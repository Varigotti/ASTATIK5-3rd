%%%%%%%%%%%%%%%%%%%%%%%%%%%%%%%%%%%%%%%%%%%%%%%%%%%%%%%%%%%%%%%%%%%%%%%%%%%%%%%%%%%%%%%%%%%%%%%%%%%
\textcolor{chapterTitleBlue}{\chapter{Betti Extended}}

In dem vorhergehenden Kapitel haben wir gesehen, dass die FE-L\"{o}sung  $u_h(x)$ die \"{U}berlagerung der gen\"{a}herten Einflussfunktion $G_h(y,x)$ mit der Belastung $p(y)$ ist
\begin{align}\label{Eq96}
u_h(x) = \int_0^{\,l} G_h(y,x)\,p(y)\,dy\,.
\end{align}
Diese f\"{u}r die finiten Elemente zentrale Gleichung beruht auf einem Satz, den wir {\em Betti extended\/} nennen. \\

\begin{theorem}[Betti extended]\index{Betti extended}
Man darf in dem {\em Satz von Betti\/} -- bei unver\"{a}nderter Belastung -- die exakten L\"{o}sungen $u_1$ und $u_2$ durch ihre FE-L\"{o}sungen $u_{1@h}$ und $u_{2@h}$ ersetzen.
\end{theorem}

Wenn also die Gleichung
\begin{align}
A_{1,2} = \int_0^{\,l} p_1\,\underset{\uparrow}{u_2}\,dx = \int_0^{\,l} p_2\,\underset{\uparrow}{u_1}\,dx = A_{2,1}
\end{align}
richtig ist, dann ist auch die Gleichung
\begin{align}
A_{1,2}^h  = \int_0^{\,l} p_1\,\underset{\uparrow}{u_{2@h}}\,dx = \int_0^{\,l} p_2\,\underset{\uparrow}{u_{1@h}}\,dx= A_{2,1}^h
\end{align}
richtig.

%----------------------------------------------------------
\begin{figure}[tbp]
\centering
\if \bild 2 \sidecaption \fi
\includegraphics[width=0.6\textwidth]{\Fpath/U128}
\caption{Der Satz von Maxwell} \label{U128}
\end{figure}%%
%----------------------------------------------------------

Wir behaupten nicht, dass $A_{1,2} = A_{1,2}^h$ ist, sondern nur, dass wenn $A_{1,2} = A_{2,1}$ gilt, dass dann auch $A_{1,2}^h = A_{2,1}^h$ richtig ist; knapp gesagt gilt also
\begin{align}
(p_1,u_2) = (p_2,u_1) \qquad \Rightarrow \qquad (p_1,u_{2@h}) = (p_2,u_{1@h})\,.
\end{align}
Was {\em Betti extended\/} bedeutet, macht man sich am einfachsten an Hand des {\em Satzes von Maxwell\/} klar, s. Abb. \ref{U128}, der ja nur eine spezielle Variante des {\em Satzes von Betti\/} ist
\begin{align}
P_1 \cdot w_2(x_1) = P_2 \cdot w_1(x_2)\,.
\end{align}
Wenn man die beiden Biegelinien $w_1(x)$ und $w_2(x)$ mit finiten Elementen berechnet, dann erh\"{a}lt man nicht die exakten Durchbiegungen in den beiden Punkten $x_1$ und $x_2$
\begin{align}
w_{1@h}(x_2) \neq w_1(x_2) \qquad w_{2@h}(x_1) \neq w_2(x_1)\,,
\end{align}
aber der {\em Satz von Maxwell\/}, das \glq \"{u}ber Kreuz gleich\grq{}, gilt gem\"{a}{\ss} {\em Betti extended\/} auch f\"{u}r diese N\"{a}herungen
\begin{align}
 P_1\cdot w_{2@ h}(x_1 ) = P_2\cdot w_{@1h}(x_2) \,.
\end{align}
Damit ist im \"{u}brigen auch gezeigt, dass der {\em Satz von Maxwell\/} auch f\"{u}r FE-L\"{o}sungen gilt, man setze $P_1 = P_2 = 1$, was ja nicht unbedingt selbstverst\"{a}ndlich ist. Dass er f\"{u}r die Knotenwerte gelten muss, war, wegen der Symmetrie der Steifigkeitsmatrizen, klar. {\em Betti extended\/} garantiert dies aber auch f\"{u}r alle Punkte dazwischen.

%%%%%%%%%%%%%%%%%%%%%%%%%%%%%%%%%%%%%%%%%%%%%%%%%%%%%%%%%%%%%%%%%%%%%%%%%%%%%%%%%%%%%%%%%%%%%%%%%%%
{\textcolor{sectionTitleBlue}{\section{Beweis}}}}

Der Beweis f\"{u}r {\em Betti extended\/} beruht auf den beiden Gleichungen
\begin{subequations}
\begin{align}
\int_{\Omega} p_{1@h}\,u_{2@h} \,d\Omega &= \int_{\Omega} p_1\,u_{2@h} \,d\Omega  \label{Eq91}\\
\int_{\Omega} p_{2@h}\,u_{1@h} \,d\Omega &= \int_{\Omega} p_2\,u_{1@h} \,d\Omega\,,
\end{align}
\end{subequations}
und dem {\em Satz von Betti\/}
\beq
\text{\normalfont\calligra B\,\,}(u_{1@h},u_{2@h}) = \int_{\Omega} p_{1@h}\,u_{2@h} \,d\Omega - \int_{\Omega} p_{2@h}\,u_{1@h} \,d\Omega = 0\,,
\eeq
so dass
\beq
  \int_{\Omega} p_1 \,u_{2@h} \,d\Omega = \int_{\Omega} p_{1@h}\, u_{2@h} \,d\Omega = \int_{\Omega} p_{2@h}\, u_{1@h} \,d\Omega = \int_{\Omega} p_2 \, u_{1@h}\,d\Omega\,,
\eeq
oder
\begin{align}
\int_{\Omega} p_1 \,u_{2@h} \,d\Omega = \int_{\Omega} p_2 \, u_{1@h}\,d\Omega\,,
\end{align}
was die Erweiterung von
\begin{align}
\int_{\Omega} p_1 \,u_2 \,d\Omega = \int_{\Omega} p_2 \, u_1\,d\Omega
\end{align}
auf die FE-L\"{o}sungen ist.

Zu (\ref{Eq91}) kommt man wie folgt: Die {\em Galerkin-Orthogonalit\"{a}t\/} besagt, dass
\begin{align}
\delta A_i = a(u_1 - u_{1@h},\Np_i) = 0
\end{align}
oder, wenn man das mit \"{a}u{\ss}erer statt mit innerer Arbeit schreibt, $\delta A_i = \delta A_a$,
\begin{align}
\int_{\Omega} (p_1 - p_{1@h})\,\Np_i \,d\Omega = 0 \quad i = 1,2,\ldots n \quad \Rightarrow \quad \int_{\Omega}(p_1 - p_{1@h})\, u_{2@h}\,d\Omega = 0\,,
\end{align}
weil ja $u_{2@h} $ eine Linearkombination der $\Np_i$ ist. Sinngem\"{a}{\ss} gilt das auch f\"{u}r die zweite Gleichung.

Mit {\em Betti extended\/} ist der Beweis
der zentralen Gleichung (\ref{Eq96}) sehr einfach, denn in der Einflussfunktion f\"{u}r $u(x)$
\begin{align}
A_{1,2} = 1 \cdot u(x) = \int_0^{\,l} \delta(y-x)\,u(y)\,dy = \int_0^{\,l} G(y,x)\,p(y)\,dy = A_{2,1}
\end{align}
darf man $u$ und $G$ durch die beiden FE-L\"{o}sungen $u_h$ und $G_h$ ersetzen
\begin{align}
A_{1,2}^h = \int_0^{\,l} \delta(y-x)\,\underset{\uparrow}{u_h}(y)\,dy = \int_0^{\,l} \underset{\uparrow}{G_h}(y,x)\,p(y)\,dy = A_{2,1}^h
\end{align}
und somit gilt
\begin{align}
u_h(x) = \int_0^{\,l} G_h(y,x)\,p(y)\,dy \,.
\end{align}
Diese Substitutionen, $u \to u_h$ und $G \to G_h$, kann man bei allen linearen Funktionalen, also allen Integralen wie
\begin{align}\label{Eq47}
J(u) = \int_0^{\,l} \delta(y-y)\,u(y)\,dy = \int_0^{\,l} G(y,x)\,p(y)\,dy
\end{align}
vornehmen, d.h. man darf jederzeit $u$ und $G(y,x)$ durch ihre FE-N\"{a}herungen ersetzen und erh\"{a}lt so
\begin{align}\label{Eq76}
J(u_h) = \int_0^{\,l} \delta(y-y)\,u_h(y)\,dy = \int_0^{\,l} G_h(y,x)\,p(y)\,dy\,.
\end{align}

%%%%%%%%%%%%%%%%%%%%%%%%%%%%%%%%%%%%%%%%%%%%%%%%%%%%%%%%%%%%%%%%%%%%%%%%%%%%%%%%%%%%%%%%%%%%%%%%%%%
{\textcolor{sectionTitleBlue}{\section{In welchen Punkten ist die FE-L\"{o}sung exakt?}}
Mit Hilfe von{\em Betti extended\/} haben wir nun auch Klarheit dar\"{u}ber, wann und wo FE-Ergebnisse exakt sind.

Wir studieren diese Frage an einem vorgespannten Seil und, um mit der Standard-Notation der finiten Elemente konform zu gehen, bezeichnen wir die Durchbiegung des Seils im folgenden mit dem Buchstaben $u$.

Die Einflussfunktion $G(y,x)$ f\"{u}r die Durchbiegung des Seils, s. Abb. \ref{U236}, in dem Punkt $x = 1.5$ ist die Antwort des Seils auf eine Einzelkraft $P = 1$, ein Dirac Delta $\delta(y-x)$.

Die Einzelkraft zwischen den zwei Knoten kann das FE-Programm nicht darstellen und so setzt es statt dessen zwei halb so gro{\ss}e Einzelkr\"{a}fte in die beiden Nachbarknoten. Dies ist -- in unserer Notation -- der Lastfall $\delta_h(y,x)$ und die zugeh\"{o}rige Durchbiegung $G_h(y,x)$ ist die gen\"{a}herte Einflussfunktion.

%----------------------------------------------------------
\begin{figure}[tbp]
\centering
\if \bild 2 \sidecaption \fi
\includegraphics[width=.8\textwidth]{\Fpath/U236}
\caption{Einflussfunktion f\"{u}r die Durchbiegung im Punkt $x = 1.5$, \textbf{ a)} exakte Einflussfunktion, \textbf{ b)} gen\"{a}herte Einflussfunktion \textbf{ c)}
FE-L\"{o}sung unter Gleichlast $p = 1$} \label{U236}
\end{figure}%%
%----------------------------------------------------------

Es gibt also zwei Dirac Deltas, das exakte und das gen\"{a}herte
\begin{align}
\delta(y-x)  \quad \downarrow  \qquad \delta_h(y-x) \quad \frac{1}{2}\,\downarrow  + \frac{1}{2}\,\downarrow
\end{align}
und ebenso zwei Einflussfunktionen
\begin{align} \label{Eq49}
G(y,x) \quad \text{(ein Knick)} \qquad G_h(y,x) \quad \text{(zwei Knicke)} \,.
\end{align}
Mit finiten Elementen suchen wir eine N\"{a}herungsl\"{o}sung in dem LF $p$ (Streckenlast) f\"{u}r den Seildurchhang auf dem Raum $\mathcal{V}_h$, also all den Polygonz\"{u}gen (Seilecken), die mit den drei $\Np_i(x)$ dargestellt werden k\"{o}nnen. Spiegelbildlich zu diesem Raum gibt es einen Raum $\mathcal{V}_h^*$
\index{$\mathcal{V}_h^*$}, der all die Knotenkr\"{a}fte $f_1, f_2, f_3$ enth\"{a}lt, die die Seilecke in $\mathcal{V}_h$ erzeugen.

Es gilt nun: wenn eine Funktion $u_h $ in $\mathcal{V}_h $ liegt (also ein Seileck ist), dann ist das gen\"{a}herte Dirac Delta (2 halbe Einzelkr\"{a}fte) so gut, wie  das exakte Dirac Delta (eine Einzelkraft)
\begin{align}\label{Eq94}
u_h(x) = \int_0^{\,l} \delta(y-x)\,u_h(y)\,dy = \int_0^{\,l} \delta_h(y-x)\,u_h(y)\,dy\,.
\end{align}
Konkret hei{\ss}t das also hier
\begin{align}
1 \cdot u_h(x) = \frac{1}{2}\, \cdot u_h(x_1) + \frac{1}{2}\, \cdot u_h(x_2)\,,
\end{align}
was einleuchtet, weil der Wert einer Geraden zwischen zwei Knoten gerade der Mittelwert der Knotenwerte ist.

Und weil der FE-Lastfall $p_h$ in $\mathcal{V}_h^*$ liegt, also aus drei Knotenkr\"{a}ften besteht, ist die gen\"{a}herte Einflussfunktion $G_h(y,x)$ so gut wie die exakte
\begin{align}
u_h(x) = \int_0^{\,l} G(y,x)\,p_h(y)\,dy = \int_0^{\,l} G_h(y,x)\,p_h(y)\,dy\,.
\end{align}
Auch das ist einfach zu verstehen. Weil die Lastf\"{a}lle $p_h$ nur aus Knotenlasten $f_i$ bestehen, wird bei der Auswertung der Einflussfunktion nicht integriert, sondern nur \"{u}ber die Knoten summiert
\begin{align}\label{Eq195}
u_h(x) = \int_0^{\,l} G_h(y,x)\,p_h(y)\,dy = \sum_{i = 1}^3\,G_h(y_i,x)\,f_i\,.
\end{align}
Die FE-Einflussfunktionen f\"{u}r die Durchbiegung der Knoten $y_i$ sind aber exakt, $G_h(y_i,x) = G(y_i,x)$ in jedem Punkt $x$, und das erkl\"{a}rt, warum die Formel f\"{u}r jeden Punkt $x$ genau den richtigen Wert $u_h(x)$ liefert.\\


\hspace*{-12pt}\colorbox{highlightBlue}{\parbox{0.98\textwidth}{Auf $\mathcal{V}_h$ bzw. $\mathcal{V}_hh^*$ sind die Ergebnisse, die man mit den N\"{a}herungen $\delta_h(y,x)$ bzw. $G_h(y,x)$ erzielt, exakt.}}\\

Um dieses Ergebnis richtig zu w\"{u}rdigen, muss man verstehen, dass mit $u_h$ hier nicht notwendig die FE-L\"{o}sung gemeint ist, sondern dass $u_h$ eine beliebige Funktion aus $\mathcal{V}_h$ sein kann.

Das gen\"{a}herte Dirac Delta $\delta_h$ ist auf $\mathcal{V}_h$ also so gut, wie das exakte, denn (\ref{Eq94}) gilt f\"{u}r alle $u_h \in \mathcal{V}_h$. Und ist $p_h$ der Lastfall, der dem Seil die Gestalt $u_h$ gibt, dann kann man mit der gen\"{a}herten Einflussfunktion $G_h(y,x)$ den Wert $u_h(x)$ aus $p_h$ berechnen. Dies ist der Inhalt von (\ref{Eq195}).


Es ist aber noch eine Steigerung m\"{o}glich. Das gen\"{a}hrte Dirac Delta, also die beiden \glq halben\grq{} Punktlasten in den Nachbarknoten, stellt ja ein eigenes Funktional
\begin{align}
J_h(u) = \int_0^{\,l} \delta_h(y-x)\,u(y)\,dy = \frac{1}{2}\,(u(x_1) + u(x_2))
\end{align}
dar, das man auf beliebige Funktionen anwenden kann -- nicht nur auf die Seilecke in $\mathcal{V}_h$.
Angewandt auf $u(x)= \sin\,\pi\,x/4$ erh\"{a}lt man z.Bsp. den Wert
\begin{align}
J_h(u) = \frac{1}{2}\, (\sin \frac{1.0\,\pi}{4} + \sin \frac{2.0\,\pi}{4}) = 0.85
\end{align}
w\"{a}hrend $J(u) = \sin (1.5\,\pi/4) = 0.92$ ist. Es besteht also ein Unterschied im Ergebnis und damit zwischen $J$ und $J_h$.
%----------------------------------------------------------
\begin{figure}[tbp]
\centering
\if \bild 2 \sidecaption \fi
\includegraphics[width=1.0\textwidth]{\Fpath/U390A}
\caption{Gelenkig gelagerte Platte mit zentrischer St\"{u}tze. Wenn man die vier R\"{a}der des SLW, LF $p$, auf die Einflussfl\"{a}che $G_h$ stellt, erh\"{a}lt man die St\"{u}tzenkraft $S_h$ der FE-L\"{o}sung. Dasselbe Ergebnis erh\"{a}lt man aber auch, wenn man die FE-Belastung $p_h$ (hier in einer symbolischen Darstellung als Blocklast) auf die exakte Einflussfl\"{a}che stellt. Ebenso gilt $S_h = (G_h,p) =  (G_h,p_h)$ Bild b und d} \label{U390}
\end{figure}%%
%----------------------------------------------------------

Mit Blick auf die exakte L\"{o}sung $u(x)$ und die FE-N\"{a}herung $u_h(x)$ gilt jedoch die {\em $h$-Vertauschungsregel\/}\index{$h$-Vertauschungsregel}\index{Vertauschungsregel}
\begin{align}\label{Eq56}
\boxed{J_h(u) = J(u_h)}\,,
\end{align}
die wir im Falle des Seils auch leicht verifizieren k\"{o}nnen
\begin{align}
J_h(u) &= \frac{1}{2}\,(u(1.0) + u(2.0)  ) =  \frac{1}{2}\,(1.5 + 2.0) = 1.75 \\
 J(u_h) &= u_h(1.5) = 1.75\,.
\end{align}
Das Funktional $J_h(u)$ misst $u$ in den beiden Punkten $x = 1.0$ und $x = 2.0$, w\"{a}hrend das Funktional $J(u_h)$ die Biegelinie $u_h$ nur im Aufpunkt $x = 1.5$ misst. Aber beide Messergebnisse sind gleich!

Die Vertauschungsregel basiert auf der Tatsache, dass eine FE-L\"{o}sung auf sechs verschiedene Arten darstellbar ist
\begin{align}
\label{sixways}
u_h(x) &= \int_0^{\,l}  G(y, x)\, p_h(y) \,dy = \int_0^{\,l}  G_h(y, x) \,p(y) \,dy\nn \\
 &= \int_0^{\,l}  G_h(y, x)\, p_h(y) \,dy \nn \\
  &= \int_0^{\,l}  \delta(y, x)\, u_h(y) \,dy =\int_0^{\,l}  \delta_h(y,x)\, u_h(y)\,dy \nn \\
&=\int_0^{\,l}  \delta_h(y,x)\, u(y) \,dy\,,
\end{align}
und wenn wir noch die Formeln
\beq
u_h(x) = a(G,u_h) = a(G_h,u_h) = a(G_h,u)
\eeq
mitz\"{a}hlen, die im Grunde Varianten der Mohrschen Arbeitsgleichung sind, sind es sogar neun.

Auf den ersten beiden Gleichungen
\begin{align}
J(u_h) = \int_0^{\,l} G(y, x) p_h(y) \,dy = \int_0^{\,l}  G_h(y, x) p(y) \,dy = J_h(u)
\end{align}
basiert die $h$-Vertauschungsregel, wobei wir gleich $u_h(x)$ zu $J(u_h)$ erweitert haben, denn (\ref{sixways}) gilt ja nicht nur f\"{u}r das Punktfunktional $J(u) = u(x)$, sondern f\"{u}r jedes lineare Funktional $J$.

Ob man die exakte Einflussfunktion $G$ mit den FE-Lasten $p_h$ \"{u}berlagert, oder die gen\"{a}herte Einflussfunktion $G_h$ mit der Originalbelastung $p$, macht keinen Unterschied -- das Resultat ist dasselbe.
%----------------------------------------------------------
\begin{figure}[tbp]
\centering
\if \bild 2 \sidecaption \fi
\includegraphics[width=.65\textwidth]{\Fpath/U199}
\caption{Wenn $w$ auf $l_e$ linear ist, ist das gen\"{a}herte Dirac Delta in Abb. \textbf{ b)}
genauso gut, wie das exakte, in Abb. \textbf{ a)}, und wenn $w$ auf $l_e$ ein kubisches Polynom ist, gilt dasselbe f\"{u}r \textbf{ c)} und \textbf{ d)}. Die gen\"{a}herten Dirac Deltas sind die \"{a}quivalenten Knotenkr\"{a}fte (= Festhaltekr\"{a}fte $\times (-1)$) aus dem exakten Dirac Delta, wenn man sich also das Element $l_e$ links und rechts eingespannt denkt} \label{U199}
\end{figure}%%
%----------------------------------------------------------

Abb. \ref{U390} demonstriert dies am Beispiel einer St\"{u}tze unter einer Platte, auf der ein Schwerlastwagen (SLW) steht. Dargestellt ist die exakte Einflussfl\"{a}che f\"{u}r die St\"{u}tzenkraft, Abb. \ref{U390} a, und die gen\"{a}herte, Abb. \ref{U390} b. Die Radlasten des SLW stellen den LF $p$ dar, und die Blocklast soll symbolisch f\"{u}r den FE-Lastfall $p_h$ stehen. Es gibt nur eine Formel f\"{u}r die exakte St\"{u}tzenkraft
\begin{align}
S = \int_{\Omega} G(\vek y,\vek x)\,p(\vek y)\,d\Omega_{\vek y}\,,
\end{align}
aber drei M\"{o}glichkeiten die N\"{a}herung $S_h$ zu berechnen
\begin{align}
S_h &= \int_{\Omega} G(\vek y,\vek x)\,p_h(\vek y)\,d\Omega_{\vek y} = \int_{\Omega} G_h(\vek y,\vek x)\,p_h(\vek y)\,d\Omega_{\vek y}\nn \\
 &= \int_{\Omega} G_h(\vek y,\vek x)\,p(\vek y)\,d\Omega_{\vek y}\,.
\end{align}
\\

\begin{remark}
Wie Abb. \ref{U199} am Beispiel eines Seils bzw. eines Balkens illustriert, sind die \"{a}quivalenten Knotenkr\"{a}fte aus dem Dirac Delta, das sind die Kr\"{a}fte und Momente in Abb. \ref{U199} b und d, so gut, wie das exakte Dirac Delta, wenn die Biegelinie \"{u}ber die Elementl\"{a}nge $l_e$ linear verl\"{a}uft
\begin{align}
w(x) = \int_0^{\,l} \delta(y-x)\,w(y)\,dy = \frac{1}{2}\, (w(x_i) + w(x_{i + 1}))
\end{align}
bzw. dort kubisch ist
\begin{align}
w(x) = \int_0^{\,l} \delta(y-x)\,w(y)\,dy &= \frac{1}{2}\,w(x_i) + w'(x_i) \cdot \frac{l_e}{8}\nn \\
&+ \frac{1}{2}\,w(x_{i+1}) -  w'(x_{i+1}) \cdot \frac{l_e}{8}\,.
\end{align}
Das ist die anschauliche Interpretation der $h$-Vertauschungsregel.

Aus der Sicht des FE-Programms reichen die gen\"{a}herten Dirac Deltas $\delta_h$ vollkommen aus, weil sie ja auf $\mathcal{V}_h$ die exakten Werte liefern, $(\delta,\Np_i) = (\delta_h,\Np_i)$.

\end{remark}

%%%%%%%%%%%%%%%%%%%%%%%%%%%%%%%%%%%%%%%%%%%%%%%%%%%%%%%%%%%%%%%%%%%%%%%%%%%%%%%%%%%%%%%%%%%%%%%%%%%
{\textcolor{sectionTitleBlue}{\section{Exakte Werte}}}
Wir k\"{o}nnen nun auch sagen, wann die FE-L\"{o}sung
in einem Punkt exakt ist.\\

\begin{theorem}[Exakte Werte]\index{exakte Werte} \vspace{0.3cm}
\\Hinreichende Bedingungen
\begin{enumerate}
  \item Wenn die Einflussfunktion $G$ eines Funktionals $J$ in $\mathcal{V}_h$ liegt, dann ist sie identisch mit der FE-N\"{a}herung, $G_h = G$, d.h. dann gilt
\beq
J_h(u) = J(u) \qquad \text{f\"{u}r alle}\,\,\,u \in V
\eeq
und daher auch
\beq
J(u_h) = J_h(u) = J(u)\,.
\eeq
  \item Wenn die exakte L\"{o}sung in $\mathcal{V}_h$ liegt, $u = u_h$ (die Projektion $u_h$ ist identisch mit $u$), dann ist der Fehler in jeder Einflussfunktion orthogonal zur rechten Seite $p$
      \beq\label{Eq52}
      J(u) - J(u_h) = \int_{\Omega}(G(\vek y, \vek x) - G_h(\vek y, \vek x))\,p(\vek y)\,d\Omega_{\vek y} = 0\,.
      \eeq
\end{enumerate}
Notwendige Bedingung
\begin{enumerate}
  \item Wenn ein Wert exakt ist, $J(u_h) = J(u)$, dann muss der Fehler in der Einflussfunktion orthogonal sein zur rechten Seite $p$
      \beq
      J(u) - J(u_h) = \int_{\Omega}(G(\vek y, \vek x) - G_h(\vek y, \vek x))\,p(\vek y)\,d\Omega_{\vek y} = 0\,.
      \eeq
\end{enumerate}
\end{theorem}

%%%%%%%%%%%%%%%%%%%%%%%%%%%%%%%%%%%%%%%%%%%%%%%%%%%%%%%%%%%%%%%%%%%%%%%%%%%%%%%%%%%%%%%%%%%%%%%%%%%
{\textcolor{sectionTitleBlue}{\section{Eindimensionale Probleme}}}
Der obige Satz fasst im Grunde die ganze Theorie zusammen, aber vielleicht ist es sinnvoll, auf einzelne Aspekte doch noch n\"{a}her einzugehen.

Es ist bekannt, dass bei eindimensionalen Problemen wie St\"{a}ben und Balken, die FE-L\"{o}sung mit der exakten L\"{o}sung in den Knoten \"{u}bereinstimmt. Dies liegt daran, wie wir jetzt wissen, dass die Einflussfunktionen f\"{u}r die Knotenwerte in dem Ansatzraum $\mathcal{V}_h$ liegen.

Bei Differentialgleichungen zweiter Ordnung wie St\"{a}ben, Seilen, Schubtr\"{a}gern, etc. arbeitet man mit st\"{u}ckweise linearen Ansatzfunktionen. F\"{u}r die Darstellung der Einflussfunktionen der Knoten reichen diese Ans\"{a}tze jedoch aus, wie man zum Beispiel an dem Seil in Abb. \ref{U236} sieht, weil die Einflussfunktionen ja auch nur st\"{u}ckweise linear sind.

Bei Differentialgleichungen vierter Ordnung, wie dem Biegebalken, basieren die finiten Elemente auf Hermite-Polynomen (kubische Ans\"{a}tze) mit denen man die Einflussfunktionen f\"{u}r die Durchbiegungen und Verdrehungen der Knoten exakt darstellen kann. Die FE-L\"{o}sung ist daher in den Knoten  exakt.
%------------------------------------FIGURE-----------------------
\begin{figure}
\centering
{\includegraphics[width=0.65\textwidth]{\Fpath/U130}}
  \caption{Quadratische Elemente interpolieren die exakte L\"{o}sung nicht im Mittelknoten, sondern nur an den Endknoten des Elements, weil die {\em bubble function\/} des Mittelknotens zu glatt ist, \textbf{ a)} Ansatzfunktionen, \textbf{ b)} Punktlast im Mittelknoten und FE--L\"{o}sung, \textbf{ c)} Punktlast  am Endknoten und die FE-L\"{o}sung, die in diesem Falle exakt ist}
  \label{U130}
\end{figure}%
%------------------------------------FIGURE-----------------------

Das Ganze gilt nicht mehr, wenn die homogenen L\"{o}sungen der Differentialgleichungen aus diesem Raster herausfallen, wie im Fall eines in L\"{a}ngsrichtung gebetteten ($c$) Stabes
\begin{align}
- EA\,u''(x) + c\,u(x) = p_x\,,
\end{align}
wo die homogene L\"{o}sung die Gestalt
\beq\label{Eq50}
u(x) = c_1 \,e^{\alpha x } + c_2\, e^{-\alpha x} \qquad \alpha = \sqrt{\frac{c}{EA}}
\eeq
hat oder im Fall eines elastisch gelagerten Balkens,
\begin{align}
 EI\,w^{IV} + c\,w(x) = p_z\,,
\end{align}
wo die homogene L\"{o}sung die Gestalt
\begin{align}\label{Eq51}
\!\!\!\!\!\!\!\!w(x) &= e^{\beta\,x}(c_1\,\cos \beta\,x + c_2\, \sin \beta\,x) +
e^{-\beta\,x}(c_3\,\cos \beta\,x + c_4\, \sin
\beta\,x)\\
\beta &= \sqrt[4]{\frac{c}{EI}}
\end{align}
hat.

Denn alle Einflussfunktionen setzen sich st\"{u}ckweise aus den homogenen L\"{o}sungen der zu Grunde liegenden Differentialgleichung zusammen, aber normalerweise enth\"{a}lt der Ansatzraum $\mathcal{V}_h$ nicht solche \glq exotischen\grq{} Funktionen wie (\ref{Eq50}) und (\ref{Eq51}).

Zu den Merkw\"{u}rdigkeiten geh\"{o}rt auch, dass man mit eigentlich besseren Ansatzfunktionen unter Umst\"{a}nden die F\"{a}higkeit verliert, die exakte L\"{o}sung in den Knoten zu interpolieren. Dies passiert, wenn man zum Beispiel eine Seillinie mit quadratischen Elementen ann\"{a}hert, s. Abb. \ref{U130}.

In der Mitte des Elementes ist eine quadratische FE-L\"{o}sung glatt, dort regiert die {\em bubble function\/}\index{bubble function} des Mittenknotens. Was man aber br\"{a}uchte, w\"{a}re die M\"{o}glichkeit, dort einen Sprung in der ersten Ableitung darstellen zu k\"{o}nnen, damit man die Wirkung einer Einzelkraft in dem Mittenknoten wiedergeben kann. Die Ableitung der {\em bubble function\/} ist aber glatt, sie springt nicht, und deswegen stimmt z.B. die FE-L\"{o}sung des Problems
\begin{align}
- u''(x) = x \qquad u(0) = u(l) = 0\,,
\end{align}
nicht mit der exakten L\"{o}sung in den Mittenknoten \"{u}berein. Die Einflussfunktion f\"{u}r den Mittenknoten liegt nicht in $\mathcal{V}_h$\footnote{Wenn $u$ in $\mathcal{V}_h$ liegt, z.B. wenn $u$ quadratisch ist, dann ist der Fehler $u(x) - u_h(x) = 0$, weil in dem Fall der Fehler in der Einflussfunktion orthogonal zu $p$ ist.}.
%-----------------------------------------------------------------
\begin{figure}[tbp]
\centering
\includegraphics[width=0.72\textwidth]{\Fpath/U158}
\caption{Platte im LF $g$, \textbf{ a)} Einflussfunktion f\"{u}r $m_{xx}$ in Plattenmitte, \textbf{ b)} Momente $m_{xx}$ im L\"{a}ngsschnitt, \textbf{ c)} Momente $m_{xx}$ mit St\"{u}tze} \label{U158}
\end{figure}%
%-----------------------------------------------------------------

Ein FE-Programm muss also eine Balance finden zwischen der Regularit\"{a}t, die von der Wechselwirkungsenergie verlangt wird, damit man die Beitr\"{a}ge $k_{ij} = a(\Np_i,\Np_j) $ der Steifigkeitsmatrix berechnen kann und der \glq Nicht-Regularit\"{a}t\grq{}, die man braucht, um einen Sprung in der ersten Ableitung erzeugen zu k\"{o}nnen.

%%%%%%%%%%%%%%%%%%%%%%%%%%%%%%%%%%%%%%%%%%%%%%%%%%%%%%%%%%%%%%%%%%%%%%%%%%%%%%%%%%%%%%%%%%%%%%%%%%%
{\textcolor{sectionTitleBlue}{\section{Fl\"{a}chentragwerke}}}
Wenn bei Fl\"{a}chentragwerken die Ergebnisse in einem Punkt $\vek x$ exakt sind, dann ist das in der Regel Zufall. Dann kann es nur so sein, dass der Fehler $G(\vek y,\vek x) - G_h(\vek y,\vek x)$ in der Einflussfunktion orthogonal zur Belastung $p$ ist,
\begin{align}
u(\vek x) - u_h(\vek x) = \int_{\Omega} (G(\vek y,\vek x) - G_h(\vek y,\vek x))\,p(y)\,\,d\Omega_{\vek y} = 0\,,
\end{align}
denn die exakten Einflussfunktionen liegen bei Fl\"{a}chentragwerken nicht in dem Ansatzraum $\mathcal{V}_h$ der finiten Elemente, weil die verwendeten {\em shape functions\/} keine homogenen L\"{o}sungen der Scheiben- bzw. Plattengleichung sind.

%---------------------------------------------------------------------------------
\begin{figure}[tbp]
\centering
\if \bild 2 \sidecaption \fi
\includegraphics[width=0.8\textwidth]{\Fpath/U139}
  \caption{3-D Darstellung der Momente $m_{xx}$ der Quadratplatte mit einer zentrischen St\"{u}tze im LF $g$}
  \label{U139}
\end{figure}
%---------------------------------------------------------------------------------
%---------------------------------------------------------------------------------
\begin{figure}[tbp]
\centering
\if \bild 2 \sidecaption \fi
\includegraphics[width=0.8\textwidth]{\Fpath/U138}
  \caption{Quadratplatte, 8 m $\times$ 8 m, mit Einzelst\"{u}tze in der Mitte, (dargestellt ist die untere H\"{a}lfte der Platte), Serie von Einflussfunktionen f\"{u}r $m_{xx}$ auf der $x$-Achse}
  \label{U138}
\end{figure}
%---------------------------------------------------------------------------------
Es ist aber auch klar, dass die Art der Belastung
\begin{align}
\text{{\em Gleichlast\/}} \qquad\text{{\em Linienlast\/}} \qquad \text{{\em Punktlast\/}}
\end{align}
einen Einfluss auf die Gr\"{o}{\ss}e des Fehlers hat. Je gleichm\"{a}{\ss}iger die Belastung verteilt ist, um so eher gleichen sich die Fehler in den Einflussfunktionen im Mittel aus, wie man am Beispiel einer gelenkig gelagerten Quadratplatte sehen kann.

In Abb.  \ref{U158} a ist die Einflussfunktion f\"{u}r das Moment $m_{xx}$ in Plattenmitte dargestellt. Im Lastfall $g$ ist das Moment
\begin{align}
m_{xx}(\vek x) = \int_{\Omega} G_2(\vek y,\vek x)\,g\,\,d\Omega_{\vek y} = g \cdot \int_{\Omega} G_2(\vek y,\vek x)\,d\Omega_{\vek y} = g \cdot V
\end{align}
gleich dem Volumen $V$ von $G_2$ mal $g$. Und anscheinend kann das FE-Pro\-gramm das Volumen der Einflussfunktionen relativ gut bestimmen, die
Momentenverteilung in Abb. \ref{U158} b wirkt \"{u}berzeugend.

Eine Einzelkraft $ P$ aus einer St\"{u}tze, $\Omega_S = a \times b$, ist jedoch von einem anderen Kaliber, s. Abb. \ref{U139},
\begin{align}\label{Eq134}
m_{xx}(\vek x) = \int_{\Omega} G_2(\vek y,\vek x)\,g\,\,d\Omega_{\vek y} + \frac{P}{\Omega_S} \int_{\Omega_S} G_2(\vek y,\vek x)\,d\Omega_{\vek y}\,,
\end{align}
denn $G_2$ ist ja singul\"{a}r in der St\"{u}tzenmitte.

Genau genommen rechnet das Programm zwar mit einer Ersatzlast $g_h(\vek y)$, die das Eigengewicht wie die St\"{u}tzenkraft \glq \"{a}quivalent\grq{} beinhaltet
\begin{align}\label{Eq135}
m_{xx}^h(\vek x) = \int_{\Omega} G_2^h(\vek y,\vek x)\,g_h(\vek y)\,\,d\Omega_{\vek y}\,,
\end{align}
aber auch diese Formel kommt an dem Grundproblem, $G_2(\vek x, \vek x) = \infty$, und dem scharfen Anstieg von $G_2$ zur St\"{u}tze hin nicht vorbei.

Die Abb. \ref{U138} illustriert, wie sich die Einflussfunktion f\"{u}r $m_{xx}$ \"{a}ndert, wenn sich der Aufpunkt der St\"{u}tze n\"{a}hert. Im Grunde bleibt sie sich immer gleich, nur wird sie bei der Ann\"{a}herung an die St\"{u}tze nach oben geschoben. Und das Ma{\ss}, um wieviel sie nach oben geschoben werden muss, das ist umso schwerer zu bestimmen, je n\"{a}her der Aufpunkt der St\"{u}tze kommt.\\

%%%%%%%%%%%%%%%%%%%%%%%%%%%%%%%%%%%%%%%%%%%%%%%%%%%%%%%%%%%%%%%%%%%%%%%%%%%%%%%%%%%%%%%%%%%%%%%%%%%
{\textcolor{sectionTitleBlue}{\section{Punktlager bei Scheiben und Platten und der Unterschied}}}
Bei Scheiben liegen die Dinge \"{a}hnlich. Es gibt jedoch einen bemerkenswerten Unterschied zwischen Scheibe und Platte.

Wenn eine Scheibe sich auf Punktlager abst\"{u}tzt, dann kennt man die Lagerkr\"{a}fte relativ genau (bei statisch bestimmter Lagerung sogar exakt), aber das hilft einem nicht bei der Eingrenzung der (mit dem Ingenieurverstand vertr\"{a}glichen) maximalen Spannungen in der N\"{a}he der Punktlager. Hier kann man sich aber so behelfen, dass man die Lagerkraft \"{u}ber eine gewisse Breite gleichm\"{a}{\ss}ig verteilt und die Scheibe f\"{u}r diese Spannungen bemisst, s. Abb. \ref{U133} a.

%---------------------------------------------------------------------------------
\begin{figure}[tbp]
\centering
\if \bild 2 \sidecaption \fi
\includegraphics[width=0.5\textwidth]{\Fpath/U133}
  \caption{\textbf{ a)} Die Lagerkraft $R$ und die Lagerpressung bei einer Scheibe, \textbf{ b)} bei einem Balken kann man jedoch keine Beziehung zwischen dem St\"{u}tzmoment $M$ und der Lagerkraft $R$ herstellen}
  \label{U133}
\end{figure}
%---------------------------------------------------------------------------------
Bei einer Platte ist die Situation im Grunde dieselbe. Die Lagerkr\"{a}fte in den St\"{u}tzen sind bei einer FE-Berechnung relativ genau, aber das hilft einem nicht -- und das ist der Unterschied -- bei der Eingrenzung der maximalen Biegemomente \"{u}ber der St\"{u}tze, weil es keinen direkten Zusammenhang zwischen der St\"{u}tzenkraft und den Biegemomenten gibt; die Ausmitte $e$ bleibt unbekannt.

Es reicht ein Zwischenlager eines Balkens zu betrachten, s. Abb. \ref{U133} b. Die Summe der Querkr\"{a}fte $V_l$ und $V_r$ muss gleich der Lagerkraft $R$ sein, aber es gelingt nicht, das St\"{u}tzmoment $M$ in irgendeiner Weise mit $R$ zu verkn\"{u}pfen.

Es gibt ja eine Vielzahl von Tr\"{a}gern, die \"{u}ber einem Zwischenlager bei gleicher Lagerkraft ganz unterschiedliche Momente aufweisen. Dazu passt, dass Momente keine Arbeiten leisten, wenn man sie hebt oder senkt -- Kr\"{a}fte schon. Man m\"{u}sste die Platte neigen, um die Ausmitte $e$ zu finden. Die Wirkung eines Moments geht mit der Neigung der Einflussfunktion, Wirkung = $G' \cdot M$.

%%%%%%%%%%%%%%%%%%%%%%%%%%%%%%%%%%%%%%%%%%%%%%%%%%%%%%%%%%%%%%%%%%%%%%%%%%%%%%%%%%%%%%%%%%%%%%%%%%%
{\textcolor{sectionTitleBlue}{\section{Wenn die L\"{o}sung in $\mathcal{V}_h$ liegt}}}
Wenn die L\"{o}sung in $\mathcal{V}_h$ liegt, weil zum Beispiel bilineare Ans\"{a}tze ausreichen, um die Verformungen einer Scheibe darzustellen, dann ist $\vek p_h = \vek p$, d.h. dann ist der FE-Lastfall identisch mit dem Originallastfall. Den FE-Lastfall $\vek p_h $ erh\"{a}lt man ja, indem man die FE-L\"{o}sung in die Originalgleichung einsetzt und dabei muss in diesem Falle gerade die Belastung $\vek p$ herauskommen, die man aufgebracht hat.
%---------------------------------------------------------------------------------
\begin{figure}[tbp]
\centering
\if \bild 2 \sidecaption \fi
\includegraphics[width=1.0\textwidth]{\Fpath/U237}
  \caption{FE-Einflussfunktionen f\"{u}r $\sigma_{xx}$. Diese \glq eckigen\grq{} Einflussfunktionen sind bestimmt nicht richtig, aber das Integral \"{u}ber den Lastrand ergibt in beiden F\"{a}llen, und in jedem anderen Fall auch, den korrekten Wert f\"{u}r $\sigma_{xx}$.}
  \label{U237}
\end{figure}
%---------------------------------------------------------------------------------

Es bleibt aber ein Problem. Ein FE-Programm berechnet alle Werte und also auch die Spannungen mit gen\"{a}herten Einflussfunktionen
\begin{align}
\sigma_{xx}^h(\vek x) = \int_{\Omega} \vek G_h(\vek y,\vek x)\dotprod \vek p(\vek y)\,d\Omega_{\vek y}\,.
\end{align}
Also m\"{u}ssten die FE-Spannungen doch nur N\"{a}herungswerte sein, warum sind sie aber exakt?

Der Grund ist, dass der Fehler in den Einflussfunktionen orthogonal zu der Belastung ist, wenn die L\"{o}sung in $\mathcal{V}_h$ liegt,
\begin{align}
\sigma_{xx}(\vek x) - \sigma_{xx}^h(\vek x)= \int_{\Omega} (\underbrace{\vek G(\vek y,\vek x) - \vek G_h(\vek y,\vek x)}_{Fehler})\dotprod \vek p(\vek y)\,d\Omega_{\vek y} = 0\,.
\end{align}
Jedes solches $\vek p$ \glq neutralisiert\grq{} also den Fehler in den Einflussfunktionen.\\

\hspace*{-12pt}\colorbox{highlightBlue}{\parbox{0.98\textwidth}{Der Fehler in einer gen\"{a}herten Einflussfunktion ist orthogonal zu allen Lastf\"{a}llen $p$, die sich auf $\mathcal{V}_h$ exakt l\"{o}sen lassen}}\\

Die Scheibe in Abb. \ref{U237} ist so gelagert, dass sich unter Zug ein gleichf\"{o}rmiger Spannungszustand aufbaut, den man mit bilinearen Elementen  exakt wiedergeben kann. Die  exakte L\"{o}sung liegt also in $\mathcal{V}_h $.
%----------------------------------------------------------
\begin{figure}[tbp]
\centering
\if \bild 2 \sidecaption \fi
\includegraphics[width=1.0\textwidth]{\Fpath/U129}
\caption{Dirac Delta $\delta(\vek y- \vek x)$ und das gen\"{a}herte Dirac Delta $\delta_h(\vek y, \vek x)$ bei einer Scheibe, auf $\mathcal{V}_h$ liefern beide dasselbe Ergebnis } \label{U129}
\end{figure}%%
%----------------------------------------------------------

Aber die Einflussfunktion f\"{u}r die horizontale Spannung $ \sigma_{xx} $, gleich welchen Punkt $\vek x$ man betrachtet, liegt nicht in $\mathcal{V}_h $, weil dazu Punktversetzungen n\"{o}tig sind, die sich mit bilinearen Elementen nicht erzeugen lassen, aber trotzdem sind die Ergebnisse exakt.

Das geht nur so, dass die horizontalen Randverschiebungen der FE-Einflussfunktion in der Summe (dem Integral) genau den exakten Wert treffen, denn sonst w\"{a}re $\sigma_{xx}^h$ nicht gleich $\sigma_{xx}$
\begin{align}
\sigma_{xx}^h(\vek x) = \int_{\Gamma} \vek G_h(\vek y,\vek x) \dotprod \vek t(\vek y)\,ds_{\vek y} = \sigma_{xx}(\vek x)\,.
\end{align}
Hier ist $\Gamma$ der rechte Rand und $\vek t = \{t,0\}^T$ ist die Randlast.\\

\begin{remark}
Die Gleichung
\begin{align}\label{Eq111}
\int_{\Omega} (\vek G(\vek y,\vek x) - \vek G_h(\vek y,\vek x))\dotprod \vek p_h(\vek y)\,\,d\Omega_{\vek y} = 0
\end{align}
ist das Gegenst\"{u}ck zu der {\em Galerkin-Orthogonalit\"{a}t\/}, die ja besagt, dass die Abweichung des FE-Lastfalls $\vek p_h$ vom exakten Lastfall $\vek p$ orthogonal zu allen Ansatzfunktionen $\vek \Np_i$ ist
\begin{align}
\int_{\Omega} (\vek p(\vek x) - \vek p_h(\vek x))\dotprod \vek \Np_i(\vek x)\,d\Omega = 0\,.
\end{align}
Galerkin testet mit den Ansatzfunktionen, w\"{a}hrend der Test bei den Einflussfunktionen, (\ref{Eq111}), die Lastf\"{a}lle $\vek p_h$ sind, die man auf $\mathcal{V}_h$ exakt l\"{o}sen kann. F\"{u}r jedes solche $\vek p_h$ muss  die Differenz $\vek G - \vek G_h$ orthogonal zu $\vek p_h$ sein.

Die {\em Galerkin-Orthogonalit\"{a}t\/} gilt auch f\"{u}r Punktlasten, s. Abb. \ref{U129}, und daher ist auf $\mathcal{V}_h$ kein Unterschied zwischen dem exakten und dem gen\"{a}herten Dirac Delta ($\Np_{i}^H$ = horizontale Komponente von $\vek \Np_i(\vek x)$ im Punkt $\vek x$)
\begin{align}
\Np_{i}^H(\vek x) = \int_{\Omega} \vek \delta(\vek y - \vek x) \dotprod \vek \Np_i(\vek y) \,d\Omega_{\vek y} = \int_{\Omega} \vek \delta_h(\vek y, \vek x) \dotprod \vek \Np_i(\vek y) \,d\Omega_{\vek y}\,.
\end{align}
\end{remark}
%----------------------------------------------------------------------
\begin{figure}[h]
\if \bild 2 \sidecaption \fi
\includegraphics[width=0.8\textwidth]{\Fpath/U533}
\caption{Um das Netz an den richtigen Stellen zu verfeinern, muss man die exakte L\"{o}sung nicht kennen. Dort wo die Knicke in der FE-L\"{o}sung gro{\ss} sind, dort verfeinert man. In der Statik sind die Knicke die Spannungsspr\"{u}nge zwischen den Elementen}\label{U533}
\end{figure}
%----------------------------------------------------------------------
\vspace{-0.5cm}
%%%%%%%%%%%%%%%%%%%%%%%%%%%%%%%%%%%%%%%%%%%%%%%%%%%%%%%%%%%%%%%%%%%%%%%%%%%%%%%%%%%%%%%%%%%%%%%%%%%
{\textcolor{sectionTitleBlue}{\section{Patch-Test}}}\label{Patch-Test}\index{Patch-Test}
Bei einem Patch-Test konstruiert man eine L\"{o}sung, die in $\mathcal{V}_h$ liegt, vorzugsweise sind das L\"{o}sungen mit einfachen Spannungsfeldern, und kontrolliert, ob sich die exakten Ergebnisse ergeben. Der Patch-Test lebt davon, dass der Fehler in den Einflussfunktionen orthogonal zur Belastung ist, wenn die L\"{o}sung $u$ in  $\mathcal{V}_h$ liegt, wenn also $p_h = p$ ist, denn mit (\ref{Eq92}) gilt
\begin{align}
\int_{\Omega} (G(\vek y,\vek x) - G_h(\vek y,\vek x))\,p(\vek y)\,d\Omega_{\vek y} = \int_{\Omega} G(\vek y,\vek x)\,(p(\vek y) - p_h(\vek y))d\Omega_{\vek y}\,,
\end{align}
und weil die rechte Seite null ist muss auch die linke Seite null sein. Tests, die garantiert schief gehen, sind Einzelkr\"{a}fte bei Fl\"{a}chentragwerken. Sie \"{u}berfordern jedes Netz. Kein $\mathcal{V}_h$ enth\"{a}lt die exakte L\"{o}sung.

%----------------------------------------------------------
\begin{figure}[tbp]
\centering
\if \bild 2 \sidecaption[t] \fi
\includegraphics[width=.7\textwidth]{\Fpath/U271}
\caption{Adaptive Verfeinerung einer Scheibe in der Umgebung der kritischen Punkte} \label{U271}
\end{figure}%%
%----------------------------------------------------------

%%%%%%%%%%%%%%%%%%%%%%%%%%%%%%%%%%%%%%%%%%%%%%%%%%%%%%%%%%%%%%%%%%%%%%%%%%%%%%%%%%%%%%%%%%%%%%%%%%%
{\textcolor{sectionTitleBlue}{\section{Adaptive Verfeinerung}}}\label{SecAdaptiveVerfeinerung}
Den Fehler einer FE-L\"{o}sung kann man nicht direkt berechnen, denn man kennt weder die exakten Verschiebungen noch die exakten Spannungen. Man kann nur die Abweichung zwischen dem Originallastfall $\vek p$ und dem FE-Lastfall $\vek p_h$ messen und dann das Netz dort verfeinern, s. Abb. \ref{U271}, wo diese Differenzen am gr\"{o}{\ss}ten sind.

Theoretisch geht es sogar noch einfacher, wie Abb. \ref{U533} demonstriert: Man achtet nur auf die 'Knicke' (= Spannungsspr\"{u}nge in der FE-L\"{o}sung) und verfeinert das Netz an diesen Stellen.

Da es, wie so oft, nur auf die Algebra ankommt, wollen wir die Gleichungen skalar schreiben und die Biegefl\"{a}che $u(\vek x)$ einer Membran, die unter einem Druck $p$ steht, betrachten.

Aus den Darstellungen
\begin{align}
u(\vek x)= \int_{\Omega} G(\vek y,\vek x)p(\vek y)\,d\Omega_{\vek y} \qquad u_h(\vek x)= \int_{\Omega} G_h(\vek y,\vek x)p(\vek y)\,d\Omega_{\vek y}
\end{align}
ergibt sich der Fehler der FE-L\"{o}sung zu
\begin{align}
u(\vek x) - u_h(\vek x) = \int_{\Omega} (G(\vek y,\vek x) - G_h(\vek y,\vek x))\,p(\vek y)\,d\Omega_{\vek y}\,.
\end{align}
Nun wissen wir aber auch, dass der Fehler einer FE-L\"{o}sung darauf beruht, dass ein FE-Programm den FE-Lastfall $p_h$ statt des exakten Lastfalls $p$ l\"{o}st
\begin{align}
u(\vek x) - u_h(\vek x) = \int_{\Omega} G(\vek y,\vek x)(p(\vek y) - p_h(\vek y))\,d\Omega_{\vek y}\,,
\end{align}
und somit gilt
\begin{subequations}\label{Eq92}
\begin{align}
u(\vek x) - u_h(\vek x) &= \int_{\Omega} G(\vek y,\vek x)(p(\vek y) - p_h(\vek y))\,d\Omega_{\vek y} \label{Eq92:SubEq1} \\
&= \int_{\Omega} (G(\vek y,\vek x) - G_h(\vek y,\vek x))\,p(\vek y)\,d\Omega_{\vek y} \label{Eq92:SubEq2} \\
&= \int_{\Omega} (G(\vek y,\vek x) - G_h(\vek y,\vek x))\,(p(\vek y) - p_h(\vek y))\,d\Omega_{\vek y} \label{Eq92:SubEq3}\,.
\end{align}
\end{subequations}
Den Schritt von der zweiten Gleichung zur dritten Gleichung macht die {\em Galerkin-Orthogonalit\"{a}t\/}
\begin{align}
\int_{\Omega} (G(\vek y,\vek x) - G_h(\vek y,\vek x))\,p_h(\vek y)\,d\Omega_{\vek y} = 0
\end{align}
m\"{o}glich; wir haben nur eine null addiert.

Auf der ersten Gleichung (\ref{Eq92:SubEq1}) basiert die klassische adaptive Verfeinerung, bei der das Netz dort verfeinert wird, wo die Abweichungen zwischen $p$ und dem FE-Lastfall $p_h$ gro{\ss} sind.
%----------------------------------------------------------------------------
\begin{figure}[tbp] \centering
\if \bild 2 \sidecaption \fi
\includegraphics[width=0.8\textwidth]{\Fpath/TOTTENHAM24}
\caption{Bilineare Elemente, LF $g$, Berechnung von $\sigma_{xx}$ {\bf a)} Ausgangsnetz
{\bf b)} halbe Elementl\"{a}nge {\bf c)} adaptive Verfeinerung {\bf d)} Verfeinerung mittels
Dualit\"{a}tstechnik}\label{Tottenham}
\end{figure}
%----------------------------------------------------------------------------

Die andere Strategie w\"{a}re es, den Fehler $G-G_h$ in der Einflussfunktion kleiner zu machen, wie dies (\ref{Eq92:SubEq2}) nahelegt.

Dabei stellt sich die Frage, wie wir den Abstand $G - G_h$ messen wollen, denn wir kennen das exakte $G$ nicht. Dies Problem l\"{o}sen wir wie folgt: Zun\"{a}chst ersetzen wir die Einflussfunktion (\ref{Eq92:SubEq2}) durch ihre schwache Form
\begin{align}\label{Eq180}
u(\vek x) - u_h(\vek x) &= \int_{\Omega} (G(\vek y,\vek x) - G_h(\vek y,\vek x))\,p(\vek y)\,d\Omega_{\vek y} = a(G - G_h,u)
\end{align}
und schreiben diese Wechselwirkungsenergie \"{a}quivalent als \"{a}u{\ss}ere Arbeit\footnote{Das ist der Satz von Betti, (\ref{Eq180}) = $W_{12} = W_{21}$ = (\ref{Eq181})}
\begin{align}\label{Eq181}
u(\vek x) - u_h(\vek x) &=a(G - G_h,u) = \int_{\Omega} (\delta(\vek y -\vek x) - \delta_h(\vek y -\vek x))\,u(\vek y)\,d\Omega_{\vek y}\,,
\end{align}
und die Abweichung $\delta - \delta_h$ k\"{o}nnen wir messen, s. Abb. \ref{U129}. Nicht direkt, weil $\delta$ ja nur ein fl\"{u}chtiges Symbol ist, aber die Kr\"{a}fte $\delta_h$ k\"{o}nnen wir plotten und das Netz so verfeinern, dass sie weitgehend vom Bildschirm verschwinden, sich in einem Punkt zusammenschn\"{u}ren.

Die dritte Variante ist die {\em zielorientierte adaptive Verfeinerung\/} ({\em goal oriented adaptive refinement)\/}. Bei dieser Technik wird der Fehler $G- G_h$ in der Einflussfunktion und gleichzeitig der Fehler $p - p_h$ in der Belastung kleiner gemacht. Sie basiert auf (\ref{Eq92:SubEq3}). Schreiben wir diese Gleichung in schwacher Form, dann lautet sie
\begin{align}
u(\vek x) - u_h(\vek x) &= a(\underbrace{G(\vek y,\vek x) - G_h(\vek y,\vek x)}_f, \underbrace{\vphantom{G(\vek y,\vek x)} u - u_h}_g) =: a(f,g)
\end{align}
und jetzt kann man die {\em Schwarzsche Ungleichung\/}\index{Schwarzsche Ungleichung} zu Hilfe nehmen,
\begin{align}\label{Eq169}
|u(\vek x) - u_h(\vek x)| = | a(f,g)|  \leq a(f,f)^{1/2} \cdot a(g,g)^{1/2} = \|f\| \cdot \|g\|\,,
\end{align}
um die linke Seite durch die Energienorm von $f$  und von $g$ abzusch\"{a}tzen.

Theoretisch steht (\ref{Eq169}) auf \glq wackligen F\"{u}{\ss}en\grq{}, weil die Energienorm der Einflussfunktion ja in der Regel unendlich gro{\ss} ist, aber wenn wir das schlicht ignorieren -- der Erfolg gibt uns recht -- dann ist das die schnellste Art, den Fehler der FE-L\"{o}sung in einem Punkt klein zu machen.

Technisch geht man bei der  zielorientierten adaptiven Verfeinerung so vor, dass man neben dem {\em primalen Problem\/} $- \Delta u = f$ noch das {\em duale Problem\/} $- \Delta G = \delta_0$ l\"{o}st und das Netz so optimiert, dass die Fehler in beiden L\"{o}sungen klein werden, wie bei der Scheibe in Abb. \ref{Tottenham}, wo es darum ging die Spannung $\sigma_{xx}$ in dem Aufpunkt m\"{o}glichst genau zu bestimmen.

Man k\"{o}nnte nach diesen Bemerkungen nun vermuten, dass fehlende Spr\"{u}nge eine Garantie daf\"{u}r sind, dass die FE-L\"{o}sung \glq genau\grq{} ist, aber das ist nicht garantiert, \cite{Ha5}, denn Fernfeldfehler, Stichwort {\em pollution\/}, k\"{o}nnen zu einem {\em drift\/} der FE-L\"{o}sung f\"{u}hren, den man nicht registriert, wenn man nur auf die Spr\"{u}nge achtet.

%%%%%%%%%%%%%%%%%%%%%%%%%%%%%%%%%%%%%%%%%%%%%%%%%%%%%%%%%%%%%%%%%%%%%%%%%%%%%%%%%%%%%%%%%%%%%%%%%%%
{\textcolor{sectionTitleBlue}{\section{Pollution}}}
Der englische Begriff {\em pollution\/}\index{pollution} meint das Ph\"{a}nomen, dass die L\"{o}sung in einem Teil $A$ des Tragwerks von Fehlerquellen, die in einem abliegenden Teil $B$ auftreten, negativ beeinflusst wird. Bei Fl\"{a}chentragwerken haben die Einflussfunktionen mit diesem Problem zu k\"{a}mpfen, s. Kapitel 6.10. Die Fehler in den Ecken strahlen ins Innere aus.
%---------------------------------------------------------------------------------
\begin{figure}
\centering
{\includegraphics[width=1.0\textwidth]{\Fpath/1GreenF156}}
\caption{Einflussfunktion f\"{u}r die Scherkraft $N_{yx}$ im Schnitt $A-A$, \textbf{ a)} Wandscheibe und FE-Netz, \textbf{ b)} FE-Einflussfunktion $G_h$, \textbf{ c)} verbessertes Modell, \textbf{ d)} verbesserte FE-L\"{o}sung $G_h$, \cite{Ha6}}
\label{1GreenF156}%
%
\end{figure}%
%---------------------------------------------------------------------------------

Die Wandscheibe in Abb. \ref{1GreenF156} illustriert dieses Ph\"{a}nomen sehr gut. Die Einflussfunktion f\"{u}r die Scherkraft $N_{yx}$ im horizontalen Schnitt $A-A$ ist eine Seitw\"{a}rtsbewegung des Teils oberhalb des Schnittes um einen Meter nach rechts.

Auf einem FE-Netz kann man diese Bewegung aber nicht nachfahren, weil man dann die Elemente auseinander schneiden m\"{u}sste. So produziert ein FE-Programm, das bilineare Elemente benutzt, die Verformungsfigur in Abb. \ref{1GreenF156} b, bei der sich die Oberkante der Scheibe um 2.3 m nach rechts bewegt! Die Schwierigkeiten, die das FE-Programm hat, die Gleitbewegung im Schnitt $A-A$ darzustellen, f\"{u}hren also zu einem gro{\ss}en Fehler an einer anderen Stelle, der Oberkante der Scheibe. Die Konsequenz ist, dass von einer Belastung von 1 kN an der Oberkante der Scheibe das 2.3-fache im Schnitt $A-A$ ankommt, was wahrlich ein gro{\ss}er Fehler ist.

Wie man in Abb. \ref{1GreenF156} d sieht, kann man durch eine adaptive Verfeinerung des Netz den Fehler in der oberen Auslenkung deutlich von 2.3 - 1.0 auf 1.2 - 1.0 verringern.\\
%----------------------------------------------------------------------------
\begin{figure}[tbp]
\centering
\if \bild 2 \sidecaption \fi \label{Korrektur4}
\includegraphics[width=1.0\textwidth]{\Fpath/U369}
\caption{Zugstab mit sich ver\"{a}nderndem Querschnitt und Einflussfunktion f\"{u}r das Stab\-ende \textbf{ a)} System \textbf{ b)} Einflussfunktion f\"{u}r Verschiebung des Stabendes \textbf{ c)} dieselbe Figur kann man auch an einem Stab $EA = 1$ mit Hilfe von Kr\"{a}ften $j^+$ erzeugen, s. S. \pageref{jplus} } \label{U369}
\end{figure}%
%----------------------------------------------------------------------------
Es gilt auch, dass, wenn man den Schnitt $A-A $ genau durch die {\em Mitte\/} der Elementreihe f\"{u}hrt, sich die Oberkante wirklich um 1 m nach rechts bewegt, also die Belastung an der Oberkante richtig im Schnitt $A-A$ ankommt. Die resultierende Scherkraft $N_{yx}$ in der Mitte des Elements ist dann also exakt.

Wenn man das sogenannte {\em Wilson-Element\/}\index{Wilson-Element} benutzt, \cite{Ha5}, dann werden die Ergebnisse auch f\"{u}r einen Schnitt $A-A $ richtig, der nicht durch die Mitte der Elementreihe geht. Nur den Schnitt innerhalb des Elements selbst, die abrupte Scherbewegung, kann man auch mit dem Wilson-Element nur \glq gleitend\grq{} nachvollziehen.

{\textcolor{sectionTitleBlue}{\subsubsection*{Ursachen}}}
{\em Pollution\/} hat im wesentlichen drei Ursachen, \cite{Babuska5}:\\

\begin{enumerate}
  \item Unstetigkeiten in der Belastung, also zum Beispiel Spr\"{u}nge in der Verkehrslast einer Platte (ein mildes Problem)
  \item Gro{\ss}e und kleine Elemente nebeneinander, also ein unausgewogenes Netz; andererseits gilt aber, dass man mit gradierten Netzen {\em pollution\/} d\"{a}mpfen kann
  \item Nicht-glatte Einflussfunktionen, s. Abb. \ref{U369}
\end{enumerate}
Wenn die Koeffizienten in der Differentialgleichung springen, weil z.B. die Dicke $d$ der Platte springt, dann schl\"{a}gt das auf die Einflussfunktionen durch, d.h. die Regularit\"{a}t (\glq die Gl\"{a}tte\grq{}) der Einflussfunktionen nimmt ab, es treten Knicke auf wie in Abb. \ref{U369} b und damit vergr\"{o}{\ss}ert sich die {\em pollution\/}, weil das FE-Programm mehr M\"{u}he hat, solche Einflussfunktionen zu approximieren.

Alle FE-L\"{o}sungen -- bis auf Stabtragwerke ($EA$ und $EI$ konstant) -- weisen einen globalen Fehler auf, k\"{a}mpfen mit {\em pollution\/}.

Anschaulich \"{a}u{\ss}ert sich der Fehler als {\em drift\/}\index{drift} in den Knoten, \cite{Ha5}. Die exakte L\"{o}sung und die FE-L\"{o}sung sind in den Knoten nicht deckungsgleich. Das muss im Grunde auch so sein, es ist ein \glq Qualit\"{a}tsmerkmal\grq{}, weil die Interpolierende eine schlechtere L\"{o}sung als die FE-L\"{o}sung ist -- die Fehler in den Spannungen sind im Mittel gr\"{o}{\ss}er als bei der FE-L\"{o}sung.

Nur bei Stabtragwerken fallen Interpolierende $u_I$ und FE-L\"{o}sung $u_h$ zusammen, ist der Fehler in den Spannungen der beiden L\"{o}sungen daher gleich. Dieser Fehler wird behoben, indem das FE-Programm stabweise die lokalen L\"{o}sungen zur FE-L\"{o}sung addiert und so auf die exakten Ergebnisse kommt. Vergisst man diesen Schritt, dann ist die FE-L\"{o}sung $u_h = u_I$ nicht besser als die Interpolierende und die \glq L\"{u}cken\grq{} $u - u_h$ (\glq B\"{o}gen\grq{}, die von Knoten zu Knoten spannen) sind die  {\em lokalen Fehler\/}\index{lokaler Fehler} der FE-L\"{o}sung.\\

{\em Interpolierende:\/} \index{Interpolierende} Wenn man die exakte L\"{o}sung $u$ kennt und man interpoliert sie in den Knoten mit den {\em shape functions\/} aus $\mathcal{V}_h$, dann ist das die Interpolierende $u_I$ auf dem FE-Netz.

Viele Fehlersch\"{a}tzer basieren auf der Tatsache, dass der Fehler der FE-L\"{o}sung $\|u - u_h\|$ (gemessen in der Energienorm $\sim$ Fehlerquadrat der Spannungen) kleiner als der Fehler $\|u - u_I\|$ der Interpolierenden ist. Netze, auf denen man Funktionen gut interpolieren kann, sind auch gute FE-Netze, weil die FE-L\"{o}sung immer noch eine Idee besser ist als die Interpolierende.




