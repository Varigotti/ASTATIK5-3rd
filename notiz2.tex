\documentclass[graybox,envcountchap,sectrefs]{svmono}
%\documentclass[12pt]{article}
\usepackage{latexsym}
\usepackage{ifpdf}
\oddsidemargin0.12cm \evensidemargin0.12cm \topmargin-1.2cm \headheight0.7cm \headsep 0cm
\textwidth16.5cm
\parindent0cm
\textheight25cm

\ifpdf
\newcommand{\Fpath}{d:/astatikbilder/pdf}
%\newcommand{\Fpath}{C:/Users/Friedel/OneDrive/Dokumente/ASTATIK3/Pdf}
\else
\newcommand{\Fpath}{d:/astatikbilder/pictures}
%\newcommand{\Fpath}{C:/Users/Friedel/OneDrive/Dokumente/ASTATIK3/Pictures}
\fi
\usepackage{multicol}
\usepackage{graphicx}
\usepackage{calligra}
\usepackage{amssymb}
\usepackage{amsmath}
\usepackage{wasysym}
\usepackage{amssymb}
\usepackage{amsmath}
\usepackage{dsfont}
\usepackage{blindtext}
\usepackage{german}
\usepackage{deutengi}
\usepackage{float}
\usepackage[]{graphicx}        % standard LaTeX graphics tool
\newcommand{\bild}{2}
\newcommand{\vek}[1]{\mbox{\boldmath $#1$}}
\newenvironment{Eqnarray} {\arraycolsep 0.14em\begin{eqnarray}}{\end{eqnarray}}
\newcommand {\bfo}    {\begin {Eqnarray}}
\newcommand {\efo}    {\end   {Eqnarray}}
\newcommand {\nn} {\nonumber}
\newcommand {\dotprod}{{\,\scriptscriptstyle \stackrel{\bullet}{{}}}\,}
\newcommand {\barr}   {\begin {array}}
\newcommand {\earr}   {\end {array}}
\newcommand{\Np}{\varphi}
\def\strut{\rule{0in}{.50in}}
\newcommand{\lqq}{\lq\lq}
\newcommand{\rqq}{\rq\rq \,}
\newcommand{\beq}{\begin{equation}}
\newcommand{\eeq}{\end{equation}}

\newenvironment{EqnarrayNN} {\arraycolsep 0.14em\begin{eqnarray*}}{\end{eqnarray*}}
\newcommand {\bfoo}    {\begin {EqnarrayNN}}       % bfo
\newcommand {\efoo}    {\end {EqnarrayNN}}         % efo
\newcommand{\hlq}{\glq\kern.07em\allowhyphens}   % Frank Holzwarth  Januar 2001
\input{ebild}
\begin{document}
\pagestyle{empty}
t = 0.2\,m


\end{document}
$a\,\,b\,\,x\,\,y\,\,\ell\,u_a\,\,v_a\,\,u_b\,\,v_b\,\,\Np_a\,\,\Np_b$\\
$F_{x,a}$\nn
$F_{y,a}$\nn
$F_{x,b }$\nn
$F_{y,b}$\nn
$M_a$ \\
$M_b$
Verschiebungsgr\"{o}{\ss}en \\
Kraftgr\"{o}{\ss}en\\
(fem)\\
\begin{align}
u_i\,\,u_j\,\,u_k\,\,i\,\,j\,\,k\,\,x\,\,y\,\,u_1\,\,u_2\,\,u_3\,\,v_1, v_2\,\,v_3\,a\,\,d\,1\,\,2\,\,3 \nn
\end{align}
\begin{align}
F_{x,i}\,\,F_{y,i}\,\,F_{x,j}\,\,F_{y,j}F_{x,k}\,\,F_{y,k}\,\,\nn
\end{align}
$p_{x,1}$ \\
$p_{x,2}$ \\
$p_{x,3}$ \\
$p_{y,1}$ \\
$p_{y,2}$ \\
$p_{y,3}$ \\
$p_{y,m,1}$ \\
$p_{y,m,2}$ \\
$F_{B,x}$ \\
$F_{B,y}$ \\
$M_B$ \\
$x_B$ \\
$y_B$ \\
$F_{x,1}$ \\
$F_{x,2}$ \\
$F_{x,3}$ \\
$F_{y,1}$ \\
$F_{y,2}$ \\
$F_{y,3}$ \\




\hspace*{-12pt}\colorbox{highlightBlue}{\parbox{0.98\textwidth}{...}}\\

{\textcolor{blue}{\subsubsection*{Modellfehler}}}

{\textcolor{blue}{\section{Mengenlehre}}}

\textbf{ a)}

\text{\normalfont\calligra G\,\,}(\Np_i,\Np_j)

\pageref{SecAdaptiveVerfeinerung}

\begin{subequations}

\textcolor{red}{\delta w}

\overset{?}{=}

\backmatter

\text{\normalfont\calligra B\,\,}(w,\textcolor{red}{\delta w}) =  A_{1,2} - A_{2,1} = 0

 (\ref{Eq92:SubEq3})

 %%% mondgr\"{u}n  sand
 \barr {r @{\hspace{2mm}} r @{\hspace{2mm}} r @{\hspace{2mm}} r} 1 & -1 & 0 & 0 \\ -1 & 1 & 0 & 0 \\ 0 & 0 & 1 & -1 \\ 0 & 0 & -1 & 1 \\ \earr \right] \left[ \barr{cc} u_1^a \\ u_2^a \\ u_1^b \\ u_2^b \earr \right] = \vek K_E\,\vek u_E\,.

 \boxed{O_c - O = -\frac{\Delta\,EI}{EI}\int_{x_a}^{\,x_b} \frac{M_c\,M_G}{EI_c}\,dy}

\allowdisplaybreaks{4}

CMYK 20 0 60 0

\glq fast alles\grq\

kurz nachtragen, warum
\begin{align}
\text{\normalfont\calligra G\,\,}(u_h,\Np_i) = 0 \quad i = 1,2,\ldots, n
\end{align}
die Grundgleichung der FEM ist, s. S. \pageref{Eq190}.



@{\hspace{2mm}}

$^\circ$

Um die Bewegung $\Np_i$ zu erzeugen, sind Kr\"{a}fte n\"{o}tig, die wir den Lastfall $p_i$ nennen und die wir uns hier, der Einfachheit halber als eine Streckenlast $p_i$ vorstellen. Die zu dem Lastfall $p_i$ geh\"{o}rigen \"{a}quivalenten Knotenkr\"{a}fte sind also komponentenweise ($j$)
\begin{align}
f_{ij}    = \int_0^{\,l} p_i\,\Np_j\,dx \qquad (\delta A_a)\,.
\end{align}
Wegen der ersten Greenschen Identit\"{a}t, $\text{\normalfont\calligra G\,\,}(\Np_i,\Np_j) = \delta A_a - \delta A_i = 0$, sind diese \"{a}u{\ss}eren Arbeiten gleich der inneren Arbeit, also der Wechselwirkungsenergie $f_{ij} = a(\Np_i,\Np_j) = k_{ij}$, und so sind die Spalten von $\vek K$ gerade die \"{a}quivalenten Knotenkr\"{a}ften der Kr\"{a}fte $p_i$, die die $\Np_i$ erzeugen. 