\documentclass[graybox,envcountchap,sectrefs]{svmono}
%\documentclass[12pt]{article}
\usepackage{latexsym}
\usepackage{ifpdf}
\oddsidemargin0.12cm \evensidemargin0.12cm \topmargin-1.2cm \headheight0.7cm \headsep 0cm
\textwidth16.5cm
\parindent0cm
\textheight25cm

\ifpdf
\newcommand{\Fpath}{d:/astatikbilder/pdf}
%\newcommand{\Fpath}{C:/Users/Friedel/OneDrive/Dokumente/ASTATIK3/Pdf}
\else
\newcommand{\Fpath}{d:/astatikbilder/pictures}
%\newcommand{\Fpath}{C:/Users/Friedel/OneDrive/Dokumente/ASTATIK3/Pictures}
\fi
\usepackage{multicol}
\usepackage{graphicx}
\usepackage{calligra}
\usepackage{amssymb}
\usepackage{amsmath}
\usepackage{wasysym}
\usepackage{amssymb}
\usepackage{amsmath}
\usepackage{dsfont}
\usepackage{blindtext}
\usepackage{german}
\usepackage{deutengi}
\usepackage{float}
\usepackage[]{graphicx}        % standard LaTeX graphics tool
\newcommand{\bild}{2}
\newcommand{\vek}[1]{\mbox{\boldmath $#1$}}
\newenvironment{Eqnarray} {\arraycolsep 0.14em\begin{eqnarray}}{\end{eqnarray}}
\newcommand {\bfo}    {\begin {Eqnarray}}
\newcommand {\efo}    {\end   {Eqnarray}}
\newcommand {\nn} {\nonumber}
\newcommand {\dotprod}{{\,\scriptscriptstyle \stackrel{\bullet}{{}}}\,}
\newcommand {\barr}   {\begin {array}}
\newcommand {\earr}   {\end {array}}
\newcommand{\Np}{\varphi}
\def\strut{\rule{0in}{.50in}}
\newcommand{\lqq}{\lq\lq}
\newcommand{\rqq}{\rq\rq \,}
\newcommand{\beq}{\begin{equation}}
\newcommand{\eeq}{\end{equation}}

\newenvironment{EqnarrayNN} {\arraycolsep 0.14em\begin{eqnarray*}}{\end{eqnarray*}}
\newcommand {\bfoo}    {\begin {EqnarrayNN}}       % bfo
\newcommand {\efoo}    {\end {EqnarrayNN}}         % efo
\newcommand{\hlq}{\glq\kern.07em\allowhyphens}   % Frank Holzwarth  Januar 2001
\input{ebild}
\begin{document}
\pagestyle{empty}
6.2.4 (2): Gl. 6.10:\\
Widerstand Normalkraft\\
\begin{align}
N_{c,Rd} = \frac{A \cdot f_y }{\gamma_{M0}} \nn
\end{align}
\begin{align}
N_{c,Rd} = 8091\,\text{kN} \nn
\end{align}
\begin{align}
n = N_{Ed} / N_{c,Rd} = 0.618\nn
\end{align}

6.2.5 (2): Gl. 6.13: \\
Widerstand Moment\\
\begin{align}
M_{y,Rd} = \frac{W_{y,pl} \cdot f_y}{\gamma_{M0}}\nn
\end{align}
\begin{align}
W_{y,pl} = 7094.2\, \text{cm}^3\nn
\end{align}
\begin{align}
M_{y,Rd} = 1667\,\text{kNm}\quad \frac{M_{Ed}}{M_{y,Rd}} = 0.270\nn
\end{align}

6.2.6 (2): Gl. 6.18 \\
Widerstand Querkraft \\

\begin{align}
V_{z,Rd} = \frac{A_{Vz \cdot f_y}}{\sqrt{3} \cdot \gamma_{M0}} \nn
\end{align}

\begin{align}
V_{z,Rd} = 1757\,\text{kN} \nn
\end{align}
\begin{align}
V_{z,Ed}/V_{z,Rd} = 0.797
\end{align}

6.2.6 (3)a: Die wirksame\\ Schubfl\"{a}che darf f\"{u}r\\ gewalzte Profile mit\\ I- und H-Querschnitten, \\ Lastrichtung parallel\\
 zum Steg\\

 \begin{align}
 A_{Vz} = A - 2\,b \cdot t_f + (t_w + 2 \cdot r) \cdot t_f \nn
 \end{align}

 aber mindestens $\eta \cdot h_w \cdot t_w$\\

 \begin{align}
  A_{Vz} = 344.3 - 2 \cdot 30.6 \cdot 4 = 99.5\,\text{cm}^2 \nn
 \end{align}
 \begin{align}
 a = 99.5/344.3 = 0.289\nn
 \end{align}

 6.2.10 (3) Die \\
 Momententragf\"{a}higkeit\\
  f\"{u}r auf Biegung und\\
 Normalkraft\\
  beanspruchte\\
   Querschnitte ist mit einer\\
    abgeminderten\\
Streckgrenze: f\"{u}r die\\
 wirksamen Schubfl\"{a}chen\\
  zu ermitteln.\\

\begin{align}
\rho = \bigg[ \frac{2 \cdot V_{Ed}}{V_{p,Rd,z}} - 1 \bigg]^2\nn
\end{align}
\begin{align}
\Rightarrow \,N_{V,Rd} = N_{c,Rd} \cdot (1 - a_{Vz} \cdot \rho)\nn
\end{align}

\begin{align}
\rho = \bigg[\frac{ 2 \cdot 1400}{1757} - 1 \bigg]^2 = 0.352 \nn
\end{align}
\begin{align}
\Rightarrow \,N_{V,Rd} = 8091 \cdot (1 - 0.376 \cdot 0.352) = 7020\,\text{kN}\nn
\end{align}
\begin{align}
n = \frac{N_{Ed}}{N_{V,Rd}} = \frac{5000}{7020} = 0.712 \nn
\end{align}


6 2.8 	Gl. 6 30: Bei l-\\
Querschnitten mit\\
 gleichen Flanschen\\
  und einachsiger\\
   Biegung um die\\
    Hauptachse

\begin{align}
M_{y,V,Rd} = \bigg[ W_{pl,y} - \frac{\rho \cdot A_w^2}{4 \cdot t_w} \bigg] \cdot f_y \nn
\end{align}
\begin{align}
A_w = h_w \cdot t_w\,;\quad  \frac{A_w^2}{t_w} = A_w \cdot h_w \nn
\end{align}
\begin{align}
M_{y,V,Rd} = \bigg[ 7094 - \frac{0.352 \cdot 93.24^2}{4 \cdot 2.1} \bigg] \cdot f_y \nn
\end{align}
\begin{align}
M_{y,V,Rd} = 1581\,\text{kNm}\nn
\end{align}

Das ist die Reduktion der Stegfl\"{a}che\\

\begin{align}
M_{y,NV,Rd} = \text{min}\bigg[1, \frac{ 1 - n}{1 - 0.5 \cdot a} \bigg] \cdot M_{y,V,Rd}\nn
\end{align}
\begin{align}
a = \text{min} [0.5, (A - 2 \cdot b \cdot t)/A] \nn
\end{align}
\begin{align}
M_{y,NV,Rd} = \frac{ 1 - 0.712}{1 - 0.5 \cdot 0.289} \cdot 1581 = 531\,\text{kNm}\nn
\end{align}
\begin{align}
a = (344.3 - 2 \cdot 30.6 \cdot 4) /344.3 = 0.289\nn
\end{align}
\begin{align}
\frac{M_Ed}{M_{y,Nv,Rd}} = \frac{450}{531} = 0.846 \nn
\end{align}
\end{document}


\hspace*{-12pt}\colorbox{hellgrau}{\parbox{0.98\textwidth}{...}}\\

{\textcolor{blue}{\subsubsection*{Modellfehler}}}

{\textcolor{blue}{\section{Mengenlehre}}}

\textbf{ a)}

\text{\normalfont\calligra G\,\,}(\Np_i,\Np_j)

\pageref{SecAdaptiveVerfeinerung}

\begin{subequations}

\textcolor{red}{\delta w}

\overset{?}{=}

\backmatter

\text{\normalfont\calligra B\,\,}(w,\textcolor{red}{\delta w}) =  A_{1,2} - A_{2,1} = 0

 (\ref{Eq92:SubEq3})

 %%% mondgr\"{u}n  sand
 \barr {r @{\hspace{2mm}} r @{\hspace{2mm}} r @{\hspace{2mm}} r} 1 & -1 & 0 & 0 \\ -1 & 1 & 0 & 0 \\ 0 & 0 & 1 & -1 \\ 0 & 0 & -1 & 1 \\ \earr \right] \left[ \barr{cc} u_1^a \\ u_2^a \\ u_1^b \\ u_2^b \earr \right] = \vek K_E\,\vek u_E\,.

 \boxed{O_c - O = -\frac{\Delta\,EI}{EI}\int_{x_a}^{\,x_b} \frac{M_c\,M_G}{EI_c}\,dy}

\allowdisplaybreaks{4}

CMYK 20 0 60 0

\glq fast alles\grq\

kurz nachtragen, warum
\begin{align}
\text{\normalfont\calligra G\,\,}(u_h,\Np_i) = 0 \quad i = 1,2,\ldots, n
\end{align}
die Grundgleichung der FEM ist, s. S. \pageref{Eq190}.



@{\hspace{2mm}}

$^\circ$ 