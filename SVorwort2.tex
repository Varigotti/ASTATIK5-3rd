
\begin{flushleft}\large{\bf{Zweite Auflage}} \end{flushleft}

Das elektronische Format macht es m\"{o}glich, \"{A}nderungen und Erg\"{a}nzungen, gleich einem nie endenden {\em work in progress\/}, kontinuierlich in den Text einflie{\ss}en zu lassen. Nach zwei Jahren und +150 Seiten haben wir uns jedoch dazu entschlossen, die zweite Auflage \glq einzufrieren\grq{} und auf den Server der Universit\"{a}tsbibliothek Kassel (KOBRA) zu legen. Sie kann von dort \"{u}ber den (permanent g\"{u}ltigen, \glq zitierf\"{a}higen\grq{}) link\\

\href{http://nbn-resolving.de/urn:nbn:de:hebis:34-2018030554714}{http://nbn-resolving.de/urn:nbn:de:hebis:34-2018030554714}
%http://nbn-resolving.de/urn:nbn:de:hebis:34-2018030554714\\

\begin{flushleft} heruntergeladen werden. \end{flushleft}
Die bei Kassel-University-Press erh\"{a}ltliche gedruckte Version des Buchs ist auf dem Stand der 1. Auflage vom Fr\"{u}hjahr 2016, ebenso der pdf-file auf der Seite der Kassel-University-Press.


\begin{flushleft}\large{\bf{Dritte Auflage}} \end{flushleft}

Im Fr\"{u}hjahr 2018 haben wir mit der dritten Auflage begonnen. Die jeweils tages-aktuelle pdf-Version der dritten Auflage liegt auf den Seiten\\

 \href{http://simplel.ink/go/pdf}{http://simplel.ink/go/pdf} und \href{http://simplel.ink/go/winfemBook}{http://simplel.ink/go/winfemBook}.

\begin{flushright}\noindent
Kassel  {\hfill {\it Friedel Hartmann, Peter Jahn}}\\\vspace{0.1cm}
April 2018   {\hfill {hartmann@be-statik.de, PJahn@uni-kassel.de}}\\
\end{flushright}

\vspace{1 cm}
\begin{acknowledgement}
Her Kollege Werkle, Hochschule Konstanz, hat uns bei der korrekten Formulierung der {\em \"{A}quivalenten Spannungs Tranformation\/}, Abschnitt 3.11, tatkr\"{a}ftig unterst\"{u}tzt. Daf\"{u}r sei ihm an dieser Stelle gedankt.\\
\end{acknowledgement}

