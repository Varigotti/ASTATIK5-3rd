%%%%%%%%%%%%%%%%%%%%%%%%%%%%%%%%%%%%%%%%%%%%%%%%%%%%%%%%%%%%%%%%%%%%%%%%%%%%%%%%%%%%%%%%%%%%%%%%%%%
\textcolor{chapterTitleBlue}{\chapter{Finite Elemente}}\index{finite Elemente}\label{Chap3}
%%%%%%%%%%%%%%%%%%%%%%%%%%%%%%%%%%%%%%%%%%%%%%%%%%%%%%%%%%%%%%%%%%%%%%%%%%%%%%%%%%%%%%%%%%%%%%%%%%%
Zur Vorbereitung auf das Thema finite Elemente und Einflussfunktionen wollen wir kurz die Grundlagen der finiten Elemente rekapitulieren.

%%%%%%%%%%%%%%%%%%%%%%%%%%%%%%%%%%%%%%%%%%%%%%%%%%%%%%%%%%%%%%%%%%%%%%%%%%%%%%%%%%%%%%%%%%%%%%%%%%%
{\textcolor{sectionTitleBlue}{\section{Das Minimum}}}
Das {\em Prinzip vom Minimum der potentiellen Energie\/} besagt, dass die Gleichgewichtslage eines Tragwerks die potentielle Energie des Tragwerks zum Minimum macht. Zur Konkurrenz zugelassen sind bei diesem Wettbewerb alle Funktionen, die die geometrischen Lagerbedingungen des Tragwerks erf\"{u}llen. Man nennt diese Menge \"{u}blicherweise $\mathcal{V}$ (wie  \glq Vorrat\grq{}).

So ist die Biegelinie des Seils in Abb. \ref{U33}
\begin{align}\label{Eq88}
- H\,w''(x) = p(x) \qquad w(0) = w(l) = 0 \qquad \text{$H$ = Vorspannung in dem Seil}\,,
\end{align}
der Sieger, wenn es darum geht, die potentielle Energie des Seils
\begin{align}
\Pi(w) = \frac{1}{2}\,\int_0^{\,l} \frac{V^2}{H}\,dx - \int_0^{\,l} p(x)\,w(x)\,dx \qquad (V = H\,w')
\end{align}
auf der Menge aller Funktionen, deren Durchbiegungen in den Aufh\"{a}ngepunkten null sind, $w(0) = w(l) = 0$, zum Minimum zu machen.

%----------------------------------------------------------
\begin{figure}[tbp]
\centering
\if \bild 2 \sidecaption[t] \fi
\includegraphics[width=.85\textwidth]{\Fpath/U33A}
\caption{FE-Berechnung eines Seils, \textbf{a)} System und Belastung, \textbf{ b)} Dach- oder H\"{u}tchenfunktionen, \textbf{ c)} FE-L\"{o}sung $w_h(x)$ ,
\textbf{ d)} Vergleich $w(x)$ und $w_h(x)$} \label{U33}
\end{figure}%%
%----------------------------------------------------------

Nun ist es nicht m\"{o}glich, den ganzen Raum $\mathcal{V}$ zu durchsuchen, um $w(x)$ zu finden, dazu ist er zu gro{\ss}, und so beschr\"{a}nken wir die Suche auf einen {\em endlichdimensionalen\/} Teilraum $\mathcal{V}_h \subset \mathcal{V}$ und erkl\"{a}ren den Sieger  $w_h$ des Wettbewerbs um das Minimum auf diesem Teilraum als die beste N\"{a}herung. Dies nennt man das {\em Verfahren von Ritz\/}\index{Verfahren von Ritz}.

Der Wettbewerb beginnt damit, dass wir das Seil in mehrere finite Elemente unterteilen. Das ist einfach ein St\"{u}ck Seil, (so sieht es der Ingenieur), bzw. ein St\"{u}ck der $x$-Achse, (so sieht es der Mathematiker) auf dem zwei lineare Funktionen definiert sind, die sogenannten {\em Element-Einheitsverformungen\/}, die die Auslenkung des linken bzw. des rechten Knotens des Elements beschreiben.  Indem man nun diese Verformungen \"{u}ber die Elementgrenzen hinweg geeignet fortsetzt, kann man \glq H\"{u}tchenfunktionen\grq{}\index{H\"{u}tchenfunktionen} konstruieren. Das sind st\"{u}ckweise lineare Verl\"{a}ufe $\Np_i(x)$, die in dem Knoten $x_i$  den Wert 1 haben und zu den Nachbarknoten hin auf null abfallen, s. Abb. \ref{U33}. Sie stellen die {\em Einheitsverformungen der Knoten\/}\index{Einheitsverformungen} dar, sie sind die {\em shape functions\/}.

Die vier Einheitsverformungen der vier innenliegenden Knoten bilden also den FE-Ansatz
\begin{align}\label{Eq112}
w_h(x) = w_1\,\Np_1(x) + w_2\,\Np_2(x) + w_3\,\Np_3(x) + w_4\,\Np_4(x)\,,
\end{align}
und wir bestimmen die Knotenverformungen $w_i$ so, dass die FE-L\"{o}sung die potentielle Energie
\begin{align}\label{Eq46}
\Pi(w_h)= \frac{1}{2}\,\int_0^{\,l} H\,(w_h')^2\,dx - \int_0^{\,l} p\,w_h\,dx
\end{align}
auf $\mathcal{V}_h$, das ist die Menge aller Seilecke, die sich mit den $\Np_i(x)$ darstellen lassen, zum Minimum macht.
%----------------------------------------------------------
\begin{figure}[tbp]
\centering
\if \bild 2 \sidecaption[t] \fi
\includegraphics[width=.7\textwidth]{\Fpath/U122}
\caption{FE-Modell eines Seils, \textbf{a)} Ansatzfunktionen, \textbf{ b)} Einflussfunktion (EF) f\"{u}r $w(x_1)$ und \textbf{ c)} f\"{u}r die Durchbiegung $w(x)$ im Zwischenpunkt, \textbf{ d)} die exakte Einflussfunktion f\"{u}r $w(x)$} \label{U122}
\end{figure}%%
%----------------------------------------------------------

Der Ansatz (\ref{Eq112}) gewinnt den Wettbewerb genau dann, $\partial \Pi(w_h)/ \partial w_i = 0$ f\"{u}r $i = 1,2,3,4$, wenn der Vektor $\vek w$ der Knotenverformungen (die \glq Adresse\grq{} des Ansatzes auf $\mathcal{V}_h$) dem Gleichungssystem $\vek K\,\vek w = \vek f$, oder
\beq\label{Eq69}
\frac{H}{l_e} \,  \left[\barr{r r r r} 2 & - 1 & 0 & 0 \\ - 1 & 2 & -1 & 0\\ 0 & -1 & 2 &-1 \\ 0 & 0 & -1 &2\earr\right]
\,\left[\barr{c} w_1 \\w_2 \\ w_3 \\ w_4 \earr \right] = \left[\barr{c} 1 \\ 1  \\
1  \\ 1 \earr \right]
\eeq
gen\"{u}gt. Die Elemente $k_{ij}$ der Steifigkeitsmatrix $\vek K$ sind die Wechselwirkungsenergien zwischen den Ansatzfunktionen
\begin{align}
k_{ij} = a(\Np_i,\Np_j) = \int_0^{\,l} H\,\Np_i'(x)\,\Np_j'(x)\,dx = \int_0^{\,l} \frac{V_i\,V_j}{H}\,dx\,,
\end{align}
und die \"{a}quivalenten Knotenkr\"{a}fte auf der rechten Seite, $f_i = 1$, sind die Integrale
\begin{align}
f_i = \int_0^{\,l} p(x)\,\Np_i(x)\,dx\, .
\end{align}
Das System (\ref{Eq69}) hat, bei einer Vorspannung $H = 1$ und einer Elementl\"{a}nge $l_e = 1$, die L\"{o}sung
\beq
w_1 = w_4 = 2 \qquad w_2 = w_3 =  3\,,
\eeq
und daher ist das Seileck
\beq\label{A11Resultat}
w_h(x) = 2 \cdot  \Np_1(x) + 3 \cdot \Np_2(x) + 3 \cdot \Np_3(x) + 2 \cdot \Np_4(x)
\eeq
auf $\mathcal{V}_h$ die beste Ann\"{a}herung  an die wahre Biegelinie $w(x)$.


%%%%%%%%%%%%%%%%%%%%%%%%%%%%%%%%%%%%%%%%%%%%%%%%%%%%%%%%%%%%%%%%%%%%%%%%%%%%%%%%%%%%%%%%%%%%%%%%%%%
{\textcolor{sectionTitleBlue}{\section{Warum die Knotenwerte beim Seil exakt sind}}}
Wenn man die FE-L\"{o}sung mit der exakten L\"{o}sung
\begin{align}
w(x) = \frac{1}{2}\,\cdot (5\,x - x^2)
\end{align}
vergleicht, dann f\"{a}llt auf, dass die FE-L\"{o}sung mit der exakten L\"{o}sung in den Knoten \"{u}bereinstimmt, $w_i = w(x_i)$. Das  liegt daran, dass das FE-Programm Einflussfunktionen benutzt, also die Durchbiegung des Seils mit der Formel\footnote{Wir schreiben k\"{u}rzer $G(y,x)$ statt $G_0(y,x)$ und werden das bei Gelegenheit \"{o}fter tun.}
\begin{align}
w(x) = \int_0^{\,l} G(y,x)\,p(y)\,dy
\end{align}
berechnet und die Einflussfunktionen der Knoten in $\mathcal{V}_h$\index{$\mathcal{V}_h$} liegen. $\mathcal{V}_h$ ist die Menge aller Biegelinien, die mit den $\Np_i(x)$ darstellbar sind.

Die Einflussfunktion f\"{u}r die Durchbiegung $w(x_1)$ im ersten Innenknoten ist  das Seileck $G(y,x_1)$, das sich ausbildet, wenn in dem Knoten $x_1$ eine Kraft $P = 1$ angreift, s. Abb. \ref{U122} b, und
%----------------------------------------------------------
\begin{figure}[tbp]
\centering
\if \bild 2 \sidecaption[t] \fi
\includegraphics[width=1.0\textwidth]{\Fpath/U189}
\caption{Hochbauplatte, \textbf{a)} System,  \textbf{ b)} Biegefl\"{a}che im LF $g$, \textbf{ c)} Einflussfunktion f\"{u}r die Durchbiegung $w$ in einem Knoten $\vek x$} \label{U189}
\end{figure}%%
%----------------------------------------------------------
dieses Seileck k\"{o}nnen die vier Ansatzfunktionen darstellen.

Das ist der Grund, warum in diesem, aber {\em auch in jedem anderen Lastfall\/}, die FE-L\"{o}sung mit der exakten L\"{o}sung im Knoten $x_1$ \"{u}bereinstimmt
\begin{align}
 w_h(x_1) = \int_0^{\,l} G(y,x_1)\,p(y)\,dy = A \cdot 1.0 = 2.0 \cdot 1.0 = w(x_1)\,.
\end{align}
Wir machen die Probe. Es sei $p(x) = \sin(\pi x/5)$, dann ist
\begin{align}
\vek f = \{0.569, \,0.920, \,0.920, \,0.569\}^T
\end{align}
und das System $\vek K\,\vek w = \vek f$ hat die L\"{o}sung $\vek w = \{1.489, 2.409, 2.409, 1.489\}^T$, was die Knotenwerte der exakten L\"{o}sung $w(x) = 25/\pi^2 \cdot \sin(\pi  x/5)$ sind.

Wenn der Aufpunkt aber zwischen zwei Knoten liegt wie in Abb. \ref{U122} c, er liegt im Punkt $x = 1.5$, dann hat das zu dem Punkt geh\"{o}rige Seileck seine Spitze zwischen den beiden Knoten, und ein solches Dreieck kann man mit den vier Ansatzfunktionen nicht darstellen. Das FE-Programm verbindet daher die beiden Knoten links und rechts vom Aufpunkt mit einer geraden Linie und rechnet mit dieser N\"{a}herung $G_h(y,x)$
\begin{align}
w_h(x) = \int_0^{\,l} G_h(y,x)\,p(y)\,dy = A_h \cdot 1.0 = 2.5 \neq 2.75 = w(x)\,,
\end{align}
und so ist das Ergebnis nat\"{u}rlich auch nur eine N\"{a}herung, $w_h(x) = 2.5$ m, w\"{a}hrend die exakte Durchbiegung den Wert $w = 2.75$ m hat.

Nun wird man einwenden wollen: ein FE-Programm berechnet doch die Knotenwerte durch L\"{o}sen des Gleichungssystems $\vek K\,\vek w = \vek f$ und die Werte dazwischen findet es, indem es zwischen den Knoten interpoliert.

Das ist richtig, aber die Werte in dem Vektor $\vek w$ sind genauso gro{\ss}, {\em als ob\/} das FE-Programm sie mit den gen\"{a}herten Einflussfunktionen berechnet h\"{a}tte. Das ist der entscheidende Punkt. Von der klassischen Statik zu den finiten Elementen ist es ein ganz, ganz kurzer Weg.

Und diese Vorgehensweise ist nat\"{u}rlich nicht auf die Stabstatik beschr\"{a}nkt. So hat das FE-Programm die Biegefl\"{a}che der Platte in Abb. \ref{U189} (theoretisch) so berechnet, dass es in jeden Knoten $\vek x_i$ nacheinander eine Kraft $P = 1$ gestellt hat und die sich dabei ausbildende Biegefl\"{a}che $G_h(\vek y,\vek x_i)$ mit dem Eigengewicht $g$ \"{u}berlagert hat
\begin{align}\label{Eq78}
w_h(\vek x_i) = \int_{\Omega} G_h(\vek y, \vek x_i)\,g(\vek y)\,\,d\Omega_{\vek y} = \text{Volumen von $G_h$ $\times\,g$}\,.
\end{align}
Wir sagen theoretisch, weil nat\"{u}rlich das FE-Programm die Knotenwerte durch das L\"{o}sen von $\vek K\,\vek w = \vek f$ bestimmt hat, aber diese sind  genau so gro{\ss}, {\em als ob\/} das FE-Programm die Einflussfunktion (\ref{Eq78}) benutzt h\"{a}tte.

Das System $\vek K\,\vek w = \vek f$ ist der \glq kurze\grq{} Weg zu den $w_i$, die Formel (\ref{Eq78}) ist der \glq lange\grq{} Weg, aber die Ergebnisse sind dieselben\footnote{Ungl\"{a}ubige Leser d\"{u}rfen die rechte Seite von (\ref{Eq189}) partiell integrieren}
\begin{align}\label{Eq189}
w_h(\vek x_i) = w_i = \sum_j\,k_{ij}^{(-1)}\,f_j = \int_{\Omega} G_h(\vek y, \vek x_i)\,g(\vek y)\,\,d\Omega_{\vek y}\,.
\end{align}
Dies ist das geheime, wenig bekannte Gesetz hinter den finiten Elementen. \\

\hspace*{-12pt}\colorbox{highlightBlue}{\parbox{0.98\textwidth}{So gut, wie die Einflussfunktionen sind, so gut sind die FE-Ergebnisse.}}\\

%----------------------------------------------------------
\begin{figure}[tbp]
\centering
\if \bild 2 \sidecaption[t] \fi
\includegraphics[width=.8\textwidth]{\Fpath/U163}
\caption{Seilberechnung mit zwei Elementen,  \textbf{a)} Belastung und Biegelinie, \textbf{ b)} FE-L\"{o}sung + lokale L\"{o}sungen, \textbf{ c)} lokale L\"{o}sungen, \textbf{ d)} Einheitsverformung des Knotens. Bemerkenswert ist, dass die Tangente im Mittenknoten automatisch stetig ist (kein Knick!), kein Sprung in der Querkraft $V = H\,w'$} \label{U163}
\end{figure}%%
%----------------------------------------------------------

%%%%%%%%%%%%%%%%%%%%%%%%%%%%%%%%%%%%%%%%%%%%%%%%%%%%%%%%%%%%%%%%%%%%%%%%%%%%%%%%%%%%%%%%%%%%%%%%%%%
{\textcolor{sectionTitleBlue}{\section{Addition der lokalen L\"{o}sung}}}\index{lokale L\"{o}sung}
Wenn man die Durchbiegung des Seils mit einem FE-Programm berechnet, dann sieht man auf dem Bildschirm kein Seileck, sondern eine wohl geschwungene Parabel zweiten Grades, also die exakte Kurve. Wie macht das das FE-Programm? Das Programm geht genau so vor, wie wir das beschrieben haben:\\

\begin{itemize}
  \item Es unterteilt das Seil in kleine Elemente.
  \item Es reduziert die Belastung in die Knoten, es berechnet also die $f_i$\,.
  \item Es l\"{o}st das Gleichungssystem $\vek K\,\vek w = \vek f$\,.
\end{itemize}
Wenn es jetzt stehen bleiben w\"{u}rde, dann w\"{u}rde man auf dem Bildschirm ein Seileck sehen.

Es folgt nun aber noch ein weiterer Schritt. Das Programm berechnet f\"{u}r jedes Element die sogenannte {\em lokale L\"{o}sung\/}\index{lokale L\"{o}sung} $w_{loc}$. Das ist die Durchbiegung, die die Streckenlast an dem {\em beidseitig festgehaltenen Element\/} erzeugt, und diese wird elementweise zu dem Seileck addiert. So ist die exakte Seilkurve in Abb. \ref{U163} entstanden.

Im Grund ist das genau die Technik des Drehwinkelverfahrens. Das Drehwinkelverfahren reduziert alle Belastung in die Knoten, f\"{u}hrt dann einen Knotenausgleich durch und h\"{a}ngt zum Schluss feldweise die lokalen L\"{o}sungen ein.

Die finiten Elemente machen es nicht anders, denn das Gleichungssystem, das aufgestellt wird, lautet eigentlich
\begin{align}\label{Eq67}
\vek K\,\vek w = \vek f_K + \vek d\,.
\end{align}
Die $f_{K @i}$ sind die Einzelkr\"{a}fte, die direkt in den Knoten angreifen und die $d_i$ sind die in die Knoten reduzierten Lasten, die Lagerdr\"{u}cke aus der {\em domain load\/}.  Die Begriffe Lagerdruck ({\em actio\/})\index{Lagerdr\"{u}cke} und Festhaltekraft ({\em reactio\/}) sind spiegelbildlich. Die Lagerdr\"{u}cke $\times (-1)$ sind die Festhaltekr\"{a}fte.

In der FE-Literatur wird der Unterschied zwischen den $f_{K @i}$ und $d_i$ normalerweise verwischt, steht $f_i \equiv f_{K @i} + d_i$  f\"{u}r beide Anteile. Die $f_i$ in dem obigen Beispiel sind eigentlich die $d_i$, also die in die Knoten reduzierte Streckenlast,
\begin{align}\label{Eq93}
d_i^e = \int_0^{\,l_e} p\,\Np_i^e\,dx \qquad \Np_i^e = \text{Element-Einheitsverformungen}\,,
\end{align}
w\"{a}hrend die echten $f_{K @i}$ null sind, weil keine Einzelkr\"{a}fte in den Knoten angreifen.

Der Ingenieur wendet die Formel (\ref{Eq93}) links und rechts vom Knoten an, und addiert dann die beiden Beitr\"{a}ge $d_i^{L} + d_i^{R}$, w\"{a}hrend die finiten Elemente die Belastung $p$ mit den Knoteneinheitsverformungen $\Np_i = \Np_i^{L} + \Np_i^{R}$ gleich \glq in einem St\"{u}ck\grq{} \"{u}berlagern
\begin{align}
d_i = \int_0^{\,l} p\,\Np_i\,dx = d_i^{L} + d_i^{R}\,.
\end{align}
Die enge Verwandtschaft der finiten Elemente mit dem Drehwinkelverfahren beruht auf dieser Formel, denn die Einflussfunktionen f\"{u}r die \"{a}quivalenten Lagerkr\"{a}fte $d_i$ sind genau die Element-Einheits\-verformungen $\Np_i^e$. Ob man die Belastung in die Knoten reduziert (Drehwinkelverfahren) oder die \"{a}quivalenten Knotenkr\"{a}fte berechnet, ist dasselbe.\\

\hspace*{-12pt}\colorbox{highlightBlue}{\parbox{0.98\textwidth}{Bei Stabtragwerken ist die Methode der finiten Elemente mit dem Drehwinkelverfahren identisch. }}\\

Das L\"{o}sen des Gleichungssystems $\vek K\,\vek w = \vek  f_K + \vek d$ entspricht einem Knotenausgleich in einem Schritt. Elementweise werden dann nur noch die lokalen L\"{o}sungen dazu addiert.

So gelingt es also den finiten Elementen trotz ihrer beschr\"{a}nkten Kinematik, d.h. der Verwendung von
\begin{itemize}
  \item linearen Ans\"{a}tzen f\"{u}r die Element-L\"{a}ngsverschiebungen
  \item kubischen Polynomen f\"{u}r die Element-Durchbiegungen
\end{itemize}
die exakten Verformungen zu generieren; die lokalen L\"{o}sungen bringen den fehlenden \glq Schwung\grq{} in die Verformungsfigur. Die $u_{loc}$ bzw. $w_{loc}$ stehen in einer (aus der Statik-Literatur \"{u}bernommen) Bibliothek des FE-Programms und werden von dort bei Bedarf abgerufen.

%Bei Seilen und St\"{a}ben (L\"{a}ngsverschiebungen) sieht man die lokalen L\"{o}sungen, wenn man die Knotenwerte mit einem Lineal verbindet. Dann sind die Teile, die man noch zu den geraden Linien addieren  muss, die lokalen L\"{o}sungen.

All dies gilt genau genommen nur, wenn $EA$ bzw. $EI$ konstant sind, weil nur dann die Element-Einheitsverformungen $\Np_i^e$ homogene L\"{o}sungen der Stab- bzw. Balkendifferentialgleichung sind. Bei gevouteten Tr\"{a}gern liefern die finiten Elementen  also nur eine N\"{a}herung, was aber auch f\"{u}r das Drehwinkelverfahren gilt, denn die exakte Reduktion der Belastung in die Knoten bei gevouteten Tr\"{a}gern beherrscht auch das Drehwinkelverfahren nicht. Ganz zu schweigen von der Kenntnis der exakten Fortleitungszahlen in einem solchen Fall.

Die \"{A}quivalenz {\em Finite Elemente = Drehwinkelverfahren\/} bedeutet aber auch, dass es keinen Sinn macht, die einzelnen Stiele und Riegel eines Rahmens weiter in Elemente zu unterteilen. Es bringt nichts an Genauigkeit.


\begin{remark}
Die finiten Elemente werden gerne am Balken erkl\"{a}rt. Wir machen das auch. Damit die finiten Elemente aber finite Elemente bleiben, m\"{u}ssen wir uns darauf verst\"{a}ndigen, dass alle diese Demonstrationen sich auf den Zeitpunkt beziehen, {\em bevor\/} die lokale L\"{o}sung zur FE-L\"{o}sung addiert wird.
\end{remark}
%----------------------------------------------------------------------------------------------------------
\begin{figure}[tbp]
\centering
\if \bild 2 \sidecaption \fi
\includegraphics[width=1.0\textwidth]{\Fpath/UE338A}
\caption{Lokales und globales Koordinatensystem} \label{UE338}
\end{figure}%
%----------------------------------------------------------------------------------------------------------
\vspace{-1cm}
{\textcolor{blue}{\subsubsection*{Beispiel}}}
Da es wichtig ist, dieses Vorgehen zu verstehen, wollen wir die einzelnen Schritte an Hand des Systems in Abb.  \ref{UE338} erl\"{a}utern.

Im ersten Schritt stellt das FE-Programm die globale, nicht reduzierte Steifigkeitsmatrix $ \vek K_G$ der Gr\"{o}{\ss}e $9 \times 9$ auf. Es geht dabei (theoretisch) wie folgt vor:  Es schreibt das System zun\"{a}chst in entkoppelter Form (12 Gleichungen)
\begin{align}
\left[ \barr{c c} \vek K_1 & \vek 0 \\ \vek  0 & \vek K_2 \earr \right] \,\left[ \barr{c}\vek u_1 \\ \vek u_2 \earr \right] =\left[ \barr{c} \vek f_1 \\ \vek f_2\earr \right] + \left[ \barr{c} \vek d_1\\ \vek d_2\earr \right]
\end{align}
wobei $\vek K_i$ die Elementmatrizen sind, s. (\ref{Eq166}), und die Vektoren $\vek u_i, \vek f_i$ und $\vek d_i$ die zugeh\"{o}rigen Weg- und Kraftgr\"{o}{\ss}en an den Balkenenden sind. Die Kopplung der Balkenendverformungen $\vek u_i$ an die Knotenverformungen $\vek u$ kann man durch eine Matrix $\vek A$ beschreiben
\begin{align}
 \left[ \barr{c}\vek u_1 \\ \vek u_2 \earr \right]_{(12)} = \vek A_{(12 \times 9)}\,\vek u_{(9)}\,.
\end{align}
Gleichzeitig m\"{u}ssen die Balkenendkr\"{a}fte $\vek f_i + \vek d_i$ in jedem Knoten mit den \"{a}u{\ss}eren Knotenkr\"{a}ften $\vek f$ im Gleichgewicht sein, was gerade die Gleichung
\begin{align}
\vek A^T_{(9 \times 12)}\,(\left[ \barr{c} \vek f_1 \\ \vek f_2\earr \right]_{(12)} + \left[ \barr{c} \vek d_1\\ \vek d_2\earr \right]_{(12)}) = \vek f_{(9)}
\end{align}
ist und so kommt man auf die Beziehung
\begin{align}
\vek A^T \left[ \barr{c c} \vek K_1 & \vek 0 \\ \vek  0 & \vek K_2 \earr \right]\,\vek A\,\vek u = \vek K_G\,\vek u = \vek f
\end{align}
wobei die Matrix links die globale, nicht reduzierte Steifigkeitsmatrix $\vek K_G$ des Rahmens ist. Im Anschluss streicht das Programm die Spalten und Zeilen der Matrix $ \vek K_G$, die gesperrten Freiheitsgraden entsprechen (praktisch geht es anders vor), und reduziert so die Matrix $\vek K_G$ auf eine $5 \times 5$ Matrix $\vek K$.

Im zweiten Schritt berechnet das FE-Programm f\"{u}r jedes Element die \"{a}quivalenten Knotenkr\"{a}fte aus der verteilten Belastung\footnote{Ein oberer Index $e$ bedeutet, dass sich die Gr\"{o}{\ss}e auf das lokale KS des Elements bezieht}
\begin{align}
d_{i}^{@e} = \int_0^{\,l_e} p^e(x)\,\Np_i^{@e}(x)\,dx\,.
\end{align}
Die Notation ist symbolisch zu nehmen, weil $p^e(x)$ zwei Richtungen haben kann, in lokale $x_e$- oder lokale $z_e$-Richtung und die Einheitsverschiebungen $\Np_i^{@e}(x)$ sind entsprechend die korrespondierenden Verschiebungen, zwei in $x_e$-Richtung und vier in $z_e$-Richtung.

Eine $6 \times 6$ Matrix $\vek T_e$ ($\ldots \sin\,\alpha, \cos\,\alpha, \ldots$), s. (\ref{Eq72}), transformiert diese Vektoren $\vek d^e$ in das globale Koordinatensystem $\vek d_e = \vek T_e\,\vek d^e$, wo dann aus den $2 \times 6$ Komponenten die 5 Komponenten des Vektors
\begin{align}
\{\vek d_1,\, \vek d_2\} \rightarrow \vek d
\end{align}
zusammengestellt werden, der die \"{a}quivalenten Knotenkr\"{a}fte aus der verteilten Belastung in die 5 Richtungen $u_i$ enth\"{a}lt.

Im zentralen dritten Schritt l\"{o}st das Programm das System
\begin{align}\label{Eq160}
\vek K\,\vek u = \vek f_K + \vek d
\end{align}
und bestimmt damit den Vektor $\vek u$ der Knotenverschiebungen. Der Vektor $\vek f_K$ enth\"{a}lt nur Nullen, au{\ss}er $f_{K @3} = P$.

Im vierten Schritt werden die Knotenverschiebungen $u_i$ aus dem globalen Koordinatensystem in die lokalen Koordinatensysteme der einzelnen Elemente transformiert,
\begin{align}
\vek u^e = \vek T_e^T \vek u \qquad (\vek  T_e^T = \vek T_e^{-1})\,.
\end{align}
Im f\"{u}nften Schritt bestimmt das Programm f\"{u}r jedes Element anhand des Systems (jetzt wird im lokalen KS gerechnet)
\begin{align}
\vek K^e\,\vek u^e = \vek f^e + \vek d^e
\end{align}
den Vektor der Balkenendkr\"{a}fte $\vek f^e$
\begin{align}
\vek f^e = \vek K^e\,\vek u^e - \vek d^e\,,
\end{align}
wobei $\vek K^e$ die Elementmatrix der Gr\"{o}{\ss}e $6 \times 6$ ist, (\ref{Eq167}). Man beachte, dass die $f_i^e$ Balkenendkr\"{a}fte sind, w\"{a}hrend die $f_{K @i}$ in (\ref{Eq160}) Knotenkr\"{a}fte sind.

Wenn man die Balkenendkr\"{a}fte $f_1^{@e}, f_2^{@e}, f_3^{@e}$ am Anfang des Elements kennt und die Streckenlast, dann kann man mit den Gleichgewichtsbedingungen die Schnittgr\"{o}{\ss}en $N(x), M(x)$ und $V(x)$ jedem Punkt $x$ des Elements berechnen.

Im letzten sechsten Schritt addiert das FE-Programm schlie{\ss}lich die lokale L\"{o}sung zur FE-L\"{o}sung und kommt so zur selben L\"{o}sung wie das Drehwinkelverfahren.

In einem einzelnen Balken entspricht die ganze Prozedur der Aufteilung der Biegelinie $w(x)$ in eine homogene und eine partikul\"{a}re L\"{o}sung
\begin{align}
w(x) = w_h(x) + w_p(x)\,,
\end{align}
wobei die homogene L\"{o}sung eine Entwicklung nach den Einheitsverformungen ist
\begin{align}
w_h(x) = \sum_i\,u_i\,\Np_i(x)\,,
\end{align}
und die partikul\"{a}re L\"{o}sung ist die Biegelinie am eingespannten Balken.
In Glg. (\ref{Eq159}), S. \pageref{Eq159}, haben wir diese Aufspaltung vorgenommen.

Nach diesem kurzen Ausflug in die Notation $\vek K\,\vek u = \vek f_K + \vek d$  wollen wir im Folgenden zur \"{u}blichen Notation $\vek K\,\vek  u = \vek f$ zur\"{u}ckkehren, bei der die rechte Seite f\"{u}r $\vek f \equiv \vek f_K + \vek d$ steht.

Wir werden auch meist die Knotenwerte mit $u_i$ bezeichnen, auch wenn es Ergebnisse aus einer Balkenberechnung sind, weil $\vek K\,\vek u = \vek f$ die Standardnotation ist.


%----------------------------------------------------------------------------------------------------------
\begin{figure}[tbp]
\centering
\if \bild 2 \sidecaption \fi
\includegraphics[width=.9\textwidth]{\Fpath/U161}
\caption{Der Fehlervektor $\vek e$ steht senkrecht auf der Ebene, auf die projiziert wird, und alle Vektoren, die sich in Projektionsrichtung von $\vek x$ nicht unterscheiden, haben dasselbe Bild} \label{U161}
\end{figure}%
%----------------------------------------------------------------------------------------------------------
%%%%%%%%%%%%%%%%%%%%%%%%%%%%%%%%%%%%%%%%%%%%%%%%%%%%%%%%%%%%%%%%%%%%%%%%%%%%%%%%%%%%%%%%%%%%%%%%%%%
{\textcolor{sectionTitleBlue}{\section{Projektion}}}
Wir hatten oben das System $\vek K\,\vek u = \vek f$ aus dem Prinzip vom Minimum der potentiellen Energie hergeleitet. Die Projektion der exakten L\"{o}sung auf den Ansatzraum $\mathcal{V}_h$, also das {\em Galerkin-Verfahren\/}, f\"{u}hrt jedoch, wie wir zeigen wollen, auf dasselbe System.

Das Bild bei der Projektion eines Vektors $\vek x =  \{x_1, x_2, x_3\}^T$ auf die $x\!-\!y$-Ebene ist sein Schatten $\vek x'$, s. Abb. \ref{U161}. Wir wissen nat\"{u}rlich, wo der Schatten hinf\"{a}llt, aber der Computer hat keine Augen, er rechnet. Er macht f\"{u}r den Schatten $\vek x'$ den Ansatz $\vek x' = c_1\,\vek e_1 + c_2\,\vek e_2$ und bestimmt $c_1$ und $c_2$ so, dass der Fehler senkrecht auf $\vek e_1$ und $\vek e_2$ steht
\begin{align}
(\vek x - \vek x')^T\, \vek e_i = 0 \qquad i = 1,2 \qquad \Rightarrow \qquad c_1 = x_1\,, c_2 = x_2\,,
\end{align}
was gleichbedeutend damit ist, dass der Schatten der Vektor in der Ebene ist, der den kleinstm\"{o}glichen Abstand
\begin{align}
|\vek e| = |\vek x - \vek x'| = \text{Minimum}
\end{align}
von $\vek x$ hat. Dies kann man auch schreiben als (wir vergessen einmal die Wurzel zu ziehen)
\begin{align}
|\vek e|^2 = (\vek x - \vek x')^T\,(\vek x - \vek x') = \text{Minimum}\,,
\end{align}
so sieht man besser die Verwandtschaft mit (\ref{Eq109}), denn auch bei den finiten Elementen handelt es sich, wenn man es als {\em Galerkin-Verfahren\/} interpretiert, um ein Projektionsverfahren. Nur dass die Metrik eine andere ist, nicht das Skalarprodukt zwischen zwei Vektoren, sondern die Wechselwirkungsenergie zwischen zwei Funktionen ist das Ma{\ss}.

Beim {\em Galerkin-Verfahren\/} w\"{a}hlt man als beste N\"{a}herung die Projektion der exakten L\"{o}sung $w$ auf den Ansatzraum $\mathcal{V}_h$, also die Funktion $w_h$ in $\mathcal{V}_h$, deren Fehler $w - w_h$ senkrecht auf allen $\Np_i$ steht
\begin{align}\label{Gortho}
 a(w - w_h,\Np_i) = 0 \qquad i = 1,2, \ldots, n\,.
\end{align}
Auf Grund der ersten Greenschen Identit\"{a}t, etwa des Seils,
\begin{align}\label{Eq89}
\text{\normalfont\calligra G\,\,}(w,\Np_i) = \int_0^{\,l} p\,\Np_i\,dx - a(w,\Np_i) = 0 \quad \Rightarrow \quad f_i = \int_0^{\,l} p\,\Np_i\,dx = a(w,\Np_i)
\end{align}
ist dies \"{a}quivalent mit
\begin{align}
 a(w_h,\Np_i) = f_i \qquad i = 1,2, \ldots, n
\end{align}
und diese $n$ Gleichungen entsprechen genau dem System $\vek K\,\vek u = \vek f$.

Diese Eigenschaft impliziert, dass $w_h$ auf $\mathcal{V}_h$ den kleinstm\"{o}glichen Abstand in der Energiemetrik,
\begin{align}\label{Eq109}
a(w-w_h,w-w_h) = \text{Minimum}
\end{align}
von $w$ hat, was praktisch bedeutet, dass $w_h$ im Sinne des Fehlerquadrats die kleinstm\"{o}gliche Abweichung in den Schnittgr\"{o}{\ss}en aufweist, \cite{Ha5}, S. 572.

Was man in Abb. \ref{U161} auch sieht ist, dass die nochmalige Projektion des Bildes $\vek x'$ nichts bringt, die Projektion bleibt stehen. Das ist auch der Grund, warum die Strategie, die Abweichung  $p - p_h$ in den Lasten im Sinne einer Korrektur nachtr\"{a}glich auf die Struktur aufzubringen, zu nichts f\"{u}hrt, die zu dem Lastfall $p - p_h$ geh\"{o}rigen $f_i$ sind null, weil der Schatten des Fehlers $\vek e$ der Nullvektor ist.

Glg. (\ref{Gortho}) ist die {\em Galerkin-Orthogonalit\"{a}t\/}\index{Galerkin-Orthogonalit\"{a}t}. Wegen $\delta A_i = \delta A_a$ kann man sie auch als Orthogonalit\"{a}t in den \"{a}u{\ss}eren Arbeiten schreiben
\begin{align}
a(w-w_h,\Np_i) = \int_0^{\,l} (p - p_h)\,\Np_i\,dx = 0\,.
\end{align}
Die Differenz zwischen dem Originallastfall $p$ und dem FE-Lastfall $p_h$ (dessen L\"{o}sung $w_h$ ist), ist also orthogonal zu allen $\Np_i$, d.h. die Fehlerkr\"{a}fte $p - p_h$ leisten keine Arbeit, wenn man an dem Seil mit den $\Np_i$ wackelt.


%%%%%%%%%%%%%%%%%%%%%%%%%%%%%%%%%%%%%%%%%%%%%%%%%%%%%%%%%%%%%%%%%%%%%%%%%%%%%%%%%%%%%%%%%%%%%%%%%%%
{\textcolor{sectionTitleBlue}{\section{\"{A}quivalente Knotenkr\"{a}fte}}}\index{aequivalente Knotenkr\"{a}fte}
Die $f_i $ auf der rechten Seite des Gleichungssystems $\vek K\,\vek u = \vek f $ sind keine Kr\"{a}fte, sondern Arbeiten\footnote{F\"{u}r eine alternative Interpretation in der Stabstatik s. S. \pageref{Dimensionsbetrachtung}}
\begin{align}
f_i = \int_0^{\,l} p(x)\,\Np_i(x)\,dx = [F/L] \cdot [L] \cdot [L] = [F \cdot L]\,,
\end{align}
und auch auf der linken Seite stehen Arbeiten, denn der einzelne Eintrag $k_{ij}$ in der Steifigkeitsmatrix beruht, wenn wir einen Stab als Beispiel nehmen, auf der Formel
\begin{align}
k_{ij } = \int_0^{\,l} EA\,\Np_i'\,\Np_j'\,dx = [F/L^2 \cdot L^2] \cdot [L/L] \cdot [L/L] \cdot [L] = [F \cdot L]\,,
\end{align}
und entsprechend auch beim Balken
\begin{align}
k_{ij } = \int_0^{\,l} EI\,\Np_i''\,\Np_j''\,dx = [F \cdot L^2] \cdot [1/L] \cdot [1/L] \cdot [L] = [F \cdot L]\,.
\end{align}
Bei jeder Ableitung wird mit $[L]^{-1}$ multipliziert
\begin{align}
\Np_i\,\,[L] \qquad \Np_i' = \frac{d\,\Np_i}{dx} = [\phantom{L}] \qquad \Np_i'' = \frac{d\,\Np_i'}{dx} = \frac{1}{[L]}\,.
\end{align}
Bei 3-D Problemen f\"{u}hrt die Dimensionsbetrachtung auf
\begin{align}
\int_{\Omega} \sigma_{ij}\,\varepsilon_{ij} \,dV = \frac{[F]}{[L^2]}\,\frac{[L]}{[L]} \,[L^3] = [F \cdot L]\,,
\end{align}
woraus bei Scheibenproblemen, $d = $ Dicke der Scheibe und $ dV = d\, d\Omega$, der Ausdruck
\begin{align}
\int_{\Omega} \sigma_{ij}\,\varepsilon_{ij}\, d \,d\Omega = \frac{[F]}{[L^2]}\,\frac{[L]}{[L]} \,[L \cdot L^2] = [F \cdot L]
\end{align}
wird.\\

\hspace*{-12pt}\colorbox{highlightBlue}{\parbox{0.98\textwidth}{Die Knotenverschiebungen $u_i$ sind also (intern) {\em dimensionslose\/} Gewichte an dem FE-Ansatz
\begin{align}
u_h = \sum_i\,u_i\,\Np_i(x) = [\phantom{L}] \cdot [L] = [L]\,.
\end{align}
Im Ausdruck haben Sie nat\"{u}rlich die Dimension einer L\"{a}nge, wie es der Ingenieur sehen will.
}}\\

%%%%%%%%%%%%%%%%%%%%%%%%%%%%%%%%%%%%%%%%%%%%%%%%%%%%%%%%%%%%%%%%%%%%%%%%%%%%%%%%%%%%%%%%%%%%%%%%%%%
{\textcolor{sectionTitleBlue}{\subsubsection*{Rechenpfennige}}}\label{Rechenpfennige}

F\"{u}r ein FE-Programm sind die \"{a}quivalenten Knotenkr\"{a}fte $f_i$ {\em Rechenpfennige\/}\index{Rechenpfennige} wie \glq eins im Sinn\grq{}. Es geht von Knoten zu Knoten, verschiebt den Knoten um einen Meter in horizontaler und vertikaler Richtung und notiert sich, wieviel Arbeit die Belastung dabei leistet. Das sind die $f_i$.

Hat ein horizontales $f_i$ in einem Knoten einer Scheibe den Wert 10 kNm, so bedeutet dies, dass in der N\"{a}he des Knotens Lasten so verteilt sind, dass sie bei einer horizontalen Auslenkung $\vek \Np_i$ des Knotens um 1 m die Arbeit 10 kNm leisten.

Alles, was ein FE-Programm macht, ist, dass es dann Ersatzlasten so \"{u}ber die Scheibe verteilt, dass diese bei einer Auslenkung der einzelnen Knoten um 1 Meter dieselbe Arbeit leisten, wie die Originalbelastung, was man kurz als $f_{h @i}  = f_i$ schreiben kann.

In der Notation des \"{u}bern\"{a}chsten Abschnitts ist $f_{h @i}$ die Arbeit, die die FE-Lasten
\begin{align}
\vek p_h = \sum_j\,u_j\,\vek p_j
\end{align}
auf dem Weg $\vek \Np_i$ leisten
\begin{align}
f_{h @i} = \sum_j\,u_j\,\delta A_a(\vek p_j,\vek \Np_i)\,.
\end{align}
Der FE-Lastfall $\vek p_h$ sind die Kr\"{a}fte, die n\"{o}tig sind, um dem Tragwerk die Gestalt $\vek u = \{u_1, u_2, \ldots, u_n\}^T$ zu geben.
%----------------------------------------------------------------------------------------------------------
\begin{figure}[tbp]
\centering
\if \bild 2 \sidecaption \fi
\includegraphics[width=1.0\textwidth]{\Fpath/U89}
\caption{Einheitsverformungen am Stab (l\"{a}ngs) und am Balken (quer)} \label{U89}
\end{figure}%
%----------------------------------------------------------------------------------------------------------

%%%%%%%%%%%%%%%%%%%%%%%%%%%%%%%%%%%%%%%%%%%%%%%%%%%%%%%%%%%%%%%%%%%%%%%%%%%%%%%%%%%%%%%%%%%%%%%%%%%
{\textcolor{sectionTitleBlue}{\section{Festhaltekr\"{a}fte}}}\index{Festhaltekr\"{a}fte}
Wenn man die \"{a}quivalenten Knotenkr\"{a}fte aus der Streckenlast berechnet,
\begin{align}\label{Eq85}
d_i = \int_0^{\,l} p(x)\,\Np_i(x)\,dx\,,
\end{align}
dann nennt man das die Reduktion der Belastung in die Knoten.

Wir erinnern daran, dass wir die \"{a}quivalenten Knotenkr\"{a}fte $\vek f = \vek f_K + \vek d$, in die Kr\"{a}fte $\vek f_K$ aufspalten, die direkt in den Knoten angreifen und die Kr\"{a}fte $\vek d$, die bei der Reduktion der Belastung im Feld, der {\em domain load\/}, in die Knoten entstehen. Hier behandeln wir die Kr\"{a}fte $\vek d$.

Die \"{a}quivalenten Knotenkr\"{a}fte sind die Kr\"{a}fte, die ({\em actio\/}), mit denen  die Last auf die beidseitig festgehaltenen bzw. eingespannten Enden des Stabes oder Balkens dr\"{u}ckt. Also gilt: {\em Die Einheits\-ver\-form\-ungen $\Np_i^e$ sind die Einflussfunktionen f\"{u}r die \"{a}quivalenten Knotenkr\"{a}fte\/}. Bei einem Stab sind das die Funktionen
\begin{align}
\barr{l l l} \Np_1^e(x) = {\displaystyle \frac{1 - x}{l_e}} \qquad &\Np_1^e(0) = 1\,, \qquad
&\Np_1^e(l_e) = 0\,, \\ [0.3cm] \Np_2^e(x) = {\displaystyle \frac{x}{l_e}}  \qquad &\Np_2^e(0) =
0\,, \qquad &\Np_2^e(l_e) = 1 \earr
\end{align}
und bei einem Balken sind es die kubischen Polynome, s. Abb. \ref{U89},
\bfo\label{Phi1Bis4}
\parbox{5cm}{
\bfo
\Np_1^e(x) &=& 1 - \frac{3x^2}{l_e^2} + \frac{2x^3}{l_e^3} \nn \\
\Np_2^e(x) &=& - x + \frac{2x^2}{l_e} - \frac{x^3}{l_e^2} \nn
\efo
}
%\hfill
\parbox{5cm}{
\bfo
\Np_3^e(x) &=& \frac{3x^2}{l_e^2} - \frac{2x^3}{l_e^3}\nn \\
\Np_4^e(x) &=& \frac{x^2}{l_e} - \frac{x^3}{l_e^2}\,.\nn  \label{Einheitsverformungen}
\efo
}
\efo
Die \"{a}quivalenten Knotenkr\"{a}fte $d_i^e$ sind dann die Gr\"{o}{\ss}en
\begin{align}
d_i^e &= \int_0^{\,l_e} \,\underset{\rightarrow}{p(x)}\,\Np_i^e(x)\,dx \qquad i = 1,2 &&\qquad \text{Stab} \\
d_i^e &= \int_0^{\,l_e} \,\underset{\downarrow}{p(x)}\,\Np_i^e(x)\,dx \qquad i = 1,2, 3, 4 &&\qquad \text{Balken}\,.
\end{align}
Wirkt eine Einzelkraft $P$ bzw. ein Moment $M$ in einem Punkt $x$, dann erh\"{a}lt man die zugeh\"{o}rigen $d_i^e$ einfach durch Auswertung im Punkt
\begin{align}
d_i^e = \Np_i^e(x) \cdot P \qquad  d_i^e = {\Np_i^e} '(x) \cdot M\,.
\end{align}
\"{A}quivalente Knotenkr\"{a}fte und Festhaltekr\"{a}fte  verhalten sich wie {\em actio\/} und {\em reactio\/} zueinander. Die Festhaltekr\"{a}fte sind also die $d_i^e \times (-1)$.

Eventuell muss man auch noch die $d_i^e$ in das DIN-Koordinatensystem umrechnen
\begin{align}
d_1^e &= - N(0) \qquad d_2^e = N(l_e) \\
d_1^e &= - V(0) \qquad d_2^e = - M(0) \qquad d_3^e = V(l_e) \qquad d_4^e = M(l_e) \,,
\end{align}
wenn man anders mit ihnen weiterrechnen will.\\

\hspace*{-12pt}\colorbox{highlightBlue}{\parbox{0.98\textwidth}{Bei Stabtragwerken sind die Einheitsverformungen der Knoten die Einflussfunktionen f\"{u}r die Festhaltekr\"{a}fte $\times (-1)$.}}\\

All dies nat\"{u}rlich unter der Voraussetzung, dass die Elementeinheitsverformungen $\Np_i^e$ homogene L\"{o}sungen der Stab- und Balkenl\"{o}sung sind, was man in der Regel voraussetzen kann, wenn keine gevouteten oder andere exotische Profile vorkommen, wenn also $EA$ und $EI$ konstant sind.

%----------------------------------------------------------------------------------------------------------
\begin{figure}[tbp]
\centering
\if \bild 2 \sidecaption \fi
\includegraphics[width=0.7\textwidth]{\Fpath/U73}
\caption{Ausschnitt aus einem FE-Netz: Die Kr\"{a}fte, die den Mittenknoten um eine L\"{a}ngeneinheit nach rechts verschieben, und die Bewegung an den umliegenden Knoten abstoppen, sind die {\em shape forces\/}. Die Fl\"{a}chenkr\"{a}fte $p_x$ und $p_y$ sind nur als integrale Werte angegeben. Im \"{u}brigen Netz sind die {\em shape forces\/} null, weil sich dort nichts bewegt} \label{U73}
\end{figure}%
%----------------------------------------------------------------------------------------------------------

%----------------------------------------------------------------------------------------------------------
\begin{figure}[tbp]
\centering
\if \bild 2 \sidecaption \fi
\includegraphics[width=.9\textwidth]{\Fpath/U28NEW}
\caption{Belastung einer Scheibe mit einer Einzelkraft, {\bf a)} System {\bf b)} der FE-Lastfall $\vek p_h$; die Graustufen entsprechen der Intensit\"{a}t der aufintegrierten FE-Fl\"{a}chenlast, wie in Abb. \ref{U73}, in den bilinearen Elementen} \label{U28}
\end{figure}%
%----------------------------------------------------------------------------------------------------------\\

%%%%%%%%%%%%%%%%%%%%%%%%%%%%%%%%%%%%%%%%%%%%%%%%%%%%%%%%%%%%%%%%%%%%%%%%%%%%%%%%%%%%%%%%%%%%%%%%%%%
{\textcolor{sectionTitleBlue}{\section{Shape forces und der FE-Lastfall}}}\index{shape forces}
Um einen Knoten eines Netzes um einen Meter horizontal oder vertikal zu verschieben -- und dabei gleichzeitig alle anderen Knoten festzuhalten -- sind gewisse Kr\"{a}fte n\"{o}tig, s. Abb. \ref{U73}. Wir nennen diese Kr\"{a}fte, in Analogie zu dem Begriff {\em shape functions\/}\index{shape functions}, die {\em shape forces\/} $\vek p_j = \{p_x, p_y\}^T$, die zu dem Freiheitsgrad $u_j$ geh\"{o}ren. Es sind treibende wie haltende Kr\"{a}fte. Die treibenden Kr\"{a}fte lenken den Knoten aus, und die haltenden Kr\"{a}fte sorgen daf\"{u}r, dass die Bewegung an den umliegenden Knoten zum Stillstand kommen. Es sind immer Gleichgewichtskr\"{a}fte.
%----------------------------------------------------------------------------------------------------------
\begin{figure}[tbp]
\centering
\if \bild 2 \sidecaption \fi
\includegraphics[width=1.0\textwidth]{\Fpath/U127}
\caption{Lineare FE-L\"{o}sung eines Stabes ($EA = 1$) unter konstanter Streckenlast. Die Kr\"{a}fte an den $\Np_i$ sind die {\em shape forces\/} (s.f.)} \label{U127}
\end{figure}%
%----------------------------------------------------------------------------------------------------------\\

Die Summe dieser {\em shape-forces\/} -- mit den Knotenverschiebungen $u_j$ gewich\-tet -- stellt den FE-Lastfall\index{FE-Lastfall} dar, also die Kr\"{a}fte, die $\vek u$ erzeugen
\begin{align}\label{Eq70}
\vek p_h = \sum_j\,u_j\,\vek p_j\,.
\end{align}
Nun kann man fragen, wenn die einzelnen $\vek p_j$ alle Gleichgewichtskr\"{a}fte sind, was wandert dann eigentlich in die Lager? Das kommt zum Vorschein, wenn man einen Schnitt neben einem Lager f\"{u}hrt, wie in Abb. \ref{U127} b, wo der Schnitt durch die Mitte des ersten Elements geht. Durch diesen Schnitt wird das Gleichgewicht in dem ersten Element gest\"{o}rt und eine entsprechend gro{\ss}e Schnittkraft von 2.5 kN muss das Gleichgewicht wieder herstellen. Dass es nicht die volle Belastung von 3.0 kN ist, liegt daran, dass das FE-Programm ja die \"{a}quivalente Knotenkraft $f = 0.5$ in dem Lager ignoriert, weil sie nicht statisch wirksam ist.

W\"{a}hrend bei Fl\"{a}chentragwerken einiger Aufwand n\"{o}tig ist, um den FE-Lastfall $\vek p_h$ zu berechnen, s. Abb. \ref{U28}, muss man bei Stabtragwerken ($EA$ und $EI$ konstant) gar nichts tun, denn bei Stabtragwerken ist der FE-Lastfall mit den \"{a}quivalenten Knotenkr\"{a}ften $f_i = f_{K @ i} + d_i$ identisch.\\

\begin{remark}
Die Abb. \ref{U28} illustriert auch sehr gut, die \glq Doppelb\"{o}digkeit\grq{} der Statik im Umgang mit finiten Elementen. Gehalten wird die Scheibe in der N\"{a}he der Lagerknoten eigentlich von einem konzentrierten System von Fl\"{a}chen- und Linienkr\"{a}ften. Im Ausdruck stehen aber nur die \"{a}quivalenten Knotenkr\"{a}fte $f_i$, die diese Haltekr\"{a}fte in der \glq Summe\grq{} repr\"{a}sentieren, und der Ingenieur findet (zu Recht) gar nichts dabei mit diesen $f_i$ weiter zu rechnen, aus ihnen echte Kr\"{a}fte zu machen.
\end{remark}
\vspace{-0.5cm}
%%%%%%%%%%%%%%%%%%%%%%%%%%%%%%%%%%%%%%%%%%%%%%%%%%%%%%%%%%%%%%%%%%%%%%%%%%%%%%%%%%%%%%%%%%%%%%%%%%%
{\textcolor{sectionTitleBlue}{\subsubsection*{Die Rolle der $u_i$}}}\index{die Rolle der $u_i$}

Der FE-Lastfall $\vek p_h$ wird mit Hilfe der $u_i$ so austariert, dass die zugeh\"{o}rigen Knotenkr\"{a}fte $f_{h @i}$ mit den \"{a}quivalenten Knotenkr\"{a}ften aus der Belastung \"{u}bereinstimmen, also im Fall einer Scheibe die Arbeiten $f_{h @i}$ genauso gro{\ss} sind, wie die Arbeiten $f_i$
\begin{align}\label{Eq123}
f_{h @i} = \int_{\Omega} \vek p_h^T \dotprod \vek \Np_i\,d\Omega = \int_{\Omega}\,\vek p^T \dotprod \vek \Np_i\,d\Omega = f_i\,.
\end{align}
Diese Forderung ist \"{a}quivalent mit dem System $\vek K\,\vek u = \vek f$, denn der Vektor $\vek K\,\vek u$ (die Eintr\"{a}ge sind innere Arbeiten $\delta A_i$) ist wegen {\em \glq innen = au{\ss}en\grq{}\/} identisch mit dem Vektor $\vek f_h$ der \"{a}u{\ss}eren Arbeiten $\delta A_a$.

Man stelle sich die Scheibe einmal mit der Originalbelastung $\vek p$ vor und daneben mit der FE-Belastung $\vek p_h$. Nun gehe man von Knoten zu Knoten und verschiebe den Knoten probeweise um einen Meter in horizontaler und vertikaler Richtung. Dann wird man finden, dass die Arbeiten immer gleich sind, $f_{h @i} = f_i$. In diesem Sinne gilt:\\

\hspace*{-12pt}\colorbox{highlightBlue}{\parbox{0.98\textwidth}{Der FE-Lastfall ist \glq wackel\"{a}quivalent\grq{} zu dem Originallastfall.
}}\\

Das ist wie bei einer Waage, wo bei jeder Drehung des Waagebalkens die Arbeiten des linken und rechten Gewichts gleich sind, d.h. die beiden Gewichte sind \glq wackel\"{a}quivalent\grq{}.\index{wackel\"{a}quivalent}

Ob ein Tragwerk die Originallasten tr\"{a}gt oder die FE-Lasten, kann man durch Wackeln mit den $\vek \Np_i$ allein nicht entscheiden, weil die Arbeiten jedesmal gleich sind.

Der FE-Lastfall ist der Lastfall, f\"{u}r den ein Tragwerksplaner das Tragwerk eigentlich bemisst, denn die Verformungen und die Schnittkr\"{a}fte im Ausdruck geh\"{o}ren zu diesem Lastfall $\vek p_h$.

{\em Wenn man die Belastung $\vek p_h$ auf das Tragwerk aufbringen w\"{u}rde und ein Statiker w\"{u}rde die Verformungen und Schnittkr\"{a}fte von Hand berechnen, dann w\"{u}rde er genau die FE-Ergebnisse erhalten\/}.

%----------------------------------------------------------------------------------------------------------
\begin{figure}[tbp]
\centering
\if \bild 2 \sidecaption \fi
\includegraphics[width=0.95\textwidth]{\Fpath/U29}
\caption{Vergleich der Hauptspannungen, {\bf a)} grobes Netz, {\bf b)} sehr feines Netz} \label{U29}
\end{figure}%
%----------------------------------------------------------------------------------------------------------

In Abb. \ref{U28} b ist ein solcher Vergleich des Originallastfalls und des FE-Lastfalls einmal dargestellt. Zu berechnen waren die Wirkungen einer Einzelkraft von 10 kN, die an der rechten oberen Ecke zieht. Diesen doch eigentlich einfachen Lastfall ersetzt nun das FE-Programm durch ein sehr konfus wirkendes System von Fl\"{a}chenkr\"{a}ften und Kantenkr\"{a}ften, die den Lastfall $\vek p_h$ darstellen. Weil diese Lasten so \glq merkw\"{u}rdig\grq{} aussehen, werden sie normalerweise von FE-Programmen nicht gezeigt, weil ein Anwender, der mit der Theorie der finiten Elemente nicht vertraut ist, irritiert w\"{a}re.

Nur darf man sich davon aber nicht abschrecken lassen, denn den Ingenieur interessieren prim\"{a}r die Schnittgr\"{o}{\ss}en, und es ist Spekulation aus der Differenz $\vek p - \vek p_h$ in den Lasten auf die Differenz in den Schnittgr\"{o}{\ss}en schlie{\ss}en zu wollen. Die Differenz in der Belastung kann man ausrechnen, die Differenz in den Schnittkr\"{a}ften aber leider nicht.

Dass die FE-Ergebnisse nicht so schlecht sein k\"{o}nnen, wie dies die Abb. \ref{U28} anscheinend suggeriert, macht der direkte Vergleich der Hauptspannungen in Abb. \ref{U29} klar. Diese wirken durchaus glaubhaft.

Wir wollen noch ein zweites, indirektes Argument anf\"{u}hren. Im Originallastfall sind alle $f_i = 0$, bis auf das $f_i = 10$ in der oberen rechten Ecke, und daher m\"{u}ssen auch alle $f_{h @i} = 0$ sein, bis auf den Knoten in der Ecke, $f_{h @i} = 10$
\begin{align}\label{Eq124}
\int_{\Omega} \vek p_h \dotprod \vek \Np_i\,\,d\Omega = f_{h @i} = f_i = 0\,.
\end{align}
Das FE-Programm muss also schon kr\"{a}ftig jonglieren, um diese Eigenschaft zu garantieren, und das mag das \glq Chaos\grq{} in Abb. \ref{U28} b erkl\"{a}ren, denn all die recht willk\"{u}rlich aussehenden Teile des Lastfalls $\vek p_h$ sind so ausbalanciert, dass sie keine Arbeit leisten, wenn man einen Knoten probeweise um einen Meter in horizontaler oder vertikaler Richtung verschiebt.

Der Gro{\ss}teil der FE-Lasten $\vek p_h$  ist f\"{u}r das FE-Programm im Sinne der Energiemetrik null, weil sie keine Knotenkr\"{a}fte $f_{h @i}$ generieren.\\

\begin{remark}
Die {\em shape forces\/} $\vek p_i$ sind bei einer Scheibe Fl\"{a}chenkr\"{a}fte und Linienkr\"{a}fte und die virtuelle Arbeit dieser Kr\"{a}fte ist daher eigentlich eine Summe aus Gebietsintegralen und Linienintegralen \"{u}ber die Kanten $\Gamma$ der Elemente, auf denen der Knoten liegt, zu dem $\Np_i$ geh\"{o}rt
\begin{align}
f_{h @i} = \int_{\Omega} ... \,d\Omega + \int_{\Gamma} ...\,ds\,.
\end{align}
Die Schreibweise (\ref{Eq124}) ist daher eine zusammenfassende Kurzschreibweise f\"{u}r all diese Integrale.
\end{remark}
\vspace{-0.5cm}
%%%%%%%%%%%%%%%%%%%%%%%%%%%%%%%%%%%%%%%%%%%%%%%%%%%%%%%%%%%%%%%%%%%%%%%%%%%%%%%%%%%%%%%%%%%%%%%%%%%
{\textcolor{sectionTitleBlue}{\section{Wie die Lawine ins Rollen kam}}}
Wenn wir die Gleichung $\vek f_h = \vek f $ als die Grundgleichung der finiten Elemente ansehen, dann stimmt das mit dem Zugang in der Originalarbeit \cite{Turner} von {\em Turner et alteri\/} \"{u}berein. Die Autoren betrachteten damals ein {\em CST-Element\/}\index{CST-Element} und sie leiteten eine Matrix $\vek S $ her, die die drei Spannungen in dem Element, $\vek \sigma = \{\sigma_{xx}, \sigma_{yy}, \sigma_{xy}\}^T$, mit den Knotenverschiebungen verkn\"{u}pft\footnote{Wir orientieren uns hier an der Darstellung in \cite{Kurrer} S. 882-883}
\begin{align}\label{Eq174}
\vek \sigma_{(3)} = \vek S_{(3 \times 6)}\,\vek u_{(6)} \,.
\end{align}
Das sind die Spannungen, die zu dem FE-Lastfall $\vek p_h$ geh\"{o}ren, also zu den Kr\"{a}ften, die die Knotenverschiebungen, Vektor $\vek u$, bewirken. Weil die Spannungen konstant sind, gibt es in einem solchen LF $\vek p_h$ keine Volumenkr\"{a}fte sondern nur Randkr\"{a}fte. Als n\"{a}chstes haben die Autoren die sechs \"{a}quivalenten Knotenkr\"{a}fte, Vektor $\vek f_{h}$, berechnet, die zu diesen Kr\"{a}ften geh\"{o}ren, sie haben also die Randkr\"{a}fte mit den Einheitsverformungen $\vek \Np_i(\vek x)$ der Knoten \"{u}berlagert, was auf eine Matrix $\vek T $ f\"{u}hrte
\begin{align}
\vek f_h = \vek T_{(6 \times 3)}\,\vek \sigma_{(3)}
\end{align}
oder mit (\ref{Eq174}) auf die Beziehung
\begin{align}
\vek f_h = \vek T_{(6 \times 3)}\,\vek S_{(3 \times 6)}\,\vek u_{(6)}
\end{align}
und das ist genau die Steifigkeitsmatrix $\vek K = \vek T\,\vek S$ des {\em CST-Element\/}. Dass hier die Steifigkeitsmatrix auftaucht, deren Eintr\"{a}ge virtuelle {\em innere\/} Energien sind, obwohl doch eigentlich nur \glq au{\ss}en\grq{} gerechnet wurde, liegt an dem \glq au{\ss}en = innen\grq{}, das die erste Greensche Identit\"{a}t garantiert
\begin{align}
\text{\normalfont\calligra G\,\,}(\vek \Np_j,\vek \Np_i) = \delta A_a(\vek p_j,\vek \Np_i) - \delta A_i(\vek \Np_j,\vek \Np_i) = \delta A_a(\vek p_j,\vek \Np_i)-  k_{ij} =0\,,
\end{align}
zeilenweise also
\begin{align}
f_{hi} &= \delta A_a(\vek p_h,\vek \Np_i) = \sum_j\,\delta A_a(\vek p_j,\vek \Np_i)\,u_j =  \sum_j\,\delta A_i(\vek \Np_j,\vek \Np_i)\,u_j \nn \\
&= \sum_j\,k_{ij}\,u_j = (\text{Zeile $i$ von $\vek K$) $\dotprod\,\vek u$}\,.
\end{align}

\hspace*{-12pt}\colorbox{highlightBlue}{\parbox{0.98\textwidth}{Die Elemente $k_{ij}$ einer Steifigkeitsmatrix $\vek K$ kann man als virtuelle innere Arbeit wie als virtuelle \"{a}u{\ss}ere Arbeit lesen.}}\\

Das $k_{11} = \delta A_i(\Np_1,\Np_1)$ einer Balkenmatrix
\begin{align}
 k_{11} = \int_{0}^{l}\frac{M_1^2}{EI}\,dx = a(\Np_1,\Np_1) = \frac{12\,EI}{l^3} \cdot 1 = \delta A_a(\Np_1,\Np_1)
\end{align}
ist genauso gro{\ss}, wie die Arbeit, die die Kraft $12\,EI/l^3$ -- das ist die Kraft, die den Knoten nach unten dr\"{u}ckt -- auf dem Weg $\Np_1(0) = 1$ leistet.

{\em Die \glq erste\grq{} Steifigkeitsmatrix in der Geschichte der FEM war also eine Tabelle von {\bf \"{a}u{\ss}eren Arbeiten}\/}. Erst die \"{A}quivalenz $\delta A_i = \delta A_a$ f\"{u}hrt auf die Interpretation, wie wir Steifigkeitsmatrizen $\vek K$ heute lesen, $k_{ij} = a(\vek \Np_i,\vek \Np_j)$.

Bemerkenswert ist dabei, wie simpel die finiten Elemente begonnen haben -- kein Energieprinzip, kein Galerkin, \"{u}berhaupt keine h\"{o}here Mathematik, sondern ein uraltes Prinzip, das Prinzip Waage\index{Prinzip Waage}, die \glq Wackel\"{a}quivalenz\grq{}\,
\begin{align}
 \boxed{\delta A_a(\vek p, \vek \Np_i) = \delta A_a(\vek p_h,\vek \Np_i) \qquad \text{der Start der FEM}}
\end{align}
hat die Lawine ins Rollen gebracht\footnote{
Erst sp\"{a}ter, als die Mathematiker an Bord  kamen, hat man erkannt, dass man die Elementans\"{a}tze  als finite Funktionen deuten konnte und dass die Wackel\"{a}quivalenz $\vek f_h = \vek f$ dem $\delta \Pi = 0 $ der potentiellen Energie entspricht.}.




%---------------------------------------------------------------------------------
\begin{figure}
\centering
{\includegraphics[width=0.8\textwidth]{\Fpath/U192A}}
  \caption{Gelenkig gelagerte Platte mit Innenst\"{u}tze im LF $g$, \textbf{ a)} Hauptmomente, \textbf{ b)} Biegefl\"{a}che mit \"{a}quivalenter Knotenkraft in der St\"{u}tze,  \textbf{ c)} L\"{a}ngsschnitt durch die Platte mit den FE-Lasten (symbolische Darstellung). Die \"{a}quivalente Knotenkraft im mittleren Bild steht stellvertretend f\"{u}r die aufw\"{a}rts gerichteten FE-Lasten im Bereich der St\"{u}tze, die eigentlich die Platte halten}
  \label{U192}
\end{figure}
%---------------------------------------------------------------------------------
%---------------------------------------------------------------------------------
\begin{figure}
\centering
{\includegraphics[width=0.8\textwidth]{\Fpath/U168}}
  \caption{Der FE-Lastfall $g_h$, das ist der LF, den das FE-Programm f\"{u}r den LF $g$ setzt, \textbf{ a)} die Zahlen sind die mittleren Fl\"{a}chenkr\"{a}fte in den Elementen, im Bereich der Innenst\"{u}tze sind sie negativ, dort st\"{u}tzen die Fl\"{a}chenkr\"{a}fte also die Platte, \textbf{ b)} ebenso wie die Linienkr\"{a}fte l\"{a}ngs der Elementkanten im Bereich der St\"{u}tze (rote Farbe); die Linienmomente des LF $g_h$ l\"{a}ngs der Linien sind nicht dargestellt}
  \label{U168}
\end{figure}
%---------------------------------------------------------------------------------


%---------------------------------------------------------------------------------
\begin{figure}
\centering
\if \bild 2 \sidecaption[t] \fi
{\includegraphics[width=0.75\textwidth]{\Fpath/U175}}
  \caption{Einzelkraft und Platte}
  \label{U175}
\end{figure}
%---------------------------------------------------------------------------------

%%%%%%%%%%%%%%%%%%%%%%%%%%%%%%%%%%%%%%%%%%%%%%%%%%%%%%%%%%%%%%%%%%%%%%%%%%%%%%%%%%%%%%%%%%%%%%%%%%%
{\textcolor{sectionTitleBlue}{\section{Der FE-Lastfall bei Platten}}}
Der FE-Lastfall und seine Interpretation ist bei Platten \"{a}hnlich ambivalent wie bei Scheiben, wie die Bilder \ref{U192} und \ref{U168} zeigen. Die St\"{u}tzenkraft $f_i$ im Ausdruck suggeriert, dass die Platte in der Mitte punktgenau von einer Einzelkraft gehalten wird, aber wenn man sich die Verteilung der FE-Lasten in Abb. \ref{U168} ansieht, dann erkennt man, dass das symbolisch zu nehmen ist. Rechnerisch sind es aufw\"{a}rts gerichtete Fl\"{a}chen- und Linienkr\"{a}fte in der Umgebung der St\"{u}tze, die die Platte nach oben dr\"{u}cken, keine Einzelkraft.

Denn w\"{a}re die St\"{u}tzenkraft $f_i$ eine echte Einzelkraft, dann m\"{u}ssten in dem Knoten die Schubkr\"{a}fte $v_n$ bei Ann\"{a}herung an die St\"{u}tze unendlich gro{\ss} werden, $v_n = 1/(2\,\pi\,r)$, weil anders die Bilanz\footnote{s. S. \pageref{EinzelF}}
\begin{align}
\lim_{r \to 0} \int_0^{\,2\,\pi} v_n \,r\,d\Np = f_i
\end{align}
nicht einzuhalten ist, s. Abb. \ref{U175}. Aber die {\em shape functions\/} $\Np_i(\vek x)$ sind Polynome und sie k\"{o}nnen sich nicht wie $1/r$ in einem Knoten zusammenschn\"{u}ren.

Die Knotenkraft $f_i$ im Ausdruck ist daher eine \"{a}quivalente Knotenkraft, also ein Arbeits\"{a}quivalent im Sinne von \glq soviel Arbeit wie eine Einzelkraft $f_i$ auf dem Weg 1 Meter leisten w\"{u}rde\grq{}.

Zwar hat die die St\"{u}tzenkraft $f_i$ direkt nichts mit der Bemessung der Platte zu tun, weil die Momente $m_{xx}, m_{xy}$ und $m_{xy}$ und auch die Querkr\"{a}fte $q_x$ und $q_y$ zu dem FE-Lastfall $p_h$ geh\"{o}ren, s. Abb. \ref{U168}, bei dem eben Fl\"{a}chen- und Linienkr\"{a}fte die Platte st\"{u}tzen, aber der Ingenieur wird nat\"{u}rlich mit der Knotenkraft $f_i$ einen Durchstanznachweis f\"{u}hren und sie auch als St\"{u}tzenkraft weiterleiten.

Uns scheint, dass diese Ambivalenz der finiten Elemente in der Praxis zu wenig diskutiert wird. Ein FE-Modell ist ein {\em Ersatzmodell\/}\index{Ersatzmodell} und es ist gar nicht eindeutig gekl\"{a}rt, wie man die FE-Ergebnisse auf das reale Tragwerk \"{u}bertr\"{a}gt.

Wenn man sich \"{u}berlegt,  mit welcher Akkuratesse heute der Durchstanznachweis bei Platten gef\"{u}hrt wird, und wie \glq wacklig\grq{} die FE-Ergebnisse sind, dann fragt man sich manchmal, ob die Relationen noch stimmen. Nat\"{u}rlich kann man auf der Materialseite keine R\"{u}cksicht darauf nehmen, dass Kr\"{a}fte nur n\"{a}herungsweise bekannt sind, man muss so tun, als ob sie exakt w\"{a}ren, aber exakte Resultate gibt es in der Praxis eben leider nicht.

%Es w\"{a}re dringend notwendig, \"{u}ber den Umgang mit FE-Ergebnissen zu diskutieren und nicht l\"{a}nger die Materialseite und die Numerik getrennt zu behandeln.


%%%%%%%%%%%%%%%%%%%%%%%%%%%%%%%%%%%%%%%%%%%%%%%%%%%%%%%%%%%%%%%%%%%%%%%%%%%%%%%%%%%%%%%%%%%%%%%%%%%
\textcolor{sectionTitleBlue}{\section{Kopplung von Bauteilen}}
Wenn man einen Stab an zwei Knoten einer Scheibe festmacht, dann bedeutet $f_i^{Stab} = f_i^{Scheibe}$, dass bei einer Verschiebung des Lagerknotens $\delta u_i = 1$ (alle anderen Knoten sind dabei fest) die zugeh\"{o}rigen inneren Energien in den beiden Bauteilen gleich gro{\ss} sind. Es ist keine statische Kopplung, weil das $f_i$ auf der Seite der Scheibe keine echte Einzelkraft ist.

Nur bei eindimensionalen Problemen, $EI w^{IV} = p$, $-EA u'' =  p$ etc., und dann auch nur bei konstanten Steifigkeiten, sind die $f_i$ (dimensionsbereinigt) echte Knotenkr\"{a}fte.


Wie sich die einzelnen Elemente eines Netzes zu einem Ganzen f\"{u}gen, kann man mit Hilfe der {\em Inzidenzmatrizen\/}\index{Inzidenzmatrizen} verfolgen.

%-----------------------------------------------------------------
\begin{figure}[tbp]
\centering
\includegraphics[width=1.0\textwidth]{\Fpath/U387A}
\caption{Kopplung zweier Stabelemente }
\label{U387}
\end{figure}%
%-----------------------------------------------------------------

Die beiden Elementmatrizen des Stabes in Abb. \ref{U387} stehen auf der Diagonalen einer $4 \times 4$ Matrix\footnote{Alle Elementmatrizen einzeln, unverbunden und $\vek u_{loc}$  und $\vek f_{loc}$ ist die Liste der Weg- und Kraftgr\"{o}{\ss}en an den Elementenden } $\vek K_{loc}$\index{$\vek K_{loc}$}
\begin{align}\label{Eq193}
\vek f_{loc} = \left[ \barr{c} f_1^a \\ f_2^a \\ f_1^b \\ f_2^b \earr \right] =  \frac{EA}{\ell_e} \left [
\barr {r @{\hspace{2mm}} r @{\hspace{2mm}} r @{\hspace{2mm}} r} 1 & -1 & 0 & 0 \\ -1 & 1 & 0 & 0 \\ 0 & 0 & 1 & -1 \\ 0 & 0 & -1 & 1 \\ \earr \right] \left[ \barr{cc} u_1^a \\ u_2^a \\ u_1^b \\ u_2^b \earr \right] = \vek K_{loc}\,\vek u_{loc}\,.
\end{align}
Die Verschiebungen der Elementenden sind an die Knotenverschiebungen $u_1, u_2, u_3$ gekoppelt
\begin{align}
\vek u_{loc} = \left[ \barr{c} u_1^a \\ u_2^a \\ u_1^b \\ u_2^b \earr \right] = \left [
\barr {r @{\hspace{2mm}} r @{\hspace{2mm}} r } 1 & 0 & 0  \\ 0 & 1 & 0 \\ 0 & 1 & 0 \\ 0 & 0 &  1 \\ \earr \right] \left[ \barr{cc} u_1 \\ u_2 \\ u_3 \earr \right] = \vek  A\,\vek u\,.
\end{align}
Die Kr\"{a}fte $\vek f_{loc}$ und die Knotenkr\"{a}fte $\vek f$ m\"{u}ssen bei einer virtuellen Verr\"{u}ckung $\vek \delta \vek u$ bzw. $\vek \delta \vek u_{loc} = \vek A\,\vek \delta \vek  u$ die gleiche Arbeit leisten
\begin{align}
\vek f_{loc}^T\,\vek \delta \vek u_{loc}^{\vphantom{T}} = \vek f^T\,\vek \delta \vek u \qquad \text{oder} \qquad \vek f_{loc}^T\,\vek A\,\vek  \delta\vek u = \vek f^T\,\vek  \delta\vek u\,,
\end{align}
was $\vek f = \vek A^T\,\vek f_{loc}$ ergibt und das sind nat\"{u}rlich gerade die Gleichgewichtsbedingungen zwischen den Stabendkr\"{a}ften und den Knotenkr\"{a}ften $f_i$
\begin{align}
\vek f = \left[ \barr{c} f_1 \\ f_2 \\ f_3 \earr \right] = \left [
\barr {r @{\hspace{2mm}} r @{\hspace{2mm}} r @{\hspace{2mm}} r} 1 & 0 & 0 & 0 \\ 0 & 1 & 1 & 0 \\ 0 & 0 & 0 & 1 \earr \right] \left[ \barr{c} f_1^a \\ f_2^a \\ f_1^b \\ f_2^b \earr \right] = \vek A^T\,\vek f_{loc}\,.
\end{align}
Entsprechend erh\"{a}lt man durch Multiplikation der Matrix $\vek K_{loc}$ von links und rechts mit
$\vek A^T$ bzw. $\vek A$ die Gesamtsteifigkeitsmatrix
\begin{align}
\vek K = \vek A^T\,\vek K_{loc}\,\vek A = \frac{EA}{\ell_e} \left [
\barr {r @{\hspace{2mm}} r @{\hspace{2mm}} r} 1 & -1 & 0  \\ -1 & 2 &-1 \\ 0 &-1 &1\earr \right]\,.
\end{align}

%-----------------------------------------------------------------
\begin{figure}[tbp]
\if \bild 2 \sidecaption[t] \fi
\centering
\includegraphics[width=0.99\textwidth]{\Fpath/U548Q}
\caption{Kopplung Scheibe -- Balken. Die drei Schnittkr\"{a}fte $F_{B,x}, F_{B,y}, M_B$ des Balkens werden in \"{a}quivalente Knotenkr\"{a}fte $F_{x,i}, F_{y,i}$ rechts umgerechnet, so dass sie zur Scheibe links passen \cite{Werkle3} }
\label{U548}
\end{figure}%
%-----------------------------------------------------------------
%%%%%%%%%%%%%%%%%%%%%%%%%%%%%%%%%%%%%%%%%%%%%%%%%%%%%%%%%%%%%%%%%%%%%%%%%%%%%%%%%%%%%%%%%%%%%%%%%%%
\textcolor{chapterTitleBlue}{\section{\"{A}quivalente Spannungs Transformation}}\label{AST}
Die Kopplung von gleichartigen Elementen untereinander stellt also kein Problem dar. Schwieriger ist es aber z.B. St\"{u}tzen (Balken) und Scheiben miteinander zu koppeln, weil Balkenenden Drehfreiheitsgrade haben, die den Knoten einer Scheibe fehlen.

Die  {\em \"{A}quivalente Spannungs Transformation\/} (EST) von Werkle, \cite{Werkle1}, l\"{o}st dieses Problem auf sehr elegante, ja nat\"{u}rliche Weise. Bei ihr z\"{a}umt man das Pferd sozusagen von hinten auf. Normalerweise geht man bei der Formulierung einer Steifigkeitsmatrix
\begin{align}
\vek K = \vek A^T\vek K_{loc} \vek A
\end{align}
ja so vor\footnote{Auf der Diagonalen von $\vek K_{loc} $ stehen die Elementmatrizen, s. (\ref{Eq193}) }, dass man erst die Kopplung zwischen den Weggr\"{o}{\ss}en beschreibt,
\begin{align} \label{Eq186}
\vek u_{loc} = \vek A\,\vek u_{Knoten}
\end{align}
und dann mit der Transponierten $\vek A^T$ die zugeh\"{o}rigen Gleichgewichtsbedingungen formuliert\index{Aequivalente Spannungs Transformation}
\begin{align} \label{Eq187}
\vek f_{Knoten} = \vek A^T\,\vek f_{loc}\,.
\end{align}
Bei der \"{a}quivalenten Spannungs Transformation ist es umgekehrt. Bei ihr wird zuerst die Abbildung (\ref{Eq187}) formuliert -- hier ist das statische Verst\"{a}ndnis und das Geschick des Ingenieurs gefragt -- und diese Matrix $\vek A^T$ wird dann in transponierter Form in (\ref{Eq186}) \"{u}bernommen.

An dieser Stelle zeigt sich -- und das war vorher nicht so deutlich -- dass es zwei Wege gibt, den Zusammenhang $\vek A$ der Elemente zu beschreiben, den geometrischen Pfad $\vek u_{Knoten} \to \vek u_{loc}$ oder den statischen Pfad $\vek f_{loc} \to \vek f_{Knoten}$.\\


\textcolor{chapterTitleBlue}{\subsubsection*{Beispiel}}

Wie man diese Technik nutzen kann um eine Steifigkeitsmatrix herzuleiten, die die Kopplung eines Balkens an eine Scheibe beschreibt, soll das folgende Beispiel zeigen.

Die Situation zeigt Abb. \ref{U548}; drei Knoten mit den Freiheitsgraden  $u_i, v_i$ liegen dem Balken gegen\"{u}ber. Beim geometrischen Pfad (\ref{Eq186}) macht man die Annahme, dass der Querschnitt des Balkens eben bleibt, also mit $a = d/2$, halbe Tr\"{a}gerh\"{o}he,
\begin{align}
u_1 = u_B + a\,\tan \Np_B, \quad u_2 = u_B, \quad u_3 = u_B - a\,\tan \Np_B, \quad v_1 = v_2 = v_3
\end{align}
und damit lautet (\ref{Eq186}) ausgeschrieben
\begin{align}
\left[\barr{c} u_1^{(1)} \\ v_1^{(1)}  \\u_2^{(1)} \\ v_2^{(1)} \\ u_2^{(2)} \\ v_2^{(2)} \\ u_3^{(2) } \\ v_3^{(2)}\earr \right] = \left[\barr{c @{\hspace{6mm}} c @{\hspace{2mm}} c} 1 &0 & a \\ 0 & 1 & 0 \\ 1 & 0 & 0 \\ 0 & 1 & 0  \\ 1 & 0 & 0 \\ 0 &1 & 0 \\ 1 & 0 &- a \\ 0 & 1 & 0 \earr \right] \,\left[\barr{c} u_B \\ v_B \\ \tan \Np_B \earr \right]\,.
\end{align}
Entscheidend ist, dass hier ein linearer Verlauf der Verschiebungen angenommen wurde -- eine Annahme, die nat\"{u}rlich so nicht richtig ist. Richtig im Sinne der Elastizit\"{a}tstheorie w\"{a}re ein  Verschiebungsverteilung, wie sie sich bei einer Berechnung des Anschlusses als Scheibe, d.h. der Modellierung des Balkens mit Scheibenelementen, einstellt.

Beim statischen Pfad  geht man dagegen \"{u}ber die Kr\"{a}fte. Die Schnittgr\"{o}{\ss}en $F_{Bx}, F_{By}$ und $M_B$, s. Abb. \ref{U548}, erzeugen, bei Ansatz der Biegebalkentheorie, die Spannungen\footnote{Vorzeichen gem\"{a}{\ss} Abb. \ref{U548}} (Rechteckquerschnitt, $t$ = Wandst\"{a}rke)
\begin{align}
p_x &= \frac{F_{B,x}}{A} - \frac{M_B}{I}\,y_B = \frac{F_{B,x}}{d\,t} - \frac{12\,M_B}{t\,d^3}\,y_B \\
p_y &= \frac{3}{2}\,\frac{1}{d\,t} (1 - 4\,\frac{y_B^2}{d^2})\,F_{B,y}\,,
\end{align}
in der Stirnfl\"{a}che des Balkens, die man in \"{a}quivalente Knotenkr\"{a}ften $\vek f$ auf der Seite der Scheibe umrechnen kann und so kommt man zu einer Beziehung zwischen den Kr\"{a}ften auf den beiden Seiten des Schnittufers.

Bei dieser Technik werden erst aus dem Vektor $\vek f_B = \{F_{B,x}, F_{B,y}, M_B\}^T$ die Knotenwerte $p_i$ der Spannungen $\sigma$ und $\tau$ ermittelt. Die Knotenwerte fassen wir zu einem  Vektor $\vek p $ zusammen und so kann der \"{U}bergang $\vek f_B \to \vek p$ mit einer Matrix $\vek P$ (wie Polynome) beschrieben werden.

Mit den $y$-Koordinaten der Punkte 1, 2 und 3 (Achse $y_B$ bezogen auf den Schwerpunkt des Balkens)
\begin{align}
y_{B,1} = - a, \qquad y_{B,2} = 0, \qquad y_{B,3} = a
\end{align}
erh\"{a}lt man mit den obigen Formeln die Knotenwerte der Spannungen zu
\begin{align}
\left[\barr{c} p_{x,1} \\ p_{y,1}  \\p_{y,m,1-2} \\ p_{x,2} \\ p_{y,2} \\ p_{y,m,2-3} \\  p_{x,3} \\  p_{y,3}\earr \right] = \frac{ 1}{16\,t\,a^2}\left[\barr{c @{\hspace{6mm}} c @{\hspace{2mm}} c} 8\,a &0 & 24 \\ 0 & 0 & 0 \\ 0 & 9\,a & 0 \\ 8\,a &0 & 0 \\ 0 &12\,a &0 \\ 0 & 9\,a & 0 \\ 8\,a &0 & -24 \\ 0 & 0 &0  \earr \right]\,\left[\barr{c} F_{B,x} \\ F_{B,y} \\ M_B \earr \right]
\end{align}
oder
\begin{align}\label{Eq26}
\vek p = \vek P\,\vek f_B\,.
\end{align}
Die Berechnung der \"{a}quivalenten Knotenkr\"{a}fte $\vek f_S$ aus den als Linienlasten aufgefassten Spannungen $p_x$ und $p_y$ geschieht, wie es die Regel ist, durch die \"{U}berlagerung der Spannungen mit den {\em shape functions\/}. Das Ergebnis hat formal die Gestalt
\begin{align}
\vek f_S = \vek Q\,\vek p\,.
\end{align}
mit einer Matrix, die wir $\vek Q$ (wie Quadratur) nennen.

Diese Beziehung wird zun\"{a}chst f\"{u}r jedes Element separat ermittelt und daraus dann die Gesamtmatrix $\vek Q $ gebildet. Nach Bild \ref{U548} sind hier zwei Scheibenelemente zu ber\"{u}cksichtigen. Man erh\"{a}lt\footnote{nach \cite{Werkle2} Glg. (4.78c, d), S. 259 und Glg. (4.86a, b), S. 262}, am Element 1
\begin{align}
\left[\barr{c} F_{x,1}^{(1)} \\ F_{x,2}^{(1)} \earr \right] =
\frac{a\,t}{6} \left[\barr{c @{\hspace{6mm}} c} 2 & 1 \\ 1 &2  \earr \right]\,\left[\barr{c} p_{x,1}\\ p_{x,2} \earr \right]
\end{align}
und
\begin{align}
\left[\barr{c} F_{y,1}^{(1)} \\ F_{y,2}^{(1)} \earr \right] =
\frac{a\,t}{12} \left[\barr{c @{\hspace{6mm}} c @{\hspace{6mm}} c} 3 & 4 &-1 \\ -1 &4 &3  \earr \right]\,\left[\barr{c} p_{y,1} \\ p_{y,m,1-2} \\ p_{y,2} \earr \right]\,.
\end{align}
Damit lautet am Element 1 die Beziehung
\begin{align}
\left[\barr{c} F_{x,1}^{(1)} \\ F_{y,1}^{(1)} \\ F_{x,2}^{(1)} \\ F_{y,2}^{(1)}\earr \right]
= \frac{ a\,t}{12}\,\left[\barr{c @{\hspace{3mm}} c @{\hspace{3mm}} c @{\hspace{3mm}} c @{\hspace{3mm}} c} 4 & 0 & 0 &2 & 0 \\ 0 & 3 & 4 &0 & -1\\  2 & 0 & 0 &4 & 0 \\
 0 & -1 & 4 &0 & 3  \earr \right]\,\left[\barr{c} p_{x,1} \\ p_{y,1} \\ p_{y,m,1-2} \\ p_{x,2} \\ p_{y,2} \earr \right]
\end{align}
und entsprechend am Element 2
\begin{align}
\left[\barr{c} F_{x,2}^{(2)} \\ F_{y,2}^{(2)} \\ F_{x,3}^{(2)} \\ F_{y,3}^{(2)}\earr \right]
= \frac{ a\,t}{12}\,\left[\barr{c @{\hspace{3mm}} c @{\hspace{3mm}} c @{\hspace{3mm}} c @{\hspace{3mm}} c} 4 & 0 & 0 &2 & 0 \\ 0 & 3 & 4 &0 & -1\\  2 & 0 & 0 &4 & 0 \\
 0 & -1 & 4 &0 & 3  \earr \right]\,\left[\barr{c} p_{x,2} \\ p_{y,2} \\ p_{y,m,2-3} \\ p_{x,3} \\ p_{y,3} \earr \right]\,.
\end{align}
Den Vektor der Knotenkr\"{a}fte, die auf die Scheibe an der Verbindung wirken, ergibt sich durch Addition der Elementkr\"{a}fte der einzelnen Elemente, s. Abb. \ref{U548}, und man erh\"{a}lt so die Matrix $\vek Q $ zu
\begin{align}
\left[\barr{c} F_{x,1} \\ F_{y,1} \\ F_{x,2} \\ F_{y,2} \\ F_{x,3} \\ F_{y,3}\earr \right] = \frac{a\,t}{12}\,\left[\barr{c @{\hspace{3mm}} c @{\hspace{3mm}} c @{\hspace{3mm}} c @{\hspace{3mm}} c @{\hspace{3mm}} c @{\hspace{3mm}} c @{\hspace{3mm}} c} 4 &0 &0 &2 &0 &0 &0 &0\\
0 &3 &4 &0 &-1 &0 &0 &0 \\
2 &0 &0 &8 &0 &0 &2 &0 \\ 0 &-1 &4 &0 &6 &4 &0 &-1 \\ 0 &0 & 0 &2 &0 &0 &4 &0 \\
0 & 0 &0 &0 &-1 &4 &0 &3\earr \right]
\left[\barr{c} p_{x,1} \\ p_{y,1} \\ p_{y,m,1-2} \\ p_{x,2} \\ p_{y,2} \\ p_{y,m,2-3} \\ p_{x,3} \\ p_{y,3}\earr \right]
\end{align}
oder
\begin{align}
\vek f_S = \vek Q\,\vek p\,.
\end{align}
Die Kr\"{a}fte $\vek f_S$ sind die Kr\"{a}fte rechts in Abb. \ref{U548}.
Mit (\ref{Eq26}) folgt weiter
\begin{align}
\vek f_S = \vek Q\,\vek P\,\vek f_B = \vek A^T\,\vek f_B
\end{align}
und somit lautet die Transformationsmatrix
\begin{align}\label{Eq27}
\vek A = \vek P^T\,\vek Q^T = \left[\barr{c @{\hspace{3mm}} c @{\hspace{3mm}} c @{\hspace{3mm}} c
@{\hspace{3mm}} c @{\hspace{3mm}} c} 1/4 & 0 &1/2 & 0 &1/4 & 0 \\
0 &1/8 & 0 &3/8 &0 &1/8\\
1/2\,a & 0 & 0& 0  &-1/2\,a & 0\earr \right]\,.
\end{align}
%-----------------------------------------------------------------
\begin{figure}[tbp]
\if \bild 2 \sidecaption[t] \fi
\centering
\includegraphics[width=0.5\textwidth]{\Fpath/U549}
\caption{Stabelement }
\label{U549}
\end{figure}%
%-----------------------------------------------------------------
Die Weg- und Kraftgr\"{o}{\ss}en am Stabende
\begin{align}
\vek f_B = \left[\barr{c} F_{B,x} \\ F_{B,y} \\ M_{B} \earr \right]\,, \quad
\vek u_B = \left[\barr{c} u_B \\ v_B \\ \Np_B \earr \right]\,,\quad
\vek f_S = \left[\barr{c} F_{x,1} \\ F_{y,1} \\ F_{x,2} \\ F_{y,2} \\ F_{x,3} \\ F_{y,3}\earr \right]\,,\quad
\vek u_S = \left[\barr{c} u_1 \\ v_1 \\ u_2 \\ v_2 \\ u_3 \\ v_3 \earr \right]
\end{align}
transformieren sich also wie
\begin{align} \label{Eq990}
\vek u_B = \vek A_{(3 \times 6)}\,\vek u_S \qquad \vek f_S = \vek A^T_{(6 \times 3)}\,\vek f_B\,.
\end{align}
Am Stab, s. Abb. \ref{U549}, lauten die Beziehungen zwischen den Weg- und Kraftgr\"{o}{\ss}en
\begin{align}
\left[\barr{c @{\hspace{3mm}} c @{\hspace{3mm}} c @{\hspace{3mm}} c @{\hspace{3mm}} c @{\hspace{3mm}} c @{\hspace{3mm}} c @{\hspace{3mm}} c} a_0 & 0 & 0 &- a_0 & 0 & 0 \\
0 & 12\,a_1/\ell^2 & 6\,a_1/\ell & 0 &-12\,a_1/\ell^2& 6\,a_1/\ell \\
0 &6\,a_1/\ell & 4\,a_1 & 0 &-6\,a_1/\ell & 2\,a_1 \\
-a_0 & 0 & 0 & a_0 & 0 & 0 \\
0 & -12\,a_1/\ell^2 & -6\,a_1/\ell & 0 &12\,a_1/\ell^2& -6\,a_1/\ell \\
0 &6\,a_1/\ell & 2\,a_1 & 0 &-6\,a_1/\ell & 4\,a_1
\earr \right]\,\left[\barr{c} u_a \\ v_a \\ \Np_a \\ u_b \\v_b \\ \Np_b \earr \right] =
\left[\barr{c} F_{x,a} \\ F_{y,a} \\ M_a \\ F_{x,b} \\ F_{y,b} \\ M_b\earr \right]
\end{align}
mit
\begin{align}
a_0 &= \frac{E A}{\ell}\,, \quad a_1 = \frac{E I}{\ell}\,,  \quad \ell = \text{L\"{a}nge des Stabes}.
\end{align}
Das Balkenelement besitzt die Knoten $a $ und $b $. Entsprechend wird die obige Steifigkeitsmatrix des Balkens nun in Untermatrizen, die sich auf die Knoten $a $ und $b $ beziehen, unterteilt
\begin{align} \label{Eq30}
\left[\barr{c @{\hspace{3mm}} c } \vek K_{aa} & \vek K_{ab} \\
\vek K_{ba} &\vek K_{bb}\earr \right]\,\left[\barr{c} \vek u_a \\ \vek u_b\earr \right] = \left[\barr{c} \vek f_a \\ \vek f_b\earr \right]\,.
\end{align}
Beim Anschluss des Knoten $a $ an zwei Scheibenelemente wie in Abb. \ref{U548} lauten die Weg- und Kraftgr\"{o}{\ss}en im Knoten $a$ in der Notation der EST
\begin{align}
\vek u_a = \vek u_B =\left[\barr{c} u_B \\ v_B \\ \Np_B\earr \right] \qquad \vek f_a = \vek f_B = \left[\barr{c}  F_{B,x} \\  F_{B,y} \\ M_B\earr \right]\,.
\end{align}
Wenn die Verschiebungen in den finiten Elementen linear verlaufen, transformieren sich die Gr\"{o}{\ss}en, s.o., gem\"{a}{\ss}
\begin{align}
\vek u_B = \vek A_{(3 \times 6)}\,\vek u_S
\end{align}
mit der Matrix $\vek A$ wie in (\ref{Eq27}). Setzen wir nun $\vek u_a = \vek u_B = \vek A\,\vek u_S$ in (\ref{Eq30}) ein, multiplizieren dann die erste Zeile von links mit $\vek A^T$, so ergibt sich mit
\begin{align}
\vek A^T_{(6 \times 3)}\,\vek f_a = \vek A^T_{(6 \times 3)}\,\vek f_B = \vek f_S
\end{align}
das Resultat
\begin{align}\label{Eq31}
\left[\barr{c @{\hspace{3mm}} c } \vek A^T\,\vek K_{aa}\,\vek A & \vek A^T\vek K_{ab} \\
\vek K_{ba}\,\vek A &\vek K_{bb}\earr \right]\,\left[\barr{c} \vek u_S \\ \vek u_b\earr \right] = \left[\barr{c} \vek f_S \\ \vek f_b\earr \right]\,.
\end{align}
Das ist eine $9 \times 9 $ Matrix, 6 {\em dofs\/} $\vek u_S$ an der Scheibe und 3 {\em dofs\/} $\vek u_b$ am Balken, und die $\vek f_S$ sind die sechs Knotenkr\"{a}fte an der Scheibe\footnote{{\em dofs\/} = {\em degrees of freedom\/}, Freiheitsgrade}.

Die Steifigkeitsmatrix (\ref{Eq31}) ist nun im Knoten $a $ auf die Freiheitsgrade des Scheibenmodells und im Knoten $b$ auf die Freiheitsgrade des Stabes bezogen. Knoten $b$ kann, falls er ebenfalls an ein Scheibenmodell angeschlossen ist, ebenfalls transformiert werden.

Zum Verst\"{a}ndnis sei gesagt, dass der statische Pfad hier nur zur Herleitung der Matrix (\ref{Eq31}) benutzt wird. Der  Zusammenbau aller Elementmatrizen zur Gesamtsteifigkeitsmatrix erfolgt dann wie sonst auch.\\

\begin{remark}
Der geometrische Pfad und der statische Pfad beruhen auf unterschiedlichen Annahmen, die zu unterschiedlichen Ergebnissen f\"{u}hren. Beim geometrischen Pfad werden die Verschiebungsverl\"{a}ufe vorgegeben, und die Spannungen der Scheibenelemente passen sich diesen Vorgaben an. Beim statischen Pfad werden dagegen die Spannungsverl\"{a}ufe vorgegeben und die Verschiebungen (hier die Punkte 1, 2 und 3) k\"{o}nnen sich anpassen und weichen dann aber von der linearen Verteilung des geometrischen Pfads ab.

Beim geometrischen Pfad ist die Verbindung zu steif, beim statischen Pfad ist sie zu weich. Der geometrische Pfad bedeutet jedoch -- insbesondere bei der Verbindung von St\"{u}tzen mit Platten wie bei einer Flachdecke -- einen starren Einschluss im FE-Modell. Die FEM kommt jedoch mit starren Einfl\"{u}ssen nicht gut klar, d.h. es ergeben sich im Verbindungsbereich Spannungssingularit\"{a}ten und damit stark fehlerhafte Elementspannungen. Dies ist beim statischen Pfad nicht der Fall und daher sollte man dem statischen Pfad den Vorzug geben. F\"{u}r weitere Details verweisen wir auf  \cite{Werkle2}.
\end{remark}

%----------------------------------------------------------
\begin{figure}[tbp]
\centering
\if \bild 2 \sidecaption[t] \fi
\includegraphics[width=.8\textwidth]{\Fpath/U72}
\caption{Berechnung der Einflussfunktion f\"{u}r eine Verschiebung $u(x)$, \textbf{a)} Dachfunktionen und Originalbelastung, \textbf{ b)} Ersatzkr\"{a}fte und daraus resultierende FE-Einflussfunktion} \label{U72}
\end{figure}%%
%----------------------------------------------------------



%%%%%%%%%%%%%%%%%%%%%%%%%%%%%%%%%%%%%%%%%%%%%%%%%%%%%%%%%%%%%%%%%%%%%%%%%%%%%%%%%%%%%%%%%%%%%%%%%%%
{\textcolor{sectionTitleBlue}{\section{Berechnung von Einflussfunktionen mit finiten Elementen}}}\label{InfFEM}
Wir kommen nun zu einem sehr wichtigen Thema, der Berechnung von Einflussfunktionen mit finiten Elementen.

Auch die Berechnung von Einflussfunktionen mit finiten Elementen f\"{u}hrt auf das Gleichungssystem $\vek K\,\vek u = \vek f$, nur dass wir die $u_i$ jetzt $g_i$ nennen und statt $f_i$ schreiben wir $j_i$
\begin{align}
\vek K\,\vek g = \vek j\,.
\end{align}
Dieser Namenswechsel erleichtert das Operieren mit FE-Einflussfunktionen.

Unser erstes Probest\"{u}ck ist die Berechnung der Einflussfunktion f\"{u}r die L\"{a}ngsverschiebung $u(x)$ des Stabes in Abb. \ref{U72} im Punkt $x = 2.5$. Die \"{a}quivalenten Knotenkr\"{a}fte sind in diesem Fall die Verschiebungen der Ansatzfunktionen $\Np_i(x)$ in dem Punkt $x = 2.5$
\begin{align}\label{Eq110}
\Np_1(2.5) = 0 \qquad \Np_2(2.5) = 0.5 \qquad \Np_3(2.5) = 0.5 \qquad \Np_4(2.5) = 0\,,
\end{align}
was auf  das Gleichungssystem
\begin{align}\label{Eq68}
\frac{EA}{l_e} \left[\barr{r r r r} 2 & - 1 & 0 & 0 \\ - 1 & 2 & -1 & 0\\ 0 & -1 & 2 &-1 \\ 0 & 0 & -1 &2\earr\right]
\,\left[\barr{c} g_1 \\g_2 \\ g_3 \\ g_4 \earr \right] = \left[\barr{c} 0 \\ 0.5  \\
0.5  \\ 0 \earr \right]
\end{align}
f\"{u}r die Knotenverschiebungen $g_i$ f\"{u}hrt. Dieses System  hat die L\"{o}sung
\begin{align}
g_1 = 1\qquad g_2 = 2\qquad g_3 = 2.5\qquad g_4 = 2.5\,,
\end{align}
und somit hat die Einflussfunktion die Gestalt
\begin{align}
G_0^h(y,x = 2.5) = \frac{l_e}{EA}\, [1 \cdot \Np_1(y) + 2 \cdot \Np_2(y) + 2.5 \cdot \Np_3(y) + 2.5 \cdot\Np_4(y)]\,.
\end{align}
Da $x$ schon den Aufpunkt bezeichnet, benutzen wir als Lauf- oder Summationsvariable den Buchstaben $y$.

Die FE-Einflussfunktion ist, bis auf das Element, in dem der Aufpunkt $x$ liegt, exakt.
Den Fehler in dem Element beheben die FE-Programme dadurch, dass sie zur FE-L\"{o}sung die lokale L\"{o}sung addieren
\begin{align}
G_0(y,x) = G_0^h(y,x) + \text{lokale L\"{o}sung}\,.
\end{align}
So gelingt es den FE-Programmen exakte Einflussfunktionen f\"{u}r Stabtragwerke zu generieren -- vorausgesetzt $EA$ und $EI$ sind konstant.\\

{\textcolor{sectionTitleBlue}{\subsubsection*{Der Schl\"{u}ssel zu den Knotenkr\"{a}ften $ j_i$}}}\index{Schl\"{u}ssel zu den  $j_i$}

Warum sind bei dem obigen Beispiel die \"{a}quivalenten Knotenkr\"{a}fte $j_i$ (= $f_i$) gleich den Werten der Ansatzfunktionen im Aufpunkt, $j_i = \Np_i(2.5)$? Der Schl\"{u}ssel hierzu liegt in der Definition der \"{a}quivalenten Knotenkr\"{a}fte $f_i$.

%----------------------------------------------------------
\begin{figure}[tbp]
\centering
\if \bild 2 \sidecaption[t] \fi
\includegraphics[width=.99\textwidth]{\Fpath/U451A}
\caption{Berechnung der vier Einflussfunktionen eines Balkens mit finiten Elementen. Die \"{a}quivalenten Knotenkr\"{a}fte (hier ohne Vorzeichen---das steckt in den Pfeilen) sind die Werte der Ansatzfunktionen im Aufpunkt $x = 0.5\,\ell$. Wenn zu den FE-L\"{o}sungen noch die lokalen L\"{o}sungen addiert werden, sind die Ergebnisse auch im Element mit dem Aufpunkt exakt---sonst nur au{\ss}erhalb von dem Element. Die $j_i$ findet man in (\ref{Eq219}) und (\ref{Eq219X}). } \label{U451}
\end{figure}%%
%----------------------------------------------------------
Die Knotenkraft $f_i$ ist eine Arbeit und zwar die Arbeit, die die Belastung $p(x)$  auf den Wegen der Ansatzfunktion $\Np_i(x)$ leistet
\begin{align}
f_i = \int_0^{\,l} p(x)\,\Np_i(x)\,dx\,.
\end{align}
Bei Einflussfunktionen steht auf der rechten Seite der Differentialgleichung ein Dirac Delta (eine in einem Punkt konzentrierte Linienlast)
\begin{align}
-EA\,\frac{d^2}{dy^2}\,G_0(y,x) = \delta_0(y-x) \qquad \leftarrow \,\,\text{[kN/m]}\,,
\end{align}
die eine Einzelkraft im Aufpunkt $x = 2.5$ repr\"{a}sentiert. Sie ist sozusagen das $p$, das zur Einflussfunktion geh\"{o}rt. (Wir differenzieren auf der linken Seite nach $y$, dies ist hier die Laufvariable).

Jetzt rechnen wir und finden, dass die \"{a}quivalenten Knotenkr\"{a}fte ($[\text{kN} \cdot \text{m}]$)
\begin{align}
j_i = \int_0^{\,l} \underbrace{\delta_0(y-x)}_{[\text{kN/m}]}\,\underbrace{\Np_i(y)}_{[\text{m}]}\,\underbrace{dy}_{\text{[m]}} = \Np_i(x)  \qquad x = 2.5
\end{align}
zahlenm\"{a}{\ss}ig einfach die Werte der vier Ansatzfunktionen $\Np_i$ im Aufpunkt $x = 2.5$ sind, so kommt die Liste (\ref{Eq110}) zustande, siehe auch Abb. \ref{U451}.

%%%%%%%%%%%%%%%%%%%%%%%%%%%%%%%%%%%%%%%%%%%%%%%%%%%%%%%%%%%%%%%%%%%%%%%%%%%%%%%%%%%%%%%%%%%%%%%%%%%
{\textcolor{sectionTitleBlue}{\section{Funktionale}}}
Um nun doch etwas systematischer vorzugehen, wollen wir den Begriff des Funktionals\index{Funktional} einf\"{u}hren.

Wenn wir die Komponente $u_i$ eines Vektors $\vek u$ abfragen, dann werten wir streng genommen das Funktional
\begin{align}
J_i(\vek u) = u_i = \vek e_i^T\,\vek u
\end{align}
aus, das dem Skalarprodukt zwischen dem Einheitsvektor $\vek e_i$, dem diskreten Dirac Delta, und $\vek u$ entspricht.

Geht es um die Auswertung von Funktionen, dann sind die Funktionale Integrale, wie etwa der Ausdruck
\begin{align} \label{Eq43}
J(u) = \int_0^{\,l} u(x)\,dx\,.
\end{align}
Der Wert dieses Funktionals ist gleich dem Integral der Funktion $ u(x) $ \"{u}ber das Intervall $(0,l)$.
Funktionale sind also im allgemeinen Funktionen von Funktionen.

Funktionale wie
\begin{align}
J(u) = u(0)
\end{align}
nennt man {\em Punktfunktionale\/}\index{Punktfunktionale}, weil sie einen Wert in einem Punkt zur\"{u}ckgeben
\begin{align}
J(\sin(x)) = \sin(0) = 0 \quad J(e^x) = e^0 = 1\,.
\end{align}
Lineare Funktionale\index{lineares Funktional} wie das Integral einer Funktion
\begin{align}
J(u_1 + u_2) &= \int_0^{\,l} (u_1(x) + u_2(x))\,dx \nn \\
& = \int_0^{\,l} u_1(x)\,dx + \int_0^{\,l} u_2(x)\,dx = J(u_1) + J(u_2)\,,
\end{align}
lassen sich \glq superponieren\grq{}.

Jede Lagerkraft, jede Durchbiegung, jedes Moment, etc., ist ein (nat\"{u}rlich jeweils anderes) Funktional $J(w)$
\begin{align}\label{Eq20}
J(w) = V(0) \qquad J(w) = w(x) \qquad J(w) = M(x)\,.
\end{align}
Der entscheidende Schritt ist nun, dass wir die Auswertung eines linearen Funktionals auf ein Arbeitsintegral, ein $L_2$-Skalarprodukt zur\"{u}ckf\"{u}hren und dabei das Dirac Delta als Vorbild nehmen.

Das Funktional $J(w) = w(x)$, die Durchbiegung des Tr\"{a}gers in einem Punkt $x$, ist im Ergebnis die \"{U}berlagerung von $w$ mit einem Dirac Delta $\delta(y-x)$ (= Punktlast)
\begin{align}\label{Eq125}
J(w) = 1 \cdot w(x) = \int_0^{\,l} \delta(y-x)\,w(y)\,dy \qquad [\text{kNm}]\,.
\end{align}
Das Dirac Delta spielt hier dieselbe Rolle, wie oben der Einheitsvektor $\vek e_i$.

Und diese Interpretation wenden wir jetzt konsequent an. Jedes Funktional ist f\"{u}r uns das Ergebnis der \"{U}berlagerung eines Dirac Deltas mit $w$
\begin{align}
J(w) = \int_0^{\,l} \delta(y-x)\,w(y)\,dy \,.
\end{align}
Wie wir gleich sehen werden, m\"{u}ssen wir gar nicht wissen, wie diese verschiedenen Dirac Deltas aussehen. Wir m\"{u}ssen nur wissen, was das Ergebnis $J(w)$ sein soll. Und weil wir jedes Funktional als Arbeitsintegral lesen, hat jedes so dargestellte Funktional die Dimension einer Arbeit
\begin{align}
J(w) = 1 \cdot \text{\glq irgendetwas\grq{}}
\end{align}
wobei die 1 immer eine solche Dimension hat, dass $1 \cdot\text{\glq irgendetwas\grq{}}$ die Dimension einer Arbeit hat, die 1 also konjugiert zu ihrem Begleiter ist. So lesen wir die Funktionale in (\ref{Eq20}) wie folgt
\begin{alignat}{3}
J(w) &= V(x) \cdot 1 &&= \text{kN}  \cdot \text{m}\, &&\quad\text{1 = Versetzung} \\
J(w) &= w(x) \cdot 1 &&= \text{m}   \cdot \text{kN}\,&&\quad\text{1 = Kraft}\\
J(w) &= M(x) \cdot 1 &&= \text{kNm} \cdot [\,] \,      &&\quad\text{1 = Knick }\,.
\end{alignat}
Der Knick ist ein Sprung in der Ableitung $w'$ also im Tangens, der ja als Quotient zweier L\"{a}ngen, $dw/dx$, keine Dimension hat. Im Folgenden schreiben wir die $1$ meist nicht mit an, aber wir denken sie dann immer mit.

Die Einf\"{u}hrung der Dirac Deltas hat den Vorteil, dass wir die zugeh\"{o}rige Einflussfunktion als die L\"{o}sung der Differentialgleichung (wir bleiben der Einfachheit halber beim Balken) mit dem Dirac Delta auf der rechten Seite interpretieren k\"{o}nnen
\begin{align}
EI\,\frac{d^4}{dy^4}\,G(y,x) = \delta (y-x)\,.
\end{align}
(Weil wir den Aufpunkt mit $x$ bezeichnen, benutzen wir $y$ als Laufvariable und deswegen stehen links Ableitungen nach $y$).

Die \"{a}quivalenten Knotenkr\"{a}fte, die die Einflussfunktion generieren, sind dann einfach die Zahlen
\begin{align}
j_i = \int_0^{\,l} \delta(y-x)\,\Np_i(y)\,dy = J(\Np_i)\,,
\end{align}
also die Werte $J(\Np_i)$ der Ansatzfunktionen. Einfacher geht es eigentlich nicht mehr.
Das bedeutet also:

 (1) Die Knotenkr\"{a}fte $j_i$, die die Einflussfunktion f\"{u}r die Durchbiegung eines Balkens in einem Punkt $x$ erzeugen, sind die Durchbiegungen der Ansatzfunktionen $\Np_i$ in diesem Punkt
\beq
j_i = \Np_i(x) \,.
\eeq
(2) Die Kr\"{a}fte $j_i$, die die Einflussfunktion f\"{u}r das Moment $M(x) $ in einem Punkt $x$ eines Balkens erzeugen,
\beq
j_i = - EI\,\Np_i''(x) = M(\Np_i)(x)\,,
\eeq
sind die Momente der Ansatzfunktionen in diesem Punkt $x$ -- etc.
%---------------------------------------------------------------------------------
\begin{figure}
\centering
{\includegraphics[width=0.9\textwidth]{\Fpath/U34}}
\caption{FE-Modell eines Seils, \textbf{ a)} Ansatzfunktionen,  \textbf{ b)} FE-Einflussfunktion f\"{u}r $w$ im Punkt $x = 1.25$ und exakter Wert (0.94),  \textbf{ c)} FE-Einflussfunktion f\"{u}r $w$ im ersten Knoten, die Funktion ist exakt, $G_h(y,x) = G(y,x)$}
\label{U34}
%
\end{figure}%%
%---------------------------------------------------------------------------------

Formulieren wir das als Regel:\\

\begin{theorem}[Knotenkr\"{a}fte f\"{u}r Einflussfunktionen]\index{Knotenkr\"{a}fte f\"{u}r Einflussfunktionen}
Die Einflussfunktion f\"{u}r ein lineares Funktional $J(u)$ wird durch die Knotenkr\"{a}fte $j_i = J(\Np_i)$ erzeugt. Die Knotenkr\"{a}fte sind also zahlenm\"{a}{\ss}ig einfach die Werte $J(\Np_i)$ der Ansatzfunktionen.
\end{theorem}

\hspace*{-12pt}\colorbox{highlightBlue}{\parbox{0.98\textwidth}{Die Knotenwerte der Einflussfunktionen nennen wir $g_i$ und die \"{a}quivalenten Knotenkr\"{a}fte $j_i$, so dass das System $\vek K\,\vek u = \vek f$ jetzt also
\begin{align}
\vek K\,\vek g = \vek j
\end{align}
hei{\ss}t. Die Bedeutung der $g_i \equiv u_i$ und $j_i \equiv f_i$ \"{a}ndert diese Umbenennung nat\"{u}rlich nicht.}}\\

%%%%%%%%%%%%%%%%%%%%%%%%%%%%%%%%%%%%%%%%%%%%%%%%%%%%%%%%%%%%%%%%%%%%%%%%%%%%%%%%%%%%%%%%%%%%%%%%%%%
{\textcolor{sectionTitleBlue}{\section{Schwache und starke Einflussfunktionen}}}
In Kapitel 1 haben wir \"{u}ber den Unterschied zwischen schwachen und starken Einflussfunktionen gesprochen und gezeigt, dass es nicht m\"{o}glich ist mit schwachen Einflussfunktionen Kraftgr\"{o}{\ss}en zu berechnen.

Es mag daher eine \"{U}berraschung sein, dass die finiten Elemente (scheinbar, s. S. \pageref{Eq155}) nicht zwischen schwachen und starken Einflussfunktionen unterscheiden. Eine FE-Einflussfunktion ist die L\"{o}sung des Variationsproblems
\begin{equation}\label{EE7Equationforz}
G_h \in \mathcal{V}_h: \qquad
a(G_h,\varphi_i ) = J(\varphi_i ) \qquad \text{f\"{u}r alle}\,\,\varphi_i  \in \mathcal{V}_h
\end{equation}
und diese Funktion $G_h$ kann in beide Formulierungen eingesetzt werden
\begin{equation}
J(u_h) = \underbrace{\int_0^{\,l} G_h(y,x)\,p(y)\,dy}_{stark} =
\underbrace{\vphantom{\int_0^{\,l} }a(G_h,u_h)}_{schwach}\,,
\end{equation}
weil auf $\mathcal{V}_h$ die beiden Formeln -- stark und schwach -- zusammenfallen
\begin{equation}
J(u_h) = \int_0^{\,l}
G_h(y,x)\,p(y)\,dy = \vek g^T\,\vek f = \vek g^T\,\vek K\,\vek
u = a(G_h,u_h)\,.
\end{equation}
In Matrizenschreibweise besteht der Unterschied nur darin, wie man die Gleichungen liest
\begin{equation}
J(u_h) = \underbrace{\vek g^T\,\vek f}_{stark} = \underbrace{\vek g^T\,\vek K\,\vek u}_{schwach}\,.
\end{equation}
In der schwachen Formulierung $\vek g^T\,\vek K\,\vek u$ summieren wir \"{u}ber alle Eintr\"{a}ge
\begin{align}
\sum_{i,j} g_i\cdot k_{ij}\cdot u_j = \sum_{i,j} g_i \cdot a(\Np_i,\Np_j) \cdot u_j\,,
\end{align}
was einem Gebietsintegral (wie \glq Mohr\grq{}) gleichkommt, w\"{a}hrend die starke Formulierung $\vek g^T\,\vek f$ im Gegensatz dazu die Knotenverschiebungen $g_i$ mit den Knotenkr\"{a}ften $f_i$ wichtet.

%%%%%%%%%%%%%%%%%%%%%%%%%%%%%%%%%%%%%%%%%%%%%%%%%%%%%%%%%%%%%%%%%%%%%%%%%%%%%%%%%%%%%%%%%%%%%%%%%%%
{\textcolor{sectionTitleBlue}{\section{Beispiele}}}
\begin{example}
Um die Einflussfunktion f\"{u}r die Durchbiegung $w(x)$ des Seils in Abb. \ref{U34} im Punkt $x = 1.25$ zu berechnen, $J(w) = w(1.25)$, werden die Durchbiegungen der $\Np_i$ im Punkt $x$ als Knotenkr\"{a}fte aufgebracht
\beq
j_1 = \Np_1(x) = 0.75 \quad j_2 = \Np_2(x) = 0.25 \quad j_3 = \Np_3(x) = 0 \quad j_4 = \Np_4(x) = 0\,.
\eeq
Das Gleichungssystem
\beq\label{Eq176}
 \left[\barr{r r r r} 2 & - 1 & 0 & 0 \\ - 1 & 2 & -1 & 0\\ 0 & -1 & 2 &-1 \\ 0 & 0 & -1 &2\earr\right]
\,\left[\barr{c} g_1 \\g_2 \\ g_3 \\ g_4 \earr \right] = \left[\barr{c} 0.75 \\ 0.25  \\
0  \\ 0 \earr \right]
\eeq
f\"{u}r die Knotenverschiebungen $g_i$ der Einflussfunktion hat die L\"{o}sung
\beq
g_1 = 0.75 \qquad g_2 = 0.75 \qquad g_3 = 0.5 \qquad g_4 = 0.25
\eeq
und so lautet die Einflussfunktion
\beq
G(y,x) = 0.75 \cdot \Np_1(y) + 0.75 \cdot \Np_2(y) + 0.5 \cdot \Np_3(y) + 0.25 \cdot \Np_4(y)\,.
\eeq
Ginge es um die Berechnung der Einflussfunktion f\"{u}r die Durchbiegung im ersten Knoten, $x = 1.0$, h\"{a}tten die Knotenkr\"{a}fte die Werte
\beq
j_1 = \Np_1(x) = 1.0 \quad j_2 = \Np_2(x) = 0 \quad j_3 = \Np_3(x) = 0 \quad j_4 = \Np_4(x) = 0\,,
\eeq
und dann w\"{a}re die FE-Einflussfunktion f\"{u}r $w(x)$ sogar exakt, weil die Zahlen
\beq
g_1 = 0.8 \qquad g_2 = 0.6 \qquad g_3 = 0.4 \qquad g_4 = 0.2
\eeq
genau die Knotenwerte der Einflussfunktion sind und sie dazwischen linear verl\"{a}uft, siehe Abb. \ref{U34} c.
\end{example}
%-----------------------------------------------------------------
\begin{figure}[tbp]
\centering
\if \bild 2 \sidecaption \fi
\includegraphics[width=0.9\textwidth]{\Fpath/U37}
\caption{FE-Einflussfunktion f\"{u}r \textbf{ a)} das Biegemoment $M$ und \textbf{ b)} die Querkraft $V$ im Punkt $0.5\,\ell_e$. Die Knotenkr\"{a}fte sind die $j_i = M(\Np_i)$ bzw. $j_i = V(\Np_i)$, s. (\ref{Eq219X})}
\label{U37}
\end{figure}%
%-----------------------------------------------------------------

\begin{remark}
Die $g_i$ \"{a}ndern sich mit der Lage $x$ des Aufpunktes, sie sind also  Funktionen von $x$, so dass eine FE-Einflussfunktion im allgemeinen eine separierte Gestalt hat
\beq
G(y,x) = g_1(x) \cdot \Np_2(y) + g_2(x) \cdot \Np_2(y) + g_3(x) \cdot \Np_3(y) + g_4(x) \cdot \Np_4(y)\,.
\eeq
\end{remark}

\begin{example}
Die Einflussfunktion f\"{u}r das Biegemoment $M = - EI\,w''$ des Durchlauftr\"{a}gers in Abb. \ref{U37} im Punkt $x = 4.5$  wird von den \"{a}quivalenten Knotenkr\"{a}ften, siehe Abb.  \ref{U22},
\beq
j_i = - EI\,\Np_i''(x) \cdot 1 \,\,\,\text{[kNm]} \qquad 1 = \text{Knick}
\eeq
erzeugt und die Einflussfunktion f\"{u}r die Querkraft $V(x) = - EI\,w'''(x)$ von den Kr\"{a}ften
\beq
j_i = - EI\,\Np_i'''(x) \cdot 1\,\,\,\text{[kNm]} \qquad 1 = \text{Versatz}\,,
\eeq
wobei die $\Np_i$ die Ansatzfunktionen ({\em shape functions\/}\index{shape functions}) sind, die auf einem einzelnen Element mit der L\"{a}nge $\ell$ die Gestalt
%-----------------------------------------------------------------
\begin{figure}[tbp]
\centering
\if \bild 2 \sidecaption \fi
\includegraphics[width=1.0\textwidth]{\Fpath/U22}
\caption{Balkenelement, \textbf{ a)} die vier Ansatzfunktionen $\Np_i$ und
\textbf{ b)} die dazu geh\"{o}rigen Biegemomente $M$ und \textbf{ c)}
Querkr\"{a}fte $V$, s. (\ref{Eq219X})}
\label{U22}
\end{figure}%
%-----------------------------------------------------------------



\begin{subequations}\label{Eq219}
\begin{alignat}{3}
\Np_1(x) &= 1 - \frac{3x^2}{\ell^2} + \frac{2x^3}{\ell^3} &\qquad \Np_2(x) &= - x + \frac{2x^2}{\ell} - \frac{x^3}{\ell^2} \\
\Np_3(x) &= \frac{3x^2}{\ell^2} - \frac{2x^3}{\ell^3} &\qquad \Np_4(x) &= \frac{x^2}{\ell} - \frac{x^3}{\ell^2}  \label{Einheitsverformungen}
\end{alignat}
\end{subequations}
haben. Ihre Schnittgr\"{o}{\ss}en $M_i = - EI\,\Np_i''$ und $V_i = - EI\,\Np_i'''$ sind
\begin{subequations}\label{Eq219X}
\allowdisplaybreaks{}
\begin{alignat}{3}
M_1(x) &=   (\frac{6}{\ell^2} - \frac{12x}{\ell^3}) \cdot EI  &\qquad  M_2(x) &=  (\frac{6 x}{\ell^2}-\frac{4}{\ell}) \cdot EI \\
M_3(x) &= (\frac{12x}{\ell^3}-\frac{6}{\ell^2})\cdot EI &\qquad M_4(x) &= (\frac{6 x}{\ell^2}-\frac{2}{\ell})\cdot EI \\
V_1(x) &=   - \frac{12}{\ell^3} \cdot EI &\qquad V_2(x) &=  \frac{6}{\ell^2} \cdot EI \\
V_3(x) &=  \frac{12}{\ell^3}\cdot EI &\qquad  V_4(x) &= \frac{6 }{\ell^2}\cdot EI\,.
\end{alignat}
\end{subequations}

Nur die Knoten des Elements, das den Aufpunkt $x$ enth\"{a}lt, tragen Knotenkr\"{a}fte $j_i$, weil die Ansatzfunktionen $\Np_i$ aller anderen Elemente, die weiter weg liegen, null Momente bzw. null Querkr\"{a}fte im Aufpunkt $x$ haben.
%----------------------------------------------------------------------------------------------------------
\begin{figure}[tbp]
\centering
\if \bild 2 \sidecaption[t] \fi
\includegraphics[width=1.0\textwidth]{\Fpath/U6}
\caption{CST-Element,  \textbf{ a)} Geometrie und Freiheitsgrade,  \textbf{ b)} diese Knotenkr\"{a}fte erzeugen die Einflussfunktion f\"{u}r die Spannung $\sigma_{xx}$. Sie ist konstant,  wie es bei einem CST-Element ({\em constant strain element\/}) \index{CST-Element} ja auch sein muss}
\label{U6}
\end{figure}%%
%----------------------------------------------------------------------------------------------------------

Die so erzeugten Einflussfunktionen sind au{\ss}erhalb des Elementes, auf dem der Aufpunkt liegt, exakt. Nur im Element m\"{u}ssen sie korrigiert werden. In der Praxis macht man das so, wie oben schon erl\"{a}utert, dass man zu der FE-Einflussfunktion die lokale L\"{o}sung addiert.

Mit den obigen Formeln kann man also Einflussfunktionen f\"{u}r die Momente bzw. Querkr\"{a}fte in einem beliebigen Punkt $0 \leq x \leq l$ eines Elements berechnen. Die Knotenkr\"{a}fte in den beiden Knoten des Elements sind die Momente $M_i(x)$ (EF f\"{u}r $M$) bzw. die Querkr\"{a}fte $V_i(x)$ (EF f\"{u}r $V$) der $\Np_i$ im Aufpunkt $x$. Alle anderen Knoten des Tragwerks sind lastfrei, $f_i = 0$.
\end{example}


\begin{example}
In einem CST-Element ({\em constant strain triangle\/}) wie in Abb. \ref{U6} sind die Spannungen konstant
\begin{align}
\left[\barr{r} \sigma_{xx} \\ \sigma_{yy} \\ \sigma_{xy}\earr \right]  = \frac{E}{2\,A} \left[ \barr {r @{\hspace{4mm}} r @{\hspace{4mm}} r
@{\hspace{4mm}} r  @{\hspace{4mm}} r @{\hspace{4mm}} r}
      y_{23} & 0 &y_{31} &0 & y_{12} & 0  \\
     0 & x_{23} & 0 &x_{13} &0 & x_{21}  \\
        x_{32} & y_{23} &x_{13} &y_{31} & x_{21} & y_{12}
    \earr \right]\,\left[\barr{r} u_1 \\ v_1  \\ u_2 \\ v_2 \\ u_3 \\ v_3\earr \right]\,.
\end{align}
Es bedeutet $x_{ij} = x_i - x_j$ und $y_{ij} = y_i - y_j$.
%----------------------------------------------------------------------------------------------------------
\begin{figure}[tbp]
\centering
\if \bild 2 \sidecaption[t] \fi
\includegraphics[width=0.6\textwidth]{\Fpath/U221}
\caption{Bilineares Element}
\label{U221}
\end{figure}%%
%----------------------------------------------------------------------------------------------------------

Um die Einflussfunktion f\"{u}r $\sigma_{xx}$ zu erzeugen, l\"{a}sst man in den Knoten des Elements die Spannungen $\sigma_{xx}$ aus den Einheitsverformungen $\vek \Np_i(\vek x)$ der Knoten wirken, $j_i = \sigma_{xx}(\vek \Np_i)(\vek x)$, also
\begin{align}
j_1 = \frac{E}{2\,A}\,y_{23} \cdot 1 \quad j_3 =  \frac{E}{2\,A}\,y_{31} \cdot 1 \quad j_5 =  \frac{E}{2\,A}\,y_{12} \cdot 1\quad j_2 = j_4 = j_6 = 0 \,.
\end{align}
Man beachte, dass die Summe der $j_i$ null ist
\begin{align}
j_1 + j_3 + j_5 =  \frac{E}{2\,A}( y_2 - y_3 + y_3 - y_1 + y_1 - y_2) = 0\,.
\end{align}
Das ist bei allen Einflussfunktionen f\"{u}r Kraftgr\"{o}{\ss}en so, weil man im Grunde das Element mit dem Aufpunkt durch gegengleiche Kr\"{a}fte $j_i$ spreizt.
\end{example}
\begin{example}
Bei den n\"{a}chsten beiden Beispielen rechnen wir mit bilinearen Scheibenelementen. Ein bilineares Element hat vier Knoten und $2 \cdot 4$ Freiheitsgrade, s. Abb. \ref{U38G}. Zu jedem Freiheitsgrad  (FG)\index{FG} geh\"{o}rt ein Verschiebungsfeld $\vek \Np_i(\vek x)$, das den Knoten in horizontaler oder vertikaler Richtung auslenkt
%----------------------------------------------------------------------------------------------------------
\begin{figure}[tbp]
\centering
\if \bild 2 \sidecaption \fi
\includegraphics[width=1.0\textwidth]{\Fpath/U38G}
\caption{Bilineare Elemente, \textbf{ a)} Einflussfunktion f\"{u}r $u_x(\vek x)$ und \textbf{ b)} f\"{u}r $\sigma_{yy}(\vek x)$}
\label{U38G}
\end{figure}%%
%----------------------------------------------------------------------------------------------------------
\beq
\vek \Np_1(\vek x) = \left[\barr{c} \psi_1(\vek x) \\ 0 \earr \right] \quad \vek \Np_2(\vek x) = \left[\barr{c} 0 \\ \psi_1(\vek x)  \earr \right] \quad\vek \Np_3(\vek x) = \left[\barr{c} \psi_2(\vek x) \\ 0 \earr \right] \quad \mbox{etc.}
\eeq
Die $\psi_i(\vek x)$ sind die vier Ansatzfunktionen der vier Eckpunkte, siehe Abb. \ref{U221},
\begin{alignat}{2}\label{Eq179}
\psi_1(\vek x) &= \frac{1}{4\,a\,b} \,(a - 2 x)(b - 2 y)\qquad &\psi_2(\vek x) &= \frac{1}{4\,a\,b} \,(a + 2 x)(b - 2 y) \\
\psi_3(\vek x) &= \frac{1}{4\,a\,b} \,(a + 2 x)(b + 2 y)\qquad &\psi_4(\vek x) &= \frac{1}{4\,a\,b}
\,(a - 2 x)(b + 2 y)\,.
\end{alignat}
Wenn also ein Verschiebungsfeld einen Knoten in horizontaler Richtung dr\"{u}ckt, dann sind alle vertikalen Verschiebungen null und umgekehrt. Solche Verschiebungsfelder machen es leicht, die Bewegungen der Knoten zu kontrollieren.

{\textcolor{sectionTitleBlue}{\subsubsection*{Einflussfunktion f\"{u}r $u_x$}}}\index{Einflussfunktion f\"{u}r $u_x$}

Um die Einflussfunktion f\"{u}r die horizontale Verschiebung in dem Viertelspunkt eines Elementes mit der L\"{a}nge $a = 2$ und H\"{o}he $b = 1$ zu generieren, l\"{a}sst man vier horizontale Kr\"{a}fte in den vier Ecken des Elementes wirken. Diese Kr\"{a}fte sind die Verschiebungen der vier horizontalen Verschiebungsfelder, Indices $1, 3, 5, 7$, im Aufpunkt $\vek x = (-0.5, -0.25)$ (Element-Koordinaten)
\beq
j_1 = 0.5625 \qquad j_3 = 0.1875 \qquad j_5 = 0.0625 \qquad j_7 = 0.1875 \,,
\eeq
und sie erzeugen die Verformung in Abb. \ref{U38G}. (Die vier vertikalen Verschiebungsfelder haben nat\"{u}rlich null Horizontalverschiebungen im Aufpunkt und daher sind auch die $j_i$ in vertikaler Richtung, $j_2, j_4, j_6, j_8$, alle null).

{\textcolor{sectionTitleBlue}{\subsubsection*{Einflussfunktion f\"{u}r  $\sigma_{xx}$}}}\index{Einflussfunktion f\"{u}r  $\sigma_{xx}$}

Die Einflussfunktion f\"{u}r die Spannung $\sigma_{yy}$ in dem Viertelspunkt entsteht, wenn man die Spannungen $\sigma_{yy}(\vek \Np_i)$ der $4 \times 2$ Verschiebungsfelder $\vek \Np_i$ als Knotenkr\"{a}fte $j_i$ aufbringt.
%-----------------------------------------------------------------
\begin{figure}[tbp]
\centering
\if \bild 2 \sidecaption \fi
\includegraphics[width=1.0\textwidth]{\Fpath/U75}
\caption{Einflussfunktion f\"{u}r das Integral von $\sigma_{xy}$ in einem senkrechten Schnitt, \textbf{ a)} \"{a}quivalente Knotenkr\"{a}fte, \textbf{ b)} Einflussfunktion, \cite{Ha6}}
\label{U75}
%
\end{figure}%
%-----------------------------------------------------------------

In einem bilinearen Element mit der L\"{a}nge $a$ und H\"{o}he $b$, wie in Abb. \ref{U221}, haben die Spannungen den Verlauf
\begin{align}\label{SigBilinear1}
\sigma_{xx}(x,y)&=\frac{E}{a\,
     b\,( -1 + \nu^2) }\cdot  \bigg[
     b\,( {u_1} - {u_3}
          )  + a\,\nu\,
        ( {u_2} - {u_8} )\,+\nn\\&\hphantom{=}
        + x\, \nu\,(-u_2+u_4 -u_6 +u_8) +y\,(-u_1 + u_3 -u_5+ u_7)\bigg]\\
\sigma_{yy}(x,y)&=\frac{E}{a\,b\,
     ( -1 + \nu^2)}\cdot  \bigg[
      b\,\nu\,
        ( {u_1} - {u_3} )  +
       a\,( {u_2} - {u_8} )\,+\nn \\&\hphantom{=}
        +x\,(-u_2 + u_4-u_6+ u_8) +y\,\nu\,(-u_1+u_3-u_5+u_7)  \bigg]
\end{align}
und
\begin{align}
\sigma_{xy}(x,y)&=\frac{- E}{2\,a\,b\,
     ( 1 + \nu )}\cdot  \bigg[
        b\,( {u_2} - {u_4}
             )  + a\,
          ( {u_1} - {u_7} )\,+ \nn \\&\hphantom{=}
        +x\,(-u_1 +u_3-u_5+ u_7)+ y\,(-u_2+ u_4-u_6+ u_8) \bigg]\,.
\end{align}
Setzen wir $u_1 = 1$ setzen und alle anderen $u_i = 0$, so erhalten wir die Spannungen, die zu dem Verschiebungsfeld $\vek \Np_1(\vek x)$ geh\"{o}ren. So betragen am ersten Knoten die \"{a}quivalenten Knotenkr\"{a}fte $j_i$ in horizontaler Richtung ($u_1 = 1$)
\begin{equation}
j_1 = \sigma_{yy}(x,y) = \frac{E}{a\,b\,
     ( -1 + \nu^2)}\cdot  \bigg[
      b\,\nu\,  {u_1} +y\,\nu\,(-u_1)  \bigg] = -3.07 \cdot 10^6\,\mbox{kNm}
\end{equation}
und in vertikaler Richtung ($u_2 = 1$)
\begin{equation}
j_2 = \sigma_{yy}(x,y) =\frac{E}{a\,b\, ( -1 + \nu^2)}\cdot  \bigg[
             a\,( {u_2} ) +x\,(-u_2) \bigg] = -3.85 \cdot 10^7\,\mbox{kNm}\,.
\end{equation}
Die anderen $j_i$ ergeben sich nach demselben Muster. Das Ergebnis und die Knotenkr\"{a}fte sind in Abb. \ref{U38G} b dargestellt.
%-----------------------------------------------------------------
\begin{figure}[tbp]
\centering
\if \bild 2 \sidecaption \fi
\includegraphics[width=0.85\textwidth]{\Fpath/U368}
\caption{Generierung der Einflussfunktionen in einem CST-Element, \textbf{ a)} f\"{u}r $\sigma_{xx}$, \textbf{ b)} f\"{u}r $\sigma_{yy}$; weil die Spannungen \"{u}berall gleich sind, gelten die Einflussfunktionen f\"{u}r jeden Punkt }
\label{U368}
%
\end{figure}%
%-----------------------------------------------------------------

Die $j_i$ verhalten sich wie \glq inverse\grq{} Gummib\"{a}nder\index{Gummib\"{a}nder}, was ein Charakteristikum der Einflussfunktionen f\"{u}r Spannungen ist, also f\"{u}r den Differenzenquotient des Verschiebungsfeldes im Aufpunkt.\label{rubberband}

Man kann die $j_i$, die das Element spreizen, auch {\em \glq geometrische Kr\"{a}fte\grq{}\/}\index{geometrische Kr\"{a}fte} nennen, weil ihre Gr\"{o}{\ss}e vom Zuschnitt des Elements abh\"{a}ngt.
%-----------------------------------------------------------------
\begin{figure}[tbp]
\centering
\if \bild 2 \sidecaption \fi
\includegraphics[width=0.9\textwidth]{\Fpath/U88}
\caption{Lokale L\"{o}sungen = ein-elementrige Einflussfunktionen am festgehaltenen Stab bzw. Balken}
\label{U88}
\end{figure}%
%-----------------------------------------------------------------

Bei einem langgezogenen Dreieck (CST-Element) mit dem Seitenverh\"{a}ltnis $l_x:l_y = 4:1$ braucht man z.Bsp. f\"{u}r die Spreizung in vertikaler Richtung das vierfache an Kraft gegen\"{u}ber einer Spreizung in horizontaler Richtung, s. Abb. \ref{U368}. Entsprechend unausgewogen sind auch die Elementbeitr\"{a}ge in der Steifigkeitsmatrix des Elements, weil da ja noch quadriert wird
\begin{align}
\int_{\Omega_e} \varepsilon_{yy}^2 \,d\Omega : \int_{\Omega_e} \varepsilon_{xx}^2 \,d\Omega = 16:1\,.
\end{align}
Bei der Berechnung einer Steifigkeitsmatrix wird zwar nichts gespreizt, sondern die Knoten werden um $u_i = 1$ ausgelenkt, aber das ist beim CST-Element dasselbe, weil die Spannungen konstant sind.

{\textcolor{sectionTitleBlue}{\subsubsection*{Einflussfunktion f\"{u}r $N_{xy}$}}}\index{Einflussfunktion f\"{u}r $N_{xy}$}

Nun soll die Einflussfunktion f\"{u}r das Integral der Schubspannungen
\beq
N_{xy} = \int_0^{\,l} \sigma_{xy}\,dy
\eeq
in einem vertikalen Schnitt, der durch einen vorgegebenen Punkt $\vek x = (x,y)$ l\"{a}uft, berechnet werden. Die \"{a}quivalenten Knotenkr\"{a}fte sind jetzt Integrale, siehe Abb. \ref{U75},
\beq
j_i = \int_0^{\,l} \sigma_{xy}(\vek \Np_i)\,dy\,,
\eeq
also die aufintegrierten Schubspannungen der Verschiebungsfelder, die zu den vier Ecken des Elementes geh\"{o}ren. In den vier Ecken jedes Elements, durch das der Schnitt f\"{u}hrt, werden die folgenden \"{a}quivalenten Knotenkr\"{a}fte aufgebracht
\begin{align}
j_i^e &= \int_0^{\,b} \sigma_{xy}(\vek \Np_i)\,dy = \frac{- E}{2\,a\,
     ( 1 + \nu )}\cdot  \bigg[
        b\,( {u_2} - {u_4}
             )  + a\,
          ( {u_1} - {u_7} )\,+ \nn \\
                &  +x\,(-u_1 +u_3-u_5+ u_7)+ \frac{b}{2}\,(-u_2+ u_4-u_6+ u_8) \bigg]\,.
\end{align}
mit $x $ als der $x$-Koordinate des Schnittes.

Um $j_1^e$ zu berechnen, setzen wir $u_1 = 1$ und alle anderen $u_i = 0$. F\"{u}r $j_2^e$ setzen wir $u_2 = 1$ und alle anderen $u_i = 0$, etc. Der Index $e$ an $j_i^e$ soll darauf hinweisen, dass dies Elementbeitr\"{a}ge sind. Die resultierende Knotenkraft ergibt sich durch die Summation \"{u}ber alle an den Knoten angeschlossenen Elemente.
\end{example}

%%%%%%%%%%%%%%%%%%%%%%%%%%%%%%%%%%%%%%%%%%%%%%%%%%%%%%%%%%%%%%%%%%%%%%%%%%%%%%%%%%%%%%%%%%%%%%%%%%%
{\textcolor{sectionTitleBlue}{\section{Die lokale L\"{o}sung}}}\index{lokale L\"{o}sung}\label{lokaleL\"{o}sung}
Die lokale L\"{o}sung ist die Einflussfunktion am beidseitig eingespannten Element, s. Abb. \ref{U88}. Diese L\"{o}sung muss zu der FE-Einflussfunk\-ti\-on in dem Element, in dem der Aufpunkt liegt, addiert werden.
%-----------------------------------------------------------------
\begin{figure}[tbp]
\centering
\if \bild 2 \sidecaption \fi
\includegraphics[width=1.0\textwidth]{\Fpath/U91A}
\caption{Berechnung der lokalen L\"{o}sung der Einflussfunktion f\"{u}r die Querkraft $V$ }
\label{U91}
\end{figure}%
%-----------------------------------------------------------------
%-----------------------------------------------------------------
\begin{figure}[tbp]
\centering
\if \bild 2 \sidecaption \fi
\includegraphics[width=.99\textwidth]{\Fpath/U471}
\caption{Einflussfunktion f\"{u}r das Moment bzw. die Querkraft im beidseitig eingespannten Stab, links und rechts vom Aufpunkt $x$ }
\label{U471}
\end{figure}%
%-----------------------------------------------------------------
Zum Exempel berechnen wir die Einflussfunktion f\"{u}r die Querkraft $V$, s. Abb. \ref{U91}, in einem beidseitig festgehaltenen Balken. Der linke Teil der Funktion, $w_L(x)$, ist im Punkt $x = 0$ eingespannt und gen\"{u}gt den statischen Randbedingungen\footnote{Das sind die Festhaltekr\"{a}fte aus der Spreizung des Aufpunkts}
\begin{align}
M(0) = - \frac{6}{3^2} EI  \qquad V(0) = - \frac{12}{3^3}\,EI\,,\
\end{align}
so dass $ w_L(x) = a\,\Np_3(x) + b\,\Np_4(x)$ der nat\"{u}rliche Ansatz f\"{u}r diese Funktion ist und die Koeffizienten $a = -1$ und $b = 0$ sind die L\"{o}sung des Systems
\begin{align}
\left[ \barr {r @{\hspace{4mm}}r @{\hspace{4mm}}r
@{\hspace{4mm}}r @{\hspace{4mm}}r}
      M_3(0) & M_4(0)  \\
      V_3(0) & V_4(0) \\
     \earr \right]\left [\barr{c}  a \\ b\earr \right ]
=  EI\,\left [\barr{c}  -6/3^2 \\  -12/3^3\earr \right ]
\end{align}
wobei $M_i$ und $V_i$ die Biegemomente bzw. Querkr\"{a}fte der {\em shape functions\/} $\Np_i(x)$ sind.

Rechts vom Aufpunkt w\"{a}hlen wir den Ansatz $w_R(x) = c\,\Np_1(x) + d\,\Np_2(x)$ und bestimmen $c = 1$ und $d = 0$ so, dass die statischen Randbedingungen
\begin{align}
M(l) = - \frac{6}{3^2} EI  \quad V(l) = - \frac{12}{3^3}\,EI\,,
\end{align}
erf\"{u}llt sind. Die Einflussfunktion hat somit die Gestalt
\begin{align}
w(x) = \left \{ \barr{r r} -\Np_3(x)  \qquad &0 < x < 1.5\\ \Np_1(x) \qquad &1.5 < x < 3.0 \earr \right.
\end{align}
Die lokale L\"{o}sung  basiert immer auf den $2 \times 2$ Einheitsverformungen des Elements; links auf den Funktionen $\Np_3(x)$ und $\Np_4(x)$ und rechts auf den Funktionen $\Np_1(x)$ und $\Np_2(x)$. Nat\"{u}rlich muss der Aufpunkt $x$ nicht in der Mitte des Elements liegen, s. Bild \ref{U471}.

%%%%%%%%%%%%%%%%%%%%%%%%%%%%%%%%%%%%%%%%%%%%%%%%%%%%%%%%%%%%%%%%%%%%%%%%%%%%%%%%%%%%%%%%%%%%%%%%%%%
{\textcolor{sectionTitleBlue}{\section{Die zentrale Gleichung}}}
FE-Einflussfunktionen liegen in separierter Form vor
\begin{align}
G_h(y,x) = g_1(x)\,\Np_1(y) + g_2(x)\,\Np_2(y) + \ldots + g_n(x)\,\Np_n(y)\,.
\end{align}
Das ist der Grund, warum bei der \"{U}berlagerung mit der Belastung $p(y)$ die \"{a}quivalenten Knotenkr\"{a}fte erscheinen
\begin{align}
w_h(x) = \int_0^{\,l} G_h(y,x)\,p(y)\,dy = \sum_{i = 1}^n\,g_i(x) \int_0^{\,l} p(y)\,\Np_i(y)\,dy = \sum_{i = 1}^n\,g_i(x)\,f_i\,,
\end{align}
und somit die Auswertung einer  Summation \"{u}ber die Knoten gleichkommt. F\"{u}r die Belastung werden die \"{a}quivalenten Knotenkr\"{a}fte gesetzt und der Einfluss einer Knotenkraft $f_i$ auf das Funktional $J(w_h) = w_h(x)$ ist gleich der Arbeit, die $f_i$ auf der zur Einflussfunktion geh\"{o}rigen Knotenverschiebung $g_i $ leistet.

Nun kann man $w_h(x)$ aber auch berechnen, indem man das Dirac Delta mit der Biegelinie $w_h$ \"{u}berlagert
\begin{align}
w_h(x) = \int_0^{\,l} \delta(y-x)\,w_h(y)\,dy\,.
\end{align}
Verkn\"{u}pfen wir diese beiden Darstellungen, dann haben wir die zentrale Gleichung zu dem Thema Einflussfunktionen und finite Elemente vor uns. \\


\begin{theorem}[Die zentrale Gleichung]\index{zentrale Gleichung}
\begin{align} \label{EE1LongChain}
w_h(x) &= \int_0^{\,l} G_h(y,x)\,p(y)\,dy = \int_0^{\,l} \sum_i\,g_i(x)\,\Np_i(y)\,p(y)\,dy = \sum_i\,g_i(x)\,f_i \nn\\
&= \vek g^T\,\vek f = \vek g^T\,\vek K\,\vek w = \vek g^T\,\vek K^T\,\vek w = \vek
j^T\,\vek w =  \sum_i\,j_i\,w_i \nn \\
&= \sum_i\,\Np_i(x)\,w_i =   \int_0^{\,l} \sum_i\, w_i\,\Np_i(y)\,\delta(y-x)\,dy
=\int_0^{\,l} w_h(y)\,\delta(y-x)\,dy\,. \nn \\
\end{align}
\end{theorem}

Die Durchbiegung $w_h(x)$ in einem Punkt $x$ ist, so lesen wir, das Skalarprodukt zwischen dem Vektor $\vek g$ der Knotenwerte der Einflussfunktion und dem Vektor der \"{a}quivalenten Knotenkr\"{a}fte $\vek f$ oder, umgekehrt, zwischen den Knotenwerten $w_i$ der FE-L\"{o}sung und den \"{a}quivalenten Knotenkr\"{a}ften $j_i = \Np_i(x)$ der Einflussfunktion
\beq\label{Eq84A}
w_h(x) = \left \{ \begin{array}{l } {\displaystyle  \int_0^{\,l} w_h(y)\,\delta(y-x)\,dy = \sum_i\,\Np_i(x)\,w_i = \vek j^T\,\vek w }          \\
{\displaystyle \int_0^{\,l} G_h(y,x)\,p(y)\,dy = \vek
g^T\,\vek f}\,.
\end{array} \right.
\eeq
Diese Darstellung gilt f\"{u}r alle linearen Funktionale
\beq
J(u) = \left \{ \begin{array}{l } {\displaystyle  \vek j^T\,\vek u }          \\
{\displaystyle \vek
g^T\,\vek f}\,,
\end{array} \right.
\eeq
und sie ist die denkbar knappste Darstellung eines linearen Funktionals.

Die erste Formel
\begin{align}
J(\vek u) = \vek j^T\,\vek u &= j_1\,u_1 + j_2\,u_2 + \ldots + j_n\,u_n \nn \\
&= J(\Np_1)\,u_1 + J(\Np_2)\,u_2 + \ldots + J(\Np_n)\,u_n
\end{align}
spielt die Berechnung von $J(\vek u)$ auf die Einzelwerte $j_i = J(\Np_i)$ zur\"{u}ck. Setzt man die $j_i$ als Kr\"{a}fte in die Knoten, wie in Abb. \ref{U37}, dann ist $J(\vek u)$ die Arbeit, die die $j_i$ auf den Wegen $u_i$ leisten.

In der zweiten Formel, $J(\vek u) = \vek g^T\,\vek f$, werden die \"{a}quivalenten Knotenkr\"{a}fte mit den Einflusskoeffizienten $g_i $ gewichtet, also den Knotenverschiebungen der FE-Einflussfunktion.

%%%%%%%%%%%%%%%%%%%%%%%%%%%%%%%%%%%%%%%%%%%%%%%%%%%%%%%%%%%%%%%%%%%%%%%%%%%%%%%%%%%%%%%%%%%%%%%%%%%
{\textcolor{sectionTitleBlue}{\section{Zustandsvektoren und Messungen}}}\label{Zustandsvektoren}

In einem \"{u}bertragenen Sinne repr\"{a}sentiert der Verschiebungsvektor $\vek u$ einen Zustandsvektor des Tragwerks und die Auswertung eines Funktionals
\begin{align}
J(\vek u) = \vek g^T\,\vek f = \vek g^T\,\vek K\,\vek u
\end{align}
kann man als eine Messung an dem Tragwerk betrachten. Die Frage ist nun, wie sich die Messwerte ver\"{a}ndern, wenn sich die Steifigkeitsmatrix ver\"{a}ndert, wenn das Tragwerk aus einem System $\vek K$ in ein System $\vek K \to \vek K + \vek \Delta \vek K$ \"{u}bergeht?

Die urspr\"{u}ngliche schwache Formulierung (Variationsformulierung)
\begin{align}
\vek \delta\,\vek u^T\,\vek K\,\vek  u= \vek \delta \vek u^T \,\vek f
\end{align}
und die ge\"{a}nderte Formulierung haben dieselbe rechte Seite
\begin{align}
\vek \delta\,\vek u^T\,(\vek K + \vek \Delta \vek K) \,\vek  u_c= \vek \delta \vek u^T \,\vek f\,,
\end{align}
so dass die Differenz der beiden Gleichungen, $\vek e = \vek u_c - \vek u$, den Ausdruck
\begin{align}
\vek \delta \vek u^T\,\vek K\,\vek e = - \vek \delta \,\vek u^T\,\vek \Delta\,\vek K\,\vek u_c
\end{align}
ergibt. Setzen wir f\"{u}r $\vek  \delta \vek u$ den Vektor $\vek g$ der Einflussfunktion dann folgt
\begin{align}
J(\vek e)= -  \vek g^T\,\vek \Delta\,\vek K\,\vek u_c\,,
\end{align}
was ein lokales Resultat ist, zumindest so lokal wie die Matrix $\vek \Delta \vek K$, weil wir zur Auswertung von $J(\vek e)$ nur \"{u}ber das ge\"{a}nderte Element $\Omega_e$ integrieren m\"{u}ssen (sinngem\"{a}{\ss} ist die rechte Seite ja so etwas wie das Mohrsche Arbeitsintegral beim Balken).

Man stelle sich einen gro{\ss}en ebenen Rahmen vor, in dem ein einzelner Riegel rei{\ss}t, $EI \to EI + \Delta EI$, $\Delta EI < 0$. Der Riss bedeutet einen \"{U}bergang von der Matrix $\vek K$ zu einer neuen Matrix $\vek K + \vek \Delta \vek K$, wobei die Zusatzmatrix $\vek \Delta \vek K = \Delta EI/EI \cdot \vek K_e$ die kleine, urspr\"{u}ngliche Steifigkeitsmatrix des gerissenen Elementes, gewichtet mit dem Faktor $\Delta EI/EI$ ist -- klein verglichen mit der Gr\"{o}{\ss}e von $\vek K$.

Wenn der Rahmen $2\,n$ Freiheitsgrade $u_i$ hat, dann k\"{o}nnten wir theoretisch $2\,n$ Messungen $J_i(\vek e) = u_{ic} - u_i$ mit den $2\,n$ Knotenvektoren $\vek g_i$ an dem gerissenen Element
\begin{align}
J_i(\vek e) = u_{ic} - u_i = -\vek g_i^T\,\vek \Delta\,\vek K\,\vek u_c
\end{align}
vornehmen und so die ge\"{a}nderten Knotenverschiebungen
\begin{align}
u_{ic} = u_i + J_i(\vek e)
\end{align}
-- in Matrizenschreibweise ist das
\begin{align}\label{Eq161}
\vek u_c = \vek u + \vek K^{(-1)}\,\vek \Delta\,\vek K\,\vek u_c
\end{align}
berechnen, weil die Spalten der Inversen gerade die Knotenverschiebungsvektoren $\vek g_i$ der Einflussfunktionen f\"{u}r die Knotenverschiebungen $u_i$ sind.

In Kapitel 5 schreiben wir das als
\begin{align}\label{Eq7}
\vek K\,\vek u_c = \vek K\,\vek u + \vek \Delta\,\vek K\,\vek u_c = \vek f + \vek f^+\,.
\end{align}
Das Problem dabei ist, dass wir nat\"{u}rlich den Vektor $\vek u_c$ nicht kennen, den wir brauchen um $J_i(\vek e)$ zu berechnen. In dieser Situation k\"{o}nnten wir f\"{u}r den Vektor $\vek u_c$ n\"{a}herungsweise den urspr\"{u}nglichen Vektor $\vek u$ setzen
\begin{align}
J_i(\vek e)  \simeq \vek g_i^T\,\vek \Delta\,\vek K\,\vek u\,,
\end{align}
oder $\vek u_c$ durch Iteration bzw. eine (stark verk\"{u}rzte) direkte Berechnung bestimmen, s. Kapitel 5, S. \pageref{Eq59}. \\

\begin{remark}
Das obige Resultat (\ref{Eq7}) impliziert, dass
\begin{align}\label{Eq162}
(\vek I + \vek K^{-1}\,\vek \Delta \,\vek K)\,\vek u_c = \vek u
\end{align}
gelten muss, was bedeutet, dass ein Tragwerk im \"{U}bergang von  $\vek K$ zu \vek K + \vek \Delta \vek K den Zuwachs $\vek \Delta \vek u = \vek u_c - \vek u$ nicht frei w\"{a}hlen kann, sondern dass der neue Zustand $\vek u_c$ mit dem vorherigen Zustand $\vek u$ kompatibel sein muss.

Theoretisch  k\"{o}nnte man mit einer ganz einfachen Steifigkeitsmatrix  $\vek K_0$ starten, die man dann mit weiteren Matrizen $\vek \Delta \vek K_i$ anreichert
\begin{align}
\vek K_i = \vek K_o + \vek \Delta \vek K_1 + \vek \Delta \vek K_2 + \ldots \vek \Delta \vek K_i
\end{align}
und so w\"{u}rde man eine Kette von L\"{o}sungen $\vek u_i$ produzieren, wo jeder neue Vektor $\vek u_{i + 1}$ mit dem vorhergehenden Vektor verschr\"{a}nkt ist
\begin{align}\label{Eq163}
(\vek I + \vek K_i^{-1}\,\vek \Delta \,\vek K_{i + 1})\,\vek u_{i + 1} = \vek u_i\,.
\end{align}
\end{remark}


%%%%%%%%%%%%%%%%%%%%%%%%%%%%%%%%%%%%%%%%%%%%%%%%%%%%%%%%%%%%%%%%%%%%%%%%%%%%%%%%%%%%%%%%%%%%%%%%%%%
{\textcolor{sectionTitleBlue}{\section{Der Satz von Maxwell}}}\label{Maxwell}
Eigentlich geh\"{o}rt der {\em Satz von Maxwell\/} in das Kapitel 2, aber wir mussten auf den Begriff des Funktionals warten.

%---------------------------------------------------------------------------------
\begin{figure}
\centering
\if \bild 2 \sidecaption \fi
\includegraphics[width=1.0\textwidth]{\Fpath/U354A}
\caption{Zwei Einflussfunktionen, \textbf{ a)} eine Einzelkraft generiert die Einflussfunktion $\vek G_1$ f\"{u}r $u_x(\vek x_a)$ und \textbf{ b)} ein Versatz erzeugt die Einflussfunktion $\vek G_2$ f\"{u}r die Spannung $\sigma_{xx}$ im Punkt $\vek x_b$; die beiden Kerne sind adjungiert, $J_2(\vek G_1) = J_1(\vek G_2)$}
\label{UE354}%
\end{figure}%
%---------------------------------------------------------------------------------
Den {\em Satz von Maxwell\/} kennt der Ingenieur als die Gleichung
\begin{align} \label{Eq126}
w_1(x_2) = w_2(x_1)\,.
\end{align}
{\em Die Durchbiegung, die eine Kraft $P = 1$ am Ort $x_1$ in einem abliegenden Punkt $x_2$ erzeugt, ist genauso gro{\ss}, wie die Durchbiegung, die eine Kraft $P = 1$ am Ort $x_2$ im Punkt $x_1$ erzeugt, s. Abb. \ref{U128} S. \pageref{U128}\/}.
%---------------------------------------------------------------------------------
\begin{figure}
\centering
\if \bild 2 \sidecaption \fi
\includegraphics[width=1.0\textwidth]{\Fpath/U317}
\caption{Zwei Einflussfunktionen und ihre Gleichheit $J_2(G_1) = J_1(G_2)$ \"{u}ber Kreuz}
\label{U317}%
\end{figure}%
%---------------------------------------------------------------------------------

Wir k\"{o}nnen dies dahingehend verallgemeinern, dass wir sagen, dass die Kerne zweier Funktionale \glq \"{u}ber Kreuz\grq{} gleich sind, was bedeuten soll
\begin{align}\label{Eq127}
\boxed{J_1(G_2) = J_2(G_1)}\,.
\end{align}
In Worten: {\em Was das erste Funktional $J_1$ angewandt auf die Einflussfunktion $G_2$ liefert, ist derselbe Wert, den das zweite Funktional $J_2$ angewandt auf die Einflussfunktion $G_1$ liefert\/}.

Die beiden Biegelinien $w_1$ bzw. $w_2$ in (\ref{Eq126}), Einzelkraft $P = 1$ in $x_1$ bzw. $x_2$, sind ja gerade die Einflussfunktionen $G_1$ und $G_2$ f\"{u}r die beiden Funktionale,
\begin{align}
J_1(w) = w(x_1) = \int_0^{\,l} G_1(y,x_1)\,p\,dy \qquad J_2(w) = w(x_2)= \int_0^{\,l} G_2(y,x_2)\,p\,dy
\end{align}
und so kommen wir auf den Ausdruck (\ref{Eq127}), der im \"{u}brigen f\"{u}r alle Paare von linearen Funktionalen und ihre Kerne gilt. Der {\em Satz von Maxwell\/} ist nicht auf Durchbiegungen begrenzt.

In einem gewissen Sinne ist das Resultat $w_1(x_2) = w_2(x_1)$ dasselbe, wie die Feststellung, dass die Entfernung von einem Punkt $A$ zu einem Punkt $B$ genauso gro{\ss} ist, wie die Entfernung von $B$ nach $A$\footnote{Diese Bemerkung ist nicht so trivial, wie sie klingt.}.

Glg. (\ref{Eq127}) ist die Grundgleichung. Sie ist der Satz von Betti auf den Punkt gebracht.

Um diese Gleichung in Aktion zu sehen, betrachten wir eine quadratische  Scheibe auf der zwei Punkte $\vek x_a$ und $\vek x_b$ markiert sind, s. Abb. \ref{UE354}. Im Punkt $\vek x_a$ messen wir die horizontale Verschiebung eines Verschiebungsfeldes $\vek u$
\begin{align}
J_1(\vek u) = u_x(\vek x_a)
\end{align}
und im Punkt $\vek x_b$ messen wir die Spannung $\sigma_{xx}$
\begin{align}
J_2(\vek u) = \sigma_{xx}(\vek u) (\vek x_b)
\end{align}
dieses Feldes.

Die Einflussfunktion f\"{u}r das Funktional $J_1$ ist das Verschiebungsfeld $\vek G_1$, das von einer horizontalen Einzelkraft $P = 1$ erzeugt wird, und die Einflussfunktion $\vek G_2$ f\"{u}r $J_2$ wird von einer horizontalen Versetzung im Punkt $\vek x_b$ erzeugt.

Gem\"{a}{\ss} Maxwell (= Satz von Betti) gilt
\begin{align}\label{Eq172}
J_1(\vek G_2) = J_2(\vek G_1)
\end{align}
oder\\

{\em Die Verschiebung im Punkt $\vek x_a$ verursacht durch die Versetzung im Punkt $\vek x_b$ ist gleich der Spannung im Punkt $\vek x_b$ infolge der Einzelkraft im Punkt $\vek x_a$.\/}\\

%---------------------------------------------------------------------------------
\begin{figure}
\centering
{\includegraphics[width=0.7\textwidth]{\Fpath/U370}}
  \caption{FE-Einflussfunktion f\"{u}r die vertikale Spannung $\sigma_{yy}$ in der Elementmitte, \"{a}quivalente Knotenkr\"{a}fte und zugeh\"{o}rige Lagerkr\"{a}fte aus der Spreizung des Aufpunkts.}
  \label{U370}\label{Korrektur19}
\end{figure}
%---------------------------------------------------------------------------------
In Abb. \ref{U317} ist das erste Funktional
\begin{align}
J_1(w) = M(x_c)
\end{align}
das Moment einer Biegelinie $w$ im Punkt $x_c$ und das zweite Funktional
\begin{align}
J_2(w) = B
\end{align}
ist die Lagerkraft, die dieselbe Biegelinie $w$ im Lager $B$ hat.

Zu $G_1$ (= Einflussfunktion f\"{u}r $J_1$) geh\"{o}rt die Lagerkraft $J_2(G_1) = -8\, 112$ kN$\times 1$\,m und zur Einflussfunktion $G_2$ (= Einflussfunktion f\"{u}r $B$) geh\"{o}rt ein Moment $J_1(G_2) = -8\,112$ kNm und beide Werte sind zahlenm\"{a}{\ss}ig gleich\footnote{Das Ergebnis von Einflussfunktionen hat immer die Dimension {\em Arbeit\/}.}
\begin{align}
J_1(G_2) = J_2(G_1)\,.
\end{align}
{\em Betti extended\/}, s. Kapitel 4, garantiert \"{u}brigens, dass dies auch f\"{u}r die FE-L\"{o}sungen gilt, d.h. in der Gleichung
\begin{align}
J_1(G_2) = \int_0^{\,l} \delta_1\, G_2\,dy = \int_0^{\,l} \delta_2\,G_1\,dy  = J_2(G_1)
\end{align}
darf man $G_1$ und $G_2$ durch die FE-L\"{o}sungen ersetzen, $J_1(G_2^h) = J_2(G_1^h)$.\\

\begin{remark}
Den klassischen Maxwell, $\delta_{12} = \delta_{21}$ unter Einzelkr\"{a}ften s. Abb. \ref{U128} S. \pageref{U128}, kann man auch aus der Mohrschen Arbeitsgleichung herleiten, wenn man zur Berechnung der Durchbiegungen $\delta_{ij}$ schwache Einflussfunktionen benutzt, denn dann ist die Symmetrie im Ergebnis eine einfache Konsequenz der Symmetrie der Wechselwirkungsenergie
\begin{align}
\delta_{12} = a(w_1,w_2) = a(w_2,w_1) = \delta_{21} \,.
\end{align}
Bei \glq h\"{o}heren\grq{} Dirac Deltas, $\delta_3, \delta_4$ (Balken), muss man allerdings \"{u}ber den Satz von Betti gehen, weil es f\"{u}r $M(x)$ und $V(x)$ keine schwachen Einflussfunktionen gibt.
\end{remark}

\vspace{-0.5cm}
{\textcolor{blue}{\subsubsection*{Lagersenkung}}}
Der Satz von Maxwell\index{Lagersenkung} beantwortet auch die Frage, wie man Einflussfunktionen auswertet, wenn sich ein Lager setzt, wie z.B. das mittlere Lager der Wandscheibe in Abb. \ref{U370}. Dargestellt ist die Einflussfunktion f\"{u}r $\sigma_{yy}$ in einem Element. Die Spreizung des Aufpunktes erzeugt in dem Lager eine vertikale Lagerkraft $R_y$ von $119\,259$ kN und damit ergibt sich die Spannung zu
\begin{align}
\sigma_{yy} = -119\,259 \cdot \Delta_y\,,
\end{align}
wenn $\Delta_y$ (in $y$-Richtung positiv) die Lagerbewegung ist. Das Minus haben wir auf S. \pageref{LagerWeg} erkl\"{a}rt.

%---------------------------------------------------------------------------------
\begin{figure}
\centering
{\includegraphics[width=0.8\textwidth]{\Fpath/U191A}}
  \caption{\textbf{ a)} Seil aus $n = 5$ Elementen, \textbf{ b-e)} die Durchbiegungen sind die Spalten der inversen Steifigkeitsmatrix (alle Werte mal $l_e/(5\,H)$), \textbf{ f)} wenn $n$ w\"{a}chst, werden die Spalten von $\vek K^{-1}$ immer \"{a}hnlicher, $\det(\vek K^{-1}) \to 0$, d.h. $\vek K^{-1}$ wird singul\"{a}r}\label{Korrektur6}
  \label{U191}
\end{figure}
%---------------------------------------------------------------------------------

%%%%%%%%%%%%%%%%%%%%%%%%%%%%%%%%%%%%%%%%%%%%%%%%%%%%%%%%%%%%%%%%%%%%%%%%%%%%%%%%%%%%%%%%%%%%%%%%%%%
{\textcolor{sectionTitleBlue}{\section{Die inverse Steifigkeitsmatrix}}}\index{inverse Steifigkeitsmatrix}
Die FE-Einflussfunktion f\"{u}r die Verschiebung $u(x)$ in einem Knoten $x_k$ hat die Form
\beq
G_h(y,x_k) = \sum_i g_i(x_k)\,\Np_i(y)\,.
\eeq
Der Vektor $\vek g = \{g_{1},g_{2}, \ldots, g_{n}\}^T $ ist die L\"{o}sung des $n\times n$ Systems
\beq
\vek K\,\vek g = \vek e_k \qquad \text{(Einheitsvektor $\vek
e_k$)}\,,
\eeq
was bedeutet, dass die Spalten $\vek g_k$ der inversen Steifigkeitsmatrix $\vek K^{-1}$
\beq
\vek g = \vek K^{-1} \vek e_k = \vek g_k
\eeq
die Knotenverschiebungen sind, die zu den $n$ Einflussfunktionen $G_h(y, x_k)$ der $n$ Knoten $x_k$ geh\"{o}ren
\beq
 G_h(y,x_k) = \sum_i g_{k @i}\,\Np_i(y)  = \vek g_k^T\,\vek \Phi (y)\,,
\eeq
mit $\vek \Phi(y) = \{\Np_1(y), \Np_2(y), \ldots, \Np_n(y)\}^T. $

Das erkl\"{a}rt, warum die Inverse einer tri-diagonalen Matrix voll besetzt ist.
Schon eine einzelne Punktlast $P = 1$ zwingt das Seil zu einer Ausgleichsbewegung, die alle Knoten erfasst.  Die Inverse einer Differenzenmatrix wie $\vek K$ (man denke an ein Seil ($\ldots 0\,\,-1\,\,\,2\,\,-1\,\,\,0\,\,\ldots$) ist also eine Summenmatrix.\\

\hspace*{-12pt}\colorbox{highlightBlue}{\parbox{0.98\textwidth}{Eine Steifigkeitsmatrix $\vek K$ \glq differenziert\grq{} und ihre Inverse $\vek K^{-1}$ \glq integriert\grq{}. Die Inverse ist {\em immer\/} voll besetzt und sie ist symmetrisch (wegen Maxwell).
}}\\

%%%%%%%%%%%%%%%%%%%%%%%%%%%%%%%%%%%%%%%%%%%%%%%%%%%%%%%%%%%%%%%%%%%%%%%%%%%%%%%%%%%%%%%%%%%%%%%%%%%
{\textcolor{sectionTitleBlue}{\section{Beispiele}}}
Das Seil in Abb. \ref{U191}\,a, das mit einer Kraft $H$ vorgespannt wird, ist in f\"{u}nf lineare Elemente unterteilt. Die Steifigkeitsmatrix
$\vek K$
\beq
    \vek K = \frac{H}{l_e}
    \left[ \barr {r @{\hspace{4mm}}r @{\hspace{4mm}}r
@{\hspace{4mm}}r}
      2 & -1 & 0 & 0  \\
      -1 & 2 & -1 & 0 \\
      0 & -1 & 2 & -1 \\
      0 & 0 & -1 & 2  \\
    \earr \right]\,
\eeq
ist tri-diagonal, aber die Inverse
\beq
   \vek K^{-1} = \frac{l_e}{5\,H}
    \left[ \barr {r @{\hspace{4mm}}r @{\hspace{4mm}}r
@{\hspace{4mm}}r}
      4 & \phantom{-}3 & \phantom{-}2 & \phantom{-}1  \\
      3 & 6 & 4 & 2 \\
      2 & 4 & 6 & 3 \\
      1 & 2 & 3 & 4  \\
    \earr \right]\,
\eeq
ist dagegen voll besetzt. Die Spalte $\vek g_k$ der Inversen, siehe die Bilder \ref{U191}\,b---f, sind die Durchbiegungen der Knoten, wenn im Knoten $x_k$ eine Einzelkraft $P = 1$ steht.

Die Zeilensumme der Inversen verr\"{a}t im \"{u}brigen, wie oben gezeigt, welche Knoten sich am st\"{a}rksten verformen, denn
\begin{align}
w_h(x_i) &= \int_0^{\,l} G_h(y,x_i)\,p(y)\,dy = \sum_{j = 1}^4 g_j(x_i) \int_0^{\,l} \Np_j(y)\,p(y)\,dy = \sum_{j = 1}^4 k_{ij}^{(-1)} f_j\nn \\
&\simeq  \sum_{j = 1}^4 k_{ij}^{(-1)} \qquad (\text{setze alle $f_j = 1$)}\,.
\end{align}
Das sind hier die Knoten 2 und 3.
%---------------------------------------------------------------------------------
\begin{figure}
\centering
{\includegraphics[width=1.0\textwidth]{\Fpath/U25}}
  \caption{\textbf{ a)} Unterteilung eines Balkens in drei Elemente,
  \textbf{ b)} Biegelinie aus $f_1 = 1$ (Spalte 1 von $\vek K^{-1}$),  \textbf{ c)} aus $f_2 = 1$ (Spalte 2 von $\vek K^{-1}$),   \textbf{ d)} aus $f_3 = 1$ (Spalte 3)}
  \label{U25}
\end{figure}
%---------------------------------------------------------------------------------
%---------------------------------------------------------------------------------
\begin{figure}
\centering
{\includegraphics[width=0.9\textwidth]{\Fpath/U360}}
  \caption{Die Knotenverschiebungen dieser drei Einflussfunktionen bilden die Spalten $\#13, \,\#14\,, \#15$ der Inversen $\vek K^{-1}$}
  \label{UE360}
\end{figure}
%---------------------------------------------------------------------------------

Zu dem Balken in Abb. \ref{U25}, es sei $EI = 1$ und $l_e = 1$, geh\"{o}rt die Steifigkeitsmatrix
\begin{align}
\vek K = \left[ \barr {r @{\hspace{4mm}}r @{\hspace{4mm}}r
@{\hspace{4mm}}r @{\hspace{4mm}}r @{\hspace{4mm}}r}
4    & 6    & 2    & 0   &  0     &0\\
     6    & 24    &  0   & -12    & -6     &0\\
     2    &  0    &  8   &   6   &   2     &0\\
     0   & -12    &  6   &  24   &   0    &-6\\
     0   &  -6    &  2   &  0    &  8     &2\\
     0   &   0    &  0   &  -6    &  2   &  4
    \earr \right]
\end{align}
und die Spalten der Inversen
\begin{align}
\vek K^{-1} = \left[ \barr {r @{\hspace{4mm}}r @{\hspace{4mm}}r
@{\hspace{4mm}}r @{\hspace{4mm}}r @{\hspace{4mm}}r}
 1.00  &-0.56   & 0.17  & -0.44  & -0.33  & -0.50\\
   -0.56  & 0.44   &-0.22  &  0.39   & 0.28   & 0.44\\
    0.17   &-0.22   & 0.33  & -0.28   &-0.17  & -0.33\\
   -0.44  & 0.39  & -0.28  &  0.44   & 0.22  &  0.56\\
   -0.33   & 0.28  & -0.17  &  0.22   & 0.33  &  0.17\\
   -0.50   & 0.44  & -0.33  &  0.56  &  0.17  &  1.00
    \earr \right]
    \end{align}
sind die Knotenbewegungen in den sechs Grundlastf\"{a}llen $\vek f = \vek e_i$, wenn also jeweils ein Knoten belastet wird, $i = 1,2,\ldots 6$.

Dies gilt f\"{u}r Rahmen beliebiger Gr\"{o}{\ss}e, siehe Abb. \ref{UE360}. Die einzelnen Spalten der Inversen $\vek K^{-1}$ sind immer die Knotenwerte der Einflussfunktionen f\"{u}r die Knotenverschiebungen.

Eine Einzelkraft $f_i = 1$ in Richtung eines Freiheitsgrades $u_i$ lenkt die Knoten aus. Die Energie, die dabei erzeugt wird, steht auf der Diagonalen $\{g_{ii}\}$ von $\vek K^{-1}$, denn
\begin{align}
a(\vek g_i, \vek g_i) = \vek g_i^T\,\vek K\,\vek g_i = \vek g_i^T\,\vek K\,\vek K^{-1}\,\vek e_i = g_{ii}
\end{align}
Je gr\"{o}{\ss}er $g_{ii}$ ist, desto weiter ist der Ausschlag, desto weicher ist das Tragwerk in Richtung von $u_i$, denn $2 \cdot A_a = 1 \cdot u_i = 2 \cdot A_i = g_{ii}$. \"{U}berraschend ist das nicht, denn $g_{ii} = f_{ii}$ ist ja das Diagonalelement der Flexibilit\"{a}tsmatrix $\vek F = \vek K^{-1}$.

%%%%%%%%%%%%%%%%%%%%%%%%%%%%%%%%%%%%%%%%%%%%%%%%%%%%%%%%%%%%%%%%%%%%%%%%%%%%%%%%%%%%%%%%%%%%%%%%%%%
{\textcolor{sectionTitleBlue}{\section{Allgemeine Form einer FE-Einflussfunktion}}}
Das folgende Theorem fasst die Ergebnisse zusammen.\\
%----------------------------------------------------------------------------
\begin{figure}
\centering
{\includegraphics[width=0.95\textwidth]{\Fpath/U83}}
  \caption{Die Biegelinie unter der Streckenlast $p$ ist die Einh\"{u}llende der Seilecke aus den Einzelkr\"{a}ften $dP$}
  \label{U83}
\end{figure}%%
%----------------------------------------------------------------------------

\begin{theorem}[Allgemeine Form einer FE-Einflussfunktion]
Es sei $\vek K$ die Steifigkeitsmatrix des Tragwerks.\\
 (i) Die FE-Einflussfunktion f\"{u}r $u_h(x)$, der Wert der FE-L\"{o}sung in einem Punkt $x$, ist
\beq
G_h(y,x) = \vek \phi(y)^T\, \vek K^{-1}\,\vek \phi(x)
\eeq
wobei
\begin{align}
\vek \phi(x) = \{\Np_1(x), \Np_2(x), \ldots, \Np_n(x)\}^T
\end{align}
die Liste mit den Werten der Ansatzfunktionen in dem Punkt $x$ ist, und $\vek \phi(y)$ ist dieselbe Liste, nur dass $x$ durch $y$ (die Integrationsvariable) ersetzt wird. \\
(ii) Die Einflussfunktion f\"{u}r ein lineares Punktfunktional $J$ ist
\beq
G_h(y,x) = \vek \phi(y)^T\, \vek K^{-1}\,\vek j(x)
\eeq
wobei der Vektor
\beq
\vek j(x) =  \{J(\Np_1), J(\Np_2), J(\Np_3), \ldots, J(\Np_n) \}^T
\eeq
die Liste der Werte $J(\Np_i)(x)$ ist.
\end{theorem}
Die Biegelinie eines Seils ist die Einh\"{u}llende der unendlich vielen Einflussfunktionen, die jede f\"{u}r sich den Einfluss eines infinitesimalen Teils $p(y)\,dy$ der Belastung auf die Durchbiegung $w(x)$ beschreiben, s. Abb. \ref{U83},
\beq
w(x) = \int_0^{\,l} G(y,x)\,p(y)\,dy\,.
\eeq
Im Unterschied hierzu ist die FE-L\"{o}sung $w_h(x) = \vek w^T\,\vek \phi(x) $ darstellbar als eine Summe von {\em endlich vielen\/} Einflussfunktionen, die einzeln mit den \"{a}quivalenten Knotenkr\"{a}ften $f_i$ der Knoten $x_i$ gewichtet werden
\beq
w_h(x) =  f_1\,G_h(x_1, x) + f_2\,G_h(x_2, x) + \ldots + f_n\,G_h(x_n, x)\,,
\eeq
denn der Vektor $\vek w$ der Knotenwerte ist
\begin{align}
\vek w &= \vek K^{-1} \vek f =  \vek K^{-1} (f_1\,\vek e_1 + f_2\,\vek e_2 + \ldots + f_n\,\vek e_n) \nn \\
&= f_1\,\vek g_1 + f_2\,\vek g_2 + \ldots + f_n\,\vek g_n = \sum_{k = 1}^n\,f_k \cdot \left [\barr{c} \phantom{.} \\  \vek g_k \\ \phantom{.} \earr \right ]
\end{align}
und die Spalte  $\vek g_k$ von $\vek K^{-1}$ entspricht $G_h(x_k,x)$.

%---------------------------------------------------------------------------------
\begin{figure}
\centering
\if \bild 2 \sidecaption \fi
\includegraphics[width=1.0\textwidth]{\Fpath/UE355}
\caption{Einflussfunktionen eine Stabes \textbf{ a)} f\"{u}r $u(x_i)$, \textbf{ b)} f\"{u}r $u(x_{i+1})$ und Reaktion des Stabes auf Punktlasten $\mp EA/l_e$ im \textbf{ c)} Knoten $x_i$ und \textbf{ d)} im Knoten $x_{i+1}$}
\label{UE355}%
\end{figure}%
%---------------------------------------------------------------------------------

%%%%%%%%%%%%%%%%%%%%%%%%%%%%%%%%%%%%%%%%%%%%%%%%%%%%%%%%%%%%%%%%%%%%%%%%%%%%%%%%%%%%%%%%%%%%%%%%%%%
{\textcolor{sectionTitleBlue}{\section{Finite Differenzen und Einflussfunktionen}}}\index{finite Differenzen}
Im Folgenden konzentrieren wir uns der Einfachheit halber auf einen Stab, aber die Ergebnisse lassen sich nat\"{u}rlich verallgemeinern.

Das erste, was wir notieren wollen, ist, dass die Knotenkr\"{a}fte $f_i$, die die Einflussfunktionen f\"{u}r die Verschiebung im Mittelpunkt eines Elementes erzeugen, immer gleich gro{\ss} sind, $f_i = 1/2$, unabh\"{a}ngig davon, wie lang, $l_e$, das Element ist. Bei der Einflussfunktionen f\"{u}r die Normalkraft $N = EA\,u'$ ist das (in \"{U}bereinstimmung mit dem umgekehrten Gummibandeffekt, S. \pageref{rubberband}) anders
\begin{align}
f_i = -\frac{EA}{l_e} \qquad f_{i+1} = \frac{EA}{l_e}\,.
\end{align}
Die Ansatzfunktion $\Np_i(x)$ (eine H\"{u}tchenfunktion) hat eine negative Steigung $-1/l_e$ im Punkt $x$ und $\Np_{i+1}$ hat eine positive Steigung $1/l_e$ im Punkt $x$.

Angenommen eine Kraft $P$ wirkt im Punkt $y$ des Stabes. Um die Normalkraft im Punkt $x$ zu berechnen, werten wir die FE-Einflussfunktion $G_1^h(y,x)$ f\"{u}r $N_h(x)$ im Punkt $y$ aus
\begin{align}
N_h(x) &= G_1^h(y,x) \cdot P = [G_0^h(y,x_i) \cdot (-\frac{EA}{l_e}) + G_0^h(y,x_{i+1}) \cdot (+\frac{EA}{l_e})] \cdot P\nn \\
 &= \frac{EA}{l_e}\,(G_0^h(y,x_{i+1}) - G_0^h(y,x_i)) \cdot P\,.
\end{align}
Und dies ist offenbar eine Finite-Differenzen-N\"{a}herung der wahren Gleichung
\begin{align}
N(x) = G_1(y,x)\cdot P = EA\,\frac{d}{dx}\,G_0(y,x) \cdot P\,.
\end{align}
\hspace*{0pt}\colorbox{highlightBlue}{\parbox{0.98\textwidth}{Bei der Verwendung von linearen Elementen n\"{a}hern wir die Einflussfunktionen f\"{u}r die Schnittgr\"{o}{\ss}en durch finite Differenzen der Einflussfunktionen der Knotenverschiebungen $G_0^h$ an.}}\\

Die $G_0^h$ der Knotenverschiebungen $u_i$ sind sozusagen die Universalschl\"{u}ssel. Es sei daran erinnert, dass ja die Knotenvektoren der Einflussfunktionen der $u_i$ die Spalten der Inversen $\vek K^{-1}$ bilden.

%---------------------------------------------------------------------------------
\begin{figure}
\centering
\if \bild 2 \sidecaption \fi
\includegraphics[width=1.0\textwidth]{\Fpath/UE356}
\caption{Die Knotenkr\"{a}fte $j_i$, die die FE-Einflussfunktion f\"{u}r $u(\vek x)$ auf einem bilinearen Netz (Poisson Gleichung) erzeugen}
\label{UE356}%
\end{figure}%
%---------------------------------------------------------------------------------


Einflussfunktionen f\"{u}r Verschiebungen interpolieren die Knotenwerte
\begin{align}
u(x) = \frac{1}{2}\,(G_0(y,x_i) + G_0(y,x_{i+1})) \cdot P\,.
\end{align}
Dasselbe gilt f\"{u}r die Einflussfunktionen selbst, s. Abb. \ref{UE356}. Die Knotenwerte $g_i(\vek x)$ der Einflussfunktion
\begin{align}
G_h(\vek y,\vek x) = \sum_i g_i(\vek x)\,\Np_i(\vek y)
\end{align}
f\"{u}r die Durchbiegung $u_h(\vek x)$ der Membran im Aufpunkt $\vek x$ ist der Vektor
\begin{align}
\vek g(\vek x) = j_a(\vek x) \cdot \left [\barr{c} \phantom{.} \\  \vek g_a \\ \phantom{.} \earr \right ] + j_b(\vek x) \cdot \left [\barr{c} \phantom{.} \\  \vek g_b \\ \phantom{.} \earr \right ]+  j_c(\vek x) \cdot \left [\barr{c} \phantom{.} \\  \vek g_c \\ \phantom{.} \earr \right ] + j_d(\vek x) \cdot \left [\barr{c} \phantom{.} \\  \vek g_d \\ \phantom{.} \earr \right ]
\end{align}
wobei die Vektoren $\vek g_a, \vek g_b, \vek g_c, \vek g_d$ die entsprechenden Spalten der Inversen $\vek K^{-1}$ sind und die Gewichte sind die Werte der vier {\em shape functions\/} $\psi_i(\vek x)$ in (\ref{Eq179}) im Aufpunkt $\vek x = (x_1,x_2)$
\begin{align}
j_a = \psi_1(\vek x) \qquad j_b = \psi_2(\vek x)   \qquad j_c = \psi_3(\vek x) \qquad j_d = \psi_4(\vek x) \,.
\end{align}
%---------------------------------------------------------------------------------
\begin{figure}
\centering
\if \bild 2 \sidecaption \fi
\includegraphics[width=0.8\textwidth]{\Fpath/U35}
\caption{FE-Einflussfunktion f\"{u}r $\sigma_{yy}$ in zwei benachbarten Punkten}
\label{U35}%
\end{figure}%
%---------------------------------------------------------------------------------

\vspace{-1cm}
%%%%%%%%%%%%%%%%%%%%%%%%%%%%%%%%%%%%%%%%%%%%%%%%%%%%%%%%%%%%%%%%%%%%%%%%%%%%%%%%%%%%%%%%%%%%%%%%%%%
{\textcolor{sectionTitleBlue}{\section{Die Natur macht keine Spr\"{u}nge, aber die finiten Elemente}}}\label{Jumps}
Wenn man die Trennlinie zwischen zwei Elementen \"{u}berschreitet, dann springen die Spannungen. Das bedeutet aber doch, dass auch die Einflussfunktionen springen m\"{u}ssen. Wie kommt das?

Den Grund sieht man in Abb. \ref{U35}. Die \"{a}quivalenten Knotenkr\"{a}fte, die die Einflussfunktion f\"{u}r $\sigma_{yy}$ in dem oberen Punkt $\vek x_1$ generieren, sind die Spannungen $\sigma_{yy}$ der Knotenverschiebungen $\vek \Np_i$ in diesem Punkt. Weil nur die Ansatzfunktionen des Elements, in dem $\vek x_1$ liegt, Spannungen in dem Punkt $\vek x_1$ erzeugen, werden nur die vier Knoten des Elementes belastet. Wenn der Punkt in das n\"{a}chste Element wandert, $\vek x_1 \to \vek x_2$, dann verschwinden diese Knotenkr\"{a}fte und tauchen an den vier Knoten des Nachbarelementes auf. Dieser pl\"{o}tzliche Sprung in den belasteten Knoten ist der Grund, warum die Spannungen springen: {\em Die Einflussfunktionen springen\/}.
%---------------------------------------------------------------------------------
\begin{figure}
\centering
\if \bild 2 \sidecaption \fi
\includegraphics[width=.75\textwidth]{\Fpath/U458}
\caption{Einflussfunktionen an einer Platte und einem Stab \textbf{ a)} Lage der Knotenkr\"{a}fte f\"{u}r die Durchbiegung der Platte in einem Knoten und in Elementmitte -- das punktgenau gelingt in einem Knoten am besten, \textbf{ b)} Einflussfunktion f\"{u}r eine Knotenverschiebung, \textbf{ c)} f\"{u}r eine Verschiebung in Elementmitte, \textbf{ d)} f\"{u}r die Normalkraft in Elementmitte und \textbf{ e)} f\"{u}r den Mittelwert der Normalkraft in einem Knoten, Mittel aus links und rechts}\label{U458}%
\end{figure}%
%---------------------------------------------------------------------------------


Verschiebungen springen beim \"{U}berschreiten der Elementlinie nicht, weil sich die Einflussfunktionen (im unmittelbarer N\"{a}he der Linie) nicht \"{a}ndern. W\"{a}re es anders, dann w\"{a}ren die Elemente nicht konform -- dann w\"{a}ren die {\em shape functions\/} unstetig.

Technisch ist es so, dass bei einer Verschiebungs-Einflussfunktion die beiden Knotenkr\"{a}fte, die in Abb. \ref{U35} springen, gleich bleiben und die anderen $f_i = \vek \Np_i(\vek x \pm 1 \text{mm}) = 0$ sind null, wenn $\vek x$ auf der Kante liegt.

An Hand der Einflussfunktionen, s. Abb. \ref{U458}, erkennt man im \"{u}brigen, dass Verschiebungen  in den Knoten am genauesten sind (bei 1-D Problemen sind sie dort sogar exakt, wenn $EA$ oder $EI$ konstant sind) und Spannungen sind es in der Mitte der Elemente. Wenn man Spannungen -- gezwungenerma{\ss}en -- an Knoten mittelt\index{Mittelung der Spannungen}, dann ist das so, als ob man bei der Berechnung der Einflussfunktionen f\"{u}r die Spannungen die Elementgr\"{o}{\ss}e in der Umgebung des Knotens verdoppelt h\"{a}tte.
%---------------------------------------------------------------------------------
\begin{figure}
\centering
\if \bild 2 \sidecaption \fi
\includegraphics[width=.75\textwidth]{\Fpath/U288}
\caption{Auf dem Weg vom Aufpunkt zum Fu{\ss}punkt der Einzelkraft m\"{u}ssen alle Steifig\-keiten richtig modelliert werden, denn nur dann werden die Einflusskoeffizienten (die Fortleitungszahlen) richtig erfasst}
\label{U288}%
\end{figure}%
%---------------------------------------------------------------------------------

Hierhin geh\"{o}rt auch das Thema {\em Gausspunkte\/}\index{Gausspunkte}, also die Beobachtung, dass in den Integrationspunkten die Ergebnisse genauer sind, als in den \"{u}brigen Punkten. Der Grund ist, dass der Fehler einer FE-L\"{o}sung eine \"{a}hnliche Verteilung aufweist, wie die partikul\"{a}re L\"{o}sung am allseits eingespannten Element -- bei eindimensionalen Problemen stimmt das sogar genau -- und die Nullstellen der Schnittkr\"{a}fte dieser partikul\"{a}ren L\"{o}sungen genau in den Gausspunkten liegen, \cite{Ha6}.
%----------------------------------------------------------
\begin{figure}[tbp]
\centering
\if \bild 2 \sidecaption[t] \fi
\includegraphics[width=0.99\textwidth]{\Fpath/U441}
\caption{Hochbaudecke \textbf{ a)} Unterkonstruktion \textbf{ b)} Biegemomente $m_{xx}$} \label{U441}
\end{figure}%%
%----------------------------------------------------------

Auch das ist ein Ergebnis, das nur auf partieller Integration beruht. Beim Balken ist die Abweichung in den Momenten zwischen der FE-L\"{o}sung und der exakten L\"{o}sung gleich dem Momentenverlauf $M_p(x)$ am eingespannten Balken
\begin{align}
M(x) - M_h(x) = M_p(x)\,.
\end{align}
Wegen der Einspannung, $w_p'(0) = w_p'(l) = 0$,  ist das Integral des Moments $M_p$ null, s. (\ref{Eq33}),
\begin{align}
\int_0^{\,l_e} M_p(x)\,dx = 0\,.
\end{align}
Nun ist $M_p(x)$ ein Polynom zweiten Grades, wenn wir annehmen, dass $p$ konstant ist, und das muss sich mit einer Gauss-Quadratur exakt berechnen lassen\footnote{$n = 2$ Punkte k\"{o}nnen Polynome bis zur Ordnung $2\,n - 1 = 3$ exakt integrieren.} und das geht nur so, dass das Moment $M_p$ in den beiden Integrationspunkten null ist. Das ist ein \"{u}berraschendes Resultat, aber es l\"{a}sst sich leicht verifizieren. (Kubische Momente, $p$ = linear, sind antimetrisch in den Gausspunkten, aber nicht null).

Bei Scheiben und Platten gilt das unter Umst\"{a}nden nur noch n\"{a}herungsweise, aber immer noch hinreichend deutlich, was den Gausspunkten ihren guten Ruf eingebracht hat.

%%%%%%%%%%%%%%%%%%%%%%%%%%%%%%%%%%%%%%%%%%%%%%%%%%%%%%%%%%%%%%%%%%%%%%%%%%%%%%%%%%%%%%%%%%%%%%%%%%%
\textcolor{sectionTitleBlue}{\section{Ein merkw\"{u}rdiges Ergebnis}}\label{Dimensionsbetrachtung}
An diese Stelle passt vielleicht auch die folgende Beobachtung: Die Platte in Bild \ref{U441} wurde mit finiten Elementen berechnet und im Nachlauf sollen nun die Biegemomente $m_{xx}$ in den Elementen ermittelt werden. Wir nehmen einmal an, dass sie elementweise linear sind, $m_{xx} = a + b\,x + c\,y$.

In jedem Element kann man den Momentenverlauf theoretisch mit der  Momenten-Einflussfunktion ermitteln
\begin{align}
m_{xx} = a + b\,x + c\,y = \sum_e \int_{\Omega_e} G_2(\vek y,\vek x) p_e(\vek y)\,d\Omega_{\vek y} + \ldots\,.
\end{align}
Die rechte Seite dieses Ausdrucks wird sich allerdings \"{u}ber viele, viele Seiten Papier erstrecken, weil wir \"{u}ber jedes der (gesch\"{a}tzt) 1000 Elemente $\Omega_e$ zu integrieren haben und danach auch noch \"{u}ber alle Kanten des Netzes, (Linienlasten und -momente aus $p_h$). Das merkw\"{u}rdige ist nun, dass all diese vielen Beitr\"{a}ge sich in der Summe auf ein lineares Polynom reduzieren, das durch drei Zahlen $a, b$  und $c$ charakterisiert ist.


%%%%%%%%%%%%%%%%%%%%%%%%%%%%%%%%%%%%%%%%%%%%%%%%%%%%%%%%%%%%%%%%%%%%%%%%%%%%%%%%%%%%%%%%%%%%%%%%%%%
{\textcolor{sectionTitleBlue}{\section{Der Weg vom Aufpunkt zur Belastung}}}\index{vom Aufpunkt zur Belastung}
F\"{u}r eine korrekte Kommunikation zwischen dem Aufpunkt und der Belas\-tung ist es wichtig, dass die Steifigkeiten auf dem Weg vom Aufpunkt zur Belastung richtig erfasst werden, weil davon sehr viel abh\"{a}ngt, s. Abb. \ref{U288}.
%---------------------------------------------------------------------------------
\begin{figure}
\centering
\if \bild 2 \sidecaption \fi
\includegraphics[width=.95\textwidth]{\Fpath/U174A}
\caption{Durchlauftr\"{a}ger mit Kragarmlast, \textbf{ a)} Einflusslinie f\"{u}r das Einspannmoment mit $2 \cdot EI$ im mittleren Feld  und \textbf{ b)} bei konstantem $EI$}
\label{U174}%
\end{figure}%
%---------------------------------------------------------------------------------

Die Abb. \ref{U174} demonstriert dies am Beispiel der Einflussfunktion f\"{u}r das Einspannmoment eines Durchlauftr\"{a}gers mit Kragarm. Eine Verdopplung von $EI$ im mittleren Feld f\"{u}hrt zu einer sp\"{u}rbaren Senkung des Einspannmoments im Verh\"{a}ltnis von 2:3 , wie man an den unterschiedlichen Durchbiegungen des Kragarms ablesen kann.
%---------------------------------------------------------------------------------
\begin{figure}
\centering
\if \bild 2 \sidecaption \fi
\includegraphics[width=.95\textwidth]{\Fpath/U176}
\caption{Eine Spreizung der St\"{u}tze erzeugt die Einflussfunktion f\"{u}r die St\"{u}tzenkraft. Die korrekte Propagierung \"{u}ber das Tragwerk h\"{a}ngt von der korrekten Modellierung der Steifigkeiten ab, \cite{Sof}}
\label{U176}%
\end{figure}%
%---------------------------------------------------------------------------------

Es ist anschaulich klar, dass diese Kommunikation um so \glq wackliger\grq{} wird, je weiter der Aufpunkt und die Last auseinander liegen, weil mit wachsender Entfernung immer mehr Bauteile zu passieren sind und sich so die Fehler aus nur n\"{a}herungsweise richtig erfassten Steifigkeiten, $EA \pm \Delta EA$, $EI \pm \Delta EI$, oder Einspanngraden $k_\Np \pm \Delta k_\Np$, kumulieren k\"{o}nnen, s. Abb. \ref{U176}. Zum Gl\"{u}ck ist es aber so, dass in der Regel mit der Entfernung die Einflusskoeffizienten abnehmen und damit auch die Auswirkungen von m\"{o}glichen Fehlern.

%%%%%%%%%%%%%%%%%%%%%%%%%%%%%%%%%%%%%%%%%%%%%%%%%%%%%%%%%%%%%%%%%%%%%%%%%%%%%%%%%%%%%%%%%%%%%%%%%%%
{\textcolor{sectionTitleBlue}{\section{Die Spalten von $\vek K$ und $\vek K^{-1}$}}

Nachdem wir jetzt auch Einflussfunktionen mit finiten Elementen berechnen k\"{o}nnen, wollen wir hier die wesentlichen Eigenschaften der Spalten der beiden Matrizen $\vek K$ und $\vek K^{-1}$ im Vergleich zusammenstellen.\index{Spalten von $\vek K$}\index{Spalten von $\vek K^{-1}$}

{\textcolor{sectionTitleBlue}{\subsubsection*{Die Spalten von $\vek K$}}}
Die Matrix $\vek K = [\vek f_1, \vek f_2, \ldots, \vek f_n]$ bildet den Vektor $\vek u$ auf einen Vektor $\vek f_h$ ab
\begin{align}
\vek K \vek u = u_1 \cdot \vek f_i + u_2 \cdot \vek f_2 + \ldots + u_n \cdot \vek f_n = \vek f_h\,.
\end{align}
Warum wir den Vektor $\vek f_h$ nennen, wird gleich deutlich.

%---------------------------------------------------------------------------------
\begin{figure}
\centering
\if \bild 2 \sidecaption \fi
\includegraphics[width=.90\textwidth]{\Fpath/U455}
\caption{Der Mittenknoten hat den FG $u_y = u_{7}$ \textbf{ a)} Spalte 7 der Steifigkeitsmatrix $\vek K$ in vektorieller Darstellung (je zwei Eintr\"{a}ge ($x$-, $y$-Komponenten)) in Spalte 7 bilden eine Kraft), die Kraft $k_{7, 7}$ in der Mitte lenkt den Knoten um $u_{7} = 1$ aus und die umliegenden Kr\"{a}fte $k_{i,7}$ stoppen die Bewegung an den n\"{a}chsten Knoten ab \textbf{ b)} Spalte 7 der Flexibilit\"{a}tsmatrix $\vek F = \vek K^{-1}$, die Eintr\"{a}ge in der Spalte $\vek f_{7}$ sind die Verschiebungen $u_i$ der Knoten verursacht durch die vertikale Kraft $f_{7} = 1$. Au{\ss}erhalb der Umgebung des belasteten Knotens in Bild a herrscht \glq Windstille\grq{}, die Matrix $\vek K$ ist schwach besetzt, w\"{a}hrend in Bild b die ganze Scheibe \glq durchweht wird\grq{}, die ganze Scheibe sp\"{u}rt die Punktlast, $\vek F$ ist voll besetzt }
\label{U447}%
\end{figure}%
%---------------------------------------------------------------------------------
Die Spalte $\vek f_i$ ist der Vektor der \"{a}quivalenten Knotenkr\"{a}fte, der zur Einheitsverformung $\Np_i$ geh\"{o}rt und die gewichtete ($u_i$) Summe  $\vek f_h$ sind daher die \"{a}quivalenten Knotenkr\"{a}fte, die zur FE-L\"{o}sung $\sum_i u_i\,\Np_i$ geh\"{o}ren, s. Abb. \ref{U447} und \ref{U448}.

Um die Bewegung $\Np_i$ zu erzeugen, sind Kr\"{a}fte n\"{o}tig, die wir den Lastfall $p_i$ nennen und die wir uns hier, der Einfachheit halber als eine Streckenlast $p_i$ vorstellen. Die zu dem Lastfall $p_i$ geh\"{o}rigen \"{a}quivalenten Knotenkr\"{a}fte sind also komponentenweise ($j$)
\begin{align}
f_{ij}    = \int_0^{\,l} p_i\,\Np_j\,dx \qquad (\delta A_a)\,.
\end{align}
Wegen der ersten Greenschen Identit\"{a}t, $\text{\normalfont\calligra G\,\,}(\Np_i,\Np_j) = \delta A_a - \delta A_i = 0$, sind diese \"{a}u{\ss}eren Arbeiten gleich der inneren Arbeit, also der Wechselwirkungsenergie $f_{ij} = a(\Np_i,\Np_j) = k_{ij}$, und so sind die Spalten von $\vek K$ gerade die \"{a}quivalenten Knotenkr\"{a}ften der Kr\"{a}fte $p_i$, die die $\Np_i$ erzeugen.

Die Grundgleichung $\vek K\,\vek u = \vek f$ der finiten Elemente bedeutet also $\vek f_h = \vek f$.  Das ist die \glq Wackel\"{a}quivalenz\grq{} zwischen dem Originallastfall und dem FE-Lastfall.

{\textcolor{sectionTitleBlue}{\subsubsection*{Die Spalten von $\vek K^{-1}$}}}
Die Matrix $\vek K^{-1} = [\vek g_1, \vek g_2, \ldots, \vek g_n]$ bildet den Vektor $\vek f$ der \"{a}quivalenten Knotenkr\"{a}fte auf den Vektor $\vek u$ ab
\begin{align}
\vek K^{-1} \vek f = f_1 \cdot \vek g_1 + f_2 \cdot \vek g_2 + \ldots + f_n \cdot \vek g_n = \vek u\,,
\end{align}
oder komponentenweise
\begin{align}
u_i = \vek g_i^T\,\vek f = \sum_j g_{ij} f_j  = \int_0^{\,l} \sum_j g_{ij}  \,\Np_j(y)\,p\,dy\,,
\end{align}
was belegt, dass
\begin{align}
G_h(y,x_i)= \sum_j\, g_{ij} \,\Np_i(y)
\end{align}
die Einflussfunktion f\"{u}r $u_i$ ist
\begin{align}
u_i = \int_0^{\,l} G_h(y,x_i)\,p(y)\,dy\,.
\end{align}
Die $g_{ij}$ sind die Elemente von $\vek K^{-1}$ und die Vektoren $\vek g_i$ sind die Spalten von $\vek K^{-1}$. Normalerweise w\"{u}rde man die Spalten mit $\vek u_i$ bezeichnen, weil es Verschiebungen sind, aber weil die $g_{ij}$ die Knotenwerte der Einflussfunktionen f\"{u}r die Knotenverschiebungen sind,  schreiben wir $\vek g_i$.
%---------------------------------------------------------------------------------
\begin{figure}
\centering
\if \bild 2 \sidecaption \fi
\includegraphics[width=.99\textwidth]{\Fpath/U448}
\caption{Der Drehfreiheitsgrad $u_{3}$ \textbf{ a)} die Spalte 3 der Steifigkeitsmatrix $\vek K$ enth\"{a}lt die \"{a}quivalenten Knotenkr\"{a}fte zum LF $u_{3} = 1$ und $u_i = 0$ sonst, $\vek K\,\vek e_{3} = \text{Spalte 3}$; das Moment $k_{3,3}$ ist der Ausl\"{o}ser der Bewegung, die von den $k_{i,3}$ an den Nachbarknoten gestoppt wird, \textbf{ b)} Spalte 3 = $\vek F\,\vek e_{3}$ der Flexibilit\"{a}tsmatrix $\vek F = \vek K^{-1}$, die Eintr\"{a}ge sind die Verschiebungen und Verdrehungen der Knoten verursacht durch das Moment $f_{3} = 1$ ($f_i = 0$ sonst). Hier wird klar, warum $\vek K$ schwach und $\vek F$ voll besetzt ist }
\label{U448}%
\end{figure}%
%---------------------------------------------------------------------------------
Wir notieren noch
\begin{align}
\vek e_i = \vek K^{-1}\,\vek f_i \qquad \vek g_i = \vek K^{-1}\,\vek e_i \qquad \vek K\,\vek K\,\vek g_i = \vek f_i\,.
\end{align}

Die Inverse hei{\ss}t wegen der Richtung $\vek K^{-1}\,\vek f = \vek u$ die Flexibilit\"{a}tsmatrix $\vek F = \vek K^{-1}$, aber man k\"{o}nnte sie auch {\em Greensche Matrix\/} $\vek G= \vek K^{-1}$\index{Greensche Matrix}  nennen.

Der Mittenknoten der Scheibe in Abb. \ref{U447} hat in horizontaler Richtung den FG 7, $u_x = u_{7}$\footnote{Realiter wird der FG nat\"{u}rlich eine zwei- oder dreistellige Zahl sein, aber um die Notation einfach zu halten, w\"{a}hlen wir 7 als FG und in Abb. \ref{U448} die Zahl 3}. In Abb. \ref{U447} a sieht man schematisch die Spalte $\vek f_{7}$ der Matrix $\vek K$ und in Bild b die Spalte $\vek g_{7}$ der Matrix $\vek F = \vek K^{-1}$. Die Spalte $\vek f_{7}$ enth\"{a}lt komponentenweise die Kr\"{a}fte in $x$- und $y$-Richtung, die die Auslenkung des FG $u_{7} = 1$ bewirken und zwar so, dass die Verschiebung in den Nachbarknoten zum Stillstand kommt.

Die Spalte $\vek g_{7}$ der Flexibilit\"{a}tsmatrix $\vek F = \vek K^{-1}$ enth\"{a}lt die $x$- und $y$-Verschiebungen, die durch eine horizontale Einzelkraft $f_{7} = 1$ in Richtung von $u_{7}$ verursacht wird. Was hier dicht beieinander liegt, kann nat\"{u}rlich in den beiden Spalten durch L\"{u}cken voneinander getrennt sein, weil die Nummerierung dar\"{u}ber entscheidet, was wo steht.

Die Einheitsverformung $\Np_3$ in Abb. \ref{U448} a ist die Einflusslinie f\"{u}r die \"{a}quivalente Knotenkraft $f_3$ (das Knotenmoment), also das Moment, mit dem die Belastung $p(x)$ den festgehaltenen Knoten zu verdrehen versucht
\begin{align}
f_3 = \int_0^{\,l} \Np_3(x)\,p(x)\,dx\,.
\end{align}
Die Verformungen in Abb. \ref{U448} b sind nat\"{u}rlich die Einflussfunktion f\"{u}r die Verdrehung des Knotens.

%%%%%%%%%%%%%%%%%%%%%%%%%%%%%%%%%%%%%%%%%%%%%%%%%%%%%%%%%%%%%%%%%%%%%%%%%%%%%%%%%%%%%%%%%%%%%%%%%%%
{\textcolor{sectionTitleBlue}{\section{Die inverse Steifigkeitsmatrix als Analysetool}}}
Es geh\"{o}rt zu den ehernen Regeln in der {\em computational mechanics\/}, dass man die Inverse $\vek K^{-1}$ nicht berechnet und nicht abspeichert, aber man verpasst so doch einfache M\"{o}glichkeiten ein Tragwerk zu analysieren.

%%%%%%%%%%%%%%%%%%%%%%%%%%%%%%%%%%%%%%%%%%%%%%%%%%%%%%%%%%%%%%%%%%%%%%%%%%%%%%%%%%%%%%%%%%%%%%%%%%%
{\textcolor{sectionTitleBlue}{\subsection{Maximale Verformungen}}}\index{maximale Verformungen}\label{Korrektur12}
Weil die Elemente $g_{ij}$ der Inversen die Knotenwerte der Einflussfunktionen sind, s. Abb. \ref{U275},
\begin{align}
u_i = \int_0^{\,l} G_h(y,x_i)\,p(y)\,dy = \sum_{j = 1}^n \int_0^{\,l} g_{ij}\,\Np_j(y)\,p(y)\,dy = \sum_{j = 1}^n\,g_{ij}\,f_j
 = \vek g_i^T\,\vek f\,.
\end{align}
%---------------------------------------------------------------------------------
\begin{figure}
\centering
\if \bild 2 \sidecaption \fi
\includegraphics[width=.95\textwidth]{\Fpath/U275}
\caption{Der Eintrag $g_{ij}$ in $\vek K^{-1}$ beschreibt den gegenseitigen Einfluss }
\label{U275}%
\end{figure}%
%---------------------------------------------------------------------------------
kann man die $g_{ij}$ dazu benutzen, um \"{u}berschl\"{a}gig die Laststellungen zu finden, die die maximale Verformung in einem Knoten $x_i$ erzeugen, denn angenommen alle $f_i$ sind eins (oder zumindest gleich gro{\ss}), dann ist die maximal auftretende Verformung in dem Knoten die Summe \"{u}ber die Werte $g_{ij} > 0$ in der Spalte $i$
\begin{align}
\text{max}\,\,u_i = \sum_{j = 1}^n g_{ij} \qquad g_{ij} > 0\,,
\end{align}
also die \glq Quersumme\grq{} in Zeile $i$ (Spalte = Zeile) \"{u}ber die positiven Werte. Diese Knoten sind demnach als Lastknoten zu w\"{a}hlen.
%---------------------------------------------------------------------------------
\begin{figure}
\centering
\if \bild 2 \sidecaption \fi
\includegraphics[width=1.0\textwidth]{\Fpath/U388}
\caption{Einflussfunktion $G_0$ f\"{u}r eine Durchbiegung, den FG $u_7$. Wo $G_0 > 0$ ist, sind die Eintr\"{a}ge $g_{7j}$ in $\vek K^{-1}$ (Zeile 7, Spalte $j$) positiv und im umgekehrten Fall negativ. So kann ein Programm automatisch, indem es die Eintr\"{a}ge in Zeile 7 der Inversen $\vek K^{-1}$ Revue passieren l\"{a}sst, die g\"{u}nstigen oder ung\"{u}nstigen Laststellungen finden}
\label{U388}%
\end{figure}%
%---------------------------------------------------------------------------------

Die typische Frage, ob denn eine Last  $f_j$ im 3. Stock einen Einfluss auf die Verformung $u_i$ eines Knotens im 1. Stock hat, kann man also einfach an Hand der Gr\"{o}{\ss}e des Elementes $g_{ij}$ der Inversen beantworten, wie in Abb. \ref{U388} am Beispiel eines Rahmens gezeigt wird.

Man k\"{o}nnte auch Karten (Plots) generieren, wie weit die Kr\"{a}fte $f_j$ maximal von einem fest gew\"{a}hlten Knoten  entfernt sein d\"{u}rfen, damit noch etwas in $u_i$ sp\"{u}rbar ist. Das w\"{a}ren dann alle Knoten mit $|g_{ij}| > \varepsilon$, also gr\"{o}{\ss}er als eine gewisse Schranke $\varepsilon$.


%%%%%%%%%%%%%%%%%%%%%%%%%%%%%%%%%%%%%%%%%%%%%%%%%%%%%%%%%%%%%%%%%%%%%%%%%%%%%%%%%%%%%%%%%%%%%%%%%%%
{\textcolor{sectionTitleBlue}{\subsection{Maximale Momente}}}\index{maximale Momente}
Die Einflussfunktion f\"{u}r das Feldmoment in einem Riegel hat die Knotenwerte
 \begin{align}
 \vek g = \vek K^{-1}\,\vek j
 \end{align}
wobei die $j_i = M(\Np_i)$ die Momente der {\em shape functions\/} in der Mitte des Riegels sind. Das werden nicht mehr als vier Gr\"{o}{\ss}en $j_i \neq 0$ sein und so ist der Vektor $\vek g$ einfach zu berechnen. Das Feldmoment $M(x) = \vek g^T\,\vek f$ wird dann maximal/minimal, wenn man alle $f_i \gtreqless 0$ mitnimmt, die zu positiven/negativen Eintr\"{a}gen $g_i \gtreqless 0$ passen. So kann man also zu jedem interessierenden Punkt $x$ einen Vektor $\vek g$ berechnen und mit diesem Vektor die $f_i$ nach $g_i \cdot f_i \gtreqless 0$ sortieren, um die Extremwerte der Momente zu finden.

%%%%%%%%%%%%%%%%%%%%%%%%%%%%%%%%%%%%%%%%%%%%%%%%%%%%%%%%%%%%%%%%%%%%%%%%%%%%%%%%%%%%%%%%%%%%%%%%%%%
{\textcolor{sectionTitleBlue}{\subsection{Beliebige Funktionale}}}
Das geht mit jeder beliebigen Gr\"{o}{\ss}e, sprich jedem Funktional $J(u)$. Man muss ja einfach nur rechnen
\begin{align}
\vek g = \vek K^{-1}\,\vek j \qquad j_i = J(\Np_i)(x)
\end{align}
und kann dann das \glq Zeilenkriterium\grq{} auf den Vektor $\vek g$ der Einflussfunktion $J(u) = \vek g^T\,\vek f$ anwenden, sprich die Komponenten $g_i$ nach positiven und negativen Werten sortieren
\begin{align}
g_i \gtrless 0
\end{align}
und wei{\ss} dann, welche $f_i\gtrless 0$ einen positiven bzw. negativen Beitrag zu $J(u)$ leisten.

\pagebreak
%%%%%%%%%%%%%%%%%%%%%%%%%%%%%%%%%%%%%%%%%%%%%%%%%%%%%%%%%%%%%%%%%%%%%%%%%%%%%%%%%%%%%%%%%%%%%%%%%%%
{\textcolor{sectionTitleBlue}{\section{Lokale \"{A}nderungen und die Inverse}}}
Das Thema Inverse gibt uns auch Gelegenheit darauf hinzuweisen, dass lokale Steifigkeits\"{a}nderungen die ganze Inverse $\vek K^{-1}$ \"{a}ndern, auch wenn sich nur eine Zahl in $\vek K$ \"{a}ndert.
%----------------------------------------------------------------------------
\begin{figure}
\centering
{\includegraphics[width=.99\textwidth]{\Fpath/U273}}
  \caption{Stab aus vier Elementen}
  \label{U273}
\end{figure}%%

%----------------------------------------------------------

Der Stab in Abb. \ref{U273} hat im letzten Element die L\"{a}ngssteifigkeit $EA = a$ im Unterschied zu den ersten drei Elementen, in denen $EA = 1$ ist, und so kommt $a$ nur einmal vor
\begin{align}
\vek K = \left[ \barr{r @{\hspace{4mm}}r @{\hspace{4mm}}r} 2 &-1 &0 \\ -1 &2 &-1 \\ 0 &-1 &1 + a\earr\right]\,,
\end{align}
aber das $a$ taucht in jedem Element der Inversen $\vek G = \vek K^{-1}$ auf
\begin{align}
\vek K^{-1} = \left[
\begin{array}{c@{\hspace{4mm}}c@{\hspace{4mm}}c}
 \displaystyle{\frac{2 a+1}{3 a+1}} & \displaystyle{\frac{a+1}{3 a+1}} & \displaystyle{\frac{1}{3 a+1} }\vspace{0.3cm}\\
 \displaystyle{\frac{a+1}{3 a+1}} & \displaystyle{\frac{2 a+2}{3 a+1}} & \displaystyle{\frac{2}{3 a+1}} \vspace{0.3cm}\\
 \displaystyle{\frac{1}{3 a+1}} & \displaystyle{\frac{2}{3 a+1}}& \displaystyle{\frac{3}{3 a+1}}
\end{array}
\right]\,.
\end{align}

\hspace*{-12pt}\colorbox{highlightBlue}{\parbox{0.98\textwidth}{Eine {\em lokale\/} Steifigkeits\"{a}nderung \"{a}ndert also den Verlauf der Einflussfunktionen im {\em ganzen\/} Tragwerk.}}\\

Wir werden aber in Kapitel 5 sehen, dass es trotzdem einen Weg gibt, wie man Effekte von Steifigkeits\"{a}nderungen nur durch Betrachtung des betroffenen Elements verfolgen kann -- zumindest n\"{a}herungsweise.
\vspace{-0.5cm}
%%%%%%%%%%%%%%%%%%%%%%%%%%%%%%%%%%%%%%%%%%%%%%%%%%%%%%%%%%%%%%%%%%%%%%%%%%%%%%%%%%%%%%%%%%%%%%%%%%%
{\textcolor{sectionTitleBlue}{\section{Das Weggr\"{o}{\ss}enverfahren}}% hardwired

Das Weggr\"{o}{\ss}enverfahren basiert auf der Beobachtung, dass die Kenntnis der Weggr\"{o}{\ss}en $u_i$ an den \glq R\"{a}ndern\grq{}, in den Knoten, ausreicht, um die Schnittkr\"{a}fte in den St\"{a}ben zu berechnen, s. Abschnitt 1.18.
 Die Zahl der unbekannten $u_i$ ist der {\em Grad der kinematischen Unbestimmtheit\/}\index{Grad der kinematischen Unbestimmtheit}. Das Ziel ist es daher, herauszufinden, wie sich die $u_i$ unter Last einstellen. Dies f\"{u}hrt auf dasselbe System, $\vek K\,\vek u = \vek f$, wie bei den finiten Elementen.

{\textcolor{blue}{\subsection{Wie kommt man auf $\vek K \vek u = \vek f$ ?}}}

Wir stellen uns einen Rahmen vor, der aus mehreren St\"{a}ben besteht, zu denen je eine horizontale Verschiebung $u(x)$ und vertikale Biegelinie $w(x)$ geh\"{o}rt, also ein \glq Sammelsurium\grq{} von Funktionen. Wir werden uns aber im folgenden kurz halten und in der Notation so tun, als h\"{a}tten wir es nur mit einem Stab zu tun und alles auf eine Funktion $u(x)$ reduzieren, die stellvertretend f\"{u}r all die anderen Funktionen in dem Rahmen stehen m\"{o}ge. Wir betreiben ja hier mehr Algebra als Mathematik.
%----------------------------------------------------------------------------
\begin{figure}
\centering
{\includegraphics[width=0.80\textwidth]{\Fpath/U382}}
  \caption{Weggr\"{o}{\ss}enverfahren \textbf{ a)} System, \textbf{ b)} \"{A}quivalente Knotenkr\"{a}fte am festgehaltenen System und Momente der lokalen L\"{o}sung, \textbf{ c)} Momente $M_R$ im Lastfall Knotenlasten = $\vek f_K + \vek p$, \textbf{ d)} Momente $M = M_R + M_{loc}$; $M_R$ sind die Momente der Biegelinie $w_R$, wenn also erst die Streckenlasten in die Knoten reduziert wurden und dann die Knoten gel\"{o}st wurden}
  \label{U382}
\end{figure}%%

%----------------------------------------------------------

Zuerst werden alle Knoten festgehalten und die Belastung in die Knoten reduziert, d.h. es werden die \"{a}quivalenten Knotenkr\"{a}fte, die {\em actio\/},
\begin{align}
d_i = \int_0^{\,l} p\,\Np_i\,dx
\end{align}
berechnet, s. Abb. \ref{U382} b. Die Festhaltekr\"{a}fte sind die {\em reactio\/}, $- d_i$. Wichtig ist, dass man die exakten $\Np_i(x)$ benutzt, weil die {\em shape functions\/} ja die exakten Einflussfunktionen f\"{u}r die \"{a}quivalenten Knotenkr\"{a}fte sind.

Wir erinnern daran, dass das $\vek f$ auf der rechten Seite von $\vek K\,\vek u = \vek f$ eine Summe von zwei Vektoren
\begin{align} \label{Eq95}
\vek f = \vek f_K + \vek d
\end{align}
ist. In dem Vektor $\vek f_K$ stehen die Kr\"{a}fte, die direkt in den Knoten angreifen und der Vektor $\vek d$ enth\"{a}lt die \"{a}quivalenten Knotenkr\"{a}fte aus der verteilten Belastung. Die Belastung in den Knoten ist also die Summe (\ref{Eq95}).

L\"{a}sst man dann die Knoten los, so stellt sich in dem Rahmen eine Verformungsfigur $u_R(x)$ ($R$ = reduziert) ein, die die Gleichgewichtslage des Rahmens unter der Wirkung der Knotenlasten $f_i = f_{K @i} + d_i$ ist, s. Abb. \ref{U382} c. \\

\hspace*{-12pt}\colorbox{highlightBlue}{\parbox{0.98\textwidth}{Das Weggr\"{o}{\ss}enverfahren berechnet nur die Figur $u_R(x)$. Was fehlt, wird am Schluss als lokale L\"{o}sung stabweise zu $u_R(x)$ addiert, $u(x) = u_R(x) + u_{loc}$.}}\\

Weil $u_R(x)$ eine homogene L\"{o}sung ist, kann man sie stabweise mit den {\em shape functions\/} $\Np_i(x)$ darstellen,
\begin{align}
u_R(x) = \sum_i\,u_i\,\Np_i(x)\,,
\end{align}
denn die $\Np_i(x)$ bilden in jedem Stab ein {\em vollst\"{a}ndiges\/} System von homogenen L\"{o}sungen.

Es fehlen noch die Knotenverschiebungen $u_i$. Diese bestimmen wir mit Hilfe der ersten Greenschen Identit\"{a}t. Die Identit\"{a}t $\text{\normalfont\calligra G\,\,}(u,v) = 0$ ist f\"{u}r alle Paare $(u,v)$ von hinreichend glatten Funktionen null, und daher muss sie auch f\"{u}r jedes Paar $(u_R,\Np_j), j = 1,2,\ldots n$ null sein
\begin{align}\label{Eq74}
\text{\normalfont\calligra G\,\,}(u_R,\Np_j) = f_j - a(u_R,\Np_j) = f_j - \sum_i\,u_i\,a(\Np_i,\Np_j) = 0\,,
\end{align}
was das System
\begin{align}\label{Eq75}
\vek K\,\vek u = \vek f
\end{align}
oder hier
\begin{align}
\frac{EI}{l} \left[\barr{c @{\hspace{4mm}}c @{\hspace{4mm}}c} 4 & 2 & 0 \\ 2 & 8 & 2 \\ 0 & 2 & 4 \earr \right]\left[\barr{c c c} u_2 \\ u_4 \\ u_6 \earr \right] = \left[\barr{c c c} 0 \\ 0 \\ 10 \earr \right] + \left[\barr{c c c} -16 \\ -9 \\ 25 \earr \right] = \vek f_K + \vek d
\end{align}
ergibt. So kommt die Steifigkeitsmatrix in das Weggr\"{o}{\ss}enverfahren hinein.

Wenn man das System (\ref{Eq75}) in Einzelschritten l\"{o}st (Verfahren von {\em Gauss-Seidel\/}), dann entspricht das dem Vorgehen des {\em Drehwinkelverfahren\/} \index{Drehwinkelverfahren} und wenn man in es einem Schritt l\"{o}st, dann macht man es wie die finiten Elemente heute.

\begin{remark}
Dass sich in (\ref{Eq74}) die \"{a}u{\ss}ere Arbeit $\delta A$ auf
\begin{align}
\delta A_a = f_j = [V_R\,\Np_j - M_R\,\Np_j']_0^l = \sum_{i = 1}^4\,f_i(u_R)\,u_i(\Np_j)
\end{align}
reduziert, ist hoffentlich evident, denn $u_i(\Np_j) = \delta_{ij}$ (Kronecker Delta). Hier steht $u_i(\Np_j) $ f\"{u}r die Weggr\"{o}{\ss}en $u_1(\Np_j) = \Np_j(0), u_2(\Np_j) = \Np_j'(0), u_3(\Np_j) = \Np_j(l), u_4(\Np_j) = \Np_j'(l)$.
\end{remark}
\vspace{-0.5cm}
{\textcolor{sectionTitleBlue}{\subsection{Handberechnung von $\vek K$}}}
Bei einer Handberechnung stellt man die Steifigkeitsmatrix $\vek K$ spaltenweise von Hand auf. Die Spalte $\vek f_i$ enth\"{a}lt ja die \"{a}quivalenten Knotenkr\"{a}fte, die zur Einheitsverformung $\Np_i(x)$ geh\"{o}ren. Also lenkt man einen Knoten aus, $u_i = 1$, h\"{a}lt alle anderen Knoten fest und berechnet, welche Kr\"{a}fte $f_{ij}$ dies in Richtung der $u_j$ ergibt. Das $f_{ii}$ ist die \"{a}quivalente Knotenkraft, die den Knoten auslenkt und die anderen $f_{ij}$ sind die \"{a}quivalenten Knotenkr\"{a}fte in Richtung der \"{u}brigen $u_j$, die die Bewegung abstoppen.

So kann man die Steifigkeitsmatrix $\vek K = [\vek f_1, \vek f_2, \ldots , \vek f_n]$ spaltenweise direkt berechnen, was den kleinen Vorteil hat, dass man unter Umst\"{a}nden Freiheitsgrade sparen kann, man nicht jeden Stab als Vollstab mit $6 = 3 \times 2$ Freiheitsgraden ansetzen muss, sondern z.B. einen Pendelstab als ein $2 \times 2$ Element behandeln kann.

Die Handberechnung wird erg\"{a}nzt von der Berechnung der \"{a}quivalenten Knotenkr\"{a}fte $\vek f$ und am Schluss wird, wie zuvor, das System $\vek K\,\vek u = \vek f$ gel\"{o}st.

{\textcolor{sectionTitleBlue}{\subsection{Finite Elemente}}}

Wiederholen wir noch einmal, zum Vergleich, wie die finiten Elemente bei Stabtragwerken vorgehen: Zuerst wird die Belastung in die Knoten reduziert. Wir bezeichnen die zugeh\"{o}rige L\"{o}sung, also die Gleichgewichtslage des Rahmens unter den Knotenkr\"{a}ften mit $u_R$.

Mit den {\em shape functions\/} wird dann die FE-L\"{o}sung $u_h $ dieses Lastfalls gebildet
\begin{align}
 u_h = \sum_i u_i\,\Np_i(x)
\end{align}
und so eingestellt, dass bei jeder virtuellen Verr\"{u}ckung $\Np_i(x)$ die beiden L\"{o}sungen dieselbe virtuelle \"{a}u{\ss}ere Arbeit leisten,
\begin{align}
f_i = \delta A_a(u_R,\Np_i) = \delta A_a(u_h,\Np_i) = f_{h @i}\,,
\end{align}
was wegen
\begin{align}
\text{\normalfont\calligra G\,\,}(u_h,\Np_i) = \delta A_a - \delta A_i = f_{h @i} - \sum_j k_{ij} \,u_j = 0
\end{align}
mit dem Gleichungssystem
\begin{align}\label{Eq79}
\sum_j k_{ij} \,u_j = f_i \qquad i = 1,2\ldots n
\end{align}
identisch ist. Weil die L\"{o}sung $u_R(x)$ nach Konstruktion eine homogene L\"{o}sung ist (keine Belastung im Feld) kann sie mit den {\em shape functions\/} $\Np_i(x)$ dargestellt werden. Die $u_j$ aus (\ref{Eq79}) liefern also die exakte L\"{o}sung
\begin{align}
u_h(x) = \sum_i\,u_i\,\Np_i(x) = u_R(x)\,.
\end{align}
Was noch fehlt, sind die lokalen L\"{o}sungen, die man in einem zweiten Schritt zu $u_R(x)$ addiert und so ist die \"{U}bereinstimmung mit dem Weggr\"{o}{\ss}enverfahren perfekt.

Ein FE-Programm berechnet also automatisch die exakte L\"{o}sung $u_R(x)$, wenn die Steifigkeiten abschnittsweise konstant sind, denn dann liegt $u_R(x)$ in $\mathcal{V}_h$ und es liefert eine N\"{a}herung, wenn die Tr\"{a}ger gevoutet sind oder andere exotische Profile haben, weil dann $u_R(x)$ nicht mehr in $\mathcal{V}_h$ liegt.

{\textcolor{blue}{\subsection{Drehwinkelverfahren}}}
Fr\"{u}her hat man das Aufstellen und L\"{o}sen des Systems $\vek K\,\vek u = \vek f$ vermieden, ist man iterativ vorgegangen, wie in den Verfahren von {\em Cross\/} und {\em Kani\/}, \cite{Hirschfeld}. Exemplarisch wollen wir die einzelnen Schritte am Drehwinkelverfahren erl\"{a}utern.\\

\begin{enumerate}
  \item Erst werden alle Knoten festgehalten und die Schnittkraftverl\"{a}ufe (= lokale L\"{o}sungen) und die Festhaltekr\"{a}fte in den Knoten berechnet.
  \item Knotenweise werden dann die Knoten ausgeglichen: Man l\"{o}st einen Knoten und stellt das Gleichgewicht her, erlaubt dem Knoten also sich so zu verdrehen, dass die inneren Momente dem Lastmoment das Gleichgewicht halten k\"{o}nnen. Der Ausgleich an einem Knoten f\"{u}hrt zur Weiterleitung von Momenten an die (weiterhin festgehaltenen) Nachbarknoten.
  \item Danach wird der Knoten wieder gesperrt und der n\"{a}chste Knoten ausgeglichen. Nach drei- bis viermaligen Durchl\"{a}ufen konvergiert das Verfahren in der Regel, weil die Gr\"{o}{\ss}e der weiterzuleitenden Momente relativ schnell abnimmt.
\end{enumerate}

Man beginnt also, wie bei den finiten Elementen, mit der Reduktion der Belastung in die Knoten. Aber w\"{a}hrend die finiten Elemente das System $\vek K\,\vek u = \vek f$ heute in einem Schritt l\"{o}sen, hat der Praktiker das System im Grunde fr\"{u}her iterativ gel\"{o}st, zeilenweise, was mathematisch dem Verfahren von {\em Gauss-Seidel\/} entspricht. Das ist, aus mathematischer Sicht, im Grunde der einzige Unterschied zwischen dem Drehwinkelverfahren und den finiten Elementen. Wenn sich nat\"{u}rlich der Ingenieur so auch das Aufstellen der Steifigkeitsmatrix $\vek K$ erspart hat.

\begin{remark}
Das Weggr\"{o}{\ss}enverfahren bildet traditionsgem\"{a}{\ss} den Gegenpol zum Kraftgr\"{o}{\ss}enverfahren. Die starke \glq Magnetwirkung\grq{} der finiten Elementen hat dazu gef\"{u}hrt, dass das Verfahren heute meist in einer Matrizenschreibweise dargestellt wird.

Wir geben daher zu Bedenken, ob man das Weggr\"{o}{\ss}enverfahren nicht gleich in FE-Schreibweise formulieren sollte, denn dann spart man Zeit, kann schon sehr fr\"{u}h die finiten Elemente (noch als exakte, homogene L\"{o}sungen) einf\"{u}hren und hat den ganzen Apparat zur Hand, um aus den Elementmatrizen die Steifigkeitsmatrix zu erzeugen.

Die \glq echten\grq{} finiten Elemente kann man ja dann als Galerkin-Verfahren einf\"{u}hren, als Projektion der exakten L\"{o}sung auf den Ansatzraum $\mathcal{V}_h$, mit derselben Gleichung $\vek K\,\vek u = \vek f$ wie beim Weggr\"{o}{\ss}enverfahren und der Student hat den Vorteil, dass er in dem Neuen das Vertraute wiedererkennt.
\end{remark}
\vspace{-0.5cm}
%%%%%%%%%%%%%%%%%%%%%%%%%%%%%%%%%%%%%%%%%%%%%%%%%%%%%%%%%%%%%%%%%%%%%%%%%%%%%%%%%%%%%%%%%%%%%%%%%%%
{\textcolor{sectionTitleBlue}{\section{Mohr und die Flexibilit\"{a}tsmatrix $\vek K^{-1}$}}
Es sei $\vek K$ die Steifigkeitsmatrix eines Fachwerks. Die Steifigkeit $k_i$ des Fachwerks in Richtung des Freiheitsgrades $u_i$ ist definiert als
\begin{align}
k_i = \frac{1}{u_i}\,,
\end{align}
wobei $u_i$ die Verschiebung ist, die aus einer Einzelkraft $f_i = 1$ resultiert.

Der Verschiebungsvektor in diesem Lastfall ist $\vek u = \vek K^{-1}\,\vek e_i$ und dies ist der Grund, warum die Verschiebung
\begin{align}\label{Eq168}
u_i = \vek e_i^T\,\vek u = \vek e_i^T\,\vek K^{-1}\,\vek e_i = g_{ii}
\end{align}
mit dem Eintrag $g_{ii}$ auf der Diagonalen der Flexibilit\"{a}tsmatrix $\vek F = \vek K^{-1}$ identisch ist.

Wir k\"{o}nnten $u_i$ aber auch mit dem Mohrschen Arbeitsintegral
\begin{align}
u_i = g_{ii} = \sum_e\,\frac{N_e^2\,l_e}{EA_e}
\end{align}
berechnen, wobei $N_e$ die Normalkraft in dem Stab $e$ im Lastfall $\vek f = \vek e_i$ ist.

Bei Stockwerkrahmen w\"{u}rde dieselbe Gleichung wie folgt lauten
\begin{align}
u_i = g_{ii} =  \sum_e\,\int_0^{\,l_e} (\frac{N_e^2}{EA_e} + \frac{M_e^2}{EI_e})\,dx\,,
\end{align}
wobei $N_e$ und $M_e$ die korrespondierende Bedeutung haben.

In moderner Notation k\"{o}nnten wir das schreiben als
\begin{align}
g_{ii} = a(u^{(i)}, u^{(i)})
\end{align}
wobei $u^{(i)}$ die FE-L\"{o}sung des Lastfalls $\vek f = \vek e_i$ ist. Die Verallgemeinerung auf die Nebendiagonalen ist offensichtlich
\begin{align}
g_{ij} =  \vek e_j^T\,\vek K^{-1}\,\vek e_i = \sum_e\,\frac{N_e^{(i)}\,N_e^{(j)}\,l_e}{EA_e} = a(u^{(i)}, u^{(j)})\,.
\end{align}
Auf den ersten Blick ist es erstaunlich, dass wir dasselbe Ergebnis einmal mit $\vek K$ und einmal mit $\vek K^{-1}$ schreiben k\"{o}nnen
\beq
u_i =  \left \{ \begin{array}{l } {\displaystyle \vek g^T\,\vek K\,\vek u }   \vspace{0.3 cm}       \\
{\displaystyle \vek e_i^T\,\vek K^{-1}\,\vek e_i}\,,
\end{array} \right.
\eeq
aber etwas lineare Algebra macht schnell klar, warum das geht.

Der Vektor $\vek g$ ist die L\"{o}sung des Gleichungssystems
\begin{align}
\vek K\,\vek g = \vek j = \vek e_i
\end{align}
und daher sind in der Tat die beiden Formen gleich
\begin{align}
J(\vek u) = u_i = \vek g^T\,\vek K\,\vek u = \vek e_i^T\,\vek u = \vek e_i^T\,\vek K^{-1}\,\vek e_i\,.
\end{align}
Es ist auch hilfreich sich daran zu erinnern, dass die Spalten $\vek g_i$ der Inversen $\vek K^{-1}$ die Knotenvektoren $\vek g$ der Einflussfunktionen der Knotenverschiebungen sind und daher gilt nat\"{u}rlich $u_i = g_i = g_{ii}$.

%----------------------------------------------------------------------------
\begin{figure}
\centering
{\includegraphics[width=0.9\textwidth]{\Fpath/UE255D}}
  \caption{Einflussfunktion f\"{u}r $ \sigma_{yy}$ im mittleren Element, \textbf{ a)} die Elementsteifigkeit hat sich verdoppelt und \textbf{ b)} hat ihren normalen Wert, \textbf{ c)} Einflussfunktion f\"{u}r die Normalkraft $N(x)$ in einem Stab---die Funktion \"{a}ndert sich nicht}
  \label{U255}
\end{figure}%%
%----------------------------------------------------------

%%%%%%%%%%%%%%%%%%%%%%%%%%%%%%%%%%%%%%%%%%%%%%%%%%%%%%%%%%%%%%%%%%%%%%%%%%%%%%%%%%%%%%%%%%%%%%%%%%%
{\textcolor{sectionTitleBlue}{\section{Querschnitts\"{a}nderungen}}}\index{Querschnitts\"{a}nderungen}
Wenn die St\"{a}rke einer Scheibe abschnittsweise unterschiedlich ist, dann \"{a}ndert sich an der Berechnung der Einflussfunktionen technisch nichts. Man kann nur sehr sch\"{o}n beobachten, wie steife Zonen Kr\"{a}fte anziehen und weiche Zonen vermieden werden.

Wenn, wie in Abb. \ref{U255}, das Material in der e\'{\i}ngebetteten Zone h\"{a}rter ist als die Umgebung, dann wird die Einflussfunktion f\"{u}r die Spannung $\sigma_{yy}$ in der Mitte der Zone weit ausstrahlen, d.h. ein relativ gro{\ss}er Anteil der Belastung flie{\ss}t durch die steife Zone. Umgekehrt, wenn die Zone sehr viel weicher ist als die Umgebung, s. Abb. \ref{UE324}, dann behindert die Umgebung die Ausbreitung der Spreizung des Aufpunktes, d.h. nur ein kleiner Anteil der Belastung wird durch den weichen Kern flie{\ss}en.
%----------------------------------------------------------------------------
\begin{figure}
\centering
{\includegraphics[width=0.85\textwidth]{\Fpath/U495}}
  \caption{Hochbauplatte mit zwei abgeminderten Bereichen, 20 cm statt 40 cm im LF $g$, {\bf a)} Hauptmomente, {\bf b)} Einflussfunktion f\"{u}r $m_{xx}$ im Aufpunkt 1 und {\bf c)} im Aufpunkt 2} \label{UE324}
\end{figure}
%----------------------------------------------------------------------------
%----------------------------------------------------------------------------
\begin{figure}
\centering
{\includegraphics[width=0.9\textwidth]{\Fpath/KOPFMOMENTED}}
  \caption{Platte mit St\"{u}tzenkopfverst\"{a}rkung, Verteilung der \textbf{a)} Momente $m_{xx}$ (as in $x$-Richtung) und  \textbf{b)} $m_{yy}$ (as in $y$-Richtung), \cite{Ha5}} \label{Kopfmomente}
\end{figure}
%----------------------------------------------------------------------------
Das erscheint logisch, aber es verbleibt doch eine Frage: Wenn der E-Modul des Elementes sich verdoppelt, dann kostet es doppelt so viel M\"{u}he das Element auseinander zu rei{\ss}en, um den Verschiebungssprung in vertikaler Richtung (= Einflussfunktion f\"{u}r $\sigma_{yy}$) n\"{a}herungsweise zu generieren. Es ist klar, dass die Kr\"{a}fte $\vek f$ dann doppelt so gro{\ss} sein m\"{u}ssen wie im einfachen Fall. Aber es muss anscheinend so sein, dass diese doppelt so gro{\ss}en Kr\"{a}fte sich nicht allein in dem Spreizen des Elements verbrauchen, sondern dass ein Teil \"{u}brig bleibt, die obere Kante der Scheibe \"{u}berproportional weit nach oben zu dr\"{u}cken. Die Kr\"{a}fte $2 \cdot \vek f$ kommen da anscheinend weiter als die Kr\"{a}fte $\vek f$.


Ein Blick auf einen Zugstab, Abb. \ref{U255} c, kann helfen. Wenn man in einem Element die Steifigkeit verdoppelt, dann \"{a}ndert das nicht die Einflussfunktion $G(y,x)$ f\"{u}r $N(x)$ am Stabende $y = l$, weil die Verdopplung von $EA$ die Steigung der Einflussfunktion $G_c$ in dem Element halbiert und wegen
\begin{align}
G_c' \cdot 2 \cdot f \cdot l_e = \frac{1}{2}\, G' \cdot 2 \cdot f \cdot l_e = (G(y_b,x) - G(y_a,x)) \cdot f
\end{align}
\"{a}ndert sich nichts, ($y_a$ und $y_b$ sind die Endpunkte des Elements mit der L\"{a}nge $l_e$).

Bei der Scheibe ist das anscheinend anders. Eine Verdopplung des E-Moduls halbiert nicht die Steigung der Einflussfunktion f\"{u}r $\sigma_{yy}$ (jetzt in vertikaler Richtung), sondern der Abfall der Steigung muss geringer sein, vielleicht weil sich die Steifigkeiten in den Elementen links und rechts von dem Element $2\cdot E$ ja nicht ge\"{a}ndert haben, und so kommt es zu dem \glq \"{U}berschuss\grq{} an der oberen Kante.

Eine St\"{u}tzenkopfverst\"{a}rkung verursacht Spr\"{u}nge in den Momenten $m_{yy}$, aber -- wie beim
Balken -- nicht in den Momenten $m_{xx}$, s. Abb. \ref{Kopfmomente}.


%----------------------------------------------------------------------------
\begin{figure}
\centering
{\includegraphics[width=0.99\textwidth]{\Fpath/U438}}
  \caption{Einflussfunktionen f\"{u}r Spannungen $\sigma_{yy}$ und $\sigma_{xx}$ in zwei Scheiben. Die Pfeile sind die Knotenverschiebungen $\vek g_i$ in den Knoten $\vek x_i$ aus der Spreizung des Aufpunkts. Knotenkr\"{a}fte $\vek f_i$ aus der Belastung, die in Richtung der $\vek g_i$ weisen, haben maximalen Einfluss und Knotenkr\"{a}fte, die senkrecht auf den $\vek g_i$ stehen, keinen Einfluss}  \label{U438}\label{Korrektur21}
\end{figure}
%----------------------------------------------------------------------------
%----------------------------------------------------------------------------
\begin{figure}
\centering
{\includegraphics[width=1.0\textwidth]{\Fpath/U11}}
  \caption{ Plot der Knotenvektoren $\vek g_i$ des Funktionals $J(u_h) = \sigma_{xx}$, der horizontalen Spannungen in der Scheibe nahe der \"{O}ffnung}
  \label{U11}
\end{figure}%%
%----------------------------------------------------------------------------
%----------------------------------------------------------------------------
\begin{figure}
\centering
{\includegraphics[width=0.8\textwidth]{\Fpath/U416}}  %% Pos. KURZ
  \caption{Scheibe aus vier bilinearen Elementen, links festgehalten, Einflussfunktion f\"{u}r $\sigma_{xx}$ im Knoten oben links. Knotenkr\"{a}fte, die auf den roten Linien liegen, senkrecht zu den Verschiebungslinien der Knoten, verursachen keine Spannungen $\sigma_{xx}$ in dem Knoten}
  \label{U416}
\end{figure}%%
%----------------------------------------------------------
%----------------------------------------------------------------------------
\begin{figure}
\centering
{\includegraphics[width=0.8\textwidth]{\Fpath/U222}}
  \caption{Plot der Knotenvektoren $\vek g_i$ des Funktionals $J(u_h) = \sigma_{yy}$, also der vertikalen Spannung im Rissgrund. Die {\em Lagrangepunkte\/} sind die Punkte, in denen der Einfluss der Knotenkr\"{a}fte $\vek f_i$ auf $\sigma_{yy}$ praktisch null ist}
  \label{U222}
\end{figure}%%
%----------------------------------------------------------

%%%%%%%%%%%%%%%%%%%%%%%%%%%%%%%%%%%%%%%%%%%%%%%%%%%%%%%%%%%%%%%%%%%%%%%%%%%%%%%%%%%%%%%%%%%%%%%%%%%
{\textcolor{sectionTitleBlue}{\section{Sensitivit\"{a}tsplots}}}\index{Sensitivit\"{a}tsplots}
Das Ergebnis $J(u_h) = \vek g^T\,\vek f$ ist das Skalarprodukt aus dem Vektor $\vek g$, also den Knotenwerten der Einflussfunktion, und dem Vektor $\vek f$ der \"{a}quivalenten Knotenkr\"{a}fte aus der Belastung. Dieses Skalarprodukt kann als eine Summe \"{u}ber die $N$ Knoten des FE-Netzes geschrieben werden
\beq
J(u_h) =  \sum_{i = 1}^N \vek g_i^T\,\vek f_i \qquad i = \text{Knoten}\,,
\eeq
wobei die Vektoren $\vek g_i$ und $\vek f_i$ die Anteile aus den gro{\ss}en Vektoren $\vek g$ and $\vek f$ sind, die sich auf den Knoten $i$ beziehen
\beq
\vek g = \{\underbrace{g_1, g_2}_{\vek g_1}, \underbrace{g_3, g_4}_{\vek g_2}, \ldots, g_{2N}\}^T \qquad 2-D\,.
\eeq
Wenn daher $\vek f_i$ in einem Knoten orthogonal zu $\vek g_i$ ist, dann ist der Beitrag des Knotens zu $J(u_h)$ null. Der Plot der Vektoren $\vek g_i$ gleicht somit einem {\em Sensitivit\"{a}tsplot\/} des Funktionals $J(u_h)$, siehe die Bilder \ref{U438}, \ref{U11} und \ref{U416}. Knotenkr\"{a}fte $\vek f_i$, die in dieselbe Richtung zeigen wie die $\vek g_i$, \"{u}ben einen maximal gro{\ss}en Einfluss auf $J(u_h)$ aus\footnote{Das Verformungsbild einer Scheibe gleicht -- auch in normalen Lastf\"{a}llen -- einer eingefrorenen Str\"{o}mung. Nur sind die FE-Netze meist zu grob, um das zu sehen. Die Bilder hier wurden mit Randelementen erzeugt. }.
%-----------------------------------------------------------------
\begin{figure}[tbp]
\centering
\includegraphics[width=.99\textwidth]{\Fpath/U434}
\caption{Fachwerk mit biegesteifen Knoten \textbf{ a)} Einflussfunktion f\"{u}r die vertikale Verschiebung im Knoten 2 \textbf{ b)} Sensitivit\"{a}tsplot der Verschiebung} \label{U434}
\end{figure}%
%-----------------------------------------------------------------

In Abb. \ref{U222} ist die Einflussfunktion f\"{u}r die Spannung $\sigma_{yy}$ im Rissgrund einer Zugscheibe dargestellt. Auff\"{a}llig ist, dass es zwei ruhige Zonen gibt, in denen der Einfluss der Knotenkr\"{a}fte auf $J(u_h)$ praktisch null ist. Wir nennen diese Punkte {\em  Lagrangepunkte\/}\index{Lagrangepunkt}. In der Astronomie sind die Lagrangepunkte die Punkte, in denen sich die Gravitationskr\"{a}fte der Sonne und des Mondes das Gleichgewicht halten, weil sie mit gegengleichen Kr\"{a}ften an einem Satelliten ziehen, der dort geparkt ist. Solche Lagrangepunkte findet man in fast allen diesen Plots.\\

\begin{remark}
Mit einem FE-Programm erzeugt man diese Bilder wie folgt:
\begin{enumerate}
  \item Man bringt die $j_i = J(\vek \Np_i)$ als \"{a}quivalente Knotenkr\"{a}fte auf und l\"{o}st das System $\vek K\,\vek g = \vek j$.
  \item Man plottet in jedem Knoten $k$ den Vektor $\vek g_k = \{g_x^{(k)}, g_y^{(k)}\}^T$, also die horizontale und vertikale Verschiebung des Knotens.
\end{enumerate}
\end{remark}
%-----------------------------------------------------------------
\begin{figure}[tbp]
\centering
\includegraphics[width=.99\textwidth]{\Fpath/U436}
\caption{Stockwerkrahmen mit Diagonalstreben \textbf{ a)} Einflussfunktion f\"{u}r die horizontale Lagerkraft im linken Lager \textbf{ b)} f\"{u}r die vertikale Lagerkraft im vorletzten Lagerknoten} \label{U436}
\end{figure}%
%-----------------------------------------------------------------

Bei eindimensionalen Problemen wie Rahmen entstehen die Sensitivit\"{a}tsplots, wenn man die Einflussfunktionen als ebene Verschiebungsfiguren antr\"{a}gt, wie in Abb. \ref{U434}, also in einer Figur beide Anteile, horizontal wie vertikal, zeichnet. Wanderlasten, die maximalen Effekt erzielen wollen, m\"{u}ssen in Richtung der roten Pfeile zeigen -- senkrecht dazu ist die Wirkung null.

Abb. \ref{U436} enth\"{a}lt die Sensitivit\"{a}tsplots f\"{u}r zwei Lagerkr\"{a}fte eines Stockwerkrahmens mit  Diagonalstreben.


%%%%%%%%%%%%%%%%%%%%%%%%%%%%%%%%%%%%%%%%%%%%%%%%%%%%%%%%%%%%%%%%%%%%%%%%%%%%%%%%%%%%%%%%%%%%%%%%%%%
{\textcolor{sectionTitleBlue}{\section{Die Lagerkr\"{a}fte der FE-L\"{o}sung}}}\label{Korrektur20}
Das Thema Lagerkr\"{a}fte und FE-L\"{o}sungen muss man mit Vorsicht angehen. Nichts liegt n\"{a}her, als die Knotenkr\"{a}fte direkt in Lagerkr\"{a}fte zu \"{u}bersetzen. Bei Stabtragwerken sind es echte Kr\"{a}fte, bei Fl\"{a}chentragwerken sind es jedoch in der Regel nur \"{a}quivalente Knotenkr\"{a}fte, also Kr\"{a}fte, die nicht wirklich im strengen Punktsinn vorhanden sind.

%-----------------------------------------------------------------
\begin{figure}[tbp]
\centering
\includegraphics[width=0.8\textwidth]{\Fpath/U142}
\caption{Starre St\"{u}tze, die gesamte St\"{u}tzenkraft ist die Summe aus der St\"{u}tzenkraft $R_{FE}$ der FE-L\"{o}sung plus dem direkt in die St\"{u}tze reduziertem Anteil aus der Last $p$} \label{U142}
\end{figure}%
%-----------------------------------------------------------------

Die $f_i$ in den Lagerknoten berechnet ein FE-Programm im Nachlauf, nachdem es das System $\vek K\,\vek u = \vek f$ gel\"{o}st hat, wie folgt: \\
\begin{itemize}
  \item Es erweitert den Vektor $\vek u$ zun\"{a}chst um die zuvor gestrichenen $u_i = 0$ in den Lagerknoten, $\vek u \to \vek u_{G}$,
  \item und multipliziert die nicht-reduzierte, globale Steifigkeitsmatrix $\vek K_{G}$\index{nicht-reduzierte Steifigkeitsmatrix} mit dem vollen Vektor $\vek u_{G}$,
  \item die Eintr\"{a}ge $f_i$ in dem Vektor $\vek f_{G} = \vek K_{G}\,\vek u_{G}$, die zu den gesperrten Freiheitsgraden geh\"{o}ren, sind die Knotenkr\"{a}fte in den Lagern {\em ohne\/} die Anteile der Last, die direkt in die Lager reduziert wurden. Zu diesen muss man also noch die Lagerkr\"{a}fte aus der direkten Reduktion addieren, die wir $R_{d}$ nennen, s. Abb. \ref{U142},
      \begin{align}
      f_i(komplett) = f_i +  R_{d} = R_{FE} + R_{d}\,.
      \end{align}
      \item Wenn allerdings die Lager nachgiebig gerechnet wurden, dann ist das letzte Man\"{o}ver nicht notwendig, dann beinhaltet $f_i = R_{FE}$ die volle Lagerkraft.
\end{itemize}

Die Summe der $f_i (komplett)$ in den Lagern muss gleich der aufgebrachten Belastung sein, weil die {\em shape functions\/} im Regelfall eine {\em partition of unity\/} bilden oder bilden sollten, die Starrk\"{o}rperbewegungen des frei geschnittenen Tragwerks also in $\mathcal{V}_h^+$ liegen, s. \cite{Ha5} Chapter 1.35 und 1.44. \index{Gleichgewicht}\index{$\mathcal{V}_h^+$}

Die Ansatzfunktionen $\Np_i$ eines Netz bilden den Raum $\mathcal{V}_h^+$. Wenn man die Ansatzfunktionen $\Np_i$ streicht, die zu gesperrten Freiheitsgraden geh\"{o}rt, dann erh\"{a}lt man den Unterraum $\mathcal{V}_h \subset \mathcal{V}_h^+$ auf dem wir die FE-L\"{o}sung suchen.

Lokal ist das Gleichgewicht, also der Vergleich der Belastung $p$ mit den Schnittkr\"{a}ften der FE-L\"{o}sung, nicht erf\"{u}llt, weil ja die FE-L\"{o}sung zu einem anderen Lastfall, dem FE-Lastfall $p_h$ geh\"{o}rt. Technisch ist der Grund der, dass die Starrk\"{o}rperbewegungen der lokalen {\em patchs\/} nicht in $\mathcal{V}_h$ liegen.

Was es mit der Zweiteilung der Lagerkr\"{a}fte in \glq echte\grq{} und \glq nur gedachte\grq{} auf sich hat, soll zun\"{a}chst an Hand der Stabstatik untersucht werden.

%%%%%%%%%%%%%%%%%%%%%%%%%%%%%%%%%%%%%%%%%%%%%%%%%%%%%%%%%%%%%%%%%%%%%%%%%%%%%%%%%%%%%%%%%%%%%%%%%%%
{\textcolor{sectionTitleBlue}{\section{Lagersenkung}}}\label{Korrektur17}\index{Lagersenkung}
Es sei $u_5$ der Freiheitsgrad, der zu dem abgesenkten ($\Delta w$) Lager geh\"{o}re. Das FE-Programm bringt die Spalte $\vek f_5$ von $\vek K$ auf die rechte Seite, streicht die Zeile 5 und Spalte 5 in dem System und bestimmt den Vektor $\vek u$ aus dem verk\"{u}rzten System $\vek K\,\vek u = - \Delta w\cdot \vek f_5$ und findet so die Biegelinie
\begin{align}
w(x) = \sum_{\stackrel{i \neq 5}{i = 1}}^n\, u_i\,\Np_i(x) + \Delta w \cdot \Np_5(x) =: w_V(x) + \Delta w \cdot \Np_5(x)\,.
\end{align}
Die Umstellung  $\vek K\,\vek u = - \Delta w \cdot \vek f_5$ ist nachvollziehbar und schl\"{u}ssig, aber auch sie hat einen mathematischen Hintergrund.

Es gilt n\"{a}mlich, dass Biegelinien $w$ aus Lagersenkung orthogonal sind zu den virtuellen Verr\"{u}ckungen des Systems, $\delta A_i(w,\delta w) = 0$, s. Glg. (\ref{Eq120}) S. \pageref{Eq120}. W\"{a}hlen wir als $\delta w$ die $\Np_i$ -- mit Ausnahme von $\Np_5$ selbst, weil man das abgesenkte aber feste Lager ja nicht verr\"{u}cken kann -- dann gilt also
\begin{align}
\delta A_i(w,\Np_i) = a(w_V + \Delta w \cdot \Np_5,\Np_i) = a(w_V,\Np_i) + \Delta w \cdot a(\Np_5,\Np_i) = 0
\end{align}
und somit
\begin{align}
a(w_V,\Np_i) = - \Delta w \cdot a(\Np_5,\Np_i) \qquad i = 1,2,\ldots, n \,\,\,\, i \neq 5
\end{align}
und das ist genau das verk\"{u}rzte System $\vek K\,\vek u = -\Delta w \,\vek f_5$, weil $w_V$ ja eine Entwicklung nach den {\em shape functions\/} $\Np_i$ ist (mit L\"{u}cke bei $\Np_5$).
%-----------------------------------------------------------------
\begin{figure}[tbp]
\centering
\includegraphics[width=.99\textwidth]{\Fpath/U443}
\caption{Lagersenkung} \label{U443}
\end{figure}%
%-----------------------------------------------------------------

Rechentechnisch einfacher ist vielleicht das folgende Vorgehen: Man entfernt den Knoten nicht, belastet wie zuvor die Knoten mit den Kr\"{a}ften
\begin{align}
f_j = -a(\Np_5,\Np_j) \cdot \Delta w= - k_{5j} \cdot \Delta w
\end{align}
und addiert am Schluss zu der Biegefigur die mit der Lagersenkung $\Delta w$ skalierte Einheitsverformung $\Np_5 \cdot \Delta w$ des Lagers.

Das kann man sich so zurechtlegen: In Spalte 5 stehen die Kr\"{a}fte $f_j = k_{5j}$, die n\"{o}tig sind, um den FG $u_5$ um eine Einheit auszulenken und gleichzeitig die Bewegung an den n\"{a}chsten Knoten zum Stillstand zu bringen. Indem man die Knoten mit den $-f_j$ belastet, hebt man die Sperre auf (alles nat\"{u}rlich mal $\Delta w$), und weil man das Lager nicht weggenommen hat, muss man noch $\Np_5 \cdot \Delta w$ zur Biegelinie addieren.

{\em Beispiel:\/} Das rechte Lager in Abb. \ref{U443} senkt sich um 1 m. An der nicht verk\"{u}rzten Steifigkeitsmatrix
\begin{align}
\left[ \barr{l  @{\hspace{4mm}}l   @{\hspace{4mm}}l} 8 &6\,\leftarrow &2 \\ 6 &12 &6 \\ 2 & 6\,\leftarrow &4 \earr \right] \left[ \barr{c} u_3 \\u_5 \\u_6 \earr\right] = \left[ \barr{c} f_3 \\f_5 \\f_6 \earr\right]
\end{align}
kann man ablesen, wie man das System (mit gesperrtem FG 5) zu belasten hat
\begin{align}
\left[ \barr{c  @{\hspace{4mm}}c} 8 &2 \\ 2  &4 \earr \right] \left[ \barr{c} u_3 \\u_6 \earr\right] = -\left[ \barr{c} 6 \\6 \earr\right]
\end{align}
Mit der L\"{o}sung $u_3 = -0.4286$, $u_6 = -1.2857$ und $u_5 = 1.0$ ergibt sich so die Biegelinie als die Summe der drei Einheitsverformungen
\begin{align}
w(x) = u_3\,\Np_3(x) + u_5\,\Np_5(x) + u_6\,\Np_6(x)\,.
\end{align}

%%%%%%%%%%%%%%%%%%%%%%%%%%%%%%%%%%%%%%%%%%%%%%%%%%%%%%%%%%%%%%%%%%%%%%%%%%%%%%%%%%%%%%%%%%%%%%%%%%%
{\textcolor{sectionTitleBlue}{\section{Einflussfunktion f\"{u}r ein starres Lager}}}
Vom Standpunkt der \glq Schulstatik\grq{} aus ist alles klar: Man entfernt das Lager und dr\"{u}ckt den Balken um einen Meter nach unten, s. den vorhergehenden Abschnitt. Die sich dabei einstellende Biegelinie ist die Einflussfunktion. Das bleibt nat\"{u}rlich weiterhin richtig.

Nun kann man sich dem Problem aber auch aus der Sicht der finiten Elemente n\"{a}hern. Eine Lagerkraft $R$ ist ein Funktional
\begin{align}
R = J(w)\,,
\end{align}
angewandt auf die Biegelinie $w$ und die FE-Einflussfunktion f\"{u}r $R$ erh\"{a}lt man, so lautet die Regel, wenn man als Knotenkr\"{a}fte $j_i$ die Lagerkr\"{a}fte der Ansatzfunktionen w\"{a}hlt
\begin{align}
j_i = R(\Np_i)\,.
\end{align}
Also die Lagerkraft, die zur Biegelinie $w = \Np_i$ geh\"{o}rt.

Diese Lagerkraft ist aber gleich der Wechselwirkungsenergie $\times (-1)$ zwischen $\Np_i $ und der Funktion $\Np_l$
\begin{align}\label{Eq53}
a(\Np_i,\Np_l) = \int_0^{\,l} EI\,\Np_i''\,\Np_l''\, dx = - R(\Np_i) \cdot 1\,,
\end{align}
wenn $\Np_l$ die Einheitsverformung des Lagerknotens ist, also die Funktion, bei der sich das Lager um eine L\"{a}ngeneinheit nach unten bewegt. Die Terme (\ref{Eq53}) stehen in der Spalte $l$ der nicht reduzierten globalen Steifigkeitsmatrix $\vek K_G$.

%-----------------------------------------------------------------
\begin{figure}[tbp]
\centering
\includegraphics[width=.99\textwidth]{\Fpath/U106}
\caption{Tr\"{a}ger, \textbf{ a)} Lagerreaktion $R$, \textbf{ b)} Einflussfunktion $G$ f\"{u}r $R$, \textbf{ c)} FE-Einflussfunktion $G_h$ auf $\mathcal{V}_h$; die Knotenkr\"{a}fte $j_i$, die $G_h$ erzeugen, sind die Lagerkr\"{a}fte $R$ der Ansatzfunktionen} \label{U106}
\end{figure}%
%-----------------------------------------------------------------

Dieses Ergebnis beruht auf der ersten Greenschen Identit\"{a}t, denn (\ref{Eq53}) ist die Identit\"{a}t in der Gestalt $\delta A_i = \delta A_a$
\begin{align}
\text{\normalfont\calligra G\,\,}(\Np_i,\Np_l) &= \text{\normalfont\calligra G\,\,}(\Np_i,\Np_l)_{(0,x)} + \text{\normalfont\calligra G\,\,}(\Np_i,\Np_l)_{(x,l)}\nn \\
&=\underbrace{V(\Np_i)(x_{-}) - V(\Np_i)(x_{+})}_{-R(\Np_i)} -\int_0^{\,l} EI\,\Np_i''\,\Np_l''\, dx = 0\,.
\end{align}
Die $\Np_i $ sind homogene L\"{o}sungen der Balkengleichung und $\Np_l$ ist in allen Knoten, bis auf den Lagerknoten, null. Dort ist $\Np_l = 1$ und $V$ springt. Das erkl\"{a}rt, warum die virtuelle \"{a}u{\ss}ere Arbeit sich auf den Ausdruck $ -R(\Np_i) \cdot 1$ verk\"{u}rzt.

\hspace*{-12pt}\colorbox{highlightBlue}{\parbox{0.98\textwidth}{Die Lagerkraft $R$ einer Ansatzfunktion $\Np_i(x)$ ist gleich der Wechselwirkungsenergie $a(\Np_i,\Np_l) \times (-1)$  zwischen $\Np_i(x)$ und $\Np_l(x)$. Wenn $u_l$ der Freiheitsgrad des Lagers ist, dann ist also $\vek j = -\vek K_G\,\vek e_l$.}}\\

Mit dem Vektor\footnote{Im Vektor $\vek K_G\,\vek e_l$ werden die Eintr\"{a}ge gestrichen, die zu gesperrten FG geh\"{o}ren.} $\vek g = \vek K^{-1}\,\vek j = -\vek K^{-1}\,\vek K_G\,\vek e_l$ lautet also die Einflussfunktion f\"{u}r die Lager\-kraft im Knoten $x$ (FG $u_l$)
\begin{align}
G_h(y,x) = \sum_i\,g_i\,\Np_i(y) = \vek \Phi(y)^T\,\vek g = \vek \Phi(y)^T\,\vek K^{-1}\,\vek j  \,.
\end{align}
Dieser Einflussfunktion fehlt aber offensichtlich, s. Abb. \ref{U106} c, das St\"{u}ck $\Np_l$ direkt unter  dem Lager. Das muss aber so sein, weil der Ansatzraum $\mathcal{V}_h$ ja die Funktion $\Np_l$ nicht enth\"{a}lt -- der Knoten wird ja festgehalten. Aber warum kommen die finiten Elemente dann trotzdem auf die richtige Lagerkraft? Das liegt daran, wie FE-Programme vorgehen.


Ist eine verteilte Last $p(x)$ gegeben, dann stellen sie in jeden Knoten die zugeh\"{o}rige \"{a}quivalente Knotenkraft und daher in den Lagerknoten die Kraft
\begin{align}
f_l = \int_0^{\,l} p(x)\,\Np_l(x)\,dx\,.
\end{align}
Diese Kraft wandert aber direkt in das Lager und beeinflusst somit die FE-Berechnung gar nicht. Alle Verformungen und alle Schnittkr\"{a}fte der FE-L\"{o}sung kommen aus dem Lastfall, bei dem dieser Anteil fehlt. Und die Einflussfunktion in Abb. \ref{U106} ist genau die Einflussfunktion  f\"{u}r die Lagerkraft in solchen \glq amputierten\grq{} Lastf\"{a}llen.
%-----------------------------------------------------------------
\begin{figure}[tbp]
\centering
\includegraphics[width=0.9\textwidth]{\Fpath/U194}
\caption{Seil, \textbf{ a)} System, \textbf{ b)} FE-L\"{o}sung \textbf{ c)} -   \textbf{ d)} lokale L\"{o}sungen, \textbf{ e)} exakte L\"{o}sung} \label{U194}
\end{figure}%
%-----------------------------------------------------------------

Die direkt in die St\"{u}tze flie{\ss}ende Knotenkraft $f_l$ erzeugt eine Gegenkraft in der St\"{u}tze, die wir $R_{d} = - f_l$ nennen.

Am Ende der Berechnung addiert das FE-Programm zu der Lagerkraft $R_{FE}$ des \glq amputierten\grq{} Lastfalls die Lagerkraft $R_{d}$ hinzu und so stimmt am Schluss wieder alles
\begin{align}
R = R_{FE} + R_{d}\,.
\end{align}
Man kann das ganze nat\"{u}rlich auch \glq von hinten\grq{} aufz\"{a}umen, indem man einfach zur FE-Einflussfunktion $G_h$ den fehlenden Anteil $\Np_l$ addiert, und so genau die Einflussfunktion $G$ erh\"{a}lt
\begin{align}
G = G_h + \Np_l\,,
\end{align}
wie sie der Ingenieur sehen will. Wenn man diese mit der Belastung \"{u}berlagert, ist das Ergebnis die volle Lagerkraft
\begin{align}
R = R_{FE} + R_d = \int_0^{\,l} G(y,x)\,p(y)\,dy\,.
\end{align}
In der Stabstatik geschieht die Addition $R = R_{FE} + R_{d}$ \glq automatisch\grq{}, n\"{a}mlich in dem Moment, in dem die lokalen L\"{o}sungen elementweise zur FE-L\"{o}sung addiert werden.

Das Seil in Abb. \ref{U194} a wird mit einer konstanten Streckenlast $p = 10$\,kN/m belastet. Eine Unterteilung in zwei lineare Elemente\footnote{Das ist die nicht-reduzierte globale Steifigkeitsmatrix $\vek K_G$.}
\begin{align}
        \left[ \barr {r @{\hspace{4mm}}r @{\hspace{4mm}}r}
      2 & -1 & 0  \\
      -1 & 2 & -1 \\
      0 & -1 & 2  \\
      \earr \right]\,\left[ \barr {r} 0 \\ 5 \\ 0 \earr \right] = \left[ \barr {r} f_1 \\ 10 \\ f_3 \earr \right]
      \end{align}
hat das Ergebnis $u_2 = 5$ und damit erh\"{a}lt man in der Nachlaufrechnung als Lagerkr\"{a}fte $f_1 = f_3 = - 5$. Dazu m\"{u}sste man noch die Lagerkr\"{a}fte aus der direkten Reduktion addieren. Das geschieht aber automatisch wie folgt: Elementweise werden die lokalen L\"{o}sungen bestimmt, s. Abb. \ref{U194} c und d, und zur FE-L\"{o}sung addiert. Die Folge ist, dass der Seildurchhang $w = w_{FE} + w_{loc}$ jetzt sch\"{o}n rund ist, und die Lagerkr\"{a}fte genau \glq passen\grq{}.

%%%%%%%%%%%%%%%%%%%%%%%%%%%%%%%%%%%%%%%%%%%%%%%%%%%%%%%%%%%%%%%%%%%%%%%%%%%%%%%%%%%%%%%%%%%%%%%%%%%
{\textcolor{sectionTitleBlue}{\section{Einflussfunktion f\"{u}r ein nachgiebiges Lager}}}
Der Fusspunkt eines elastischen Lagers ist ein fester Punkt, ein festes Lager und so kann man die Einflussfunktion f\"{u}r die Fusspunktskraft mit der obigen Technik berechnen, s. Abb. \ref{U179}.

Einfacher ist es nat\"{u}rlich, wenn man in den Kopf des nachgiebigen Lagers eine Kraft $\bar{P} = 1$ stellt, die Reaktion $\bar{w}$ des Tragwerks darauf berechnet
\begin{align}
\bar{w} = \frac{1}{3\,EI/l^3 + k}
\end{align}
und diese Figur (= Einflussfunktion f\"{u}r die Zusammendr\"{u}ckung der Feder) mit der Steifigkeit $k$ der Feder multipliziert, s. Abb. \ref{U475} a,
\begin{align}
F = k \int_{0}^{l} G_0(y,l)\,p(y)\,dy\,.
\end{align}

%-----------------------------------------------------------------
\begin{figure}[tbp]
\centering
\includegraphics[width=0.9\textwidth]{\Fpath/U179}
\caption{Gelenkig gelagerte Platte mit Innenst\"{u}tzen, \textbf{ a)} System \textbf{ b)} Einflussfunktion f\"{u}r die vordere, linke St\"{u}tzenkraft. Von der angesetzten Spreizung von 1000 mm werden $\sim 200$\,mm von der St\"{u}tze \glq verschluckt\grq{}, also rund $80\, \% \doteq 800 \, \text{mm}/1000\,\text{mm}$ einer Punktlast gehen direkt in die St\"{u}tze darunter. Die restlichen $20 \%$ tr\"{a}gt die Platte} \label{U179}
\end{figure}%
%-----------------------------------------------------------------

%-----------------------------------------------------------------
\begin{figure}[tbp]
\centering
\includegraphics[width=0.8\textwidth]{\Fpath/U475}
\caption{Elastische Einspannung, $k_\Np$ ist die Steifigkeit der Drehfeder, die Steifigkeit der Einspannung ist $3\,EI/l + k_\Np$, \textbf{ a)} Einflussfunktion $G_1$ f\"{u}r die Verdrehung der Einspannung \textbf{ b)} Momentenverteilung aus der Belastung} \label{U475}
\end{figure}%
%-----------------------------------------------------------------
Sinngem\"{a}{\ss} dasselbe gilt f\"{u}r Drehfedern, s. Abb. \ref{U475} b. Man verdreht die Einspannung mit einem Moment $\bar{M} = 1$, arbeitet also gegen die Drehsteifigkeit des Balkens plus der Drehsteifigkeit $k_\Np$ der Drehfeder
\begin{align}
\tan\,\bar{\Np} = \frac{1}{3\,EI/l + k_\Np}
\end{align}
und diese Biegelinie $G_1(y,l)$, gewichtet mit $k_\Np$, ist die Einflussfunktion f\"{u}r das Moment in der Feder aus einer Streckenlast $p$
\begin{align}
M = k_\Np \int_0^{\,l} G_1(y,l)\,p(y)\,dy\,.
\end{align}

%-----------------------------------------------------------------
\begin{figure}[tbp]
\centering
\includegraphics[width=0.7\textwidth]{\Fpath/U422}
\caption{Innenwand, \textbf{ a)} Lagerkr\"{a}fte, \textbf{ b)} Einflussfunktion f\"{u}r die Knotenkraft (= \glq halbe Lagerkraft\grq{}) am Wandende; sie wird ausgel\"{o}st durch eine Absenkung des Elements} \label{U422}
\end{figure}%
%-----------------------------------------------------------------

%%%%%%%%%%%%%%%%%%%%%%%%%%%%%%%%%%%%%%%%%%%%%%%%%%%%%%%%%%%%%%%%%%%%%%%%%%%%%%%%%%%%%%%%%%%%%%%%%%%
{\textcolor{sectionTitleBlue}{\section{Wandknoten}}\label{Korrektur31}
Mit den Knotenkr\"{a}ften an den Enden der W\"{a}nde\index{Wandknoten} werden die Durchstanznachweise\index{Durchstanznachweise} gef\"{u}hrt. Wenn man den Praktiker fragt, wie denn die Einflussfunktion f\"{u}r eine solche Knotenkraft aussieht, dann wird er sagen: \glq Man muss den Knoten um 1 m absenken\grq{}.

Das ist richtig, aber wir wollen doch die Situation etwas genauer betrachten. Statt nach der Knotenkraft fragen wir nach dem Wert der \glq halben\grq{} Lagerkraft im letzten Element der Wand, also dem Integral
\begin{align}
J(w)= \int_0^{\,l_e} \frac{x}{l_e}\,l(x)\,dx  \qquad l(x) = \text{Lagerkraft}\,.
\end{align}
Die Funktion $x/l_e$, die von $0$ auf den Endwert $1$ ansteigt, ist die Wichtungsfunktion. Wir stellen uns das so vor, dass wir das letzte Element entfernen und die Platte im Bereich des (entfernten) Elementes so belasten, dass die Biegefl\"{a}che dort diesen linearen Verlauf aufweist. Diese Biegefl\"{a}che $G(\vek x)$ ist die Einflussfunktion.  Aus dem Satz von Betti
\begin{align}
A_{1,2} = \int_{\Omega} G(\vek x)\,p(\vek x)\,d\Omega - \int_0^{\,l_e} \frac{x}{l_e}\,l(x)\,dx = A_{2,1} = 0
\end{align}
ergibt sich f\"{u}r die \glq halbe\grq{} Lagerkraft der Wert
\begin{align}
J(w) = \int_{\Omega} G(\vek x)\,p(\vek x)\,d\Omega\,.
\end{align}
Die Arbeit $A_{2,1} = 0 $ ist null, weil die Linienkr\"{a}fte $\delta^{@l}$, die die Platte l\"{a}ngs des Elements nach unten dr\"{u}cken, um die Biegefl\"{a}che $G(\vek x)$ zu erzeugen, auf den Wegen $w = 0$ des Wandelementes keine Arbeit leisten.

Gem\"{a}{\ss} {\em Betti extended\/} lautet das FE-Ergebnis
\begin{align}\label{Eq138}
J(w_h)= \int_{\Omega} G_h(\vek x)\,p(\vek x)\,d\Omega\,,
\end{align}
wenn $G_h$ die FE-Biegefl\"{a}che ist, die l\"{a}ngs des Elementes den Verlauf $x/l_e $ hat.  Diese Biegefl\"{a}che wird durch die Absenkung des Wandendes um 1 m erzeugt. Das ist die Einflussfl\"{a}che des Praktikers.

Das FE-Ergebnis (\ref{Eq138}) kann man im \"{u}brigen, wie wir kurz zeigen wollen, direkt aus dem Satz von Betti
\begin{align}
A_{1,2} = \int_{\Omega} G_h(\vek x)\,p(\vek x)\,d\Omega - \int_0^{\,l_e} \frac{x}{l_e}\,l(x)\,dx = A_{2,1}  = 0
\end{align}
herleiten.

Zun\"{a}chst bemerken wir, dass wir f\"{u}r $p$ auch $p_h$ in dem obigen Integral setzen k\"{o}nnen. Mit dem {\em switch\/} $p \to p_h$ ist die Arbeit $A_{2,1}$ das Integral
\begin{align}
A_{2,1} = \int_{\Omega} \delta_h^{@l}(\vek x)\,w_h(\vek x)\,d\Omega\,,
\end{align}
wenn $\delta_h^l(\vek x)$ die Last ist, die die N\"{a}herung $G_h(\vek x)$ mit dem $x/l_e$ l\"{a}ngs des Elements erzeugt. Nun ist $w_h$ l\"{a}ngs des Elements null und somit ist es auch das Integral.

Wenn man dasselbe mit einem Innenknoten der Wand macht, dann erzeugt die Absenkung des Knotens nat\"{u}rlich die Einflussfunktion f\"{u}r die Summe der beiden \glq halben\grq{} Wandkr\"{a}fte links und rechts vom Knoten.

%-----------------------------------------------------------------
\begin{figure}[tbp]
\centering
\includegraphics[width=1.0\textwidth]{\Fpath/U311}
\caption{Punktgest\"{u}tzte Platte, LF g, \textbf{ a)} Biegefl\"{a}che, \textbf{ b)} Lagerkr\"{a}fte} \label{U311}
\end{figure}%
%-----------------------------------------------------------------
%-----------------------------------------------------------------
\begin{figure}[tbp]
\centering
\includegraphics[width=1.0\textwidth]{\Fpath/U312}
\caption{W\"{a}nde unter einer Hochbaudecke, auch diese Lagerkr\"{a}fte kann ein FE-Programm relativ genau ermitteln} \label{U312} % Position 503
\end{figure}%
%-----------------------------------------------------------------

%%%%%%%%%%%%%%%%%%%%%%%%%%%%%%%%%%%%%%%%%%%%%%%%%%%%%%%%%%%%%%%%%%%%%%%%%%%%%%%%%%%%%%%%%%%%%%%%%%%
{\textcolor{sectionTitleBlue}{\section{Genauigkeit der Lagerkr\"{a}fte}}
Die Einflussfunktionen f\"{u}r St\"{u}tzen sind einfache Senken in der Platte, die durch eine Spreizung der Gr\"{o}{\ss}e Eins zwischen St\"{u}tzenkopf und Platte ausgel\"{o}st werden. Sie \"{a}hneln vom Typ her den Einflussfunktionen f\"{u}r Durchbie\-gungen und sie sind daher auch schon auf relativ groben Netzen gut anzun\"{a}hern. Das ist der Grund, warum man bei einer FE-Berechnung keine gro{\ss}en Zweifel an der H\"{o}he der ausgewiesenen St\"{u}tzenkr\"{a}fte haben muss, s. Abb. \ref{U311}. Das gilt sinngem\"{a}{\ss} auch f\"{u}r W\"{a}nde, s. Abb. \ref{U312} und auch f\"{u}r Unterz\"{u}ge, obwohl da in der Praxis zum Teil heftig diskutiert wird, was die Modellierung von Unterz\"{u}gen angeht, s. Abb. \ref{U308}.

%%%%%%%%%%%%%%%%%%%%%%%%%%%%%%%%%%%%%%%%%%%%%%%%%%%%%%%%%%%%%%%%%%%%%%%%%%%%%%%%%%%%%%%%%%%%%%%%%%%
{\textcolor{sectionTitleBlue}{\section{Punktlasten und Punktlager bei Scheiben}}
Punktkr\"{a}fte bei Scheiben f\"{u}hren zu unendlich gro{\ss}en Spannungen und so k\"{o}nnen Scheiben auch nicht auf Punktlager gesetzt werden, weil das Material einfach zu flie{\ss}en anfangen w\"{u}rde. Andererseits erh\"{a}lt man aber mit finiten Elementen doch sinnvolle Ergebnisse in Punktlagern, s. Abb. \ref{U84}. Wie geht das?

Es geht, weil auch die finiten Elemente einen geb\"{u}hrenden Abstand von echten Punktkr\"{a}ften einhalten. Im Ausdruck stehen zwar \"{a}quivalente Knotenkr\"{a}fte $f_i$, aber sie sind ja nur Stellvertreter f\"{u}r die wahren St\"{u}tzkr\"{a}fte, die als Fl\"{a}chen- und Linienkr\"{a}fte in der Umgebung des Knotens die Scheibe halten.
%-----------------------------------------------------------------
\begin{figure}[tbp]
\centering
\includegraphics[width=0.8\textwidth]{\Fpath/U84}
\caption{Einflussfunktion f\"{u}r die FE-Lagerkraft in dem Punktlager einer Scheibe} \label{U84}
\end{figure}%
%-----------------------------------------------------------------

%---------------------------------------------------------------------------------
\begin{figure}
\centering
\if \bild 2 \sidecaption[t] \fi
{\includegraphics[width=0.9\textwidth]{\Fpath/U109}}
\caption{Original und FE-Netz mit \"{a}quivalenten Knotenkr\"{a}ften}
\label{U109}%
\end{figure}%
%---------------------------------------------------------------------------------

Auf der Lastseite gilt dasselbe. Wenn der Anwender eine Knotenkraft eingibt, dann behilft sich ein FE-Programm damit eine Schar von \"{a}quivalenten Fl\"{a}chen- und Linienkr\"{a}ften in die N\"{a}he des Knotens zu plazieren.

Dies ist die Stelle, wo man sieht, wie die finiten Elemente ihr Eigenleben entwickeln. Formal sind sie nur ein numerisches Werkzeug, aber der Ingenieur findet gar nichts dabei, die Knotenkr\"{a}fte f\"{u}r real zu nehmen und so kommt er mit Hilfe der finiten Elemente sehr elegant \"{u}ber die Stolpersteine hinweg, die ihm die Elastizit\"{a}tstheorie in den Weg legt.

%---------------------------------------------------------------------------------
\begin{figure}
\centering
\if \bild 2 \sidecaption[t] \fi
{\includegraphics[width=1.0\textwidth]{\Fpath/U281}}
\caption{Wandscheibe auf Punktlagern (Ausschnitt)}  % Pos. NE2
\label{U281}%
\end{figure}%
%---------------------------------------------------------------------------------

So ist das statische Problem der Wandscheibe in Abb. \ref{U28}, die sich auf zwei Punktlager st\"{u}tzt, vom  Standpunkt der Mathematik aus ein schlecht gestelltes Problem, weil die Spannungen und Verformungen der Scheibe in den Lagerpunkten gem\"{a}{\ss} der Elastizit\"{a}tstheorie unendlich gro{\ss} werden und echte Punktlager die Scheibe auch nicht festhalten k\"{o}nnten.

Mit finiten Elementen ist man jedoch noch ein gutes St\"{u}ck von dieser Grenze entfernt. Zudem ist
die Scheibe statisch bestimmt gelagert und so ist es nur nat\"{u}rlich, die $f_i$ wie echte Kr\"{a}fte zu behandeln.

Viele Ingenieure interpretieren das FE-Modell einer Scheibe ja als eine Art \glq Fachwerk\grq{}\index{Fachwerkmodell}, wo kleine Scheibenelemente in den Knoten miteinander verbunden sind und wo das Gleichgewicht in den Knoten, $\vek K\,\vek u = \vek f$, die Gleichgewichtslage $\vek u$ bestimmt, s. Abb. \ref{U109}. Das ist aber eine Interpretation \glq als ob\grq{}, schon deswegen, weil auf beiden Seiten von $\vek K\,\vek u = \vek f$ Arbeiten stehen und keine Kr\"{a}fte; es werden Arbeiten bilanziert.

Das (scheinbar) merkw\"{u}rdige ist, dass man f\"{u}r diese \glq fiktiven\grq{} \"{a}quivalenten Knotenkr\"{a}fte $f_i$ Einflussfunktionen aufstellen kann. Einfach so, wie man das auch bei einem Fachwerk machen w\"{u}rde: man verschiebt den Lagerknoten um einen Meter und bilanziert so die Arbeiten, die von der Belastung und der Lagerkraft bei einer Verr\"{u}ckung des Lagers geleistet werden.

%-----------------------------------------------------------------
\begin{figure}[tbp]
\centering
\includegraphics[width=1.0\textwidth]{\Fpath/U308}
\caption{Hochbauplatte, \textbf{ a)} Unterkonstruktion, \textbf{ b)} Hauptmomente im LF $g$} \label{U308}
\end{figure}%
%-----------------------------------------------------------------

%%%%%%%%%%%%%%%%%%%%%%%%%%%%%%%%%%%%%%%%%%%%%%%%%%%%%%%%%%%%%%%%%%%%%%%%%%%%%%%%%%%%%%%%%%%%%%%%%%%
{\textcolor{sectionTitleBlue}{\section{Punktlager sind hot spots}}}\index{hot spots}\label{Punktlager}
Wenn man einen Knoten festh\"{a}lt, dann wird die Scheibe dort praktisch \glq geerdet\grq{}. Das ist so, als ob man mit der einen Hand eine Hochspannungsleitung ber\"{u}hrt und mit der anderen die Erde. Der steile Anstieg der Verschiebungen vom festen Lager zu den freien Knoten produziert gro{\ss}e Spannungen in den Elementen, die mit dem festen Knoten verbunden sind, s. Abb. \ref{U281}.

Je kleiner die Elemente in der N\"{a}he des Festpunktes werden, um so steiler ist der Verschiebungsgradient in den Elementen und um so gr\"{o}{\ss}er sind somit auch die Spannungen in den Elementen.
%---------------------------------------------------------------------------------
\begin{figure}
\centering
\if \bild 2 \sidecaption[t] \fi
{\includegraphics[width=0.4\textwidth]{\Fpath/U20}}
\caption{Durch das letzte Element vor dem Punktlager muss die ganze Lagerkraft flie{\ss}en...}
\label{U20}%
\end{figure}%
%---------------------------------------------------------------------------------

Warum die Spannungen unendlich gro{\ss} werden, ja unendlich gro{\ss} werden m\"{u}ssen, versteht man, wenn man sich die finiten Elemente anschaut.
%---------------------------------------------------------------------------------
\begin{figure}
\centering
\if \bild 2 \sidecaption[t] \fi
{\includegraphics[width=0.9\textwidth]{\Fpath/U337}}
\caption{Einflussfunktion f\"{u}r die Querkraft $q_x$, \textbf{ a)} die Scherbewegung kann sich frei ausbilden, \textbf{ b)} die starre St\"{u}tze behindert die Scherbewegung, die weiter abliegenden Knotenkr\"{a}fte sind im \"{U}bergewicht und treiben so die Platte nach oben---auch in Bereichen die vom Aufpunkt weiter weg entfernt liegen}
\label{U337}%
\end{figure}%
%---------------------------------------------------------------------------------

%---------------------------------------------------------------------------------
\begin{figure}
\centering
\if \bild 2 \sidecaption[t] \fi
{\includegraphics[width=0.9\textwidth]{\Fpath/U350}}
\caption{Dieselbe Platte wie in Abb. \ref{U337}; die Einflussfunktion f\"{u}r das Biegemoment $m_{xx}$ (im selben Punkt) zeigt dasselbe Verhalten, aber links und rechts in \textbf{ a)} und \textbf{ b)} sind ausgewogener, weil die Einflussfunktion durch eine symmetrische Quelle (ein Quattropol) erzeugt wird}
\label{U350}%
\end{figure}%
%---------------------------------------------------------------------------------

Angenommen in dem Punktlager wirkt eine vertikale Kraft $f_i = 10 $ kN. Wenn man also den Lagerknoten um einen Meter nach oben dr\"{u}ckt (das ist rein rechnerisch), dann leistet die Knotenkraft dabei die Arbeit $\delta A_a =10$ kNm, s. Abb. \ref{U20}.
%---------------------------------------------------------------------------------
\begin{figure}
\centering
\if \bild 2 \sidecaption[t] \fi
{\includegraphics[width=0.65\textwidth]{\Fpath/U294}}
\caption{Generierung der Einflussfunktion f\"{u}r $\sigma_{yy}$, \textbf{ a)} die Knotenkr\"{a}fte, die die Spreizung (n\"{a}herungsweise) erzeugen sind jeweils in allen vier Knoten gleich und h\"{a}ngen nur von der Maschenweite $h$ ab, \textbf{ b)} je kleiner die Elemente werden, um so gr\"{o}{\ss}er werden die Knotenkr\"{a}fte und damit die Verformung der Scheibe. Das Verh\"{a}ltnis zwischen den Kr\"{a}ften steht $2:1$, weil das feste Lager eine Kraft neutralisiert (\glq amputierter Dipol\grq{})}
\label{U294}%
\end{figure}%
%---------------------------------------------------------------------------------

%---------------------------------------------------------------------------------
\begin{figure}
\centering
%\if \bild 2 \sidecaption[t] \fi
{\includegraphics[width=0.60\textwidth]{\Fpath/U295}}
\caption{Generierung der Einflussfunktionen f\"{u}r $\sigma_{yy}$, \textbf{ a)} und \textbf{ b)} bei der Spreizung der Nachbarelemente des Lagerelementes dr\"{u}cken zwei Kr\"{a}fte nach oben und zwei Kr\"{a}fte nach unten und so bleiben die  Verschiebungen (= Spannungen als Einflussfunktion) auch in der Grenze, $h \to 0$, endlich (\glq echter Dipol\grq{})}
\label{U295}%
\end{figure}%
%---------------------------------------------------------------------------------
Die Bewegung des Lagerknotens teilt sich dem Element $\Omega_e$ mit, auf dem das Lager liegt, und so muss die virtuelle innere Energie $\delta A_i$ in dem Element gleich $\delta A_a$ sein
\begin{align}
\delta A_a = 1\cdot f_i = \int_{\Omega_e} \sigma_{ij}\,\delta \varepsilon_{ij}\,d\Omega = \delta A_i\,.
\end{align}
Die Verzerrungen $\delta \varepsilon_{ij}$ resultieren dabei aus der Lagerbewegung $u_i = 1$.

Alle anderen Elemente sp\"{u}ren nichts davon, weil alle anderen Knoten bei dem Man\"{o}ver festgehalten werden. \\

\hspace*{-12pt}\colorbox{highlightBlue}{\parbox{0.98\textwidth}{Dieses letzte vor dem Lager liegende Element muss also ganz allein die n\"{o}tige Energie aufbringen, um die Lagerarbeit ins gleiche zu setzen!}}\\

Wenn nun das Element immer kleiner wird, weil man ja genaue Ergebnisse haben will..., dann m\"{u}ssen die Spannungen in dem Element immer mehr anwachsen, weil immer weniger Fl\"{a}che vorhanden ist, \"{u}ber die man integrieren kann, und so hat man keine Chance irgend etwas vern\"{u}nftiges zu berechnen. Man muss dann umschalten und in \"{a}quivalenten Knotenkr\"{a}ften denken.

%%%%%%%%%%%%%%%%%%%%%%%%%%%%%%%%%%%%%%%%%%%%%%%%%%%%%%%%%%%%%%%%%%%%%%%%%%%%%%%%%%%%%%%%%%%%%%%%%%%
{\textcolor{sectionTitleBlue}{\section{Der amputierte Dipol}}}\index{amputierter Dipol}\label{Korrektur22}
Um die Singularit\"{a}t in Punktlagern besser zu verstehen, betrachten wir die Einflussfunktion f\"{u}r die Querkraft $q_x$ in einer Platte, s. Abb. \ref{U337}, die ja durch eine Scherbewegung entsteht. Wenn der Aufpunkt frei im Innern liegt, dann halten sich die beiden Scherbewegungen die Balance und die Auslenkung der Platte bleibt, bis auf den Aufpunkt selbst, beschr\"{a}nkt. R\"{u}ckt jedoch der Aufpunkt in die N\"{a}he eines Punktlagers, dann wird diese Balance durch die St\"{u}tze gest\"{o}rt, und die Folge davon ist, dass bei Ann\"{a}herung an die St\"{u}tze die Auslenkung der Platte \"{u}ber alle Grenzen w\"{a}chst, s. Abb. \ref{U337} b.

Tendenziell sieht man den Effekt auch bei der \"{u}ber den Tr\"{a}ger wandernden Einflussfunktion f\"{u}r die Querkraft in Abb. \ref{1GreenF74}, S. \pageref{1GreenF74}. Je mehr der Aufpunkt sich dem Zwischenlager n\"{a}hert, desto mehr wird die Einflussfunktion nach oben gedr\"{u}ckt, ger\"{a}t die Balance zwischen links und rechts aus dem Lot und am Ende m\"{u}ndet die Scherbewegung in einem einseitigen Versatz, einem einseitigen {\em uplift\/}. Nur ist es so, dass bei der Platte die Spitze dieses {\em uplifts\/} unendlich weit \"{u}ber der St\"{u}tze liegt.

Diese Argumentation k\"{o}nnen wir nun direkt auf die Scheibe \"{u}bertragen, denn die Einflussfunktion f\"{u}r die Spannungen $\sigma_{ij}$ werden auch durch solche Scherbewegungen erzeugt, s. Abb. \ref{U294}.  Numerisch sind es vier Kr\"{a}fte, die dort die Einflussfunktion f\"{u}r $\sigma_{yy}$ erzeugen.

Liegt der Aufpunkt in dem Element mit dem Lagerknoten, dann steht es 2:1 f\"{u}r die nach oben treibenden Kr\"{a}fte, d.h. {\em zwei\/} Knotenkr\"{a}fte dr\"{u}cken nach oben, aber nur {\em eine\/} Knotenkraft dr\"{u}ckt nach unten, weil die Knotenkraft im Lager ausf\"{a}llt. So gelingt es also den $f_i$ die Oberkante der Scheibe in der Grenze in \glq den Himmel\grq{} zu verschieben. In den Nachbarelementen wirken hingegen,  s. Abb. \ref{U295}, alle {\em vier = zwei + zwei\/} Kr\"{a}fte gleichzeitig und halten so die Balance mit der Konsequenz, dass die Auslenkung (= Spannung) endlich bleibt.

Um die Tendenz $\sigma_{ij} \to \infty$ auch statisch zu verstehen, denken wir uns der Einfachheit halber das Element als eine kleine Kreisscheibe mit einem Radius $R$. Die Elementverzerrungen aus der Verschiebung des Lagerknotens verhalten sich wie
\begin{align}
\delta \varepsilon_{ij} = O( \frac{1}{R})\,.
\end{align}
(Macht man den Durchmesser $2\,R$ eines Zeltes kleiner, beh\"{a}lt aber die H\"{o}he 1 bei, dann wird die Neigung der Zeltbahn gr\"{o}{\ss}er).
%---------------------------------------------------------------------------------
\begin{figure}
\centering
\if \bild 2 \sidecaption[t] \fi
{\includegraphics[width=1.0\textwidth]{\Fpath/U86}}
\caption{Je kleiner die vier Elemente um die Kraft herum werden, um so gr\"{o}{\ss}er m\"{u}ssen die Spannungen werden, um dieselbe Knotenkraft $f_i$ auf schrumpfender Fl\"{a}che zu erzeugen}
\label{U86}%
\end{figure}%
%---------------------------------------------------------------------------------
Sinngem\"{a}{\ss} gilt daher
\begin{align}
 \int_{\Omega_e} \sigma_{ij}\,\delta \varepsilon_{ij}\,d\Omega \sim\int_0^{\,2\,\pi} \int_0^{\,R}   \sigma_{ij}\,\frac{1}{R}\,r\,dr\,d\Np = \int_0^{\,2\,\pi} \sigma_{ij} \,\frac{1}{2}\,R\,d\Np\,,
\end{align}
und daher muss sich $\sigma_{ij}$ wie $1/R$ verhalten, damit in der Grenze, $R \to 0$, die Knotenkraft $f_i$ \"{u}brig bleibt
\begin{align}
\lim_{R \to 0} \int_{\Omega_e} \sigma_{ij}\,\delta \varepsilon_{ij}\,d\Omega = f_i\,.
\end{align}

\begin{remark}
Der Vollst\"{a}ndigkeit halber sei noch erw\"{a}hnt, dass Linien\-lager\index{Linienlager $3-D$} im $3-D$ dasselbe Schicksal erleiden, wie Punktlager bei Scheiben, weil die Spannungen in einem Linienlager (= unendlich feiner Draht) das Material zum Flie{\ss}en bringen. Aus diesem Grund kann man theoretisch auch kein Linienlager um ein oder zwei Zentimeter senken. In der Praxis geht es nat\"{u}rlich schon, weil die finiten Elementen weit weg sind von solchen Fallstricken der Elastizit\"{a}tstheorie.
\end{remark}

%---------------------------------------------------------------------------------
\begin{figure}
\centering
\if \bild 2 \sidecaption[t] \fi
{\includegraphics[width=1.0\textwidth]{\Fpath/U417}}
\caption{LF Eigengewicht: Die adaptive Verfeinerung in der N\"{a}he der Punktlager m\"{u}ndet im Grenzfall $h \to 0$ theoretisch in Einzelkr\"{a}ften. Damit w\"{u}rde die Scheibe ihren Halt verlieren, weil die unendlich gro{\ss}en Spannungen das Material zum Flie{\ss}en bringen w\"{u}rden}
\label{U417}%
\end{figure}%
%---------------------------------------------------------------------------------
\vspace{-1cm}
%%%%%%%%%%%%%%%%%%%%%%%%%%%%%%%%%%%%%%%%%%%%%%%%%%%%%%%%%%%%%%%%%%%%%%%%%%%%%%%%%%%%%%%%%%%%%%%%%%%
{\textcolor{sectionTitleBlue}{\section{Einzelkr\"{a}fte als Knotenkr\"{a}fte}}}\index{Einzelkr\"{a}fte als Knotenkr\"{a}fte}
Eine Einzelkraft $ P$ \"{u}bersetzt das FE-Programm in eine \"{a}quivalente Knotenkraft $f_i = P \cdot 1$\,[F$\cdot$ L]. Eine solche \"{a}quivalente Knotenkraft $f_i$ ist f\"{u}r ein FE-Programm zun\"{a}chst nur ein Signal, dass in der N\"{a}he des Knotens Lasten vorhanden sind, die bei einer virtuellen Verr\"{u}ckung des Knotens die Arbeit $f_i $ leisten, s. Abb. \ref{U86}. Sollten die $f_i$ in den Nachbarknoten null sein, dann kann es nur so sein, dass es wirklich eine Einzelkraft ist, die genau in dem Knoten angreift.

Wenn nun der Benutzer die Elemente immer kleiner macht und weiterhin darauf beharrt, dass in dem Knoten -- und nur in diesem einen Knoten -- eine \"{a}quivalente Knotenkraft $f_i $ steht, dann ist das f\"{u}r das FE-Programm ein Signal, dass es die Spannungen in der N\"{a}he dieses Knotens tendenziell unendlich gro{\ss} machen muss, weil sonst die Bilanz
\begin{align}
 \int_{\Omega_\square} \sigma_{ij}\,\delta \varepsilon_{ij} \,d\Omega = f_i
\end{align}
nicht einzuhalten ist, s. Abb. \ref{U417}.

Die Verzerrungen $\delta \varepsilon_{ij}$ kommen aus der Einheitsverschiebung des Knotens in Richtung der Kraft und weil sie nur auf den Elementen nicht null sind, auf denen der Knoten liegt, muss immer weniger Gebiet immer gr\"{o}{\ss}ere Spannungen $\sigma_{ij}$ und Verzerrungen $\delta \varepsilon_{ij}$ produzieren. Dass {\em beide\/} gegen $\infty$ gehen m\"{u}ssen, haben wir oben gezeigt. Auch so kommen singul\"{a}re Spannungen in die finiten Elemente hinein. Sie sind in dieser Situation eine \glq Schutzma{\ss}nahme\grq{}, um bei immer kleiner werdenden Elementen am Ziel $f_i$ festhalten zu k\"{o}nnen.

Verfeinert man ein Netz in Gegenwart von Fl\"{a}chenkr\"{a}ften, dann werden die $f_i$ kleiner und  eine Verfeinerung ist daher unsch\"{a}dlich. Hier ist es aber so, dass $f_i$ seine H\"{o}he {\em beibeh\"{a}lt\/} und so erzwingt $h \to 0$ die Reaktion $\sigma \to \infty$.

%%%%%%%%%%%%%%%%%%%%%%%%%%%%%%%%%%%%%%%%%%%%%%%%%%%%%%%%%%%%%%%%%%%%%%%%%%%%%%%%%%%%%%%%%%%%%%%%%%%
\textcolor{sectionTitleBlue}{\section{Vorverformungen}}\index{Vorverformungen}\label{Korrektur42}
Wir wollen auch noch kurz erl\"{a}utern, wie Vorverformungen bei St\"{a}ben, die nach Theorie II. Ordnung gerechnet werden, im Rahmen der FEM behandelt werden.

 \glq {\em Bei Ansatz von Vorverformungen bedeutet Theorie II. Ordnung die Formulierung des Gleichgewichts am {\em gesamtverformten\/} System, wobei Gesamtverformung = Vorverformung + Lastverformung ist'\/}, \cite{Rubin} S. 77. Wir schreiben das als
\begin{align}
w(x) = s(x) + w_{L}(x) \qquad s\,\text{wie \glq Schlangenlinie\grq{}}
\end{align}
 Die so zweigeteilte Biegelinie muss der Differentialgleichung
\begin{align}
EI\,w^{IV}(x) + P\,w''(x) = p(x) \qquad  P = |P| \,\,\text{als Druckkraft}
\end{align}
oder
\begin{align}
EI\,w_{L}^{IV}(x) + P\,w_{L}''(x) = p(x) - (EI\,s^{IV}(x) + P\,s''(x))
\end{align}
gen\"{u}gen. Die Vorverformung wird durch ihre Interpolierende $s_I$ ersetzt
\begin{align}\label{Eq86}
s_I(x) = \sum_j s_j\,\Np_j(x)\,,
\end{align}
wobei die $s_j$ die Knotenwerte von $s(x)$ sind (Durchbiegungen und Verdre\-hungen -- Hermite Interpolation!). Weil die {\em shape functions\/} elementweise homogene L\"{o}sungen sind, $EI\,\Np_i^{IV} = 0$, reduziert sich das in jedem Element auf
\begin{align}
EI\,w_{L}^{IV}(x) + P\,w_{L}''(x) = p(x) - P\,s_{I}''(x)\,.
\end{align}
Zu den \"{a}quivalenten Knotenkr\"{a}ften aus der Last sind also noch die \"{a}quivalenten Knotenkr\"{a}fte der Interpolierenden zu addieren. Auf jedem Element gilt (partielle Integration)
\begin{align}
\text{\normalfont\calligra G\,\,}_e(s_I,\Np_i) = \int_0^{\,l_e} -P\,s_I''\,\Np_i\,dx + [P\,s_I'\,\Np_i]_0^{l_e} - P \int_0^{\,l_e} s_I'\,\Np_i'\,dx = 0\,.
\end{align}
Nun hat die Hermite-Interpolierende $s_I$ stetige erste Ableitungen in den Knoten und die $\Np_i(x)$ sind stetig in den Knoten, so dass sich bei der Summation \"{u}ber die Elemente
die Randarbeiten in den Innenknoten wegheben\footnote{An freien Enden verbleibt ein zus\"{a}tzlicher Anteil $P\,s_I' \cdot \Np_i$ der zur Knotenkraft $f_{si}$ zu addieren ist}
\begin{align}
\sum_e \text{\normalfont\calligra G\,\,}_e(s_I,\Np_i) = \int_0^{\,l} -P\,s_I''\,\Np_i\,dx  - P \int_0^{\,l} s_I'\,\Np_i'\,dx = 0\,,
\end{align}
was mit (\ref{Eq86}) gerade das Resultat
\begin{align}
f_{si} - \sum_j P \int_0^{\,l} \Np_i'\,\Np_j'\,dx\,s_j =  f_{si} - \sum_j\,G_{ij}\,s_j = 0\qquad G_{ij} = P\,\int_0^{\,l} \Np_i'\,\Np_j'\,dx
\end{align}
ergibt. Man erh\"{a}lt also den Vektor der \"{a}quivalenten Knotenkr\"{a}fte aus der Vorverformung, wenn man die {\em geometrische Steifigkeitsmatrix\/}, (\ref{IIN}) S. \pageref{IIN}, mit den Knotenwerten $\vek s$ der Vorverformung multipliziert
\begin{align}
\vek f_s = \vek G\,\vek s\,,
\end{align}
und mit Vorverformungen l\"{o}st man das System $\vek K_{II}\,\vek u = \vek f + \vek f_s$. Hierbei ist $\vek K_{II}$ die Steifigkeitsmatrix nach Theorie II. Ordnung, die aber meist durch die Steifigkeitsmatrix $\vek K_I$ nach Theorie I. Ordnung und die geometrische Steifigkeitsmatrix $\vek G$ ersetzt wird
\begin{align}
\vek K_{II} \simeq \vek K_I + \vek G\,.
\end{align}


%%%%%%%%%%%%%%%%%%%%%%%%%%%%%%%%%%%%%%%%%%%%%%%%%%%%%%%%%%%%%%%%%%%%%%%%%%%%%%%%%%%%%%%%%%%%%%%%%%%
{\textcolor{sectionTitleBlue}{\section{Die Grenzen von FE-Einflussfunktionen}}}\index{Grenzen von FE-Einflussfunktionen}
Wenn wir die Einflussfunktion f\"{u}r die zweite Ableitung $u''(x)$ der L\"{a}ngsverschiebung eines Stabes aufstellen wollen, und mit der Berechnung der \"{a}quivalenten Knotenkr\"{a}ften $j_i$ beginnen, dann merken wir schnell, dass alle
\begin{align}
j_i = \Np_i''(x) = 0
\end{align}
null sind und wegen $\vek K\vek g = \vek 0$ daher auch die Knotenverschiebungen $g_i$ der Einflussfunktion, d.h. die FE-Einflussfunktion f\"{u}r die zweite Ableitung ist identisch null
\begin{align}
G_2^h(y,x ) = 0\,.
\end{align}
Das ist immer so. Man kann mit finiten Elementen Einflussfunktionen nur f\"{u}r Ableitungen bis zu der Ordnung der Ansatzfunktionen berechnen
\begin{align}
\text{Stab} \qquad \Np_i(x) &= \text{linear} &&\qquad \rightarrow \qquad \text{max $u'(x)$}\\
\text{Balken} \qquad \Np_i(x) &= \text{kubisch} &&\qquad \rightarrow \qquad \text{max $u'''(x)$}\,.
\end{align}
Die h\"{o}heren Ableitungen kennen die finiten Elemente nicht oder, anders gesagt, sie haben keine Vorstellung davon, dass es so etwas geben k\"{o}nnte.\\

Das stimmt im \"{u}brigen mit der $h$-{\em Vertauschungsregel\/}\index{$h$-Vertauschungsregel} \"{u}berein, s. S. \pageref{Eq56},
\begin{align}
J_h(u) = J(u_h)\,.
\end{align}
Denn sei $J(u) = u''(x)$, und $u_h$ eine lineare FE-L\"{o}sung, dann ist die zweite Ableitung null, $J(u_h) = 0$, und daher muss auch $J_h(u) = 0$ null sein -- f\"{u}r alle $u$, also etwa alle Polynome beliebig hoher Ordnung. Das geht nur so, dass $G_2^h(y,x)$ identisch null ist, was eben bedeutet, dass man eine Einflussfunktion f\"{u}r zweite Ableitungen mit linearen Elementen nicht darstellen kann.

%Man kann es auch anders sagen: Ein FE-Programm reduziert alle Lasten in die Knoten, es kennt also nur Knotenkr\"{a}fte und Knotenmomente. Im Feld gibt es keine Belastung, keine Streckenlast $p$ und deswegen ist die vierte Ableitung zwischen den Knoten null, $EI\,w_h^{IV} = 0$. Warum eine Einflussfunktion f\"{u}r $w_h^{IV}$ konstruieren, wenn man wei{\ss}, dass die Funktion null ist?

%Bei Fl\"{a}chentragwerken gilt sinngem\"{a}{\ss} dasselbe, wenn auch die Situation etwas komplizierter wird, wenn man h\"{o}here Ans\"{a}tze benutzt.

%Mit quadratischen Scheibenelementen kann man schon Einflussfunktionen f\"{u}r zweite Ableitungen berechnen, weil die zugeh\"{o}rigen \"{a}quivalenten Knotenkr\"{a}fte nicht null sind. Nur ist die Frage, was man mit diesen Einflussfunktionen will? Man k\"{o}nnte mit ihnen den FE-Lastfall $p_h$ punktweise berechnen, obwohl es viel einfacher w\"{a}re, direkt \"{u}ber die FE-L\"{o}sung $\vek u_h$ zu gehen.


%%%%%%%%%%%%%%%%%%%%%%%%%%%%%%%%%%%%%%%%%%%%%%%%%%%%%%%%%%%%%%%%%%%%%%%%%%%%%%%%%%%%%%%%%%%%%%%%%%%
{\textcolor{sectionTitleBlue}{\section{Bemerkung zu den finiten Elementen}}}
Die finiten Elemente haben eine beispiellose Erfolgsgeschichte hinter sich, und der Grund ist, wie wir meinen, dass das Konzept des finiten Elements in nat\"{u}rlicher Weise die Kr\"{a}fte freigelegt hat, die in dem Begriff der Funktion schlummern. Wir wollen hier einen Erkl\"{a}rungsversuch wagen.

Jede $C^1$-Funktion \"{u}ber einem Intervall $(0,l) $ besitzt die Integraldarstellung
\begin{align}
u(x) = u(0) + \int_{0}^{x}\,u'(y)\cdot 1\,dy = \int_{\Gamma} \ldots + \int_{\Omega} \ldots
\end{align}
-- das ist einfach nur partielle Integration -- und jede $C^2$-Funktion \"{u}ber einem Gebiet $\Omega$ besitzt die Integraldarstellung
\begin{align} \label{Eq192}
u(\vek x) = &\int_{\Gamma} [g(\vek y, \vek x) \,\frac{\partial u(\vek y)}{\partial n} - \frac{\partial g(\vek y, \vek x)}{\partial n}\,u(\vek y)]\,ds_{\vek y} + \int_{\Omega} g(\vek y, \vek x)\,(- \Delta u(\vek y))\,\,d\Omega_{\vek y}\,.
\end{align}
Der Kern $g(\vek y, \vek x) = -1/(2\,\pi) \ln |\vek x - \vek y|$ ist die Fundamentall\"{o}sung der Laplacegleichung, $- \Delta g = \delta(\vek y - \vek x)$. (Der Kern oben ist die {\em Heaviside-function\/} $g = 1$ bis zum Punkt $x$)\index{Heaviside-function}.

Eine $C^2$-Funktion in einem Gebiet $\Omega $ ist also eindeutig durch ihren Randwert $u$, ihre \glq Spur\grq, und die Normalableitung $\partial u/\partial n$ auf dem Rand $\Gamma$ und die \glq Last\grq\ im Feld $\Omega$, die Summe der zweiten Ableitungen $\Delta u = u,_{y_1 y_1} + u,_{y_2 y_2}$, bestimmt.

Der Kern, der diese Daten zum Leben erweckt, aus ihnen $u(\vek x)$ generiert, und somit auch bestimmt, wie sich $u(\vek x)$ \"{a}ndert, wenn man sich von einem Punkt $\vek x_1$ zu einem Punkt $\vek x_2$ bewegt, ist in der Ebene der $\ln r$ und im Raum der Abstand $r = |\vek y - \vek x|$ selbst vom Aufpunkt $\vek x$ zu den Daten in $\vek y$. (Woher der Unterschied?)

%Der Abstand $r = |\vek y - \vek x|$, genauer der {\em Logarithmus naturalis\/}, bestimmt, wie sich eine Funktion \"{a}ndert, wenn man sich von einem Punkt $\vek x_1$ zu einem Punkt $\vek x_2$ bewegt.

Wir halten die Formel (\ref{Eq192}) f\"{u}r den Schl\"{u}ssel zur Diffe\-ren\-tial- und Integralrechnung: Gebiet, Rand und Funktion bilden mathematisch eine Einheit und die finiten Elementen sind die logische Umsetzung dieser Idee -- deswegen sind sie so erfolgreich
\begin{align}\label{Eq191}
\text{Finites Element} = \text{Gebiet} + \text{Rand} + \text{Funktion}\,.
\end{align}
Das hat der Bauingenieur {\em Clough\/} instinktiv richtig erfasst und so ist er zu den finiten Elementen gekommen. Mathematiker oder Elektroingenieure denken nicht in {\em shapes\/}, sie h\"{a}tten die finiten Elemente nicht erfinden k\"{o}nnen. Ihnen fehlt -- man verzeihe uns das Vorurteil -- das intuitive Verst\"{a}ndnis f\"{u}r diesen Zusammenhang, das einen Bauingenieur oder Maschinenbauer auszeichnet.

Es ging nicht darum, $\sin (x)$ und $\cos (x)$ durch kurze H\"{u}tchen-Funktion zu ersetzen -- das haben auch andere Autoren vorgeschlagen --  sondern das Bauteilkonzept, die Idee \glq alles in einem\grq, \glq {\em tutto insieme\/}\grq, das wie nat\"{u}rlich aus (\ref{Eq191}) erw\"{a}chst, ist die eigentlich Idee hinter den finiten Elementen und von dieser Idee geht die Faszination der finiten Elemente aus. 