%%%%%%%%%%%%%%%%%%%%%%%%%%%%%%%%%%%%%%%%%%%%%%%%%%%%%%%%%%%%%%%%%%%%%%%%%%%%%%%%%%%%%%%%%%%%%%%%%%%
\textcolor{chapterTitleBlue}{\chapter{Steifigkeits\"{a}nderungen und Reanalysis}}}\label{Steifigkeits\"{a}nderungen}
%%%%%%%%%%%%%%%%%%%%%%%%%%%%%%%%%%%%%%%%%%%%%%%%%%%%%%%%%%%%%%%%%%%%%%%%%%%%%%%%%%%%%%%%%%%%%%%%%%%

Das Thema in diesem Kapitel sind lokale Steifigkeits\"{a}nderungen in einzelnen Elementen, die sinngem\"{a}{\ss} zu einem neuen Satz von Gleichungen f\"{u}hren
 \begin{align}
(\vek K�+ \vek  \Delta \vek K)\,(\vek u + \vek \Delta \vek u) = \vek  f
\end{align}
und wir wollen die Effekte studieren, die solche Steifigkeits\"{a}nderungen in einem Tragwerk bewirken. Weil wir den Verschiebungsvektor $\vek u_c = \vek u + \vek \Delta \vek u$ des modifizierten Tragwerkes mittels der Matrix $\vek K$ des urspr\"{u}nglichen Tragwerkes
\begin{align}
\vek K\,\vek u = \vek f
\end{align}
berechnen, nennt man das auch {\em Reanalysis\/}\index{Reanalysis}.

In der Literatur werden viele Techniken diskutiert, mit denen sich solche \glq St\"{o}rungsrechnungen\grq{} numerisch behandeln lassen, s. z.B. \cite{Haug}. Wir konzentrieren uns hier aber auf die M\"{o}glichkeiten, die die Einflussfunktionen bieten, d.h. wir konzentrieren uns auf die Inverse $\vek K^{-1}$, denn Einflussfunktionen und Inverse sind Synonyme.

Die wesentliche Einsicht ist es, dass der neue Verschiebungsvektor $\vek u_c$ mit der urspr\"{u}nglichen Steifigkeitsmatrix berechnet werden kann, wenn man zur rechten Seite einen Vektor $\vek f^+$ addiert\index{$\vek f^+$}
\begin{align}
\vek K\,\vek u_c = \vek f + \vek f^+
\end{align}
und dass dieser Vektor $\vek f^+$ zu allen Starrk\"{o}rperbewegungen $\vek u_0 = \vek a + \vek x \times \vek b$ orthogonal ist, was impliziert, dass die Effekte von lokalen Steifigkeits\"{a}nderungen in der Regel rasch abklingen.

%----------------------------------------------------------------------------------------------------------
\begin{figure}[tbp]
\centering
\if \bild 2 \sidecaption \fi
\includegraphics[width=0.7\textwidth]{\Fpath/U230}
\caption{Steifigkeits\"{a}nderung} \label{U230}
\end{figure}%
%----------------------------------------------------------------------------------------------------------

%%%%%%%%%%%%%%%%%%%%%%%%%%%%%%%%%%%%%%%%%%%%%%%%%%%%%%%%%%%%%%%%%%%%%%%%%%%%%%%%%%%%%%%%%%%%%%%%%%%
\textcolor{sectionTitleBlue}{\section{Ein erster Versuch}}

Wir beginnen mit einem einfachen Beispiel, einem Stab, dessen L\"{a}ngsverschiebung $u(x)$ ja der Differentialgleichung $- EA\,u''(x) = p(x)$ gen\"{u}gt. Wenn man an dem Stab in Abb. \ref{U230} mit einer Kraft $f = 1 $ kN zieht, dann verl\"{a}ngert er sich, bei einer angenommenen L\"{a}ngssteifigkeit von $EA = 1$ kN, um den Betrag
\begin{align}
u = \frac{f \cdot l}{EA} = \frac{1 \cdot 2}{1} = 2\,\text{m}.
\end{align}
Wenn wir die L\"{a}ngssteifigkeit $EA$ verdoppeln, $EA = 2$, dann halbiert sich die L\"{a}ngsverschiebung
\begin{align}
u_c = \frac{f \cdot l}{EA} = \frac{1 \cdot 2}{2} = 1\,\text{m}.
\end{align}
Wir stellen uns jetzt die folgende Frage: Wie muss man die Zugkraft $f$ \"{a}ndern, $f \to f_c$, damit an dem urspr\"{u}nglichen Stab unter der Wirkung der modifizierten Zugkraft $f_c = f + f^+$ sich dieselbe  L\"{a}ngsverschiebung einstellt, wie an dem Stab mit der doppelten Steifigkeit?

Das f\"{u}hrt auf die Gleichung
\begin{align}
\frac{(f + f^+) \cdot l}{EA} = \frac{f \cdot l}{2 \cdot EA}= u_c
\end{align}
oder
\begin{align}
f^+ = -\frac{f}{2}\,.
\end{align}
Man darf also an dem urspr\"{u}nglichen System nur mit der halben Kraft ziehen,
\begin{align}
f_c = f + f^+ = f - \frac{f}{2} = \frac{f}{2}\,,
\end{align}
wenn man denselben Effekt haben will, wie an dem versteiften Stab.

Dieses Beispiel ist nat\"{u}rlich sehr einfach, aber wir erkennen hier schon die Grundidee.
%----------------------------------------------------------------------------------------------------------
\begin{figure}[tbp]
\centering
\if \bild 2 \sidecaption \fi
\includegraphics[width=0.65\textwidth]{\Fpath/U231}
\caption{Steifigkeits\"{a}nderung} \label{U231}
\end{figure}%
%----------------------------------------------------------------------------------------------------------
Steifigkeits\"{a}nderungen f\"{u}hren dazu, dass sich die Steifigkeitsmatrix \"{a}ndert, $\vek K \to \vek K + \vek \Delta \vek K$, und damit auch der Vektor der Knotenverschiebungen, $\vek u \to \vek u_c$,
\begin{align}
(\vek K + \vek \Delta \vek K)\,\vek u_c = \vek f\,.
\end{align}
Durch einfaches Umstellen sieht man, dass diese Gleichung mit dem System
\beq
\vek K\,\vek u_c = \vek f - \vek \Delta\,\vek K\,\vek u_c = \vek f + \vek f^+
\eeq
identisch ist. Der neue Vektor $\vek u_c$ kann also als L\"{o}sung des urspr\"{u}nglichen Systems gelten, wenn man zu der rechten Seite $\vek f$ den Vektor
\beq
\vek f^+ := - \vek \Delta\,\vek K\,\vek u_c
\eeq
addiert.

%%%%%%%%%%%%%%%%%%%%%%%%%%%%%%%%%%%%%%%%%%%%%%%%%%%%%%%%%%%%%%%%%%%%%%%%%%%%%%%%%%%%%%%%%%%%%%%%%%%
\textcolor{sectionTitleBlue}{\section{Zweites Beispiel}}
Bevor wir das weiter diskutieren, wollen wir noch ein zweites, einfaches Beispiel studieren. Der Stab in Abb. \ref{U231} ist in zwei Elemente mit der glei\-chen L\"{a}ngssteifigkeit $EA = 1$ kN unterteilt und eine Kraft $f_2 = 10$ kN zieht an seinem rechten Ende
\begin{align} \label{Eq42}
\left[ \barr {r @{\hspace{4mm}}r @{\hspace{4mm}}r
@{\hspace{4mm}}r @{\hspace{4mm}}r}
      2 & -1  \\
      -1 & 1 \\
     \earr \right]\left [\barr{c}  u_1 \\  u_2\earr \right ]
=  \left [\barr{c}  0 \\  10\earr \right ]\,.
\end{align}
Dieses System hat die L\"{o}sung $u_1 = 10\,\text{m}, u_2 = 20$ m.

Nun verdoppeln wir die Steifigkeit des rechten Elementes, $EA \to 2 EA$,
\begin{align}
\left[ \barr {r @{\hspace{4mm}}r @{\hspace{4mm}}r
@{\hspace{4mm}}r @{\hspace{4mm}}r}
      3 & -2  \\
      -2 & 2 \\
     \earr \right]\left [\barr{c}  u_1^c \\  u_2^c\earr \right ]
=  \left [\barr{c}  0 \\  10\earr \right ]
\end{align}
und wir erhalten so die L\"{o}sung $u_1^c = 10, u_2^c = 15$ m.

Wir fragen jetzt, wie muss man die rechte Seite $\vek f$ des Systems (\ref{Eq42}) modifizieren, damit das urspr\"{u}ngliche System dieselbe L\"{o}sung hat,
\begin{align}
\left[ \barr {r @{\hspace{4mm}}r @{\hspace{4mm}}r
@{\hspace{4mm}}r @{\hspace{4mm}}r}
      2 & -1  \\
      -1 & 1 \\
     \earr \right]\left [\barr{c}  u_1^c \\  u_2^c\earr \right ]
=  \left [\barr{c}  0 \\  10\earr \right ] + X\, \left[\barr{r}  1 \\  -1\earr \right ]\,.
\end{align}
Der letzte Vektor auf der rechten Seite ist der Vektor $\vek f^+$. Wir haben ihn von Anfang an so konstruiert, dass er ein Gleichgewichtsvektor ist, sich seine Komponenten gegenseitig aufheben.
Die Rechnung ergibt in der Tat, dass diese Annahme richtig war, denn mit $X = 5 $ ergibt sich eine L\"{o}sung.

Die gesuchten Knotenkr\"{a}fte lauten also
\begin{align}
\vek f + \vek f^+ = \left [\barr{c}  0 \\  10\earr \right ] + \left [\barr{r}  5 \\  -5\earr \right ] = \left [\barr{c}  5 \\  5\earr \right ]\,,
\end{align}
und wir \"{u}berzeugen uns leicht, dass die neuen Knotenverschiebungen $u_i^c $ genau die L\"{o}sungen des so modifizierten urspr\"{u}nglichen Systems sind
\begin{align}
\left[ \barr {r @{\hspace{4mm}}r @{\hspace{4mm}}r
@{\hspace{4mm}}r @{\hspace{4mm}}r}
      2 & -1  \\
      -1 & 1 \\
     \earr \right]\left [\barr{c}  10 \\  15\earr \right ]
=  \left [\barr{c}  5 \\  5\earr \right ]\,.
\end{align}

%%%%%%%%%%%%%%%%%%%%%%%%%%%%%%%%%%%%%%%%%%%%%%%%%%%%%%%%%%%%%%%%%%%%%%%%%%%%%%%%%%%%%%%%%%%%%%%%%%%
\textcolor{sectionTitleBlue}{\section{Strategie}}
Diese Ergebnisse nehmen wir nun zum Anlass unsere Strategie wie folgt zu formulieren: Wir \"{a}ndern nicht die Steifigkeitsmatrix, sondern die rechte Seite.
Wir suchen also eine Erg\"{a}nzung $\vek f^+$ zu dem Knotenkraftvektor $\vek f $ so, dass $\vek  f+ \vek f^+$ an dem System $\vek K $ dieselben Knotenverschiebungen verursacht, wie $\vek f $ an dem System $\vek K_c$.

Die Schwierigkeit dabei ist nat\"{u}rlich, dass der Vektor $\vek f^+ = -\vek \Delta\,\vek K\,\vek u_c$ von dem neuen Vektor $\vek u_c$ abh\"{a}ngt, den wir ja nicht kennen. (N\"{a}herungsweise kann man f\"{u}r $\vek u_c$ den Vektor $\vek u$ setzen).
Aber es geht uns nicht darum, eine neue Rechenmethode zur Nachverfolgung von Steifigkeits\"{a}nderungen in die Praxis einzuf\"{u}hren, sondern es geht uns prim\"{a}r um das statische Verst\"{a}ndnis.

Wir werden sehen, dass alle Steifigkeits\"{a}nderungen verstanden werden k\"{o}nnen, als die Addition von solchen {\em Gleichgewichts\-kr\"{a}ften\/} $\vek f^+$.

Das bedeutet: wenn $\vek u_0 = \vek a + \vek b \times \vek x$ eine Starrk\"{o}rperbewegung des Tragwerks ist, also eine Translation $\vek a$ plus einer m\"{o}glichen Drehung um eine Achse $\vek b$, dann ist die Arbeit der Kr\"{a}fte $\vek f^{+}$ null,
\begin{align}
\vek f^{+@T}\,\vek u_0 = 0\,.
\end{align}
Gleichgewichtskr\"{a}fte wie der Vektor $\vek f^+$ leisten also keine Arbeit auf Starrk\"{o}rperbewegungen. Daraus k\"{o}nnen wir, wie wir sehen werden, den folgenden Schluss  ziehen: \\

\hspace*{-12pt}\colorbox{highlightBlue}{\parbox{0.98\textwidth}{
Steifigkeits\"{a}nderungen sind in ihren Auswirkungen lokal begrenzt, weil sich die Wirkungen der Gleichgewichtskr\"{a}fte $\vek f^+$ in der Ferne aufheben.}}\\

%-----------------------------------------------------------------
\begin{figure}
\centering
\if \bild 2 \sidecaption \fi
{\includegraphics[width=0.6\textwidth]{\Fpath/U110}} \caption{Eine Steifigkeits\"{a}nderung $\vek K + \vek \Delta \vek K $ bedeutet, dass man ein Zusatzelement $\Omega_e^+$ mit der
 Steifigkeit $\vek \Delta \vek K $ an das Tragwerk anheftet, \cite{Ha6}}
\label{U110}%
\end{figure}%\\
%-----------------------------------------------------------------



%%%%%%%%%%%%%%%%%%%%%%%%%%%%%%%%%%%%%%%%%%%%%%%%%%%%%%%%%%%%%%%%%%%%%%%%%%%%%%%%%%%%%%%%%%%%%%%%%%%
\textcolor{sectionTitleBlue}{\section{Addition oder Subtraktion von Steifigkeiten}}

Die \"{A}nderung der Steifigkeit in einem Element kann man so deuten, dass man vor das urspr\"{u}ngliche Element ein zweites Element legt, dessen Steifigkeit gerade so gro{\ss} ist, dass die beiden Elemente zusammen die angezielte Steifigkeit haben, s. Abb. \ref{U110}.
%-----------------------------------------------------------------
\begin{figure}
\centering
\if \bild 2 \sidecaption \fi
{\includegraphics[width=0.8\textwidth]{\Fpath/U97}}
\caption{Zunahme und Abnahme der L\"{a}ngssteifigkeit $EA$ in dem ersten Element. Bei den Lastf\"{a}llen in der zweiten Reihe sind die L\"{a}ngsverformungen gleich den Lastf\"{a}llen in der ersten Reihe}
\label{U97}%
\end{figure}%
%-----------------------------------------------------------------

Das zus\"{a}tzliche Element muss nun durch Koppelkr\"{a}fte mit dem urspr\"{u}ng\-lichen Tragwerk synchron geschaltet werden, und die dazu n\"{o}tigen Koppelkr\"{a}fte sind gerade die Kr\"{a}fte $\vek f^+$. Das macht auch verst\"{a}ndlich, warum die Kr\"{a}fte $\vek f^+$ Gleichgewichtskr\"{a}fte sind, denn w\"{a}ren sie das nicht, dann w\"{u}rde das vorgeschaltete Element wegfliegen.

Dies gilt f\"{u}r eine Zunahme der Steifigkeit ebenso, wie f\"{u}r eine Abnahme der Steifigkeit. Wenn die Steifigkeit des Elementes gr\"{o}{\ss}er wird, dann haben die Koppelkr\"{a}fte $\vek f^+ $ die Tendenz, die Verformung des Elementes zu behindern, sie steifen sozusagen das Element aus.
%-----------------------------------------------------------------
\begin{figure}
\centering
\if \bild 2 \sidecaption \fi
{\includegraphics[width=0.8\textwidth]{\Fpath/U98}}
\caption{Zunahme und Abnahme der Biegesteifigkeit $EI$ in dem ersten Element. }
\label{U98}%
\end{figure}%
%-----------------------------------------------------------------
Umgekehrt, wenn die Steifigkeit in dem Element abnimmt, dann erh\"{o}hen die Koppelkr\"{a}fte $\vek f^+$ noch die Verformungen des Elementes, sie wirken wie zus\"{a}tzliche Lasten in den Knoten des Elementes, s. Abb. \ref{U97} und Abb. \ref{U98}.


%%%%%%%%%%%%%%%%%%%%%%%%%%%%%%%%%%%%%%%%%%%%%%%%%%%%%%%%%%%%%%%%%%%%%%%%%%%%%%%%%%%%%%%%%%%%%%%%%%%
\textcolor{sectionTitleBlue}{\section{Dipole und Monopole}}

Zwei gegengleiche Kr\"{a}fte $f_i^+ = \pm 1/h$, die \"{u}ber alle Grenzen wachsen, wenn ihr Abstand $h$ gegen null geht, bilden ein Dipol.

Bleiben die beiden gegengleichen Kr\"{a}fte hingegen auch im Grenzfall $h = 0$ endlich, dann nennen wir dies einen {\em Pseudo-Dipol\/}\index{Pseudo-Dipol}. Das Proton (+) und das Elektron (-) in einem Wasserstoffatom bilden einen solchen Pseudo-Dipol, und der Abstand der beiden entgegengesetzten Elementarladungen ist so klein, dass sich ihre Wirkungen auf eine Punktladung au{\ss}erhalb des Atomes praktisch aufheben.

In \"{a}hnlicher Weise stellen die Kr\"{a}fte $f_i^+$ Pseudo-Dipole dar, d.h. zu jeder aufw\"{a}rts gerichteten Kraft $f_i^+$ gibt es eine entgegengesetzt wirkende Kraft $f_j^+$, so dass die beiden Kr\"{a}fte $f^+$ aus der Ferne betrachtet einem Pseudo-Dipol gleichen, s. Abb. \ref{U113}.

Die Wirkung der Kr\"{a}fte $f_i^+$ auf irgendeinen Punkt $x$ des Tragwerks h\"{a}ngt davon ab, wie gro{\ss} die Laufzeitunterschiede von dem Punkt $x$ zu der Kraft $+f_i^+$ und der Gegenkraft $-f_i^+$ sind.
Wenn zwei Kr\"{a}fte $\pm f_i^+$ fast deckungsgleich sind, weil das Element $\Omega_e$ sehr klein ist, dann heben sich ihre Wirkungen nahezu auf, weil die Einflussfunktion sich auf dem winzigen Element kaum \"{a}ndert, $g' \simeq 0$.

Betrachten wir ein einfaches Beispiel. Ein Stabelement \"{a}ndere seine Steifigkeit, $EA_c = EA + \Delta EA$. In einem entfernten Element wird durch die Spreizung eines Punktes die Einflussfunktion f\"{u}r die Normalkraft in dem entfernten Element erzeugt.

Diese Einflussfunktion pflanzt sich nun bis zu dem Element $EA_c = EA + \Delta EA$ fort und wir beobachten jetzt, was dort passiert. Vereinbarungsgem\"{a}{\ss} wirken dort zwei Zusatzkr\"{a}fte $\pm f^+$, die den Effekt der Steifigkeits\"{a}nderung in dem Element nachbilden.
Am Anfang des Stabelementes mit der L\"{a}nge $l_e$ ziehe die Kraft $f_i^+$ nach links und am Ende ziehe eine gleichgro{\ss}e Kraft $f_{i + 1}^+$ nach rechts, und die Einflussfunktion f\"{u}r die Normalkraft
\begin{align}
G(y,x) = \sum_j g_j(x)\,\Np_j(y)
\end{align}
habe im linken Knoten den Wert $g_i$ und im rechten Knoten den Wert $g_{i + 1}$. Dann betr\"{a}gt der  Unterschied in der Normalkraft $N$, die wir als Funktional $N = J(u)$ lesen, also der Unterschied $N_{neu} - N_{alt} = N_c - N$,
\begin{align}
J(u_c) - J(u) = f_i^+ g_i - f_{i+1}^+ @g_{i+1} = f_i^+ \cdot (g_i - g_{i+1}) \simeq f_i^+ \cdot G'\cdot l_e\,.
\end{align}
Die Wirkung der Steifigkeits\"{a}nderung wird also nur dann merkbar sein, wenn die Einflussfunktion
in dem ge\"{a}nderten Element halbwegs merkbar ansteigt oder f\"{a}llt, $G' \gg 0$. Die Wirkung der $\pm f^+$ lebt also von der Differenz zwischen Elementanfang und Elementende, also kurz gesagt von $G'$.\\

\hspace*{-12pt}\colorbox{highlightBlue}{\parbox{0.98\textwidth}{Die Kr\"{a}ftepaare $\pm f_i^+$ registrieren die Unterschiede in den Einflussfunktionen (z.B.) am Elementanfang und Elementende, sie \glq differenzieren\grq{} die Einflussfunktionen. }}\\

Bei einem Balken registrieren die $\pm f^+$, es sind jetzt Momente, die Unterschiede in der ersten Ableitung der Einflussfunktion an den Balkenenden,
\begin{align}
J(u_c) - J(u) = f_i^+ g_i' - f_{i+1}^+ @g_{i+1}' = f_i^+ \cdot (g_i' - g_{i+1}') \simeq f_i^+ \cdot G''\cdot l_e
\end{align}
sie reagieren also auf die Gr\"{o}{\ss}e der zweiten Ableitung, der  Momente der Einflussfunktion in dem Element. (Der Balken sei fest gelagert).
%-----------------------------------------------------------------
\begin{figure}[tbp]
\centering
\includegraphics[width=0.9\textwidth]{\Fpath/U113}
\caption{Steifigkeits\"{a}nderung in einem Element und die zugeh\"{o}rigen Koppelkr\"{a}fte $\vek f_i^+$. Diese Kr\"{a}fte folgen den Hauptspannungsrichtungen (- - -)  und es sind Gleichgewichtskr\"{a}fte, die Pseudo-Dipolen gleichen.}
\label{U113}
\end{figure}%
%-----------------------------------------------------------------

%%%%%%%%%%%%%%%%%%%%%%%%%%%%%%%%%%%%%%%%%%%%%%%%%%%%%%%%%%%%%%%%%%%%%%%%%%%%%%%%%%%%%%%%%%%%%%%%%%%
\textcolor{sectionTitleBlue}{\section{Weggr\"{o}{\ss}en und Kraftgr\"{o}{\ss}en}}

Betrachten wir nun die \"{A}nderungen und die Rolle, die der Vektor $\vek f^+$ dabei spielt, etwas systematischer. Es sei zun\"{a}chst $J(u)$ eine Weggr\"{o}{\ss}e, also etwa $u(x)$ oder $w(x)$. Die Ergebnisse vor und nach der Steifigkeits\"{a}nderung lauten
\begin{align}
J(u) = \vek g^T \vek f = \vek u^T \vek j
\end{align}
bzw.
\begin{align}
J(u_c) = \vek g_c^T \vek f = \vek u_c^T \vek j_c\,.
\end{align}
Nun ist kein Unterschied zwischen den beiden Vektoren $\vek j$ und $\vek j_c$, weil in die Definition der Weggr\"{o}{\ss}en die Steifigkeiten nicht eingehen.
%-----------------------------------------------------------------
\begin{figure}[tbp]
\centering
\includegraphics[width=0.9\textwidth]{\Fpath/U383}
\caption{Steifigkeits\"{a}nderung in einem Stab unter einer Gleichstreckenlast, die $f_i^+$ sind antimetrisch, die EF-$u$ ist symmetrisch und die EF-$N$ ist antimetrisch. Man beachte aber, dass die genaue \"{A}nderung $N_c - N$ nicht einfach die $f_i^+$ mal den Knotenverschiebungen der Einflussfunktion EF-$N$ ist, s. (\ref{Eq137})}\label{Korrektur28}
\label{U383}
\end{figure}%
%-----------------------------------------------------------------

Ist z.B. $J(u) = u(x)$ die Verschiebung in einem Punkt $x$, dann ist
\begin{align}
\vek j = \{\Np_1(x), \Np_2(x), \ldots, \Np_n(x) \}^T = \vek j_c
\end{align}
und so folgt
\begin{align}
J(u_c) - J(u) &= \vek j^T (\vek u_c - \vek u) = \vek j^T\vek K^{-1}(\vek f + \vek f^+) - \vek j^T \vek K^{-1} \vek f\nn\\
 &= \vek j^T\,\vek K^{-1} \vek f^+ = \vek g^T \vek f^+\,.
\end{align}
Bei Kraftgr\"{o}{\ss}en ist das unter Umst\"{a}nden anders. Wenn der Aufpunkt $x$ auf dem Element liegt, dessen Steifigkeit sich \"{a}ndert, $EA \to EA_c$, dann sind, etwa im Fall $J(u) = EA\,u'(x)$, die Vektoren $\vek j$
\begin{align}
\vek j = \{EA\,\Np_1'(x), EA\, \Np_2'(x), \ldots, EA\,\Np_n'(x) \}^T
\end{align}
und
\begin{align}
\vek j_c = \{EA_c\,\Np_1'(x), EA_c\, \Np_2'(x), \ldots, EA_c\,\Np_n'(x) \}^T
\end{align}
nicht gleich\footnote{Nur die $\Np_i(x)$, die einen \glq Fu{\ss}\grq{} auf dem Element mit dem Aufpunkt $x$ haben, sind in dem Vektor $\vek j$ bzw. $\vek j_c$ nicht null, so dass diese Vektoren ziemlich leer sind. }, so dass nun das Ergebnis lautet
\begin{align}\label{Eq102}
J(u_c) - J(u) &= \vek j_c^T \vek K^{-1}(\vek f + \vek f^+) - \vek j^T \vek K^{-1}\vek f\nn \\
&= (\vek j_c - \vek j)^T \vek u + \vek j_c^T \vek K^{-1}\,\vek f^+  \,.
\end{align}
Auf der rechten Seite stehen die Korrekturterme, um von $J(u)$ auf $J(u_c)$ zu kommen, wobei man die erste Korrektur, um den Fehler in der Steifigkeit, $EA$ statt $EA_c$, zu beheben, an $\vek u$ vornehmen kann, und der zweite Term ist der Einfluss der Kr\"{a}fte $\vek f^+$ auf die Schnittgr\"{o}{\ss}en.

Setzen wir $\vek u^+ = \vek K^{-1}\,\vek f^+$\index{$\vek u^+$} und sei $\vek j_c = \alpha\,\vek j$, dann ist dasselbe wie
\begin{align}\label{Eq137}
J(u_c) - J(u) = \vek j_c^T (\vek u + \vek u^+) - \vek j^T\,\vek u = (\alpha -1)\,\vek g^T\,\vek f + \alpha\,\vek g^T\,\vek f^+\,.
\end{align}
Wenn der Punkt $x$ auf einem Element liegt, dessen Steifigkeit sich nicht \"{a}ndert, dann ist wegen $\vek j_c = \vek j$ die Formel dieselbe, wie bei den Weggr\"{o}{\ss}en
\begin{align}
J(u_c) - J(u) = \vek j^T\,\vek u^+ = \vek j^T\,\vek K^{-1} \vek f^+ = \vek g^T \vek f^+\,.
\end{align}
Bei dem Stab in Abb. \ref{U383} halbiert sich im zweiten Element die L\"{a}ngssteifigkeit von $EA = 1$ auf $EA_c = 0.5$. Die Belastung ist eine Gleichlast von 1 kN/m. Die Elementl\"{a}nge betr\"{a}gt $l_e = 1.0$ m. Die Matrizen lauten
\begin{align}
\vek K = \left[ \barr {r @{\hspace{4mm}}r @{\hspace{4mm}}r @{\hspace{4mm}}r} 2 & -1 & 0 & 0 \\ -1 & 2 & -1 & 0\\ 0 &-1 &2 &-1\\ 0 & 0 &-1 &2\earr \right]
\qquad \vek  \Delta \vek K = \left[ \barr {r @{\hspace{4mm}}r @{\hspace{4mm}}r @{\hspace{4mm}}r} -0.5 & 0.5 & 0 & 0 \\ 0.5 & -0.5 & 0 & 0\\ 0 &0 &0 &0\\ 0 & 0 &0 &0\earr \right]
\end{align}
und die Vektoren haben die Gestalt
\begin{align}
\vek u = \left[ \barr {r } 2.0 \\ 3.0 \\ 3.0 \\ 2.0 \earr \right]\quad \vek u_c = \left[ \barr {r } 1.833 \\3.500 \\3.333 \\2.167 \earr \right] \quad \vek f^+ = \left[ \barr {r } -0.833 \\ 0.833 \\ 0.0 \\ 0.0 \earr \right] \quad \vek j =\left[ \barr {r } -1.0 \\ 1.0 \\ 0.0 \\ 0.0 \earr \right]\,.
\end{align}
Es ist $\vek j_c = 0.5 \cdot \vek j$ wegen $EA_c = 0.5\,EA$.

Die Differenz der Normalkr\"{a}fte in der Mitte des zweiten Elements ergibt sich somit zu
\begin{align}
N_c - N = J(\vek u_c) - J(\vek u) = - 0.5\,\vek j^T\,\vek u + 0.5\,\vek j^T\,\vek K^{-1}\,\vek f^+ = -0.1665\,.
\end{align}
In der Praxis berechnet man $N_c - N$ nat\"{u}rlich direkt aus $\vek u_c$ und $\vek u$. Weil die Elemente linear sind ist $N_c = 0.5 \cdot (3.500 - 1.833) = 0.8335$ und $N = 1.0 \cdot (3.0 - 2.0) = 1.0 $. Es ging hier prim\"{a}r um den Nachweis, dass die obige Formel (\ref{Eq102}) das richtige Ergebnis liefert.

%%%%%%%%%%%%%%%%%%%%%%%%%%%%%%%%%%%%%%%%%%%%%%%%%%%%%%%%%%%%%%%%%%%%%%%%%%%%%%%%%%%%%%%%%%%%%%%%%%%
\textcolor{sectionTitleBlue}{\section{Symmetrie und Antimetrie}}\label{Korrektur26}
Die Einflussfunktionen f\"{u}r Verschiebungen bei einem Stab (Scheibe) werden durch Monopole erzeugt
und die Einflussfunktionen f\"{u}r Normalkr\"{a}fte (Spannungen) durch Dipole. Nun sind ja die $f_i^+$ antimetrisch und somit sollten die Effekte von Steifigkeits\"{a}nderungen bei den Spannungen ({\em antimetrisch $\times$ antimetrisch\/}) gr\"{o}{\ss}er sein als bei den Verschiebungen ({\em antimetrisch $\times$ symmetrisch\/}). Mit
\begin{align}
 \frac{N_c - N}{N} = \frac{0.8335 - 1.0}{1.0}= - 0.1665 \qquad \frac{u_c - u}{u} = \frac{2.667 - 2.5}{2.5} = 0.0666
\end{align}
best\"{a}tigt sich das bei diesem Beispiel.

Bei statisch bestimmten Systemen ist es aber gerade umgekehrt: Die Verformungen \"{a}ndern sich, aber die Kr\"{a}fte nicht. Der Grund ist, dass die $f_i^+$ Gleichgewichtskr\"{a}fte sind, die auf den st\"{u}ckweise linearen Einflussfunktionen (f\"{u}r $N, M, V$) in der Summe null Arbeit leisten. Die Einflussfunktionen f\"{u}r die Verformungen sind jedoch \glq krumme\grq{} Linien zu denen die $f_i^+$ im energetischen Sinne nicht orthogonal sind. Deswegen \"{a}ndern sich die Verformungen. \label{Korrektur8}

%-----------------------------------------------------------------
\begin{figure}[tbp]
\centering
\includegraphics[width=0.99\textwidth]{\Fpath/U428}
\caption{Einflussfunktion f\"{u}r eine Querkraft. In den Stielen des obersten Stockwerks ist der Verlauf praktisch konstant, was bedeutet, dass \"{A}nderungen $EI \pm \Delta EI$ in den Stielen sehr geringen Einfluss auf die Querkraft haben werden, weil die zugeh\"{o}rigen $f_i^+$ zu diesen Bewegungen orthogonal sind}\label{Korrektur34}
\label{U428}
\end{figure}%
%-----------------------------------------------------------------

%-----------------------------------------------------------------
\begin{figure}[tbp]
\centering
\includegraphics[width=0.99\textwidth]{\Fpath/U429}
\caption{Einflussfl\"{a}che f\"{u}r das Moment $m_{yy}$ einer zweiseitig gelagerten Decke. Die lange Seite ist eingespannt, die kurze gelenkig gelagert. Im rechten Teil verl\"{a}uft die Fl\"{a}che nahezu linear, so dass Steifigkeits\"{a}nderungen in diesem Teil einen geringen Einfluss auf das Moment $m_{yy}$ im Aufpunkt haben sollten}\label{Korrektur35}
\label{U429}
\end{figure}%
%-----------------------------------------------------------------
%%%%%%%%%%%%%%%%%%%%%%%%%%%%%%%%%%%%%%%%%%%%%%%%%%%%%%%%%%%%%%%%%%%%%%%%%%%%%%%%%%%%%%%%%%%%%%%%%%%
\textcolor{sectionTitleBlue}{\section{Das Abklingen der Effekte}}

Je weiter man sich vom Aufpunkt entfernt, um so \glq linearer\grq{} wird die Einflussfunktion, um so mehr dominieren die linearen Anteile in den Einflussfunktionen, d.h. der Vektor $\vek g$ der Knotenverschiebungen gleicht mit wachsendem Abstand mehr und mehr einem Vektor $\vek u_0= \vek a + \vek x \times \vek b$ und das bedeutet, weil der Vektor $\vek f^+$ orthogonal zu den Vektoren $\vek u_0$ ist, dass der Einfluss der Kr\"{a}fte $\vek f^+$ mit wachsendem Abstand vom Aufpunkt gegen null tendiert\\

\hspace*{-12pt}\colorbox{highlightBlue}{\parbox{0.98\textwidth}{
\beq
J(e) =  J(u_c) - J(u)  = \vek g^T\,\vek f^+ \simeq (\vek a + \vek x \times \vek b)^T\,\vek f^+ = 0\,.
\eeq
Dies ist der Grund, warum es m\"{o}glich ist, mit gemittelten Materialparametern genaue Ergebnisse zu erhalten.}}\\

\vspace{0.7cm}

Wenn $J(u)$ eine Kraftgr\"{o}{\ss}e ist und der Aufpunkt auf dem betroffenen Element liegt, dann ist, wie oben gezeigt, die Formel um die Korrektur $EA_c - EA$ auf dem Element zu erweitern
\begin{align}
J(e) = J(u_c) - J(u) &= \underbrace{(\vek j_c - \vek j)^T \vek u}_{Korrektur} + \vek j_c^T \vek K^{-1}\,\vek f^+\,.
\end{align}
Die erste Korrektur ist rein lokal, geschieht nur im Aufpunkt, w\"{a}hrend der zweite Term den Einfluss der Kr\"{a}fte $f_i^+$ erfasst, deren Effekte aber mit der Entfernung relativ rasch abklingen.


%%%%%%%%%%%%%%%%%%%%%%%%%%%%%%%%%%%%%%%%%%%%%%%%%%%%%%%%%%%%%%%%%%%%%%%%%%%%%%%%%%%%%%%%%%%%%%%%%%%
\textcolor{sectionTitleBlue}{\section{Die Bedeutung f\"{u}r die Praxis}}
Die Bedeutung dieser Ergebnisse f\"{u}r die Praxis liegt darin, dass sie erkl\"{a}ren, warum {\em Homogenisierungsmethoden\/}\index{Homogenisierungsmethoden} erfolgreich sind.

Beton setzt sich aus den unterschiedlichsten Kiessorten und Zementstein zusammen. Jedes Zuschlagskorn hat ja einen anderen Elastizit\"{a}tsmodul und daher m\"{u}ssten wir eigentlich jedes Zuschlagskorn durch ein eigenes finites Element modellieren. Stattdessen rechnen wir aber mit einem gemittelten Elastizit\"{a}tsmodul und erhalten durchaus glaubhafte Ergebnisse.
%-----------------------------------------------------------------
\begin{figure}[tbp]
\centering
\includegraphics[width=0.9\textwidth]{\Fpath/U114}
\caption{Eigengewicht und Kr\"{a}fte $f_i^+$. Die Scheibe wurde erst mit einem einheitlichen E-Modul $E = 1$ berechnet und dann wurde der E-Modul in den Elementen
zuf\"{a}llig, $0.5 < E_i < 1.5$, variiert, und es wurden die Kr\"{a}fte $f_i^+$ berechnet. Diese Kr\"{a}fte $f_i^+$ plus den Kr\"{a}ften $f_i$ aus dem Lastfall Eigengewicht
erzeugen in dem Modell $\vek K$ den Verschiebungsvektor $\vek u_c$ der Scheibe, $\vek K\,\vek u_c = \vek f + \vek f^+$. Es ist derselbe Vektor $\vek u_c$ wie in dem
Modell $\vek K_c\,\vek u_c = \vek f$ wobei die Matrix $\vek K_c$ auf den zuf\"{a}llig gestreuten Werten $E_i$ beruht. }
\label{U114}
\end{figure}%
%-----------------------------------------------------------------

%-----------------------------------------------------------------
\begin{figure}[tbp]
\centering
\includegraphics[width=0.9\textwidth]{\Fpath/U421}\label{Korrektur27}
\caption{Wandscheibe unter Eigengewicht, in den Elementen mit den Knotenkr\"{a}ften $f_i^+$ wurde der E-Modul um 90 \% verringert. Im Grunde verhalten sich die geschw\"{a}chten Bereiche wie \"{O}ffnungen. Bemerkenswert ist, dass die Kr\"{a}fte $f_i^+$ auf den Rand konzentriert sind, sie ziehen die \glq \"{O}ffnung\grq{} zusammen. Die Analogie ist naheliegend: In einem St\"{u}ck Blech, das man zur Verst\"{a}rkung mit Heftn\"{a}hten auf eine Stahlwand schwei{\ss}t, d\"{u}rften dieselben Kr\"{a}fte auftreten, nur dass sie kontinuierlich \"{u}ber den geschwei{\ss}ten Rand verteilt sind und die umgekehrte Richtung haben   }
\label{U421}
\end{figure}%
%-----------------------------------------------------------------
Dies d\"{u}rfte wesentlich daran liegen, dass die Knotenkr\"{a}fte $f_i^+$ auf der H\"{u}lle des Zuschlagskorns, mit denen wir ja die Abweichungen des Elastizit\"{a}tsmoduls vom Mittelwert korrigieren, Gleichgewichtskr\"{a}fte sind, die nahe beieinanderliegen, und ihre Fernwirkungen daher gegen null tendieren.
%-----------------------------------------------------------------
\begin{figure}[tbp]
\centering
\includegraphics[width=0.95\textwidth]{\Fpath/U362}  % war UE234
\caption{Ausfall einer St\"{u}tze zwischen zwei Decken und zwischen der untersten Decke und dem Fundament}
\label{U362}
\end{figure}%
%-----------------------------------------------------------------

Eine Scheibe $\Omega$ bestehe z.Bsp. aus einer Reihe von unterschiedlichen Elementen, $\Omega = \Omega_1 \cup \Omega_2 \cup \ldots \Omega_n$, die alle einen eigenen $E$-Modul $E_i$ aufweisen, der um einen Betrag $\Delta E_i = E - E_i$ von dem Mittelwert $E$ abweicht, s. Abb. \ref{U114}. Der exakte Knotenverschiebungsvektor $\vek u_c$ w\"{a}re daher die L\"{o}sung des Systems
\beq
\vek K_c\,\vek u_c = \vek f\,,
\eeq
wobei die Matrix  $\vek K_c$ sich aus den unterschiedlichen Elementmatrizen $\vek K_e(E_i)$ zusammensetzt. Wenn man hingegen mit einem einheitlichen $E$-Modul rechnet, also einer vereinfachten Matrix $\vek K$,
\beq
\vek K\,\vek u = \vek f\,,
\eeq
dann ist der Fehler in einer Verschiebung
\beq
J(u_c) - J(u) = \vek g^T\,(\vek f + \vek f^+) -  \vek g^T\,\vek f = \vek g^T\,\vek f^+
\eeq
relativ klein, weil die Kr\"{a}fte $f_i^+$, die von den Fehlertermen $\Delta E_i = E_i - E$ herr\"{u}hren
\beq
\vek K\,\vek u_c = \vek f + \vek f^+\,,
\eeq
zum einen $(1)$ Gleichgewichtsgruppen bilden und $(2)$ zum andern sich positive Abweichungen $\Delta E_i > 0$ und negative Abweichungen $\Delta E_j < 0$ ungef\"{a}hr die Waage halten werden, so dass diese beiden Effekte zusammen daf\"{u}r sorgen, dass die Fernfeldfehler klein sein werden, s. Abb. \ref{U114}. Man muss nicht jedes Zuschlagskorn durch ein eigenes Element darstellen, die Mathematik sorgt daf\"{u}r, dass sich die Fehler aufheben.

Bei der ersten Wandscheibe in Abb. \ref{U114} war die Streuung in den Steifigkeiten zuf\"{a}llig verteilt, jetzt in den Bildern \ref{U421} a und \ref{U421} b gibt es zwei Bereiche, in denen die Steifigkeit aller Elemente um denselben Betrag, um $90 \%$ reduziert wurde. Bemerkenswert ist hierbei die Dominanz der Kr\"{a}fte $f_i^+$ am Rand der beiden Bereiche, s. Abb. \ref{U421}.

%-----------------------------------------------------------------
\begin{figure}[tbp]
\centering
\includegraphics[width=0.75\textwidth]{\Fpath/U243}
\caption{Ausfall einer Eckst\"{u}tze, \textbf{ a)} Momente im LF $g$, \textbf{ b)} Kraft $f^+$ und zugeh\"{o}rige Momente, \textbf{ c)} a+ b = Momente nach Ausfall der Eckst\"{u}tze}
\label{U243}
\end{figure}%
%-----------------------------------------------------------------

Man ist versucht, weil die Kr\"{a}fte $f_i^+$ Gleichgewichtskr\"{a}fte sind, die immer paarweise auftreten, die durch die $f_i^+$ ausgel\"{o}sten Effekte zu ignorieren. Wenn, wie in  Abb. \ref{U362},  eine St\"{u}tze zwischen zwei Geschossen ausf\"{a}llt, dann kann man das am Originaltragwerk durch den Angriff von zwei gegengleichen Knotenkr\"{a}ften $f_i^+$ korrigieren und weil beide gleich gro{\ss} sind, heben sich ihre Wirkungen in der Ferne auf. Das ist richtig.


Wenn aber eine Fundamentst\"{u}tze ausf\"{a}llt, dann wirken zwar auch wieder zwei Kr\"{a}fte $\pm f_i^+$, aber die untere Kraft ist am Boden verankert und so bleibt von dem Paar $\pm f_i^+$ nur die Kraft $f_i^+$ am St\"{u}tzenkopf \"{u}brig, die, weil sie keinen Antagonisten hat, weiter ausstrahlen wird, als ein gegengleiches Kr\"{a}ftepaar $\pm f_i^+$. \\

\hspace*{-12pt}\colorbox{highlightBlue}{\parbox{0.98\textwidth}{
Der Ausfall einer Fundamentst\"{u}tze ist kritischer, als der Ausfall einer Zwischenst\"{u}tze.}}\\

%%%%%%%%%%%%%%%%%%%%%%%%%%%%%%%%%%%%%%%%%%%%%%%%%%%%%%%%%%%%%%%%%%%%%%%%%%%%%%%%%%%%%%%%%%%%%%%%%%%
\textcolor{sectionTitleBlue}{\section{Rahmen}}
In Abb. \ref{U243} a sind die Momente eines Rahmens im LF $g$ dargestellt und darunter, in Abb. \ref{U243} b, die Momente im selben Lastfall, wenn eine Gescho{\ss}st\"{u}tze ausf\"{a}llt. Statisch entspricht der \"{U}bergang von Bild a zu Bild c der Wirkung von Zusatzkr\"{a}ften $f_i^+$, wie in Abb. \ref{U243} b gezeigt,
\begin{align}
(\vek K + \vek \Delta \,\vek K)\,\vek u_c = \vek f \qquad \vek K\,\vek u_c = \vek f + \vek f^+\,.
\end{align}
Die Momente, die diese Kr\"{a}fte $f_i^+$ erzeugen, zu den Momenten des Ausgangszustandes addiert, ergeben die Momente des geschw\"{a}chten Systems.

%Das Beispiel des aufgest\"{a}nderten Balkens in Abb. \ref{U246} soll demonstrieren, dass lokale \"{A}nderungen in den Steifigkeiten zu wellenf\"{o}rmigen Effekten f\"{u}hren, die aber rasch abklingen.

%-----------------------------------------------------------------
\begin{figure}[tbp]
\centering
\includegraphics[width=0.8\textwidth]{\Fpath/U296}
\caption{Biegebalken, \textbf{ a)} Gleichlast, \textbf{ b)} Einflussfunktion f\"{u}r $M(x)$ erzeugt durch zwei gegengleiche Momente $M_l = M_r$, \textbf{ c)} Lagersenkung $\Delta = 1$}
\label{U296}
\end{figure}%
%-----------------------------------------------------------------
\pagebreak
%%%%%%%%%%%%%%%%%%%%%%%%%%%%%%%%%%%%%%%%%%%%%%%%%%%%%%%%%%%%%%%%%%%%%%%%%%%%%%%%%%%%%%%%%%%%%%%%%%%
\textcolor{sectionTitleBlue}{\section{Starre Lager}}
Ein elastisches Lager ($k$) gibt unter Last um einen Weg $u$ nach. Ein eventueller Ausfall des Lagers entspricht einem Sprung $\Delta k = -k$ in der Lagersteifigkeit und so muss eine Kraft $f^+ = - (-k)\,u_c = k\,u_c$ zur rechten Seite addiert werden, um rechnerisch den Ausfall des Lagers zu kompensieren.

Bei starren Lager ist die $f^+$-Technik nicht anwendbar, weil ein Lagerweg $u_c $ n\"{o}tig ist, um eine Kraft $f^+ = \Delta k\,u_c$ zu generieren. In einer solchen Situation w\"{u}rde man die Lagerkraft in umgekehrter Richtung auf das modifizierte Tragwerk aufbringen (nach dem Ausfall des Lagers) und die Ergebnisse zu den urspr\"{u}nglichen Ergebnissen addieren.

Wie wir die Effekte erfassen, die der Ausfall eines starren Lagers auf die inneren Kr\"{a}fte hat, wollen wir beispielhaft an dem Tr\"{a}ger in Abb. \ref{U296} zeigen. Es soll untersucht werden, wie sich das Biegemoment in der Mitte des Balkens \"{a}ndert, wenn das starre Lager am Ende des Balkens ausf\"{a}llt. (Der Praktiker w\"{u}rde dieses Problem nat\"{u}rlich ganz anders und schneller l\"{o}sen, aber es geht hier um die Systematik, die auch noch greift, wenn 10 Stockwerke \"{u}bereinander liegen).

Zuerst zeigen wir, dass die Lagerreaktion $R_G$, die zur Einflussfunktion f\"{u}r das Moment $M(x)$ in Abb. \ref{U296} b geh\"{o}rt, gleich dem Moment $M(x)$ ist, wenn sich das Lager um einen Meter senkt, $\Delta = 1$,\footnote{Das entspricht der Formel $J_1(G_2) = J_2(G_1)$ Mit $J_1(w) = R$ (Lagerreaktion die zu $w$ geh\"{o}rt) und $J_2(w) = M(x)$ (Moment von $w$ in Balkenmitte) mit $G_2$ wie in Abb. \textbf{ b)} und $G_1 = w_\Delta$ wie in Abb. \textbf{ c)}.}
\begin{align}
R_G = M(x) \qquad \text{aus Setzung $\Delta = 1$}\,.
\end{align}
Das ergibt sich aus dem {\em Satz von Betti\/} $\text{\normalfont\calligra B\,\,}(G_2,w_\Delta) = 0$
\begin{align}
A_{1,2} = \underbrace{M_l\,w_\Delta'(x_{-}) - M_r\,w_\Delta'(x_+)}_{= \,0} + R_G \cdot 1 = M(x) \cdot 1 + R_\Delta \cdot 0 = A_{2,1} \,.
\end{align}
Wenn das Lager unter einer Last $p$ ausf\"{a}llt, dann muss die vorherige Lagerkraft $R_p$ als \"{a}u{\ss}ere Kraft und in umgekehrter Richtung aufgebracht werden. Das f\"{u}hrt dazu, dass sich das Balkenende um den Betrag
\begin{align}\label{Eq118}
w(l) = R_p \cdot \frac{1}{k_S} \qquad k_S =  \frac{3\,EI}{l^3}
\end{align}
senkt. Wenn eine Setzung um einen Meter, $\Delta = 1$, ein Moment $M(x) = R_G$ verursacht, dann verursacht die Durchbiegung in (\ref{Eq118}) das Moment
\begin{align}\label{Eq119}
M(x) = R_G \cdot R_p \cdot \frac{1}{k_S} = R_G \cdot \text{Durchbiegung aus $R_p$} \,.
\end{align}
Das Moment $M(x) $ ist das $\Delta M(x) $, das zu dem urspr\"{u}nglichen Moment im Punkt $x$ addiert werden muss, um den Verlust des starren Lagers auszugleichen.
%-----------------------------------------------------------------
\begin{figure}[tbp]
\centering
\includegraphics[width=0.99\textwidth]{\Fpath/U540}
\caption{Das rasche Abklingen der Momente aus den $X_i$ bei einem Durchlauftr\"{a}ger (ein \glq Urph\"{a}nomen\grq{} der Statik) weist auf die enge Verwandtschaft zwischen den $f_i^+$ und den $X_i $ hin; die $f_i^+$ sind sozusagen die $X_i$ beim \"{U}bergang vom System $\vek K $ zum System $ \vek K + \vek \Delta \vek K$}
\label{U540}
\end{figure}%
%-----------------------------------------------------------------

Die Zahl $k_S$ ist die Steifigkeit der Struktur -- nach dem Ausfall des Lagers -- in Richtung von $w$. Wenn in dem Lager eine gewisse Reststeifigkeit $k_R$ verbleibt, dann bilden $k_S$ und $k_R$ eine Federkette und dann muss $1/k_S$ sinngem\"{a}{\ss} durch $1/k$ ersetzt werden
\begin{align}
\frac{1}{k} = \frac{1}{k_S} + \frac{1}{k_R}\,.
\end{align}

%-----------------------------------------------------------------
\begin{figure}[tbp]
\centering
\includegraphics[width=0.8\textwidth]{\Fpath/U394a}
\caption{Berechnung der Einflussfunktion f\"{u}r eine Eckverschiebung mit einem Dirac Delta, das in diskreter Form einer Knotenkraft $j$ gleich ist. \"{A}ndert sich in einem Element die Steifigkeit, dann braucht man auf der rechten Seite, $\vek K\,\vek g_c = \vek j + \vek j^+$, zus\"{a}tzliche Knotenkr\"{a}fte $\vek j^+$, wenn man mit der Matrix $\vek K$ rechnet; s. auch Abb. \ref{U369} auf S. \pageref{U369}}
\label{U394}
\end{figure}%
%-----------------------------------------------------------------

%%%%%%%%%%%%%%%%%%%%%%%%%%%%%%%%%%%%%%%%%%%%%%%%%%%%%%%%%%%%%%%%%%%%%%%%%%%%%%%%%%%%%%%%%%%%%%%%%%%
\textcolor{sectionTitleBlue}{\section{Das Kraftgr\"{o}{\ss}enverfahren}}

Zwischen dem Kraftgr\"{o}{\ss}enverfahren und den  $f_i^+$ besteht ein enger Zusammenhang, denn beim Rechnen mit den $f_i^+$ \"{a}ndern wir nicht die Steifigkeitsmatrix, sondern wir \"{a}ndern die rechte Seite, aus dem Vektor $\vek f$ wird der Vektor $\vek f + \vek f^+$. Genauso geht auch das Kraftgr\"{o}{\ss}enverfahren vor.

Das Kraftgr\"{o}{\ss}enverfahren w\"{a}hlt ein statisch bestimmtes {\em Hauptsystem\/} und in der Folge findet alles Rechnen an diesem System statt. Die statisch \"{U}berz\"{a}hligen $X_i $ spielen dabei dieselbe Rolle wie die $f_i^+$. W\"{a}hrend die $f_i^+ $ das zus\"{a}tzliche Element an die Struktur koppeln, beseitigen die  $X_i $ die Klaffungen. Beide, die $X_i$ wie die $f_i^+$, sind {\em Zusatzlasten\/}, die auf der rechten Seite erscheinen, w\"{a}hrend die eigentliche Analyse am unver\"{a}nderten Hauptsystem (Matrix $\vek K$) vor sich geht.

Das hat den Vorteil, dass wir nicht zwei S\"{a}tze von Einflussfunktionen brauchen: Einen Satz f\"{u}r die Schnittgr\"{o}{\ss}en am Hauptsystem (Matrix $\vek K$) und einen zweiten Satz f\"{u}r das statisch unbestimmte System (Matrix $\vek K_c$).

Es gibt aber noch, wie Abb. \ref{U540} zeigt, eine weitere Verwandtschaft zwischen den $f_i$ und den $X_i$. Die $f_i^+$ sind sozusagen die $X_i$, die das System $\vek K $ auf das Niveau $\vek K + \vek \Delta \vek K $ heben.

%%%%%%%%%%%%%%%%%%%%%%%%%%%%%%%%%%%%%%%%%%%%%%%%%%%%%%%%%%%%%%%%%%%%%%%%%%%%%%%%%%%%%%%%%%%%%%%%%%%
\textcolor{sectionTitleBlue}{\section{Kr\"{a}fte $\vek j^+$}} \label{Korrektur1}\label{jplus}

Die Formel $J(e) = \vek g^T\,\vek f^+$ legt nahe zu probieren, ob es nicht auch eine Formel
\begin{align}
J(e) = \vek u^T\,\vek j^+
\end{align}
gibt, s. Abb. \ref{U394}.\index{$\vek j^+$}

Und die gibt es in der Tat, denn die Verschiebung $u(x)$ in einem Punkt kann man auf zwei Arten berechnen
\begin{align}
u(x) = \int_0^{\,l} G(y,x)\,p(y)\,dy = \int_0^{\,l} u(y)\,\delta(y-x)\,dy\,.
\end{align}
In der linearen Algebra der finiten Elemente lautet die zweite Gleichung
\begin{align}
J(u) = u(x) = \vek u^T\,\vek j(x)\,,
\end{align}
wenn $\vek j(x)$ die \"{a}quivalenten Knotenkr\"{a}fte zu einem Dirac Delta sind, das im Punkt $x$ angreift. Aus $(\vek K + \vek \Delta \vek K)\,\vek g_c = \vek j$ folgt
\begin{align}
\vek K\,\vek g_c = \vek j - \vek \Delta \vek K\,\vek g_c = \vek j + \vek j^+
\end{align}
und nach Multiplikation von links mit $\vek u$ (es ist $\vek u^T\,\vek K = \vek f^T$)
\begin{align}
\underbrace{\vek u^T\,\vek K\,\vek g_c}_{J(\vek u_c)} = \underbrace{\vek u^T\vek j}_{J(\vek u)} - \vek u^T\vek \Delta \vek K\,\vek g_c = \vek u^T\vek j + \vek u^T\vek j^+\,,
\end{align}
ergibt sich
\begin{align}\label{Eq104}
J(\vek e) = u_c(x) - u(x) = \vek u^T\,\vek j^+
\end{align}
mit
\begin{align}
\vek j^+ = -\vek \Delta \,\vek K \,\vek g_c\,.
\end{align}
Wenn man z.B. mit Knotenkr\"{a}ften $\vek j$ die Einflussfunktion f\"{u}r die Querkraft in einem Riegel im 5. Stock generiert und im 1. Stock \"{a}ndert sich die Steifigkeit eines Stiels, dann muss man die Knoten dieses Stiels mit Knotenkr\"{a}ften $\vek j^+$ belasten, um am  urspr\"{u}nglichen Tragwerk durch L\"{o}sen des Systems $\vek K\,\vek g_c = \vek j + \vek j^+$ die ge\"{a}nderte Einflussfunktion, $\vek g \to \vek g_c$, zu berechnen. Von der Spreizung $\vek g_c$ im 5. Stock wird aber nicht viel im 1. Stock ankommen, so dass die notwendigen Zusatzkr\"{a}fte $\vek j^+$ klein sein werden.

Das ganze ist insofern theoretisch, weil man ja $\vek g_c$ nicht kennt (man k\"{o}nnte $\vek g_c\sim \vek g $ setzen) und somit den Vektor $\vek j^+$ nicht berechnen kann, aber es veranschaulicht, was die $j_i^+$ bedeuten.

Nehmen wir eine Verschiebung als Beispiel. Um auf dem \glq alten\grq{} Netz $G_c$ zu erzeugen, muss man zus\"{a}tzlich die Kr\"{a}fte $j_i^+$ anbringen. Formuliert man den Satz von Betti mit der {\em alten\/} L\"{o}sung $u$ und dem {\em neuen\/} $G_c$ so entsteht
\begin{align}
1 \cdot u + \sum_i\,j_i^+\,u_i = \int_{\Omega} G_c\,p\,d\Omega_{\vek y} = 1 \cdot u_c\,,
\end{align}
was bedeutet, dass die Arbeit der Kr\"{a}fte $\vek j$ und $\vek j^+$ auf den alten Wegen $\vek u$  gleich dem neuen $u_c$ im Aufpunkt ist.

%%%%%%%%%%%%%%%%%%%%%%%%%%%%%%%%%%%%%%%%%%%%%%%%%%%%%%%%%%%%%%%%%%%%%%%%%%%%%%%%%%%%%%%%%%%%%%%%%%%
\textcolor{sectionTitleBlue}{\section{Austausch als Alternative}}
Statt mit den Kr\"{a}ften $f_i^+$ zu operieren, gibt es auch andere Techniken.
Man stelle sich vor, dass eine St\"{u}tze zwischen zwei Decken, s. Abb. \ref{U126}, ausgetauscht werden muss. Wenn die neue St\"{u}tze dieselbe Steifigkeit $k = EA$ hat, dann
\begin{itemize}
  \item muss sie auf dem Bauhof dieselbe L\"{a}nge haben, wie die alte unbelastete St\"{u}tze
  \item muss die neue St\"{u}tze so vorgespannt werden, dass sie bei dem Einbau dieselbe L\"{a}nge hat wie die alte St\"{u}tze.
\end{itemize}
Nach dem Einbau k\"{o}nnen die Pressen an der neuen St\"{u}tze weggenommen werden und  nichts hat sich ge\"{a}ndert\footnote{Eigentlich ist $k = EA/l$, aber zu Vereinfachungen lassen wir den Faktor $1/l$ weg}.\label{Korrektur3}
%-----------------------------------------------------------------
\begin{figure}
\centering
\if \bild 2 \sidecaption \fi
{\includegraphics[width=0.9\textwidth]{\Fpath/U126}}
\caption{Zunahme und Abnahme der L\"{a}ngssteifigkeit $EA$ in einer St\"{u}tze}
\label{U126}%
\end{figure}%
%-----------------------------------------------------------------

Wenn stattdessen die neue St\"{u}tze eine Steifigkeit $EA_c > EA$ hat, dann dr\"{u}cken Zusatzkr\"{a}fte (= $f_i^+$) gegen die obere und untere Decke, sobald die Pressen entfernt werden, s. Abb. \ref{U126}. Denn um eine St\"{u}tze mit $EA_c > EA$ zusammenzudr\"{u}cken, sind gr\"{o}{\ss}ere Kr\"{a}fte notwendig als im Fall $EA_c = EA$. Und wenn die Pressen entfernt werden, dann dr\"{u}cken diese Kr\"{a}fte gegen die Decken.

Wenn die neue St\"{u}tze eine geringere Steifigkeit hat, $EA_c < EA$, dann ist die Situation im Grunde dieselbe, wir m\"{u}ssen nur das Vorzeichen der $f_i^+$ umdrehen.

Statt also ein zweites Element $\Omega_e^+$ vor das erste Element $\Omega_e$ zu legen, k\"{o}nnen wir uns auch vorstellen, dass wir das Element als Ganzes ersetzen. Weil aber die Steifigkeit {\em neu -- alt\/} unterschiedlich ist, $EA_c \neq EA$, sind Zusatzkr\"{a}fte $f_i^+$ an den Enden des Elements notwendig. Es gibt vier m\"{o}gliche Szenarien:

\begin{itemize}
  \item $EA_c > EA$ und $N > 0$ (Zug), dann ziehen die $f_i^+$ die St\"{u}tze zusammen, ($\rightarrow\,\,\leftarrow$).
  \item $EA_c > EA$ und $N < 0$ (Druck), dann dr\"{u}cken die $f_i^+$ die St\"{u}tze auseinander, ($\leftarrow \,\,\rightarrow$).
  \item $EA_c < EA$ und $N > 0$, dann ziehen die $f_i^+$ die St\"{u}tze auseinander, ($\leftarrow \,\,\rightarrow$).
\item $EA_c < EA$ und  $N < 0$, dann ziehen die $f_i^+$ die St\"{u}tze zusammen, ($\rightarrow\,\,\leftarrow$).
\end{itemize}
Wenn $EA_c > EA$ ist, dann nehmen die Verformungen der St\"{u}tze ab und wenn  $EA_c < EA$ ist, dann nehmen sie zu.


%-----------------------------------------------------------------
\begin{figure}[tbp]
\centering
\includegraphics[width=.99\textwidth]{\Fpath/UE352}
\caption{Eine lokale Steifigkeits\"{a}nderung, \"{a}ndert die Einflussfunktion f\"{u}r  $V(x)$ im ganzen Rahmen}
\label{UE352}
\end{figure}%
%-----------------------------------------------------------------

%-----------------------------------------------------------------
\begin{figure}[tbp]
\centering
\includegraphics[width=.99\textwidth]{\Fpath/UE366}
\caption{Risse in einem Balken}
\label{UE366}
\end{figure}%
%-----------------------------------------------------------------

%%%%%%%%%%%%%%%%%%%%%%%%%%%%%%%%%%%%%%%%%%%%%%%%%%%%%%%%%%%%%%%%%%%%%%%%%%%%%%%%%%%%%%%%%%%%%%%%%%%%%%%%
\textcolor{sectionTitleBlue}{\section{Integration \"{u}ber das defekte Element}}
Auch wenn sich nur eine Zahl $k_{ij}$ in der Steifigkeitsmatrix $\vek K$ \"{a}ndert, \"{a}ndert sich die ganze Inverse $\vek K^{-1}$. Praktisch bedeutet dies z.B., dass die Schw\"{a}chung eines Elementes, $ EI \to EI + \Delta EI$, die Querkraft $V(x)$ in allen Punkten des Rahmens \"{a}ndert
\begin{align}
V_c(x) - V(x) = \sum_e \int_0^{\,l_e} (G_3^c(y,x) - G_3(y,x))\,p(y)\,dy\,,
\end{align}
weil sich eben die Einflussfunktion, $G_3(y,x) \rightarrow G_3^c(y,x)$, \"{u}berall \"{a}ndert, und somit die \"{U}berlagerung der Belastung mit $G_3^c(y,x)$ andere Werte ergibt, s. Abb. \ref{UE352}.

Es gibt aber die M\"{o}glichkeit, das Nachrechnen auf das Element, das sich \"{a}ndert, zu beschr\"{a}nken. Wir betrachten hierzu den Zweifeldtr\"{a}ger in Abb. \ref{UE366}. Zu Anfang war die Steifigkeit $EI$ in beiden Feldern gleich gro{\ss} und so lautet die schwache Form der Balkengleichung $EI\,w^{IV} = p$
\begin{align}
\int_0^{\,l} EI\,w''\,\delta w''\,dx = \int_0^{\,l} p\,\delta w\,dx\,,
\end{align}
oder kurz
\begin{align}
a(w, \delta w) = (p, \delta w)\,.
\end{align}
Dann \"{a}ndert sich die Steifigkeit im zweiten Feld auf einen Wert $EI + \Delta EI$, die Biegelinie \"{a}ndert sich mit, $w \to w_c$, und die schwache Form geht \"{u}ber in
\begin{align}\label{Eq177}
\int_0^{\,l} EI\,w_c''\, \delta w''\,dx + \underbrace{\int_{l/2}^{\,l} \Delta EI\,w_c''\, \delta w''\,dx}_{d(w_c,\delta w)} = \int_0^{\,l} p\,\delta w\,dx
\end{align}
oder\footnote{Diese additive Zerlegung, $a(.,.) + d(.,.)$ ist der Schl\"{u}ssel zu $J(e) = - d(G,u_c)$.}
%-----------------------------------------------------------------
\begin{figure}[tbp]
\centering
\includegraphics[width=.90\textwidth]{\Fpath/U465}
\caption{Die Integration \"{u}ber das gerissene Element oben links reicht aus, um die \"{A}nderung in der horizontalen Verschiebung zu bestimmen \textbf{ a)} Spannungsverteilung nach dem Auftreten der Risse im Element oben links \textbf{ b)} Spannungen aus der Einzelkraft $P = 1$ (Einflussfunktion f\"{u}r $u_x$) am oberen Kragarmende}
\label{U465}
\end{figure}%
%-----------------------------------------------------------------
\begin{align}
a(w_c, \delta w) + d(w_c, \delta w) = (p, \delta w)\,.
\end{align}
Wenn wir die beiden Gleichungen voneinander abziehen, dann folgt
\begin{align}
a(w_c - w, \delta w) + d(w_c, \delta w) = 0
\end{align}
oder mit $e = w_c - w$
\begin{align}\label{Eq10}
a(e, \delta w) = -  d(w_c, \delta w)\,.
\end{align}
W\"{a}hlen wir als virtuelle Verr\"{u}ckung $\delta w$ die Einflussfunktion $G$ eines Funktionals $J(w)$, dann ist das Ergebnis die \"{A}nderung in dem Funktional
$J(e) = a(e, G) = - d(w_c, G)$, also
\begin{align}\label{Eq165}
J(e) = - d(w_c, G)\,.
\end{align}
Man beachte, dass
\begin{align}
d(w_c,\delta w) = \int_{l/2}^{\,l} \Delta EI\,w_c''\, \delta w''\,dx = \frac{\Delta EI }{EI} \int_{l/2}^{\,l} \frac{M_c\,M}{EI_c}\,dx\,.
\end{align}
In der Notation der linearen Algebra ist die Herleitung noch einfacher. Bei einer \"{A}nderung der Steifigkeitsmatrix, $\vek K \to \vek K + \vek \Delta \vek K$, wird aus der urspr\"{u}nglichen schwachen Formulierung
\begin{align}
\vek \delta \vek u^T\,\vek K\,\vek u = \vek \delta \vek u^T\,\vek f
\end{align}
die Gleichung
\begin{align}
\vek \delta \vek u^T\,(\vek K + \vek \Delta\,\vek K)\,\vek u_c = \vek \delta \vek u^T\,\vek f\,,
\end{align}
so dass  der Vektor $\vek e = \vek u_c - \vek u$ der Gleichung
\begin{align}\label{Eq143}
\vek  \delta \vek u^T\,\vek K\,\vek e = - \vek \delta \vek u^T\,\vek \Delta \vek K\,\vek u_c \qquad \text{f\"{u}r alle} \,\vek \delta \vek u
\end{align}
gen\"{u}gt. An dem Zustandsvektor $\vek u$ eines Systems k\"{o}nnen wir mittels einer Einflussfunktion, dem Vektor $\vek g$, Messungen vornehmen
\begin{align}
J(\vek u) = \vek g^T\,\vek K\,\vek u
\end{align}
und so folgt, wenn wir in (\ref{Eq143}) f\"{u}r $\vek  \delta \vek u$ den Vektor $\vek g$ setzen, das Ergebnis
\begin{align}\label{Eq188}
J(\vek e) =  - \vek g^T\,\vek \Delta \vek K\,\vek u_c \qquad (= - d(w_c, G))\,,
\end{align}
zu dessen Berechnung wir nur auf dem oder den betroffenen Element(en) messen m\"{u}ssen, s. Abb. \ref{U465}. Diese Gleichung entspricht (\ref{Eq165}).

Im Grunde steht in (\ref{Eq188}) die Wechselwirkungsenergie, denn die \"{A}nderung $J(\vek e)$ kann man, und das gilt f\"{u}r alle Bauteile, in der Form
\begin{align}
J(\vek e) = - d(w_c, G) = -\alpha\cdot a(G,w_c)_{\Omega_e}
\end{align}
schreiben, also als die Wechselwirkungsenergie zwischen der Einflussfunktion $G$ f\"{u}r $J(w)$ und der neuen L\"{o}sung $w_c$ in dem Element $\Omega_e$, dessen Steifigkeit sich \"{a}ndert, wobei
\begin{align}
\alpha = \frac{\Delta E}{E} =  \frac{\Delta EI}{EI} =  \frac{\Delta EA}{EA} =\frac{\Delta k}{k} = \frac{\Delta K}{K} \qquad \text{etc.}
\end{align}
das Verh\"{a}ltnis der Steifigkeits\"{a}nderung zur urspr\"{u}nglichen Steifigkeit ist.

%%%%%%%%%%%%%%%%%%%%%%%%%%%%%%%%%%%%%%%%%%%%%%%%%%%%%%%%%%%%%%%%%%%%%%%%%%%%%%%%%%%%%%%%%%%%%%%%%%%%%%%%
\textcolor{sectionTitleBlue}{\section{Observable}}
Nachdem nun klar ist, dass man die Effekte von \"{A}nderungen durch eine lokale Analyse ermitteln kann, wollen wir uns im folgenden etwas systematischer und im Kontext der klassischen Statik mit dieser Strategie besch\"{a}ftigen.

Jede Gr\"{o}{\ss}e $O$ (\glq Observable\grq{})\index{Observable} in einem Tragwerk ist im Rahmen der Theorie I. Ordnung mittels einer Einflussfunktion berechenbar. Diese Einflussfunktionen sind aber selbst Verschiebungen, d.h. zu ihnen geh\"{o}ren Biegemomente $M_G$, Querkr\"{a}fte $V_G$ und Normalkr\"{a}fte $N_G$. Das ist wichtig, weil in der Methode, die wir jetzt erl\"{a}utern wollen, das Augenmerk auf die innere Energie gerichtet ist. Wir benutzen eine Variante des Mohrschen Arbeitsintegral, bei der die \"{A}nderungen in der Gr\"{o}{\ss}e $O$, der \glq Observablen\grq{},  anhand der Wechselwirkungsenergie zwischen der Einflussfunktion und den Tragwerksverschiebungen gemessen wird.

Nehmen wir an, dass die Gr\"{o}{\ss}e $O = w(x)$ die Durchbiegung des Balkens in einem Punkt  $x$ ist.  Das Mohrsche Arbeitsintegral liefert f\"{u}r $O = w(x)$ den Wert
\bfo\label{Eq142}
O = \int_0^{\,l} \frac{M(y)\,M_G(y,x)}{EI} \,dy\,.
\efo
Dann entsteht eine neue Situation: In einem Element des Balkens \"{a}ndert sich die Steifigkeit, $EI \to EI + \Delta EI$. Das bedeutet, dass sich Momente $M \to M_c$ und $M_G \to M_G^c$ \"{a}ndern und wir m\"{u}ssen wieder von vorne anfangen
\bfo
O_c = \int_0^{\,l} \frac{M_c(y)\,M_G^c(y,x)}{EI_c} \,dy\,.
\efo
Hierbei ist $EI_c$ die {\em step function\/}
\begin{align}
EI_c = \left \{ \barr{l} EI + \Delta EI \qquad \,\,\,\text{im Intervall $[x_a,x_b]$} \\EI \qquad \qquad\qquad \text{sonst}  \earr \right. \,.
\end{align}
Nun kommt der interessante Punkt. \\

\hspace*{-12pt}\colorbox{highlightBlue}{\parbox{0.98\textwidth}{Wie wir oben gezeigt haben, (\ref{Eq188}), kann man die \"{A}nderung von $O$ alleine durch Integration \"{u}ber das Element, in dem sich die Steifigkeit \"{a}ndert, berechnen}}\\

\bfo\label{Eq100}
\boxed{O_c - O = -\frac{\Delta\,EI}{EI}\int_{x_a}^{\,x_b} \frac{M_c\,M_G}{EI_c}\,dy}
\efo
Hier ist $M_c$ das Moment in dem Element {\em nach\/} der \"{A}nderung der Steifigkeit und $M_G$ ist das Moment der Einflussfunktion {\em vor\/} der \"{A}nderung.

Wenn wir das Biegemoment durch seine Ableitungen ersetzen, $M = - EI w''$, wird die Gleichung vielleicht transparenter
\bfo
O_c - O = -\frac{\Delta\,EI}{EI}\int_{x_a}^{\,x_b} \frac{M_c\,M_G}{EI_c}\,dy = - \Delta EI\,\int_{x_a}^{\,x_b} w_c''\,G''\,dy\,.
\efo
Es ist auch nicht wichtig, ob wir $M_c$ mit $M_G$ kombinieren oder $M$ mit $M_G^c$
\begin{align}
M \times M_G^c \equiv M_c \times M_G\,,
\end{align}
das Ergebnis ist dasselbe
\bfo
\boxed{O_c - O = -\frac{\Delta\,EI}{EI}\int_{x_a}^{\,x_b} \frac{M\,M_G^c}{EI_c}\,dy}
\efo
Es sei betont, dass der Punkt $x$ {\bf an beliebiger Stelle} im Tragwerk liegen kann. Die \"{A}nderung von $O(x)$ im Punkt $ x$ kann man {\em allein\/} durch Integration \"{u}ber das gesch\"{a}digte Element $[x_a,x_b]$ berechnen. Der Punkt $x$ kann im zehnten Stock liegen und das Element $[x_a,x_b]$ im dritten Stock eingebaut sein. Entscheidend ist, was von der Einflussfunktion f\"{u}r $O(x)$, die im zehnten Stock startet, im dritten Stock ankommt, wie gro{\ss} das Moment $M_G$ der Einflussfunktion im Element $[x_a,x_b]$ noch ist -- und nat\"{u}rlich wie gro{\ss} das Lastmoment $M_c$ dort ist.

Die Formel (\ref{Eq100}) hat allerdings den Nachteil, dass man den Momentenverlauf $M_c$ am ge\"{a}nderten System kennen muss. Aber wenn man den kennt, dann braucht man die Formel nicht mehr...

Also ersetzen wir $M_c$ durch das Moment $M$ am urspr\"{u}nglichen Tragwerk und kommen so zur N\"{a}herung
\bfo
O_c - O \simeq -\frac{\Delta\,EI}{EI}\int_{x_a}^{\,x_b} \frac{M\,M_G}{EI_c}\,dy\,.
\efo
Je nach der Art der Steifigkeits\"{a}nderung muss man diese Formel nat\"{u}rlich modifizieren. Wenn sich die L\"{a}ngssteifigkeit $EA$ einer St\"{u}tze \"{a}ndert, dann lautet die Formel
\bfo
O_c - O \simeq  -\frac{\Delta EA}{EA}\int_{x_a}^{\,x_b} \frac{N  N_G}{EA_c} \,dy\,,
\efo
und \"{a}ndert sich bei einer Scheibe in einem Element $\Omega_e$ der E-Modul, dann lautet die Formel
\begin{align}
O_c - O \simeq -\frac{\Delta E}{E} \int_{\Omega_e} \sigma_{ij}\,\varepsilon_{ij}^G \,d\Omega_{\vek y}\,,
\end{align}
und so kann diese N\"{a}herung auf alle Tragwerke und alle interessierenden Gr\"{o}{\ss}en angewandt werden.

%-----------------------------------------------------------------
\begin{figure}[tbp]
\centering
\includegraphics[width=0.9\textwidth]{\Fpath/U186}
\caption{Momente $M$, \textbf{ a)} aus Wind und \textbf{ b)} Momente $\bar{M}$ durch die Verdrehung des Fu{\ss}punktes, Einflussfunktion f\"{u}r das Fu{\ss}punktsmoment. Stabweise wird das Integral von $M$ und $M_G$ mit dem Quotienten $\Delta EI/EI$ gewichtet und man bestimmt so den Einfluss, den eine \"{A}nderung $\Delta EI$ in dem Stab auf das Fu{\ss}punktsmoment haben w\"{u}rde. Genau genommen m\"{u}sste man $M$ durch $M_c$ ersetzen, aber dies kann durch einen Korrekturfaktor, $M_c \simeq \alpha M$ n\"{a}herungsweise ausgeglichen werden.}
\label{U186}
\end{figure}%
%-----------------------------------------------------------------

Das Erstaunliche an der Formel (\ref{Eq100}), die ja auf der ersten Greenschen Identit\"{a}t in der Formulierung als {\em Prinzip der virtuellen Kr\"{a}fte\/} beruht,
\begin{align}\label{Eq121}
\text{\normalfont\calligra G\,\,}(G,w) = \delta A_a^* - \delta A_i^* = 0\,,
\end{align}
ist, dass man mit ihr auch die Auswirkungen von Steifigkeits\"{a}nderungen auf Schnittgr\"{o}{\ss}en, etwa $V(x) \to V(x) + \Delta V$, voraussagen kann, w\"{a}hrend man ja sonst mit dem {\em Prinzip der virtuellen Kr\"{a}fte\/} Kraftgr\"{o}{\ss}en eigentlich nicht berechnen kann.


%%%%%%%%%%%%%%%%%%%%%%%%%%%%%%%%%%%%%%%%%%%%%%%%%%%%%%%%%%%%%%%%%%%%%%%%%%%%%%%%%%%%%%%%%%%%%%%%%%%
\textcolor{sectionTitleBlue}{\section{Nah und fern}}

Die \"{A}nderung $O_c - O$ wird nur dann auffallend sein, wenn die Wechselwirkungsenergie $\delta A_i = a(G,w_c)_{\Omega_e}$ in dem Element \glq gro{\ss}\grq{} ist\footnote{Wir lassen das Sternchen an $\delta A_i^*$ weg, weil es mathematisch \glq Verzierung\grq{} ist}. Das ist dann der Fall, wenn die Schnittgr\"{o}{\ss}en aus der  Einflussfunktion und gleichzeitig die Schnittgr\"{o}{\ss}en aus der Belastung in dem Bauteil gro{\ss} sind.

Weil die Einflussfunktionen (in der Regel) rasch abklingen, kann man davon ausgehen, dass Steifigkeits\"{a}nderungen in weit abliegenden Bauteilen nur sehr geringen Einfluss auf die Schnittgr\"{o}{\ss}en im \glq Vordergrund\grq{} haben werden.

Und weil $\delta A_i$ ein Skalarprodukt ist, kann es, wie bei Vektoren, Funktionen $M$ und $M_G$ geben, die senkrecht aufeinander stehen, deren \"{U}berlagerung also null ergibt; wenn etwa $M$ antimetrisch ist und $M_G$ symmetrisch. In solchen F\"{a}llen ist, unabh\"{a}ngig von der Gr\"{o}{\ss}e von $M$ und $M_G$, die \"{A}nderung $O_c - O$ null.

Vor jeder Berechnung kann man also, allein durch das Studium der Einflussfunktionen,
absch\"{a}tzen, welche Steifigkeits\"{a}nderungen Effekt machen und welche nicht.


Der Rahmen in Abb. \ref{U186} wird seitlich vom Wind angeblasen. Es soll abgesch\"{a}tzt werden, welche Steifigkeits\"{a}nderungen $\Delta EI$ in den Riegeln und Stielen das Anschnittsmoment im linken Fu{\ss}punkt am meisten beeinflussen ({\em Focus auf einen Punkt\/}).

Hierzu wird die Einflussfunktion $G$ f\"{u}r das Moment aufgestellt. Allerdings interessiert nicht, wie sonst \"{u}blich, der Verlauf von $G$, sondern es interessieren die {\em Momente\/} $M_G$ aus der Einflussfunktion.

\"{A}ndert sich in einem Stiel oder Riegel die Biegesteifigkeit, $EI \to EI + \Delta EI$, so ist die \"{A}nderung im Anschnittsmoment n\"{a}herungsweise gleich
\bfo
M_c - M \simeq -\frac{\Delta EI}{EI} \int_0^{\,l_e} \frac{M_G\,M}{EI}\,dy\,.
\efo
Die Bauteile, in denen die Momente $M_G$ und $M$ {\em beide\/} gro{\ss} sind, (und nicht orthogonal zueinander), sind die Bauteile mit dem gr\"{o}{\ss}ten Einfluss auf das Anschnittsmoment. Nicht \"{u}berraschend sind das die beiden Stiele im Erdgescho{\ss}.

Eine verwandte Fragestellung ist die Suche nach dem maximalen Effekt den die Steifigkeits\"{a}nderung in {\em einem\/} Element in dem Rahmen hervorruft ({\em Focus auf ein Element\/}). Das kann man schreiben als
\begin{align}
max\,\, J(e) \qquad \text{{\em alle interessierenden\/} $G$}\,,
\end{align}
also als Suche nach der Kurve $M_G$, die mit $M_c$ auf dem Element \"{u}berlagert den gr\"{o}{\ss}ten Wert $J(e)$ liefert, wobei $M_G $ das Moment der Einflussfunktion ist, die zu $J(w)$ geh\"{o}rt.

%-----------------------------------------------------------------
\begin{figure}[tbp]
\centering
\includegraphics[width=0.9\textwidth]{\Fpath/U466}
\caption{Elastische Einspannung eines Tr\"{a}gers}
\label{U466}
\end{figure}%
%-----------------------------------------------------------------


%%%%%%%%%%%%%%%%%%%%%%%%%%%%%%%%%%%%%%%%%%%%%%%%%%%%%%%%%%%%%%%%%%%%%%%%%%%%%%%%%%%%%%%%%%%%%%%%%%%
\textcolor{sectionTitleBlue}{\section{Federn}}
Beispiele f\"{u}r Federn sind drehelastische Einspannungen in Rahmenecken oder Fundamenten. Wenn sich die Drehsteifigkeit \"{a}ndert, muss man nicht das ganze Tragwerk neu berechnen, sondern man kann sich den Effekt zu nutze machen, dass die \"{A}nderung $J(e) = - d(G,w_c)$ eines Funktionals -- also {\em eines Moments, einer Querkraft, einer Durchbiegung\/} -- direkt an der Feder verfolgt werden kann. Gerade bei komplexen 3-D Modellen mit einer gro{\ss}en Anzahl von Freiheitsgraden ist das ein nicht zu untersch\"{a}tzender Vorteil.

Die schwache Form der Balkengleichung des Tr\"{a}gers in Abb. \ref{U466} lautet\footnote{Wir schreiben die Drehfedersteifigkeit hier $k$ und nicht wie sonst $k_{\Np}$}
\begin{align}
\int_0^{\,l} \frac{M\,\delta M}{EI}\,dx + \delta w'\,w'\,k = \int_0^{\,l} p\,\delta w\,dx\,.
\end{align}
Vertrauter ist dem Leser wahrscheinlich der Ausdruck
\begin{align}
 \frac{\delta M\,M}{k} = \delta w'\,w'\,k \qquad (\delta M = k\,\delta w'\,, M = k\, w')\,.
\end{align}
\"{A}ndert sich die Drehfedersteifigkeit, $k \to k + \Delta k$, dann wird daraus
\begin{align}
\underbrace{\int_0^{\,l} \frac{M_c\,\delta M}{EI}\,dx + \delta w'\,w_c'\,k}_{a(w_c,\delta w)} +  \underbrace{\vphantom{\int_0^{\,l}} \Delta k\,w_c' \,\delta w'}_{d(w_c,\delta w)} = \int_0^{\,l} p\,\delta w\,dx\,.
\end{align}
Die \"{A}nderung in einem Funktional $J(w)$, $M_G$ ist das Moment der Einflussfunktion in der Feder vor der \"{A}nderung, ergibt sich somit zu
\begin{align}\label{Eq152}
J(e) = J(w_c) - J(w)  = - d(w_c,w_G) = - \Delta k\,w_c'\,w_G' = - \Delta k\,\frac{M_c}{k_c}\,\frac{M_G}{k} \,.
\end{align}
Zur Anwendung dieser Formel fehlt uns das Moment $M_c = k_c\,w_c'$. Dieses kann man jedoch, wenn man nicht die N\"{a}herung $M_c \sim M$ verwenden will, auf zwei Arten bestimmen: Durch {\em Iteration\/}, oder, wenn man auf die Inverse $K^{-1}$ Zugriff hat, durch {\em direkte L\"{o}sung\/}, \cite{Carl2}.

Den Zugang zur Iteration finden wir, wenn wir als Funktional das Moment in der Feder w\"{a}hlen, $J(w) = M$. Dann ist (\ref{Eq152}) \"{a}quivalent mit
\begin{align}
J(e) = M_c - M = - \Delta k\,\frac{M_c}{k_c}\,\frac{M_G}{k} \,.
\end{align}
Daraus kann man eine Iterationsvorschrift f\"{u}r eine Folge $M_c^{(i)}$ ableiten, die, wie sich zeigt, schnell gegen $M_c$ konvergiert
\begin{align}
M_c^{(i+1)} = - \Delta k\,\frac{M_c^{(i)}}{k_c}\,\frac{M_G}{k}  + M \qquad M_c^{(0)} = M\,.
\end{align}
Hat man $M_c = k_c\,w_c'$ bestimmt, kennt man auch $w_c'$ und dann kann man mit (\ref{Eq152}) die \"{A}nderung in jedem Funktional $J(w)$ verfolgen. Man braucht nur das Moment $M_G$ der Einflussfunktion des Funktionals $J(w)$ in der Feder. Das kann man aber am \glq alten\grq{} System bestimmen.

\"{A}ndern sich $m$ Federn in $m$ Punkten $x_i$, dann geht (\ref{Eq152}) in ein lineares Gleichungssystem \"{u}ber, das man wieder per Iteration l\"{o}sen kann, eventuell mit einer Konvergenzbeschleunigung wie auf S.  \pageref{Eq59} beschrieben,
\begin{align} \label{Eq156}
M_{c}^{(i+1)}(x_j) &= - \sum_{l = 1}^m \Delta k(x_l)\,\frac{M_{c}^{(i)}(x_l)}{k_{c}(x_l)}\,\frac{M_{G}(x_l,x_j)}{k(x_l)}  + M(x_j)  \quad j &= 1,2, \ldots m\,.
\end{align}
Hier ist $M_{G}(x_l,x_j)$ das Moment in der Feder $x_l$ (dem Ort), das von der Einflussfunktion f\"{u}r das Moment im Punkt $x_j$ (der Ursache) in $x_l$ erzeugt wird (Berechnung am \glq alten\grq{} System).

Die Stahlhallen in Abb. \ref{CarlStahlbau1} und \ref{CarlStahlbau2} wurden so untersucht. Bei dem ersten System ging es um den \"{U}bergang von einem starren Anschluss in den Rahmenecken zu einer drehelastischen Einspannung. Bei dem zweiten System ging es um den Einfluss der drehelastischen Einspannung der St\"{u}tzen auf die horizontale Verschiebung der Kranbahn in einem ausgew\"{a}hlten Punkt $x_P$. In einer ersten Statik waren die St\"{u}tzen voll eingespannt gerechnet worden.

Zuerst wurde der Punkt mit einer horizontalen Einzelkraft $P = 100$ kN belastet (100 wegen der \glq Sichtbarkeit\grq{}) und die Einspannmomente $M_{G}^P(x_i), i = 1,2,\ldots 8$ dieser Einflussfunktion (am \glq alten\grq{}, dem voll eingespannten System) in den acht Fundamenten berechnet. Dann wurde die Iterationsvorschrift (ein $8 \times 8$-System) zur Bestimmung der acht Einspannmomente $M_c(x_i)$ aus der Verkehrslast ({\em Kran (25t-Hublast) - vert. Randlasten + Schr\"{a}glaufkr\"{a}fte in Reihe 4\/}) an dem drehelastischen System aufgestellt und die $M_c(x_i)$ wie in (\ref{Eq156}) iterativ bestimmt. Wegen der urspr\"{u}nglichen Festeinspannung $k(x_i) = \infty$ aber mit der Modifikation wie in (\ref{Eq153}) beschrieben
\begin{align}
M_{c}^{(i+1)}(x_j) &= - \sum_{l = 1}^m \frac{1}{k_{c}(x_l)}\,M_{c}^{(i)}(x_l)\,M_{G}(x_l,x_j)  + M(x_j)  \quad j &= 1,2, \ldots m\,.
\end{align}
Mit den auf Eins normierten Momenten $M_{G}^P(x_i)/100$ und den $M_c(x_i)$ kann man dann die horizontale Verschiebung $u_c(x_P)$ berechnen
\begin{align}\label{Eq157}
u_c(x_P) = - \sum_{l = 1}^8 \frac{1}{k_c(x_l)} M_c(x_l)\, M_G^P(x_l) \frac{1}{100}\,.
\end{align}
Hat man einmal die Einspannmomente $M_c(x_l)$ der St\"{u}tzen aus dem ma{\ss}gebenden Lastfall ermittelt, dann kann man die \"{A}nderung in jeder anderen Weg- oder Kraftgr\"{o}{\ss}e $J(w)$ der Halle berechnen. Was man braucht, sind nur die acht Einspannmomente $M_G(x_l)$ (am \glq alten\grq{} System), die zu der Einflussfunktion f\"{u}r die jeweilige Weg- oder Kraftgr\"{o}{\ss}e $J(w)$ geh\"{o}ren, s. Abb. \ref{U474}. Diese mit den ge\"{a}nderten Momenten $M_c(x_l)$ aus der Verkehrslast \"{u}berlagert, sinngem\"{a}{\ss} wie in \ref{Eq157}, ergibt die \"{A}nderung $J(e)$.
%-----------------------------------------------------------------
\begin{figure}[tbp]
\centering
\includegraphics[width=0.9\textwidth]{\Fpath/CarlStahlbau1}
\caption{3D-Modell eines Rahmens mit 47 St\"{a}ben und 480 Freiheitsgraden. Die dreh\-elastische Einspannung reduzierte das Einspannmoment von $M = -86,6$ kNm auf $M_c = - 75,0$ kNm, \cite{Carl4} }
\label{CarlStahlbau1}
\end{figure}%
%-----------------------------------------------------------------

Die Alternative zur Iteration ist die direkte L\"{o}sung. Ist das Funktional $J(w) = w'$ der Tangens des Drehwinkels in der Feder,
\begin{align}\label{Eq151}
J(e) = w_c' - w' = -\Delta k\,w_c'\,w_G'\,,
\end{align}
dann kann man das nach $w_c'$ aufl\"{o}sen
\begin{align}
w_c' =   w'\,\frac{1}{1 + \Delta k\,w_G'}\,.
\end{align}
Wenn die Feder den Drehfreiheitsgrad $u_7$ hat, dann ist $w_G'$ der Eintrag $f_{7,7}$ der Flexibilit\"{a}tsmatrix $\vek F = \vek K^{-1}$. Mit $w_c'$ kann man dann das ge\"{a}nderte Moment $M_c = (k + \Delta k)\,w_c'$ in der Feder berechnen oder die \"{A}nderung $J(e)$ in jedem anderen Funktional, z.B. der Verschiebung $u_3$ in einem Knoten. Man muss  nur f\"{u}r $w_G'$ in (\ref{Eq152}) den entsprechenden Wert einsetzen. Hat die Drehfeder, wie angenommen, den Freiheitsgrad $u_7$, dann ist $w_G'$ gleich dem Eintrag $f_{3,7}$ in der Flexibilit\"{a}tsmatrix $\vek F = \vek K^{-1}$
\begin{align}
J(e) = u_{c 3} - u_3 = - \Delta k\,w_c'\,f_{3,7}\,.
\end{align}
\"{A}ndern sich die Steifigkeiten in zwei Federn, die an den Stellen $x_1$ und $x_2$ liegen, dann lautet der Zusatzterm zur schwachen Form
\begin{align}
d(w_c, \delta w) = \Delta k_1\,w_c'(x_1)\,\delta w'(x_1) + \Delta k_2\,w_c'(x_2)\,\delta w'(x_2)\,.
\end{align}
Das Ziel ist die Bestimmung von $w_c'(x_1)$ und $w_c'(x_2)$, um daraus z.B. die Momente
\begin{align}
M_c(x_1) = (k_1 + \Delta k_1)\,w_c'(x_1) \qquad M_c(x_2) = (k_2 + \Delta k_2)\,w_c'(x_2)
\end{align}
in den Federn zu berechnen.

Die Verdrehungen in den Federn sind zwei Funktionale
\begin{align}
J_1(w) &= w'(x_1) = \int_0^{\,l} G_1(y,x_1)\,p(y)\,dy \\
J_2(w) &= w'(x_2) = \int_0^{\,l} G_2(y,x_2)\,p(y)\,dy
\end{align}
zu denen die Einflussfunktionen $G_i(y,x)$ geh\"{o}ren. Die \"{A}nderungen in den Verdrehungen ergeben sich zu
%-----------------------------------------------------------------
\begin{figure}[tbp]
\centering
\includegraphics[width=0.99\textwidth]{\Fpath/CarlStahlbau2}
\caption{Stahlhalle, 1\,643 St\"{a}be, 5\,340 FG, 121 LF. Zu bestimmen war der Einfluss der drehelastischen Einspannung der acht St\"{u}tzen auf die horizontale Verschiebung in H\"{o}he der Kranbahn. Die hier vorgestellte Technik reduzierte das Problem auf ein $8 \times 8$ Gleichungssystem, das per Iteration gel\"{o}st wurde, \cite{Carl4}}
\label{CarlStahlbau2}
\end{figure}%
%-----------------------------------------------------------------
\begin{align}
J_1(e) &= w_c'(x_1) - w'(x_1) = - \Delta k_1\,w_c'(x_1)\,w_{G_1}'(x_1) - \Delta k_2\,w_c'(x_2)\,w_{G_2}'(x_1) \nn \\
J_2(e) &= w_c'(x_2) - w'(x_2) = - \Delta k_1\,w_c'(x_1)\,w_{G_1}'(x_2) - \Delta k_2\,w_c'(x_2)\,w_{G_2}'(x_2)\,,
\end{align}
was dem linearen (symmetrischen) Gleichungssystem
\begin{align}
\left[ \barr {r @{\hspace{4mm}}r @{\hspace{4mm}}r
@{\hspace{4mm}}r @{\hspace{4mm}}r}
      1 + a_{11} &  a_{12} \\
      a_{21} & 1 + a_{22}
     \earr \right]\left [\barr{c}  w_c'(x_1) \\  w_c'(x_2)\earr \right ]
=  \left [\barr{c}  w'(x_1) \\  w'(x_2)\earr \right ]
\end{align}
mit
\begin{align}
a_{11} = \Delta k_1\,w_{G_1}'(x_1) \qquad a_{12} = \Delta k_2\,w_{G_2}'(x_1)\qquad a_{22} = \Delta k_2\,w_{G_2}'(x_2)
\end{align}
entspricht. W\"{a}ren die Drehfreiheitsgrade in den beiden Federn $u_7$ und $u_9$, dann findet man die Koeffizienten $w_{G_i}'(x_j)$ wieder an den entsprechenden Stellen in der Inversen $\vek F = \vek K^{-1}$
\begin{align}
w_{G_1}'(x_1) = f_{7,7} \qquad w_{G_1}'(x_2) = f_{7,9} \qquad w_{G_2}'(x_1) = f_{9,7} \qquad w_{G_2}'(x_2) = f_{9,9}\,.
\end{align}
%-----------------------------------------------------------------
\begin{figure}[tbp]
\centering
\includegraphics[width=0.99\textwidth]{\Fpath/U474}
\caption{Die Einflussfunktion ($G$) f\"{u}r ein Moment, eine Querkraft, eine Durchbiegung erzeugt Momente ($M_y^G$) in den acht Fusspunkten der St\"{u}tzen, und dieser Vektor $\vek M_y^G$ skalar multipliziert mit dem Vektor $\vek M_c$ der acht Fusspunkts-Momente $k^{-1}_c(x_i)M_c(x_i)$ ergibt die \"{A}nderung $J(e) = \vek M_c^T \vek M_y^G$ des Moments, der Querkraft, etc. im Aufpunkt}
\label{U474}
\end{figure}%
%-----------------------------------------------------------------

Die \"{A}nderung in der horizontalen Verschiebung $u_3$ erg\"{a}be sich dann sinngem\"{a}{\ss} wie folgt
\begin{align}
J(e) = u_{c 3} - u_3 =- \Delta k_1\,w_c'(x_1)\,f_{3,7} - \Delta k_2\,w_c'(x_2)\,f_{3,9} \,.
\end{align}

\begin{remark}
Das hier dargestellte Verfahren ist theoretisch nicht anwendbar, wenn ein urspr\"{u}nglich starres Lager nachgibt, weil der neue Freiheitsgrad in dem System $\vek K$ nicht vorkommt. Dies sollte jedoch eine Ausnahme sein, weil man ja Lager mit ihrer nat\"{u}rlichen Steifigkeit $k < \infty$ rechnen sollte.
%-----------------------------------------------------------------
\begin{figure}[tbp]
\centering
\if \bild 2 \sidecaption \fi
\includegraphics[width=0.7\textwidth]{\Fpath/U469}
\caption{Bestimmung des Einspannmoments $X_1$} \label{U469}
\end{figure}%%
%-----------------------------------------------------------------

Man kann aber eine Umformung vornehmen, denn in der Grenze $k \to \infty$ gilt
\begin{align}\label{Eq153}
J(e) = -\Delta k\,w_c' \, w_G' = -\frac{k_c - k}{k_c\,k} M_c\, M_G = -(\frac{1}{k} - \frac{1}{k_c})\,M_c\, M_G = \frac{1}{k_c}\,M_c\, M_G\,,
\end{align}
wenn $M_G$ das Einspannmoment der Einflussfunktion ist.

Angewandt auf das Lager selbst mit dem unbekannten $M_c$ ergibt sich
\begin{align}\label{Eq154}
M_c - M = \frac{1}{k_c}\,M_c\, M_G\,,
\end{align}
wobei $M_G$ das Moment ist, das das gelenkig gemachte Balkenende um $\tan\,\Np = 1$ spreizt, und nun kann man per Iteration eine Folge $M_i$  bestimmen, die gegen $M_c$ konvergiert
\begin{align}
M_{i+1} = \frac{1}{k_c}\,M_i\, M_G + M \qquad i = 0, 1, 2, \ldots \qquad M_0 = M\,.
\end{align}
Ist $M_c$ bestimmt, dann kann man mit (\ref{Eq153}) jede \"{A}nderung $J(e)$ verfolgen, wenn man f\"{u}r $M_G$ das Einspannmoment setzt, das zur Einflussfunktion f\"{u}r $J(w)$ am urspr\"{u}nglichen Tragwerk geh\"{o}rt.

Das Moment $M_G$ in (\ref{Eq154}) ist das Moment $X_1$, das zwischen dem Balken und der Feder \"{u}bertragen wird, s. Abb. \ref{U469}. Mit
\begin{align}
\delta_{10 } = 1 \qquad \delta_{11} = \frac{\ell}{3\,EI} + \frac{1}{k} \qquad \delta_{11}\,X_1 + \delta_{10} = 0
\end{align}
folgt
\begin{align}
X_1 = \frac{-1}{\frac{\ell}{3\,EI} + \frac{1}{k}}
\end{align}
und damit in der Grenze, $k \to \infty$, $X_1 = 3\,EI/\ell = M_G$.
\end{remark}


%-----------------------------------------------------------------
\begin{figure}[tbp]
\centering
\includegraphics[width=1.0\textwidth]{\Fpath/U461}
\caption{Semi-integrale Br\"{u}cke auf Pf\"{a}hlen\textbf{ a)} FE-Modell \textbf{ b)} Einflussfunktion f\"{u}r das Moment $M(x)$}
\label{U461}
\end{figure}%
%-----------------------------------------------------------------

%%%%%%%%%%%%%%%%%%%%%%%%%%%%%%%%%%%%%%%%%%%%%%%%%%%%%%%%%%%%%%%%%%%%%%%%%%%%%%%%%%%%%%%%%%%%%%%%%%%
\textcolor{sectionTitleBlue}{\section{Integrale Br\"{u}cken}}
Eine beispielhafte Anwendung finden diese Ideen auch bei integralen Br\"{u}cken, bei denen die Widerlager, die Pfeiler und der \"{U}berbau monolithisch miteinander verbunden sind, um die Wartungsarbeiten zu vereinfachen.

Die Idee ist relativ neu und als daher im Zuge der Autobahnerneuerung der A3 die {\em Fahrbachtalbr\"{u}cke\/} bei Aschaffenburg, eine semi-integrale Br\"{u}cke (keine monolithische Verbindung mit den Widerlagern) errichtet werden sollte, hat das Br\"{u}ckenbauamt f\"{u}r Nordbayern darauf bestanden, dass Untersuchungen an Pf\"{a}hlen vorgenommen wurden, um den Einfluss der Grenzwerte $c \pm \Delta c$ der elastischen Bettung $c$ auf den \"{U}berbau absch\"{a}tzen zu k\"{o}nnen, \cite{Schiefer}.

In einer Diplomarbeit wurden die hier entwickelten Ideen benutzt, um dies rechnerisch zu verfolgen, \cite{Sopoth}. Das Beispiel eignet sich gut, weil ja zwischen den Aufpunkten im \"{U}berbau der Br\"{u}cke und den Pf\"{a}hlen eine relativ lange \glq Strecke\grq{} liegt, s. Abb. \ref{U461}, und der Einfluss den Weg praktisch zweimal gehen muss, vom \"{U}berbau zu den Pf\"{a}hlen um die Kr\"{a}fte $\vek f^+$ zu erzeugen und die Kr\"{a}fte $\vek f^+$ gehen denselben Weg zur\"{u}ck, um die Schnittgr\"{o}{\ss}en zu \"{a}ndern, $M(x) \to M_c(x)$, $V(x) \to V_c(x)$ etc. Eine Situation, die typisch f\"{u}r {\em Substrukturen\/}\index{Substrukturen} ist.

In Substrukturen spielen die Einfl\"{u}sse nach einer Steifigkeits\"{a}nderung \glq Ping-Pong\grq{}, wechseln die Signale zwischen dem Lastgurt und der Substruktur hin und her bis die Signale ausgeglichen sind. Die Kr\"{a}fte $\vek f + \vek f^+$ erzeugen die neue Gleichgewichtslage $\vek u_c$ und die muss genau so gro{\ss} sein, dass $\vek u_c$ die Kr\"{a}fte $\vek f^+$ erzeugt.

Zur\"{u}ck zu der Br\"{u}cke: Betrachten wir zum Beispiel das Moment $M(x)$ im \"{U}berbau an einer Stelle $x$, das sich durch die \"{U}berlagerung der Einflussfunktion $G_2(y,x)$ mit der Belastung $p$ berechnen l\"{a}sst
\begin{align}
M(x) = \int_0^{\,l} G_2(y,x)\,p(y)\,dy = \vek g^T\,\vek f\,.
\end{align}
Die Frage, wie sich das Moment \"{a}ndert, wenn sich der Bettungsmodul der Pf\"{a}hle \"{a}ndert, $c \to c + \Delta c$, zielt auf den Unterschied zwischen der Einflussfunktion $G_2$ (Modul $c$, Matrix $\vek K$) und der Einflussfunktion $G_{2 c}$, die am System $\vek K + \vek \Delta \vek K$ mit dem Modul $c + \Delta c$ berechnet wird
\begin{align}
M_c(x) = \int_0^{\,l} G_{2 c}(y,x)\,p(y)\,dy= \vek g_c^T\,\vek f \,.
\end{align}
Wegen
\begin{align}
\vek g_c^T\,\vek f = \vek g^T\,(\vek f + \vek f^+)
\end{align}
kann man das auf die Bedeutung des Vektors $\vek f^+$ f\"{u}r das System $\vek K$ zur\"{u}ckspielen.

Die Vektoren $\vek g$ und $\vek g_c$ sind die Knotenwerte der beiden Einflussfunktionen am Modell $\vek K$ bzw. $\vek K_c = \vek K + \vek \Delta \,\vek K$. Der Vektor $\vek f^+ = -\vek \Delta \vek K\,\vek u_c$  sind die Zusatzkr\"{a}fte in den Knoten der Pf\"{a}hle aus der \"{A}nderung des Bettungsmoduls, $c \to c + \Delta$.

Schreiben wir die Biegeverformung im Bereich der Pf\"{a}hle elementweise als Taylorreihe
\begin{align}
w_c(x) = w_c(0) + w_c'(0) \cdot x + \frac{1}{2}\,w_c''(0) \cdot x^2 + \ldots
\end{align}
dann liefern nur die quadratischen Terme Beitr\"{a}ge zu $\vek f^+ = -\vek \Delta \vek K\,\vek u_c$, weil in jedem Element die Zeilen von $\vek \Delta\,\vek K_e$ orthogonal zu Starrk\"{o}rperbewegungen sind -- den ersten beiden Termen. Die quadratischen Terme werden jedoch die kleinsten der drei Terme sein, so dass auch $\vek f^+$ relativ klein sein wird. Dazu kommt noch, dass die Pf\"{a}hle relativ weit vom \"{U}berbau entfernt liegen, so dass der Einfluss der Pfahlkr\"{a}fte $\vek f^+$ auf den \"{U}berbau, unabh\"{a}ngig von ihrer Gr\"{o}{\ss}e, relativ klein sein wird. Die Rechenergebnisse best\"{a}tigten das, die Momente $M(x)$ und $M_c(x)$ weichen kaum voneinander ab.

Man kann das Ganze auch andersherum aufz\"{a}umen, indem man direkt die \"{A}nderungen $G_2(y,x) \to G_{2c}(y,x)$ verfolgt. Die Knotenwerte $\vek g$ der Einflussfunktion $G_2$ sind die L\"{o}sung des Systems $\vek K\vek g = \vek j$ mit $j_i = M(\Np_i)(x)$ und die Knotenwerte $\vek g_c$ der Einflussfunktion $G_{2 c}$ sind die L\"{o}sung des Systems
\begin{align}
\vek K\,\vek g_c = \vek j + \vek j^+
\end{align}
mit dem Vektor $\vek j^+ = -\Delta \vek K\,\vek g_c$. Die durch die Spreizung des Aufpunktes $x$  erzeugte Bewegung $g_c(x) = g_c(0) + g_c'(0)\,x + 0.5 * g_c''(0)^2/x\ldots$ (in den Pf\"{a}hlen) d\"{u}rfte aber demselben Argument unterliegen wie oben. Die Eintr\"{a}ge in dem Vektor $\vek j^+ =-
\vek \Delta\,\vek K\,\vek g_c$ sollten relativ klein sein und weil die Knoten der Pf\"{a}hle vom \"{U}berbau relativ weit weg liegen, sollte der Einfluss der $\vek j^+$ klein sein und somit auch der Unterschied zwischen den Einflussfunktionen $G_2(y,x)$ und $G_{2c}(y,x)$.

%%%%%%%%%%%%%%%%%%%%%%%%%%%%%%%%%%%%%%%%%%%%%%%%%%%%%%%%%%%%%%%%%%%%%%%%%%%%%%%%%%%%%%%%%%%%%%%%%%%
\textcolor{sectionTitleBlue}{\section{Klassische Formulierung}}
In der Optimierung\index{Optimierung} nennt man das Operieren mit Einflussfunktionen die {\em adjoint method of analysis\/}\index{adjoint method of analysis}. Wir wollen daher zeigen, dass die Formel
\begin{align}\label{Eq12}
J(e) = J(u_c) - J(u) \simeq - d(G,u)
\end{align}
mit den klassischen Ergebnissen der Sensitivit\"{a}tsanalyse identisch ist. Die exakte Formel w\"{a}re $J(e) = -d(G_c,u)$ oder $J(e) = - d(G,u_c)$.


Wir nehmen an, dass die Steifigkeitsmatrix
\begin{align}
\vek K = \vek K(p_1, p_2, \ldots, p_m)
\end{align}
eine Funktion von $m$ Modellparametern $p_i$ ist. Wir wollen die Sensitivit\"{a}t des Funktionals $J(\vek u) = \vek j^T\,\vek u $ bez\"{u}glich der Parameter $p_i$ bestimmen
\begin{align} \label{Eq9}
\frac{d\,J}{d p_i} = \frac{d\,}{d p_i} (\vek j^T\,\vek u) = \frac{d}{d p_i} \vek j^T\,\vek u + \vek j^T\,\frac{\partial }{\partial  p_i}\,\vek u\,.
\end{align}
Weil $\vek K\,\vek u - \vek f = \vek 0$ ein konstanter Vektor ist (als Funktion der $p_i$), sind die Ableitungen null
\begin{align}
\frac{d}{d p_i} (\vek K\,\vek u - \vek f) = \frac{\partial \vek K}{\partial p_i}\,\vek u + \vek K\,\frac{\partial \vek u}{\partial p_i} - \frac{\partial \vek f}{\partial p_i} =0\,,
\end{align}
was
\begin{align} \label{F2}
\frac{\partial \vek u}{\partial p_i} = - \vek K^{-1}\,\frac{\partial \vek K}{\partial p_i}\,\vek u
\end{align}
ergibt, wenn wir annehmen dass $\partial \vek f/\partial p_i = 0$ ist. Wenn wir dieses Ergebnis in (\ref{Eq9}) einsetzen, dann folgt
\begin{align}\label{Eq11}
\frac{d\,J}{d p_i} &= \frac{d}{d p_i} \vek j^T\,\vek u - \vek j^T\,\vek K^{-1}\,\vek u
= \frac{d}{d p_i}\,\vek j^T\,\vek u - \vek j^T\,\vek K^{-1} \frac{\partial \vek K}{\partial p_i}\,\vek u \nn \\
&= \frac{d}{d p_i} \vek j^T\,\vek u - \vek g^T \,\frac{\partial \vek K}{\partial p_i}\,\vek u\,.
\end{align}
Der Vektor
\begin{align}
\frac{\partial \vek K}{\partial p_i}\,\vek u
\end{align}
der genau dem Ausdruck $\vek \Delta \vek K\,\vek u$ entspricht, wird {\em Pseudo-Lastvektor\/}\index{Pseudo-Lastvektor} genannt.

Man beachte, dass wir hier nicht die \"{A}nderung $J(u_c) - J(u)$ berechnen, sondern nur den Gradienten
\begin{align}
\vek \nabla J(u) = \{\frac{d\,J}{d p_1}, \ldots, \frac{d\,J}{d p_m}\}^T
\end{align}
 von $J(u)$, der dann erst mittels Taylor die N\"{a}herung
\begin{align}
J(u_c) - J(u) \simeq \vek \nabla J(u) \dotprod \vek \Delta \vek p = d(G,u)
\end{align}
ergibt, wenn $\vek p$ sich nach $\vek p + \vek \Delta \vek p$ entwickelt. Der Ausdruck $d(G,u) \equiv J' \cdot \Delta p$ in (\ref{Eq12}) ist sozusagen das  Differential, das Inkrement des Funktionals, w\"{a}hrend $dJ/dp_i = J'$ nur die Ableitung ist.

Der erste Term in (\ref{Eq11}) spielt eine Rolle, wenn das Funktional von den $p_i$ abh\"{a}ngt. Das ist z.B. der Fall, wenn $J(\vek u)$ eine Kraftgr\"{o}{\ss}e in dem Element ist, dessen Steifigkeit sich \"{a}ndert, denn weil in die Definition der Schnittkr\"{a}fte die Steifigkeiten eingehen, etwa $J(u) = N(x)  = EA\,u'(x)$ muss auch das Funktional korrigiert werden, wenn sich $EA$ \"{a}ndert.

Der Vektor $\vek g$ in (\ref{Eq11}) ist nat\"{u}rlich der Vektor $\vek g$, der zu dem Funktional
geh\"{o}rt,
\begin{align}
J(\vek u) = \vek j^T\,\vek u = (\vek K\,\vek g)^T\,\vek u = \vek g^T\,\vek K\,\vek u = \vek g^T\,\vek f\,.
\end{align}
In der {\em adjoint method of analysis\/} hei{\ss}t $\vek g$ die {\em adjoint variable\/}\index{adjoint variable} und wird oft mit $\vek \lambda$ bezeichnet, also $\vek K\,\vek \lambda = \vek j$ statt $\vek K\,\vek g = \vek j$.

%%%%%%%%%%%%%%%%%%%%%%%%%%%%%%%%%%%%%%%%%%%%%%%%%%%%%%%%%%%%%%%%%%%%%%%%%%%%%%%%%%%%%%%%%%%%%%%%%%%
\textcolor{sectionTitleBlue}{\section{Direkte Differentiation}}
Noch k\"{u}rzer kann man die Ergebnisse mit der {\em direkten Differentiation\/}\index{direkte Differentiation} herleiten. Das Bauteil setze sich aus zwei Elementen zusammen und jedes Element habe einen eigenen E-Modul $E_i$, so dass die Steifigkeitsmatrix $\vek K$ eine Funktion dieser zwei Werte $E_i$ ist
\begin{align}
\vek K(E_1,E_2) \,\vek u = \vek f\,.
\end{align}
Mit $\vek K$ ist nat\"{u}rlich auch der Vektor $\vek u$ eine Funktion der $E_i$. Nun kann man fragen: Wie \"{a}ndert sich der Vektor $\vek u \rightarrow\,\vek u_c$, wenn sich der Wert $E_1$ im ersten Element \"{a}ndert?

Bezeichnen wir die Ableitung nach $E_1$ mit $(')$, dann gilt, wenn wir annehmen, dass der Vektor $\vek f$ nicht von $E_1$ abh\"{a}ngt
\begin{align}
\vek K'\,\vek u + \vek K\,\vek u' = \vek 0
\end{align}
oder
\begin{align}
\vek u' = -\vek K^{(-1)}\,\vek K'\,\vek u\,.
\end{align}
Brechen wir die Taylor-Entwicklung nach dem ersten Glied ab, dann gilt n\"{a}herungsweise
\begin{align}
\vek u_c - \vek u \sim \vek u' \cdot \Delta E_1 = -\vek K^{(-1)}\,\vek K'\,\vek u \cdot \Delta E_1 = -\vek K^{(-1)} \vek \Delta \vek K\,\vek u = -\vek K^{(-1)}\,\tilde{\vek f}^+\,.
\end{align}
Wenn hier $\vek \Delta \vek K\,\vek u_c$ st\"{u}nde, dann w\"{a}re $\tilde{\vek f}^+$ der exakte Vektor $\vek f^+$.
Der naheliegende Vorschlag, dass man $\vek f^+ = \vek \Delta\,\vek K\,\vek u_c$ durch $\vek f^+ \sim \vek \Delta\,\vek K\,\vek u$ ann\"{a}hert, entspricht also einer linearen Interpolation.

%%%%%%%%%%%%%%%%%%%%%%%%%%%%%%%%%%%%%%%%%%%%%%%%%%%%%%%%%%%%%%%%%%%%%%%%%%%%%%%%%%%%%%%%%%%%%%%%%%%
\textcolor{sectionTitleBlue}{\section{Berechnung von $\vek u_c$}}\label{Eq59}
Das Grundproblem bei der Reanalysis,
\begin{align}
J(\vek e) = - \vek g^T\,\vek \Delta \vek K\,\vek u_c
\end{align}
ist, dass die neue Gleichgewichtslage $\vek u_c$ des Tragwerks unbekannt ist bzw. ersatzweise durch den alten Vektor $\vek u$ angen\"{a}hert werden muss. Es gibt aber zwei Techniken {\em Iteration\/} und {\em direkte Bestimmung\/}, mit denen sich das Problem l\"{o}sen l\"{a}sst.

\textcolor{sectionTitleBlue}{\subsection{Iteration}}
Wir multiplizieren die Gleichung
\begin{align}\label{F2}
(\vek K + \vek \Delta \vek K)\,\vek u_c = \vek f
\end{align}
von links mit $\vek K^{-1}$
\begin{align} \label{F7}
(\vek I + \vek K^{-1}\,\vek \Delta\,\vek K)\,\vek u_c = \vek u
\end{align}
und schreiben Sie in der Form
\begin{align}\label{F6}
\vek u_c = - \vek K^{-1}\,\vek \Delta\,\vek K\,\vek u_c + \vek u
\end{align}
was die folgende {\em fixed-point iteration\/} f\"{u}r $\vek u_c$
\begin{align}
\vek u_c^{i+1} = - \vek K^{-1}\,\vek \Delta\,\vek K\,\vek u_c^i + \vek u \qquad i = 1,2,\ldots
\end{align}
mit $\vek u^0 = \vek u$ als Startvektor nahelegt.

Die Konvergenz kann man beschleunigen, wenn man den Vektor $\vek u_c^{i+1}$ mit dem vorhergehenden Vektor $\vek u_c^i$ wichtet, \cite{Carl2},
\begin{align}
\vek u_c^{i+1} = \vek u_c^i \cdot c_{max} \cdot q + \vek u_c^{i+1} \cdot q = \frac{1}{1 + c_{max}} \cdot (\vek u_c^i \cdot c_{max} + \vek u_c^{i+1})
\end{align}
wobei $c_{max}$ der gr\"{o}{\ss}te Skalenfaktor $c_e$ unter den \"{A}nderungen, $\vek \Delta \vek K_e = c_e\,\vek K_e$ ist und
\begin{align}
q = \frac{1}{1 + c_{max}}\,.
\end{align}
Theoretisch versagt die Iteration, wenn das Tragwerk durch die Wegnahme eines Elementes kinematisch wird, wenn also die Steifigkeitsmatrix
\begin{align}
\vek K_c = \vek K + \vek \Delta \vek  K
\end{align}
singul\"{a}r ist. Denn dann gibt es einen Vektor $\vek u_0 \neq \vek 0$ so, dass
\begin{align}
(\vek K +  \vek \Delta \vek K) \vek u_0 = \vek 0\,,
\end{align}
und wenn wir diese Gleichung von links mit der Inversen $\vek K^{-1}$ multiplizieren, so zeigt sich,
\begin{align}
\vek K^{-1} (\vek K +  \vek \Delta \vek K) \vek u_0 = (\vek I - \vek K^{-1}  \vek K) \vek u_0 = \vek 0\,,
\end{align}
dass der Vektor $\vek u_0$ ein Eigenvektor von $\vek K^{-1} \vek \Delta \vek K$  mit dem Eigenwert 1 ist und der Fehler $\vek e_{i+1} = \vek u_{i+1} - \vek u_i$ daher nicht schrumpft
\begin{align}
\vek e_{i+1} = - \vek K^{-1}  \vek \Delta \vek K \vek e_i   \qquad    i = 1,2,\ldots\,,
\end{align}
weil eine Fixpunktiteration nur Erfolg hat, wenn die Eigenwerte der Iterationsmatrix kleiner als 1 sind.

Man beachte, dass
\begin{align}
\vek e_{i+1} &= \vek u^{i+1} - \vek u^i = - \vek K^{-1} \vek \Delta  \vek K \vek u_i  + \vek u  + \vek K^{-1} \vek \Delta  \vek K \vek u_{i-1} - \vek u\\
 &= - \vek K^{-1} \vek \Delta  \vek K (\vek u_i - \vek u_{i-1}) = - \vek K^{-1} \vek \Delta  \vek K \vek e_i\,.
\end{align}
Aber es ist bemerkenswert, dass bei nicht zu gro{\ss}en St\"{o}rungen die Iteration konvergiert, selbst dann wenn das Entfernen eines Elementes das Tragwerk instabil macht, weil das Programm nicht versucht, die nicht existierende Inverse der singul\"{a}ren Steifigkeitsmatrix $\vek K + \vek \Delta \vek K$ zu berechnen.

\textcolor{sectionTitleBlue}{\subsection{Direkte Berechnung -- \glq von 2 auf 100\grq{}}}
Wiederholen wir noch einmal die Gleichung, die zu l\"{o}sen ist
\begin{align} \label{FX7}
(\vek I + \vek K^{-1}\,\vek \Delta\,\vek K)\,\vek u_c = \vek u\,.
\end{align}
Das Produkt $ \label{F1} \vek K^{-1}\,\vek \Delta \vek K $
ist eine schwach besetzte Matrix, weil die zur vollen Gr\"{o}{\ss}e $n \times n$ erweiterte Elementmatrix $\vek \Delta\,\vek K$ viel \glq Luft\grq{}, viele Nullen enth\"{a}lt. Nehmen wir an, dass die Matrix $\vek \Delta \vek K$ nur vier Eintr\"{a}ge enth\"{a}lt, die nicht null sind
\begin{align}
\vek \Delta \vek K = \left[\barr{r r r r r r r r}
\phantom{0} & \phantom{0}  & \phantom{0}& ... & \phantom{0} &\phantom{0} &\phantom{0}\\
.&0 & 0  & 0 & 0 &0 &0 &.\\
.&0 & k^\Delta_{33}  & 0 & 0 &k^\Delta_{35} & 0&.\\
.&0 & 0  & 0 & 0 &0 &0 &.\\
.&0 & k^\Delta_{53}  & 0 & 0 &k^\Delta_{55} & 0&.\\
.&0 & 0  & 0 & 0 &0 &0&.\\
\phantom{0} &\phantom{0} & \phantom{0}  & ... & \phantom{0} &\phantom{0} &\phantom{0}
\earr\right]
\end{align}
zwei in Zeile $3$ und zwei in Zeile $5$ (wenn wir zum Beispiel ein Federelement \"{a}ndern).

Das Produkt von zwei $n \times n$ Matrizen $\vek A$ und $\vek B$ kann als die Summe von $n$ Matrizen geschrieben werden,
\begin{align}
\vek A \vek B = \vek c_1\,\vek r_1 + \vek c_2\,\vek r_2 + \ldots + \vek c_n \vek r_n\,,
\end{align}
Spalte $\vek c_1$ von $\vek A$ mal Zeile $\vek r_1$ von $\vek B$ plus Spalte $\vek c_2$ von $\vek A$ mal Zeile $\vek r_2$ von $\vek B$ etc. Jedes Produkt $\vek c_i\,\vek r_i$ ist eine $(n \times n)$ Matrix.

Weil $\vek \Delta\vek K$ schwach besetzt ist, es enth\"{a}lt ja nur zwei Zeilen, die nicht null sind, folgt
\begin{align}
\vek K^{-1} \vek \Delta\vek K = \vek c_3\, \vek r_3 + \vek c_5\, \vek r_5 \qquad (\text{Summe von zwei $n \times n$ Matrizen})
\end{align}
was, wenn wir das in Spaltenform schreiben, das  Ergebnis
\begin{align}
\vek K^{-1} \vek \Delta\vek K = \left[\vek 0\quad \vek  0\quad \vek s_3 \quad \vek 0 \quad\vek s_5\quad\vek 0\,\ldots \,\vek 0\right] \qquad \text{Spaltenvektoren}
\end{align}
ergibt, mit den beiden Spalten
\begin{align}
\vek s_3 = k^\Delta_{33} \cdot \vek c_3 + k^\Delta_{5 3}\cdot \vek c_5  \qquad \vek s_5 = k^\Delta_{3 5}\cdot\vek c_3 + k^\Delta_{5 5}\cdot\vek c_5\,,
\end{align}
die Linearkombinationen der beiden Spalten $\vek c_3$ und $\vek c_5$ von $\vek K^{-1}$ sind.
%-----------------------------------------------------------------
\begin{figure}[tbp]
\centering
\includegraphics[width=1.0\textwidth]{\Fpath/U353}
\caption{One-Click Reanalysis (ohne Neuberechnung der Steifigkeitsmatrix) \textbf{ a)} die angeklickten Elemente \textbf{ b)} die Momentenverteilung ohne die beiden St\"{u}tzen}
\label{U353}
\end{figure}%
%-----------------------------------------------------------------
Somit geht das System (\ref{FX7}) \"{u}ber in
\begin{align}\label{Eq8}
\vek u_c + \vek s_3\,u_{c 3} + \vek s_5 \,u_{c 5} = \vek u\,.
\end{align}
Wir w\"{a}hlen die Zeilen drei und f\"{u}nf dieses Systems und bestimmen an Hand dieser beiden Gleichungen die beiden Werte $u_{c 3}$ und $u_{c 5}$
\begin{align} \label{F3}
 \left[\barr{c c  } 1 + s_{33} & s_{35} \\
 s_{53} & 1 + s_{55} \earr\right] \left[\barr{c } u_{c3} \\u_{c5}\earr\right]= \left[\barr{c } u_{3} \\u_{5}\earr\right]\,.
\end{align}
Die Kenntnis der beiden Zahlen $u_{c 3}$ und $u_{c 5}$ reicht aus, das ist das Erstaunliche, um die ganze L\"{o}sung zu bestimmen
\begin{align}\label{Eq148}
\vek u_c &= - \vek K^{-1}\,\vek \Delta \vek K\,\vek u_c + \vek u = - u_{c 3}\,\vek s_3 - u_{c 5}\,\vek s_5 + \vek u \nn \\
&= - (u_{c3}\,k^\Delta_{33} + u_{c5}\,k^\Delta_{35})\,\vek c_3 - (u_{c5}\,k^\Delta_{53} + u_{c5}\,k^\Delta_{55})\,\vek c_5 + \vek u \nn \\
&= \alpha_3 \cdot \vek c_3 + \alpha_5 \cdot \vek c_5 + \vek u\,.
\end{align}
Bei Betrachtung von (\ref{Eq148}) und (\ref{Eq149}) (s.u.) erkennt man das Muster: \\

\hspace*{-12pt}\colorbox{highlightBlue}{\parbox{0.98\textwidth}{
Der neue Vektor $\vek u_c$ ist eine Summe aus den Spalten $\vek c_i$ der Steifigkeiten $k_i$, mit Vorfaktoren $\alpha_i$, die sich aus einem linearen System wie (\ref{F3}) ergeben, plus dem alten Vektor $\vek u$.}}\\

Um die Technik allgemeiner zu fassen, schreiben wir die beiden Matrizen als eine Folge von Spaltenvektoren $\vek c_i$ bzw. Zeilenvektoren $\vek r_i$
\begin{align}
\vek K^{-1} = [\vek c_1, \vek c_2, \ldots, \vek c_n] \qquad \vek \Delta \vek K = [\vek r_1, \vek r_2, \ldots, \vek r_n]^T\,.
\end{align}
Das System $\vek K^{-1}\,\vek \Delta K = [\vek s_1, \vek s_2, \ldots, \vek s_n] $ hat dann die Spalten
\begin{align}
\vek s_i = \sum_{j = 1}^n\,\vek c_j\,r_{ji}
\end{align}
und so lautet das Gleichungssystem (\ref{FX7}) zeilenweise
\begin{align}
u_{ci} + \sum_{j = 1}^n s_{ij}\,u_{cj} = u_i \qquad i = 1,2,\ldots, n\,.
\end{align}
Weil die meisten Zeilen $\vek r_i$ null sein werden, wird sich das wieder auf ein wesentlich kleineres System reduzieren lassen, das man wie eben erl\"{a}utert behandeln kann.

Man kann so auch die ganze Inverse $\vek K_c^{-1} = [\vek c_1^c, \vek c_2^c, \ldots, \vek c_n^c]$ spaltenweise berechnen, indem man f\"{u}r $\vek u$ nacheinander die alten Spalten $\vek c_i$ setzt.

Die \glq offizielle\grq{} Methode um $(\vek K + \vek \Delta \vek K)^{-1}$ aus $\vek K^{-1}$ zu berechnen, ist die {\em Sherman-Morrison-Woodbury\/}-Formel, \cite{Golub},\index{Sherman-Morrison-Woodbury}\label{Korrektur14}
\begin{align}
(\vek K + \vek \Delta\,\vek K)^{-1} = \vek K^{-1} - \vek K^{-1} \vek U\,(\vek I + \vek V\,\vek K^{-1}\,\vek U)^{-1} \,\vek V\,\vek K^{-1}
\end{align}
wobei $\vek \Delta \vek K = \vek U\,\vek V^T$. Sie gleicht einer {\em black box\/}, bei der es schwerf\"{a}llt, den statischen Gehalt hinter der Formel zu entdecken.

\"{A}ndert sich nur die L\"{a}ngssteifigkeit $k \to k + \Delta k$ in einer St\"{u}tze, dann muss man sogar nur eine skalare Gleichung l\"{o}sen, um die neue Gleichgewichtslage $\vek u_c$ zu finden.

Sei $u_3$ die Zusammendr\"{u}ckung der St\"{u}tze, dann reduziert sich das obige System  auf die Gleichung
\begin{align}
(1 + s_{33})\, u_{c3} = u_3 \qquad s_{33} = \Delta k \cdot c_{33}\,,
\end{align}
wenn $\vek c_{3}$ die Spalte 3 der Inversen $\vek K^{-1}$ ist, und die ge\"{a}nderte L\"{o}sung ergibt sich somit zu
\begin{align}\label{Eq149}
\vek u_c = - u_{c 3}\,\vek s_3 + \vek u = \boxed{-\frac{u_3}{(1 + \Delta k \cdot c_{33})}\,\Delta k } \cdot \vek c_3\ + \vek u\,.
\end{align}
Der Vektor $\vek c_3$ sind die Koeffizienten der Einflussfunktion f\"{u}r die Zusammendr\"{u}ckung $u_3$ der St\"{u}tze und eine \"{A}nderung $\Delta k$ ist daher dann merklich, wenn der Vorfaktor vor $\vek c_3$ gro{\ss} ist und die Einflussfunktion weit ausstrahlt.

%%%%%%%%%%%%%%%%%%%%%%%%%%%%%%%%%%%%%%%%%%%%%%%%%%%%%%%%%%%%%%%%%%%%%%%%%%%%%%%%%%%%%%%%%%%%%%%%%%%%%%%%
\textcolor{sectionTitleBlue}{\section{One-Click Reanalysis}}
In dem Programm BE-FRAMES, s. S. \pageref{SoftwareDownload}, sind beide Techniken, Iteration und direkte L\"{o}sung, zu Lehrzwecken als {\em One-Click Reanalysis\/} implementiert, s. Abb. \ref{U353}. Der Student kann durch einfache Klicks Ver\"{a}nderungen an einem Rahmen vornehmen und die Effekte, die dadurch entstehen, studieren\index{BE-FRAMES}.

Implementiert sind \"{A}nderungen der Art $\vek \Delta \vek K_e = c \cdot \vek K_e$,
also eine Skalierung einzelner Elementmatrizen mit unterschiedlichen Faktoren $c$, wobei $c = 0$ einem kompletten Verlust des Elements entspricht.

Eine Serie von Modifikationen, etwa \"{A}nderungen der Elemente  5, 7 und 9, bedeutet einfach, dass $\vek \Delta \vek K$ eine Ansammlung von skalierten Elementmatrizen $\vek \Delta \vek K_e$ ist
\begin{align}
\vek u_c = - \vek K^{-1}\,(\vek \Delta\,\vek K_5 + \vek \Delta\,\vek K_7 + \vek \Delta\,\vek K_9 )\,\vek u_c + \vek u \qquad \text{(direkte Lsg.)}
\end{align}
Nat\"{u}rlich k\"{o}nnen auch einzelne Lager entfernt werden, nachtr\"{a}glich Gelenke eingebaut werden (das geht mit Dirac Deltas) und Einflussfunktionen f\"{u}r alle interessierenden Gr\"{o}{\ss}en berechnet werden.
%-----------------------------------------------------------------
\begin{figure}[tbp]
\centering
\includegraphics[width=1.0\textwidth]{\Fpath/U363}
\caption{Verschieblichkeiten in einem Rahmen lassen sich mit dem \glq zweifachen\grq{} Gauss entdecken}
\label{U363}
\end{figure}%
%-----------------------------------------------------------------

\textcolor{sectionTitleBlue}{\subsection{Wenn die Last \glq getroffen\grq{} wird}}

Wenn der Student ein Element $\Omega_e$ entfernt, das eine Belastung tr\"{a}gt, dann ist der neue Vektor $\vek u_c$ die L\"{o}sung des Systems
\begin{align}
(\vek K + \vek \Delta \vek K)\,\vek u_c = \vek f - \vek f_e\,,
\end{align}
wobei die Eintr\"{a}ge in dem Vektor $\vek f_e$ die vorherigen \"{a}quivalenten Knotenkr\"{a}fte aus dem Element $\Omega_e$ sind. In diesem Fall muss das Programm also auch das Original der rechten Seite \"{a}ndern, $\vek f \to  \vek f - \vek f_e$.

%%%%%%%%%%%%%%%%%%%%%%%%%%%%%%%%%%%%%%%%%%%%%%%%%%%%%%%%%%%%%%%%%%%%%%%%%%%%%%%%%%%%%%%%%%%%%%%%%%%%%%%%
\textcolor{sectionTitleBlue}{\subsection{Singul\"{a}re Steifigkeitsmatrizen}}
Weil es in dem Programm BE-FRAMES implementiert ist, sei noch erw\"{a}hnt, dass man mit dem Gau{\ss} Algorithmus auch Beweglichkeiten in Rahmen aufdecken kann. Der Gau{\ss} Algorithmus formt ja die Steifigkeitsmatrix in eine obere Dreiecksmatrix um. Geht man noch einen Schritt weiter, und wendet den Gau{\ss} Algorithmus auch auf die obere Dreiecksmatrix an, dann erh\"{a}lt man am Schluss -- bei regul\"{a}ren Matrizen -- die Einheitsmatrix. Wenn allerdings die Matrix singul\"{a}r ist, dann f\"{u}hren die letzten Spalten in dem Ergebnis auf die Eigenvektoren zu dem Eigenwert $\lambda = 0$, also einfach die  m\"{o}glichen Starrk\"{o}rperbewegungen, die in dem Rahmen stecken, $\vek K\,\vek u = \vek 0$. Die kann man dann auf dem Bildschirm anzeigen und so den Benutzer auf fehlende Festhaltungen aufmerksam machen, s. Abb. \ref{U363}.

Die Matrix, die bei diesem \glq zweimaligen\grq{} Gauss entsteht, nennt man die {\em row reduced echelon form\/} \index{row reduced echelon form} einer Matrix und unter diesem Namen findet man den Algorithmus auch in der Literatur.

%-----------------------------------------------------------------
\begin{figure}[tbp]
\centering
\includegraphics[width=0.95\textwidth]{\Fpath/U326D}
\caption{Der Einbau von plastischen Gelenken in einem Durchlauftr\"{a}ger \textbf{ a)} urspr\"{u}ngliche Momentenverteilung, \textbf{ b)} nach dem Einbau der Gelenke, \textbf{ c)} Biegelinie}
\label{U326}
\end{figure}%
%-----------------------------------------------------------------

%%%%%%%%%%%%%%%%%%%%%%%%%%%%%%%%%%%%%%%%%%%%%%%%%%%%%%%%%%%%%%%%%%%%%%%%%%%%%%%%%%%%%%%%%%%%%%%%%%%%%%%%
\textcolor{sectionTitleBlue}{\section{Nachtr\"{a}glicher Einbau von Gelenken}}
Andere m\"{o}gliche Defekte, die sich unter Last ausbilden, sind plastische Gelenke. Mathematisch ist  ein Gelenk eine Nullstelle im Momentenverlauf, $M(x) = 0$. Was wir daher machen ist, dass wir ein Dirac Delta $\delta_2$ im Punkt $x$ wirken lassen, d.h. wir berechnen die Einflussfunktion f\"{u}r  $M(x)$ in diesem Punkt und wir berechnen, wie gro{\ss} das Moment $M_2(x)$ der Einflussfunktion selbst in diesem Punkt $x$ ist. Dann skalieren wir das Dirac Delta mit einem Faktor $a$ derart, dass $a \cdot M_2(x) + M(x) = 0$.

So wurden die Ergebnisse in Abb. \ref{U326} erzielt. Auch dies steht als \glq {\em one-click\/}\grq{} operation in dem Programm zur Verf\"{u}gung.

Im Detail geht es wie folgt: Es sei $x_0$  der Punkt, an dem ein Gelenk eingebaut werden soll -- oder besser -- ein Moment $M_p$ aus einem LF $p$ zu Null gemacht werden soll. Der Punkt muss nicht am Ende eines Feldes liegen, wie die Punkte in Abb. \ref{U326}.

(1) Man berechnet zun\"{a}chst die Einflussfunktion $G_2$ f\"{u}r das Moment $M$ im Punkt $x_0$, indem man in den  Nachbarknoten die \"{a}quivalenten Knotenkr\"{a}fte, die zu der Einflussfunktion geh\"{o}ren, s. Abb. \ref{U451}, aufbringt. (2) Man addiert zu dieser L\"{o}sung die lokale L\"{o}sung, also die Einflussfunktion am beidseitig eingespannten Tr\"{a}ger. (3) Man berechnet das Moment $M_G(x)$ der so zusammengesetzten Einflussfunktion im Punkt $x_0$. (4) Man skaliert dann den LF Einflussfunktion so, dass das Moment $M_{G}(x_0)$ gerade $- M_p(x_0)$ ist. (5) Die Ergebnisse dieses Lastfalls zu dem LF $p$ addiert, sind die Ergebnisse mit einem Gelenk im Punkt $x_0$ im LF $p$.

Werden mehrere Gelenke eingebaut, dann muss man die Spreizung der Gelenke untereinander abstimmen, also ein lineares Gleichungssystem l\"{o}sen, so dass in allen Punkten gleichzeitig die Momente gegengleich zu den Momenten im LF $p$ sind.

%%%%%%%%%%%%%%%%%%%%%%%%%%%%%%%%%%%%%%%%%%%%%%%%%%%%%%%%%%%%%%%%%%%%%%%%%%%%%%%%%%%%%%%%%%%%%%%%%%%
\textcolor{sectionTitleBlue}{\section{Knicklasten}}
Auch die Knicklasten\index{Knicklasten} -- also  die Eigenwerte der Steifigkeitsmatrix $\vek K$ (nach Theorie II. Ordnung) -- \"{a}ndern sich, wenn sich Steifigkeiten \"{a}ndern. Das ist ein komplexes Thema, weil ja die Rechentechnik des Ingenieurs, Stichwort: {\em Nachweis am Einzelstab\/}, mindestens eine genauso gro{\ss}e Rolle spielt, wie die Mathematik f\"{u}r die wir z.B. auf \cite{Bangerth} verweisen.

Wir wollen hier nur das {\em Hellmann-Feynman-Theorem\/} zitieren, das aus der Quantenmechanik stammt, und das festh\"{a}lt, wie sich die Eigenwerte eines Systems \"{a}ndern, wenn sich die Modellparameter \"{a}ndern.

Gegeben sei das Eigenwertproblem
\bfo
(\vek K(\vek p)  - \lambda\,\vek  I)\,\vek x = \vek 0\,.
\efo
Die $(n \times n)$-Matrix $\vek K(\vek p)$ h\"{a}nge von $m$ Modellparametern $p_i$ ab. Es
sei $\lambda$ ein Eigenwert und $\vek x$ ein zugeh\"{o}riger, normierter Eigenvektor, $|\vek
x| = 1$.

Dann gilt
\bfo
\frac{\partial \lambda}{\partial p_i} = \vek x^T\,\vek K,_{p_i}\,\vek x\,.
\efo
{\em Beweis:\/} Weil $\lambda$ ein Eigenwert ist, gilt
\bfo
\lambda = \vek x^T\,\vek K\,\vek x\,.
\efo
Differenziert man beide Seiten nach $p_i$ so folgt
\bfo
\frac{\partial \lambda}{\partial p_i} = \vek x^T\,\vek K,_{p_i}\,\vek x + \vek
x,_{p_i}^T\,\vek K\,\vek x + \vek x^T\,\vek K\,\vek x,_{p_i}\,.
\efo
Nachdem $\vek x = \vek x(\vek p)$ ein Eigenvektor ist, kann dies geschrieben werden als
\begin{align}
\frac{\partial \lambda}{\partial p_i} &= \vek x^T\,\vek K,_{p_i}\,\vek x + \lambda(\vek
p)\,\vek x,_{p_i}^T\,\vek x + \lambda(\vek p)\,\vek x^T\,\vek x,_{p_i} \nn \\
&=  \vek x^T\,\vek
K,_{p_i}\,\vek x + \lambda(\vek p) \left[\vek x,_{p_i}^T\,\vek x + \vek x^T\,\vek
x,_{p_i}\right].
\end{align}
Weil die (normierte) L\"{a}nge des Eigenvektors invariant ist gegen\"{u}ber den Parametern $p_i$
\bfo
|\vek x| = \vek x^T\,\vek x = 1 \qquad \Rightarrow \qquad \vek x,_{p_i}^T\,\vek x + \vek
x^T\,\vek x,_{p_i} =0
\efo
folgt schlie{\ss}lich
\bfo
\frac{\partial \lambda}{\partial p_i} &=& \vek x^T\,\vek K,_{p_i}\,\vek x\,.
\efo
Der praktische Nutzen dieses Theorems besteht darin, dass uns versichert wird, dass die \"{A}nderungen in den Eigenwerten proportional zu den Ableitungen der Steifigkeitsmatrix nach den Parametern $p_i$ sind.

%%%%%%%%%%%%%%%%%%%%%%%%%%%%%%%%%%%%%%%%%%%%%%%%%%%%%%%%%%%%%%%%%%%%%%%%%%%%%%%%%%%%%%%%%%%%%%%%%%%
\textcolor{sectionTitleBlue}{\section{Die Vektoren $\vek f^+, \vek u^+, \vek g^+, \vek j^+$}}\label{Korrektur2}
Wir fassen zusammen: Die Vektoren $\vek u$ und $\vek g$ sind Wege, die Vektoren $\vek f$ und $\vek j$ sind \"{a}quivalente Knotenkr\"{a}fte.
\index{$\vek f^+$}\index{$\vek j^+$}\index{$\vek u^+$}\index{$\vek g^+$}
Das System $(\vek K + \vek \Delta \vek K)\,\vek u_c = \vek f$ ist identisch mit $\vek K\,\vek u_c = \vek f - \vek \Delta \vek K\,\vek u_c$ und hat die L\"{o}sung
\begin{align}
\vek u_c &= \vek u + \vek u^+ = \vek K^{-1}\,(\vek f + \vek f^+)
\end{align}
wobei
\begin{align}
\vek f^+ = - \vek \Delta\, \vek K\,\vek u_c \qquad \vek u^+ = \vek K^{-1}\,\vek f^+\,.
\end{align}
Das System $(\vek K + \vek \Delta \vek K)\,\vek g_c = \vek j_c$ zur Berechnung der Knotenwerte $g_i$ der Einflussfunktion f\"{u}r ein Funktional $J_c(\vek u_c) = \vek j_c^T\,\vek u_c = \vek g_c^T\,\vek f$ hat die L\"{o}sung
\begin{align}
\vek g_c &= \bar{\vek g} + \vek g^+ = \vek K^{-1}\,(\vek j_c + \vek j_c^+)\,,
\end{align}
wobei
\begin{align}
\bar{\vek g} = \vek K^{-1}\,\vek j_c \qquad \vek j_c^+ = - \Delta\, \vek K\,\vek g_c \qquad \vek g^+ = \vek K^{-1}\,\vek j_c^+\,.
\end{align}
Wenn sich die Steifigkeit des Elements, in dem $J_c(\vek u)$ ausgewertet wird, nicht \"{a}ndert, wenn also $J(\vek u) = J_c(\vek u)$ ist, dann ist $\vek j = \vek j_c$, und entsprechend nennen wir $\vek j_c^+ = \vek j^+$, und $\bar{\vek g} = \vek g$, so dass dann also mit (\ref{Eq104}) gilt
\begin{align}
J(\vek u_c) = (\vek g + \vek g^+)^T\,\vek f = (\vek j + \vek j^+)^T\,\vek u\,.
\end{align}
Vorsicht!  Es ist nicht $\vek j_c = \vek j + \vek j^+$. Der Vektor $\vek j^+$ ist nur eine Hilfsgr\"{o}{\ss}e, der die Formel
\begin{align}\label{Eq80}
J(\vek u_c) = (\vek j + \vek j^+)^T\,\vek u
\end{align}
m\"{o}glich macht, mit der man aus dem alten $\vek u$ die Werte des neuen $\vek u_c$ berechnen kann.
%-----------------------------------------------------------------
\begin{figure}[tbp]
\centering
\includegraphics[width=0.95\textwidth]{\Fpath/U473}
\caption{Das Federgesetz $u = 1/k \cdot f$ impliziert, dass Steifigkeits\"{a}nderungen, $\pm \Delta k$, zu unterschiedlich gro{\ss}en Korrekturen f\"{u}hren}
\label{U473}
\end{figure}%
%-----------------------------------------------------------------

Wenn $\vek j_c = \vek j + \vek j^+$ der richtige Vektor $\vek j_c$ w\"{a}re, dann m\"{u}sste man mit ihm $J(\vek u_c)$ aus $\vek u_c$ berechnen k\"{o}nnen,
\begin{align}
J(\vek u_c) = (\vek j + \vek j^+)^T\,\vek u_c \qquad \text{(?)}
\end{align}
was aber (\ref{Eq80}) widerspricht. Es kann nicht beides richtig sein.

Wie man an Abb. \ref{U394} sieht, sind die Kr\"{a}fte $j_i^+$ in den Ecken des ge\"{a}nderten Elements Zusatzkr\"{a}fte, um auf dem \glq alten\grq{} Netz (Matrix $\vek K$) die Wirkung des Dirac Deltas (Einzelkraft in der oberen rechten Ecke) darstellen zu k\"{o}nnen, also $\vek g_c$ zu generieren. Ohne die $j_i^+$ w\"{u}rde das Dirac Delta den Vektor $\vek g$ erzeugen, mit den $j_i^+$ nimmt das \glq alte\grq{} Netz die Form $\vek g_c$ an.

In Abb. \ref{U369} auf S. \pageref{U369} wurde diese Technik bei dem 1-D Problem eines Stabes benutzt, um mit Hilfe von Kr\"{a}ften $j^+$ die Einflussfunktion des {\em stepped bars\/} an einem Stab mit konstantem Querschnitt zu berechnen.


%%%%%%%%%%%%%%%%%%%%%%%%%%%%%%%%%%%%%%%%%%%%%%%%%%%%%%%%%%%%%%%%%%%%%%%%%%%%%%%%%%%%%%%%%%%%%%%%%%%
\textcolor{sectionTitleBlue}{\subsection{Unsymmetrie in den Ausgleichsbewegungen}}
Zum Schluss wollen wir noch anmerken, dass, wegen $u = 1/k \cdot f$, \"{A}nderungen $k \pm \Delta k$ unsymmetrisch ablaufen, s. Abb. \ref{U473}. Die Ausschl\"{a}ge $\Delta u$ bei einer Abnahme $-\Delta k$ sind gr\"{o}{\ss}er als die Verk\"{u}rzungen $\Delta u$ bei einer Zunahme $+\Delta k$. Anders gesagt: Wenn man die Steifigkeit $k$  einer Feder um $\Delta k$ verringert und dann wieder $\Delta k$ dazu addiert, ist man nicht da, wo man vor dem Man\"{o}ver war, sondern $f$ h\"{a}ngt tiefer.

Der Grund ist, dass man sich bei der Abnahme auf der Kurve $u = 1/k \cdot f$ befindet und beim R\"{u}ckweg auf der Kurve $u = 1/(k - \Delta k) \cdot f$.

Wenn man aus einem Reifen (5 l) ein Liter Luft (= 20 \%) herausl\"{a}sst und dann wieder hinzuf\"{u}gt, dann hat der Wagen seine alte H\"{o}he. Aber bezogen auf das reduzierte Volumen (4 l) ist 1 Liter Luft 25 \%. Rechnerisch h\"{a}tte man nur 0.8 l hinzuf\"{u}gen d\"{u}rfen, was nicht gereicht h\"{a}tte, um auf die alte H\"{o}he zu kommen. 