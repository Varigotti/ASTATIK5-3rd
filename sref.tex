\begin{thebibliography}{}
\bibitem{Altenbach} Altenbach H, Altenbach J, Kissing W (2004) Mechanics of Composite Structural Elements. Springer Verlag
\bibitem{Altenbach2} Altenbach H, Altenbach J, Naumenko K (2016) Ebene Fl\"{a}chentragwerke. Springer Verlag, 2. Auflage
\bibitem{Alastuey} Alastuey A, Clusel M, Magro M, Pujol P (2016) Physics and Mathematical Tools. World Scientific Publishing
\bibitem{Baar} Baar S (2015) Lohmeyer Baustatik 1, Lohmeyer Baustatik 2. Springer Verlag
\bibitem{Babuska5} Babu\v{s}ka I, Strouboulis T (2001) The Finite Element Method and its Reliability. Oxford University Press
\bibitem{Babuska6} Babu\v{s}ka I (1997) Der Doktorand war Roland Maucher, Ort: Cafeteria der Bauhaus-Universit\"{a}t in Weimar, IKM-Tagung, s. S. \pageref{Eq145}
\bibitem{Bangerth} Bangerth W, Rannacher R. (2003) Adaptive Finite Element Methods for Differential
Equations. Birkh\"{a}user Verlag Basel Boston Berlin
\bibitem{Barth} Barth C, Walter R (2013) Finite Elemente in der Baustatik-Praxis. Beuth
\bibitem{Basar} Ba\c{s}ar Y, Kr\"{a}tzig WB (1985) Mechanik der Fl\"{a}chentragwerke. Vieweg \& Sohn
\bibitem{Bebr} Bebr A (1971) \lqq Einflusslinien torsionssteifer Tr\"{a}gerroste\rqq, Bautechnik 48, Heft 7, 233-237
\bibitem{Beyer} Beyer K (1956) Die Statik im Stahlbetonbau. Reprint Springer Verlag 1987
\bibitem{Bletzinger} Bletzinger K U, Dieringer F (2014) Aufgabensammlung zur Baustatik: \"{U}bungsaufgaben zur Berechnung ebener Stabtragwerke. Hanser Verlag
\bibitem{Blaauwendraad} Blaauwendraad J (2010) Plates and FEM. Springer Verlag
\bibitem{Block} Block P, Gengnagel C (2015) Faustformel Tragwerksentwurf. DVA
\bibitem{Bochmann} Bochmann F, Kirsch W (2011) Statik im Bauwesen. Huss Medien
\bibitem{Borst} de Borst R, Crisfield MA, Remmers J, Verhoosel C (2014) Nichtlineare Finite-Elemente-Analyse von Festk\"{o}rpern und Strukturen. Wiley VCH
\bibitem{Bouma} Bouma A L (1993) Mechanik schlanker Tragwerke. Springer Verlag
\bibitem{Carl3} Carl O, Zhang C (2010) \lqq Static and dynamic analysis of cracked or weakened structures\rqq, Proc. Appl. Math. Mech. 10:145--146
\bibitem{Carl2} Carl O (2011) Statische und dynamische Sensitivit\"{a}tsanalysen von gesch\"{a}digten Tragwerken mit Greenschen Funktionen, Dissertation, Universit\"{a}t Siegen
\bibitem{Carl1} Carl O, Villamil P, Zhang C (2011) \lqq Stress and free vibration analysis of functionally graded beams using static Green's functions\rqq, Proc. Appl. Math. Mech. 11, 199--200
\bibitem{Carl4} Carl O, Hartman F, Zhang C (2017) \lqq Schnelle Berechnung von \"{A}nderungen und Varianten bei komplexen Tragsystemen (3D-Modellen) - Neue Ans\"{a}tze in der Baustatik unter Verwendung von Einflussfunktionen\rqq, Stahlbau (M\"{a}rz 2017) 217-224
\bibitem{Ca} https://de.wikipedia.org/wiki/Satz\_von\_Castigliano (7. 8. 2017)
\bibitem{Cirak2} \c{C}irak F, Ramm E (2000) \lqq A posteriori error estimation and adaptivity for elastoplasticity using the reciprocal theorem\rqq, Int. J. Num. Methods in Eng.  47:379--393
\bibitem{Dallmann} Dallmann R (2015) Baustatik 1, 2, 3. Hanser Verlag
\bibitem{Dhatt} Dhatt G, Touzot G, Lefran\c{c}ois (2012) Finite Element Method. ISTE, Wiley
\bibitem{Dinkler} Dinkler D (2014) Grundlagen der Baustatik. Springer Verlag
\bibitem{Eddy} Eddy W (2013) Baustatik -- einfach und anschaulich. Beuth
\bibitem{Ebel} Ebel H (1979) \lqq Zur Ber\"{u}cksichtigung von Verformungslastf\"{a}llen in den Reziprozit\"{a}tss\"{a}tzen von Betti, Maxwell und Krohn-Land\rqq, Stahlbau 49, Heft 5, 137-140
\bibitem{FK} Franke W, Kunow T (2007) Kleines Einmaleins der Baustatik. Kassel University Press
\bibitem{Gaul} Gaul L, Fiedler C (2013) Methode der Randelemente in Statik und Dynamik. Springer, 2. Auflage
\bibitem{Fuchs} Fuchs, M B (2016) Structures and Their Analysis. Springer Verlag
    \bibitem{Girkmann} Girkmann K (2013) Fl\"{a}chentragwerke. Springer Wien, Nachdruck der 6. Auflage von 1986
\bibitem{Golub} Golub GH, van Load CF (2013) Matrix Computations, 4th ed., The John Hopkins University Press Baltimore
\bibitem{Graf} Graf W, Vassilev T (2006) Einf\"{u}hrung in computerorientierte Methoden der Baustatik. Ernst \& Sohn
\bibitem{G1} Gr\"{a}tsch T, Hartmann F (2000) \lqq Zum Gleichgewicht bei finiten Elementen\rqq, Bautechnik 77, 30-36
\bibitem{G2} Gr\"{a}tsch T, Hartmann F, Katz, C (2004) \lqq Einflussfunktionen und finite Elemente\rqq, Bauingenieur 11, 489-497
\bibitem{G3} Gr\"{a}tsch T, Hartmann F (2006) \lqq Pointwise error estimation and adaptivity for the finite element method using fundamental solutions\rqq, Computational Mechanics, 37, 5, 394-407
\bibitem{G4} Gr\"{a}tsch T, Hartmann F (2001) \lqq \"{U}ber ein Fehlerbild bei der Schnittgr\"{o}{\ss}enermittlung mit finiten Elementen, Teil 1: Scheiben\rqq, Bautechnik 78, 327-332
\bibitem{G5} Gr\"{a}tsch T, Hartmann, F (2003) \lqq \"{U}ber ein Fehlerbild bei der Schnittgr\"{o}{\ss}enermittlung mit finiten Elementen, Teil 2: Platten\rqq, Bautechnik 80, 162-168
\bibitem{G8} Gr\"{a}tsch T (2002) $L_2$-Statik. Dissertation Universit\"{a}t Kassel
\bibitem{G6} Gr\"{a}tsch T, Hartmann F (2003) \lqq Finite element recovery techniques for local quantities of linear problems using fundamental solutions\rqq, Computational Mechanics, 33:15--21
\bibitem{G7} Gr\"{a}tsch T, Hartmann F (2004) \lqq Duality and Finite Elements\rqq, Finite Elements in Analysis and Design, 40, 1005--1020
\bibitem{Gr7} Gr\"{a}tsch T, Bathe KJ (2005) \lqq Influence functions and goal-oriented error estimation for finite element analysis of shell structures\rqq, International Journal for Numerical Methods in Engineering, 63(5), 631--788
\bibitem{Hake} Hake E, Meskouris K (2001) Statik der Fl\"{a}chentragwerke. Springer Verlag
\bibitem{Ha1} Hartmann F (1985) The Mathematical Foundation of Structural Mechanics. Springer Verlag
\bibitem{Ha2} Hartmann F, (1986) Methode der Randelemente. Springer Verlag
\bibitem{Ha3} Hartmann F, (1989) Introduction to Boundary Elements. Springer Verlag
\bibitem{HaM2} Hartmann F, Maucher R (1997) \lqq Zum Momentengleichgewicht bei Tragwerksberechnungen nach Theorie zweiter Ordnung\rqq,  Tagung IKM Weimar
\bibitem{HaJa2} Hartmann F, Jahn, P (1999) \lqq Integral Representations for the Deflection and the Slope of a Plate on an  Elastic Foundation\rqq,  Journal of Elasticity 56, 145-158
\bibitem{HaJa3} Hartmann F, Jahn P (2001) \lqq Boundary Element Analysis of Raft Foundations on Piles\rqq, Meccanica 36, 351-366
\bibitem{Ha4} Hartmann F, Katz C (2002) Statik mit finiten Elementen. Springer Verlag
\bibitem{HaK} Hartmann F, Kunow T (2005) \lqq The shift of Green's functions and the domain of influence\rqq, 2nd MIT Conference on Computational Fluids and Solid Mechanics
\bibitem{Ha5} Hartmann F, Katz C (2010) Structural Analysis with Finite Elements, 2nd ed. Springer Verlag
\bibitem{Ha6} Hartmann F (2013) Green's Functions and Finite Elements. Springer Verlag
\bibitem{HaJa} Hartmann F, Jahn P (2014) \lqq Steifigkeits\"{a}nderungen bei finiten Elementen\rqq, Bau\-ingenieur 89, 209-215
\bibitem{Ha7} Hartman F, Jahn P (2017) Statics and Influence Functions---From a Modern Perspective. Springer Verlag
\bibitem{HaJa2} Hartman F, Jahn P (2018) Statik und Einflussfunktionen---vom modernen Standpunkt, 2. Aufl. http://nbn-resolving.de/urn:nbn:de:hebis:34-2018030554714
\bibitem{Hake} Hake E, Meskouris K (2007) Statik der Fl\"{a}chentragwerke. Springer Verlag
\bibitem{Hartsuijker} Harsuijker C, Welleman J W (1999) Engineering Mechanics. Springer Verlag
\bibitem{Haug} Haug E J, Choi K K, Komkov V (1986) Design Sensitivity Analysis of Structural Systems. Academic Press
\bibitem{Hirschfeld} Hirschfeld K (2006) Baustatik - Theorie und Beispiele. Springer
\bibitem{Hoehland} Hoehland G (1957) St\"{u}tzmomenten-Einflussfelder durchlaufender Platten. Springer Verlag
\bibitem{Holzer} Holzer S (2015) Statische Beurteilung historischer Tragwerke, 1 und 2. Ernst \& Sohn
\bibitem{Hsiao} Hsiao GC, Wendland WL (2008) Boundary Integral Equations. Springer Verlag
\bibitem{Int1} http://phys.org/news/2015-10-multigrid-method-simulation.html
\bibitem{Irslinger} Irslinger J (2013) Mechanische Grundlagen und Numerik dreidimensionaler Schalenelemente. Diss. Uni Stuttgart, http://nbn-resolving.de/urn:nbn:de:bsz:93-opus-89596
\bibitem{Karnovsky} Karnovsky IA, Lebed O (2010) Advanced Methods of Structural Analysis. Springer Verlag
\bibitem{Katz1} Katz C, Stieda J (1993) \lqq Praktische FE-Berechnung mit Plattenbalken\rqq. Bauinformatik 1/92 30-34
\bibitem{Katz2} Katz C, Werner H (1982) \lqq Implementation of nonlinear boundary conditions in
finite element analysis\rqq. Computers \& Structures Vol. 15, No. 3, 299-304
\bibitem{Katz3} Katz C (1995) \lqq Kann die FE-Methode wirklich alles?\rqq
    FEM 95 - Finite Elemente in der Baupraxis, Ed. E. Ramm, E. Stein, W. Wunderlich
    Ernst \& Sohn, Berlin
\bibitem{Katz4} Katz C (1986) \lqq Berechnung von allgemeinen Pfahlwerken\rqq. Bauingenieur 61 Heft 12
\bibitem{Katz5} Katz C (1997) \lqq Flie{\ss}zonentheorie mit Interaktion aller Stabschnittgr\"{o}{\ss}en bei Stahltragwerken\rqq.
 Stahlbau 66, Heft 4, pp. 205-213, (1997)
\bibitem{Katz6} Katz C (1996) \lqq  Vertrauen ist gut, Kontrolle ist besser\rqq, in:
Software f\"{u}r Statik und Konstruktion, Eds. C. Katz, B. Protopsaltis, A.A. Balkema
Rotterdam, (1996)
\bibitem{Kemmler} Kemmler R, Ramm E (2001) \lqq Modellierung mit der Methode der Finiten Elemente\rqq, Beton-Kalender 2001. Ernst \& Sohn Berlin
\bibitem{Kersten} Kersten R (1962) Das Reduktionsverfahren der Baustatik. Springer Verlag, 2. Auflage
\bibitem{Kindmann} Kindmann R, Kraus M (2007) Finite-Elemente-Methoden im Stahlbau. Ernst \& Sohn
\bibitem{Koeppl} K\"{o}ppl C (2009) Einflusslinien und ihre Anwendung. Bachelor Projekt TU Graz
\bibitem{Kolar} Kol\'{a}\v{r} V (1970) \lqq The Influence Functions in the Finite Element Method\rqq, ZAMM 50 T 129--T 131\label{Korrektur13}
\bibitem{Kiener} Kiener G, Wunderlich W (2004) Statik der Stabtragwerke. Teubner Verlag
\bibitem{Kraetzig1} Kr\"{a}tzig W B,  Wittek U (1995) Tragwerke 1. Springer Verlag
\bibitem{Kraetzig2} Kr\"{a}tzig W B, Harte R (2016)  Tragwerke 2. Springer Verlag
\bibitem{Kraetzig3} Kr\"{a}tzig W B, Ba\c{s}ar Y (1997) Tragwerke 3. Springer Verlag
\bibitem{Krings} Krings W (2015) Kleine Baustatik. Springer Verlag
\bibitem{Ku} Kurrer K-E (2016) Geschichte der Baustatik. Ernst \& Sohn, 2. Auflage
\bibitem{Kurrer} Kurrer K-E (2018) The History of the Theory of Structures. Wiley Ernst \& Sohn
\bibitem{Lawo} Lawo M, Klingm\"{u}ller O, Thierauf G (1980) Stabtragwerke, Matrizenmethoden der Statik und Dynamik, 2 B\"{a}nde. Vieweg
\bibitem{Lehmann} Lehmann C, Maurer B (2006) Karl Culmann und die graphische Statik. Ernst \& Sohn
\bibitem{Link} Link M (2014) Finite Elemente in der Statik und Dynamik. Teubner Verlag, 4. Auflage
\bibitem{Lumpe} Lumpe G, Gensichen V (2014) Evaluierung der linearen und nichtlinearen Stabstatik in Theorie und Software. Ernst \& Sohn
\bibitem{Mann} Mann W (1997) Vorlesungen \"{u}ber Statik und Festigkeitslehre. Teubner-Verlag
\bibitem{Marti} Marti P (2014) Baustatik: Grundlagen--Stabtragwerke--Fl\"{a}chentragwerke. Ernst \& Sohn, 2. Auflage
\bibitem{Mehlhorn} Mehlhorn G, (1995) Der Ingenieurbau, 9 Bde, Baustatik/Baudynamik. Ernst \& Sohn
\bibitem{Merkel} Merkel M, \"{O}chsner A (2015) Eindimensionale Finite Elemente: Ein Einstieg in die Methode. Springer Verlag
\bibitem{Meskouris} Meskouris K, Butenweg C, Haker E, Holler S (2005) Baustatik in Beispielen. Springer Verlag
\bibitem{Meskouris2} Meskouris K (2009) Statik der Stabtragwerke: Einf\"{u}hrung in die Tragwerkslehre. Springer Verlag
\bibitem{Moersch} M\"{o}rsch E (1947) Statik der Gew\"{o}lbe und Rahmen, Teil A und Teil B, Wittwer Stuttgart
\bibitem{Nasdala} Nasdala L (2015) FEM-Formelsammlung Statik und Dynamik. Springer Verlag
\bibitem{Palkowski} Palkowski S (1989) Statik der Seilkonstruktionen. Springer Verlag
\bibitem{Peters} Peters K-H (2004) Der Zusammenhang von Mathematik und Physik am Beispiel der Geschichte der Distributionen. Dissertation. Universit\"{a}t Hamburg
\bibitem{Petersen0} Petersen C (1966) Beitrag zur praktischen Berechnung zylindrischer Tonnenschalen mit ver\"{a}nderlichem Kr\"{u}mmungshalbmesser. Dissertation TH M\"{u}nchen
\bibitem{Petersen1} Petersen C (2011) Statik und Stabilit\"{a}t der Baukonstruktionen. Vieweg+Teubner Verlag
\bibitem{Petersen3} Petersen C (1990) Stahlbau. Vieweg
\bibitem{Petersen2} Petersen C, Werkle H (2018) Dynamik der Baukonstruktionen. 2. Aufl. Springer Vieweg
\bibitem{Ramm} Ramm E, Hofmann TJ (1996) Stabtragwerke, in Der Ingenieurbau, Bd. 5, 1-350, Ed. G. Mehlhorn. Ernst \& Sohn
\bibitem{Pfl\"{u}ger} Pfl\"{u}ger A (1978) Statik der Stabtragwerke. Springer Verlag
\bibitem{Pucher} Pucher A (1977) Einflussfelder elastischer Platten. Springer Wien, 7. Auflage
\bibitem{Rene} Ren\'{e} H (2012) Statik im Bauwesen. Beuth
\bibitem{Rothert} Rothert H, Gensichen V (1987) Nichtlineare Stabstatik. Springer Verlag
\bibitem{Rubin} Rubin H (1993) Baustatik ebener Stabwerke, in Stahlbau-Handbuch Teil A. Stahlbau-Verlagsgesellschaft
\bibitem{Rust} Rust W (2011) Nichtlineare Finite-Elemente-Berechnungen. Vieweg 2. Auflage
\bibitem{Schade} Schade D (2003) \lqq Einflusslinien f\"{u}r Ausnutzungsgrade in Stabwerken\rqq. Stahlbau 72, Heft 2, 79-82
\bibitem{Schiefer} Schiefer S, Fuchs M, Brandt B, Maggauer G, Egerer A (2006) \lqq Besonderheiten beim Entwurf semi-integraler Spannbetonbr\"{u}cken\rqq, Beton und Stahlbetonbau, 790-802
\bibitem{Schwartpaul} Schwartpaul K, Zhang C, Carl O (2012) \lqq Sensitivity Analysis of Weakened Beams on Elastic Foundation with Green's Functions\rqq. Proc. Appl. Math. Mech. Vol. 12, 233-234
\bibitem{Silber} Silber G, Steinwender F (2005) Bauteilberechnung und Optimierung mit der FEM. Teubner
\bibitem{Sof}  SOFiSTiK AG, www.sofistik.de
\bibitem{Sopoth} Sopoth M, Sopoth G (2008) Sensitivit\"{a}tsanalyse an einem Br\"{u}ckenbauwerk in semi-integraler Bauweise. Diplomarbeit Universit\"{a}t Kassel
\bibitem{Spitzer} Spitzer P, Horschig R (2012) Statik im Bauwesen, 3 B\"{a}nde + Aufgabensammlung. Beuth
\bibitem{Strang0} Strang G, Fix GJ (2008) An Analysis of the Finite Element Method.
Wellesley-Cambridge Press, 2nd ed.
\bibitem{Strang4} Strang G (2010)  Computational Science and Engineering. Wellesley-Cambridge Press. Dt. Ausgabe: Wissenschaftliches Rechnen. Springer Verlag
\bibitem{Szabo2} Szabo I (1997) Geschichte der mechanischen Prinzipien. Birkh\"{a}user Verlag
\bibitem{Szabo3} Szabo G, Babuska I (1991) Finite Element Analysis. John Wiley \& Sons, Inc.
\bibitem{Taylor} Robert Taylor machte diese Bemerkung in der Vollversammlung des
 Ninth US Congress in Computational Mechanics in San Francisco in 2007
\bibitem{Tottenham} Tottenham H (1970) \lqq Basic Principles\rqq,
in: Finite Element Techniques in Structural Mechanics. (Eds. Tottenham H, Brebbia C), Southampton University Press
\bibitem{Turner} Turner MJ, Clough RW, Martin HC, Topp LJ (1956) \lqq Stiffness and deflection analysis of complex structures\rqq. Journal of the Aeronautical Sciences, Vol. 23, No. 9, 805-823
\bibitem{VA} https://de.wikipedia.org/wiki/Virtuelle\_Arbeit
\bibitem{Wagner} Wagner R (2016) Bauen mit Seilen und Membranen. Beuth
\bibitem{Werner} Werner K (2011) Statik im Bauwesen. Beuth
\bibitem{Werkle1} Werkle H (2000) \lqq Konsistente Modellierung von St\"{u}tzen bei der Finite-Element-Berechnung von Flachdecken\rqq, Bautechnik 77, 416--425
\bibitem{Werkle2} Werkle H (2008) Finite Elemente in der Baustatik. Springer Vieweg. 3. Auflage
\bibitem{Werkle3} Werkle H (2006) Vorlesung Baustatik III, Skriptum

\bibitem{Wigner} Wigner E P (1960) 'The unreasonable effectiveness of mathematics in the natural sciences. Richard Courant lecture in mathematical sciences delivered at New York University, May 11, 1959'. Communications on Pure and Applied Mathematics. 13: 1--14.
\bibitem{Wiki1} https://de.wikipedia.org/wiki/Prinzip\_von\_St.\_Venant
\bibitem{Wriggers} Wriggers P (2001) Nichtlineare Finite-Element-Methoden. Springer
\bibitem{Z1} Zienkiewicz OC, Taylor RL, Zhu JZ (2006) Finite Element Method: Volume 1 -- Its Basis \& Fundamentals. Butterworth Heinemann
\end{thebibliography}

{\em The History of the Theory of Structures\/} von Kurrer \cite{Kurrer} enth\"{a}lt eine ausf\"{u}hrliche Bibliographie zu dem Thema Statik und Einflussfunktionen in seiner geschichtlichen Entwicklung.

Die Diplomarbeiten aus Kassel sind \"{u}ber www.winfem.de/products.htm erreichbar. 