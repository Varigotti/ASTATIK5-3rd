%\documentclass[graybox,envcountchap,sectrefs]{svmonoFH}
\documentclass[envcountchap,monohd,sectrefs]{svmonoFH}

\newcommand{\Fpath}{d:/astatikbilder/pdf}
\newcommand{\Springerpath}{"d:/astatik5 Modern 3rdEd/springer"}

\usepackage{\Springerpath/cropmark}
\usepackage{helvet}
\usepackage{courier}
\usepackage{type1cm}
\usepackage[makeindex]{imakeidx}         % allows index generation
\usepackage{multicol}        % used for the two-column index
\usepackage[bottom]{footmisc}% places footnotes at page bottom
%\usepackage[draft]{graphicx}        % standard LaTeX graphics tool
\usepackage[]{graphicx}        % standard LaTeX graphics tool
\usepackage{booktabs}
\usepackage{amssymb}
\usepackage{amsmath}
\usepackage{dsfont}
\usepackage{wasysym}
\usepackage{rotate}  % FH
\usepackage[T1]{fontenc}
\usepackage{calligra}
\usepackage{color}
\usepackage{blindtext}
\usepackage{german}
\usepackage{deutengi}
\usepackage{float}
\usepackage{textcomp}
\usepackage[bookmarksnumbered, hyperindex]{hyperref}

\hypersetup{pdftitle={Statik und Einflussfunktionen - vom modernen Standpunkt aus},
pdfsubject = {Statik, Einflussfunktionen, Stabtragwerke, Fl\"{a}chentragwerke},
pdfauthor = {Friedel Hartmann, Peter Jahn},
pdfkeywords = {Statik, Einflussfunktionen, Satz von Betti, Maxwell, finite Elemente, Randelemente, Energieprinzipe, Variationsprinzipe, Prinzip der virtuellen Verr\"{u}ckung, Prinzip der virtuellen Kr\"{a}fte, potentielle Energie, Gleichgewicht, Greensche Identit\"{a}ten, Dualit\"{a}t, Stabtragwerke, Fl\"{a}chentragwerke}}


%\hypersetup{pdftoolbar=true}
\definecolor{red}{rgb}{1,0,0}
\definecolor{blau}{rgb}{.1,.1,.1}
\definecolor{green2}{rgb}{0,0.39,0.0}

\definecolor{chapterTitleBlue}{rgb}{.0,.2,.8}
%\definecolor{sectionTitleBlue}{rgb}{.25,.5,1.0}
\definecolor{sectionTitleBlue}{rgb}{.25,.5,1.0}  % helleres blau
\definecolor{highlightBlue}{rgb}{.85,.964,1.0}

% Make @ behave as per catcode 13 (active).  TeXbook p. 155.
\mathcode`@="8000
{\catcode`\@=\active\gdef@{\mkern1mu}}

\definecolor{weiss}{gray}{1.0}
\newcommand{\bild}{2}
\newcommand{\vek}[1]{\mbox{\boldmath $#1$}}
\newenvironment{Eqnarray} {\arraycolsep 0.14em\begin{eqnarray}}{\end{eqnarray}}
\newcommand {\bfo}    {\begin {Eqnarray}}
\newcommand {\efo}    {\end   {Eqnarray}}
\newcommand {\nn} {\nonumber}
\newcommand {\dotprod}{{\,\scriptscriptstyle \stackrel{\bullet}{{}}}\,}
\newcommand {\barr}   {\begin {array}}
\newcommand {\earr}   {\end {array}}
\newcommand{\Np}{\varphi}
\def\strut{\rule{0in}{.50in}}
\newcommand{\lqq}{\lq\lq}
\newcommand{\rqq}{\rq\rq \,}
\newcommand{\beq}{\begin{equation}}
\newcommand{\eeq}{\end{equation}}

\newenvironment{EqnarrayNN} {\arraycolsep 0.14em\begin{eqnarray*}}{\end{eqnarray*}}
\newcommand {\bfoo}    {\begin {EqnarrayNN}}       % bfo
\newcommand {\efoo}    {\end {EqnarrayNN}}         % efo
\newcommand{\hlq}{\glq\kern.07em\allowhyphens}   % Frank Holzwarth  Januar 2001
%  Bildobjekte

\newsavebox{\quadratbox}
\newsavebox{\rechteckbox}
\newsavebox{\doppeldreieckoben}
\newsavebox{\doppeldreieckunten}
\newsavebox{\dreiecklinksoben}
\newsavebox{\dreiecklinksobenKurz}
\newsavebox{\dreiecklinksunten}
\newsavebox{\dreieckrechtsoben}
\newsavebox{\rampelinks}
\newsavebox{\ramperechts}
\newsavebox{\dreieckrechtsunten}
\newsavebox{\knickk}
\newsavebox{\knickksmall}
\newsavebox{\versetzungg}
\newsavebox{\seilecka}
\newsavebox{\rechtecklanga}

\sbox{\quadratbox}{%
\begin{picture}(12,10)%
  \put(0,0){\framebox(10,10){}}%
\end{picture}}

\sbox{\rechteckbox}{%
\begin{picture}(22,10)%
  \put(0,0){\framebox(20,10){}}%
\end{picture}}

\sbox{\doppeldreieckoben}{%
\begin{picture}(32,10)%
  \put(0,0){\line(1,0){30}}
  \put(0,0){\line(0,1){10}}
  \put(30,0){\line(0,-1){10}}
  \put(0,10){\line(3,-2){30}}
\end{picture}}

\sbox{\doppeldreieckunten}{%
\begin{picture}(32,10)%
  \put(0,0){\line(1,0){30}}
  \put(0,0){\line(0,-1){10}}
  \put(30,0){\line(0,1){10}}
  \put(0,-10){\line(3,2){30}}
\end{picture}}

\sbox{\dreiecklinksoben}{%
\begin{picture}(32,10)%
  \put(0,0){\line(1,0){30}}
  \put(0,0){\line(0,1){10}}
  \put(0,10){\line(3,-1){30}}
\end{picture}}

\sbox{\dreiecklinksunten}{%
\begin{picture}(32,10)%
  \put(0,0){\line(1,0){30}}
  \put(0,0){\line(0,-1){10}}
  \put(0,-10){\line(3,1){30}}
\end{picture}}

\sbox{\dreieckrechtsoben}{%
\begin{picture}(32,10)%
  \put(0,0){\line(1,0){30}}
  \put(30,0){\line(0,1){10}}
  \put(0,0){\line(3,1){30}}
\end{picture}}

\sbox{\rampelinks}{%
\begin{picture}(12,10)%
  \put(0,0){\line(1,0){10}}
  \put(10,0){\line(0,1){10}}
  \put(0,0){\line(1,1){10}}
\end{picture}}

\sbox{\ramperechts}{%
\begin{picture}(12,10)%
  \put(0,0){\line(1,0){10}}
  \put(10,0){\line(-1,1){10}}
  \put(0,0){\line(0,1){10}}
\end{picture}}

\sbox{\dreieckrechtsunten}{%
\begin{picture}(32,10)%
  \put(0,0){\line(1,0){30}}
  \put(30,0){\line(0,-1){10}}
  \put(0,0){\line(3,-1){30}}
\end{picture}}

\sbox{\knickk}{%
\begin{picture}(20,10)%
\thicklines  \put(0,10){\line(1,-1){10}}
  \put(10,0){\line(1,1){10}}
\end{picture}}

\sbox{\knickksmall}{%
\begin{picture}(12,12)%
 \put(0,6){\line(1,-1){6}}
  \put(6,6){\line(-1,-1){6}}
%\thicklines  \put(0,8){\line(1,-1){8}}
%  \put(8,0){\line(1,1){8}}
\end{picture}}

\sbox{\versetzungg}{%
\begin{picture}(20,10)%
\thicklines  \put(0,5){\line(1,1){10}}
  \put(10,15){\line(0,-1){20}}
  \put(10,-5){\line(1,1){10}}
\end{picture}}

\sbox{\rechtecklanga}{%
\begin{picture}(52,10)%
  \put(0,0){\framebox(50,12.5){}}%
\end{picture}}

\sbox{\seilecka}{%
\begin{picture}(52,10)%
  \put(0,0){\line(1,0){50}}
  \put(50,0){\line(-2,-1){25}}
  \put(25,-12.5){\line(-2,1){25}}
\end{picture}}

\newcommand{\quadrat}{\usebox{\quadratbox}}
\newcommand{\cross}{\usebox{\rechteckbox}}
\newcommand{\doppeldo}{\usebox{\doppeldreieckoben}}
\newcommand{\doppeldu}{\usebox{\doppeldreieckunten}}
\newcommand{\dreiecklo}{\usebox{\dreiecklinksoben}}
\newcommand{\dreiecklu}{\usebox{\dreiecklinksunten}}
\newcommand{\dreieckro}{\usebox{\dreieckrechtsoben}}
\newcommand{\rampeL}{\usebox{\rampelinks}}
\newcommand{\rampeR}{\usebox{\ramperechts}}
\newcommand{\dreieckru}{\usebox{\dreieckrechtsunten}}
\newcommand{\knick}{\usebox{\knickk}}
\newcommand{\knicks}{\usebox{\knickksmall}}
\newcommand{\versetzung}{\usebox{\versetzungg}}
\newcommand{\seileck}{\usebox{\seilecka}}
\newcommand{\rechtecklang}{\usebox{\rechtecklanga}}


%\newtheoremstyle{dotlessP}{}{}{\color{db}}{}{\color{db}\bfseries}{}{ }{}
%\theoremstyle{dotlessP}

%\newtheorem{theorem}{{\textcolor{chapterTitleBlue}{Theorem}}
%\newtheorem{theorem}{Theorem}[section]

\floatstyle{boxed}
\newfloat{Box}{h}{lop}[chapter]
\makeindex

\begin{document}

\author{F. Hartmann, P. Jahn}
\title{Statik und Einflussfunktionen --\\
vom modernen Standpunkt aus\\
\begin{large}{3. Auflage (3.73), Last edit: \today}\end{large}}
%\begin{large}{3. Auflage}\end{large}}
\maketitle

%\frontmatter

%
%%%%%%%%%%%%%%%%%%%%%%% dedic.tex %%%%%%%%%%%%%%%%%%%%%%%%%%%%%%%%%
%
% sample dedication
%
% Use this file as a template for your own input.
%
%%%%%%%%%%%%%%%%%%%%%%%% Springer-Verlag %%%%%%%%%%%%%%%%%%%%%%%%%%

\thispagestyle{empty}
\vspace*{3.5cm}
\begin{center}

% write your text here
{\large
The principle of virtual displacements is nothing else than\\ integration by parts.\\
\vspace*{0.5cm}
Robert Taylor
}

\end{center}




%\include{foreword}aaa
\preface
{\em Die neue Statik ist die alte Statik\\
und im Grunde ist sie heute m\"{a}chtiger\\
als je zuvor.\/}\\

Einflussfunktionen sind ein klassisches Werkzeug der Statik. Sie verkn\"{u}p\-fen Statik mit Anschauung, denn mit ein paar geschickten Skizzen -- wenn es sein muss auf einem Bierdeckel -- kann man leicht dem Tragverhalten einer Struktur nachsp\"{u}ren und so Klarheit \"{u}ber kritische Punkte finden.

Leider ist aber die Anwendung der Einflussfunktionen etwas in den Hinterland getreten, denn in Zweifelsf\"{a}llen spielt man dann doch lieber Varianten mit dem Computer durch und umgeht so die M\"{u}he, nach dem warum und wieso zu fragen und tiefer in das Verst\"{a}ndnis des Tragverhaltens einzudringen.

Neue Ergebnisse haben jedoch das Interesse an den Einflussfunktionen wieder belebt, denn es ist nun klar, dass finite Elemente mit Einflussfunktionen rechnen. Das gleicht einer Rolle r\"{u}ckw\"{a}rts. Man dachte, man sei der klassischen Rechenverfahren ledig, und pl\"{o}tzlich sieht man, dass sie in den finiten Elementen wieder auferstanden sind.

In der klassischen Statik beschr\"{a}nkt sich das Thema Einflussfunktionen auf den {\em Satz von Land\/} und seine Modifikationen und man verliert bald das Interesse, weil sich die Einflussfunktionen so schwer berechnen lassen.

Heute benutzen wir finite Elemente und bei finiten Elementen ist der Begriff viel weiter gefasst. Das Stichwort hei{\ss}t {\em Funktionale\/}. Die Durchbiegung in der Feldmitte, das Moment \"{u}ber der St\"{u}tze, die Kraft im Lager, all dies sind Funktionale. Alles, was man berechnen kann, ist f\"{u}r die finiten Elemente ein Funktional. Und zu jedem linearen Funktional geh\"{o}rt eine Greensche Funktion, eine Einflussfunktion.

Nun sind Einflussfunktionen aber Biegelinien, also Verformungen und das wird mit finiten Elementen zum Problem, denn FE-Netze besitzen nur eine eingeschr\"{a}nkte Kinematik. Es gibt ja nur einen beschr\"{a}nkten Vorrat an Ansatzfunktionen ({\em shape functions\/}), um Verformungen darzustellen. Und das ist der Grund, warum FE-Ergebnisse in der Regel nur N\"{a}herungen sind, denn das FE-Programm kann mit der eingeschr\"{a}nkten Kinematik eines Netzes die exakten Einflussfunktionen nicht generieren, es \"{u}berlagert daher gezwungenerma{\ss}en gen\"{a}herte Einflussfunktionen mit der Belastung und so sind die Ergebnisse auch nur N\"{a}herungen.

Die Einflussfunktionen sind demnach die eigentlichen, die wahren {\em shape functions\/}, die {\em physikalischen shape functions\/}. Diese muss das FE-Programm  m\"{o}glichst gut ann\"{a}hern. Wenn das gelingt, dann sind auch die FE-Ergebnisse gut.

In der Computerstatik geht das Thema Einflussfunktionen also weit \"{u}ber den {\em Satz von Land\/} hinaus und um diesem Umfang einigerma{\ss}en gerecht zu werden, haben wir dieses Buch geschrieben.

Es ist kein Buch f\"{u}r Erstsemester, der Leser sollte mit dem Thema Einflussfunktionen schon etwas vertraut sein, dem Thema in den Statik- oder Mechanikvorlesungen schon begegnet sein.

Wir behandeln das Thema auch scheinbar mit einem sehr spitzen Bleistift. Das ist aber im Grunde Notwehr, weil sich in der Statik doch viele Dinge  im Laufe der Zeit eingeschliffen haben und der mathematische Hintergrund der Formeln nicht immer offenkundig und evident ist.

\begin{flushright}\noindent
Kassel  {\hfill {\it Friedel Hartmann, Peter Jahn}}\\\vspace{0.1cm}
Fr\"{u}hjahr 2016   {\hfill {hartmann@be-statik.de, PJahn@uni-kassel.de}}\\
\end{flushright}


\vspace{1.7cm}
PS. Urspr\"{u}nglich sollte der Titel nur {\em Einflussfunktionen --- vom modernen Standpunkt aus\/} hei{\ss}en. Im Zeitalter der Suchmaschinen schien es uns jedoch sinnvoll, das Wort {\em Statik\/} mit in den Titel aufzunehmen. Es war nicht unsere Absicht einen (nicht existierenden) Gegensatz zwischen alter und neuer Statik zu konstruieren. Nur der Blickwinkel auf die Einflussfunktionen hat sich mit dem Computer ge\"{a}ndert.\\



\begin{flushleft}\large{\bf{Zweite Auflage}} \end{flushleft}

Das elektronische Format macht es m\"{o}glich, \"{A}nderungen und Erg\"{a}nzungen, gleich einem nie endenden {\em work in progress\/}, kontinuierlich in den Text einflie{\ss}en zu lassen. Nach zwei Jahren und +150 Seiten haben wir uns jedoch dazu entschlossen, die zweite Auflage \glq einzufrieren\grq{} und auf den Server der Universit\"{a}tsbibliothek Kassel (KOBRA) zu legen. Sie kann von dort \"{u}ber den (permanent g\"{u}ltigen, \glq zitierf\"{a}higen\grq{}) link\\

\href{http://nbn-resolving.de/urn:nbn:de:hebis:34-2018030554714}{http://nbn-resolving.de/urn:nbn:de:hebis:34-2018030554714}
%http://nbn-resolving.de/urn:nbn:de:hebis:34-2018030554714\\

\begin{flushleft} heruntergeladen werden. \end{flushleft}
Die bei Kassel-University-Press erh\"{a}ltliche gedruckte Version des Buchs ist auf dem Stand der 1. Auflage vom Fr\"{u}hjahr 2016, ebenso der pdf-file auf der Seite der Kassel-University-Press.


\begin{flushleft}\large{\bf{Dritte Auflage}} \end{flushleft}

Im Fr\"{u}hjahr 2018 haben wir mit der dritten Auflage begonnen. Die jeweils tages-aktuelle pdf-Version der dritten Auflage liegt auf den Seiten\\

 \href{http://simplel.ink/go/pdf}{http://simplel.ink/go/pdf} und \href{http://simplel.ink/go/winfemBook}{http://simplel.ink/go/winfemBook}.

\begin{flushright}\noindent
Kassel  {\hfill {\it Friedel Hartmann, Peter Jahn}}\\\vspace{0.1cm}
April 2018   {\hfill {hartmann@be-statik.de, PJahn@uni-kassel.de}}\\
\end{flushright}

\vspace{1 cm}
\begin{acknowledgement}
Her Kollege Werkle, Hochschule Konstanz, hat uns bei der korrekten Formulierung der {\em \"{A}quivalenten Spannungs Tranformation\/}, Abschnitt 3.11, tatkr\"{a}ftig unterst\"{u}tzt. Daf\"{u}r sei ihm an dieser Stelle gedankt.\\
\end{acknowledgement}


%%%%%%%%%%%%%%%%%%%%%%%acknow.tex%%%%%%%%%%%%%%%%%%%%%%%%%%%%%%%%%%%%%%%%%
% sample acknowledgement chapter
%
% Use this file as a template for your own input.
%
%%%%%%%%%%%%%%%%%%%%%%%% Springer %%%%%%%%%%%%%%%%%%%%%%%%%%

\extrachap{Edition August 2017}

In dieser Edition wurden kleinere Korrekturen und \"{A}nderungen vorgenommen und das Kapitel 5, 
{\em Steifigkeits\"{a}nderungen und Reanalysis\/}, um Verfahren zur Berechnung des Vektors $\vek u_c$ erweitert.



\tableofcontents
%\include{acronym}

\hyphenation{Gleich-ge-wicht Aus-gangs-punkt w\"{a}hr-end L\"{a}ngs-richtung ge-trost Gleich-ge-wichts-be-ding-ungen auf-ad-diert
selbst-ver-st\"{a}ndlich Wechsel-wir-kungs-ener-gie Steif-ig-keiten Dif-feren-tial-gleich-ung Dif-fer-ential-gleich-ungen
Ein-fluss-funk-ti-onen Ein-fluss-funk-ti-on}

%\mainmatter
% Der Mathematiker sagt, wenn eine Gleichung null ist, $g = 0$, dann kann ich sie doch mit einer beliebigen Zahl $\delta u$ multiplizieren, und das Ergebnis \"{a}ndert sich nicht, $\delta u \cdot g = 0$,

 Mit dem Arbeitsbegriff kommt die \"{u}berragende Rolle des Skalarproduktes in die Statik hinein. Wir kennen das Skalarprodukt, als das Skalarprodukt zweier Vektoren

 Wir denken uns also den Stab wie ein Laib Brot in lauter kleine Scheibenelemente der Dicke $ dx $ zerlegt. Vorder- und R\"{u}ckseite eines Scheibeelementes wird von einer Normalkraft $N(x) + dN$ bzw. $ N (x)$ gedr\"{u}ckt oder gezogen (der Einfachheit halber nehmen wir an, dass der Zuwachs $ dN $ Null ist) und sie bewegt sich um die Strecke $d\,\delta\,u(x)$, so dass die Arbeit der beiden Normalkr\"{a}fte gleich
\begin{align}
- N(x)\,\delta u(x) + N(x) \,(\delta u(x) + d\, \delta u(x))  = N(x)\,d\,\delta u(x)
\end{align}

\section{??}
Wir beginnen mit dem Gleichgewicht. Die L\"{a}ngsverschiebung eines beidseitig eingespannten Stabes, der eine konstante Streckenlast $p$ [kN/m] tr\"{a}gt, ist die L\"{o}sung des Randwertproblems
\begin{align}
- EA\,u''(x) = p(x) \qquad u(0) = u(l) = 0\,.
\end{align}
Nun ist es nicht schwer, die L\"{o}sung dieser Differentialgleichung zu finden
\begin{align}
u(x) =
\end{align}
 und damit auch den Verlauf der Normalkraft
 \begin{align}
 N(x) = p\,\frac{l}{2} -
 \end{align}
Die L\"{o}sung wird normalerweise so verifiziert, dass man die Funktion $u$ in die Differentialgleichung einsetzt. Was ist aber mit dem Gleichgewicht? Wer kontrolliert denn eigentlich die Normalkr\"{a}fte an den beiden Stabenden? Bei der L\"{o}sung der Aufgabe hat man doch an keiner Stelle einen Gedanken an die Gleichgewichtsbedingung
\begin{align}
N(l) - N(0) + \int_0^{\,l} p\,dx = 0
\end{align}
verschwendet. Nun, die Gleichgewichtsbedingung ist garantiert, von der ersten Greenschen Identit\"{a}t, denn setzen wir f\"{u}r die Funktion $\hat{u}$ die konstante Funktion $\hat{u} = 1$, dann folgt
\begin{align}
G(u,1) = \int_0^{\,l} p\,dx + N(l) - N(0) = 0\,.
\end{align}
Weil jedes zul\"{a}ssige Paar $\{u, \hat{u}\}$ eine Nullstelle der ersten Greenschen Identit\"{a}t ist, ist es auch das Paar $\{u, 1\}$, d.h. die Gleichgewichtsbedingungen sind {\em automatisch\/} erf\"{u}llt.\\

Wenn man genauer hinschaut, wir setzen einfachheitshalber $EA = 1$, dann dr\"{u}ckt die Gleichgewichtsbedingung nur die Tatsache aus, dass die Normalkraft die Stammfunktion der zweiten Ableitung ist
\begin{align}
\int_0^{\,l} - u''(x)\,dx = - u'(l) + u(0)
\end{align}
Anders gesagt, weil Gau{\ss} die Regeln der partiellen Integration entdeckt hat, ist garantiert, dass die Normalkr\"{a}fte an den Stabenden mit der Streckenlast im Gleichgewicht sind. \\

Das kann man mit jeder Funktion $u$, wie $u(x) = e^x,$ ausprobieren
\begin{align}
\int_0^{\,l} e^x\,dx = e^l - e^0 = e^l  - 1\,.
\end{align}
\subsection{Das Prinzip der virtuellen Verr\"{u}ckungen}
Das {\em Prinzip der virtuellen Verr\"{u}ckungen\/} besagt: Wenn ein Stab im Gleichgewicht ist, dann ist bei jeder virtuellen Verr\"{u}ckung $ \delta u $ die virtuelle \"{a}u{\ss}ere Arbeit $ \delta W_a $ gleicht der virtuellen inneren Energie $\delta W_i$ in dem Stab.
\begin{align}
\delta A_a = \delta A_i
\end{align}
mit
\begin{align}
\delta A_a = \int_0^{\,l} p\,\delta u\,dx = N(l)\, \delta u(l) - N(0)\,\delta u(0)
\end{align}
und
\begin{align}
\delta A_i = \int_0^{\,l} \frac{N\,\delta N}{EA}\,dx
\end{align}
Nun diese Gleichheit der virtuellen inneren und \"{a}u{\ss}eren Arbeiten ist nichts anderes als die Aussage, dass die beiden Funktionen $ u $ und $\hat{u}$ eine Nullstelle der ersten Greenschen Identit\"{a}t sind, denn $G(u,\delta u) = 0$ garantiert eben
\begin{align} \label{Eq1}
G(u,\delta u) = \int_0^{\,l} p\,\delta u\,dx + N(l)\,\delta u(l) - N(0)\,\delta u(0)  - \int_0^{\,l} \frac{N\,\delta N}{EA}= 0\,.
\end{align}
was das Prinzip der virtuellen Verr\"{u}ckung ist.\\

Man beachte, dass wir hier keine Einschr\"{a}nkungen an die virtuelle Verr\"{u}ckung $ \delta u $ machen. In der Literatur wird das Prinzip der virtuellen Verr\"{u}ckung oft auf sogenannte {\em zul\"{a}ssige virtuelle Verr\"{u}ckungen\/} eingeschr\"{a}nkt, das sind Verr\"{u}ckungen, die mit den Lagerbedingungen des Stabes vertr\"{a}glich sind also hier mit den Festhaltungen links und rechts an den Stabenden, was bedeutet, dass die virtuellen Verr\"{u}ckungen an den Stabenden Null sind. \\

F\"{u}r solche virtuellen Verr\"{u}ckungen $ \delta u $ reduziert sich Glg. (\ref{Eq1}) auf
\begin{align}
G(u,\delta u) = \int_0^{\,l} p\,\delta u\,dx  - \int_0^{\,l} \frac{N\,\delta N}{EA}= 0\,.
\end{align}
Das ist aber, wie gesagt, eine Einschr\"{a}nkung, die die Mechanik macht und nicht die Mathematik. Die Mathematik l\"{a}sst auch virtuelle Verr\"{u}ckungen $\delta u \in C^1$ zu, die die Lager verschieben. Auch f\"{u}r diese gilt noch $\delta A_a = \delta A_i$. \\

\subsection{Das Prinzip der virtuellen Kr\"{a}fte}
Dieses Prinzip besagt, dass die \"{a}u{\ss}ere Arbeit $\delta A_a^*$ von virtuellen Kr\"{a}ften auf den Wegen $ u $ des Stabes gleich der virtuellen inneren Energie $\delta A_i^*$ ist.\\

Der Hintergrund dabei ist der folgende: man gibt sich ein System von virtuellen Kr\"{a}ften vor, also eine Streckenlast $ p^* $ und die zugeh\"{o}rigen Normalkr\"{a}fte $ N(l)^*, N(0)^*$ an den Stabenden
und l\"{a}sst nun diese auf den Wegen $ u $ des Stabes \"{a}u{\ss}ere und innere Arbeiten verrichten. Dann ist die Bilanz
\begin{align}
\delta A_a^+ = \delta A_i^*\,.
\end{align}

 Unsere Vorgehensweise ist hier etwas atypisch, oder besser gesagt typisch Universit\"{a}t, weil den Statiker normalerweise nicht die L\"{a}ngsverschiebung interessiert sondern eigentlich nur der Verlauf der Normalkraft
 Den Statiker interessiert normalerweise nicht die L\"{a}ngsverschiebung, sondern eigentlich nur der Verlauf der Normalkraft. Der Statiker betrachtet also gar nicht die Differentialgleichung,

Uns interessiert eigentlich nur der Fall, wenn alle die Koeffizienten $c_i$ und $d_i$ eins sind
\begin{align}
a(u_1 + u_2,\hat{u}_1 + \hat{u}_2) = a(u_1,\hat{u}_1) + a(u_2,\hat{u}_1) + a(u_2,\hat{u}_1) + a(u_2,\hat{u}_2)
\end{align}

. So wollen wir also die erste Greensche Identit\"{a}t lesen
\begin{align}
G(w,\hat{w}) = G(w,\hat{w})_{(0,l/2) }+ G(w,\hat{w})_{(l/2,l)} = 0 + 0 = 0\,.
\end{align}


, wenn es Punkte gibt, in denen die Voraussetzungen verletzt werden, dann nehmen wir das zum Anlass, die erste Greensche Identit\"{a}t abschnittsweise zu formulieren.

Wenn sie null sind, dann sind auch die Momente und Querkr\"{a}fte an den Balkenenden null. Bei einem Kragtr\"{a}ger mit einer Einzelkraft am rechten Ende kann die Biegelinie h\"{o}chstens ein Polynom dritten Grades sein (wegen $EI\,w^{IV} = 0$).

Ist der Kragtr\"{a}ger aber elastisch gebettet, dann muss die Biegelinie mindestens ein Polynom vierten Grades sein, weil sonst wegen $EI\,w^{IV} = 0 $ auch die Querkr\"{a}fte an den Balkenenden null sein m\"{u}ssten, s.o., was mit der Belastung nicht kompatibel ist.

Wenn sie Null sind, dann muss auch das Integral von $EI\,w^{IV} $ null sein und somit folgt
\begin{align}
G(w,1) &= \int_0^{\,l} (EI\,w^{IV} + c\,w)\,dx - \int_0^{\,l} c\,w(x)\,dx =\nn \\
& \int_0^{\,l} c\,w(x)\,dx - \int_0^{\,l} c\,w(x)\,dx = 0\,.
\end{align}
d.h. wir lernen weiter nichts dabei.

Dies ist ein Test und dieses noch ein Test und noch ein TestUnd noch ein Und noch ein Test und noch ein Test noch ein Test
Wenn wir den elastisch gebetteten Balken auf seiner ganzen L\"{a}nge um eine L\"{a}ngeneinheit nach dr\"{u}cken, $w(x) = 1$, durch Aufbringen einer konstanten Streckenlast $p = c\,w = c\,1$
\begin{align}
EI
\end{align}

Wenn die Funktionen $w(x)$ und $\hat{w}(x)$ diese Voraussetzungen nicht erf\"{u}llen, weil zum Beispiel die Querkr\"{a}fte oder die Momente in irgendeinem Punkt springen, dann unterbrechen wir die Integration an diesem Punkt und setzte sie hinter dem Punkt fort, d.h. wir formulieren die erste Greensche Identit\"{a}t dann in zwei Teilen.

Das ist im Grunde \"{a}hnlich 'trivial' wie zum Beispiel die durch Umstellen der Gleichung $\hat{x} \cdot 3 \cdot x = x\cdot 3 \cdot \hat{x}$ erhaltene Identit\"{a}t
\begin{align}
G(x,\hat{x}) = \hat{x} \cdot 3\,x - x\cdot 3 \cdot \hat{x} = 0\,.
\end{align}
Die erste Greensche Identit\"{a}t beruht auf der partiellen Integration der Arbeitsgleichung des Balkens, dem ersten Integral auf der rechten Seite. Damit die partielle Integration angewandt werden darf, m\"{u}ssen die beiden Funktionen $w \in C^4(0,l)$ und $\hat{w} \in C^2(0,l)$ liegen. Wenn dies erf\"{u}llt ist, dann ist das Ergebnis garantiert null.

Um Einzelkr\"{a}fte zu beschreiben, benutzen Mathematiker das Dirac-Delta und so w\"{a}re die Biegelinie des Balkens die L\"{o}sung des Randwertproblems
\begin{align}
EI\,w^{IV}(x) = \delta( 0.5\,l - x)\,P \qquad w(0) = w(l) = 0 \qquad M(0) = M(l) = 0
\end{align}

\begin{align}
G(w,\hat{w}) &= \int_0^{\,l} (EI\,w^{IV}(x) + P\,w'(x))\,\hat{w}(x)\,dx \nn \\
&+ [(- EI\,w'''(x) - P\,w'(x))\,\hat{w} + EI\,w''(x)\,\hat{w}'(x)]_{@0}^{@l}\nn \\
&- \int_0^{\,l} \frac{M\,\hat{M}}{EI} - P\,w'(x)\,\hat{w}(x)\,dx = 0\,.
\end{align}

Vom didaktischen Standpunkt aus ist das voll gelungen. Man w\"{u}sste es nicht besser zu machen, denn ohne diese Prinzipe w\"{u}rde man sich bei der statischen Untersuchung von Stockwerkrahmen mittels Arbeitsprinzipen (Mohr, Betti, etc.) hoffnungslos in der Formulierung der Greenschen Identit\"{a}ten verstricken.

Das Rechnen in der Statik ist zu 100 \% Mathematik, weil Zahlen in die Mathematik geh\"{o}ren. Der Tr\"{a}ger auf dem Papier ist eine mathematische Idealisierung und seine Eigenschaften kann man nur aus mathematischen Gesetzen herleiten nicht aber aus mechanischen Prinzipien.

Kraft und Weg nennt man duale Gr\"{o}{\ss}en, weil eine Kraft auf einem Weg eine Arbeit leistet. Der Arbeitsbegriff hat eine fundamentale Bedeutung f\"{u}r die Statik und Mechanik. Er kommt eigentlich spielerisch in die Statik hinein. Wenn zwei gegengleiche Kr\"{a}fte $\pm P$ an einem Lineals ziehen, sie also im Gleichgewicht sind
\begin{align}
P - P = 0\,,
\end{align}
dann kann man diese Gleichung mit einer beliebigen Zahl $ \delta u $ multiplizieren und sie bleibt richtig
\begin{align}
\delta u \,(P - P) = 0\,.
\end{align}
Anschaulich gesprochen hei{\ss}t dies, dass wenn man das Lineal \"{u}ber den Tisch schiebt, ihm also eine 'virtuelle' Verr\"{u}ckung $ \delta u $ erteilt, dass dann die virtuelle Arbeit der beiden Kr\"{a}fte null ist.

Virtuell soll hei{\ss}en, dass es sich dabei um eine gedachte Verschiebung handelt, um ein, wie Einstein sagen w\"{u}rde, blo{\ss}es Gedankenexperiment. Aber in unserem Beispiel ist diese virtuelle Verr\"{u}ckung nat\"{u}rlich ganz real.  \"{U}ber virtuelle Verr\"{u}ckung und virtuelle Verschiebungen und ihre Bedeutung f\"{u}r die Statik werden wir gleich noch N\"{a}heres zu sagen haben. Vorab sei aber schon einmal gesagt, dass ihnen nichts geheimnisvolles anhaftet, es sind einfach Funktionen, mit denen man eine Gleichung testet.

Wichtig ist an dieser Stelle nur, dass wir sehen, wie der Arbeitsbegriff und damit die Dualit\"{a}t zwischen Kraft und Weg, in die Statik und in die Mechanik hinein kommt---spielerisch.//

Das gilt \"{u}brigens auch f\"{u}r das {\em Prinzip der virtuellen Verr\"{u}ckungen\/}. Die Balkenendverformungen $\delta w$ und $\delta w'$ in $G(w,\delta w)$, s. (\ref{EqPvV}), m\"{u}ssen zu der Funktion $\delta w$ geh\"{o}ren, deren Moment $\delta M = - EI \,\delta w''(x)$ in der inneren Energie steht und sie d\"{u}rfen nicht frei erfunden sein. Normalerweise ist es aber so, dass man von einer bekannten Funktion $\delta w$ ausgeht und alles weitere aus dieser Funktion durch Differentiation berechnet. Dann passt es nat\"{u}rlich automatisch.//

Das gilt \"{u}brigens auch f\"{u}r das Prinzip der virtuellen Verr\"{u}ckungen. Die Balkenendverformungen $\delta w$ und $\delta w'$ in $G(w,\delta w)$, s. (\ref{EqPvV}), m\"{u}ssen zu der Funktion $\delta w$ geh\"{o}ren, deren Moment $\delta M = - EI \,\delta w''(x)$ in der inneren Energie steht und sie d\"{u}rfen nicht frei erfunden sein. Normalerweise ist es aber so, dass man von einer bekannten Funktion $\delta w$ ausgeht und alles weitere aus dieser Funktion durch Differentiation berechnet. Dann passt es nat\"{u}rlich automatisch. //

Und es ist nicht so, dass erst das Prinzip der virtuellen Verr\"{u}ckungen da war und auf der n\"{a}chsten Stufe die Greensche Identit\"{a}t, sondern es ist umgekehrt: Am Anfang war die Differentialgleichung aus der man durch partielle Integration des Arbeitsintegrals die Greensche Identit\"{a}t abgeleitet hat und man hat dann im Nachhinein die Ausdr\"{u}cke in der ersten Greenschen Identit\"{a}t als virtuelle \"{a}u{\ss}ere Arbeit bzw. als virtuelle innere Arbeit interpretiert und so kommt man hat
genau genommen sind es immer mathematische Beziehungen, die den Ausgangspunkt bilden und die dann nach eventuellen identischen Umformung
man immer finden wird, dass die virtuell \"{a}u{\ss}ere Arbeit gleich der virtuell inneren Energie ist//

Man beachte, dass $\delta w = x^2 $ sicherlich keine kleine virtuelle Verr\"{u}ckung ist, noch weniger, dass sie infinitesimal klein ist, denn die Auslenkung am Kragarmende betr\"{a}gt stolze 5 m. Wir k\"{o}nnen daher getrost den Einwand, dass das Prinzip der virtuellen Verr\"{u}ckungen nur f\"{u}r kleine oder infinitesimal kleine Verr\"{u}ckungen gilt, fallen lassen.
Es gibt mathematisch keinen Grund, warum die virtuellen Verr\"{u}ckungen 'klein' sein m\"{u}ssen.
//

\subsubsection*{Prinzip der virtuellen Verr\"{u}ckung}
\vspace{-0.7cm}
\begin{align}
G(w, \delta w) = \delta A_a - \delta A_i = 0
\end{align}
\\

Zu jedem der beiden St\"{a}be 1, 2, aus denen der Rahmen besteht geh\"{o}ren L\"{a}ngsverformungen $u_1, u_2$ und Biegeverformungen $w_1, w_2$ und zu jeder dieser Verformungen geh\"{o}rt eine Greensche Identit\"{a}t. Jede dieser  Identit\"{a}ten ist f\"{u}r sich null und daher kann man sie einfach addieren
\begin{align}
0 + 0 + \ldots + 0 = 0\,.
\end{align}
Ordnet man diese Identit\"{a}ten nach \"{a}u{\ss}erer und innerer Arbeit, dann sind sie, je nach Kontext, von der Gestalt
\begin{align} \label{Eq13}
A_a = A_i \qquad \delta A_a = \delta A_i \qquad \delta A_a^* = \delta A_i^*
\end{align}\\



Diese Identit\"{a}ten sehen zum Teil sehr kompliziert aus und der Leser wird sich jetzt vielleicht fragen muss ich denn all diese Terme mitnehmen, wenn ich einen Rahmen analysiere? Kann ich denn nicht bei meinem gewohnten Zugang bleiben?

Ja, nat\"{u}rlich, denn wenn man alle diese Terme zusammentr\"{a}gt und aufaddiert, dann zeigt sich, dass  viele Terme sich einfach an den Balkenenden gegenseitig aufheben und man am Schluss wieder bei den einfachen Ausdr\"{u}cken ist, die man aus der Statik kennt.

Jetzt kann man sich nat\"{u}rlich fragen, warum dieser ganze Aufwand, warum nicht gleich im alten Gleis bleiben, wenn doch dasselbe herauskommt? Aber nur auf diesem Wege verstehen wir, wo die einzelnen Terme herkommen und wie sich das alles zu einem Ganzen f\"{u}gt.\\


was bedeutet, dass sie in beiden Argumenten linear ist
und dass  die Reihenfolge der Argumente egal ist, $a(w,\hat{w}) = a(\hat{w},w)$\\

Formal ist $a(w,\hat{w})$ eine {\em symmetrische Bilinearform\/}, was bedeutet, dass sie in beiden Argumenten linear ist
und dass  die Reihenfolge der Argumente egal ist, $a(w,\hat{w}) = a(\hat{w},w)$.

Und man hat auch noch die Wahl der Greenschen Identit\"{a}t frei, d.h. man kann auch noch die Differentialgleichung frei w\"{a}hlen: So ist auch die Funktion $\sin\,(a\,x) $, wenn sie als L\"{a}ngsverschiebung eines Stabes genommen wird, im 'horizontalen' Gleichgewicht
\begin{align}
G(\sin\,a\,x,1) = \int_0^{\,l} - EA\,(- a^2\sin\,a\,x)\,dx + [EA\,\cos\,a\,x]_{@0}^{@l} = 0\,.
\end{align}
Das kann man mit allen Differentialgleichungen, also allen Identit\"{a}ten ausprobieren. Man wird immer finden, dass $G(\sin (ax), 1) = 0 $ ist.


Dass glatte alle glatten Funktionen $u$ oder $w$ im Gleichgewicht sind ist mehr oder minder trivial, denn dahinter steckt der Hauptsatz der Differential- und Integralrechnung
\begin{align}
\int_0^{\,l} F'(x)\,dx = F(l) - F(0)\,.
\end{align}
Am einfachsten sieht man das bei Differentialgleichungen zweiter Ordnung, wie beim Stab
\begin{align}
\int_0^{\,l} - EA\,u''(x)\,dx = - EA\,u'(l) + EA\,u'(0)
\end{align}
\\

In der Statik ist A die Belastung und B ist die virtuelle Verr\"{u}ckung. Wir lernen etwas \"{u}ber die Gr\"{o}{\ss}e einer Belastung, und wie sie verteilt ist, indem wir mit verschiedenen virtuellen Verr\"{u}ckungen an ihr wackeln. Darin liegt die gro{\ss}e Bedeutung der virtuellen Verr\"{u}ckungen. Um etwas \"{u}ber eine Kraft $\vek F$ zu klassifizieren, berechnen wir die Arbeiten, die die Kraft $\vek F $ auf den Wegen $\vek e_i $ leistet, wenn man sie also nacheinander in Richtung der Einheitsvektoren verschiebt. Die Koordinaten $ $ einer Kraft sind also eigentlich Arbeiten
\begin{align}
F_i = \vek e_i \dotprod \vek F
\end{align}
. Wir haben ein $A$ und ein $B$, also zwei Terme, zwei Gr\"{o}{\ss}en. Und das Skalarprodukt zwischen diesen Termen (Arbeit!) benutzen wir, um etwas \"{u}ber A zu erfahren.
, die Breite eines Schranks, indem wir einen Zollstock dagegen halten, das Gewicht eines Koffers, indem wir ihn anheben, die Gestalt eines Hauses, in dem wir es auf m\"{o}glichst viele Ebenen projizieren (Vorderansicht, Seitenansichten, R\"{u}ckansicht, etc.)
\\

Vereinfachen wir die erste Greensche Identit\"{a}t einmal auf den Ausdruck
\begin{align}
G(w, \delta w) = \int_0^{\,l} EI w^{IV}\,\delta w\,dx + [V\,\delta w]_{@0}^{@l} - a(w, \delta w) = 0
\end{align}
dann sieht man, dass man durch  geeignete Wahl der virtuellen Verr\"{u}ckung $\delta w$ eine der beiden Querkr\"{a}fte an den Balkenenden berechnen kann.

Vertauschen wir die Pl\"{a}tze
\begin{align}
G(w, \delta w) = \int_0^{\,l} EI \delta w^{IV}\, w\,dx + [\delta V\, w]_{@0}^{@l} - a(w, \delta w) = 0
\end{align}
dann sieht man, dass man durch eine geeignete Einzelkraft $P = 1 $ etwa in der Balkenmitte die Durchbiegung $w(0.5\,l)$ in der Balkenmitte berechnen kann.

\begin{align} \label{Eq13}
A_a = A_i \qquad \delta A_a = \delta A_i \qquad \delta A_a^* = \delta A_i^*
\end{align}
W\"{a}hlt man f\"{u}r $\hat{w} $ eine virtuelle Verr\"{u}ckung $\delta w $, dann entsteht das Prinzip der virtuellen Verr\"{u}ckungen. Bleibt man auf der Diagonalen, $\hat{w} = w $, dann formuliert man den Energieerhaltungssatz (bis auf den Faktor $1/2$), und setzt man $\delta w $ an die erste Stelle und $w $ an die zweite Stelle, dann formuliert man das {\em Prinzip der virtuellen Kr\"{a}fte\/}. Durch das Vertauschen sind es jetzt die Kr\"{a}fte, die zu $\delta w $ geh\"{o}ren, (also $EI\,\delta w^{IV}, \delta V, \delta M$), die virtuelle Arbeit auf den Wegen der Originalfunktion $w $ leisten.

In der Literatur wird bei der Formulierung von $G(\delta w^*,w) = 0$ die Funktion $\delta w $ meist mit einem Stern geschrieben.
Die gro{\ss}e Bedeutung der Identit\"{a}ten f\"{u}r die Statik beruht darauf, dass sie das Fundament der Arbeits- und Energieprinzipe der Statik bilden. \\



Das {\em Prinzip der virtuellen Kr\"{a}fte\/} ist im Grunde identisch mit dem Prinzip der virtuellen Verr\"{u}ckungen. Nur das wir uns diesem jetzt aus einer anderen Richtung n\"{a}hern. Jetzt spielt die Biegelinie $w$ des Tr\"{a}gers die virtuelle Verr\"{u}ckung an einem System von Kr\"{a}ften $\delta K^*$.

An virtuellen Kr\"{a}ften ist nichts geheimnisvolles oder ungew\"{o}hnliches. Es sind normale Lasten + zugeh\"{o}rigen Lagerkr\"{a}ften, an denen mit $w$ gewackelt wird und so wie in jedem Lastfall das Prinzip der virtuellen Verr\"{u}ckungen gilt, so auch hier.

Wertvoll ist das Prinzip deswegen, weil man durch geschickte Wahl des Systems $\delta K^*$ aus der Identit\"{a}t $G(\delta w^*, w) = 0$ Informationen \"{u}ber Einzelverformungen ziehen kann. Wir werden das an einem Beispiel erl\"{a}utern.


\subsubsection{Addition der inneren Energien}
Wie sieht es mit der inneren Energie aus, wenn federnde Lager vorhanden sind? Eine Schraubenfeder, die einen Durchlauftr\"{a}ger st\"{u}tzt kann man in Gedanken einem Pendelstab gleichsetzen, also wie ein vollwertiges weiteres Bauteil ansehen. Nur dass die L\"{a}ngssteifigkeit der Feder nicht $EA $ ist, sondern gleich der Federsteifigkeit $k$. Bei einem Stab wird integriert
\begin{align}
\delta A_i = \int_0^{\,l} \frac{N\,\delta N}{EA}\,dx\,,
\end{align}
bei einer Feder, die ja nur eine Kopf- und Fu{\ss}verschiebung hat, $u_1$ und $u_2$, wird dagegen multipliziert, das Skalarprodukt
\begin{align}
\delta A_i = \vek \delta \,\vek u^T \,\vek K\,\vek u =   \left [ \delta u_1 \,\, \delta u_2 \right ]   \left[ \barr {r @{\hspace{4mm}}r @{\hspace{4mm}}r
@{\hspace{4mm}}r @{\hspace{4mm}}r}
      k & -k  \\
      -k & k \\
     \earr \right]\left [\barr{c}  u_1 \\  u_2\earr \right ]
  \, ,
\end{align}
ausgewertet, wobei $\vek K $ die Steifigkeitsmatrix der Feder ist.

Nun sind wir neugierig geworden, wie lautet denn die erste Greensche Identit\"{a}t einer Feder? Die gibt es nicht, wenn wir alles durch die Bewegungen $u_1, u_2 $ an den Federenden beschreiben. In
matrizieller Schreibweise lautet das Federgesetz
\begin{align}
\vek K\,\vek u = \vek f
\end{align}
und zu dieser symmetrischen Matrix geh\"{o}rt die Identit\"{a}t
\begin{align}
G(\vek u, \vek \delta\,\vek u) = \vek \delta\,\vek u^T\,\vek K\,\vek u -  \vek u^T\,\vek K\,\vek \delta\,\vek u = 0\,,
\end{align}
die alles beinhaltet, was man f\"{u}r die Energie- und Arbeitsprinzipe der Feder ben\"{o}tigt.

Wir werden sp\"{a}ter, im Zusammenhang mit dem Thema lineare Algebra und Statik noch einmal darauf zur\"{u}ckkommen.

Bei einer Feder, die sich auf dem Boden abst\"{u}tzt, ist ein Freiheitsgrad null, etwa $u_2 = 0$, und so reduziert sich die virtuelle innere Energie eines Balkens mit einem federnden Endlager auf den Ausdruck
\begin{align}
\delta A_i = \int_0^{\,l} \frac{M\,\delta M}{EI}\,dx + \delta u_1\,k\,u_1
\end{align}
und die innere Energie kann man daraus sofort ablesen, wenn man auf die Diagonale geht, $\delta u = u$ und $\delta u_1 = u_1$, und pflichtgem\"{a}{\ss} alles mit dem Faktor $1/2$ multipliziert
\begin{align}
A_i = \frac{1}{2}\, \int_0^{\,l} \frac{M\, M}{EI}\,dx + \frac{1}{2}\, u_1\,k\,u_1
\end{align}
%%%%%%%%%%%%%%%%%%%%%%%%%%%%%%%%%%%%%%%%%%%%%%%%%%%%%%%%%%%%%%%%%%%%%%%%%%%%%%%%%%%%%%%%%%%%%%%%%%%
\section{Details}

Im folgenden wollen wir nun auch im Detail zeigen, wie die Arbeits- und Energieprinzipe der Statik auf der ersten Greenschen Identit\"{a}t beruhen.\\

Ein noch einfacheres Beispiel f\"{u}r diesen \"{U}bergang von leeren Formalismen zur Statik ist das folgende: Eine Feder habe die Federsteifigkeit $k$. Bei der Multiplikation von $k$ mit zwei Zahlen $u$ und $\hat{u}$ spielt die Reihenfolge keine Rolle und deswegen ist
\begin{align} \label{Eq23}
G(u,\hat{u}) = u\,k\,\hat{u} - \hat{u}\,k\,u = 0
\end{align}
eine Identit\"{a}t. Nun sei $u$ die Auslenkung der Feder bei einer Belastung mit einer Kraft $f$, gen\"{u}ge also der Gleichung $k\,u = f$, dann folgt aus (\ref{Eq23}) die Gleichung
\begin{align}
\delta A_i = u\,k\,\hat{u} = \hat{u}\,f = \delta A_a \,,
\end{align}
also die G\"{u}ltigkeit des Prinzips der virtuellen Verr\"{u}ckungen f\"{u}r die Feder. Die Ausgangs\-gleich\-ung (\ref{Eq23}) ist trivial, mit der Ersetzung $k\,u = f$ wird daraus Statik.\\

%----------------------------------------------------------------------------------------------------------
\subsubsection{Die Arbeitsgleichung}
Die Mohrsche Arbeitsgleichung ist das Universalwerkzeug, um Verformungen an rahmenartigen Tragwerken zu berechnen
\begin{align}
\bar{1}\cdot\delta = \int_0^{\,l} \frac{M\,\bar{M}}{EI}\,dx\,.
\end{align}
Wir wollen an dieser Stelle, wenigstens einmal, einen Beweis f\"{u}r diese Formel geben, also zeigen, wie genau dieses Ergebnis zustande kommt.

Zun\"{a}chst erstellt man eine 1:1 Kopie des Tr\"{a}gers und belastet diese Kopie in Richtung der gesuchten Verformung mit einer Kraft $\bar{P} = \bar{1} $. Die Biegelinie $ $ unter der Einzelkraft kann man nicht in geschlossener Form angeben, sie liegt nicht in $C^4$, weil die Querkraft in der Balkenmitte springt
\begin{align}
\bar{V}_1 - \bar{V}_2 = \bar{P}\,.
\end{align}
Die beiden Teile $ $ und $ $ der Biegelinie sind homogene L\"{o}sungen der Balkengleichung (keine Belastung auf der freien Strecke) und in der Mitte des Balkens gehen sie, bis auf den Sprung in der Querkraft stetig ineinander \"{u}ber, $\bar{w}_1 = \bar{w}_2$, $\bar{w}_1' = \bar{w}_2'$ und $\bar{M_1} = \bar{M_2}$, so dass sich bei der abschnittsweisen Formulierung der ersten Greenschen Identit\"{a}t in der Reihenfolge $\bar{w},w $. F\"{u}r den ersten Abschnitt erhalten wir, $\dotprod = l/2$
\begin{align}
G(\bar{w}_1,w)_{(0,\dotprod)} &= \int_0^{\dotprod} EI\,\bar{w}_1^{IV}\,w\,dx + \bar{V}_1(\dotprod)\,w(\dotprod) - M_1(\dotprod)\,w'(\dotprod)\nn \\
 &- \bar{V}_1(0)\,w(0) + \bar{M}_1(0)\,w'(0) - \int_0^{\,\dotprod} \frac{M\,\bar{M}_1}{EI}\,dx = 0
\end{align}
Wegen $EI\,\bar{w}_1^{IV} = 0$ und $w(0) = 0$ und $\bar{M}_1(0) = 0$ reduziert sich das auf
\begin{align}
G(\bar{w}_1,w)_{(0,\dotprod)} &= \bar{V}_1(\dotprod)\,w(\dotprod) - M_1(\dotprod)\,w'(\dotprod)  - \int_0^{\dotprod} \frac{M\,\bar{M}_1}{EI}\,dx = 0
\end{align}
und analog f\"{u}r das zweite Intervall
\begin{align}
G(\bar{w}_2,w)_{(\dotprod,l)} &= -\bar{V}_2(\dotprod)\,w(\dotprod) + M_2(\dotprod)\,w'(\dotprod)  - \int_{\dotprod}^{\,l} \frac{M\,\bar{M}_1}{EI}\,dx = 0
\end{align}
In der Mitte ist $\bar{M}_1 = \bar{M}_2$ und $w'$ ist stetig, so dass
\begin{align}
G(\bar{w}_1,w)_{(0,\dotprod)} + G(\bar{w}_2),w)_{(\dotprod,l)} = \bar{P}\,w(\dotprod) - \int_0^{\,l} \frac{\bar{M}\,M}{EI}\,dx = 0
\end{align}
oder
\begin{align}
w(\dotprod) =  \int_0^{\,l} \frac{\bar{M}\,M}{EI}\,dx\,.
\end{align}
Das ist die Mohrsche Arbeitsgleichung. In der Literatur wird zum Beweis dieser Formel an das {\em Prinzip der virtuellen Kr\"{a}fte\/} appelliert, das Ergebnis durch Anwendung eines mechanischen Prinzips hergeleitet.

Das {\em Prinzip der virtuellen Kr\"{a}fte\/} besagt das folgende: wenn $w$ die Biegelinie eines Balkens ist, (zu vorgegebener \"{a}u{\ss}erer Belastung), das unter der Einwirkung von \"{a}u{\ss}eren Kr\"{a}ften im Gleichgewicht ist


 kann man Durchbiegungen berechnen. Auch diese Gleichung basiert auf der ersten Greenschen Identit\"{a}t und zwar auf dem {\em Prinzip der virtuellen Kr\"{a}fte\/}
\begin{align}
G(\delta w^*,w) = \int_0^{\,l} EI\,\delta w^{*IV}\,w\,dx + [\delta V^*\,w - M*\,\delta w'] - \int_0^{\,l} \frac{M\,M*}{EI} dx = 0
\end{align}
Das erste Argument ist also $\delta u*$ und erst das zweite Argument ist die Biegelinie des Tr\"{a}gers.

Anschaulich geschieht das Folgende: Man belastet eine Kopie des Tr\"{a}gers mit einer Einzelkraft $\bar{P} = \bar{1}$ (in Richtung der gesuchten Verschiebung). Die Arbeit, die diese 'virtuelle' Kraft auf dem Weg $w(x)$ am Ort von $\bar{P}$ leistet, ist gleich der virtuellen inneren Energie zwischen $w*$, der Biegelinie zu $\bar{P} = \bar{1}$, und der Biegelinie $w$ des Tr\"{a}gers.\\

So weit die verbale Beschreibung, aber mit Begriffen l\"{a}sst sich nichts beweisen, dazu muss man Mathematik machen.
\\

\subsubsection{Prinzip der virtuellen Verr\"{u}ckungen}
\begin{align}
G(w,\delta w) = \int_0^{\,l} EI\,w^{IV}(x)\,\delta w(x)\,dx + [V\,\delta w - M\,\delta w']_{@0}^{@l} - \int_0^{\,l} \frac{M\, \delta M}{EI}\,dx = 0
\end{align}
\subsubsection{Prinzip der virtuellen Kr\"{a}fte}
\begin{align}
G(\delta w, w) = \int_0^{\,l} EI\,\delta w^{IV}(x)\, w(x)\,dx + [\delta V\, w - \delta M\, w']_{@0}^{@l} - \int_0^{\,l} \frac{\delta M\,  M}{EI}\,dx = 0
\end{align}
\subsubsection{Energieerhaltung}
\begin{align}
\frac{1}{2}\,G(w, w) &= \frac{1}{2}\, \int_0^{\,l} EI\,w^{IV}(x)\, w(x)\,dx + \frac{1}{2}\, [V\, w - M\, w']_{@0}^{@l} \nn\\
&- \frac{1}{2}\, \int_0^{\,l} \frac{M\,  M}{EI}\,dx = 0
\end{align}
\subsubsection{{\em Satz von Betti\/}}
\begin{align}
\text{\normalfont\calligra B\,\,}(w_1, w_2) &= \text{\normalfont\calligra G,\,}(w_1,w_2) - \text{\normalfont\calligra G\,\,}(w_2,w_1) = \int_0^{\,l} EI\,w_1^{IV}\,w_2\,dx\nn \\
& + [V_1\,w_2 - M_1\,w_2']_{@0}^{@l} - [V_2\,w_1 - M_2\,w_1']_{@0}^{@l} - \int_0^{\,l} w_1\,EI\,w_2^{IV}dx = 0
\end{align}

\subsection{Die Balkengleichung}
Wenn die Biegesteifigkeit $ EI $ l\"{a}ngs des Balkens konstant ist, dann wird der Zusammenhang zwischen der Durchbiegung $ w(x) $ und der Belastung $ p(x) $ beschrieben durch die Differentialgleichung vierter Ordnung
\begin{align}
EI\,w^{IV}(x) = p(x)\,.
\end{align}
Die erste Greensche Identit\"{a}t dieser Differentialgleichung lautet
\begin{align}
G(w,\hat{w}) = \int_0^{\,l} EI\,w^{IV}(x)\,\hat{w}(x)\,dx + [V\,\hat{w} - M\,\hat{w}']_{@0}^{@l} - \int_0^{\,l} \frac{M\,\hat{M}}{EI}\,dx = 0
\end{align}
mit
\begin{align}
M(x) = - EI\,w''(x) \qquad V(x) = - EI\,w'''(x)
\end{align}
als dem Biegemoment bzw. der Querkraft der Biegelinie.

Diese erste Greensche Identit\"{a}t erh\"{a}lt man, wenn man das Integral
\begin{align}
\int_0^{\,l} EI\,w^{IV}(x)\,\hat{w}(x)
\end{align}
mittels partieller Integration umformt. Damit man partielle Integration anwenden kann, setzen wir voraus, dass die beiden Funktionen $w$ und $\hat{w}$ hin\-reich\-end glatt sind, $w(x) \in C^4(0,l)$ und $\hat{w}(x) \in C^2(0,l)$. Die Forderung $w \in C^4 $ ist ziemlich restriktiv. Schon wenn die Belastung $p$ in der Mitte des Balkens springt, also dort eine Unstetigkeitsstelle hat, liegt $ w $ nicht mehr in $C^4$, weil die vierte Ableitung von $ w $ diesen Sprung ja enthalten muss.

Aber das kann man leicht umgehen, indem man die erste Greensche Identit\"{a}t abschnittsweise formuliert, also f\"{u}r das Intervall $(0, l/2) $, vom linken Lager bis zur Balkenmitte, und dann noch einmal von der Balkenmitte bis zum rechten Lager, also das Intervall $(l/2,l)$
\begin{align}
G(w,\hat{w}) = G(w,\hat{w})_{(0,l/2) }+ G(w,\hat{w})_{(l/2,l)} = 0 + 0 = 0\,.
\end{align}
Insbesondere Einzel\-kr\"{a}fte $P$ oder Einzelmomente $M$ machen solche Zwangs\-pausen n\"{o}tig. Die Bilanz $G(w,\hat{w}) = 0$ gilt wirklich nur f\"{u}r Intervalle, in denen $w$ in $C^4$ liegt und $\hat{w}$ in $C^2$.

Ein Beispiel soll f\"{u}r viele gelten. Der Balken in Bild X wird in seiner Mitte durch eine Einzelkraft $ P = 10 $ belastet. Um dieses Problem zu l\"{o}sen, muss man den Balken in zwei Teile unterteilen, $w_L$ und $w_R$, vom linken Lager bis zur Mitte und von der Mitte bis zum rechten Lager. An der \"{U}bergangsstelle $\dotprod = l/2$ zwischen diesen beiden Intervallen gilt
\begin{align}
w_L(\dotprod) = w_R(\dotprod) \qquad w_L'(\dotprod) = w_R'(\dotprod) \qquad M_L(\dotprod) = M_R(\dotprod)
\end{align}
aber die Querkr\"{a}fte springen um den Betrag der Einzelkraft $P$
\begin{align}
V_R(\dotprod ) - V_L(\dotprod ) = P
\end{align}
Den Energieerhaltungssatz formuliert man jetzt f\"{u}r das linke Intervall und f\"{u}r das rechte Intervall separat und addiert die beiden Gleichungen und erh\"{a}lt so
\begin{align}
G(w,w)_{(0,l/2)} + G(w_R,w_R)_{(l/2, 0) } = P \,w(\dotprod) - \int_0^{\,l} \frac{M^2}{EI}\,dx = 0
\end{align}
Mit dem weiteren k\"{o}nnen wir uns kurz fassen, weil es nur eine Wiederholung des oben schon Gesagten ist.


%----------------------------------------------------------------------------------------------------------
\begin{figure}[tbp]
\centering
\if \bild 2 \sidecaption \fi
\includegraphics[width=0.9\textwidth]{\Fpath/S2}
\caption{Einfeldtr\"{a}ger} \label{S2}
%
\end{figure}%

%\end{document}

%%%%%%%%%%%%%%%%%%%%%%%%%%%%%%%%%%%%%%%%%%%%%%%%%%%%%%%%%%%%%%%%%%%%%%%%%%%%%%%%%%%%%%%%%%%%%%%%%%%
\subsection{Das Prinzip vom Minimum der potentiellen Energie}

Um dieses Prinzip herzuleiten, beginnen wir am besten mit dem einfachsten m\"{o}glichen statischen Element, einer Feder, s. Bild \ref{FederEnergie}.
Das Federgesetz
\begin{align}
k\,u = f
\end{align}
besagt, dass die Auslenkung $ u $ der Feder proportional zur aufgebrachten Kraft $ f $ ist. Der Faktor $ k $ hat die Dimension [F/L] und wird die Steifigkeit der Feder genannt.

Die Kraft $ f $, die die Feder nach unten zieht, leistet eine Arbeit und weil es Eigenarbeit ist tr\"{a}gt sie den Faktor $1/2$
\begin{align}
A_a = \frac{1}{2}\,f\,u\,.
\end{align}
Diese \"{a}u{\ss}ere Arbeit wird als innere Energie in der Feder gespeichert
\begin{align}\label{Eq5}
A_i = \frac{1}{2}\, k\,u^2\,.
\end{align}
Und wir erwarten nat\"{u}rlich, dass in der Gleichgewichtslage die \"{a}u{\ss}ere Arbeit und die innere Energie \"{u}bereinstimmen, $A_a = A_i$
\begin{align}
A_a = \frac{1}{2}\, f\,u = \frac{1}{2}\, k\,u\,u = \frac{1}{2}\, k\,u^2 = A_i\,.
\end{align}
Um diesen \"{U}bergang von $A_a$ zu $A_i$ m\"{o}glich zu machen, muss die innere Energie genau die Form (\ref{Eq5}) haben.

Tr\"{a}gt man den Verlauf der Funktion $1/2\,f\,u $ und der Funktion $1/2\,k\,u^2 $ in einem Koordinatenkreuz auf,
dann ist die Auslenkung $ u$ der Feder  unter der Wirkung der Kraft $ f $ genau der Punkt $u$, in dem sich die beiden Kurven schneiden, siehe Bild \ref{FederEnergie}.

Nun gibt es noch eine weitere Kurve in Bild \ref{FederEnergie} und das ist die potentielle Energie $\Pi$ der Feder
\begin{align}
\Pi(u) = \frac{1}{2}\, k\,u^2 - f\,u\,.
\end{align}
Der Faktor $1/2$ macht, dass sich bei der Bildung der Ableitung die 2 wegk\"{u}rzt
\begin{align}
\Pi'(u) = k\,u - f
\end{align}
und so, weil die Auslenkung $ u $ der Feder dem Federgesetz $k\,u = f $ gen\"{u}gt, die potentielle Energie im Gleichgewichtspunkt $u$ eine horizontale Tangente, $\Pi'(u) = 0$, hat.

Die interessante Beobachtung ist nun, siehe Bild \ref{FederEnergie}, dass der Punkt, in dem sich die \"{a}u{\ss}ere und innere Arbeit schneiden,  auch gleichzeitig der Punkt ist, in dem die potentielle Energie ihr Minimum hat (Es ist wirklich ein Minimum denn $\Pi'' = k > 0$).

Wie man im Bild \ref{FederEnergie} sieht, steigt am Anfang die \"{a}u{\ss}ere Arbeit schneller als die innere Energie, aber dann passieren die beiden Kurven einen Punkt, von dem ab die innere Energie schneller w\"{a}chst als die \"{a}u{\ss}ere Arbeit. Dieser Schnittpunkt ist genau der Gleichgewichtspunkt. Nur in diesem Punkt gilt $A_a = A_i$.

W\"{u}rde von Anfang an die innere Energie schneller steigen, als die \"{a}u{\ss}ere Arbeit, dann w\"{u}rde sich die Feder \"{u}berhaupt nicht bewegen, dann w\"{a}re schon im Nullpunkt der Wettlauf zu Ende.

Setzen wir alles auf eins, also $ k = 1$ und $ f = 1 $, dann liegt der Gleich\-gewichtspunkt genau bei $ u = 1$. Woraus folgt, dass die Mechanik im Grunde auf der Tatsache beruht, dass im Intervall $(0,1)$
die Ungleichung $u > u^2$ gilt und danach $u^2 > u$ ist. Die Zahl $u = 0.5$ ist gr\"{o}{\ss}er als ihr Quadrat $u^2 = 0.25$, aber $u = 1.5$ ist kleiner als sein Quadrat $u^2 = 2.25$. Einzig im Punkt $u = 1$ ist $u = u^2$.

Das Prinzip vom Minimum der potentiellen Energie fasst nun diese Beobachtungen wie folgt zusammen: Die Auslenkung $ u $ der Feder unter der Wirkung der Kraft $ f $ macht die potentielle Energie der Feder zum Minimum. Wenn man also nur lange genug Zufallszahlen $ u $ in die Funktion $\Pi(u)$ einsetzen w\"{u}rde, sich eine Liste der Wert $\Pi(u)$ machen w\"{u}rde, dann w\"{u}rde man automatisch zu der gesuchten Gleichgewichtslage $u$ der Feder gef\"{u}hrt.

\subsubsection{Minimum oder Maximum?}
Nun kommt eine wichtige Beobachtung. In der tiefsten Lage ist die potentielle Energie negativ, wie man durch Einsetzen ($ k\,u = f$) verifiziert
\begin{align}
\Pi(u) = \frac{1}{2}\,k\,u^2 - f\,u = \frac{1}{2}\, f\,u - f\,u = - \frac{1}{2}\, f\,u\,.
\end{align}
Nun ist aber die Auslenkung $ u $ der Sieger in einem Wettbewerb, es gibt keine andere Zahl, die die potentielle Energie kleiner macht. Und das hei{\ss}t doch anschaulich, dass $ u $ den Abstand $|\Pi(u)|$ vom Nullpunkt m\"{o}glichst gro{\ss} macht. Also ist doch das Prinzip vom Minimum der potentiellen Energie eigentlich ein Maximumsprinzip: M\"{o}glichst weit weg von null mit $|\Pi(u)|$, das ist das Bestreben von $u$. Nur weil die potentielle Energie in der Gleichgewichtslage negativ ist, ist das dasselbe, wie das Minimum der potentiellen Energie. Aber viele Ingenieure interpretieren das Prinzip eben so, wie es die Wortwahl (anscheinend)  suggeriert, mit m\"{o}glichst wenig Anstrengung zum Ziel kommen, die potentielle Energie m\"{o}glichst klein machen, m\"{o}glichst nahe an Null r\"{u}cken, w\"{a}hrend die wahre Bedeutung genau das Gegenteil ist. Die Kraft $ f $ strebt danach m\"{o}glichst viel Energie aus dem Federsystem herauszuziehen, $|\Pi(u)|$ m\"{o}glichst gro{\ss} zu machen, und sie entscheidet sich f\"{u}r die Auslenkung, die Zahl $u$, die diesem Zweck am besten dient.\\

Diese Umkehr dessen, was das Minimum der potentiellen Energie eigentlich bedeutet, zieht sich durch die ganze Statik und Mechanik. Wir werden sp\"{a}ter sehen, dass man die statischen Probleme in zwei Klassen einteilen kann: Ent\-weder werden Kr\"{a}fte aufgebracht oder Verformungen. Wenn Kr\"{a}fte aufgebracht werden, dann ist die potentielle Energie in der Gleichgewichtslage negativ und das Ziel der Belastung ist es, $|\Pi(u)|$ m\"{o}glichst gro{\ss} zu machen, m\"{o}glichst weit von null zu kommen. Wenn Verformungen eingepr\"{a}gt werden, dann ist die potentielle Energie positiv, liegt also rechts vom Nullpunkt und wenn man jetzt die potentielle Energie minimiert, dann sucht man den Verformungszustand $ u $, der die potentielle Energie m\"{o}glichst nahe an Null r\"{u}ckt. Dann hat das Prinzip vom Minimum der potentiellen Energie genau die Bedeutung, die der Ingenieur ihm normalerweise unterlegt. Das Tragwerk versucht mit m\"{o}glichst wenig Aufwand an innerer Energie die Verformungen zu erdulden, die ihm aufgezwungen werden.


%%%%%%%%%%%%%%%%%%%%%%%%%%%%%%%%%%%%%%%%%%%%%%%%%%%%%%%%%%%%%%%%%%%%%%%%%%%%%%%%%%%%%%%%%%%%%%%%%%%
\subsection{Der Stab}
Wir wollen nun diese Ergebnisse auf einen Stab \"{u}bertragen und zwar den Stab in Bild eins. Der Einfachheit halber, nehmen wir an das die Zugkraft am rechten Ende des Stabes null ist, dass die L\"{a}ngsverschiebung des Stabes also die folgenden Eigenschaften hat
\begin{align} \label{Eq10}
-EA\,u''(x) = p(x) \qquad u(0) = 0\,, \qquad N(l) = 0\,.
\end{align}
(Mit einer Zugkraft am rechten Ende geht der Beweis aber genauso).

Den Ausdruck
\begin{align}
\Pi(u) = \frac{1}{2}\, \int_0^{\,l} \frac{N^2}{EA}\,dx - \int_0^{\,l} p(x)\,u(x)\,dx
\end{align}
nennt man die potentielle Energie des Stabes. Warum hier ein Faktor $1/2$ vor dem ersten Integral steht, aber bei dem zweiten Integral fehlt hat damit zu tun, dass unter dem ersten Integral
ein Quadrat steht, $N^2 = (EA\,u')^2$, und wenn man dieses nach $u$ ableitet, dann k\"{u}rzt sich der Faktor $1/2$ weg.
Dies soll an dieser Stelle als Hinweis gen\"{u}gen. Zur genauen Kl\"{a}rung der Details ben\"{o}tigt man den Begriff der Gateaux-Ableitung, was hier zu weit f\"{u}hren w\"{u}rde.

Mathematisch ist die potentielle Energie ein sogenanntes {\em Funktional\/}, ein Ausdruck, in den man eine Funktion einsetzt und eine Zahl zur\"{u}ckbekommt, also, eine Funktion von Funktionen.

\"{A}hnlich wie bei der Feder gilt auch f\"{u}r den Stab ein Minimumsprinzip: Die L\"{a}ngsverschiebung $ u(x) $ des Stabes macht die potentielle Energie auf $V$ zum Minimum, ist also der Sieger in einem Wettbewerb, der auf $ V $ stattfindet. Was ist $V$? Die Menge $V$, oder was besser klingt, der Ansatzraum $V$, wird gebildet von allen Funktionen $ u(x)$, die den Lagerbedingungen des Stabes gen\"{u}gen, die also an der Stelle $ x = 0$ den Wert null haben.

Eine Schar von solchen Funktionen sind zum Beispiel die Funktionen
\begin{align}\label{Eq6}
\sin(x), \sin(x^2), \sin(x^3), \ldots, x, x^2, x^3, \ldots x^n, e^x - 1, e^{x^2} - 1,
\end{align}
Alle diese Funktionen, und viele weitere mehr, nehmen also an dem Wettbewerb teil, bei dem es darum geht die Funktion $u(x)$ zu finden, die den kleinsten Wert f\"{u}r die potentielle Energie liefert. Wir behaupten, dass der Sieger die L\"{a}ngsverschiebung $u(x)$ des Stabes ist.

Um dies zu zeigen, argumentieren wir wie folgt: Wenn $u(x)$ den tiefsten Punkt markiert, dann muss in jedem Nachbarpunkt  $u(x) + \hat{u}(x)$ die potentielle Energie gr\"{o}{\ss}er sein, es muss also gelten
\begin{align} \label{Eq7}
\Pi(u + \hat{u}) - \Pi(u) > 0\,.
\end{align}
Die Funktion, die wir zu $ u(x) $ hinzu addieren ist irgendeine Funktion $\hat{u}$ aus $V$, so dass die Bedingung $u(0) + \hat{u}(0) = 0 $ erhalten bleibt.

Um (\ref{Eq7}) zu zeigen, ist es sinnvoll, die potentielle Energie in eine abk\"{u}rzenden Form
\begin{align}
\Pi(u) = \frac{1}{2}\, \int_0^{\,l} \frac{N^2}{EA}\,dx - \int_0^{\,l} p(x)\,u(x)\,dx = \frac{1}{2}\, a(u,u) - (p,u)
\end{align}
zu schreiben. Dabei ist (man beachte, dass das zweite Argument $\hat{u}$ zun\"{a}chst nicht dasselbe ist wie das erste)
\begin{align}
a(u,\hat{u}) = \int_0^{\,l} EA\,u'\,\hat{u}'\,dx = \int_0^{\,l} \frac{N\,\hat{N}}{EA}\,dx
\end{align}
und
\begin{align}
(p,u) = \int_0^{\,l} p(x)\,u(x)\,dx\,.
\end{align}
Der Ausdruck $a(u,\hat{u})$ ist in beiden Argumenten linear. Das bedeutet, wenn $ u(x) $ und $ \hat{u}(x)$ zusammengesetzte Funktionen sind,
\begin{align}
u(x)= c_1\,u_1(x) + c_2\,u_2(x) \qquad \hat{u}(x) = d_1\,\hat{u}_1(x) + d_2\,\hat{u}(x)\,,
\end{align}
hier sind $c_1, c_2$ und $d_1, d_2$  beliebige Zahlen, dass dann die Form $a(u,\hat{u}$) in ihre einzelnen Bestandteile zerlegt werden kann
\begin{align}
a(u,\hat{u}) = c_1\,d_1\,a(u_1,\hat{u}_1) + c_1\,d_2\,a(u,\hat{u}_2) + c_2\,d_1\,a(u_1,\hat{u}_1) + c_2\,d_2\,a(u_2,\hat{u}_2)
\end{align}
Wegen dieser 'zweifachen' Linearit\"{a}t, im ersten und im zweiten Argument, nennt man $ a(u,\hat{u}) $ eine {\em Bilinearform\/} und weil $a(u,\hat{u})$ dasselbe ist wie $a(\hat{u},u)$, eine
{\em symmetrische Bilinearform\/}.

Das Arbeitsintegral $(p,u)$ hei{\ss}t eine Linearform, weil es in dem zweiten Argument linear ist,
\begin{align}
(p,u_1 + u_2) = (p,u_1) + (p,u_2)\,.
\end{align}
So vorbereitet k\"{o}nnen wir nun an den Beweis von (\ref{Eq7}) gehen. Die potentielle Energie an der Nachbarstelle hat den Wert
\begin{align} \label{Eq8}
\Pi(u + \hat{u}) &= \frac{1}{2}\,a(u + \hat{u},u + \hat{u}) - (p,u + \hat{u}) \nn \\
&= \frac{1}{2}\,a(u,u) + a(u,\hat{u}) + \frac{1}{2}\, a(\hat{u},\hat{u}) - (p,u) - (p,\hat{u})\,.
\end{align}
Hierbei haben wir die Symmetrie der Bilinearform ausgenutzt
\begin{align}
\frac{1}{2}\, a(u,\hat{u}) + \frac{1}{2}\, a(\hat{u},u)  = a(u,\hat{u})\,.
\end{align}
Man sieht leicht, dass (\ref{Eq8}) identisch ist mit
\begin{align}
\Pi(u + \hat{u}) &=\Pi(u) + a(u,\hat{u}) - (p,\hat{u}) + \frac{1}{2}\, a(\hat{u},\hat{u})
\end{align}
und somit folgt
\begin{align} \label{Eq9}
\Pi(u + \hat{u}) -\Pi(u) = a(u,\hat{u}) - (p,\hat{u})+ \frac{1}{2}\, a(\hat{u},\hat{u})
\end{align}
Nun ist aber, ausgeschrieben,
\begin{align}
a(u, \hat{u}) - (p,\hat{u}) = \int_0^{\,l} \frac{N(x)\,\hat{N}(x)}{EA})\,dx - \int_0^{\,l} p(x)\,\hat{u}(x)\,dx
\end{align}
was ja gerade die erste Greenschen Identit\"{a}t ist, von der wir wissen, dass sie null ist,
\begin{align}
G(u,\hat{u}) = \int_0^{\,l} p(x)\,\hat{u}(x)\,dx + \underbrace{N(l)\,\hat{u}(l) - N(0)\,\hat{u}(0)}_{= 0} - \int_0^{\,l} \frac{N(x)\,\hat{N}(x)}{EA}\,dx = 0
\end{align}
Die Arbeiten an den beiden Enden des Stabes sind Null, weil $N(l) = 0$ und $\hat{u}(0)= 0$ und wir d\"{u}rfen f\"{u}r $- EA\,u''(x) = p(x)$ setzen, weil wir wissen das die L\"{a}ngsverschiebung die Differentialgleichung in (\ref{Eq10}) erf\"{u}llt (der Ingenieur w\"{u}rde sagen, weil der Stab im Gleichgewicht ist).

Somit reduziert sich (\ref{Eq9}) auf
\begin{align}
\Pi(u + \hat{u}) -\Pi(u) = \frac{1}{2}\, a(\hat{u},\hat{u}) = \frac{1}{2}\, \int_0^{\,l} \frac{\hat{N}^2(x)}{EA}\,dx > 0
\end{align}
und dieser Ausdruck ist immer gr\"{o}{\ss}er null, solange $\hat{N} \neq 0$. In jedem Nachbarpunkt $u + \hat{u} $ ist also die potentielle Energie gr\"{o}{\ss}er als im Punkt $u$, d.h. $\Pi(u)$ muss wirklich der tiefste Punkt sein.

Man beachte, dass wir auf diese Art und Weise das Problem umgangen haben, die potentielle Energie differenzieren zu m\"{u}ssen, wie das bei normalen Minimax-Aufgaben usus ist.

Dasselbe, was wir bei der Feder \"{u}ber das Minimum und Maximum der potentiellen Energie gesagt haben, gilt auch hier. In der tiefsten Lage, in der Gleichgewichtslage $u(x)$, ist wegen
\begin{align}
G(u,u) = \int_0^{\,l} p\,u\,dx - \int_0^{\,l} \frac{N^2}{EA}\,dx = 0
\end{align}
die potentielle Energie negativ
\begin{align}
\Pi(u) = \frac{1}{2}\, \int_0^{\,l} \frac{N^2}{EA}\,dx - \int_0^{\,l} p\,u\,dx = \frac{1}{2}\,\int_0^{\,l} p\,u\,dx - \int_0^{\,l} p\,u\,dx = - \frac{1}{2}\,\int_0^{\,l} p\,u\,dx
\end{align}
denn das letzte Integral ist die Eigenarbeit der Kr\"{a}fte $p$ auf ihren Wegen $u$ und diese Eigenarbeit ist, wenn wir nicht eine ganz exotische Mechanik betreiben, immer positiv.

Wenn wir dagegen dem Stab einer Verformung aufzwingen, etwa das rechte Ende um eine Strecke $\bar{u} $ ziehen, dann m\"{u}ssen wir eine L\"{a}ngsverschiebung $ u(x) $ finden, die den Gleichungen
\begin{align}
- EA\,u''(x) = 0 \qquad u(0) = 0 \qquad u(l) = \bar{u}
\end{align}
gen\"{u}gt.

In der potentiellen Energie kommt nun keine \"{a}u{\ss}ere Kraft vor, und daher reduziert sich die potentielle Energie auf den positiven Ausdruck
\begin{align}
\Pi(u) = \frac{1}{2}\, \int_0^{\,l} \frac{N^2}{EA}\,dx
\end{align}
Wenn wir jetzt die potentielle Energie zum Minimum machen wollen, dann bedeutet das wirklich, dass wir die Funktion $ u(x) $ finden m\"{u}ssen, die $\Pi(u) $ m\"{o}glichst nahe an Null r\"{u}ckt. Der Stab will also mit m\"{o}glichst wenig Anstrengung der Bewegung $\bar{u} $ nachgeben.

\subsubsection{Der Formalismus}
Der mathematische Hintergrund ist der folgende: Alle Funktionen $u(x)$, die an dem Wettbewerb teilnehmen, m\"{u}ssen an den Enden die Werte $u(0) = 0$ und $u(l) = \bar{u}$ haben. Wir nennen diesen Raum den Raum $ V $. Ihn kann man sich so entstanden denken, dass man zu einer fest gew\"{a}hlten Funktionen $v(x)$ mit den Eigenschaften $v(0) = 0$ und $v(l) = \bar{u}$ lauter Funktionen $u$ mit der Eigenschaft $u(0) = 0$ und $u(l) = 0$ hinzu addiert.

Beim Nachweis
\begin{align}
\Pi(u + \hat{u}) - \Pi(u) =  a(u,\hat{u}) + \frac{1}{2}\, a(\hat{u},\hat{u}) > 0
\end{align}
ist $\hat{u}$ dann eine solche Testfunktion, $\hat{u}(0) = \hat{u}(l) = 0$. Wegen
\begin{align}
G(u,\hat{u}) = \int_0^{\,l} 0\cdot \hat{u}\,dx + N(l)\cdot 0 - N(0)\cdot 0 - a(u,\hat{u}) = - a(u,\hat{u}) = 0
\end{align}
reduziert sich das aber auf
\begin{align}
\Pi(u + \hat{u}) - \Pi(u) =  \frac{1}{2}\, a(\hat{u},\hat{u}) > 0
\end{align}
also einen positiven Ausdruck und damit ist alles gezeigt, dass $u$ die potentielle Energie in der Tat zum Minimum macht.\\

%%%%%%%%%%%%%%%%%%%%%%%%%%%%%%%%%%%%%%%%%%%%%%%%%%%%%%%%%%%%%%%%%%%%%%%%%%%%%%%%%%%%%%%%%%%%%%%%%%%
\section{Der Balken}
Nachdem wir also jetzt zuerst am Beispiel einer Feder und eines Stabes die Energie-und Arbeitsprinzipe formuliert haben, wollen wir nun dasselbe auch f\"{u}r Balken, dem vielleicht wichtigsten statischen Element, tun.

Das sch\"{o}ne an den Arbeitsprinzipien ist, dass die Vorgehensweise immer dieselbe ist und auch die mathematischen Strukturen immer dieselben sind. Wir beginnen mit einer Differentialgleichung, die den Zusammenhang zwischen der \"{a}u{\ss}eren Kraft und der Verformung des Bauteils beschreibt und leiten die erste Greensche Identit\"{a}t f\"{u}r diese Differentialgleichung her. All die dann folgenden Schritte sind im Grunde identisch mit den Schritten, die wir bei der Feder bzw. bei dem Stab gemacht haben. Und es wird sich zeigen, dass auch bei Fl\"{a}chentragwerken die Logik immer dieselbe ist.\\

Wie ist das, wenn federende Lager vorhanden sind? Dann sind die $\delta w $ (in der Regel) in den Lagern nicht null und ein solches Lager im Punk $x_L$ mit der Federsteifigkeit $c $ liefert somit einen Beitrag
\begin{align} \label{Eq22}
\underbrace{c\,w(x_L)}_{Kraft}\,\underbrace{\delta w(x_L)}_{Weg}
\end{align}
Ist das nun virtuelle innere Energie oder virtuelle \"{a}u{\ss}ere Arbeit? Es ist
virtuelle innere Energie, wird also auf der Seite von $\delta A_i $ verbucht. Die Energie ist positiv, wenn die Zusammendr\"{u}ckung des Lagers, $w(x_L) $, dieselbe Richtung hat, wie die virtuelle Verr\"{u}ckung $\delta w(x_L) $.

Man kann aber (\ref{Eq22}) auch mit der Lagerkraft $F$ schreiben, also mit der  Kraft, mit der der Boden die Feder st\"{u}tzt
\begin{align} \label{Eq22}
-F\,\delta w(x_L) \,.
\end{align}
Das Minus ist der Tatsache geschuldet, dass $F $ und $w(x_L) $ immer entgegengesetzte Richtungen haben.

Im Falle des federnd gelagerten Tr\"{a}gers, s. Bild \ref{S8}, gibt es also zwei Schreibweisen f\"{u}r $\delta A_a = \delta A_i$
\begin{align}
\int_0^{\,l} p\,\delta w\,dx = \int_0^{\,l} \frac{M\,\delta M}{EI}\,dx + c\,w(l)\,\delta w(l)
\end{align}
oder
\begin{align}
\int_0^{\,l} p\,\delta w\,dx = \int_0^{\,l} \frac{M\,\delta M}{EI}\,dx - F\,\delta w(l)
\end{align}
Man gewinnt am besten Klarheit \"{u}ber das Problem, wenn man die Steifigkeitsmatrix der Feder zur Hilfe nimmt
\begin{align}
\vek  K\,\vek u = \vek f\,.
\end{align}
am Boden ist $u_2 = 0 $ und so reduziert sich alles auf
\begin{align}
k\,u_1 = f_1 \qquad -k\,u_1 = f_2\,.
\end{align}
\\

Das ist die Mohrsche Arbeitsgleichung. In der Literatur wird zum Beweis dieser Formel an das {\em Prinzip der virtuellen Kr\"{a}fte\/} appelliert, das Ergebnis durch Anwendung eines mechanischen Prinzips hergeleitet.

Das {\em Prinzip der virtuellen Kr\"{a}fte\/} besagt das folgende: wenn $w$ die Biegelinie eines Balkens ist, (zu vorgegebener \"{a}u{\ss}erer Belastung), das unter der Einwirkung von \"{a}u{\ss}eren Kr\"{a}ften im Gleichgewicht ist


 kann man Durchbiegungen berechnen. Auch diese Gleichung basiert auf der ersten Greenschen Identit\"{a}t und zwar auf dem {\em Prinzip der virtuellen Kr\"{a}fte\/}
\begin{align}
G(\delta w^*,w) = \int_0^{\,l} EI\,\delta w^{*IV}\,w\,dx + [\delta V^*\,w - M*\,\delta w'] - \int_0^{\,l} \frac{M\,M*}{EI} dx = 0
\end{align}
Das erste Argument ist also $\delta u*$ und erst das zweite Argument ist die Biegelinie des Tr\"{a}gers.

Anschaulich geschieht das Folgende: Man belastet eine Kopie des Tr\"{a}gers mit einer Einzelkraft $\bar{P} = \bar{1}$ (in Richtung der gesuchten Verschiebung). Die Arbeit, die diese 'virtuelle' Kraft auf dem Weg $w(x)$ am Ort von $\bar{P}$ leistet, ist gleich der virtuellen inneren Energie zwischen $w*$, der Biegelinie zu $\bar{P} = \bar{1}$, und der Biegelinie $w$ des Tr\"{a}gers.\\

So weit die verbale Beschreibung, aber mit Begriffen l\"{a}sst sich nichts beweisen, dazu muss man Mathematik machen.
\\

\begin{align}
A\,\delta w(x_a) &=  [V\, \delta w - \ldots]_{x_a}^{x_b} \\
B\,\delta w(x_b) &=  [\ldots + V\,\delta w]_{x_a}^{x_b} + [V\,\delta w + \ldots]_{x_b}^{x_c}\\
P\,\delta w(x_P) &=  [\ldots + V\,\delta w]_{x_b}^{x_P} + [V\,\delta w + \ldots]_{x_P}^{x_c}\\
C\,\delta w(x_c) &=  [\ldots + V\,\delta w]_{x_b}^{x_c}
\end{align}
\\

Weil nun f\"{u}r kleine Winkel der Tangens und der Winkel nahezu gleich gro{\ss} sind
\begin{align}
\tan\,\Np \cong \Np
\end{align}
wird $\Np = \tan\,\Np$ gesetzt, was dann zu dem merkw\"{u}rdigen f\"{u}hrt, dass der eigentlich dimensionslose Tangens $w'$ pl\"{o}tzlich die Dimension Rad hat. Kaum ein Autor, der es wagt $w'$ ohne Dimension zu schreiben, wie es richtig w\"{a}re.
Denn es ist nicht der Drehwinkel $\Np $, der zu dem Biegemoment konjugiert ist, sondern der Tangens dieses Winkels
\begin{align}
w'(x) = \tan\,\Np(x)\,.
\end{align}
In der ersten Greensche Identit\"{a}t steht $M \cdot w'$ und nicht $M \cdot \Np$.
Autoren trennen hier nicht sauber. Besonders bei der Berechnung von Einflusslinien kommen so ganz merkw\"{u}rdige und h\"{o}chst irritierende Effekte zustande. Wir werden dar\"{u}ber sp\"{a}ter noch mehr zu sagen haben.



\colorbox{hellgrau}{Box mit hellgrauem Hintergrund}

\colorbox{hellgrau}{\parbox{0.8\textwidth}{Im Prinzip macht der Tragwerkskplaner nichts anders, nur vereinfacht er das Anschreiben der Identit\"{a}ten ganz wesentlich. Zun\"{a}chst gibt es bei ihm nur ein $u$ und ein $w$, denn man wei{\ss} ja automatisch welches $u_i $ oder $w_i $ gemeint ist, wenn man auf den oder den Stiel bzw. Riegel zeigt. Dasselbe gilt f\"{u}r die virtuellen Verr\"{u}ckungen $\delta u$ und $\delta w$. Dann hat der Tragwerksplaner einen intuitiven Begriff von dem, was \"{a}u{\ss}ere Arbeit ist, und so kann er die Summe der virtuellen \"{a}u{\ss}eren Arbeiten anschreiben, ohne viel Mathematik betreiben zu m\"{u}ssen}}\\
\\

%%%%%%%%%%%%%%%%%%%%%%%%%%%%%%%%%%%%%%%%%%%%%%%%%%%%%%%%%%%%%%%%%%%%%%%%%%%%%%%%%%%%%%%%%%%%%%%%%%%
\section{Rahmen (praktisch)}
. Zu jeder L\"{a}ngsverschiebung $u_i $ und Durchbiegung $w_i $ eines Stiels oder Riegels geh\"{o}rt ja eine eigene Identit\"{a}t und erst ihre Summe repr\"{a}sentiert dann die ganze Statik des Rahmens.\\

Im Prinzip macht der Tragwerkskplaner nichts anders, nur vereinfacht er das Anschreiben der Identit\"{a}ten ganz wesentlich. Zun\"{a}chst gibt es bei ihm nur ein $u$ und ein $w$, denn man wei{\ss} ja automatisch welches $u_i $ oder $w_i $ gemeint ist, wenn man auf den oder den Stiel bzw. Riegel zeigt. Dasselbe gilt f\"{u}r die virtuellen Verr\"{u}ckungen $\delta u$ und $\delta w$. Dann hat der Tragwerksplaner einen intuitiven Begriff von dem, was \"{a}u{\ss}ere Arbeit ist, und so kann er die Summe der virtuellen \"{a}u{\ss}eren Arbeiten anschreiben, ohne viel Mathematik betreiben zu m\"{u}ssen
\begin{align}
\delta A_a = P\,\delta w(x_P) + \int p\,\delta w\,dx + \ldots
\end{align}
Man sieht ja auf dem Plan, wo die Streckenlast $p$ wirkt, und deswegen kann man die Integrationsgrenzen weglassen, und so wird das Gesicht der \"{a}u{\ss}eren Arbeit $\delta A_a $ einfach durch die Verteilung der Lasten auf dem Rahmen diktiert.

Genauso geht der Tragwerkskplaner mit der virtuellen inneren Energie vor. Weil es nur noch ein $u$ bzw. $w$ gibt, gibt es auch nur noch eine Normalkraft $N$ und ein Moment $M$ in allen Stielen und Riegeln, so dass sich die virtuelle innere Energie zu
\begin{align}
\delta A_i = \int \frac{N\,\delta N}{EA}\,dx + \int \frac{M\,\delta M}{EI}\,dx
\end{align}
ergibt. Die fehlenden Integrationsgrenzen weisen darauf hin, dass \"{u}ber alle Stiele und Riegel zu integrieren ist. Feldweise sind dabei die lokalen Steifigkeiten, $EA $ und $EI $ einzusetzen.\\

Dies gilt auch dann, wenn federnde Lager vorhanden sind. Eine Feder ist ja ein zus\"{a}tzlicher Stiel (\"{a}hnlich einem Pendelstab) und der Punkt, in dem die Feder den Balken st\"{u}tzt, ist ein innenliegender Knoten. Das eigentliche Lager liegt tiefer, es ist der Punkt, in dem sich die Feder auf den Baugrund abst\"{u}tzt.

Das Tragwerk besteht jetzt aus Balken und Feder und dementsprechend muss man die virtuelle innere Energie beider Bauteile mitnehmen
\begin{align}
G(w, \delta w) + G(u, \delta u) = \int_0^{\,l} p(x)\,w(x)\,dx  - \int_0^{\,l} \frac{M\,\delta M}{EI}\,dx - \delta u\,k\,u = 0\,.
\end{align}
Wir haben hier die Randarbeiten gleich weggelassen, weil sie sich zu Null ergeben und ausgenutzt, dass man die virtuelle innere Energie der Feder als $\delta u\,k\,u $ schreiben kann,
wenn $u $ und $\delta u $ die Bewegungen am Kopf der Feder sind.

Auf dieses Ergebnis kommt man wie folgt: Die Federsteifigkeit eines Stabes ist $k = EA/l$ und daher ist umgekehrt das $EA $, das zu einer Feder geh\"{o}rt, $EA = k\,l$. Die Normalkraft in der Feder ist
$N\,= k\,u $, und es ist $\delta N\,= k\,\delta u$, so dass
\begin{align}
\int_0^{\,l} \frac{N\,\delta N}{EA}\,dx = \int_0^{\,l} \frac{k\,u \cdot k\,\delta u}{k\,l} \,dx = \delta u\,k\,u\,.
\end{align}
Man kann dieses Ergebnis nat\"{u}rlich auch aus der Steifigkeitsmatrix der Feder ableiten.
\\

\begin{framed}
  \begin{align*}
    \mbox{"`}1\mbox{"'} \cdot \delta_{ik}
    = &
    \underbrace{\int\frac{M_{i}\,M_{k}}{EI}\;}_{\text{Biegemomente}}
    + \underbrace{\int\frac{N_{i}\,N_{k}}{EA}\;x}_{\text{Normalkr\"{a}fte}}
    \\[1em]
    &
    + \underbrace{\sum_m \frac{F_{mi}F_{mk}}{c_{m}}}_{\text{Normalkraftfedern}}
    + \underbrace{\sum_m \frac{M_{mi}M_{mk}}{c_{\varphi m}}}_{\text{Biegemomentenfedern}}
    \\[2em]
    &
    + \underbrace{\int N_i\,\alpha_T\,T\;}_{\text{Gleichm. Erw\"{a}rmung}}
    + \underbrace{\int M_i\,\alpha_T\,\frac{\Delta T}{h}\;}_{\text{Ungleichm. Erw\"{a}rmung}}
    - \underbrace{\sum_n F_{ni}\,\delta^0_{n}}_{\text{Lagerverschiebungen}}
    - \underbrace{\sum_n M_{ni}\,\varphi^0_{n}}_{\text{Lagerverdrehungen}}
  \end{align*}
\end{framed}


\colorbox{hellgrau}{\parbox{0.8\textwidth}{Die Arbeitss\"{a}tze der Statik beruhen auf diesen Identit\"{a}ten.}}\\


\colorbox{hellgrau}{\parbox{0.8\textwidth}{Die Arbeitss\"{a}tze der Statik beruhen auf diesen Identit\"{a}ten.}}\\

Die erste Greenschen Identit\"{a}t erlaubt .
\begin{align}
G(w, \delta w) = 0 \qquad G(w,w) = 0 \qquad G(\delta w^*, w) = 0\,.
\end{align}
Diese entsprechen dem Prinzip der virtuellen Verr\"{u}ckungen, dem Energieerhaltungssatz und dem {\em Prinzip der virtuellen Kr\"{a}fte\/}
\begin{align}
\delta A_a - \delta A_i = 0 \qquad A_a - A_i = 0 \qquad \delta A_a^* - \delta A_i^* = 0\,.
\end{align}\\

So gilt f\"{u}r die Lagerkr\"{a}fte des Tr\"{a}gers in Bild \ref{U204},
\begin{align}
A - V_l = 0\, \quad  - V_l + B + V_r = 0 \quad - V_l + C = 0
\end{align}
und f\"{u}r die Spr\"{u}nge links und rechts von $P$ und $M$
\begin{align}
\uparrow\,- V_l + \downarrow\,P + \downarrow\,V_r = 0\qquad\qquad  \curvearrowright\, M_l + \curvearrowright\,M - \curvearrowleft\,M_r = 0\,,
\end{align}
so dass bei der Addition der Randarbeiten an den Intervallgrenzen die folgenden Arbeiten \"{u}brig bleiben
\begin{align}\label{Eq21}
A\,\delta w(x_1) + B\,\delta w(x_3) + C\,\delta w(x_6) + P\,\delta w(x_2) + M\,\delta w'(x_3)\,.
\end{align}
Diese Summe plus der virtuellen Arbeit der Streckenlast ist die gesamte virtuelle \"{a}u{\ss}ere Arbeit
\begin{align}
\delta A_a = \text{(\ref{Eq21})} + \int_{x_5}^{\,x_6} p\,\delta w\,dx\,.
\end{align}




Zum Abschluss nun noch die Anwendung des Prinzips der virtuellen Kr\"{a}fte auf diesen Tr\"{a}ger.

In Bild \ref{S4} d wirkt im Viertelspunkt $x_0$ eine Einzelkraft $\bar{P} = 1$ und erzeugt die Biegelinie $\delta w^*$. Die Biegelinie $w(x)$ des urspr\"{u}nglichen Lastfalls, Bild \ref{S4} a, ist eine zul\"{a}ssige virtuelle Verr\"{u}ckung f\"{u}r den Lastfall $\bar{P} = 1$, und so ergibt die Anwendung der ersten Greenschen Identit\"{a}t das Resultat
\begin{align}
G(\delta w^*,w) = \underbrace{\phantom{\int_0^{\,l} }\bar{P} \cdot w(x_0)}_{\delta A_a^*} - \underbrace{\int_0^{\,l} \frac{\delta M^*\,M}{EI}\,dx}_{\delta A_i^*} = 0
\end{align}
oder
\begin{align}
w(x_0) = \int_0^{\,l} \frac{\delta M^*\,M}{EI}\,dx\,,
\end{align}
was der Ingenieur als eine Anwendung des Prinzips der virtuellen Kr\"{a}fte interpretiert.\\

%----------------------------------------------------------------------------------------------------------
\begin{figure}[tbp]
\centering
\if \bild 2 \sidecaption \fi
\includegraphics[width=0.9\textwidth]{\Fpath/S8}
\caption{Federndes Lager} \label{S8}
%
\end{figure}%
%----------------------------------------------------------------------------------------------------------\\


Im Grunde empfiehlt es sich sowieso, vor der Formulierung der Identit\"{a}ten in Gedanken alle Lager zu entfernen, denn dann kann man den Balken beliebige virtuelle Verr\"{u}ckungen $\delta w$ erteilen ohne  fragen zu m\"{u}ssen, ob denn die virtuelle Verr\"{u}ckung zul\"{a}ssig sei. Das Entfernen der Lager hat auch noch den positiven Effekt, dass die Lagerkr\"{a}fte als \"{a}u{\ss}ere Kr\"{a}fte, sie halten den Balken ja in  der Schwebe, sichtbar werden.
\\

Die wesentlichen Informationen bei der Formulierung der ersten Greenschen Identit\"{a}t kommen aus den Randarbeiten. In der einen Variante steht dort
\begin{align}
G(w, \delta w) = \ldots + [V\,\delta w - M\,\delta w']_{@0}^{@l} - \ldots
\end{align}
und in der anderen Variante
\begin{align}
G(\delta w, w) = \ldots + [\delta V\, w - \delta M\, w']_{@0}^{@l} - \ldots
\end{align}
In der ersten Variante kann man durch geeignete Wahl von $\delta w$ bzw. $\delta w'$ Informationen \"{u}ber die Randwerte der Querkraft bzw. des Moments gewinnen, typischerweise die Lagerkr\"{a}fte, $\delta w = 1$ bzw. $\delta w' = 1$ im Lager.

In der zweiten Variante kann man durch geeignete Wahl von $\delta V $ bzw. $\delta M $ Informationen \"{u}ber die Weggr\"{o}{\ss}en an den Intervallgrenzen bekommen. Die Verl\"{a}ufe $\delta V$ und $\delta M$ geh\"{o}ren zu Einzelkr\"{a}ften bzw. Einzelmomenten mit denen man die Durchbiegungen oder die Verdrehung in einem bestimmten Punkt berechnen will. Weil in einem solchen Fall die zugeh\"{o}rige Biegelinie nicht mehr in $C^4 $ liegt, muss man den Tr\"{a}ger in zwei Teile teilen und so entstehen die Intervallgrenzen.


%%%%%%%%%%%%%%%%%%%%%%%%%%%%%%%%%%%%%%%%%%%%%%%%%%%%%%%%%%%%%%%%%%%%%%%%%%%%%%%%%%%%%%%%%%%%%%%%%%%
\subsection{Der Umgang mit den Identit\"{a}ten in der Praxis}

%%%%%%%%%%%%%%%%%%%%%%%%%%%%%%%%%%%%%%%%%%%%%%%%%%%%%%%%%%%%%%%%%%%%%%%%%%%%%%%%%%%%%%%%%%%%%%%%%%
\subsection{Rahmen}

Die Greenschen Identit\"{a}ten bilden in der Tat den Hintergrund eines gro{\ss}en Teils des Rechnens in der Statik, aber es w\"{a}re viel zu m\"{u}hsam die Identit\"{a}ten abschnittsweise f\"{u}r  $u$ und $w$, zu formulieren, eventuelle Sprungterme, die aus Einzelkr\"{a}ften oder Einzelmomenten resultieren, an den Intervallgrenzen zu ber\"{u}cksichtigen und das Ganze zu einem geschlossenen Ausdruck zu summieren.
%----------------------------------------------------------------------------------------------------------
\begin{figure}[tbp]
\centering
\if \bild 2 \sidecaption \fi
\centering
\includegraphics[width=0.9\textwidth]{\Fpath/S1}
\caption{Abgewinkelter Tr\"{a}ger} \label{S1}
\end{figure}%
%----------------------------------------------------------------------------------------------------------

Ingenieure haben sich daf\"{u}r das Prinzip der virtuellen Verr\"{u}ckung in und das {\em Prinzip der virtuellen Kr\"{a}fte\/} so zurechtgelegt, dass man all diese Schwierigkeiten vermeidet.

Der Rahmen in Bild \ref{S1} besteht aus zwei Balken und jeder Balken kann sich in seiner Achsrichtung $(u)$ und quer dazu $(w)$ verformen, so dass wir es mit vier Funktionen $u_1, w_1, u_2, w_2$ zu tun haben, die die Verformungen des Rahmens beschreiben und die innere Energie enth\"{a}lt Beitr\"{a}ge von allen vier Funktionen
\begin{align}
A_i = \frac{1}{2}\,\int_0^{\,l_1} \frac{N_1^2}{EA}\,dx + \frac{1}{2}\,\int_0^{\,l_1} \frac{M_1^2}{EI}\,dx + \frac{1}{2}\,\int_0^{\,l_2} \frac{N_2^2}{EA}\,dx + \frac{1}{2}\,\int_0^{\,l_2} \frac{M_2^2}{EI}\,dx
\end{align}
In der Statik schreiben wir aber viel k\"{u}rzer
\begin{align}
A_i = \frac{1}{2}\,\int_0^{\,l} \frac{N^2}{EA}\,dx + \frac{1}{2}\,\int_0^{\,l} \frac{M^2}{EI}\,dx
\end{align}
Genauso ist es mit der \"{a}u{\ss}eren Arbeit
\begin{align}
A_a  = \frac{1}{2}\,P_1\,u_1(x_1) + \frac{1}{2}\,\int_0^{\,x_2} p\,w_2\,dx + \frac{1}{2}\,P_2\,w_2(x_2)
\end{align}
wo die Unterscheidung zwischen $u_1$ und $u_2$ bzw. $w_1$ und $w_2$ weggelassen wird
\begin{align}
A_a = \frac{1}{2}\,P_1\,u(x_1) + \frac{1}{2}\,\int_0^{\,l} p\,w\,dx + \frac{1}{2}\,P_2\,w(x_2)
\end{align}
und die genauen Integrationsgrenzen, wie \"{u}blich, dem Bild entnommen werden m\"{u}ssen.
\\

Das Prinzip vom Minimum der potentiellen Energie fasst nun diese Beobachtungen wie folgt zusammen: Die Auslenkung $ u $ der Feder unter der Wirkung der Kraft $ f $ macht die potentielle Energie der Feder zum Minimum. Wenn man also nur lange genug Zufallszahlen $ u $ in die Funktion $\Pi(u)$ einsetzen w\"{u}rde, sich eine Liste der Wert $\Pi(u)$ machen w\"{u}rde, dann w\"{u}rde man automatisch zu der gesuchten Gleichgewichtslage $u$ der Feder gef\"{u}hrt.\\



Der Stab in Bild 1 a tr\"{a}gt eine Streckenlast und der Stab in Bild 1 b eine Einzelkraft. Der Satz von Bett besagt nun, dass Arbeiten, die Streckenlasten im Bild 1H auf den Wegen der Verformungen des Systems in Bild 1B leisten, genauso gro{\ss} ist, wie die Arbeit, die die Einzelkraft in Bild 1B auf den Wegen des Stabes in Bild 1A leisten
\begin{align}
A_{1,2} = \int_0^{\,l} p(x)\,u_2(x)\,dx = P \,u_1(l) = A_{2,1}\,.
\end{align}
Der {\em Satz von Betti\/} beruht auf der zweiten Greenschen Identit\"{a}t, die man einfach durch Spiegelung der ersten Identit\"{a}t erh\"{a}lt
\begin{align}
\text{\normalfont\calligra B\,\,}(u,\hat{u}) = \text{\normalfont\calligra G\,\,}(u,\hat{u}) - \text{\normalfont\calligra G\,\,}(\hat{u}.u) = 0 - 0 = 0
\end{align}
oder
\begin{align}
B(u,\hat{u}) &= \int_0^{\,l} - EA\,u''(x)\,\hat{u}(x)\,dx + [N\,\hat{u}]_{@0}^{@l} \nn \\
&- [u\,\hat{N}]_{@0}^{@l} - \int_0^{\,l} u(x)\,(- EA\,\hat{u}''(x))\,dx = 0
\end{align}
In der zweiten Zeile wiederholt sich also die erste Zeile, nur dass eben die Pl\"{a}tze von $ u $ und $ \hat{u} $ vertauscht sind. Diese Identit\"{a}t ist g\"{u}ltig f\"{u}r alle Paare von Funktionen $u$ und $\hat{u}$, die im Integrationsintervall in $C^2$ liegen.

Wir formulieren und diesen {\em Satz von Betti\/} mit den beiden Verschiebungen $ u_1 $ und $ u_2 $. Die linke Verschiebung (Streckenlast) gen\"{u}gt den Gleichungen
\begin{align}
- EA\,u_1''(x) = p(x) \qquad u_1(l) = 0 \qquad N_1(l) = 0
\end{align}
und die rechte Verschiebung (Einzelkraft) den Gleichungen
\begin{align}
- EA\,u_2''(x) = 0 \qquad u_2(0) = 0 \qquad N_2(l) = P\,.
\end{align}
Beide L\"{o}sungen liegen in $C^2$ und somit ist die zweite Greensche Identit\"{a}t anwendbar
\begin{align}
B(u_1,u_2) = \int_0^{\,l} p(x)\,u_2(x)\,dx - P\,u_1(l) = 0
\end{align}
oder umgestellt
\begin{align}
A_{1,2} = \int_0^{\,l} p(x)\,u_2(x)\,dx = P\,u_1(l) = A_{2,1}\,,
\end{align}
was genau der Inhalt des Satzes von Betti ist.
\\

Wie sieht es in einem solchen Falle mit dem Prinzip der virtuellen Verr\"{u}ckungen bzw. dem {\em Prinzip der virtuellen Kr\"{a}fte\/} aus?\\

Dass zul\"{a}ssige virtuelle Verr\"{u}ckungen im rechten Lager nicht null sein m\"{u}ssen, kann man wie folgt verstehen: der Lastfall Lagersenkung ist identisch mit einem Lastfall, bei dem am rechten Ende eine Einzelkraft $B$ den Tr\"{a}ger, jetzt als Kragtr\"{a}ger gedacht, nach unten zieht, und die zul\"{a}ssigen virtuellen Verr\"{u}ckungen an dem Kragtr\"{a}ger m\"{u}ssen nur die Lagerbedingungen $\delta w(0) = \delta w'(0) = 0$ erf\"{u}llen.

Wie ist das aber nun, wenn die (zul\"{a}ssige) virtuelle Verr\"{u}ckung im rechten Lager zuf\"{a}llig gerade null ist, was dann also
\begin{align} \label{Eq31}
G(w, \delta w) &= - \int_0^{\,l} \frac{M \delta M}{EI}\,dx = -\delta A_i = 0
\end{align}
zur Folge hat. Um dieses Ergebnis zu verstehen, argumentieren wir wie folgt: Ent\-weder ist $ \delta w'(l) = 0$ oder $ \delta w'(l) \neq 0$. Im ersten Fall kann $ \delta w(x)$ nur zwischen den Lagern von Null verschieden sein, da ja $\delta w$ in den Lagern 'stumm' ist. Da aber keine Streckenlast vorhanden ist, die auf dem Weg $\delta w(x)$ eine Arbeit leisten k\"{o}nnte, ist $\delta A_a = 0$. Im zweiten Falle k\"{o}nnte das rechte Balkenendmoment $M(l)$ auf der Drehung $\delta w'(l)$ eine Arbeit leisten, aber da das Moment null ist, ist auch diese Arbeit null und somit gilt auch in diesem Fall $\delta A_a = 0$ und (\ref{Eq31}) ist best\"{a}tigt.
\\
Das ist das Grundschema bei der Anwendung von Einflussfunktionen. Wir kennen die Belastung $p = EI\,w^{IV}$ auf einen Tr\"{a}ger, kennen also das Bild von $w$ unter der Abbildung $EI\,w^{IV}$ und wollen nun wissen, wie gro{\ss} die Durchbiegung $w$ an einer gewissen Stelle $x$ ist.\\

Es sind ja wieder die Randarbeiten, die die interessierenden Werte $M(x) $ und $V(x) $ oder $N(x) $ liefern und jetzt muss also das, was konjugiert ist zu den Kraftgr\"{o}{\ss}en, an den beiden Intervallgrenzen, links und rechts vom Aufpunkt springen, damit sich das gew\"{u}nschte Resultat ergibt
\begin{align}
M(x) \cdot (w'(x_{-}) - w'(x_+)) = M(x) \cdot 1\\
V(x) \cdot (w(x_{-}) - w(x_+)) = V(x)\cdot 1\\
N(x) \cdot (u(x_{-}) - u(x_+)) = N(x) \cdot 1
\end{align}
Ein Knick, also ein Sprung in der Neigung der Tangente, liefert die Einflussfunktionen f\"{u}r $M(x) $ und ein Sprung in der Biegelinie bzw. der L\"{a}ngsverschiebung liefert die Einflussfunktionen f\"{u}r $V(x) $ bzw. $N(x)$.

Wie das im Detail geht, wollen wir am Beispiel des Tr\"{a}gers in Bild X zeigen. Zun\"{a}chst wird der Tr\"{a}ger dupliziert, also eine getreue Kopie des Tr\"{a}gers erstellt.\\

Und wenn man dabei einen Fehler macht, wenn man ein falsches Signal aussendet, weil vielleicht die Spreizung nicht genau 1 ist, oder die Ausbreitung der Welle auf k\"{u}nstliche Hindernisse st\"{o}{\ss}t (falsch angesetzte Steifigkeiten), dann ist auch das Signal, das den Fu{\ss}punkt der Last erreicht falsch und dann ist auch die Schnittkraft falsch.



Der erste Schritt bei der Berechnung von Einflussfunktionen f\"{u}r Schnittkr\"{a}fte wie auch Lagerkr\"{a}ften ist, dass man die Kraft sichtbar macht, also zu einer \"{a}u{\ss}eren Kraft macht.
Dies erreicht man durch den Einbau eines entsprechenden Gelenks, das den Kraftfluss unterbricht und dazu zwingt mit einem \"{a}u{\ss}eren Kr\"{a}ftepaar das Gleichgewicht wieder herzustellen.

Diesem so modifizierten System erteilt man dann eine virtuelle Verr\"{u}ckung derart, dass die beiden Kr\"{a}fte links und rechts vom Gelenk den Weg $(-1)$ zur\"{u}ck legen.
Das f\"{u}hrt z.B. bei der Berechnung einer Einflussfunktion f\"{u}r eine Querkraft zu einem Beitrag
\begin{align}
V(x) \cdot 1
\end{align}
in der Bilanz $A_{1,2} = A_{2,1}$ oder, wenn man den Term auf eine Seite alleine bringt, auf die Einflussfunktion
\begin{align}
V(x) \cdot 1 = \ldots\,.
\end{align}\\


In der Praxis werden Einflussfunktionen nat\"{u}rlich nicht mehr mit dem Kraftgr\"{o}{\ss}enverfahren berechnet, sondern es werden FE-Programme benutzt und in diese sind die Routinen zur Berechnung der Einflussfunktionen fertig eingebaut. Welche Technik dabei zum Einsatz kommt, diskutieren wir in Abschnitt X.\\


Der Symmetrie der Wechselwirkungsenergie
\beq
 a(u,\hat{u})  = a(\hat{u},u)
\eeq
entspricht die Symmetrie der Steifigkeitsmatrix $\vek K$ und daher folgt, dass wenn $u$ und $G$ die Variationsl\"{o}sungen des {\em primalen\/} und des {\em dualen\/} Problems sind
\begin{align}
u \in \mathcal{V} \qquad a(u,v) &= (p,v) \qquad \forall\,v \in \mathcal{V} \qquad \mbox{{\em primal problem\/}} \\
G \in \mathcal{V} \qquad a(G,v) &= (\delta,v) \qquad \forall\,v \in \mathcal{V} \qquad \mbox{{\em dual problem\/}}
\end{align}
dies das Ergebnis
\beq
u(\vek x) = (p,G) = (\delta,u) \,.
\eeq
impliziert. Glg. (\ref{Eq41}) ist dasselbe Ergebnis 'ausgeschrieben'.



In einigen B\"{u}chern wird der Aufpunkt mit dem Buchstaben $\xi$ bezeichnet und man hat so den Buchstaben $x$ als Integrationsvariable frei, aber dann wird die Durchbiegungen eine Funktion von $\xi$, was auch ungewohnt ist
\beq
u(\xi) = \int_0^{\,l} G(x, \xi)\,p(x)\,dx \,.
\eeq
Wir ziehen daher die Kombination $x$ und $y$ vor.\\

%%%%%%%%%%%%%%%%%%%%%%%%%%%%%%%%%%%%%%%%%%%%%%%%%%%%%%%%%%%%%%%%%%%%%%%%%%%%%%%%%%%%%%%%%%%%%%%%%%%
\section{Was finite Elemente nicht sind}
Viele Ingenieure verstehen finite Elemente in etwa so: Man unterteilt eine Scheibe in kleine dreiecksf\"{o}rmige Elemente, die in den Knoten zusammenh\"{a}ngen, und man ersetzt die Belastung durch \"{a}quivalente Knotenkr\"{a}fte $f_i$ und bestimmt die Knotenverformungen $u_i $ so, dass in  allen Knoten Gleichgewicht herrscht
\begin{align}
\vek K\,\vek u = \vek f\,.
\end{align}
Diese Erkl\"{a}rung wird dann noch meist begleitet von einem Bild, wo man das urspr\"{u}ngliche Tragwerk sieht und daneben das FE Modell, s. Bild X.

Das ist sehr eing\"{a}ngig und deswegen ist das Modell wohl sehr popul\"{a}r, aber die Gleichung $\vek K\,\vek u = \vek f$ ist keine Gleichgewichtsbedingung und die Belastung wird auch nicht in die Knoten reduziert. Ebenso h\"{a}ngen die Elemente nicht nur in den Knoten zusammen, und daher tauschen sie \"{u}ber ihren ganzen Umfang ihre Kr\"{a}fte mit den Nachbarelementen aus.

Was die Situation so schwierig macht ist, dass man bei stabartigen Tragwerken durchaus so argumentieren kann. Dann sind die  Stiele und Riegel die einzelnen finiten Elemente und in den Knoten wirken echte Knotenkr\"{a}fte, weil die Methode der finiten Elemente bei Stabtragwerken (in Standardf\"{a}llen) im Grunde mit dem Drehwinkelverfahren identisch ist.

Wenn die Tr\"{a}ger aber ihren Querschnitt ver\"{a}ndern, also zum Beispiel gevoutet sind, dann ist auch das FE-Modell nur eine N\"{a}herung, weil es ja solche 'Feinheiten' in der Regel ignoriert.


\subsubsection{}
\subsubsection{Resum\'{e}}
Die erste und zweite Greensche Identit\"{a}t (Betti) kann man als unterschiedliche Bilanzen von acht Arbeiten lesen
\begin{align}
A_i = A_a \qquad \delta A_i = \delta A_a \qquad \delta A_i^* = \delta A_a^* \qquad A_{1,2} = A_{2,1} \qquad \text{(Betti)}
\end{align}
Damit, dass man eine Differentialgleichung w\"{a}hlt oder setzt, die das Bauteil regiert, legt man die Form der Gleichgewichtsbedingungen und s\"{a}mtliche Arbeits- und Energieprinzipe f\"{u}r das Bauteil fest. Diese folgen durch partielle Integration aus der Differentialgleichung, holen also im Grunde nur das aus der Differentialgleichung heraus, was implizit bei ihrer Herleitung in sie gesteckt wurde.

Das Rechnen der Statik besteht zu einem gro{\ss}en Teil auf der Anwendung der Greenschen Identit\"{a}ten. Um diese Identit\"{a}ten nun dem Ingenieur schmackhafter zu machen, hat man Prinzipe erfunden, das {\em Prinzip der virtuellen Kr\"{a}fte\/}, das Prinzip der virtuellen Verr\"{u}ckungen, den Energieerhaltungssatz usw. Dieses sind im Grunde nur verbale Umschreibungen der Greenschen Identit\"{a}ten, in Worte gefasste Zusammenfassungen.

Vom didaktischen Standpunkt aus ist das voll gelungen. Man w\"{u}sste es nicht besser zu machen, denn ohne diese Prinzipe w\"{u}rde man sich bei der statischen Untersuchung von Stockwerkrahmen mittels Arbeitsprinzipen (Mohr, Betti, etc.) hoffnungslos in der Formulierung der Greenschen Identit\"{a}ten verstricken.
\\

Wenn ein Vektor $\vek u $ das Gleichungssystem $\vek K\vek u = \vek f$ l\"{o}st, dann kann man das System auf beiden Seiten skalar mit einem Vektor $\vek \delta \vek u$ multiplizieren
\begin{align}
\vek K\vek u = \vek f \qquad \Rightarrow \qquad \vek \delta\vek u^T\,\vek K\,\vek u = \vek \delta \vek u^T\,\vek f
\end{align}




Angenommen $g(y,x)$ sei die Durchbiegung eines Balkens in einem festen Punkt $x$, wenn in einem abliegenden Punkt $y$ eine Einzelkraft angreift, dann ist
\begin{align}
w(x) = \int_0^{\,l} g(y,x)\,p(y)\,dy
\end{align}
die Durchbiegung im Punkt $x$, wenn eine Streckenlast $p$ wirkt.

Denn in Gedanken kann man die Streckenlast $p$ in eine Schar von lauter kleinen Einzelkr\"{a}ften $dP = p\,dy$ zerlegen.

Wenn ein Vektor $\vek u$ das Gleichungssystem
\begin{align}
\vek K\,\vek u = \vek f
\end{align}
l\"{o}st, dann ist
\begin{align}
\delta \vek u^T\,\vek K\,\vek u = \vek \delta \vek u^T\,\vek f
\end{align}
f\"{u}r alle Vektoren $\delta \vek u$.

Was wir oben an ganz elementaren Beispielen durchexerziert haben, wiederholt sich in der Folge mit komplexeren Tragwerken. Das Rechnen in der Statik ist zu einem gro{\ss}en Teil eine Anwendung dieser Identit\"{a}ten und zwar genau genommen, der ersten Greenschen Identit\"{a}t.

Immer dann, wenn, wie in der Statik der Kontinua, die Gleichgewichtslage eines Tragwerks von einer oder mehreren (Stiele, Riegel) Funktionen beschrieben wird, bildet die erste Greensche Identit\"{a}t das Kernst\"{u}ck.

Wir werden in der Folge Gelegenheit haben dieses zu vertiefen. Hier an dieser Stelle sei nur noch einmal betont, wie wichtig diese Identit\"{a}ten f\"{u}r das Rechnen in der Statik sind. Es sei dies auch noch einmal zum Anlass genommen, darauf hinzuweisen, dass die Arbeits- und Energieprinzipe der Statik und Mechanik mathematischer Natur sind. Anders gesagt: das Prinzip der virtuellen Verr\"{u}ckungen ist kein Naturgesetz, sondern umgekehrt
\subsection{Skalarprodukt}
Beginnen wir mit zwei Vektoren, einem Verschiebungsvektor $\vek u $ und einem Kraftvektor $\vek f $. Der Vektor $\vek u = \{u_1,u_2,u_3\}^T$ enth\"{a}lt die Durchbiegungen der drei Innenknoten des Seils und der Vektor $\vek f =  \{f_1,f_2,f_3\}^T$ ist die Liste der Knotenkr\"{a}fte.

Die Arbeit, die die Kr\"{a}fte $\vek f $ auf den Wegen $\vek u $ leisten, ist das Skalarprodukt
zwischen den beiden Vektoren
\begin{align}
\vek u \dotprod \vek f = \vek u^T\,\vek f = u_1\,f_1 + u_2\,f_2 + u_3\,f_3 = |\vek u| \,|\vek f| \,\cos\,\Np
\end{align}
F\"{u}r dieses gibt es wie man sieht, zwei verschiedene Schreibweisen, einmal die klassische mit Punkt, $\vek u \dotprod \vek f$, und dasselbe auch noch mal in Matrizenschreibweise
\begin{align}
\vek u^T\,\vek f = [u_1, u_2, u_3] \cdot \left [\barr{c}  f_1 \\  f_2 \\ f_3\earr \right ]
\end{align}
als das Produkt eines Zeilenvektors  mit einem Spaltenvektor.

Der Zusammenhang zwischen dem Verschiebungsvektor $\vek u $ und dem Kraftvektor $\vek f $ wird in der linearen Statik durch eine Steifigkeitsmatrix $\vek K $ beschrieben
\begin{align}
\vek K\,\vek u = \vek f\,.
\end{align}
Im Falle eines Seils, das zwischen zwei Hausw\"{a}nden h\"{a}ngt, s. Bild X, ist das zum Beispiel die Matrix
\beq\label{Eq176}
 \left[\barr{r r r} 2 & - 1 & 0 \\ - 1 & 2 & -1 \\ 0 & -1 & 2 \earr\right]
\,\left[\barr{c} u_1 \\u_2 \\ u_3 \earr \right] = \left[\barr{c} f_1 \\ f_2  \\
f_3  \earr \right]\,.
\eeq
Damit  ist praktisch schon die ganze Statik beschrieben. Das Seil wird mit Knotenkr\"{a}ften $\vek f$  belastet,  und die Knoten geben nach, sie senken sich, Knotenvektor $\vek u $.

Nun wollen wir die Arbeits- und Energieprinzipe des Seils formulieren. Das Transponierte eines Spaltenvektors ist ein Zeilenvektor und umgekehrt. Das Transponierte einer reellen Zahl wie $\pi$ jedoch, ist dieselbe Zahl, $\pi^T = \pi$.

Im folgenden seien $\vek u$ und $\vek \delta \vek u$ zwei beliebige Vektoren.  Im ersten Schritt multiplizieren wir die Steifigkeitsmatrix $\vek K $ mit dem Vektor $\vek u $ und diesen Vektor dann skalar mit dem zweiten Vektor, $\vek \delta \vek u^T\,\vek K\,\vek u$. Das Ergebnis ist also eine Zahl,  wir nennen sie $a$, die man transponieren kann, ohne die Zahl zu \"{a}ndern
\begin{align}
a^T = a
\end{align}
Der folgende Ausdruck ist daher eine Identit\"{a}t
\begin{align}
G(\vek u,\vek \delta \vek u) =  \vek u^T \,\vek K\,\vek \delta \vek u- \vek \delta \vek u^T\vek K\,\vek u = a^T - a = 0\,.
\end{align}
Auf dieser Identit\"{a}t beruhen die Arbeits- und Energieprinzipe des Seils. Um das recht w\"{u}rdigen zu k\"{o}nnen, ben\"{o}tigen wir noch zwei Definitionen: Die innere Energie des Seils ist der Ausdruck
\begin{align}
A_i = \frac{1}{2}\, \vek u^T\,\vek K\,\vek u
\end{align}
und die potentielle Energie des Seils ist die Funktion
\begin{align}
\Pi(\vek u)= \frac{1}{2}\, \vek u^T\,\vek K\,\vek u - \vek f^T\,\vek u
\end{align}
Sie ist eine Funktion der Knotenverschiebungen $u_i$ und ihre Ableitungen nach den $u_i$  bilden den Vektor
\begin{align}
\nabla \Pi(\vek u) = \{\frac{\partial \Pi}{\partial u_1}, \frac{\partial \Pi}{\partial u_2}, \frac{\partial \Pi}{\partial u_3}\}^T
\end{align}
Ihn nennt man den Gradienten von $\Pi(\vek u)$. Man findet leicht, dass
\begin{align}
\nabla \Pi(\vek u) = \vek K\,\vek u - \vek f
\end{align}
Zum Vergleich betrachte man die Funktion
\begin{align}
\Pi(u) = \frac{1}{2}\, k\,u^2 - f\,u
\end{align}
und leite sie nach $u$ ab
\begin{align}
\frac{d\Pi}{du} = k\,u - f
\end{align}
So versteht man, warum der Gradient $\nabla\,\pi(\vek u)$ die obige Gestalt hat.
\\

Oder $w_1(x)$ und $w_2(x)$ seien die Biegelinien eines gelenkig gelagerten Einfeldtr\"{a}gers in zwei verschiedenen Lastf\"{a}llen
\begin{align}
EI\,w_1^{IV}(x) = p_1(x) \qquad  EI\,w_2^{IV}(x) = p_2(x)\,.
\end{align}
Multiplikation und Integration \"{u}ber Kreuz ergibt
\begin{align}
\int_0^{\,l} w_2(x)\,EI\,w_1^{IV}(x) \,dx \qquad \int_0^{\,l} w_1(x)\,EI\,w_2^{IV}(x) \,dx\,.
\end{align}
Um jetzt weiter zu kommen, m\"{u}ssen wir partielle Integration anwenden
\begin{align}
\int_0^{\,l} w_2(x)\,EI\,w_1^{IV}(x) \,dx = \int_0^{\,l} EI\,w_2''\,w_1''\,dx
\end{align}\\

Dem Tragwerkplaner w\"{u}rde es viel zu lange dauern, bei einem dreist\"{o}ckigen Stockwerkrahmen all diese Identit\"{a}ten einzeln anzuschreiben. Daher gibt es bei ihm nur ein $u$ und ein $w$, denn man wei{\ss} ja automatisch welches $u_i $ oder $w_i $ gemeint ist, wenn man auf den oder den Stiel bzw. Riegel zeigt. Dasselbe gilt f\"{u}r die virtuellen Verr\"{u}ckungen $\textcolor{red}{\delta u}$ und $\textcolor{red}{\delta w}$. Und dann hat der Tragwerksplaner einen intuitiven Begriff von dem, was \"{a}u{\ss}ere Arbeit ist, und so kann er die Summe der virtuellen \"{a}u{\ss}eren Arbeiten anschreiben, ohne viel Mathematik betreiben zu m\"{u}ssen
\begin{align}\label{Eq33}
\textcolor{red}{\delta A_a} = P\cdot\textcolor{red}{\delta w(x_P)} + \int p_z(x)\,\textcolor{red}{\delta w(x)}\,dx + \ldots
\end{align}
Man sieht ja auf dem Plan, wo die Streckenlasten $p_z$ und $p_x$ wirken, und deswegen kann man die Integrationsgrenzen weglassen, und so wird das Gesicht der \"{a}u{\ss}eren Arbeit $\textcolor{red}{\delta A_a} $ einfach durch die Verteilung der Lasten auf dem Rahmen diktiert.

Genauso vereinfacht der Tragwerkskplaner die virtuelle innere Energie. Weil es nur noch ein $u$ bzw. $w$ gibt, gibt es auch nur noch eine Normalkraft $N$ und ein Moment $M$ in allen Stielen und Riegeln, so dass sich die virtuelle innere Energie zu
\begin{align}
\textcolor{red}{\delta A_i} = \int \frac{N\,\textcolor{red}{\delta N}}{EA}\,dx + \int \frac{M\,\textcolor{red}{\delta M}}{EI}\,dx
\end{align}
ergibt.\\

Die fehlenden Integrationsgrenzen weisen darauf hin, dass \"{u}ber alle Stiele und Riegel zu integrieren ist. Feldweise sind dabei die lokalen Steifigkeiten, $EA $ und $EI $, einzusetzen.

Diese Vereinfachung macht den Umgang mit den Identit\"{a}ten sehr bequem. In der Statik reduziert man die Beitr\"{a}ge zu den Greenschen Identit\"{a}ten auf die folgenden acht Arbeiten
\begin{align}
A_a\,, A_i \qquad \delta A_i\,, \delta A_a \qquad \delta A_i^*\,, \delta A_a^* \qquad A_{1,2}\,, A_{2,1}\,\,\text{(Betti)}
\end{align}
und es ist dann der Sorgfalt des Ingenieurs \"{u}berlassen, die \"{a}u{\ss}eren und inneren Arbeiten richtig einzusortieren und zu bilanzieren. Es m\"{o}gen noch so viele St\"{a}be und Riegel sein, die einen Rahmen bilden, jedes Bauteil mit seiner eigenen Greenschen Identit\"{a}t, aber am Schluss reduzieren sich alle Beitr\"{a}ge in den Greenschen Identit\"{a}ten auf die obigen Arbeiten.

Dass diese Regel so einfach ist, beruht darauf, dass jede Identit\"{a}t f\"{u}r sich null ist und damit auch die Summe aller Identit\"{a}ten
\begin{align}
0 + 0 + 0 + \ldots + 0 = 0
\end{align}
oder wenn man das ganze, wie ein Buchhalter nach Soll und Haben, nach au{\ss}en und innen, verteilt
\begin{align}
\delta A_a = \delta A_i \qquad A_a = A_i \qquad \delta A_a^* = \delta A_i^*\,.
\end{align}




%----------------------------------------------------------------------------------------------------------
\begin{figure}[tbp]
\centering
\if \bild 2 \sidecaption \fi
\includegraphics[width=1.0\textwidth]{\Fpath/S4}
\caption{Durchlauftr\"{a}ger} \label{S4}
%
\end{figure}%
%----------------------------------------------------------------------------------------------------------\\

%%%%%%%%%%%%%%%%%%%%%%%%%%%%%%%%%%%%%%%%%%%%%%%%%%%%%%%%%%%%%%%%%%%%%%%%%%%%%%%%%%%%%%%%%%%%%%%%%%
{\textcolor{blau2}{\subsection{Beispiel}}}
Bei der Anwendung der Greenschen Identit\"{a}ten auf die Biegelinie des Durchlauftr\"{a}gers in Bild \ref{S4} a muss man, wie oben erl\"{a}utert, an jedem Zwischenlager und im Punkt $x_P$ anhalten, weil man \"{u}ber Lagerkr\"{a}fte und Einzelkr\"{a}fte nicht einfach hinweg integrieren kann. Die gesamte Identit\"{a}t ist also eine Summe von drei Termen
\begin{align}
G(w,\textcolor{red}{\delta w}) &= G(w_1, \textcolor{red}{\delta w})_{(x_a,x_b)} + G(w_2,\textcolor{red}{\delta w})_{(x_b,x_P)} + G(w_3,\textcolor{red}{ \delta w})_{(x_P,x_c)}\nn \\
&= 0 + 0 + 0 = 0\,,
\end{align}
oder
\begin{align}
G(w,\textcolor{red}{\delta w}) &= \int_0^{\,l} p\,\textcolor{red}{\delta w(x)}\,dx + A\cdot\textcolor{red}{\delta w(x_a)} + B\cdot\textcolor{red}{\delta w(x_b)} + C\cdot\textcolor{red}{\delta w(x_c)} + P\cdot\textcolor{red}{\delta w(x_P)}\nn \\
& - \int_0^{\,l} \frac{M\,\textcolor{red}{\delta M}}{EI}\,dx = 0\,.
\end{align}
Die Einzelarbeiten $ A\,\textcolor{red}{\delta w(x_a)}$ etc. kommen dabei aus den eckigen Klammern, sind also die \"{u}brig gebliebenen Randarbeiten an den Balkenenden bzw. die aufaddierten Randarbeiten an den \"{U}bergangsstellen, wie z.B. im Punkt $x_b$
\begin{align}
[\ldots + V\,\textcolor{red}{\delta w}]_{x_a}^{x_b} + [V\,\textcolor{red}{\delta w} + \ldots]_{x_b}^{x_c} = (V(x_b^-) - V(x_b^+))\,\textcolor{red}{\delta w(x_b)} = B\,\textcolor{red}{\delta w(x_b)}\,.
\end{align}
Setzt man f\"{u}r $\textcolor{red}{\delta w(x)}$ die Biegelinie selbst, so erh\"{a}lt man (nach Multiplikation mit dem Faktor $1/2$) den Energieerhaltungssatz
\begin{align}
\frac{1}{2}\, G(w,w) = \underbrace{\frac{1}{2}\,\int_{0}^{\,l} p(x)\,w_1(x)\,dx + \frac{1}{2}\, P\,w_2(x_P)}_{A_a} - \underbrace{\frac{1}{2}\,\int_0^{\,l} \frac{M^2}{EI}\,dx}_{A_i} = 0\,.
\end{align}
Ist $\textcolor{red}{\delta w(x)}$ eine zul\"{a}ssige virtuelle Verr\"{u}ckung, also $\textcolor{red}{\delta w = 0}$ in den Lagern, dann ergibt sich
\begin{align}
G(w,\textcolor{red}{\delta w}) = \underbrace{\int_{0}^{\,l} p(x)\,\textcolor{red}{\delta w(x)}\,dx + P\, \textcolor{red}{\delta w(x_P)}}_{\delta A_a} - \underbrace{\int_0^{\,l} \frac{M\, \textcolor{red}{\delta M}}{EI}\,dx}_{\delta A_i} = 0\,.
\end{align}
Wenn die virtuelle Verr\"{u}ckung $\delta w(x)$ in den Lagern nicht null ist, (was ja mathematisch  durchaus zul\"{a}ssig ist), dann leisten auch die Lagerkr\"{a}fte  Arbeiten
\begin{align}
G(w,\textcolor{red}{\delta w}) &= \int_{0}^{\,l} p(x)\,\textcolor{red}{\delta w(x)}\,dx  + P \cdot \textcolor{red}{\delta w(x_P)} + A\cdot\textcolor{red}{\delta w(x_a)} + B\cdot\textcolor{red}{\delta w(x_b) }\nn \\ &+ C\cdot\textcolor{red}{\delta w(x_c)}
 - \int_0^{\,d} \frac{M\,\textcolor{red}{ \delta M}}{EI}\,dx = 0\,.
\end{align}
Was an \"{a}u{\ss}erer Arbeit in den Lagern hinzukommt, wird nat\"{u}rlich  dadurch kompensiert, dass sich auch $\textcolor{red}{\delta M} $ \"{a}ndert, aber insgesamt bleibt die Bilanz null.


Auch die Translation $\textcolor{red}{\delta w(x) = 1}$ ist offiziell keine zul\"{a}ssige virtuelle Verr\"{u}ckung, aber das disqualifiziert sie nicht. Im Gegenteil, ihre Anwendung liefert das  Gleichgewicht der vertikalen Kr\"{a}fte
\begin{align}
G(w,\textcolor{red}{1}) &= \int_{0}^{\,l} p(x)\,dx + P \cdot \textcolor{red}{1}  + A\cdot \textcolor{red}{1} + B\cdot \textcolor{red}{1} + C\cdot \textcolor{red}{1}  = 0
\end{align}
und die Drehung $\textcolor{red}{\delta w(x)} = x$, die weder klein noch zul\"{a}ssig ist, kontrolliert das Moment um das linke Lager
\begin{align}
G(w,\textcolor{red}{x}) &=\int_{0}^{\,l} p(x)\,\textcolor{red}{x}\,dx + P \cdot \textcolor{red}{x_P} + B\cdot \textcolor{red}{x_b} + C\cdot \textcolor{red}{x_c}  = 0\,.
\end{align}
Die Crux mit zul\"{a}ssigen und nicht zul\"{a}ssigen virtuellen Verr\"{u}ckungen kann man am einfachsten so umgehen, dass man prinzipiell alle Lager entfernt und durch das Anbringen der Lagerkr\"{a}fte den Balken in der Schwebe h\"{a}lt. Dann sind alle virtuellen Verr\"{u}ckungen, wenn sie nur hinreichend glatt sind,  $\textcolor{red}{\delta w} \in C^2(0,l) $, zul\"{a}ssige virtuelle Verr\"{u}ckungen.\\


Auf die Interpretation von $\vek u^T\,\vek K\,\vek  u$ als innerer virtueller Energie kommt man wie folgt: Die Federsteifigkeit eines Stabes ist $k = EA/l$ und daher ist umgekehrt das $EA $, das zu einer Feder geh\"{o}rt, $EA = k\,l$. Der Einfachheit halber sei nur ein Ende der Feder beweglich $(u)$. Die Normalkraft in der Feder ist
$N\,= k\,u $, und es ist $\textcolor{red}{\delta N\,= k\,\delta u}$, so dass
\begin{align}
\int_0^{\,l} \frac{N\,\textcolor{red}{\delta N}}{EA}\,dx = \int_0^{\,l} \frac{k\,u \cdot k\,\textcolor{red}{\delta u}}{k\,l} \,dx = \textcolor{red}{\delta u}\,k\,u\,.
\end{align}
Die virtuelle innere Energie in dem Tragwerk ist jetzt also
\begin{align}
\delta A_i = \int \frac{M\,\textcolor{red}{\delta M}}{EI}\,dx + \int \frac{N\,\textcolor{red}{\delta N}}{EA}\,dx + \textcolor{red}{\vek \delta u^T}\,\vek K\,\vek u
\end{align}
und sinngem\"{a}{\ss} sind $\textcolor{red}{\delta A_i^*}$ und $A_i $ zu erweitern
\begin{align}
\textcolor{red}{\delta A_i^*} = \ldots + \vek u^T\,\vek K\,\textcolor{red}{\vek\delta \vek  u^*} \qquad A_i = \ldots + \frac{1}{2}\,\vek  u^T\,\vek K\,\vek u\,.
\end{align}
Bei Federn, die sich mit einem Ende auf den Boden abst\"{u}tzen, ist ein Freiheitsgrad gesperrt und es verbleibt nur ein Freiheitsgrad, den wir $u $ nennen wollen.
Die Identit\"{a}t erlaubt verschiedene Schreibweisen
\begin{align}
G(u, \textcolor{red}{\delta u}) = \textcolor{red}{\delta u}\,f - u\,k\,\textcolor{red}{\delta \,u} = \textcolor{red}{\delta u}\,f - \frac{f\,\textcolor{red}{\delta f}}{k} = \textcolor{red}{\delta A_a} - \textcolor{red}{\delta A_i} = 0\,.
\end{align}
In der Formulierung als Prinzip der virtuellen Kr\"{a}fte
\begin{align}
G(\textcolor{red}{\delta u^*}, u) = u\,\textcolor{red}{\delta f^*} - \textcolor{red}{\delta u^*}\,k\,u =  u\,\textcolor{red}{\delta \,f^*} - \frac{\textcolor{red}{\delta \,f^*}\, f}{k} = \textcolor{red}{\delta A_a^*} - \textcolor{red}{\delta A_i^*} = 0
\end{align}
bedeutet das also
\begin{align}
\textcolor{red}{\delta A_i^*} = \frac{\textcolor{red}{\delta \,f^*}\, f}{k}= \textcolor{red}{\delta f^*} \times \frac{f}{k} = \text{Kraft}^*\,\times\,\text{Weg}\,.
\end{align}
Wir stellen uns nun vor, dass wir an einer Stelle des Tragwerks eine Einzelkraft $\textcolor{red}{\bar{P} = \bar{1}}$ aufbringen, um mit dem Prinzip der virtuellen Kr\"{a}fte eine Verformung $\delta $ aus einem Lastfall $p$ zu berechnen. Diese Kraft $\textcolor{red}{\bar{P} = \bar{1}}$ verursacht eine Kraft $\textcolor{red}{\delta f^*}$ in der Feder.

Diese Kraft mal der Zusammendr\"{u}ckung $u = f/k$ (aus dem Lastfall $p$) ist der Beitrag der Feder zur Arbeitsgleichung
\begin{align}
\textcolor{red}{\delta A_a^*} = \textcolor{red}{\bar{1}} \cdot \delta  = \int \frac{\textcolor{red}{\bar{M}}\,M}{EI}\,dx + \int \frac{\textcolor{red}{\bar{N}}\,N}{EA}\,dx + \frac{\textcolor{red}{\delta f^*} \,f}{k}= \textcolor{red}{\delta A_i^*}\,.
\end{align}
Er ist positiv, wenn $\textcolor{red}{\delta f^*} $ und $f $ dasselbe Vorzeichen haben.
\\

%%%%%%%%%%%%%%%%%%%%%%%%%%%%%%%%%%%%%%%%%%%%%%%%%%%%%%%%%%%%%%%%%%%%%%%%%%%%%%%%%%%%%%%%%%%%%%%%%%%
{\textcolor{blau2}{\section{Vereinfachung}}}
Die vielen Unterbrechungen an den Zwischenlagern und der Vorzeichenwechsel zwischen der linken und der rechten Querkraft und ebenso den Biegemomenten links und rechts an den Intervallgrenzen f\"{u}hren schnell dazu, dass man den \"{U}berblick verliert.

Bevor man jetzt anf\"{a}ngt und sich m\"{u}ht herauszufinden, wie denn die Lagerkraft zu den Querkr\"{a}ften steht, ob die Lagerkraft positiv oder negativ ist etc. empfiehlt es sich wie folgt vorzugehen:

Die erste Greensche Identit\"{a}t wird so angeschrieben, als ob alle Lagerkr\"{a}fte dieselbe Richtung h\"{a}tten wie die positiven virtuellen Verr\"{u}ckungen
\begin{align}
G(w, \textcolor{red}{\delta w}) &= M_A \cdot \textcolor{red}{\delta w'(x_1)} + A\cdot \textcolor{red}{\delta w(x_1)} + P\cdot \textcolor{red}{\delta w(x_2)} + M \cdot \textcolor{red}{\delta w'(x_3)}\nn \\
&+ B \cdot \textcolor{red}{\delta w(x_4)} + \int_{x_5}^{\,x_6} p(x)\,\textcolor{red}{\delta w(x)}\,dx + C\cdot\textcolor{red}{\delta w(x_6)} = 0\,.
\end{align}
Die positiven Weggr\"{o}{\ss}en legen also die positiven Richtungen fest. Im zweiten Schritt werden die Lagerkr\"{a}fte eingesetzt und zwar wie sie wirklich wirken, also mit Vorzeichen.
So erh\"{a}lt man automatisch das richtige Ergebnis.

In der Regel wird man zudem nur zul\"{a}ssige virtuelle Verr\"{u}ckungen benutzen, wir haben diese Einschr\"{a}nkung hier
bewusst nicht gemacht, und dann sind die Arbeiten in den Lagern sowieso null.\\


Weil das so ist, bringen wir jedem Student im ersten Semester Statik bei, dass in der linearen Statik Drehungen durch Bewegungen l\"{a}ngs der Tangenten an den Drehkreis ersetzt werden. Hier passt sich gezwungenerma{\ss}en die Statik der Mathematik an und opfert die Anschauung der Mathematik.
\\

Genau so kommen die folgenden Identit\"{a}ten zustande. Sie beruhen einfach nur auf partieller Integration, also identischen Umformungen, und wir gehen daher kein Risiko ein, wenn wir behaupten dass diese f\"{u}r alle Paare von Funktionen null sind.
\\

Nun ist es ja so, dass die Randarbeiten $[V\,\textcolor{red}{\delta w} - M\,\textcolor{red}{\delta w'}]$ alternierende Vorzeichen haben, was die Behandlung von Lagersenkungen theoretisch etwas kompliziert macht. Jedesmal m\"{u}sste man sich neu \"{u}berlegen mit welchem Vorzeichen man den Beitrag versieht und welches Vorzeichen er dann schlie{\ss}lich hat, wenn man ihn auf den rechte Seite bringt.

Man kann das ganze Problem aber umgehen, wenn man sich auf den statischen Gehalt konzentriert. Die Terme in der eckigen Klammer sind \"{a}u{\ss}ere Arbeiten, und damit sind sie im Endergebnis gleich den Arbeiten, die die virtuellen Lagerkr\"{a}fte $\textcolor{red}{\bar{F}}$ bzw. $\textcolor{red}{\bar{M}}$ (Momente) auf den vorgegebenen Lagersenkungen $w_\Delta$ und Lagerverdrehungen $\tan \Np_\Delta$ leisten. Die Bilanz lautet also
\\



Die Begriffe virtuelle Verr\"{u}ckungen und virtuelle Kr\"{a}fte sind f\"{u}r den Anf\"{a}nger eher verwirrend, weil er meint dies w\"{a}ren spezielle Zust\"{a}nde. Insbesondere wenn von den virtuellen Verr\"{u}ckungen oder den virtuellen Kr\"{a}ften ausdr\"{u}cklich verlangt wird, dass sie 'klein' sein m\"{u}ssen.

Virtuell soll ausdr\"{u}cken, dass diese Zust\"{a}nde gedachte Zust\"{a}nde sind, Gedankenexperimente. Ansonsten sind es ganz normale Lastf\"{a}lle, wie der Originallastfall auch. Nur dass sie eben so ausgew\"{a}hlt werden, dass man bei der Formulierung von $G(w, \textcolor{red}{\delta w}) = 0 $ oder $G(\textcolor{red}{\delta w^*},  w) = 0 $ die Informationen erh\"{a}lt, die man sucht.

Die Notation $\textcolor{red}{\delta w^*}$ ist nicht gl\"{u}cklich. Virtuell ist doch gedacht und warum wird das eine ohne Stern und das andere aber mit Stern geschrieben? Wichtig ist doch nur die Reihenfolge in der ersten Greenschen Identit\"{a}t. Was kommt zuerst, erst $w$ und dann $\textcolor{red}{\delta w} $ oder erst $\textcolor{red}{\delta w}$ und dann $w$?
\\

%--------------------------------------------------------------------------------------------------
\subsubsection*{Querkraft}
Die linke Querkraft kann sich nicht bewegen, weil sie auf dem Kragtr\"{a}ger sitzt, so dass die rechte Querkraft den ganzen Weg alleine gehen muss. Dies erreicht man, indem man das Gelenk so spreizt, dass
der Teil hinter dem Gelenk auf der ganzen L\"{a}nge um eine L\"{a}ngeneinheit nach unten gleitet.

\"{U}bertr\"{a}gt man diese Bewegung auf das Original, dann leisten dort die \"{a}u{\ss}eren Kr\"{a}fte die Arbeit $A_{1,2} = - V\cdot \textcolor{blau2}{1} + P \cdot \textcolor{blau2}{1}$ w\"{a}hrend
die Arbeiten an der Kopie null sind, $A_{2,1} = 0$, weil keine Kr\"{a}fte n\"{o}tig sind, um den Teil hinter dem Gelenk abzusenken
\begin{align}
A_{1,2} = - V\cdot \textcolor{blau2}{1} + P \cdot \textcolor{blau2}{1} = A_{2,1} = 0
\end{align}
 oder
\begin{align}
V = P\,.
\end{align}
%--------------------------------------------------------------------------------------------------
\subsubsection*{Moment}
Man baut in die Kopie ein Gelenk ein und kippt den \"{a}u{\ss}eren Schenkel des Kragtr\"{a}ger so nach oben, dass er einen Winkel $\textcolor{blau2}{\tan \Np = 1} $ mit der Horizontalen bildet. Damit wird der Forderung Gen\"{u}ge getan, dass die beiden Momente $M(x)$ insgesamt einen Weg von $\textcolor{blau2}{(-1)}$ zur\"{u}cklegen. Das linke Moment sitzt fest auf dem unverr\"{u}ckbaren Kragtr\"{a}ger und so muss das Moment rechts vom Gelenk den Weg alleine gehen.

Zum Spreizen des Gelenks sind keine Kr\"{a}fte n\"{o}tig und so ist $A_{2,1} = 0$. Insgesamt erh\"{a}lt man somit f\"{u}r den {\em Satz von Betti\/}
\begin{align}
A_{1,2} = - M\,\textcolor{blau2}{\tan\,\Np} - P\,\textcolor{blau2}{\tan\,\Np} \cdot 0.5\,l = A_{2,1} = 0
\end{align}
oder wegen $\textcolor{blau2}{\tan\,\Np = 1}$
\begin{align}
M = - P \cdot 0.5\,l\,.
\end{align}
%--------------------------------------------------------------------------------------------------
\subsubsection*{Normalkraft}
Jetzt wird ein Normalkraftgelenk eingebaut und um eine L\"{a}ngeneinheit gespreizt. Wieder ist es so, dass nur die rechte Normalkraft sich bewegen kann und sie daher den vollen Weg $\textcolor{blau2}{-1}$ gehen muss. Das Spreizen des Normalkraftgelenks am rechten System erfordert keinerlei Kr\"{a}fte und somit ergibt sich der {\em Satz von Betti\/} zu
\begin{align}
A_{1,2} = - N \cdot \textcolor{blau2}{1} + P\cdot \textcolor{blau2}{1} = A_{2,1} = 0
\end{align}
oder
\begin{align}
N = P\,.
\end{align}\\

Was in der linearen Algebra die Einheitsvektoren $\vek e_i$ sind, sind in der Ana\-lysis die Dirac-Deltas. Mittels der Dirac-Deltas kann man die Einflussfunktion f\"{u}r die Durchbiegung $w$ eines Tr\"{a}gers in einem Punkt $x$ als die L\"{o}sung der Differentialgleichung
\begin{align}
EI\,\frac{d^4}{dy^4}\,G_0(y,x) = \delta(y-x)
\end{align}
interpretieren.

Die Formulierung des Satzes von Betti wird dann sehr einfach
\begin{align}
B(G_0,w) = \int_0^{\,l} \delta(y-x)\,w(y)\,dy - \int_0^{\,l} G_0(y,x)\,p(y)\,dy = 0
\end{align}
oder
\begin{align}
w(x) = \int_0^{\,l} G_0(y,x)\,p(y)\,dy \,.
\end{align}
Wir wiederholen hier das, was wir oben schon gesagt haben: Das Operieren mit Dirac-Delta is ein sehr suggestiver Kalk\"{u}l, aber man muss deutlich sehen, dass es ein {\em symbolisches Rechnen\/} ist und bleibt. Mit dem Dirac-Delta kann man sehr einfach die vielen Effekte, die hinter einer Einzelkraft stehen, wie die Zweiteilung der L\"{o}sung, der Sprung in der Querkraft, etc., umgehen, aber auf der anderen Seite darf man nicht vergessen, dass man nur auf dem Weg
\begin{align}
B(G_0,w) = B(G_{@0}^{@l},w)_{(0,x)} + B(G_{@0}^{@l},w)_{(x,l)} = 0 + 0
\end{align}
die Ergebnisse begr\"{u}nden kann. Anders gesagt: Wenn man wei{\ss}, was herauskommt, kann man die Abk\"{u}rzung nehmen, aber vorher muss man wissen, was eigentlich herauskommt...
\\

%%%%%%%%%%%%%%%%%%%%%%%%%%%%%%%%%%%%%%%%%%%%%%%%%%%%%%%%%%%%%%%%%%%%%%%%%%%%%%%%%%%%%%%%%%%%%%%%%%%
{\textcolor{blau2}{\subsection{Archimedes' Hebel}}}
Archimedes wusste, wie so etwas passieren kann und er behauptete auch prompt, dass er die Welt aus den Angeln heben k\"{o}nnte, wenn man ihm nur einen festen Punkt g\"{a}be, siehe Bild \ref{Diss15}\,a.\\

{\em Die Kraft seiner Hand mal der L\"{a}nge des Hebels ist gleich dem Moment der Erde um den festen Punkt.\/}\\

Dort, wo bei Archimedes die Erde sitzt, platzieren wir ein Lager und wir fragen, welche Lagerkraft $R_A$ ist notwendig, um dem Druck von Archimedes Hand zu widerstehen.

Die Einflussfunktionen f\"{u}r die Lagerkraft $R_A$ = die Verschiebungsfigur, die sich einstellt wenn man das linke Lager entfernt, und dort den Balken um eine L\"{a}ngeneinheit nach unten dr\"{u}ckt

In der linearen Statik und auch Mechanik sind Drehungen Pseudodrehungen: Der Abstand $x$ eines Punktes von dem Drehpunkt und die vertikale Verschiebung $y$ bilden ein rechteckiges Dreieck
\beq
\tan \Np = \frac{y}{x}
\eeq
mit $\Np$ als dem Drehwinkel. Das Lager geht den Weg
$y = 1$ und daher muss der Hebel eine $90^0$ Drehung vollf\"{u}hren, wenn der Abstand $h$ zwischen den beiden Punkten infinitesimal klein wird
\beq
\tan \Np = \lim_{h \to 0} \frac{1}{h} = \infty
\eeq
und dies bedeutet, dass eine unendlich gro{\ss}e Kraft
\begin{equation}
R_A = \lim_{h \to 0} \frac{1}{h} \,l\,P
\end{equation}
notwendig ist, um selbst der kleinsten Kraft $P$ am anderen Ende des Tr\"{a}gers das Gleichgewicht zu halten.\\

%%%%%%%%%%%%%%%%%%%%%%%%%%%%%%%%%%%%%%%%%%%%%%%%%%%%%%%%%%%%%%%%%%%%%%%%%%%%%%%%%%%%%%%%%%%%%%%%%%%
{\textcolor{blau2}{\section{Betti extended}}}
Die Resultate in diesem Kapitel beruhen auf einem Satz, den wir {\em Betti extended\/} nennen. Er ist eine Erweiterung des Satzes von Betti auf finite Elemente.

Die L\"{a}ngsverschiebung $u(x)$ eines Stabes in einem Punkt $x$ kann gem\"{a}{\ss} dem {\em Satz von Betti\/} auf zwei Arten dargestellt werden
\begin{align} \label{Eq47}
u(x) = \int_0^{\,l} \delta_0(y-x)\,u(y)\,dy = \int_0^{\,l} G_0(y,x)\,p(y)\,dy\,.
\end{align}
Man kann die Funktion $u$ mit dem Dirac-Delta \"{u}berlagern oder die Belastung $p$ mit der durch die Einzelkraft ausgel\"{o}sten L\"{a}ngsverschiebung $G_0(y,x)$. Beide Formulierungen liefern dasselbe Resultat.

Der {\em Satz von Betti\/} extended besagt nun, dass man in (\ref{Eq47}) die beiden Funktionen $u$ und $G_0(y,x)$ durch ihre Projektionen auf $V_h$ ersetzen darf, also durch die FE-L\"{o}sungen $u_h$ bzw. $G_h(y,x)$
\begin{align} \label{Eq47}
u_h(x) = \int_0^{\,l} \delta(y-x)\,u_h(y)\,dy = \int_0^{\,l} G_h(y,x)\,p(y)\,dy
\end{align}
und dass das Ergebnis dann die Verschiebung $u_h(x)$ der FE-L\"{o}sung in dem Punkt $x$ ist.

Allgemeiner gefasst besagt der {\em Satz von Betti extended\/} das folgende: Ist $J(u)$ ein lineares Funktional, das die Darstellung
\begin{align}
J(u) = \int_0^{\,l} \delta(y-y)\,u(y)\,dy = \int_0^{\,l} G(y,x)\,p(y)\,dy
\end{align}
hat, dann erh\"{a}lt man den Wert $J(u_h)$, also den Wert des Funktionals bezogen auf die FE-L\"{o}sung, mit den beiden \"{a}quivalenten Formulierungen
\begin{align}
J(u_h) = \int_0^{\,l} \delta(y-y)\,u_h(y)\,dy = \int_0^{\,l} G_h(y,x)\,p(y)\,dy\,.
\end{align}
Auf diesem Satz beruht die Technik der Einflussfunktionen f\"{u}r finite Elemente.

So wie die exakte Einflussfunktion $G(y,x)$ durch \"{U}berlagerung mit $p$ den exakten Wert
\begin{align}
J(u) = \int_0^{\,l} G_h(y,x)\,p(y)\,dy
\end{align}
liefert, so liefert die gen\"{a}herte Einflussfunktion $G_h(y,x)$ den Wert $J(u_h)$ f\"{u}r die FE-L\"{o}sung
\begin{align}
J(u_h) = \int_0^{\,l} G_h(y,x)\,p(y)\,dy\,.
\end{align}
Fassen wir das im gr\"{o}{\ss}eren Rahmen. Die Technik der finiten Elemente besteht darin, dass wir die exakte L\"{o}sung auf $V_h$ projizieren,
\begin{align}
a(u-u_h,\Np_i) = 0 \qquad i = 1,2,\ldots n\,.
\end{align}
Parallel dazu projizieren wir aber auch alle Einflussfunktionen auf $V_h$
\begin{align}
a(G-G_h,\Np_i) = 0 \qquad i = 1,2,\ldots n\,.
\end{align}
Und der Zusammenhang zwischen $u$ und den diversen Einflussfunktionen $G(y,x)$
\begin{align}
J(u) = \int_0^{\,l} G(y,x)\,p(y)\,dy
\end{align}

\"{u}bertr\"{a}gt sich automatisch auf die N\"{a}herungen $u_h$ und $G_h(y,x)$, d.h. der Zusammenhang geht nicht verloren, denn
\begin{align}
J(u_h) = \int_0^{\,l} G_h(y,x)\,p(y)\,dy
\end{align}

Wenn man jetzt die Einflussfunktionen trotzdem berechnen will, dann steht man vor einem Problem, denn die zur FE-Einflussfunktion geh\"{o}rigen \"{a}quivalenten Knotenkr\"{a}fte $j_i $ sind alle null
\begin{align}
j_i = J(\vek \Np_i) = 0\,.
\end{align}
Wenn wir jetzt trotzdem versuchen, die Einflussfunktionen mit finiten Elementen zu berechnen, dann sind die \"{a}quivalenten Knotenkr\"{a}fte $f_i$ gleich den Lagerkr\"{a}ften, die zu den Verschiebungsfeldern
$\vek \Np_i$ geh\"{o}ren. Diese werden wie oben berechnet. Es sei $\vek \Np_X$ das Verschiebungsfeld, bei dem sich das Lager um eine L\"{a}ngeneinheit in vertikaler Richtung absenkt.

Die \"{a}quivalenten Knotenkr\"{a}fte $f_i$, die die (gen\"{a}herte) Einflussfunktion generieren, sind dann die Wechselwirkungsenergie zwischen den Feldern $\vek \Np_i$ und dem Feld $\vek \Np_X$
\begin{align}
a(\vek \Np_i,\vek \Np_X) = f_i \cdot 1\,.
\end{align}
Der Einflussfunktion, die die $f_i$ generieren, fehlt wieder, wie oben, das St\"{u}ck $\vek \Np_X$. Wenn man es dazu addiert, hat man die ganze Einflussfunktion.\\



%%%%%%%%%%%%%%%%%%%%%%%%%%%%%%%%%%%%%%%%%%%%%%%%%%%%%%%%%%%%%%%%%%%%%%%%%%%%%%%%%%%%%%%%%%%%%%%%%%%
{\textcolor{blau2}{\section{Einflussfunktionen f\"{u}r \"{a}quivalente Knotenkr\"{a}fte}}}
Eine Belastung wird in einen Knoten reduziert, indem man die Belastung mit der Einheitsverschiebung des Knotens \"{u}berlagert
\begin{align}
f_i = \int_0^{\,l} p(x)\,\Np_i(x)\,dx\,.
\end{align}
Die Einflussfunktionen f\"{u}r die Knotenkr\"{a}fte sind also identisch mit den Einheitsverschiebungen der Knoten. So weit, wie sich eine Einheitsverschiebung erstreckt, soweit reicht der Einfluss eines Knotens. Je kleiner die Elemente werden, um so geringer wird also auch die Knotenkraft $f_i $, die auf einen einzelnen Knoten entf\"{a}llt.

Was wir bei einer FE-Analyse als Knotenkr\"{a}fte $f_i $ bezeichnen, sind eigentlich \"{a}quivalente Knotenkr\"{a}fte, also nicht echte Knotenkr\"{a}fte im physikalischen Sinne, sondern es sind 'Rechenpfennige'. Wenn in einem Knoten eine horizontale  Knotenkraft $f = 10$ kNm wirkt, dann bedeutet es, dass in der N\"{a}he des Knotens Lasten so verteilt sind dass sie bei einer horizontalen Auslenkung des Knotens um eine L\"{a}ngeneinheit die Arbeit $f = 10$ kNm leisten.

Wie genau diese Kr\"{a}fte verteilt sind, wo sie konzentriert sind und wo nicht, das bleibt offen und so gibt es mehrere Anordnungen von Kr\"{a}ften die alle dieselben \"{a}quivalente Knotenkr\"{a}fte aufweisen.

Die \"{a}quivalente Knotenkr\"{a}fte sind also ein K\"{u}rzel f\"{u}r die Arbeiten, die die reale Belastung auf den Einheitsverformungen der Knoten leisten. Ein FE-Programm denkt und rechnet in Arbeiten, $\vek K\,\vek u = \vek f$. Die $u_i $ sind (dimensionslose) Knotenverschiebungen, aber die $f_i $ sind Arbeiten.


\begin{remark}
Der Vektor $\vek r$ ist schwach besetzt, weil ja nur die Ansatzfunktionen der Knoten, die dem Lagerknoten direkt gegen\"{u}berliegen, eine \"{a}quivalente Lagerkraft $r_i = a(\Np_X,\Np_i)$ in dem Lager aufweisen. Die weiter abliegenden $\Np_i$ '\"{u}berlappen' sich nicht mit $\Np_X$.
\end{remark}

Wir k\"{o}nnen nur gen\"{a}herte Einflussfunktionen erzeugen, aber es gilt, dass der Fehler in den Einflussfunktionen orthogonal zur Belastung ist. Anders gesagt die Abweichung zwischen $\sigma_{xx}^h$ und $\sigma_{xx}$ ist in jedem Punkt Null, obwohl der Fehler in der Einflussfunktion, $G(\vek y,\vek x) - G_h(\vek y,\vek x)$, nicht null ist!



Die physikalisch echte Momentenbedingung unter Benutzung des Drehkreises erf\"{u}llt der Balken dagegen nicht!


Im folgenden wollen wir die Berechnung von Einflussfunktionen f\"{u}r Schnitt\-gr\"{o}{\ss}en etwas systematischer fassen.
\begin{alignat}{2}
\textcolor{blau2}{G_1(y,x)} &= \text{Einflussfunktion f\"{u}r $N(x)$} &&= \phantom{-}EA\,u'(x)\nn \\
\textcolor{blau2}{G_2(y,x)} &= \text{Einflussfunktion f\"{u}r $M(x)$} &&= -EI\,w''(x)\nn \\
\textcolor{blau2}{G_3(y,x)} &= \text{Einflussfunktion f\"{u}r $V(x)$} &&= -EI\,w'''(x)\nn
\end{alignat}
Der Index 1, 2 und 3 korrespondiert dabei der Ordnung der Ableitung (letzte Spalte).

Weil die Funktionen dadurch entstehen, dass man ein Gelenk spreizt, muss man den {\em Satz von Betti\/} f\"{u}r den linken und rechten Teil
separat formulieren und dann addieren.

Die Verformung $w(x)$ oder $u(x)$ f\"{u}r die man eine Einflussfunktion sucht, hei{\ss}t auch die {\em prim\"{a}re L\"{o}sung\/} und die Einflussfunktion hei{\ss}t dementsprechend die {\em duale L\"{o}sung\/}.
\\

Zu der Balkendifferentialgleichung $EI\,w^{IV}(x) = p(x) $, die ja eine Differentialgleichung vierter Ordnung ist, geh\"{o}ren vier Dirac-Deltas und zu der Stab-Gleichung $- EA\,u''(x) = p(x)$ und allen anderen Differentialgleichungen zweiter Ordnung geh\"{o}ren zwei Dirac-Deltas, s. Bild \ref{VierBeam14},
\begin{align}
&\delta_0(y-x) \qquad \text{Einzelkraft}\\
&\delta_1(y-x) \qquad \text{Verschiebungssprung in Achsrichtung}
\end{align}
die jeweils den konjugierten Wert liefern
\begin{align}
\int_0^{\,l} \delta_0(y-x)\,u(y)\,dy = u(x) \qquad \int_0^{\,l} \delta_1(y-x)\,u(y)\,dy = N(x)\,.
\end{align}


F\"{u}r das Verst\"{a}ndnis ist es wichtig zu sehen, dass der {\em Satz von Betti\/} immer mit zwei Systemen operiert, dem Originaltr\"{a}ger und einer $1:1$ Kopie. In beide Tr\"{a}ger wird das Gelenk eingebaut, aber
nur die Kopie wird gespreizt. Und zwar so, dass, wenn man diese Bewegung auf den Originaltr\"{a}ger \"{u}bertr\"{a}gt, die Kraftgr\"{o}{\ss}en links und rechts von dem Gelenk in der Summe den Weg $(-1)$ zur\"{u}ck legen, s. Bild \ref{S17}.\\

Die Berechnung von Einflussfunktionen durch den Einbau von Gelenken und deren Spreizung, s. Bild \ref{1GreenF226}, wird in der Literatur manchmal mit dem Namen {\em Mueller-Breslau\/} verbunden, obwohl es ja doch eigentlich der {\em Satz von Betti\/} ist.\\



Beim Rechnen in der Statik kommt {\em Meter\/} vor und nat\"{u}rlich auch {\em Newton\/}, aber {\em rad\/} kommt (normalerweise) gar nicht vor. Der Tangens, ja, der kommt auch st\"{a}ndig vor, aber der hat keine Dimension...

Nat\"{u}rlich kann man jetzt einwenden, dass die Autoren, die dem Tangens $\delta$ die Dimension {\em rad\/} geben, dies in der Absicht tun, das Ergebnis f\"{u}r den Leser in eine ihm gewohnte Dimension zu \"{u}bersetzen
\begin{align}
\delta = \tan\,\Np \simeq \Np\,,
\end{align}
aber wir haben den Eindruck, dass viele Autoren wirklich meinen, dass das $\delta$ ein Winkel sei.\\

Neue Formulierungen, neue Differentialgleichung sollte man also zuerst mit den trigonometrischen Funktionen testen, weil diese ja eine Basis des Funktionenraums \"{u}ber $(0,l)$ sind.
\\

Denn jeder, der davon \"{u}berzeugt ist, dass er mittels mechanischer Prinzipe mathematische Ergebnisse voraussagen kann, muss doch erkl\"{a}ren, wie das gehen soll, wie man mit S\"{a}tzen aus einem Gebiet {\em A\/} (Mechanik) Resultate im Gebiet {\em B\/} (Mathematik) herleiten kann.\\


%%%%%%%%%%%%%%%%%%%%%%%%%%%%%%%%%%%%%%%%%%%%%%%%%%%%%%%%%%%%%%%%%%%%%%%%%%%%%%%%%%%%%%%%%%%%%%%%%%%
{\textcolor{blau2}{\section{Durchlauftr\"{a}ger}}}
Das Thema Einzelkr\"{a}fte leitet \"{u}ber zu Durchlauftr\"{a}gern, bei denen ja in den Lagern Einzelkr\"{a}fte den Tr\"{a}ger st\"{u}tzen.

Wenn die virtuellen Verr\"{u}ckungen zul\"{a}ssig sind, dann reduziert sich das Prinzip der virtuellen Verr\"{u}ckungen auf die einfache Gleichung
\begin{align}
\delta A_a = \int_0^{\,l} p \,\textcolor{red}{\delta w}\,dx = \int_0^{\,l} \frac{M \textcolor{red}{\delta M}}{EI}\,dx = \delta A_i\,.
\end{align}


Der Effekt, auf denen es uns jetzt ankommt, sind die Arbeiten, die die Lagerkr\"{a}fte leisten, wenn man eine beliebige virtuelle Verr\"{u}ckungen $ \textcolor{red}{\delta w}$ ansetzt. Wir nehmen uns ein beliebiges Zwischenlager vor, s. Bild \ref{U19}. Die Querkr\"{a}fte im Anschnitt zum Lager nennen wir $V^-$ und $V^+$, entsprechend den Seiten auf denen sie wirken. Hei{\ss}t die Lagerkraft $B$ dann gilt
\begin{align}
B  = V^+ - V^-\,.
\end{align}
Wenn wir nun das Lager  um das Ma{\ss} $\textcolor{red}{\delta w} $ verr\"{u}cken, dann ergibt sich die virtuelle \"{a}u{\ss}ere Arbeit der Lagerkraft zu
\begin{align}
B \cdot \textcolor{red}{\delta w}\,,
\end{align}
wo also die Lagerkraft in derselben Richtung positiv gez\"{a}hlt wird, wie die virtuelle Verr\"{u}ckung.

Dieses Ergebnis stimmt mit dem Resultat \"{u}berein, das man erh\"{a}lt wenn man feldweise die erste Greensche Identit\"{a}t formuliert und addiert.
\\

Das einfachste Beispiel hierf\"{u}r ist ein gelenkig gelagerter Einfeldtr\"{a}ger
\begin{align}
\text{\normalfont\calligra G\,\,}(w, \textcolor{red}{\delta w}) &= \int_0^{\,l} p(x)\,\textcolor{red}{\delta w(x)}\,dx + [V\,\textcolor{red}{\delta w} - M\,\textcolor{red}{\delta w'}]_{@0}^{@l} - \int_0^{\,l} \frac{M\,\textcolor{red}{\delta M}}{EI}\,dx = 0\,,
\end{align}
denn wegen $M(0) = M(l) = 0 $ und $\textcolor{red}{\delta w(0) = \delta w(l) = 0}$ folgt leicht
\begin{align}
[V\,\textcolor{red}{\delta w} - M\,\textcolor{red}{\delta w'}]_{@0}^{@l} &= V(l)\,\textcolor{red}{\delta w(l)} - M(l)\,\textcolor{red}{\delta w'(l)} - V(0)\,\textcolor{red}{\delta w(0)} + M(0)\,\textcolor{red}{\delta w'(0)}\nn \\
&= V(l) \cdot 0 - 0\cdot \textcolor{red}{\delta w'(l)} - V(0) \cdot 0 + 0 \cdot \textcolor{red}{\delta w'(0)} = 0\,,
\end{align}
so dass sich die erste Greensche Identit\"{a}t in der Tat auf den Ausdruck
\begin{align}
\text{\normalfont\calligra G\,\,}(w, \delta w) &= \int_0^{\,l} p(x)\,\textcolor{red}{\delta w(x)}\,dx  - \int_0^{\,l} \frac{M\,\textcolor{red}{\delta M}}{EI}\,dx = 0
\end{align}
reduziert, in dem keine Randarbeiten mehr vorkommen.



Das einfachste Beispiel hierf\"{u}r ist ein gelenkig gelagerter Einfeldtr\"{a}ger
\begin{align}
\text{\normalfont\calligra G\,\,}(w, \textcolor{red}{\delta w}) &= \int_0^{\,l} p(x)\,\textcolor{red}{\delta w(x)}\,dx + [V\,\textcolor{red}{\delta w} - M\,\textcolor{red}{\delta w'}]_{@0}^{@l} - \int_0^{\,l} \frac{M\,\textcolor{red}{\delta M}}{EI}\,dx = 0\,,
\end{align}
denn wegen $M(0) = M(l) = 0 $ und $\textcolor{red}{\delta w(0) = \delta w(l) = 0}$ folgt leicht
\begin{align}
[V\,\textcolor{red}{\delta w} - M\,\textcolor{red}{\delta w'}]_{@0}^{@l} &= V(l)\,\textcolor{red}{\delta w(l)} - M(l)\,\textcolor{red}{\delta w'(l)} - V(0)\,\textcolor{red}{\delta w(0)} + M(0)\,\textcolor{red}{\delta w'(0)}\nn \\
&= V(l) \cdot 0 - 0\cdot \textcolor{red}{\delta w'(l)} - V(0) \cdot 0 + 0 \cdot \textcolor{red}{\delta w'(0)} = 0\,,
\end{align}
so dass sich die erste Greensche Identit\"{a}t in der Tat auf den Ausdruck
\begin{align}
\text{\normalfont\calligra G\,\,}(w, \delta w) &= \int_0^{\,l} p(x)\,\textcolor{red}{\delta w(x)}\,dx  - \int_0^{\,l} \frac{M\,\textcolor{red}{\delta M}}{EI}\,dx = 0
\end{align}
reduziert, in dem keine Randarbeiten mehr vorkommen.


Die Arbeiten $\textcolor{red}{\bar{F}_{k}}\,w_{\Delta\,k}$ sind positiv, wenn $\textcolor{red}{\bar{F}_k}$ und $w_{\Delta\,k}$ gleichgerichtet sind. Analoges gilt f\"{u}r die Momente und die Verdrehungen.
\\

%%%%%%%%%%%%%%%%%%%%%%%%%%%%%%%%%%%%%%%%%%%%%%%%%%%%%%%%%%%%%%%%%%%%%%%%%%%%%%%%%%%%%%%%%%%%%%%
{\textcolor{blau2}{\section{Die Arbeitsgleichung}}}
Die Mohrsche Arbeitsgleichung hat in ihrer elementarsten Form die Gestalt
\begin{align}
\textcolor{red}{\bar{1}}\cdot\delta = \int_0^{\,l} \frac{M\,\textcolor{red}{\bar{M}}}{EI}\,dx\,.
\end{align}
In den Lehrb\"{u}chern wird zum Beweis der Formel auf das Prinzip der virtuellen Kr\"{a}fte verwiesen.

Nun ist die Formel nat\"{u}rlich ein St\"{u}ck Mathematik, und sie wird nicht nur in der Statik verwendet, sondern auch zum Beispiel in der Meteorologie, in der Thermodynamik und vielen anderen Zweigen der Physik.  Daher wollen wir an dieser Stelle einmal einen mathe\-matischen Beweis f\"{u}r diese Formel f\"{u}hren, allerdings in der Sprache des Ingenieurs.

Zun\"{a}chst erstellt man eine Kopie des Tr\"{a}gers und belastet diese Kopie in Richtung der gesuchten Verformung mit einer Kraft $\textcolor{red}{\bar{P} = \bar{1} }$. Weil die Querkraft in der Balkenmitte springt
\begin{align}
\textcolor{red}{\bar{V}_L - \bar{V}_R = \bar{P}}\,,
\end{align}
muss man die Biegelinie aus zwei Teilen, $\textcolor{red}{\bar{w}_L} $ und $\textcolor{red}{\bar{w}_R} $, zusammensetzen. Diese beiden Teile sind homogene L\"{o}sungen der Balkengleichung (keine Belastung auf der freien Strecke) und in der Mitte des Balkens gehen sie, bis auf den Sprung in der Querkraft, stetig ineinander \"{u}ber, $\textcolor{red}{\bar{w}_L = \bar{w}_R}$, $\textcolor{red}{\bar{w}_L' = \bar{w}_R'}$ und $\textcolor{red}{\bar{M_L} = \bar{M_R}}$. F\"{u}r den ersten Abschnitt ergibt sich, wir integrieren bis $x = l/2$,
\begin{align}
\text{\normalfont\calligra G\,\,}(\textcolor{red}{\bar{w}_L},w)_{(0,x)} &= \textcolor{red}{\bar{V}_L}(x)\,w(x) - \textcolor{red}{\bar{M}_L(x)}\,w'(x) - \int_0^{\,x} \frac{M\,\textcolor{red}{\bar{M}_L}}{EI}\,dx = 0
\end{align}
und analog f\"{u}r den zweiten Abschnitt
\begin{align}
\text{\normalfont\calligra G\,\,}(\textcolor{red}{\bar{w}_R},w)_{(x,l)} &= -\textcolor{red}{\bar{V}_R(x)}\,w(x) + \textcolor{red}{\bar{M}_R(x)}\,w'(x)  - \int_{x}^{\,l} \frac{M\,\textcolor{red}{\bar{M}_L}}{EI}\,dx = 0\,.
\end{align}
In der Mitte ist $\textcolor{red}{\bar{M}_L = \bar{M}_R}$ und $w'$ ist stetig, so dass sich die Arbeitsbeitr\"{a}ge der Momente aufheben, und es verbleibt
\begin{align}
\text{\normalfont\calligra G\,\,}(\textcolor{red}{\bar{w}_L},w)_{(0,x)} + G(\textcolor{red}{\bar{w}_R)},w)_{(x,l)} = \textcolor{red}{\bar{P}}\,w(x) - \int_0^{\,l} \frac{\textcolor{red}{\bar{M}}\,M}{EI}\,dx = 0
\end{align}
oder
\begin{align}
w(x) =  \int_0^{\,l} \frac{\textcolor{red}{\bar{M}}\,M}{EI}\,dx\,,
\end{align}
was die Arbeitsgleichung ist.\\

Nun k\"{o}nnte man argumentieren, gut, jetzt haben wir die Lager weggenommen und dann ist der Begriff zul\"{a}ssig und nicht zul\"{a}ssig hinf\"{a}llig, aber warum m\"{u}ssen 'mit Lager' die virtuellen Verr\"{u}ckungen klein sein und wenn man die Lager wegnimmt, dann ist jede Gr\"{o}{\ss}e, jeder Ausschlag von $\textcolor{red}{\delta w}$ erlaubt?

Darauf k\"{o}nnte man erwidern, dass ja hier mit Starrk\"{o}rperbewegungen $\textcolor{red}{\delta w = a\,x + b}$ gearbeitet wird, und damit die virtuelle innere Energie sowieso null ist. Aber niemand hindert uns daran eine virtuelle Verr\"{u}ckungen $\textcolor{red}{\delta w(x)} $ zu w\"{a}hlen, die keine Starrk\"{o}rperbewegung ist, die gro{\ss} ist und trotzdem stimmt die Bilanz $\textcolor{red}{\delta A_a - \delta A_i = 0}$.
\\
Der eigentliche sch\"{o}pferische Akt dabei ist aber nicht die partielle Integration des Arbeitsintegrals, sondern die Wahl  der virtuellen Verr\"{u}ckung $\delta w $,   der Testfunktion. Wir testen ein A, die Streckenlast, indem wir es gegen ein B, die virtuelle Verr\"{u}ckung, halten. Wie A auf B reagiert, ist Hinweis darauf, wie A beschaffen ist.
\\


Um die Breite eines Schranks zu messen, halten wir einen Zollstock dagegen, das Gewicht eines K\"{o}rpers finden wir, indem wir ihn anheben, etc. Immer sind es zwei Dinge, die miteinander in Wechselwirkung stehen und daher ist die Dualit\"{a}t der Kernbegriff.\\

{\textcolor{blau2}{\section{Das Ziel}}
Wir wollen in diesem Kapitel die Arbeits- und Energieprinzipe der Statik herleiten und zeigen, auf welchen Grundlagen sie beruhen.

%----------------------------------------------------------------------------------------------------------
\begin{figure}[tbp]
\includegraphics[width=1.0\textwidth]{\Fpath/U16}
%\caption{Balken und m\"{o}gliche virtuelle Verr\"{u}ckungen, die auch gross sein d\"{u}rfen! Jeder Faktor ist zul\"{a}ssig, $\delta w = 10^{10}\,\sin(\pi\,x/\ell)$} \label{U16}
\end{figure}%
%----------------------------------------------------------------------------------------------------------

Wir beginnen mit einem einfachen Beispiel, dem Balken in Bild \ref{U16} a. In der Gleichgewichtslage ist gem\"{a}{\ss} dem Energieerhaltungssatz die innere Energie gleich der \"{a}u{\ss}eren Arbeit, die die Streckenlast auf dem eigenen Wege verrichtet
\begin{align}
\frac{1}{2}\,\int_0^{\,l} \frac{M^2}{EI}dx = \frac{1}{2}\,\int_0^{\,l} p\,w\,dx.
\end{align}
Wieso ist das so? Warum sind die beiden Integrale gleich?

Wenn wir dem Balken dann eine zul\"{a}ssige virtuelle Verr\"{u}ckung $\delta w $ erteilen, s. Bild \ref{U16} b, das ist eine Verr\"{u}ckung, die mit den Lagerbedingungen des Tr\"{a}gers vertr\"{a}glich ist,  $\delta w(0) = \delta w(l) = 0$, wie z.Bsp.
\begin{align}
\delta w(x) = \sin \frac{\pi\,x}{l}\,,
\end{align}
dann finden wir, dass die virtuelle \"{a}u{\ss}ere Arbeit genauso gro{\ss} ist, wie die virtuelle innere Energie
\begin{align}
\delta A_a = \int_0^{\,l} p\,\delta w\,dx = \int_0^{\,l} \frac{M\,\delta M}{EI}\,dx = \delta A_i\,.
\end{align}
Und das gilt nicht nur f\"{u}r dieses $\delta w = \sin x\,\pi/l$, sondern f\"{u}r {\em alle\/} nur vorstellbaren zul\"{a}ssigen virtuellen Verr\"{u}ckungen, also etwa alle Funktionen $\delta w$ in Bild \ref{U16} b. Warum sind wir dessen so sicher?

Jede Last l\"{o}st einen Wettbewerb aus, denn auf $V$, das sind alle Funktionen, die die Lagerbedingungen einhalten, geht es, sobald die Belastung auf den Balken gebracht wurde, darum das Minimum der potentiellen Energie
\begin{align}
\Pi(w) = \frac{1}{2}\,\frac{M^2}{EI}\,dx - \int_0^{\,l} p\,w\,dx
\end{align}
zu finden. Der Sieger dieses Wettbewerbes ist die Biegelinie des Balkens.

Und wenn wir nachrechnen, so finden wir, dass die potentielle Energie der Biegelinie {\em  negativ\/} ist
\begin{align}
\Pi(w) = -\frac{1}{2}\, \int_0^{\,l} p\,w\,dx\,,
\end{align}
denn das Integral f\"{u}r sich ist positiv, weil $p$ und $w$ in die gleiche Richtung zeigen.
%----------------------------------------------------------------------------------------------------------
\begin{figure}[tbp]
\centering
\if \bild 2 \sidecaption \fi
\includegraphics[width=.6\textwidth]{\Fpath/SECONDORDER}
\caption{Theorie II. Ordnung}
\label{SecondOrder}%
\end{figure}%
%----------------------------------------------------------------------------------------------------------

Das Minimum der potentielle Energie bedeutet also {\em nicht\/} m\"{o}glichst wenig Aufwand, die potentielle Energie $\Pi(w)$ m\"{o}glichst  dicht an Null r\"{u}cken, sondern das Gegenteil: m\"{o}glichst weit weg von null, den Abstand $|\Pi(w)|$ m\"{o}glichst gro{\ss} machen! Wenn man etwas, was negativ ist, kleiner macht, dann r\"{u}ckt man es weiter weg von Null, man vergr\"{o}{\ss}ert den Betrag $|\Pi(w)|$ der potentiellen Energie.

Das Prinzip vom Minimum der potentiellen Energie ist also eigentlich ein {\em Maximumprinzip\/}, zumindest in Lastf\"{a}llen $g$ oder $p$.

Und wenn wir die Belastung verdoppeln $p \to 2\,p$, und damit auch die Auslenkung des Balkens, $w \to 2\,w$, dann {\em vervierfacht\/} sich die potentielle Energie
\begin{align}
 - \frac{1}{2}\,\int_0^{\,l} 2\,  p\,\,2 \,w\,dx = 4\,\Pi(w)\,.
\end{align}
Zum Abschluss noch eine Bemerkung zur Theorie II. Ordnung.
Wenn man ein Tragwerk nach Theorie II. Ordnung berechnet, dann stellt man das Gleichgewicht am verformten Tragwerk auf, so hei{\ss}t es. Aber durch die Zusammendr\"{u}ckung des Kragtr\"{a}gers in L\"{a}ngsrichtung, s. Bild \ref{SecondOrder}, wird der Hebelarm kleiner, und diese Verk\"{u}rzung findet keine Ber\"{u}cksichtigung bei der Berechnung des Einspannmoments
\begin{align}
M = P \cdot l  + H \cdot w(l) \qquad \text{?}
\end{align}
denn eigentlich m\"{u}sste man f\"{u}r $l$ die leicht verk\"{u}rzte L\"{a}nge
\begin{align}
l' = l - \frac{H\,l}{EA}\,
\end{align}
setzen.

Weil das ein grunds\"{a}tzlicher 'Defekt' der Theorie II. Ordnung ist, kann man---genau genommen---das Gleichgewicht eines Rahmens, der nach Theorie II. Ordnung berechnet wurde, nicht kontrollieren, weil  die Knotenverformungen auf den L\"{a}ngen $l'$ beruhen, die Haltekr\"{a}fte aber mit den 'vollen' L\"{a}ngen $l$ berechnet werden.
\\

Beginnen wir mit einer {\em statisch bestimmt gelagerten\/} Scheibe. Am rechten Rand, in der H\"{o}he $h$, greift eine Einzelkraft $P$ an. Die Einflussfunktion f\"{u}r die vertikale Lagerkraft im rechten Rollenlager ist eine Rotation der ganzen Scheibe um das linke Lager und zwar so, dass sich das rechte Lager um einen Meter nach unten bewegt. Es ist nun evident, dass die Einflussfunktion nicht davon abh\"{a}ngt, wie dick die Backsteine sind. Genauso ist es f\"{u}r die Berechnung der Lagerkraft aus $P$ irrelevant, wieviele Schichten von Backsteinen noch \"{u}ber die H\"{o}he $h$ hinaus folgen.

Eine \"{a}hnliche Situation liegt bei einem 3-Gelenk-Bogen vor. Die Lagerkr\"{a}fte des Bogens h\"{a}ngen nur von der Lage der drei Gelenke zueinander ab, aber nicht davon, welche Form die beiden Bogenh\"{a}lften dazwischen haben.\\

%%%%%%%%%%%%%%%%%%%%%%%%%%%%%%%%%%%%%%%%%%%%%%%%%%%%%%%%%%%%%%%%%%%%%%%%%%%%%%%%%%%%%%%%%%%%%%%%%%%
{\textcolor{blau2}{\section{Einflussfunktionen ohne Einbau von Gelenken}}}
Den Einbau von Gelenken bei Stabtragwerken kann man umgehen, wenn man den Aufpunkt auf ein sehr kurzes Element legt. In dem Element selbst ist die Einflussfunktion nur eine N\"{a}herung, aber in allen anschlie{\ss}enden Elementen ist sie deckungsgleich mit der exakten Einflussfunktion.

Man sieht das sehr sch\"{o}n in den Bildern \ref{1GreenF173} und \ref{U37}, wo \"{a}quivalente Knotenkr\"{a}fte $j_i $ die Einflussfunktionen f\"{u}r das Biegemoment bzw. die Querkraft in dem Element erzeugen.

Bei Fl\"{a}chentragwerken gilt das \"{u}brigens nicht.
\\

, denn die Lagerkr\"{a}fte $f_i$ berechnet ein FE-Programm mit dem Prinzip der virtuellen Verr\"{u}ckungen, d.h. durch \"{U}berlagerung der Spannungen mit den Verzerrungen der virtuellen Verr\"{u}ckungen
\begin{align}
f_i = \int_{\Omega} \sigma_{ij} \,\varepsilon_{ij}\,\,d\Omega
\end{align}
Im letzten Punkt der Lager werden die Spannungen zwar unendlich gro{\ss}, aber das Integral, also die \"{U}berlagerung der Spannungen mit den Verzerrungen aus den Knoteneinheitsverformungen ist endlich, s. Abschnitt \ref{Punktlager}.

Dass das so sein muss, sieht man auch im Lastangriffspunkt. Angenommen wir machen das letzte Element in der oberen rechten Ecke immer kleiner, dann werden die Spannungen in dem Element immer gr\"{o}{\ss}er, aber das Arbeitsintegral
\begin{align}
10\,\text{kN} \cdot 1\,\text{m} = \int_{\Omega} \sigma_{ij}\, \varepsilon_{ij}\,\,d\Omega
\end{align}
beh\"{a}lt seinen Wert bei. Wenn der Integrand beschr\"{a}nkt w\"{a}re und die Gr\"{o}{\ss}e des Elementes ging gegen null, dann w\"{u}rde auch das Ergebnis gegen null gehen. Er muss also gerade so gegen unendlich gehen, dass die schrumpfende Fl\"{a}che des Elements durch ein Anwachsen der Spannungen ausgeglichen wird.

Das ist, vereinfacht gesagt der Grund, warum man Lagerkr\"{a}fte auch in singul\"{a}ren Punkten mit finiten Elementen ausrechnen kann.\\

Im Grunde ist es ein Problem der Mechanik. Das Thema Einzelkr\"{a}fte l\"{a}sst sich bei Scheiben nicht durchhalten, weil echte Einzelkr\"{a}fte das Material einer Scheibe zum Flie{\ss}en bringen w\"{u}rden und somit die Kr\"{a}fte einfach vom Bildschirm verschwinden w\"{u}rden.\\

Welches wissenschaftliche Werk hat sich je mit dieser Frage besch\"{a}ftigt? Alle sprechen von klein und infinitesimal klein, nur wenn man nachfragt, 'Wie klein?', dann kommt keine Antwort.\\


Das linke Integral, $\delta A_a$, und das rechte Integral, $\delta A_i$, sind Zahlen, sie geh\"{o}ren also ganz in das Gebiet der Mathematik. Man kann doch nicht die Gleichheit zweier Zahlen mit einem Prinzip der Mechanik 'beweisen'.

Es sind f\"{u}nf (!) Funktionen, $w, \delta w, p, M, \delta M$, deren Integrale hier bilanziert werden. Und der Ingenieur ist k\"{u}hn genug zu behaupten, dass . Wieso sollen sich
\\

Und die Grenze wird schon gleich am Anfang \"{u}berschritten, denn die Gleichgewichtsbedingungen der klassischen Statik beruhen auf dem Begriff der Starrk\"{o}rperbewegungen, also Translationen und Pseudodrehungen.
Wenn man nur genug Stellen nach dem Komma mitnimmt

Begriffe wie klein und infinitesimal klein geh\"{o}ren in das Gebiet der N\"{a}herungsrechnung, aber die Statik l\"{o}st die Gleichungen exakt! Noch die hundertste Stelle nach dem Komma ist exakt.
\\

Man kann doch nicht Beweise dadurch f\"{u}hren, dass man den einen Term $\delta A_a$ nennt und den anderen $\delta A_i$ und dann aus dem Prinzip der virtuellen Verr\"{u}ckungen schlie{\ss}en, dass die beiden gleich sein m\"{u}ssen. Dann ist auch $0 = 1$, denn die \"{a}u{\ss}ere virtuelle Arbeit ist zuf\"{a}llig gerade 0 und die innere virtuelle Arbeit ist 1 und wegen $\delta A_a = \delta A_i$ ist also 0 = 1.\\

 eben nur auf partieller Integration beruht. Es sind genau drei Schritte, die zu diesem Ergebnis f\"{u}hren.
\begin{enumerate}
  \item Die Biegelinie gen\"{u}gt den Gleichungen
\begin{align}
EI\,w^{IV}(x) = p(x) \qquad w(0) = w(l) = 0 \qquad M(0) = M(l) = 0\,.
\end{align}
  \item Die virtuelle Verr\"{u}ckung ist 'stumm' an den Balkenenden, $\delta w(0) = \delta w(l) = 0$.
  \item Die erste Greensche Identit\"{a}t (partielle Integration) garantiert, dass f\"{u}r Funktionen $w \in C^4(0,l)$  und $\textcolor{red}{\delta w } \in C^2(0,l)$ der folgende Ausdruck null ist
\begin{align}
\text{\normalfont\calligra G\,\,}(w,\textcolor{red}{\delta w }) &= \int_0^{\,l} EI\,w^{IV}(x)\,\textcolor{red}{\delta w }\,dx + [V\,\textcolor{red}{\delta w } - M\,\textcolor{red}{\delta w '}]_{@0}^{@l} - \int_0^{\,l} \frac{M\,\textcolor{red}{\delta M}}{EI}\,dx \nn\\
&=  \int_0^{\,l} EI\,w^{IV}(x)\,\textcolor{red}{\delta w }\,dx - \int_0^{\,l} \frac{M\,\textcolor{red}{\delta M}}{EI}\,dx = 0\,,
\end{align}
\end{enumerate}
Diese drei Schritte erlauben den Schluss $\delta A_a = \delta A_i$ und zwar f\"{u}r beliebig gro{\ss}e virtuelle Verr\"{u}ckungen $\delta w(x)$
\begin{align}
\text{\normalfont\calligra G\,\,}(w,\textcolor{red}{\delta w }) &= \int_0^{\,l} EI\,w^{IV}(x)\,\textcolor{red}{\delta w }\,dx - \int_0^{\,l} \frac{M\,\textcolor{red}{\delta M}}{EI}\,dx = 0\,,
\end{align}

Man kann eigentlich nur Mitleid mit den Studenten haben, die so etwas verstehen m\"{u}ssen.

Die Folge ist eigentlich, dass die Statik durch solche 'Lehrs\"{a}tze' in eine unn\"{o}tige Schieflage kommt. Der Aberglaube, so ist man versucht zu sagen, dominiert die Statik.\\
Alle Biegelinien vom Typ $w = a\,x + b$ sind 'Null-Momenten-Linien' und deswegen m\"{u}ssen die \"{a}u{\ss}eren Kr\"{a}fte orthogonal zu all diesen Starrk\"{o}rperbewegungen sein.
\\

Mit diesen Zitaten wollen wir darauf aufmerksam machen, dass das {\em Rechnen\/} in der Statik mathematischen Gesetzen unterliegt---und nur mathematischen Gesetzen. Castigliano hat bei der Herleitung seines Satzes vom Fachwerk (wo der Satz gilt) auf den elastischen K\"{o}rper (wo er nicht gilt) geschlossen. Er hat vergessen, dass man mathematische Probleme nicht mit den Gesetzen der Mechanik beweisen kann

Mathematik und Mechanik sind also zwei getrennte Gebiete und man kann nicht mit Mitteln der Mechanik einen mathematischen Beweis f\"{u}hren.
, wie sie sich in der ersten Greenschen Identit\"{a}t darstellt
\begin{align}
\text{\normalfont\calligra G\,\,}(w,\textcolor{red}{\hat{w}}) = \int_0^{\,l} EI\,w^{IV}(x)\,\textcolor{red}{\hat{w}}\,dx + [V\,\textcolor{red}{\hat{w}} - M\,\textcolor{red}{\hat{w}'}]_{@0}^{@l} - \int_0^{\,l} \frac{M\,\textcolor{red}{\hat{M}}}{EI}\,dx = 0\,,
\end{align}
Die Starrk\"{o}rperbewegungen der Balkenstatik sind alle Biegelinien $\hat{w} = a\,x + b$ mit 'Null-Momenten', $- EI\,\hat{w}'' = 0$, und die erste Greensche Identit\"{a}t schreibt zwingend vor, dass die \"{a}u{\ss}eren Kr\"{a}fte orthogonal sein m\"{u}ssen, zu all diesen Funktionen $\hat{w} = a\,x + b$. Das ist gerade die Summe der vertikalen Kr\"{a}fte
\begin{align}
\text{\normalfont\calligra G\,\,}(w,\textcolor{red}{1}) = \int_0^{\,l}  EI\,w^{IV}(x)\cdot\textcolor{red}{1}\,dx + V(l) \cdot\textcolor{red}{1} - V(0)\cdot\textcolor{red}{1} = 0
\end{align}
bzw. die Summe der Momente um den linken Anfangspunkt $x = 0$
\begin{align}
\text{\normalfont\calligra G\,\,}(w,\textcolor{red}{x}) = \int_0^{\,l} EI\,w^{IV}(x)\,\textcolor{red}{x}\,dx + V(l) \cdot\textcolor{red}{l} - M(l) \cdot\textcolor{red}{1} + M(0) \cdot\textcolor{red}{1}= 0\,,
\end{align}
Genauso ist ein Fachwerk im Gleichgewicht, $\vek K\,\vek u = \vek f$, (das ist jetzt die nicht-reduzierte Steifigkeitsmatrix, inklusive den Lagerknoten), wenn die Knotenkr\"{a}fte $\vek f$ orthogonal sind zu allen Starrk\"{o}rperbewegungen $\vek u_0$ des Fachwerks und die Starrk\"{o}rperbewegungen sind Translationen und Pseudodrehungen
\begin{align}
\vek u_0 = \vek a + \vek \omega \times \vek x
\end{align}
Mittels partieller Integration zeigt man, dass das Arbeitsintegral
Wer sagt denn, dass $\delta A_i$ die \"{U}berlagerung der Biegemomente ist? Vielleicht gibt es ja noch eine genauere Formulierung, wo dann auch noch die Durchbiegungen \"{u}berlagert werden
\begin{align}
\delta A_i = \int_0^{\,l} (\frac{M\,\delta M}{EI} + w\,\delta\,w)\,dx
\end{align}
Warum nicht so? Weil dann die Glg. X nicht mehr stimmt, die im \"{U}brigen auf partieller Integration beruht.

Autoren sind virtuos in der Handhabung der Energieprinzipe, aber uns scheinen die Beweise oft nur darauf zu beruhen, dass der Autor den einen Term $\delta A_a$ nennt und den anderen $\delta A_i$ und weil das Prinzip der virtuellen Verr\"{u}ckung besagt, dass die virtuellen \"{a}u{\ss}eren Arbeiten und inneren Arbeiten gleich sein m\"{u}ssen, folgt $\delta A_a = \delta A_i$. Beweis gelungen!

Aber dann beweisen wir auch, dass 0 dasselbe ist, wie 1. Denn die \"{a}u{\ss}ere virtuelle Arbeit ist zuf\"{a}llig gerade 0 und die innere virtuelle Arbeit ist 1 und wegen $\delta A_a = \delta A_i$ ist 0 = 1.

Das ist nat\"{u}rlich jetzt spa{\ss}haft gemeint, aber uns kommen viele 'Beweise' in den Statikb\"{u}chern so vor, weil st\"{a}ndig auf die Energieprinzipe Bezug genommen wird. Aus dem und dem Prinzip folgt, dass die beiden Integrale gleich sein m\"{u}ssen, etc.

Nur, Castiglianos Theorem gilt nicht f\"{u}r Scheiben und auch nicht f\"{u}r elastische K\"{o}rper, weil bei diesen Bauteilen die Verformungen aus einer Einzelkraft $P = 1$, das ist die Hilfsgr\"{o}{\ss}e mit der Castigliano operiert, unendlich gro{\ss} sind.


Das Problem ist das folgende...

nicht unter einen Generalverdacht stellen...

Etwas 'wackeliger' wird die Geschichte, wenn wir auf Wikipedia den Satz von Castigliano nachlesen:

{\em Die partielle Ableitung der in einem linear elastischen K\"{o}rper gespeicherten Form\"{a}nderungsenergie nach der \"{a}u{\ss}eren Kraft ergibt die Verschiebung $v_k$ des Kraftangriffspunktes in Richtung dieser Kraft\/}.

Aber die Form\"{a}nderungsenergie eines elastischen K\"{o}rpers, der eine Einzelkraft tr\"{a}gt, ist unendlich gro{\ss}, es macht also keinen Sinn eine Ableitung berechnen zu wollen und auch die Verschiebung $v_k$ des Kraftangriffspunktes ist unendlich gro{\ss}.

Mit diesen Zitaten wollen wir darauf aufmerksam machen, dass das {\em Rechnen\/} in der Statik mathematischen Gesetzen unterliegt---und nur mathematischen Gesetzen. Castigliano hat bei der Herleitung seines Satzes vom Fachwerk (wo der Satz gilt) auf den elastischen K\"{o}rper (wo er nicht gilt) geschlossen. Er hat vergessen, dass man mathematische Probleme nicht mit den Gesetzen der Mechanik beweisen kann. \\

In der ersten Greenschen Identit\"{a}t paart man zwei gleichberechtigte Funktionen, und es h\"{a}ngt nur vom Geschick des Aufstellers ab, die zweite Funktion so zu w\"{a}hlen,
etwa $\textcolor{red}{\delta w(x)}= \sin x$, dass er die Informationen bekommt, die er sucht.

Aber der Sinus ist nat\"{u}rlich genauso real (und nicht 'virtuell') wie die Biegelinie $w$ des Balkens, so dass beide in der Formulierung
\begin{align}
G(w,\textcolor{red}{\sin x}) = 0\,.
\end{align}
gleichberechtigt nebeneinander stehen.

%----------------------------------------------------------------------------------------------------------
\begin{figure}[tbp]
\includegraphics[width=0.7\textwidth]{\Fpath/U58}
%\caption{Stablement und virtuelle Verr\"{u}ckung} \label{U58}
\end{figure}%
%----------------------------------------------------------------------------------------------------------

Aber bei Fachwerken interessieren uns nicht so sehr die Ansatzfunktionen, als vielmehr die einzelnen Stabelemente und ihre Steifigkeitsmatrizen aus denen wir die Gesamtsteifigkeitsmatrix aufbauen.


Das ist so, wie wenn man die Biegelinie $w(x)$ eines Balkens unter einer Gleichlast berechnen soll und als N\"{a}herung eine Sinuswelle w\"{a}hlt, $w_h(x) = \sin (\pi x/l)$. Eingesetzt in die Balkengleichung ergibt sich
\begin{align}
EI\,\frac{d^4}{dx^4} \sin \frac{\pi x}{l} = \frac{\pi^4}{l^4}\,\sin \frac{\pi x}{l} = p_h(x)
\end{align}
als der Lastfall, den man eigentlich gel\"{o}st hat.
\\
Die Kr\"{a}fte $\vek p_h$ und $\vek s_h$ werden von einem FE-Programm nicht ausgegeben, weil sie in der Regel so 'merkw\"{u}rdig' aussehen, s. Bild  \ref{U28}, dass ein Anwender, der mit der Theorie der finiten Elemente nicht vertraut ist, irritiert w\"{a}re.\\

\begin{remark}
Bei einer Scheibe ist die Belastung ein Vektorfeld $\vek p = \{p_x,p_y\}^T$ und auch die Verformungen messen sich nach horizontalen und vertikalen Anteilen, $\vek u = \{u_x, u_y\}^T$. Daher ist die virtuelle \"{a}u{\ss}ere Arbeit ein Skalarprodukt wie
\begin{align}
\delta A_a = \int_{\Omega} \vek p^T\,\vek  \delta \vek u \,d\Omega = \int_{\Omega} (p_x\,\delta u_x + p_y\,\delta u_y)\,d\Omega\,.
\end{align}
\end{remark}


%%%%%%%%%%%%%%%%%%%%%%%%%%%%%%%%%%%%%%%%%%%%%%%%%%%%%%%%%%%%%%%%%%%%%%%%%%%%%%%%%%%%%%%%%%%%%%%%%%%%
%{\textcolor{blau2}{\section{Die Ambivalenz der finiten Elemente}}}

Zuerst war die Methode der finiten Elemente nur ein numerisches Werkzeug, dann ist aber mit ihr ein neuer L\"{o}sungsbegriff in die Statik gekommen ist. Der Begriff der Variationsl\"{o}sung hat zunehmend in der Statik den klassischen L\"{o}sungsbegriff verdr\"{a}ngt und die sogenannten \"{a}quivalente Knotenkr\"{a}fte, die ja eigentlich nur ein bequemes Werkzeug  zur Umsetzung der Numerik sind, haben zunehmend in der Statik ein Eigenleben entwickelt und werden wie reale statische Objekte behandelt.

Die Wandscheibe in Bild \ref{U28} st\"{u}tzt sich auf zwei Punktlager ab, um die Punktlast zu tragen. Vom Standpunkt der Mathematik und der Mechanik aus ist das ein schlecht gestelltes Problem, weil, wenn wir klassisch denken, die Spannungen und Verformungen der Scheibe in dem Lastangriffspunkt bzw. den Lagerpunkten unendlich gro{\ss} werden. Je feiner man das Netz macht, um so gr\"{o}{\ss}er werden die 'Ausrei{\ss}er'.

Der Ingenieur denkt aber anders, er denkt zun\"{a}chst in \"{a}quivalenten Knotenkr\"{a}ften. Er will erst einmal wissen, wie sich die Belastung auf die Lager verteilt.

Weil die Scheibe statisch bestimmt gelagert ist, hat die Tatsache, dass die Spannungen unendlich gro{\ss} werden, keinen Einfluss auf die Lagerkr\"{a}fte.
Finite Elemente f\"{u}hren auf das Gleichungssystem
\begin{align}
\vek K\,\vek  u = \vek f\,.
\end{align}
Streichen wir nun nicht die Zeilen und Spalten, die zu gesperrten Freiheitsgraden $u_i = 0$ geh\"{o}ren, dann muss der Vektor $\vek f$ der \"{a}quivalenten Knotenkr\"{a}fte, der jetzt auch die Lagerkr\"{a}fte umfasst, orthogonal sein zu allen Knotenvektoren $\vek u_0$, die zu Translationen und (Pseudo)Rotationen der Scheibe geh\"{o}ren, also muss gelten
\begin{align}
- f_1 + 10 = - 10 + 10 = 0 \qquad f_2 + f_4 = 10 - 10 = 0\,,
\end{align}
wenn wir die Numerierung in Bild \ref{U28} c zu Grunde legen.

Bei statisch unbestimmten Tragwerken wird im allgemeinen die Verteilung der Belastung auf die Lager von der Feinheit des Netzes abh\"{a}ngen.


Zuerst war die Methode der finiten Elemente nur ein numerisches Werkzeug, dann ist aber mit ihr ein neuer L\"{o}sungsbegriff in die Statik gekommen ist. Der Begriff der Variationsl\"{o}sung hat zunehmend in der Statik den klassischen L\"{o}sungsbegriff verdr\"{a}ngt und die sogenannten \"{a}quivalente Knotenkr\"{a}fte, die ja eigentlich nur ein bequemes Werkzeug  zur Umsetzung der Numerik sind, haben zunehmend in der Statik ein Eigenleben entwickelt und werden wie reale statische Objekte behandelt.


\begin{align}
10\,\text{kNm} &= \int_{\Omega} \vek p_h^T\,\vek \Np_i\,\,d\Omega + \int_{\Gamma} \vek s_h^T\,\vek \Np_i\,ds \nn \\
&= [F/L^2] \cdot [L] \cdot [L^2] + [F/L] \cdot [L] \cdot [L]\,.
\end{align}

Es gibt nat\"{u}rlich noch viele andere Werte, die eventuell interessieren k\"{o}nnen, aber zu ihrer Berechnung muss man dann auf Einflussfunktionen oder andere Techniken zur\"{u}ckgreifen. Nur die kanonischen Werte bekommt man sozusagen auf dem Tablett (der ersten Greenschen Identit\"{a}t) serviert.


%%%%%%%%%%%%%%%%%%%%%%%%%%%%%%%%%%%%%%%%%%%%%%%%%%%%%%%%%%%%%%%%%%%%%%%%%%%%%%%%%%%%%%%%%%%%%%%%%%%
%{\textcolor{blau2}{\section{Der Pfad vom Aufpunkt zum Lastangriffspunkt}}}
So viel Weg, wie von der Spreizung des Aufpunkts im Lastangriffspunkt ankommt, ist gleich dem Einfluss, den die Last auf die Spannung im Aufpunkt hat. Somit stellt sich die Frage, wie kommunizieren die beiden miteinander. Oder anders gefragt: Wie und auf welchen Wegen erreicht die Spreizung den Fusspunkt der Last.

Um den lokalen Anteil
Um die Einflussfunktionen f\"{u}r das Biegemoment in dem Punkt $x$ zu berechnen, unterteilen wir den Balken in zwei Elemente, der L\"{a}nge $\ell_a = x$ bzw. $\ell_b = \ell - x$, die in dem Aufpunkt gelenkig miteinander verbunden sind. Mit den Bezeichnungen des Bildes \ref{U24} ergibt sich dann das Gleichungssystem zu
\begin{align}
\left[ \barr {r @{\hspace{4mm}}r @{\hspace{4mm}}r
}
      k_{11} & k_{12} & k_{13}  \\
      k_{21} & k_{22} & 0  \\
      k_{31} & 0 & k_{33}
    \earr \right] \left[ \barr {r} u_1 \\ u_2 \\ u_3 \earr \right] =  \left[ \barr {r} 0 \\ -M \\ M \earr \right]\,,
\end{align}
wobei
\begin{align}
k_{11} = \frac{12}{l_a^3} + \frac{12}{l_b^3} \quad k_{12} = - \frac{6\,EI}{l_a^2} \quad k_{13} =  - \frac{6\,EI}{l_b^2} \quad k_{22} = \frac{4\,EI}{l_a}\quad k_{33} = \frac{4\,EI}{l_b}\,.
\end{align}
Zuerst setzt man $M = 1$ und skaliert dann anschlie{\ss}end $M$ so, dass sich die gew\"{u}nschte Spreizung $\tan\,\Np_l + \tan\,\Np_r = - u_2 + u_3 = 1$ ergibt.

Man k\"{o}nnte die Spreizung auch mit einer Knotenkraft $f_1 = P$ statt mit den beiden Momenten erzielen, aber dann w\"{a}re $A_{2,1}$ nicht null. Das ist zwar auch kein Problem, aber so ist es einfacher.


Bei der Berechnung eines Durchlauftr\"{a}gers mit dem Drehwinkelverfahren werden erst alle Knoten festgehalten und dann die Knoten einzeln gel\"{o}st und ausgeglichen. Am Ende des Ausgleichs kennt man die Verdrehungen der Knoten, das Tragwerk ist, wie man sagt, geometrisch bestimmt.
Einem solchen Knotenausgleich folgt dann noch ein zweiter Schritt, bei dem die Schnittkr\"{a}fte zwischen den Knoten berechnet werden.
Dabei k\"{o}nnen aber die einzelnen Felder abschnittsweise wie ein fest eingespannter Balken behandelt werden, was die Berechnung sehr vereinfacht.

\"{U}bertragen auf Platten bedeutet das: wenn man wei{\ss}, wie sich die R\"{a}nder einer Platte verdrehen und durchbiegen, dann kann man die Schnittgr\"{o}{\ss}en im Innern der Platte berechnen. Das ist die L\"{o}sungsstrategie der Methode der Randelemente.
\\

Die Regel ist also: In der ersten Greenschen Identit\"{a}t $\text{\normalfont\calligra G\,\,}(w,\textcolor{red}{\delta w}) = \delta A_a - \delta A_i = 0$ werden bei der Formulierung des Teils $\delta A_a$ Kr\"{a}fte mit Wegen gepaart und die Kr\"{a}fte kommen von der ersten Funktion und die Wege von der zweiten, wie wir das in (\ref{Eq61}) und (\ref{Eq62}) angedeutet haben.

Mit den Werten
\begin{align}
EA = 1.0 \cdot 10^6 \,\text{kN},\,EA_c = 2 \cdot EA\qquad  l = 3,\,l_e = 1\, \qquad p = 1000\,\text{kN}/\text{m}
\end{align}
ergibt sich
\begin{align}
u_1^c &= 2.52\cdot 10^{-3}\text{m},\,  u_2^c = 3.25 \cdot 10^{-3}\text{m},\,  u_3^c = 3.74 \cdot 10^{-3}\text{m} \\
 f^+ &= \pm (3.25 - 2.52)\cdot 10^{-3}\,\text{m} \,1.0\cdot 10^6\,\text{kN} = 750\,\text{kNm}
\end{align}
und diese Belastung, s. Bild \ref{U99} c, ergibt an dem urspr\"{u}nglichen Stab dieselbe Verformung.

%-----------------------------------------------------------------
\begin{figure}[tbp]
\centering
\includegraphics[width=0.9\textwidth]{\Fpath/S27}

\label{S27}
%
\end{figure}%
%-----------------------------------------------------------------


Die Gr\"{u}nde, warum Homogenisierungsmethoden erfolgreich sind, sind also:
\begin{itemize}
  \item Die $f_i^+$ sind Gleichgewichtskr\"{a}fte.
  \item Die Fernwirkung der $f_i^+$ tendieren gegen null.
\end{itemize}

Wir argumentieren wie folgt: Steifigkeits\"{a}nderungen k\"{o}nnen durch die Wirkung von Gleichgewichtskr\"{a}ften $f^+$ beschrieben werden. Wenn das betroffene Element nicht zu gro{\ss} ist, dann liegen die Angriffspunkte dieser Kr\"{a}fte relativ dicht beieinander. Die Effekte, die diese Kr\"{a}fte in dem Trag\-werk  bewirken, h\"{a}ngt nun von der Gestalt der Einflussfunktionen ab, die zu dem Effekt geh\"{o}rt, den wir studieren wollen.\\

%%%%%%%%%%%%%%%%%%%%%%%%%%%%%%%%%%%%%%%%%%%%%%%%%%%%%%%%%%%%%%%%%%%%%%%%%%%%%%%%%%%%%%%%%%%%%%%%%%%
%{\textcolor{blau2}{\section{$N$-, $V$- oder $M$-Gelenke}}}
In einem $N$-Gelenk springt die L\"{a}ngsverschiebung $u$, in einem Querkraftgelenk springt die Durchbiegung und in einem Momentengelenk springt die Tangente an die Biegelinie.


Aber wie geht das---$\delta(x)$ ist beliebig klein, also frei w\"{a}hlbar und die $1$ ist konstant? Dann muss doch $\eta(x)$ von $\delta(x)$ abh\"{a}ngen oder \"{a}ndert sich mit dem $\delta(x)$ auch die 1? Dann sollte der Autor doch besser $\Delta\,\Np$ schreiben, statt 1.


Wenn der oben zitierte Autor einen Beweis gef\"{u}hrt h\"{a}tten, dann h\"{a}tte er gemerkt, dass die virtuellen Verr\"{u}ckungen {\em beliebig gro{\ss}\/} sein k\"{o}nnen und dass alle Einflussfunktionen (= kinematische Ketten) Ausschl\"{a}ge aufweisen, die alles andere als klein sind, wie man in jedem Tabellenwerk nachlesen kann.


Nat\"{u}rlich ist die N\"{a}he der Mathematik zur Mechanik f\"{u}r die Mathematik immer sehr fruchtbar gewesen (wie umgekehrt auch), aber die N\"{a}he liefert Ideen, f\"{u}hrt zu Vermutungen, zu neuen mathematischen S\"{a}tzen, die aber mathematisch bewiesen werden m\"{u}ssen.

Aber dieses Vermischen von Mechanik und Mathematik zieht sich durch die ganze Statik. Didaktisch ist das sicherlich sinnvoll, aber auf der anderen Seite darf man ein Resultat wie $0 = 1$ nicht automatisch deswegen f\"{u}r richtig halten, weil $\delta A_a = 0$ ist und $\delta A_i = 1$ und ja das Prinzip der virtuellen Verr\"{u}ckungen in der Statik gilt.\\


Wenn $3\,x = 12$ ist, dann kann man die Gleichung mit {\em beliebig kleinen oder gro{\ss}en \/} Zahlen $\delta u$ multiplizieren
\begin{align}
\delta u \cdot 3\,x = 12 \cdot \delta u\,.
\end{align}
Warum soll das nur f\"{u}r infinitesimal kleine $\delta u$ richtig sein?\\

%%%%%%%%%%%%%%%%%%%%%%%%%%%%%%%%%%%%%%%%%%%%%%%%%%%%%%%%%%%%%%%%%%%%%%%%%%%%%%%%%%%%%%%%%%%%%%%%%%%
%{\textcolor{blau}{\section{Finite Elemente und Projektion}}}\index{Projektion}

Wenn man einen Vektor $\vek v$, der schr\"{a}g in den Raum zeigt, auf die $x-y$-Ebene projiziert, dann entsteht sein Schattenbild $\vek v_h$.
Um von $\vek v_h$ wieder zur Spitze von $\vek v$ zu kommen, m\"{u}ssen wir einen senkrechten Vektor $\vek e$ zu $\vek v_h$ addieren
\begin{align}
\vek v = \vek v_h + \vek e\,.
\end{align}
Bemerkenswert hieran ist, dass der Vektor  $\vek e$ senkrecht auf der  $x-y$-Ebene steht. Das ist gleichbedeutend damit, dass es in der Ebene keine bessere N\"{a}herung f\"{u}r  $\vek v$ gibt, als den Vektor $\vek v_h$. Ebenso gilt, dass alle Vektoren, die \"{u}ber dem Vektor $\vek v$ liegen, in dem Sinne, dass das Lot von ihrer Spitze auf die $x-y$-Ebene in denselben Punkt f\"{a}llt, die Spitze von $\vek v_h$, denselben Schatten haben. Die Sonne, wenn sie denn genau von oben scheint, erzeugt nur einen Schatten f\"{u}r alle diese Vektoren.

ragt, auf die drei Koordinaten ebenen projiziert, dann kann man den Vektor aus seinen 'Schatten' rekonstruieren. Auch wenn der Bauzeichner Risse anfertigt, dann sind das Projektionen auf die verschiedenen Koordinatenebenen.

\"{A}hnlich gehen die finiten Elemente vor. Sie projizieren die exakte L\"{o}sung auf den Ansatzraum $V_h$. Die FE-L\"{o}sung $u_h$ ist der Schatten
der exakten L\"{o}sung. Nur wird der Abstand anders gemessen, als bei Vektoren.

Die FE-L\"{o}sung ist die Funktion $u_h$ in $V_h$
Die Koordinaten eines Vektors $\vek v$ sind die Projektionen des Vektors auf die drei Koordinatenachsen
\begin{align}
x_i = \vek v^T\,\vek e_i\,.
\end{align}
Wenn man den Vektor in die Ebene projiziert,
dan kann man den Vektor aus den drei Richtungen wieder generieren
\begin{align}
\vek  v = x_1 \, \vek e_1 + x_2 \, \vek e_2 + x_3 \, \vek e_3
\end{align}\\

 Das ist kein Widerspruch zu der Tatsache, dass die $\vek p_i$ Gleichgewichtskr\"{a}fte sind, denn durch den Schnitt wird das Gleichgewicht der $\vek p_i$ gest\"{o}rt.

Die Gegenkr\"{a}fte ziehen und dr\"{u}cken an den Elementen direkt am Rand und diese stabilisieren sich \"{u}ber ihre Festhaltung auf dem Rand.
Weil die $\vek p_i$ Gleichgewichtskr\"{a}fte sind, folgt, dass man die Lagerkr\"{a}fte nur braucht, um die Elemente direkt neben dem Rand festzuhalten.

Das ist nicht so \"{u}berraschend, wie es sich vielleicht anh\"{o}rt.

In Gedanken schneide man das innere Netz einer Scheibe heraus.

Symbolisch ausgedr\"{u}ckt hat man
\begin{align}
\int_0^{\,l} p_1\,u_{2}\,dx = \int_0^{\,l} p_2\,u_{1}\,dx \qquad \Rightarrow \qquad \int_0^{\,l} p_1\,\underset{\uparrow}{u_2^h}\,dx = \int_0^{\,l} p_2\,\underset{\uparrow}{u_{1@h}}\,dx
\end{align}


Man beachte, dass jetzt acht Funktionen im Spiel sind, vier Verschiebungen und vier Lasten
\begin{align}
u_1, u_2, u_{1@h}, u_2^h, \qquad  p_1, p_2, p_{1@h}, p_{2@h}\,,
\end{align}
und somit zwei 'normale' Paarungen m\"{o}glich sind
\begin{align}
\int_0^{\,l} u_1 \,p_2 dx = \int_0^{\,l} u_2\,p_1\,dx \qquad \int_0^{\,l} u_{1@h} \,p_{2@h}\,dx = \int_0^{\,l} u_2^h\,p_{1@h}\,dx\,,
\end{align}
aber zus\"{a}tzlich nun auch die neue Paarung
\begin{align}
\int_0^{\,l} p_1\,u_2^h\,dx = \int_0^{\,l} p_2\,u_{1@h}\,dx\,.
\end{align}\\

Die Einflussfunktion f\"{u}r die horizontale Verschiebung eines Knotens ist die Reaktion der Scheibe auf eine horizontal gerichtete Einzelkraft der Gr\"{o}{\ss}e $P = 1$, also ein Dirac-Delta, siehe Bild  \ref{U129} a. Wenn man diesen Lastfall mit finiten Elementen l\"{o}st, und sich den FE-Lastfall anschaut, der zu diesem Lastfall geh\"{o}rt, dann erh\"{a}lt man das Bild \ref{U129} b. Wir nennen diese Kr\"{a}fte $\delta_h(\vek y,\vek x)$.

W\"{a}hrend das Dirac-Delta symbolisch zu nehmen ist, ist das gen\"{a}hrte Dirac-Delta eine Funktion, die man plotten kann, die man auf dem Bildschirm darstellen kann, s. Bild \ref{U129} b.

%%%%%%%%%%%%%%%%%%%%%%%%%%%%%%%%%%%%%%%%%%%%%%%%%%%%%%%%%%%%%%%%%%%%%%%%%%%%%%%%%%%%%%%%%%%%%%%%%%%
%{\textcolor{blau2}{\section{Matrizenstatik}}}\index{Matrizenstatik}\label{Matrizenstatik}
Zur Vorbereitung auf die rechnerische Umsetzung der Berechnung von Einflussfunktionen mit finiten Elementen wollen wir kurz die Grundlagen der Matrizenstatik rekapitulieren.

Das elementarste Beispiel f\"{u}r Matrizenstatik ist die Berechnung eines Fachwerkes mit finiten Elementen, die auf das Gleichungssystem
f\"{u}hrt.

Die Einheitsverformungen $\Np_i(x)$ eines Fachwerkes haben am Ort von $u_i$ und in Richtung des Freiheitsgrades $u_i$ den Wert Eins, $u_i = 1$, und den Wert Null an allen anderen Stellen,  $u_j = 0$, und somit ist die Verformungsfigur
\begin{align}
u(x) = \sum_i u_i\,\Np_i(x)
\end{align}
die Summe der---mit den $u_i$ gewichteten---Einheitsverformungen $\Np_i(x)$.

Die Matrix $\vek K $ in (\ref{Eq67}) ist die sogenannte reduzierte Steifigkeitsmatrix, wenn man in der nicht reduzierten Matrix $ \vek K_S$ die Zeilen und Spalten streicht, die zu gesperrten Freiheitsgraden geh\"{o}ren.

Die nicht reduzierte Matrix $\vek K_S $, die Systemmatrix, ist singul\"{a}r, weil kein Lager da ist, um das Fachwerk festzuhalten. Das kommt erst beim \"{U}bergang von $\vek K_S $ zu $\vek K$. Die Vektoren $\vek u_0$, die im Kern der Matrix $ \vek K_S $ liegen, sind gerade die Starrk\"{o}rperbewegungen des ungelagerten Fachwerks. Sie haben also die Gestalt
\begin{align}
\vek u_0 = \vek a + \vek b \times \vek x\,.
\end{align}
Wobei der Vektor $\vek a $ eine Translation beschreibt und der Vektor $\vek b $ die Drehachse (samt Drehwinkel, $\Np = |\vek b|$) darstellt, um den das Fachwerk gedreht wird.


Im Kern liegen bedeutet, dass die Vektoren $\vek u_0 $ auf den Nullvektor $\vek 0$ abgebildet werden, $\vek K_S\,\vek u_0 = \vek 0$. Aus der Identit\"{a}t
\begin{align}
B(\vek u,\vek u_0) = \vek u_0^T\,\vek K\,\vek u - \vek u^T\,\vek K\,\vek u_0 = 0
\end{align}
folgt, dass der Vektor $\vek f$ der Knotenkr\"{a}fte zu den Vektoren $\vek u_0 $ orthogonal ist
\begin{align}
\vek u_0^T \,\vek f = 0\,.
\end{align}
Damit ist garantiert, dass die Vektoren $\vek f $ den Gleichgewichtsbedingungen gen\"{u}gen, denn w\"{a}hlt man als Vektor $\vek u_0$ eine Translation in horizontaler oder vertikaler Richtung, dann ist das das Gleichgewicht der Knotenkr\"{a}fte in horizontaler bzw. vertikaler Richtung
\begin{align}
\sum H = 0 \qquad \sum V = 0
\end{align}
 und w\"{a}hlt man als Vektor $\vek u_0 $ eine Pseudorotation, dann entspricht dies der Momentensumme
\begin{align}
\sum M = 0
\end{align}
um den Nullpunkt des Koordinatensystems.

Wegen $\vek K\,\vek u_0 = \vek 0$ muss im \"{U}brigen jede Zeile orthogonal zu den Vektoren $\vek u_0$ sein, also 'im Gleichgewicht' sein.\\


\hspace*{-12pt}\colorbox{hellgrau}{\parbox{0.98\textwidth}{Die Zeilen (= Spalten) einer nicht reduzierten Steifigkeitsmatrix sind orthogonal zu den Vektoren $\vek u_0 = \vek  a + \vek b \times \vek x$}}\\

Insbesondere, wenn es nur einen Typ von Starrk\"{o}rperbewegung gibt, wie bei einem Stab, $\vek u_0 = \vek a$ (eine Verschiebung nach links oder rechts), ist in jeder Zeile die Summe der Eintr\"{a}ge null.\\

Nach der $h$-Vertauschungsregel gilt jedoch
\begin{align}
S_h  = \int_{\Omega} G_h(\vek y,\vek x)\,p(\vek y)\,d\Omega_{\vek y} =  \int_{\Omega} G(\vek y,\vek x)\,p_h(\vek y)\,d\Omega_{\vek y}\,,
\end{align}
was besagt, dass $S_h$ mit der Knotenkraft $f_i$ im Ausdruck identisch ist.

Bevor wir dies diskutieren, kehren wir noch einmal zu dem Fachwerk zur\"{u}ck.\\

%\subsubsection*{Einflussfunktion f\"{u}r eine Lagerkraft---statisch}
Die Einflussfunktion f\"{u}r eine Lagerkraft ist die Verformungsfigur des Fachwerks, wenn der Lagerknoten um 1 Meter (das ist rein rechnerisch) ausgelenkt wird, s. Bild \ref{U85}.
%-----------------------------------------------------------------
\begin{figure}[tbp]
\centering
\includegraphics[width=1.0\textwidth]{\Fpath/U87}
%\caption{Durchlauftr\"{a}ger, \textbf{a)} FE-Modell, \textbf{b)} Einflussfunktion f\"{u}r die Lagerkraft, \textbf{c)} Ansatzfunktionen und Einheitsverformung des Lagerknotens} \label{U87}
%
\end{figure}%
%-----------------------------------------------------------------
Der gesperrte Lagerknoten entspreche dem Freiheitsgrad $u_k$. Der Ingenieur behandelt das Problem wie folgt: Er bringt die zu $u_k$ geh\"{o}rige Spalte $\vek s_k$ der Steifigkeitsmatrix $\vek K_{+1}$ auf die rechte Seite und l\"{o}st das System
\begin{align}
\vek K\,\vek g = - \vek s_k\,.
\end{align}
Die Matrix $\vek K$ ist die reduzierte Steifigkeitsmatrix und $\vek K_{+1}$ ist der Vorg\"{a}nger, bei dem die zu $u_k$ geh\"{o}rige Spalte noch nicht gestrichen wurde.

Die Verformungsfigur
\begin{align}
g(x,y) = \sum_i g_{i}\,\Np_i(y)
\end{align}
ist dann die gesuchte Einflussfunktion und
\begin{align}
R_k = \vek g^T\,\vek f
\end{align}
ist die zu einem Lastfall $\vek f$ geh\"{o}rige Lagerkraft $R_k$ in dem festgehaltenen Knoten in Richtung des Freiheitsgrades $u_k$.

\begin{remark}
Die Eintr\"{a}ge in Spalte $k$ der nicht reduzierten Steifigkeitsmatrix ($k$ = gesperrter Freiheitsgrad des Lagers) sind die Integrale, s. Bild \ref{U87} c, der Einheitsverformungen
\begin{align}
k_{ik} = \int_0^{\,l} EI\,\Np_i''\,\Np_k''\,dx = \delta A_i(\Np_i,\Np_k) = \delta A_a(\Np_i,\Np_k) = \text{Lagerkraft aus $\Np_i$}
\end{align}
\end{remark}

%\subsubsection*{Einflussfunktion f\"{u}r eine Lagerkraft---mathematisch}
Versuchen wir dasselbe Ergebnis mathematisch herzuleiten: Eine Lagerkraft $R_k$ (in Richtung des gesperrten Freiheitsgrades $u_k = 0$) ist ein Funktional
\begin{align}
R_k = \mathcal{J}(\vek u)\,,
\end{align}
und die Einflussfunktion f\"{u}r $R_k$, s. Bild \ref{U87}, erh\"{a}lt man, wenn man als Knotenkr\"{a}fte $j_{i}$ die Lagerkr\"{a}fte der Einheitsverformungen wirken l\"{a}sst
\begin{align}
j_{i} = R_k(\Np_i)\,.
\end{align}
Der Vektor der Knotenverschiebungen $\vek g$ (das sind die $u_i$ der Einflussfunktion) ist dann die L\"{o}sung des Systems
\begin{align}
\vek K\,\vek g = \vek j\,.
\end{align}
Wenn also in einem Lastfall $\vek f$ das Fachwerk die Knotenverformungen $\vek u$ aufweist, dann ist
\begin{align}
R_k = \sum_i\,u_i\,j_{i} = \sum_i\,u_i\,R_k(\Np_i) = \vek u^T\,\vek j = \vek j^T\,\vek K^{-1}\,\vek f = \vek g^T\,\vek f
\end{align}
die Lagerkraft $R_k$ in diesem Lastfall.

Die Berechnung der Zahlen $j_i = R_k(\Np_i)$ kann man sich wie folgt zurechtlegen. Es sei $\vek K_{+1}$ die oben erw\"{a}hnte Steifigkeitsmatrix des Fachwerks.

Die Kr\"{a}fte, die n\"{o}tig sind, um die Verformung $u_k = 1$ und $u_i = 0$ sonst, zu erzeugen, sind identisch mit der Spalte $\vek s_k \times (-1)$ der Matrix $\vek K_{+1}$. (Minus, weil wir den Vektor auf die rechte Seite bringen). Wegen des Satzes von Betti sind die Eintr\"{a}ge $s_{ki}$ (Zeile $i$) in dem Vektor $\vek s_k$ gleich die Lagerkr\"{a}fte $R_k(\Np_i)$ und somit ist der Vektor $\vek g_k$ die L\"{o}sung des Systems
\begin{align}
\vek K\,\vek g = - \vek s_k\,,
\end{align}
was dasselbe Ergebnis wie zuvor ist.\\

Platten k\"{o}nnen schubstarr (Kirchhoffplatte) oder schubweich (Reissner-Mindlin) gerechnet werden. Die Kirchhoffplatte ist die Erweiterung des klassischen Biegebalkens (Euler-Bernoulli) in die $y$-Richtung. Die Unterschiede in den Ergebnissen sind relativ gering. Wir wollen uns daher im Folgenden mit der Kirchhoffplatte besch\"{a}ftigen.


%-----------------------------------------------------------------
\begin{figure}[tbp]
\centering
\includegraphics[width=0.8\textwidth]{\Fpath/U85}
%\caption{Einflussfunktion f\"{u}r eine Lagerkraft in einem Fachwerk} \label{U85}
%
\end{figure}%
%-----------------------------------------------------------------

Denn bei der $\sum H$, $\sum V$ und der $\sum M$ geht e

Das merkw\"{u}rdige ist aber, dass, weil die Scheibe statisch bestimmt gelagert ist, die Tatsache, dass die Spannungen unendlich gro{\ss} werden, keinen Einfluss auf die Lagerkr\"{a}fte hat.

Bei statisch unbestimmten Tragwerken wird im allgemeinen die Verteilung der Belastung auf die Lager von der Feinheit des Netzes abh\"{a}ngen.
Dadurch, dass man eine Scheibe in finite Elemente unterteilt, verl\"{a}sst man praktisch den Boden der Elastizit\"{a}tstheorie und rechnet mit einem
'Metamodell', das sich der Ingenieur, \"{a}hnlich wie ein Fachwerk, aus Scheibenelementen zusammengesetzt denkt. Und pl\"{o}tzlich kann man auch bei Scheiben, der Elastizit\"{a}tstheorie zum Trotz, mit Einzelkr\"{a}ften (= Knotenkr\"{a}ften) rechnen.

Das Vehikel f\"{u}r den \"{U}bergang zwischen dem Metamodell und dem Originalmodell ist der Begriff der Arbeit. Eine Knotenkraft $f_i = 10 $ kNm in einem festgehaltenen Knoten bedeutet aber trotzdem nicht, dass dort eine Einzelkraft von $10$ kN angreift, sondern vielmehr, dass die Wolke von Fl\"{a}chen- und Kantenkr\"{a}ften, die die Scheibe dort st\"{u}tzt, bei einer Auslenkung des Knotens um eine L\"{a}ngeneinheit die Arbeit $1 \cdot 10$ kNm leistet.



Der Unterschied zwischen der Elastizit\"{a}tstheorie und dem Metamodell ist, dass man keine Einflussfunktionen f\"{u}r nicht existierende Lagerkr\"{a}fte in Punktlagern berechnen kann, aber sehr wohl Einflussfunktionen f\"{u}r die \"{a}quivalenten Knotenkr\"{a}fte $f_i$ in solchen Lagern.
\\

%%%%%%%%%%%%%%%%%%%%%%%%%%%%%%%%%%%%%%%%%%%%%%%%%%%%%%%%%%%%%%%%%%%%%%%%%%%%%%%%%%%%%%%%%%%%%%%%%%%
%{\textcolor{blau2}{\section{Punktlager bei Scheiben}}}\index{Punktlager bei Scheiben}\label{FE-LagerScheibe}
Punktlager---mathematisch unendlich feine, unendlich d\"{u}nne Nadeln---k\"{o}nnen, wenn man der Elastizit\"{a}tstheorie folgt, eine Scheibe nicht festhalten.   Die Lagerkraft ist in allen Lastf\"{a}llen null.
Andererseits erh\"{a}lt man aber mit finiten Elementen doch sinnvolle Ergebnisse in Punktlagern. Wie geht das?

\begin{remark}
Es ist eine der Merkw\"{u}rdigkeiten dieser  'technischen' Scheibentheorie, wenn wir die Mischung aus Elastizit\"{a}tstheorie und finiten Elementen einmal so bezeichnen wollen, dass auf der Au{\ss}enseite der Scheibe, also in den Lagern und l\"{a}ngs den R\"{a}ndern, alles den Ingenieurvorstellungen entspricht, aber sobald man auf die andere Seite des Punktlagers wechselt, die Spannungen tendenziell unendlich gro{\ss} werden.
\end{remark}



%%%%%%%%%%%%%%%%%%%%%%%%%%%%%%%%%%%%%%%%%%%%%%%%%%%%%%%%%%%%%%%%%%%%%%%%%%%%%%%%%%%%%%%%%%%%%%%%%%%
%{\textcolor{blau2}{\section{Linienlager bei Scheiben}}}\label{Linienlager bei Scheiben}
Wir hatten oben \"{u}ber Einflussfunktionen f\"{u}r die $f_i$ in den Lagerknoten gesprochen.
Festgehaltenen Knoten in Reihe stellen ein Linienlager dar und die Knotenkr\"{a}fte $f_i$ sind die Lagerkr\"{a}fte, die oft von den Programmen in verteilte Kr\"{a}fte (Linienkr\"{a}fte) umgerechnet werden und so ausgegeben werden.

Wie berechnet ein FE-Programm die Knotenkr\"{a}fte $f_i$ in den Lagerknoten? \\
\begin{itemize}
  \item Es erweitert den Vektor $\vek u$ zun\"{a}chst um die zuvor gestrichenen $u_i = 0$ in den Lagerknoten, $\vek u \to \vek u_{G}$,
  \item und multipliziert die nicht-reduzierte, globale Steifigkeitsmatrix $\vek K_{G}$ mit dem vollen Vektor $\vek u_{G}$,
  \item die Eintr\"{a}ge $f_i$ in dem Vektor $\vek f_{G} = \vek K_{G}\,\vek u_{G}$, die zu den gesperrten Freiheitsgraden geh\"{o}ren, sind die Knotenkr\"{a}fte in den Lagern.
\end{itemize}

Das erkl\"{a}rt, was programmtechnisch geschieht, aber nicht, wieso eine Knotenkraft $f_i$ gerade den Wert $f_i = 123.45$ kNm hat. Das Resultat beruht auf den Einflussfunktionen.

Betrachten wir ein Rollenlager wie in Bild \ref{U141} mit einer Spannungsverteilung $\sigma_{yy}^h$ in der Lagerfuge. Das Mittel betrage $\bar{\sigma}_{yy}^h$. Die Einheitsverformungen der vier Lagerknoten in vertikaler Richtung seien die vier Dachfunktionen $\Np_i(x), \,i = 1,3, 5, 7$. Die vier Knotenkr\"{a}fte
\begin{align}
f_i = \int_0^{\,l} \sigma_{yy}^h(x)\,\Np_i(x)\,dx \qquad i = 1,3,5,7
\end{align}
sind die \"{U}berlagerung von $\sigma_{yy}^h$ mit den vier Wegen $\Np_i(x)$.

Jeden einzelnen Wert $\sigma_{yy}^h(x)$ hat das FE-Programm mit der zugeh\"{o}rigen Einflussfunktion berechnet, also einer Spreizung des Punktes in vertikaler Richtung. Das ist nun sicherlich ein diffiziles Problem, das aber sehr viel einfacher wird, wenn man mit dem Mittelwert $\bar{\sigma}_{yy}$ der Spannungen rechnet, weil die zugeh\"{o}rige Einflussfunktion einfach dadurch entsteht, dass man das Lager durchschneidet und dann den Teil oberhalb um einen Meter (nur rechnerisch) {\em als Ganzes\/} nach oben dr\"{u}ckt. Diese Bewegung l\"{a}sst sich viel besser ann\"{a}hern, als eine Serie von einzelnen Punktversetzungen und deswegen d\"{u}rfte die Resultierende $R_h$ der Knotenkr\"{a}fte $f_i$ relativ genau sein
\begin{align}
R_h = f_1 + f_3 + f_5 + f_7 = \bar{\sigma}_{yy}^h \cdot \int_0^{\,l} (\Np_1 + \Np_3 + \Np_5 + \Np_7)\,dx\,.
\end{align}
Aber schon die einzelne Knotenkraft $f_i$ profitiert von diesem Effekt, weil jedes $f_i$ ja selbst schon ein gewichtetes Mittel ist
\begin{align}
f_i = \int_0^{\,l} \sigma_{yy}^{h}(x)\,\Np_i(x)\,dx \simeq \bar{\sigma}_{yy}^{(i)}\cdot\int_0^{\,l} \Np_i(x)\,dx\,,
\end{align}
also in etwa dem Mittelwert $\sigma_{yy}^{(i)} = const.$ von $\sigma_{yy}^{h}(x)$ im Bereich von $\Np_i$ entspricht und dieser Werte gewichtet mit dem Integral von $\Np_i(x)$.
\\

Wir k\"{o}nnten dieses Ergebnis jetzt verallgemeinern und statuieren: Nur die Elemente direkt neben dem Rand erzeugen die Lagerkr\"{a}fte.
Das ist aber eigentlich evident, wie man sieht, wenn man den inneren Teil einer Scheibe herausschneidet und nur die Elemente am Rand stehen l\"{a}sst. An den Schnittkanten muss man Haltekr\"{a}fte anbringen, um das Gleichgewicht wieder herzustellen. Und diese Haltekr\"{a}fte sind gerade so gro{\ss}, wie die Belastung, die auf den inneren Teil wirkt. Wenn der Schnittkreis so gro{\ss} w\"{a}re, dass die ganze Scheibe hineinpassen w\"{u}rde, dann w\"{a}re in der Tat (\ref{Eq70}) ma{\ss}gebend.\\

\begin{align}
f_i^h &= \int_{x_a}^{\,x_b} p_h \,\Np_i\,dx \qquad (x_a,x_b) = \text{Tr\"{a}ger von $\Np_i$}\nn \\
&= \underbrace{\int_{x_a}^{\,x_b} EI\,w_h^{IV}\,\Np_i\,dx + [V_h \,\Np_i - M_h\,\Np_i']_{x_a}^{x_b} }_{\delta A_a} = \underbrace{\int_{x_a}^{\,x_b} \frac{M_h\,M_i}{EI}\,dx}_{\delta A_i}\nn \\
&= \int_{x_a}^{\,x_b} \sum_j\,EI\,\Np_j''\,\Np_i''\,dx \,u_j= \sum_j\,a(\Np_j,\Np_i)\,u_j = \sum_j\,k_{ij}\,u_j\,.
\end{align}
Das Arbeitsintegral $(p_h,\Np_i)$ in der ersten Zeile ist dabei symbolisch zu nehmen. Es ist eine Kurzform f\"{u}r das $\delta A_a$ in der zweiten Zeile. Wegen $\delta A_a = \delta A_i$ kann man $f_i^h$ durch die innere Arbeit $\delta A_i$ ausdr\"{u}cken kann. So kommt die Steifigkeitsmatrix in die Gleichung hinein, $\vek K\,\vek u = \vek f_h$, was $\vek f_h = \vek f$ bedeutet.

Der Tr\"{a}ger der Funktion $\Np_i(x)$ ist der Teil der $x$-Achse, in dem $\Np_i$ nicht konstant null ist, wo also $\Np_i(x)$ 'lebt'.
\\

Stimmt dieser Verlauf im Fall der Platte mit der Kurve \"{u}berein, die man erhalten w\"{u}rde, wenn man 'von Innen k\"{a}me', also den Kirchhoffschub im Abstand von 10 cm vom Rand ausrechnet und dann die letzten 10 cm durch Extrapolation \"{u}berbr\"{u}ckt? Nein, in der Regel sind die beiden Kurven nicht gleich. Die von Innen extrapolierten Lagerkr\"{a}fte d\"{u}rften auch relativ schlecht sein, weil die rechnerischen Querkr\"{a}fte sehr schwankend sind und eigentlich nur in der Elementmitte halbwegs passabel sind.
\\

\subsubsection*{Was geht und was nicht geht}

Eine Scheibe so zu st\"{u}tzen, dass die Verschiebungen $u_i$ in einem Punkt null sind---das geht unter zuhilfenahme von Fl\"{a}chen- und Linienkr\"{a}ften.
Das Arbeits\"{a}quivalent dieser Kr\"{a}fte ist die \"{a}quivalente Knotenkraft $f_i$ im Ausdruck
\begin{align}
f_i = \int_{\Omega} \vek p_h \dotprod \vek \Np_i\,d\Omega + \int_{\Gamma} \vek  s_h\dotprod \vek \Np_i\,ds\,.
\end{align}
(Die $\vek s_h$ sind die Linienkr\"{a}fte auf den Elementkanten $\Gamma$ in der N\"{a}he des Lagerknotens).

Was aber nicht geht, ist null Lagerverschiebungen + punktf\"{o}rmige Lagerkraft. Das kann die Elastizit\"{a}tstheorie nicht und ein FE-Programm kann es noch weniger, weil eine echte Punktkraft das Material zum Flie{\ss}en bringen w\"{u}rde.\\

Bei statisch unbestimmten Tragwerken werden die Winkel $\Np_L$ und $\Np_R$ nat\"{u}rlich direkt am Gelenk gemessen. Es sind dann die Stabdrehwinkel der Endtangenten.

Bei diesen Abmessungen gilt im \"{U}brigen
\begin{align}
\Np_L = 36.86^0 \qquad \Np_R = 14.0^0\,,
\end{align}
und der Winkel zwischen den beiden Schenkeln betr\"{a}gt somit $50.86^0$ und der Tangens dieses Winkels ist 1.23.

Damit aus $\delta\,\Np = 1$ etwas vern\"{u}nftiges wird, muss man es so interpretieren:  $\delta\,\Np = 1$ steht f\"{u}r $\tan\,\Np_L + \tan\,\Np_R = 1$. Anders kommt man \"{u}ber die H\"{u}rde nicht hinweg, denn es gibt kein Additionstheorem f\"{u}r den Tangens der Art, dass
\begin{align}
\tan(\Np_L + \Np_R) \overset{?}{=} \tan\,\Np_L + \tan\,\Np_R\,.
\end{align}


In diesem Kapitel besch\"{a}ftigen wir uns mit Einflussfunktionen, ihrer Herleitung und ihrer Anwendung in der Statik.

Mathematisch beruhen Einflussfunktionen auf dem {\em Satz von Betti\/}, also der Tatsache, dass in der linearen Statik die Differentialgleichungen selbstadjungiert\index{selbstadjungiert} sind.
\\

Das Kernst\"{u}ck der Statik der Kontinua\index{Statik der Kontinua} sind die Differentialgleichungen, die den Zusammenhang zwischen der Belastung und den Verformungen beschreiben. Was der einzelnen Differentialgleichung eigen ist, was jeweils speziell an ihr ist, kommt bei der Formulierung der ersten Greenschen Identit\"{a}ten ans Licht. Auf diesen Identit\"{a}ten beruhen im Grunde alle Arbeits- und Energieprinzipe der Statik der Kontinua.



Uns kommt es hier haupts\"{a}chlich darauf an, zu sehen, dass bei solchen Formulierungen die positive Richtung der Einzelkr\"{a}fte und Lagerkr\"{a}fte mit der positiven Richtung der Weggr\"{o}{\ss}en \"{u}bereinstimmt. Das ist jetzt nicht einfach gesetzt, sondern das ergibt sich automatisch, wenn man die erste Greensche Identit\"{a}t abschnittsweise formuliert und dann addiert, $0 + 0 + 0 + 0 + 0 = 0$.

Betrachten wir das Lager $B$, s. Bild \ref{U19}, und die zugeh\"{o}rige Lagerkraft
\begin{align}
B  = V^- - V^+\,.
\end{align}
Wenn wir das Lager  um das Ma{\ss} $\textcolor{red}{\delta w} $ verr\"{u}cken und die erste Greensche Identit\"{a}t an dem Durchlauftr\"{a}ger formulieren, dann ergibt sich aus der virtuellen \"{a}u{\ss}eren Arbeit der beiden Querkr\"{a}fte links und rechts vom Lager,
\begin{align}
[... V^-\textcolor{red}{\delta w} ]^{x_4} + [V^+\textcolor{red}{\delta w}  \ldots]_{x_4} =  V^- \textcolor{red}{\delta w} - V^+ \textcolor{red}{\delta w} = ( V^- - V^+) \,\textcolor{red}{\delta w}
\end{align}
die Arbeit der Lagerkraft zu
\begin{align}
B \cdot \textcolor{red}{\delta w}\,,
\end{align}
wo also die Lagerkraft in derselben Richtung positiv gez\"{a}hlt wird, wie die virtuelle Verr\"{u}ckung.\\

Wenn wir Risse von einem Haus anfertigen, dann sind
Und alle diese 'virtuellen Verr\"{u}ckungen' sind real, nicht blo{\ss} gedacht.

Immer wenn ein Ingenieur von virtuellen Verr\"{u}ckungen spricht, muss man genau hinh\"{o}ren, weil der Ingenieur dann leider zu oft geneigt ist, Mathematik und Mechanik zu vermengen. Manchmal kommen die Argumente aus der Mathematik, und manchmal kommen sie aber aus der Mechanik und dann sind sie nicht hilfreich.

Argumente aus der Statik sind gut geeignet, um den mechanischen Hintergrund zu erl\"{a}utern, um das Verst\"{a}ndnis zu vertiefen, aber sie haben eigentlich keine Beweiskraft, weil eben
\begin{align}
\text{\normalfont\calligra G\,\,}(w,\textcolor{red}{\delta w}) = 0\,,
\end{align}
ein mathematisches Resultat ist.

Vielleicht sollten wir uns damit zufrieden geben, zu betonen, dass virtuelle Verr\"{u}ckungen einfach Testfunktionen sind und dass die einzige Restriktion, der sie unterlegen, ist, dass sie hinreichend glatt sind, damit man sie partiell integrieren kann.

\"{A}hnliches kann man \"{u}ber die virtuellen Kr\"{a}fte sagen. Sie sind weder nur gedacht noch klein, sondern sie werden durch Differentiation
aus der Testfunktion $\delta w^* $  hergeleitet, die bei der Formulierung des Prinzips der virtuellen Kr\"{a}fte jetzt die erste Stelle in der ersten Greenschen Identit\"{a}t einnimmt
\begin{align}
\text{\normalfont\calligra G\,\,}(\textcolor{red}{\delta w^*},w) = 0\,.
\end{align}

%----------------------------------------------------------------------------------------------------------
\begin{figure}[tbp]
\centering
\if \bild 2 \sidecaption \fi
\includegraphics[width=0.6\textwidth]{\Fpath/S5}
\caption{Vorzeichenregelung} \label{S5}
%
\end{figure}%
%----------------------------------------------------------------------------------------------------------
\end{document}
%%%%%%%%%%%%%%%%%%%%%%%%%%%%%%%%%%%%%%%%%%%%%%%%%%%%%%%%%%%%%%%%%%%%%%%%%%%%%%%%%%%%%%%%%%%%%%%%%%%
{\textcolor{blau2}{\section{Das Vorzeichen der Balkenendkr\"{a}fte}}}\index{Vorzeichen der Balkenendkr\"{a}fte}
In den Greenschen Identit\"{a}ten steckt sehr viel Statik, auf die wir im folgenden noch n\"{a}her eingehen wollen.

Wenn man sich einmal die M\"{u}he macht und die Randarbeiten in der ersten Greenschen Identit\"{a}t des Balkens in voller L\"{a}nge ausschreibt
\begin{align}\label{Eq24}
\text{\normalfont\calligra G\,\,}(w, w) &= \int_0^{\,l} p(x)\,w(x)\,dx + V(l)\,w(l) - M(l)\,w'(l) \nn \\
&- V(0)\,w(0) + M(0)\,w'(0) - \int_0^{\,l} \frac{M^2}{EI}\,dx = 0\,,
\end{align}
dann f\"{a}llt das alternierende Vorzeichen der Randarbeiten auf. Das ist nat\"{u}rlich der partiellen Integration geschuldet. Es verwundert dabei aber doch, dass die Arbeitsbeitr\"{a}ge
immer das richtige Vorzeichen haben, wenn man Bild \ref{S5} zu Grunde legt. Als ob der Ingenieur geahnt h\"{a}tte, wie er die Richtung der positiven Schnittkr\"{a}fte zu w\"{a}hlen h\"{a}tte.

Aber es ist nat\"{u}rlich umgekehrt, der Ingenieur hat erst das Arbeitsintegral
\begin{align}
\int_0^{\,l} EI\,w^{IV}(x)\,w(x)\,dx
\end{align}
partiell integriert, und dann hat er gewusst, wie er die positiven Richtungen w\"{a}hlen muss.

Man kann es aber auch so sehen: Die partielle Integration holt das aus der Differentialgleichung heraus, was bei ihrer Herleitung explizit oder implizit hineingesteckt wurde.
\\

Wenn diese auch jedem Ingenieur klar ist, 'mit welcher virtuellen Kraft will man eine Querkraft berechnen?'

Die Durchbiegung eines Balkens kann man also auf zwei Arten berechnen
\begin{align}
1 \cdot w(x) = \int_0^{\,l} \frac{M\,\bar{M}}{EI}\,dx = \int_0^{\,l} G_0(y,x)\,p(y)\,dy
\end{align}
indem man $M$ mit $\bar{M}$ \"{u}berlagert (Prinzip der virtuellen Kr\"{a}fte) oder indem man die Durchbiegung $G_0(y,x)$ aus der Kraft $\bar{P} = 1$ mit der Belastung \"{u}berlagert ({\em Satz von Betti\/}).

Das Moment $M(x)$ in einem Punkt $x$ kann man aber nur mit dem {\em Satz von Betti\/} berechnen
\begin{align}
M(x) = \int_0^{\,l} G_2(y,x)\,p(y)\,dy
\end{align}
indem man an der Stelle $x$ den Balken 'knickt', Biegelinie $G_2(y,x)$ , und diese Biegelinie mit der Belastung \"{u}berlagert.

Wenn man trotzdem versucht $M(x)$ mit dem Prinzip der virtuellen Kr\"{a}fte zu berechnen
\begin{align}
M(x) = \int_0^{\,l} \frac{M\,\bar{M}}{EI}\,dx
\end{align}
dann erleidet man Schiffbruch, weil der Momentenverlauf $\bar{M}$, der zu der Biegelinie $G_2$ 'mit Knick' geh\"{o}rt, nicht berechnet werden kann. An dem Knick scheitert man.\\

Interessanter wird diese Gleichung, wenn man die Terme partiell integriert, (sofern Funktionen im Spiel sind) oder aus ihr den umgekehrten Schluss zieht, wie das eine Marktfrau macht.
%----------------------------------------------------------------------------------------------------------
\begin{figure}[tbp]
\centering
\if \bild 2 \sidecaption \fi
\includegraphics[width=0.9\textwidth]{\Fpath/U100}
\caption{Verdrehung einer Waage} \label{U100}
%
\end{figure}%
%----------------------------------------------------------------------------------------------------------

Wenn eine Waage\index{Waage} im Gleichgewicht ist, s. Bild \ref{U100},
\begin{align}
P_L \cdot h_L = P_R \cdot h_R\,,
\end{align}
dann kann man die Gleichung mit beliebigen Zahlen $\textcolor{red}{x = \tan\,\Np}$ multiplizieren
\begin{align}
P_L \cdot h_L \cdot \textcolor{red}{\tan\,\Np}= P_R \cdot h_R\cdot \textcolor{red}{\tan\,\Np}\,,
\end{align}
was bedeutet, dass bei einer beliebigen Drehung $\textcolor{red}{\Np}$ des Waagebalkens die beiden Gewichte $P_L$ und $P_R$ dieselbe Arbeit leisten, denn die Wege, die die beiden Gewichte gehen, sind ja gerade
\begin{align}
\delta w_L = h_L \cdot \textcolor{red}{\tan\,\Np }\qquad \delta w_R = h_R\cdot \textcolor{red}{\tan\,\Np}\,.
\end{align}
Die Marktfrau st\"{o}{\ss}t die Waage leicht an und wenn die Waage in jeder  Lage stehen bleibt, dann herrscht Gleichgewicht. Die Marktfrau schlie{\ss}t also aus dem Prinzip der virtuellen Verr\"{u}ckungen auf das Gleichgewicht
\begin{align}
P_L \cdot h_L \cdot \textcolor{red}{\tan\,\Np}= P_R \cdot h_R\cdot \textcolor{red}{\tan\,\Np} \qquad \Rightarrow \qquad P_L \cdot h_L = P_R \cdot h_R\,.
\end{align}
%----------------------------------------------------------------------------------------------------------
\begin{figure}[tbp]
\centering
\if \bild 2 \sidecaption \fi
\includegraphics[width=0.8\textwidth]{\Fpath/U65}
\caption{Auslenkung $y'$ bei einer echten Drehung und Auslenkung $y$ bei einer Pseudodrehung} \label{U65}
%
\end{figure}%
%----------------------------------------------------------------------------------------------------------

\begin{remark}
Wir haben hier mit Pseudodrehungen gearbeitet $h = x\cdot \textcolor{red}{\tan\,\Np}$. Wenn man reale Drehungen zu Grunde legt, so wie sie die Marktfrau sieht, dann betr\"{a}gt die Auslenkung der Gewichte $\bar{h} = x\cdot\textcolor{red}{\sin\,\Np}$, ist also etwas kleiner, s. Bild \ref{U65}. Mathematisch ist es jedoch irrelevant, ob der Faktor auf beiden Seiten der Gleichung $\textcolor{red}{\tan\,\Np}$ lautet oder $\textcolor{red}{\sin\,\Np}$, da er sich wegk\"{u}rzt.
\end{remark}


Mit Blick auf ihre gemeinsame Mitte, die Einflussfunktionen, kann man die beiden Methoden etwa wie folgt charakterisieren. \\

\begin{itemize}
  \item Finite Elemente: Gen\"{a}herte Einflussfunktion, aber exakte Daten (Belastung $p$)
  \item Randelemente: Exakte Fundamentall\"{o}sung, aber gen\"{a}herte Randwerte
\end{itemize}

Die Erweiterung der partiellen Integration auf zwei und drei Dimensionen
\begin{align}
\int_{\Omega} u,_i \,v\,d\Omega = \int_{\Gamma} u\,n_i\,ds - \int_{\Omega} u\,v,_i\,d\Omega
\end{align}
impliziert, dass das Integral der Spannungen $\sigma_{xx} = E\,(\varepsilon_{xx} + \nu\,\varepsilon_{yy}) = E\,(u_1,_1 + \nu\,u_2,_2)$ in einer allseits festgehaltenen Scheibe, $u_1 = u_2 = 0$ auf dem Rand $\Gamma$, null ist
\begin{align}
\int_{\Omega} E\,(u_1,_1 + \nu\,u_2,_2) \cdot 1\,d\Omega = \int_{\Gamma} E\,(u_1\,n_1 + \nu\,u_2\,n_2)\,ds = 0\,.
\end{align}
Analog zeigt man das f\"{u}r $\sigma_{yy}$ und $\sigma_{xy}$ und f\"{u}r die Momente einer eingespannten Platte.\\

Die Ingenieure sind immer versucht, die Statik 'gerade zu r\"{u}cken', also die Tatsache, dass man in der Balkenstatik die Drehungen durch Pseudodrehungen ersetzt, damit zu rechtfertigen, dass f\"{u}r kleine Drehwinkel $\Np$ ja der Tangens ungef\"{a}hr gleich dem Winkel selbst ist und daher die Statik eigentlich keinen Fehler begeht.

Auf demselben Wege kommt die Forderung in die Statik hinein, dass die virtuellen Verr\"{u}ckungen 'klein' sein m\"{u}ssen, weil dann nicht auff\"{a}llt, dass man sich auf der Tangente statt auf dem Drehkreis bewegt, s. Bild \ref{U14}.

Aber bei der Statik handelt es sich nicht um N\"{a}herungsrechnung, sondern die Statik rechnet {\em exakt\/}. Sie hat diese 'Korrekturen' nicht n\"{o}tig. Wenn man hinter dem Dezimalpunkt nur weit genug nach rechts geht, dann weicht die Tangente doch irgendwann vom Drehkreis ab und man kommt so zu dem Schluss: {\em Tangente ist richtig, Drehkreis ist falsch\/}.

Und wenn man der Tangente folgt, dann ist es ohne Belang, ob die virtuelle Verr\"{u}ckung gro{\ss} oder klein ist. Sie kann {\em jeden\/} Wert haben.

Der 'Grundfehler' aus der Sicht des Ingenieurs ist, wenn man so will, die Tatsache, dass in der ersten Greenschen Identit\"{a}t die Momentensumme $M = 0$ auf Pseudodrehungen beruht. Aber das muss man akzeptieren. Das geh\"{o}rt zur Balkengleichung eben dazu (und eigentlich zur ganzen linearen Mechanik).

\hspace*{-12pt}\colorbox{hellgrau}{\parbox{0.98\textwidth}{Der Ingenieur stellt die Differentialgleichung $EI\,w^{IV}(x) = p(x)$ auf, aber wie die dazu passenden Gleichgewichtsbedingungen aussehen, steht nicht in seinem Belieben. Das entscheidet allein die Mathematik.}}\\

Man kann daher nicht mitten im Galopp pl\"{o}tzlich wieder auf richtige Drehungen umschwenken und so die Statik in die N\"{a}he der N\"{a}herungsrechnung r\"{u}cken. {\em Die Statik rechnet exakt, sie l\"{o}st die gestellten Aufgaben exakt!\/}

Die Einflussfunktionen von statisch bestimmten Tragwerken sind kinematische Ketten und die Punkte bewegen sich dabei nicht auf Kreisb\"{o}gen um den Drehpol, sondern auf Tangenten an die B\"{o}gen, denn der Abstand $x$ vom Drehpol und die Auslenkung $y$ bilden einen rechten Winkel
\begin{align}
\tan\,\Np = \frac{y}{x}\,.
\end{align}
Nur so erh\"{a}lt man die richtigen Einflussfunktionen und damit die korrekten Schnittgr\"{o}{\ss}en.
Das wird von vielen Ingenieure \"{u}bersehen, die immer wieder versuchen die Mathematik an die Statik zur\"{u}ckzubinden, was aus Gr\"{u}nden der Anschauung sicherlich sinnvoll und auch geboten ist. Ja unbedingt notwendig ist, um das statische Gef\"{u}hl ...\\

 Von diesen Bilanzen, oder sollen wir besser Invarianten sagen,
\begin{align}
\text{DGL} + \Omega + \Gamma + \text{part. Integration} = 0\,.
\end{align}

lebt die Mechanik und die Statik. Die Statik der Kontinua ist im Grunde

Analysis ist also nicht nur Ableitungen, wie hier Gebiet, Rand, Funktion und die  Ableitungen zu einer Einheit verschmelzen. das deutlich macht, was f\"{u}r m\"{a}chtige wieso Differentialgleichungen

 Ein Balken hat eine L\"{a}nge, eine Platte hat eine gewisse Ausdehnung. Die obigen Identit\"{a}ten sind im Grunde nur die Erweiterung der Regeln der partiellen Integration auf

Daher kann man eine Funktion $u$ und ihre Ableitungen im integralen Sinn auf dem Gebiet $\Omega$ messen, und die Gesamtbilanz ist null.
Das ist die Symbiose von Gebiet und Funktion. Zu jedem Gebiet $\Omega$ und zu jeder auf $\Omega$ definierten Funktion
geh\"{o}rt eine Bilanzgleichung.

Die obigen Identit\"{a}ten sind {\em Invarianten\/} der Differential- und Integralrechnung.

Die Regeln der partiellen Integration besagen nun, dass\\

\begin{itemize}
  \item das Gleichgewicht
  \item das Prinzip der virtuellen Verr\"{u}ckungen
  \item das Prinzip der virtuellen Kr\"{a}fte
  \item der Energieerhaltungssatz
  \item der Satz von Betti
\end{itemize}\\

%---------------------------------------------------------------------------------
\begin{figure}[tbp]
\centering
\if \bild 2 \sidecaption \fi
\includegraphics[width=0.8\textwidth]{\Fpath/U131}
  \caption{Quadratplatte 8 m $\times $ 8 m, FE-Einflussfunktion f\"{u}r das Biegemoment $m_{xx}$ im St\"{u}tzenanschnitt, \textbf{ a)} 3-D Darstellung, \textbf{ b)} L\"{a}ngsschnitt}
  \label{U131}
%
\end{figure}
%---------------------------------------------------------------------------------\\

 l\"{a}ngs
\bfo\label{Phi1Bis4}
\Np_1^e(x) = \frac{1 - x}{l} \qquad  \Np_2^e(x) = \frac{x}{l}
\efo
und quer
\bfo\label{Phi1Bis4}
\parbox{5cm}{
\bfo
\Np_1^e(x) &=& 1 - \frac{3x^2}{l^2} + \frac{2x^3}{l^3} \nn \\
\Np_2^e(x) &=& - x + \frac{2x^2}{l} - \frac{x^3}{l^2} \nn
\efo
}
\parbox{5cm}{
\bfo
\Np_3^e(x) &=& \frac{3x^2}{l^2} - \frac{2x^3}{l^3}\nn \\
\Np_4^e(x) &=& \frac{x^2}{l} - \frac{x^3}{l^2}\,.\nn  \label{Einheitsverformungen}
\efo
}
\efo



%%%%%%%%%%%%%%%%%%%%%%%%%%%%%%%%%%%%%%%%%%%%%%%%%%%%%%%%%%%%%%%%%%%%%%%%%%%%%%%%%%%%%%%%%%%%%%%%%%%
{\textcolor{blau2}{\section{Drehungen}}}\index{Drehungen}
Eine Merkw\"{u}rdigkeit weist die Balkenstatik auf und zwar sind das Drehungen, genauer gesagt Starrk\"{o}rperdrehungen. So ist die Einflussfunktion f\"{u}r das Biegemoment in einem Kragtr\"{a}ger eine solche Drehung der rechten H\"{a}lfte des Kragtr\"{a}ger um 45$^0$ nach oben. Wenn man auf dem Kragtr\"{a}ger nur weit genug nach au{\ss}en geht, also den Kragtr\"{a}ger nur lang genug macht, dann kann man in dem Kragtr\"{a}ger beliebig gro{\ss}e Momente erzeugen.

Verl\"{a}ngert man den Strahl bis zum Fixsternhimmel, so entstehen aus winzigen Drehungen unendlich gro{\ss}e Verschiebungen.
Richtet man einen sehr starken Laserstrahl auf den Mond, dann bewegen sich die Lichtpunkte auf dem Mond bei einer winzigen Drehung des Lasers mit einer Geschwindigkeit, die gr\"{o}{\ss}er ist als die Lichtgeschwindigkeit. (Was kein Widerspruch zur Relativit\"{a}tstheorie von Einstein ist).\\




die Statik kennt keine Einflussfunktionen f\"{u}r die \"{a}quivalenten Lagerkr\"{a}fte $f_i$ bei Fl\"{a}chentragwerken
\begin{align}
f_i \overset{?}{=}  \int_{\Omega} G(\vek y,\vek x)\,p(\vek y)\,d\Omega_{\vek y}\,.
\end{align}
Die $f_i$ geh\"{o}ren ja nach der Vorstellung des Ingenieurs zu Punktlagern, aber man kann Fl\"{a}chentragwerke (mit der oben erw\"{a}hnten Ausnahme) nicht auf Punktlager stellen.
Die $f_i$ geh\"{o}ren also ganz in das FE-Modell

Die Sache wird dadurch verkompliziert, dass Ingenieure sich inzwischen daran gew\"{o}hnt haben, die Knotenkr\"{a}fte $f_i$ in den Lagerknoten als die Lagerkr\"{a}fte einer Platte oder Scheibe anzusehen. Sie haben kein Problem damit, gedanklich die klassische Statik mit dem Model 'als ob' zu mischen.

Um die $f_i$ am Leben zu erhalten, muss man also das Modell wechseln und

Das ist ein Punkt, wo das FE-Modell dem Denken des Ingenieurs viel n\"{a}her ist, als die Theorie, die gleich alles auf die 'Spitze' treibt und Punktlager unendlich d\"{u}nnen Nadeln gleichsetzt, was nat\"{u}rlich kein Material aush\"{a}lt.

F\"{u}r eine Luftst\"{u}tze kann man keine Einflussfunktion berechnen, sie ist identisch null, aber f\"{u}r die Knotenkr\"{a}fte $f_i$ in einem Punktlager schon. Diese Zweigleisigkeit w\"{u}rde sich erst aufl\"{o}sen, wenn man die Maschenweite $h$ der Elemente gegen Null gehen lassen w\"{u}rde, dann w\"{u}rde auch die Einflussfunktion f\"{u}r die Knotenkraft gegen null gehen, aber so weit geht man in der Praxis nicht.

Dies ist ein Punkt, wo die Maschenweite $h$ \"{u}ber ihren reinen Zahlenwert hinaus, auch den Charakter des Modells steuert
\begin{itemize}
  \item $h > 0$, ein Linienlager auf einer unter Umst\"{a}nden sehr kleinen Fl\"{a}che $h \times d$ ($d$ = St\"{a}rke der Scheibe)
  \item $h \to 0$, eine unendlich d\"{u}nne Nadel als Lager\,.
\end{itemize}
Das hat man sonst bei finiten Elementen nicht.

\\

Bei Stabtragwerken kommen Einzelkr\"{a}fte, Einzelmomente in nat\"{u}rlicher Weise vor und daher ordnen sich Knotenkr\"{a}fte und Knotenmomente ohne Probleme in die Statik ein. Numerik und Statik gehen parallel.

Bei Scheiben und Platten sind die Knotenkr\"{a}fte $f_i$ jedoch nur Rechengr\"{o}{\ss}en, 'eins im Sinn', die stellvertretend die Wirkungen der Fl\"{a}chen- oder Linienkr\"{a}fte beschreiben, die in der Umgebung des Knotens stehen. Es sind Arbeits\"{a}quivalente. Eine \"{a}quivalente Knotenkraft von $f_i = 10$ kNm signalisiert, dass die in der N\"{a}he des Knotens verteilten Kr\"{a}fte bei einer Auslenkung des Knotens um 1 m die Arbeit $10$ kNm leisten w\"{u}rden, also soviel, wie eine einzelne Kraft von $f_i = 10$ kN leisten w\"{u}rde, die im Knoten steht.
\\

An Hand des Systems $\vek K_{G}\,\vek u_{G} = \vek f_{G}$ berechnet ein FE-Programm die $f_i$ in den Lagerknoten und daraus dann durch Interpolation der Knotenwerte die Lagerkraft der Platte bzw. der Scheibe l\"{a}ngs des Randes.

Diese Vorgehensweise vermeidet auch die Frage, wie man denn eine Einflussfunktion f\"{u}r eine Lagerkraft in einem Punkt berechnet, denn wie wollte man einen Punkt des Randes um eine L\"{a}ngeneinheit absenken aber den Rest stehen lassen?

Das geht nur so, dass man von Innen kommt, also den Kirchhoffschub in 10 cm Abstand vom Rand berechnet und dann auf den  Rand extrapoliert. Wenn man das f\"{u}r alle Punkte l\"{a}ngs des Randes macht, dann erh\"{a}lt man die Lagerkr\"{a}fte l\"{a}ngs des Randes.

Wir wollen hier nicht untersuchen, inwieweit die so von Innen auf den Rand extrapolierten Lagerkr\"{a}fte mit dem Verlauf \"{u}bereinstimmen, den man aus den $f_i$ erh\"{a}lt, und uns lieber damit besch\"{a}ftigen, wie man Einflussfunktionen f\"{u}r die $f_i$ berechnen kann.\\

%%%%%%%%%%%%%%%%%%%%%%%%%%%%%%%%%%%%%%%%%%%%%%%%%%%%%%%%%%%%%%%%%%%%%%%%%%%%%%%%%%%%%%%%%%%%%%%%%%%
{\textcolor{blau2}{\section{Goal oriented adaptive refinement}}\index{Betti und finite Differenzen}
Zu Beginn dieses Kapitels haben wir, s. Bild \ref{U122}, die Durchbiegung eines Seils in einem Punkt berechnet, der zwischen zwei Knoten lag. Dies zwang uns dazu mit einer gen\"{a}herten Einflussfunktion $G_h(y,x)$ zu rechnen, was eine Abweichung in der berechneten Durchbiegung zur Folge hatte
\begin{align}
w(x) - w_h(x) = \int_0^{\,l} (G(y,x) - G_h(y,x))\,p(y)\,dy\,.
\end{align}
Hier ist $p$ die Original-Belastung auf dem Seil. Wie wir wissen, ersetzt das FE-Programm aber die Streckenlast $p$ durch Einzelkr\"{a}fte $f_i$ in den Knoten, die den FE-Lastfall $p_h$ darstellen. Um jetzt nicht zu weit ausholen zu m\"{u}ssen, nehmen wir der Einfachheit halber an, dass der FE-Lastfall $p_h$  aus einem auf und ab von Streckenlasten $p_h$ besteht. (Was in diesem Fall ja nicht richtig ist, aber auch mit Knotenkr\"{a}ften bleibt die Logik dieselbe).

Nun kann man zeigen, dass sich die obige Formel nicht \"{a}ndert, wenn man f\"{u}r $p$ die Differenz $p - p_h$ setzt
\begin{align} \label{Eq85}
w(x) - w_h(x) = \int_0^{\,l} (G(y,x) - G_h(y,x))\,(p(y) - p_h(y))\,dy\,.
\end{align}
Das Integral, die \"{a}u{\ss}ere Arbeit, kann man durch die gleich gro{\ss}e innere  Arbeit
\begin{align} \label{Eq86}
w(x) - w_h(x) &= H \int_0^{\,l} (G'(y,x) - G_h'(y,x))\,(w'(y) - w_h'(y))\,dy\nn \\
 &= a(G-G_h,w-w_h)
\end{align}
ersetzen. Nach ein paar weiteren Schritten f\"{u}hrt dies auf die Absch\"{a}tzung
\begin{align}
|w(x) - w_h(x)| \leq c \cdot a(G-G_h,G-G_h)\cdot a(w-w_h,w-w_h)
\end{align}
mit einer Konstanten $c$. Die Botschaft dieser Ungleichung ist, dass der Fehler beschr\"{a}nkt ist durch die inneren Energie des Fehlers $G-G_h$ in der Einflussfunktion und der inneren Energie des Fehlers $w - w_h$ der FE-L\"{o}sung, was im Grunde bedeutet: Je schlechter sich die Einflussfunktion bzw. die exakte L\"{o}sung ann\"{a}hern l\"{a}sst, um so gr\"{o}{\ss}er ist der Fehler auf der linken Seite---was evident klingt.
\\

Dieses Buch hat daher die Absicht zu zeigen, wie die Arbeits- und Energieprinzipe der Statik

\begin{itemize}
  \item das Prinzip der virtuellen Verr\"{u}ckungen
  \item der Energieerhaltungssatz
  \item das Prinzip der virtuellen Kr\"{a}fte
  \item der Satz von Betti
\end{itemize}
sich aus der Mathematik entwickeln, weil diese S\"{a}tze keine Naturgesetze sind, sonder sie in allen ihren Ausformungen nur auf Mathematik beruhen. Wenn man zu ihren Quellen will, dann muss man die mathematischen Grundlagen dieser S\"{a}tze und Prinzipe studieren.\\

Wir wollen aus den Ingenieuren keine Mathematiker machen, aber sie sollen doch an einem Punkt die Gelegenheit bekommen zu verstehen, warum virtuelle Verr\"{u}ckungen oder virtuelle Kr\"{a}fte nicht klein sein m\"{u}ssen. Warum das Rechnen in der Statik auf mathematischen Gesetzen beruht und nicht auf Naturgesetzen.\\

%%%%%%%%%%%%%%%%%%%%%%%%%%%%%%%%%%%%%%%%%%%%%%%%%%%%%%%%%%%%%%%%%%%%%%%%%%%%%%%%%%%%%%%%%%%%%%%%%%%
{\textcolor{blau2}{\section{Gl\"{a}tten von Oszillationen}}
Eines der popul\"{a}rsten Ver\"{o}ffentlichungen zu dem Thema finite Elemente ist der Aufsatz von Zhu und Zienkiewicz, \cite{Z2}, der zeigt, wie man Oszillationen in den Spannungen gl\"{a}tten kann.

Der umgekehrte Schluss lautet dann, wenn die Spannungen nicht oszillieren, dann sind sie 'richtig'. Und das kann ein Trugschluss sein, wie in \cite{Babuska5} gezeigt wurde. Pollution kann die Spannungen in einem Bauteil in eine gewisse Richtung verf\"{a}lschen, ohne dass es dabei zu Oszillationen kommt. Es werden einfach nur alle Werte um einen gewissen Betrag nahezu gleichm\"{a}{\ss}ig angehoben oder gesenkt.

Und der Ingenieur ahnt nicht, dass diese glatten Spannungen einen systematischen Fehler aufweisen.

%----------------------------------------------------------------------------------------------------------
\begin{figure}[tbp]
\centering
\if \bild 2 \sidecaption \fi
\includegraphics[width=1.0\textwidth]{\Fpath/S2}
\caption{Anwendung des Prinzips der virtuellen Kr\"{a}fte} \label{S2}
%
\end{figure}%
%----------------------------------------------------------------------------------------------------------

%----------------------------------------------------------------------------------------------------------
\begin{figure}[tbp]
\centering
\if \bild 2 \sidecaption \fi
\includegraphics[width=0.6\textwidth]{\Fpath/S11}
\caption{Anwendung des Prinzips der virtuellen Verr\"{u}ckungen zur Berechnung der Lagerkraft $A$} \label{S11}
%
\end{figure}%
%----------------------------------------------------------------------------------------------------------\\

Die Einflussfunktion f\"{u}r eine Durchbiegung $w(x)$ ist gleich der Biegelinie, die von einer Einzelkraft $P = 1$ erzeugt wird, die in dem Aufpunkt $x$ angreift und die Einflussfunktion f\"{u}r eine Verdrehung $w'(x)$ ist die Biegelinie, die von einem Moment $M = 1$ erzeugt wird,
das in dem Aufpunkt $x$ angreift, s. Bild \ref{U150}.

Es ist also immer die zur Weggr\"{o}{\ss}e konjugierte Kraftgr\"{o}{\ss}e, die die Einflussfunktion produziert. Diese einfache Regel basiert auf dem Satz von Betti.\\

Um die Querkraft im Punkt $x$ des Tr\"{a}gers in Bild  \ref{U164} zu berechnen, bauen wir im Punkt $x$ ein Querkraftgelenk ein, unterbrechen also den Kraftfluss und m\"{u}ssen dies dadurch korrigieren, dass wir jetzt von au{\ss}en zwei gegengleiche Querkr\"{a}fte $V(x)$ wirken lassen. Dann erteilen wir dem so modifizierten Tr\"{a}ger eine Bewegung derart, dass die beiden Querkr\"{a}fte insgesamt den Weg $(-1)$ zur\"{u}cklegen, sie also die Arbeit
\begin{align}
V(x)\,(\textcolor{blau2}{-1})
\end{align}
leisten. Bei dieser Bewegung leistet die \"{a}u{\ss}ere Belastung die Arbeit
\begin{align}
P\cdot \textcolor{blau2}{w}
\end{align}
und gem\"{a}{\ss} dem Satz von Betti muss die Summe dieser beiden Arbeiten null sein, weil, wie wir sp\"{a}ter sehen werden, $A_{2,1}$ immer null ist,
\begin{align}
A_{1,2} = V(x)\cdot (\textcolor{blau2}{-1}) + P\cdot\textcolor{blau2}{w} = 0 \qquad (\leftarrow \,\,\,\,A_{2,1})
\end{align}
oder
\begin{align}
V(x) \cdot \textcolor{blau2}{1} = P\,\textcolor{blau2}{w}\,.
\end{align}\\

%----------------------------------------------------------
\begin{figure}[tbp]
\centering
\includegraphics[width=1.0\textwidth]{\Fpath/1GREENF73D}
\caption{Wie die Einflussfunktion f\"{u}r das Biegemoment $M$ \"{u}ber den Tr\"{a}ger wandert und  dabei im Grunde ihre Gestalt beibeh\"{a}lt (Gleichlast)}
\label{1GreenF73}%
%
\end{figure}%%
%----------------------------------------------------------

%----------------------------------------------------------
\begin{figure}[tbp]
\centering
\includegraphics[width=1.0\textwidth]{\Fpath/1GREENF74D}
\caption{Wie die Einflussfunktion f\"{u}r die Querkraft $V$ \"{u}ber den Tr\"{a}ger wandert und dabei im Grunde ihre Gestalt beibeh\"{a}lt (Gleichlast) }
\label{1GreenF74}%
%
\end{figure}%%
%----------------------------------------------------------\\

%%%%%%%%%%%%%%%%%%%%%%%%%%%%%%%%%%%%%%%%%%%%%%%%%%%%%%%%%%%%%%%%%%%%%%%%%%%%%%%%%%%%%%%%%%%%%%%%%%%
{\textcolor{blau2}{\section{Varianten im Entwurf}}}
Wenn man einen Backstein wegnimmt, \"{a}ndert sich dann die Einflussfunktion oder nicht? Wenn sie sich nicht \"{a}ndert, dann kann man den Backstein weglassen. Das ist---etwas (stark) verk\"{u}rzt---die Frage, die wir hier anschneiden wollen.

Weil Einflussfunktionen f\"{u}r Lager- und Schnittkr\"{a}fte in statisch bestimmten Tragwerken kinematische Ketten sind, ist es klar, dass die Lagerkr\"{a}fte nicht davon abh\"{a}ngen, wie die Bauteile dimensioniert sind. \"{A}hnliches gilt f\"{u}r die Schnittkr\"{a}fte, obwohl hier die spezielle Gestalt eines Tr\"{a}gers einen Einfluss haben kann. In einem Fachwerktr\"{a}ger spaltet sich eine Querkraft anders auf, als in einem geraden Biegebalken, aber vom Prinzip her
bestimmt nur die Kinematik die Gr\"{o}{\ss}e der Schnittkr\"{a}fte.
%--------------------------------------------------------------------------------------
\begin{figure}
\centering
\includegraphics[width=0.7\textwidth]{\Fpath/U26}
\caption{Statisch bestimmtes Fachwerk, die Stabkr\"{a}fte h\"{a}ngen nur von der Geometrie des Fachwerkes ab, aber nicht von den Steifigkeiten $EA_i$ der St\"{a}be}
\label{U26}%
%
\end{figure}%
%--------------------------------------------------------------------------------------

%%%%%%%%%%%%%%%%%%%%%%%%%%%%%%%%%%%%%%%%%%%%%%%%%%%%%%%%%%%%%%%%%%%%%%%%%%%%%%%%%%%%%%%%%%%%%%%%%%%
{\textcolor{blau2}{\subsection{Statisch bestimmtes Fachwerk}}}

Die Situation l\"{a}sst sich am besten an Hand eines Fachwerks illustrieren.
Viele Probleme in der Mechanik m\"{u}nden bei der Behandlung mit finiten Elementen in dem dreifachen Produkt
\beq
\vek K_{n \times n}\,\vek u_{n \times 1} = \vek A^T_{n \times m}\,\vek C_{m \times m}\,\vek A_{m \times n}\,\vek u_{n \times 1} = \vek f_{n \times 1}\,,
\eeq
wobei $\vek A$ eine rechteckige Matrix ist und $\vek C$ eine quadratische Matrix, die von den Materialparametern abh\"{a}ngt, \cite{Strang4}.

In einem Fachwerk aus $m$ St\"{a}ben sind die Komponenten des Vektor $\vek A\,\vek u$ die Dehnungen $\varepsilon_i = u_i'$ der $m$ Fachwerkst\"{a}be und die Matrix $\vek C$ ist eine Diagonalmatrix mit den Eintr\"{a}gen $c_{ii} = EA_i$, einer f\"{u}r jeden Stab, und der Vektor
\beq
\vek C\vek A\,\vek u = \vek n = \{N_1, N_2, \ldots N_m\}^T
\eeq
ist die Liste der Normalkr\"{a}fte $N_i$ in den Fachwerkst\"{a}ben und das Gleichungssystem
\beq
\vek A^T\,\vek n = \vek f
\eeq
formuliert die Gleichgewichtsbedingungen in den Knoten. Wenn die Matrix $\vek A$ quadratisch ist wie in einem statisch bestimmten Fachwerk, dann reichen die Gleichgewichtsbedingungen alleine aus, um die Stabkr\"{a}fte zu berechnen
\beq
\vek n = (\vek A^T)^{-1} \,\vek f
\eeq
und diese wiederum bestimmen die Knotenverschiebungen
\beq
\vek u = (\vek C\,\vek A)^{-1} \,\vek n\,.
\eeq
Weil die Matrix $\vek A$ nur von der Geometrie des Fachwerkes abh\"{a}ngt, also der L\"{a}nge  $l_i$ der St\"{a}be und ihren Winkeln $\alpha_i$ gegen\"{u}ber der Horizontalen, folgt, dass in einem statisch bestimmten Fachwerk die Einflussfunktionen f\"{u}r die Normalkr\"{a}fte, die ja die Spalten der Matrix $(\vek A^T)^{-1}$ bilden, nur von der Geometrie des Fachwerkes abh\"{a}ngen, aber nicht von den Steifigkeiten $EA_i$ der einzelnen Elemente, s. Bild \ref{U26}.

Anders ist es bei den Knotenverschiebungen. Ihre Einflussfunktionen h\"{a}ngen von der Matrix $\vek C$ ab.\\

\hspace*{-12pt}\colorbox{hellgrau}{\parbox{0.98\textwidth}{In statisch bestimmten Tragwerke h\"{a}ngen die Einflussfunktionen f\"{u}r die Schnitt- und Lagerkr\"{a}fte nicht von den Steifigkeiten der einzelnen Bauteile ab. Die Einflussfunktionen f\"{u}r Verformungen dagegen schon.}}\\

%%%%%%%%%%%%%%%%%%%%%%%%%%%%%%%%%%%%%%%%%%%%%%%%%%%%%%%%%%%%%%%%%%%%%%%%%%%%%%%%%%%%%%%%%%%%%%%%%%%
{\textcolor{blau2}{\subsection{Scheibe}}}
Bei statisch bestimmten Tragwerken bilden die Einflussfunktionen kinematische Ketten und daraus folgt, dass die Gestalt einer Einflussfunktion
nicht von den Steifigkeiten der einzelnen Bauteilen abh\"{a}ngt.

Daraus m\"{u}ssen wir schlie{\ss}en, dass eine Scheibe hochgradig statisch unbestimmt ist, denn die Gestalt der Einflussfunktion f\"{u}r eine Spannung $\sigma_{xx}$, die ja durch eine Spreizung des Aufpunktes ausgel\"{o}st wird, h\"{a}ngt deutlich von der Steifigkeit der Scheibe ab.

Eine solche Einflussfunktion ist ein 'Gesamtkunstwerk', an deren Auspr\"{a}gung und Gestalt die ganze Scheibe beteiligt ist. Jede noch so kleine Modifikation in einer Steifigkeit $k_{ij}$ \"{a}ndert die Einflussfunktion ab, denn die Zeilen (= Spalten) der inversen Steifigkeitsmatrix $\vek K^{-1}$ sind ja die Knotenverschiebungen  der Einflussfunktionen.\\

%---------------------------------------------------------------------------------
\begin{figure}[tbp]
\centering
\if \bild 2 \sidecaption \fi
\includegraphics[width=0.8\textwidth]{\Fpath/U132}
  \caption{Wandscheibe, FE-Einflussfunktion f\"{u}r die Spannung $\sigma_{xx}$ in einem Punkt nahe dem unteren Rand, \textbf{ a)} Aufriss, \textbf{ b)} Knotenverschiebungen (Vektoren $\vek g$) aus der Einflussfunktion, $\sigma_{xx} = \vek g^T\,\vek f$}
  \label{U132}
%
\end{figure}
%---------------------------------------------------------------------------------\\

%----------------------------------------------------------------------------------------------------------
\begin{figure}[tbp]
\centering
\if \bild 2 \sidecaption \fi
\includegraphics[width=1.0\textwidth]{\Fpath/U43}
\caption{Das Rohr mit vier Durchmessern muss in vier Abschnitte unterteilt werden. F\"{u}r jeden Abschnitt wird separat die erste Greensche Identit\"{a}t aufgestellt und dann werden die Identit\"{a}ten addiert, 0 + 0 + 0  + 0 = 0} \label{U43}
%
\end{figure}%
%----------------------------------------------------------------------------------------------------------

%%%%%%%%%%%%%%%%%%%%%%%%%%%%%%%%%%%%%%%%%%%%%%%%%%%%%%%%%%%%%%%%%%%%%%%%%%%%%%%%%%%%%%%%%%%%%%%%%%%
{\textcolor{blau2}{\section{Unterschiedliche Steifigkeiten}}}\index{unterschiedliche Steifigkeiten}
Wenn sich die Steifigkeiten l\"{a}ngs eines Tr\"{a}gers \"{a}ndern, wie in Bild \ref{U43}, dann kann man die erste Greensche Identit\"{a}t nur abschnittsweise anschreiben. Weil aber die Randarbeiten an den Intervallgrenzen wegen $u_L = u_R$ und $N_L = N_R$ bei der Addition der Identit\"{a}ten wegfallen, bleibt am Schluss der Ausdruck
\begin{align}
\text{\normalfont\calligra G\,\,}(u,u) &= \int_0^{\,x_1} \frac{N^2}{EA_1}\,dx + \int_{\,x_1}^{x_2} \frac{N^2}{EA_2}\,dx +\int_{\,x_2}^{x_3} \frac{N^2}{EA_3}\,dx + \int_{\,x_3}^{x_4} \frac{N^2}{EA_4}\,dx \nn \\
&-P\,u(l) = 0\,.
\end{align}
\"{u}brig, der, bis auf den Faktor $1/2$, dem Energieerhaltungssatz entspricht.

%---------------------------------------------------------------------------------
\begin{figure}
\centering
{\includegraphics[width=0.9\textwidth]{\Fpath/U104}}
  \caption{\textbf{ a)} Unterteilung eines Stabes in f\"{u}nf lineare Elemente
  \textbf{ b-f)} die Verschiebungen sind die Spalten der inversen Steifigkeitsmatrix (alle Werte mal $l/(EA)$).}
  \label{U104}
%
\end{figure}%
%---------------------------------------------------------------------------------
\\


Ist diese Kraft nun gleich $R_{FE} + R_{X}$ oder nur gleich $R_{FE}$? Es ist die volle St\"{u}tzenkraft, also $R_h = R_{FE} + R_{X}$, wenn wir die Bezeichnungen des obigen Beispiels w\"{a}hlen. Dies sieht man, wenn man die {\em $h$-Vertauschungsregel\/} anwendet
\begin{align}\label{Eq99}
R_h = \int_{\Omega} G_h(\vek y,\vek x)\,p(\vek y)\,d\Omega_{\vek y} = \int_{\Omega} G(\vek y,\vek x)\,p_h(\vek y)\,d\Omega_{\vek y}\,,
\end{align}
denn gem\"{a}{\ss} dem zweiten Integral ist die Kraft $R_h$ gleich der mit der exakten Einflussfunktion $G(\vek y,\vek x)$ ermittelten St\"{u}tzenkraft des FE-Lastfalls $p_h$, und daher fehlt nichts.

Dabei ist aber ein kleiner Trick im Spiel. Das $p_h$ in (\ref{Eq99}) ist der FE-Lastfall des Systems {\em ohne St\"{u}tze\/}, weil ja am Anfang die St\"{u}tze weggenommen wurde, um die Platte dort um 1 Meter nach unten dr\"{u}cken zu k\"{o}nnen.\\

Betrachten wir eine Platte, in deren Mitte eine starre St\"{u}tze steht. Der Knoten habe in vertikaler Richtung den Freiheitsgrad $u_i$, s. Bild \ref{U142}. Es sei $\Np_i(\vek x)$ die Einheitsverformung, die zu dem Freiheitsgrad $u_i$  geh\"{o}re. Dann ist $f_i$ die Summe
\begin{align}
\sum_j k_{i j} u_j = f_i\,.
\end{align}
Diese Gleichung entspricht der Bilanz
\begin{align}
\delta A_i(w_h,\Np_i) = \delta A_a(p_h,\Np_i)\,,
\end{align}
was die umgestellte erste Greensche Identit\"{a}t ist
\begin{align}
\text{\normalfont\calligra G\,\,}(w_h,\Np_i) = \delta A_a(p_h,\Np_i) - \delta A_i(w_h,\Np_i) = 0\,.
\end{align}
Wir nennen diese St\"{u}tzkraft $R_{FE}$. Zu ihr muss nun noch die St\"{u}tzkraft $R_X$ addiert werden, die n\"{o}tig ist, um der Kraft, die direkt in die St\"{u}tze flie{\ss}t, das Gleichgewicht zu halten
\begin{align}
R_X = -\int_{\Omega} p\,\Np_i\,d\Omega\,,
\end{align}
so dass sich die gesamte St\"{u}tzkraft zu
\begin{align}
R_h = R_{FE} + R_{X}
\end{align}
ergibt.

All das gilt nat\"{u}rlich auch sinngem\"{a}{\ss} f\"{u}r die R\"{a}nder von Scheiben und Platten, wo randnahe Lasten direkt in die Lager reduziert werden. Sie gehen in den Vektor $\vek f_{G}$, der dem System $\vek K_{G}\,\vek u_{G} = \vek f_{G}$ zu Grunde liegt, nicht ein. Sie m\"{u}ssen 'von Hand' dazu addiert werden.\\

Mit finiten Elementen erh\"{a}lt man nat\"{u}rlich nur eine N\"{a}herung $G_h(\vek y,\vek x)$ f\"{u}r die Biegefl\"{a}che  und so ist auch die FE-St\"{u}tzenkraft
\begin{align}
R_h = \int_{\Omega} G_h(\vek y,\vek x)\,p(\vek y)\,d\Omega_{\vek y}
\end{align}
nur eine N\"{a}herung.

Auch dieser Einflussfunktion fehlt der Anteil $\Np_X(\vek y)$, weil diese Funktion ja nicht in $V_h$ liegt. Gedanklich ist es aber ein leichtes, die Funktion zu $G_h(\vek y,\vek x)$  hinzu zu addieren.
\\

Schlie{\ss}lich und endlich l\"{a}uft das Ganze darauf hinaus, dass man die Einflussfunktion f\"{u}r die \"{a}quivalente Lagerkraft in dem Knoten einer Platte oder Scheibe so berechnet, wie man das naiverweise vermuten w\"{u}rde: Man spreizt den Spalt zwischen dem Lagerknoten und seinem Widerlager (Wand, St\"{u}tze) um 1 Meter entgegen der Richtung der gesuchten Lagerkraft. Die Verformungsfigur der Platte oder Scheibe, die sich dabei einstellt,
\begin{align}
G_h(\vek y,\vek x)
\end{align}
ist die Einflussfunktion, s. Bild \ref{U180}, f\"{u}r die Lagerkraft $R_h = R_{FE} + R_X$, also die vollst\"{a}ndige Lagerkraft, inklusive dem Anteil $R_X$, der dem Anteil der Belastung entspricht, der direkt in das Lager reduziert wird.\\

%---------------------------------------------------------------------------------
\begin{figure}
\centering
\if \bild 2 \sidecaption \fi
\includegraphics[width=1.0\textwidth]{\Fpath/1GREENF20}
\caption{FE-Einflussfunktion f\"{u}r die Durchbiegung eines Seils. Leicht unterschiedliche Lage \textbf{ a)} und \textbf{ b)} des Aufpunktes $x$. Die Einflussfunktion \"{a}ndert sich praktisch nicht.}
\label{1GreenF20}%
%
\end{figure}%
%---------------------------------------------------------------------------------

Bei Verschiebungen $u(\vek x)$ gibt es keinen Sprung. Die \"{a}quivalenten Knotenkr\"{a}fte sind die Verschiebungen $\Np_i(x)$ der Ansatzfunktionen und wenn man das Element wechselt, dann \"{a}ndern sich diese Knotenkr\"{a}fte nicht sprungweise.

\\

Die Berechnung von Einflussfunktionen
Oben haben wir die Einflussfunktion f\"{u}r eine St\"{u}tzenkraft berechnet, indem wir einfach die Einflussfunktion f\"{u}r die Absenkung des St\"{u}tzenkopfs mit der St\"{u}tzensteifigkeit $k$ multipliziert haben.

Bei Lagerknoten von Aussen- oder Innenw\"{a}nden l\"{a}sst sich diese Technik nicht anwenden, weil solche W\"{a}nde nicht als Abfolge von St\"{u}tzen gedacht werden k\"{o}nnen. Hier kommen wir anders zum Ziel.

%%%%%%%%%%%%%%%%%%%%%%%%%%%%%%%%%%%%%%%%%%%%%%%%%%%%%%%%%%%%%%%%%%%%%%%%%%%%%%%%%%%%%%%%%%%%%%%%%%%
{\textcolor{blau2}{\section{Kopplung Wand---Scheibe}}
Bei der sogenannten {\em Positionsstatik\/}, wo man jeden Unterzug, jede Deckenplatte f\"{u}r sich alleine untersucht, wird man die Auflagerung einer Deckenplatte auf die W\"{a}nde durch Federn simulieren und dann kann man, wie oben gezeigt, sehr einfach die Einflussfunktionen f\"{u}r die \"{a}quivalenten Knotenkr\"{a}fte in diesen Federn berechnen.

Bei einer {\em 3-D Statik\/}, m\"{u}sste man jedoch anders vorgehen.\\

'zerrei{\ss}t', wenn man also einen Balken
aber die Momente und auch die Kr\"{u}mmungen gehen nur wie $-\ln\,r$ gegen Unendlich und dieses Quadrat ist noch me{\ss}bar
\begin{align}
\int_0^{\,2\,\pi} \int_0^{\,1} \ln^2\,r\,dr\,d\Np \leq \infty\,.
\end{align}

Das k\"{o}nnen wir nicht direkt tun, weil $G$ keine $C^1$ Funktion ist. Wir m\"{u}ssen also eine $\varepsilon$-Umgebung des Aufpunktes aussparen
\begin{align}
\text{\normalfont\calligra G\,\,}(G, G)_{\Omega_\varepsilon} = \int_{\Omega_\varepsilon} - \Delta G\,G\,d\Omega_{\vek y} + \int_{\Gamma}
\end{align}
und dann den Grenzprozess
\begin{align}
\lim_{\varepsilon \to 0} \text{\normalfont\calligra G\,\,}(G, G)_{\Omega_\varepsilon}
\end{align}
analysieren.

In dem gelochten Gebiet $\Omega_\varepsilon$ ist $- \Delta G = 0$, auf dem Rand $\Gamma$ ist $G = 0$, so dass von der \"{a}u{\ss}eren Arbeit nur das Integral \"{u}ber den Rand $\Gamma_{N_\varepsilon}$ des Lochs um den Aufpunkt verbleibt und dessen Grenzwert
\begin{align}
\lim_{\varepsilon \to 0} \int_{\Gamma_{N_\varepsilon}} \nabla G \dotprod  \vek n \,G\,ds = 1 \cdot \infty
\end{align}
ist unendlich gro{\ss}, genauso wie die innere Arbeit
\begin{align}
A_i = \lim_{\varepsilon \to 0} \int_{\Omega_\varepsilon } \nabla G \dotprod  \nabla G \,d\Omega = \lim_{\varepsilon \to 0}\int_0^{\,2 \pi} \int_\varepsilon^{\,R} \frac{1}{r^2} r\,dr\,d\Np = \infty\,,
\end{align}
denn das Integral
\begin{align}
\int_0^{\,1} \frac{1}{r}\,dr = \infty
\end{align}
ist unendlich.

In der Praxis macht man sich nat\"{u}rlich nicht die M\"{u}he all diese Grenzprossese genau nachzuvollziehen, sondern man schaut nur auf die innere Energie.

Die Funktionen, bei denen man vermutet, dass sie unendlich gro{\ss}e Energie haben, sind die Einflussfunktionen f\"{u}r Verschiebungen oder Spannungen, etc. Diese haben alle die Struktur
\begin{align}
r^2\,\ln r f(\Np)\qquad \ln r\,\,f(\Np) \qquad \frac{1}{r^n}\,f(\Np)
\end{align}
wobei die Potenz $n$ der St\"{a}rke der Singularit\"{a}t entspricht.\\

Die Methode des {\em goal oriented adaptive refinement\/} basiert auf diesem Fehlersch\"{a}tzer. Dabei wird das Netz adaptiv so verfeinert, dass die beiden Fehler m\"{o}glichst klein werden, \cite{Ha5}.\\

Das ist auch das mathematische Gegenargument zu dem Argument des Ingenieurs, dass das Material kl\"{u}ger sei, denn die Singularit\"{a}ten schlagen direkt bis zu den Einflussfunktionen durch und verf\"{a}lschen somit die Ergebnisse.

Wir w\"{u}rden uns aber trotzdem auf der Seite des Ingenieurs halten, weil in der Regel im Bauwesen viele andere Effekte die Genauigkeit eines FE-Modells bestimmen und man m\"{u}sste schon in einem theoretischen 'Reinraum' operieren, wenn man sich \"{u}bertriebene Gedanken \"{u}ber den Einfluss von Singularit\"{a}ten auf die FE-L\"{o}sung machen wollte, sind doch die Genauigkeitsanforderungen im Bauwesen wesentlich geringer als z.B. im Maschinenbau. Gleichwohl kann es auch in Standardf\"{a}llen, siehe die Berechnung der Einflussfunktion f\"{u}r $N_{yx}$ in Bild \ref{U200}, Probleme mit der Genauigkeit geben. Man muss also mitdenken!

\\

%%%%%%%%%%%%%%%%%%%%%%%%%%%%%%%%%%%%%%%%%%%%%%%%%%%%%%%%%%%%%%%%%%%%%%%%%%%%%%%%%%%%%%%%%%%%%%%%%%%
{\textcolor{blau2}{\section{Modellfehler und numerischer Fehler}}}
Wenn man einen konischen Schaft mit einem konstanten, mittleren Durchmesser rechnet, dann begeht man einen {\em Modellfehler\/}. Wenn man die L\"{a}ngsverschiebung in dem FE-Modell des (gleichf\"{o}rmigen) Schafts durch einen Polygonzug (= lineare Elemente) ann\"{a}hert, dann begeht man einen weiteren Fehler, den {\em numerischen Fehler\/}, so dass sich die folgende Kette von Fehlern ergibt
\begin{align}
u_{exakt} - u_{h} = \underbrace{u_{exakt} - u_{uniform}}_{Modellfehler} +  \underbrace{ u_{uniform} - u_h}_{numerischer Fehler}\,.
\end{align}
In der englischsprachigen Literatur lauten die daran ankn\"{u}pfenden Untersuchungen {\em verification and validation\/}. Verification untersucht die Gr\"{o}{\ss}e des numerischen Fehlers und validation fragt, ob \"{u}berhaupt die richtigen Gleichungen gel\"{o}st wurden. Andernfalls hat man auch auf einem noch so feinen Netz keine Chance, in die N\"{a}he der exakten L\"{o}sung zu kommen.\\

{\textcolor{blau2}{\subsubsection*{Modellfehler}}}
Den Ingenieur interessiert vor allem der Modellfehler, weil man immer das Tragwerk vereinfachen muss, 'um es in den Rechner zu bekommen'. Hier kann der Mathematiker wenig helfen. Nur der Ingenieur kann die Vereinfachungen im Modell rechtfertigen oder auf kritische Punkte im Modell hinweisen.

Der Modellfehler ist eigentlich der Ingenieurfehler schlechthin. Eigentlich interessiert den Ingenieur nur dieser Fehler. Welche Auswirkungen hat es, wenn man eine St\"{u}tze im dritten Geschoss entfernt, oder wie lagern sich die Kr\"{a}fte um, wenn man eine Wand verr\"{u}ckt?

Die Untersuchungen zu den Auswirkungen von Steifigkeits\"{a}nderungen in Kapitel \ref{Steifigkeits\"{a}nderungen} m\"{o}gen bei der Beantwortung solcher Fragen hilfreich sein.

{\textcolor{blau2}{\subsubsection*{Numerische Fehler}}}
Dagegen ist der numerische Fehler relativ gut erforscht. Er setzt sich aus zwei Anteilen zusammen

\begin{enumerate}
  \item dem Interpolationsfehler, wenn man also Kurven und Fl\"{a}chen st\"{u}ckweise durch Polynome ann\"{a}hert
  \item dem Fehler, der darauf beruht, dass man ja die Knotenwerte der Kurve, der Fl\"{a}che, die man interpolieren will, nicht kennt und sich durch L\"{o}sen des  Systems $\vek K\,\vek u = \vek f$ erst N\"{a}herungen daf\"{u}r beschaffen muss.
\end{enumerate}

Der numerische Fehler h\"{a}ngt
\begin{itemize}
  \item von der Maschenweite $h$ ab
  \item davon, wie glatt die L\"{o}sung ist, die man approximieren will. Die Biegelinie eines Balkens im LF $g$ ist glatter, als die Biegelinie unter einer Einzelkraft, weil letztere einen Knick in der zweiten Ableitung ($M = - EI\,w''$) und einen Sprung in der dritten Ableitung ($V = - EI\,w'''$) aufweist,
  \item von der Ordnung der Ableitungen ab. Je h\"{o}her die Ableitung der Zielgr\"{o}{\ss}e,  um so  gr\"{o}{\ss}er ist der Fehler. Das ergibt z.B. bei einer Platte die folgende Reihenfolge:
  \begin{itemize}
    \item $w$
    \item $w,_x$; $w,_y$
    \item $m_{xx}$, $m_{xy}$, $m_{yy}$
    \item $q_x$, $q_y$
  \end{itemize}
  \item Einflussfunktionen, die differenzieren, ungerade Ableitungen des Zielwertes, $w,_x$ und $w,y$ bzw. $q_x$ und $q_y$, sind schlechter zu approximieren als solche, die integrieren, gerade Ableitungen des Zielwertes, $w$  (0-te Ableitung) und die Momente $m_{xx}, m_{xy}, m_{yy}$ (zweite Ableitung).
\end{itemize}

Einen wichtigen Schluss kann man aus diesen Bemerkungen aber ziehen, n\"{a}mlich dass St\"{u}tzenkr\"{a}fte von einem FE-Programm gut angen\"{a}hert werden k\"{o}nnen, weil die Einflussfunktion f\"{u}r eine St\"{u}tzenkraft ja dadurch entsteht, dass man die St\"{u}tze wegnimmt und die Platte an dieser Stelle um eine L\"{a}ngeneinheit nach unten dr\"{u}ckt. Die Einflussfunktion ist praktisch die zu Eins normierte Einflussfunktion f\"{u}r die Durchbiegung am Ort der St\"{u}tze, wenn die St\"{u}tze fehlt. Eine solche Biegefl\"{a}che kann man jedoch mit einem FE-Programm gut ann\"{a}hern.

\"{A}hnliches gilt f\"{u}r frei stehende W\"{a}nde, weil die Einflussfunktion f\"{u}r die \"{u}ber die Wandl\"{a}nge integrierten St\"{u}tzkr\"{a}fte dadurch entsteht, dass man die Wand als ganzes um 1 m absenkt. Auch diese Bewegung kann man schon auf einem relativ groben Netz gut darstellen.




%-----------------------------------------------------------------
\begin{figure}[tbp]
\centering
\includegraphics[width=0.9\textwidth]{\Fpath/U235}
\caption{Schw\"{a}chung zweier Riegel und die dadurch ausgel\"{o}sten Kr\"{a}fte und Momente $f_i^+$}
\label{U235}
%
\end{figure}%
%----------------------------------------------------------------- \\

{\textcolor{blau2}{\subsubsection*{Der Zusammenhang mit den Einflussfunktionen}}}

Eine \"{A}nderung der L\"{a}ngssteifigkeit $EA$ in einem Element, f\"{u}hrt, wie wir gesehen haben, zu zwei zus\"{a}tzlichen Knotenkr\"{a}ften $\pm f_i^+$ an den Enden des Stabes. Das ist aber eine \"{a}hnliche Situation, wie bei der Berechnung der Einflussfunktion f\"{u}r die Normalkraft $N(x)$ in dem Stab, wo ja ebenfalls zwei gegengleiche Knotenkr\"{a}fte $\pm EA/l_e$ die Knoten belasten.

Verglichen mit den $\pm EA/l_e$ sind die $f_i^+$ klein, aber das ist ein einfacher Skalenfaktor und so kann man doch an Hand der Einflussfunktion f\"{u}r $N$ einen Eindruck gewinnen, wie weit die $f_i^+$ ausstrahlen.

In analoger Weise kann man bei einem Balken argumentieren. Eine \"{A}nderung von $EI $ f\"{u}hrt zu Zusatzkr\"{a}ften $f_1^+$ und $f_3^+$ und Zusatzmomenten $f_2^+$ und $f_4^+$ kNm, die man in ein symmetrisches und antimetrisches Paar aufspalten kann
und so bekommen sie \"{A}hnlichkeit mit den Knotenkr\"{a}ften $j_i$, die die Einflussfunktionen erzeugen. Die $j_i$ sind ja auch Gleichgewichtslasten.
\\

, sinngem\"{a}{\ss} also mit der Formel (hier in einer etwas symbolischen Form)
\begin{align}\label{Eq71}
w(x) = \int_0^{\,l} G_0(y,x)\,p(y)\,dy + \sum_i (X_i\,G_0'^L(y_i,x) - X_i\,G_0'^R(y_i,x))\,.
\end{align}
$G_0' (= \tan \Np)$ ist die Ableitung der Einflussfunktion, passend zu den Momenten $X_i$. Wenn die $X_i$ andere Gr\"{o}{\ss}en sind, dann muss man nat\"{u}rlich andere Werte von $G_0(y_i,x)$ abgreifen.

Diese Zweiteilung in der Summe ist n\"{o}tig, weil ja das Moment $X_i$ links nicht mit demselben Faktor $G_0'(y_i,x)$ 'weitergeleitet' wird, wie das Moment $X_i$ rechts. Die Einzelkraft $P = 1 $, die die Einflussfunktion f\"{u}r die Durchbiegung im Punkt $x $ erzeugt,  s. Bild \ref{U101}, bewirkt ja unterschiedliche Verdrehungen in dem Gelenk und das bedeutet umgekehrt, dass das Moment $X_i $ links vom Gelenk eine andere Durchbiegung im Aufpunkt $x$ erzeugt, als das gleich gro{\ss}e Moment $X_i $ rechts vom Gelenk. \\

Mit der modifizierten rechten Seite
\begin{align}
\vek K\,\vek u_c = \vek f + \vek f^+
\end{align}
kann man am Modell $\vek K$ die Knotenverformungen $\vek u_c$ des Modells $\vek K_c$ berechnen, aber nur die Knotenverformungen. Um aus $\vek u_c$ die Schnittkr\"{a}fte in den einzelnen Elementen zu berechnen, muss man nat\"{u}rlich die Steifigkeiten $EA_c$ und $EI_c$ ansetzen, die die Elemente in dem Modell $\vek K_c$ haben
\begin{align}
M_h^c(x) = - EI_c\,w_h''(x)\,.
\end{align}
Aber es gilt:

\hspace*{-12pt}\colorbox{hellgrau}{\parbox{0.98\textwidth}{Steifigkeits\"{a}nderungen bei statisch bestimmten Tragwerken bewirken keine Umlagerung der Kr\"{a}fte, sondern nur eine \"{A}nderung der Verformungen.}}\\

%-----------------------------------------------------------------
\begin{figure}[tbp]
\centering
\includegraphics[width=0.9\textwidth]{\Fpath/U99}
\caption{\"{A}nderung der L\"{a}ngssteifigkeit im mittleren Element,  \textbf{ a)} Originalsystem und Belastung,  \textbf{ b)} Einflussfunktion f\"{u}r $u(l)$,  \textbf{ c)} neue und alte L\"{o}sung im Vergleich}
\label{U99}
%
\end{figure}%
%-----------------------------------------------------------------



%%%%%%%%%%%%%%%%%%%%%%%%%%%%%%%%%%%%%%%%%%%%%%%%%%%%%%%%%%%%%%%%%%%%%%%%%%%%%%%%%%%%%%%%%%%%%%%%%%%
\textcolor{blau2}{\section{Elementares Beispiel am Stab}}
Dieselben \"{U}berlegungen gelten sinngem\"{a}{\ss} f\"{u}r \"{A}nderungen in der L\"{a}ngssteifigkeit $EA $ eines Stabes. Um den vorgelegten Stab an das Tragwerk anzuschlie{\ss}en, brauchen wir gegengleiche Knotenkr\"{a}fte $f_i^+$ oder $N_a^+$ und $N_b^+$
\begin{align}
N_a^+ = - N_b^+
\end{align}
und so folgt, dass der Einfluss dieser beiden Kr\"{a}fte
\begin{align}
&N_a^+ \cdot G(y_a,x) + N_b^+ \cdot G(y_b,x) = N_a^+ \cdot (G(y_a,x) - G(y_b,x)) \nn\\
&\simeq N_a^+ \,G'(y_a,x)\,l_e
\end{align}
proportional zur Relativverschiebung der beiden Stabenden unter der Wirkung der Einflussfunktion ist. Auch dieser Unterschied d\"{u}rfte vernachl\"{a}ssigbar sein, wenn der Aufpunkt weit genug weg liegt.\\

\hspace*{-12pt}\colorbox{hellgrau}{\parbox{0.98\textwidth}{\"{A}nderungen der L\"{a}ngssteifigkeit, $EA + \Delta EA $, in einem Element f\"{u}hren zu gegengleichen Zusatzkr\"{a}ften $f_i^+$ (in Achsrichtung) an den Elementenden.}}\\
Eine konstante Streckenlast $p$ zieht an einem Stab, s. Bild \ref{U99} a. Die \"{U}berlagerung der Einflussfunktion $G_0(l,y)$ f\"{u}r $u(l)$ mit der Belastung, s. Bild \ref{U99} b, ergibt
\begin{align}
u(l) = \int_0^{\,l} G_0(y,l)\,p(y)\,dy\,.
\end{align}
Eine Erh\"{o}hung der L\"{a}ngssteifigkeit, $EA_c > EA$, in dem zweiten Element kann durch zwei gegengleiche Knotenkr\"{a}fte $\pm f^+$ kompensiert werden, s. Bild \ref{U99} c, und somit lautet die neue L\"{a}ngsverschiebung am Stabende (berechnet mit der 'alten' Einflussfunktion)
\begin{align}
u_c(l) &= \int_0^{\,l} G_0(y,l)\,p(y)\,dy + f^+ G_0(x_a,l) - f^+ G_0(x_b,l)\nn \\
&= u(l) + f^+ (G_0(x_a,l) - G_0(x_b,l)) = u(l) - f^+ G_0'(x_a,l)\,l_e\,.
\end{align}
Die Differenz $u_c(l) - u(l)$ ist also proportional zu den Faktoren $f^+$, $G_0' = 1/EA$ und der L\"{a}nge $l_e$ des Elements. Die Abh\"{a}ngigkeit von $l_e$ best\"{a}tigt die Vermutung, dass die Effekte umso eher zu vernachl\"{a}ssigen sind, je n\"{a}her die beiden Kr\"{a}fte $f_i^+$ beieinander liegen, aber auch die Steigung $G_0' = 1/EA = 10^{-6}$ ist, wenn wir einen realistischen Wert f\"{u}r $EA$ zu Grunde legen, sehr klein. Eine kleine Steigung $G_0'$ bedeutet ja, dass sich die Werte der Einflussfunktion $G_0$ in den Fusspunkten der beiden gegengleichen Kr\"{a}fte $\pm f^+$ kaum unterscheiden.
%-----------------------------------------------------------------
\begin{figure}[tbp]
\centering
\includegraphics[width=0.7\textwidth]{\Fpath/U111}
\caption{Elementendkr\"{a}fte und Einflussfunktion $g(y)$ auf dem Element}
\label{U111}
%
\end{figure}%
%-----------------------------------------------------------------


Mit den Zahlen
\begin{align}
EA = 1.0 \cdot 10^6 \,\text{kN},\,EA_c = 2 \cdot EA\qquad  l = 3,\,l_e = 1\, \qquad p = 10\,\text{kN}/\text{m}
\end{align}
ergibt sich z.B.
\begin{align}
u_1^c &= 2.52\cdot 10^{-5}\,\text{m},\,  u_2^c = 3.25 \cdot 10^{-5}\,\text{m},\,  u_3^c = 3.74 \cdot 10^{-5}\,\text{m} \\
 f^+ &= \pm (3.25 - 2.52)\cdot 10^{-5}\,\text{m}\cdot 1.0\cdot 10^6\,\text{kN} = \pm 7.5\,\text{kNm}
\end{align}
und die Streckenlast $p$ plus den $f^+$, s. Bild \ref{U99} c, erzeugt an dem Original dieselben Knotenverschiebungen, wie $p$ alleine an dem verst\"{a}rkten Stab.
\\

Bleibt noch das Prinzip der virtuellen Verr\"{u}ckungen als (scheinbare) Alternative zum Satz von Betti, um Kraftgr\"{o}{\ss}en zu berechnen. Aber alle Einflussfunktionen f\"{u}r Kraftgr\"{o}{\ss}en, die auf dem Prinzip der virtuellen Verr\"{u}ckungen basieren, sind im Grunde Anwendung des Satzes von Betti, wie das Bild \ref{U156} demonstriert, wo, wie in Bild \ref{U148}, die Gr\"{o}{\ss}e der Lagerkraft $A$ mit dem Satz von Betti berechnet wird.

Hierzu wiederholt man im Bild \ref{U156} b das urspr\"{u}ngliche System, entfernt aber alle Lager, so dass der gewichtslose Balken frei schwebt. Eine Verdrehung um das rechte Ende
\begin{align}
w_2(x) = 1 - \frac{x}{l}
\end{align}
erfordert also keinerlei Kr\"{a}fte am System 2. Das System 1 ist der Balken links mit der Streckenlast und der Biegelinie $w(x) = w_1(x)$. Die Arbeit der Lasten am System 1 auf den Wegen des Systems 2 ist gem\"{a}{\ss} Betti gleich der Arbeit der (nicht vorhandenen) Lasten am System 2 auf den Wegen $w_1(x)$ und so folgt
\begin{align}
A_{1,2} = - A\cdot 1 + \int_0^{\,l} p(x)\,w_2(x)\,dx = \int_0^{\,l} 0\cdot w_2(x)\,dx = A_{2,1} = 0
\end{align}
oder
\begin{align}
A_{1,2} = \int_0^{\,l} p(x)\,w_2(x)\,dx\,.
\end{align}
Dass die Arbeit $A_{2,1} = 0$ ist, ist kein Zufall, sondern das ist bei der Berechnung von Einflussfunktionen f\"{u}r Kraftgr\"{o}{\ss}en mit dem Satz von Betti immer so---auch bei statisch unbestimmten Tragwerken.\\

Bei einem (statisch unbestimmten) Durchlauftr\"{a}ger braucht man nat\"{u}rlich Kr\"{a}fte, um einen Versatz der Gr\"{o}{\ss}e Eins (= Einflussfunktion f\"{u}r $V(x)$) zu erzeugen, da diese Kr\"{a}fte aber gegengleich sind, ist die Arbeit, die sie auf der Durchbiegung $w_1(x)$ leisten, null
\begin{align}
 A_{2,1} = (V_L(x) - V_R(x)) \cdot w_1(x) = 0\,.
\end{align}

\begin{remark}
Bei Drehfedern
\begin{align}
M = k_{\Np} \cdot \Np
\end{align}
hat man diese Zweideutigkeit, $\Np$ oder $\tan\,\Np$, pur, denn oft wird nicht eindeutig gesagt, welche Einheit die Drehsteifigkeit hat
\begin{align}
k_{\Np} = \left \{ \begin{array}{l } {\displaystyle  \text{{\em Moment\/}}/\text{{\em Drehwinkel\/}}  }      \\
{\displaystyle \text{{\em Moment\/}}/\text{{\em Tangens des Drehwinkels\/}}}\,.
\end{array} \right.
\end{align}
nur wegen $\tan \Np \simeq \Np$ f\"{a}llt das nicht auf. Meist ist es aber der Tangens, insbesondere, wenn es sich um den Einspanngrad handelt.
\end{remark}

%-----------------------------------------------------------------
\begin{figure}
\centering
\if \bild 2 \sidecaption \fi
{\includegraphics[width=0.7\textwidth]{\Fpath/U140}}
\caption{Bestimmung von $u_c$ des Systems $S_c$ am System $S$ mit konstantem $EA$}
\label{U140}%
%
\end{figure}%
%-----------------------------------------------------------------

%%%%%%%%%%%%%%%%%%%%%%%%%%%%%%%%%%%%%%%%%%%%%%%%%%%%%%%%%%%%%%%%%%%%%%%%%%%%%%%%%%%%%%%%%%%%%%%%%%%
\textcolor{blau2}{\subsection{Direkte Berechnung als weitere M\"{o}glichkeit}}
{\em Das geht so nicht!\/}

Eine weitere Variante, wie man sich die Dinge zurecht legen kann, ist in Bild \ref{U140} dargestellt, wo ein Stab mit zwei unterschiedlichen Steifigkeiten, $EA_1 = 1$ und $EA_2 = 2$, gezogen wird (System $S_c$).

Man l\"{o}st den Lastfall erst am System $EA_1 = EA_2 = 1$ (System $S$) und geht mit der L\"{o}sung $u$ in das System $S_c$. Das ergibt die Kr\"{a}fte in Bild \ref{U140} c, die aber nicht den Lastfall darstellen, den man l\"{o}sen wollte und die auch nicht im Gleichgewicht sind. Um das zu korrigieren l\"{o}st man am System $S$  einen Zusatzlastfall, s. Bild \ref{U140} e, und addiert dessen L\"{o}sung $u_+$ zur L\"{o}sung $u$, und am Ende hat man $u_c = u + u_+$.

%----------------------------------------------------------------------------
\begin{figure}
\centering
{\includegraphics[width=1.0\textwidth]{\Fpath/U257}}
  \caption{Einflussfunktion f\"{u}r die Eckverschiebung, \textbf{a)} Verformung der Scheibe, \textbf{b)} die Knotenverschiebungen $\vek g_i$ der Einflussfunktion, Knotenkr\"{a}fte, die in Richtung der $\vek g_i$ wirken, haben maximalen Einfluss und Kr\"{a}fte, die senkrecht auf den $\vek g_i$ stehen keinen Einfluss} \label{U257}
\end{figure}\\


%----------------------------------------------------------------------------
\begin{figure}
\centering
{\includegraphics[width=1.0\textwidth]{\Fpath/U79}}
  \caption{Plot der Knotenvektoren  $\vek g_i$ des Funktionals $J(u_h) = \sigma_{xx}$ einer punktgelagerten Scheibe. Wenn die Knotenkr\"{a}fte in Richtung der Pfeile weisen, erzielen sie die gr\"{o}{\ss}tm\"{o}glichen Wirkung hinsichtlich der Spannung $\sigma_{xx}$ im Aufpunkt. Links oberhalb des Aufpunktes liegt anscheinend ein Lagrange-Punkt}
  \label{U79}
\end{figure}

%----------------------------------------------------------------------------------------------------------
\begin{figure}[tbp]
\centering
\if \bild 2 \sidecaption \fi
\includegraphics[width=.6\textwidth]{\Fpath/U107}
\caption{Starrer Stempel auf Halbraum. An den Kanten des Stempels werden die Spannungen
unendlich gro{\ss}, weil dort die Verzerrungen im Boden unendlich gro{\ss} sind} \label{U107}
\end{figure}%%
%----------------------------------------------------------------------------------------------------------
\\

, wie etwa im Fall des B\"{u}rogeb\"{a}udes in Bild \ref{U203}, wo die Energiebilanz
\begin{align}
\frac{1}{2}\,P \cdot \Delta u = \frac{1}{2}\,\sum_i\, [ \int_0^{\,l_i} (\frac{M_i^2}{EI}+ \frac{N_i^2}{EA})\,dx \,]
\end{align}
sich nur durchhalten l\"{a}sst, wenn die $M_i$ und $N_i$ gegen Null tendieren, wenn die Zahl der Stiele und Riegel w\"{a}chst. (?) Richtig ?

%---------------------------------------------------------------------------------
\begin{figure}
\centering
{\includegraphics[width=.7\textwidth]{\Fpath/U203}}
\caption{Einzelkraft an Geb\"{a}udeecke (Stockwerkrahmen) }
\label{U203}%
%
\end{figure}%
%---------------------------------------------------------------------------------\\

\subsection{Anschauliche  Herleitung der Gleichungen}

Die Situation: In dem Dreifeldtr\"{a}ger in Bild \eqref{VerySimple1} a \"{a}ndert sich die Steifigkeit, $EI \to EI + \Delta EI$ in der Mitte des zweiten Feldes auf einem Teilst\"{u}ck $[x_a, x_b]$. Wir wollen voraussagen, welche \"{A}nderungen sich daraus f\"{u}r die Durchbiegung am Kragarmende ergibt.\\

Zun\"{a}chst sei daran erinnert, dass man am urspr\"{u}nglichen Tragwerk die Durchbiegung am Kragarmende wie folgt berechnen kann: Man bringt zun\"{a}chst eine Kraft $P = 1$ in Richtung der gesuchten Durchbiegung auf. Die zugeh\"{o}rige Biegelinie $G$ (= Greensche Funktion) ist die Einflussfunktion f\"{u}r die Durchbiegung am Kragarmende, d.h. die \"{U}berlagerung von $G$ mit der Streckenlast $p$ ergibt die Durchbiegung
\bfo
w(l) = \int_0^{\,l} G\,p\,dx \,.
\efo
Nach dem Satz von Mohr kann man aber auch statt dessen die Momente $M$ aus der Belastung und $M_G$ aus der Einflussfunktion miteinander \"{u}berlagern
\bfo
w(l) = \int_0^{\,l} \frac{M\,M_G}{EI}\,dx = \int_0^{\,l} EI\,w''\,G''\,dx\,.
\efo
Dies, wie gesagt, nur zur Erinnerung. \\

Der Index $c$ (= {\em changed\/}) bezeichne im Folgenden die Gr\"{o}{\ss}en, die sich auf das ver\"{a}nderte Tragwerk beziehen.\\

Wir benutzen die folgende Logik:\\
\begin{enumerate}
  \item Ist ein Tragwerk im Gleichgewicht, dann sind bei jeder virtuellen Verr\"{u}ckung die virtuellen \"{a}u{\ss}eren Arbeiten gleich den virtuellen inneren Arbeiten
\bfo
\delta A_a = \delta A_i\,.
\efo
  \item Dies gilt auch f\"{u}r das modifizierte Tragwerk
\bfo
\delta A_a^c = \delta A_i^c\,.
\efo
  \item Nachdem sich aber die Belastung nicht \"{a}ndert, m\"{u}ssen bei gleicher virtueller Verr\"{u}ckung der beiden Tragwerke die virtuellen \"{a}u{\ss}eren Arbeiten gleich gro{\ss} sein
\bfo
\delta A_a = \delta A_a^c\,,
\efo
und wegen des Prinzips der virtuellen Verr\"{u}ckungen, m\"{u}ssen daher auch die virtuellen inneren Arbeiten gleich gro{\ss} sein
\bfo
\delta A_i = \delta A_a = \delta A_a^c = \delta A_i\,.
\efo
\end{enumerate}
Der Anschaulichkeit halber wollen wir $\delta A_a$ und $\delta A_i$ mit 'Argumenten' schreiben, also
\bfo
\delta A_a(w,G)  \qquad \delta A_i(w,G)
\efo
Vor dem Komma steht die Biegelinie des Systems und hinter dem Komma die virtuelle Verr\"{u}ckung.
%----------------------------------------------------------------------------------------------------------
\begin{figure}[tbp]
\includegraphics[width=1.0\textwidth]{\Fpath/VERYSIMPLE1}
\caption{Berechnung der \"{A}nderung der Durchbiegung am Kragarmende infolge einer Steifigkeits\"{a}nderung im zweiten Feld.} \label{VerySimple1}
\end{figure}%
%----------------------------------------------------------------------------------------------------------

Die oben eingef\"{u}hrte Biegelinie $G$ aus der Einzelkraft \"{u}bernimmt nun die Rolle der virtuellen Verr\"{u}ckung. 'Wackelt' man also mit der Greenschen Funktion $G$ an dem urspr\"{u}nglichen System dann gilt
\bfo
\delta A_i(w,G) = \int_0^{\,l} EI\,G''\,w''\,dx = \int_0^{\,l} G\,p\,dx = \delta A_a(w,G)
\efo
und am modifizierten System bei derselben virtuellen Verr\"{u}ckung
\bfo
\delta A_i^c(w_c,G) = \int_0^{\,l} EI_c\,G''\,w_c''\,dx = \int_0^{\,l} G\,p\,dx = \delta A_a^c(w_c,G)\,.
\efo
Nun ist die Biegesteifigkeit $EI_c$ am modifizierten System bis auf den Abschnitt $[x_a,x_b]$ mit dem urspr\"{u}nglichen $EI$ identisch und somit folgt
\bfo
\delta A_i^c(w_c,G) = \int_0^{\,l} EI_c\,G''\,w_c''\,dx = \int_0^{\,l} EI\,G''\,w_c''\,dx + \int_{x_a}^{\,x_b} \Delta EI\,G''\,w_c''\,dx\,.
\efo
Das erste Integral auf der rechten Seite ist gerade $w_c(l)$, s. Anhang, und damit folgt
\bfo
\delta A_i^c(w_c,G) = w_c(l) + \int_{x_a}^{\,x_b} \Delta EI\,G''\,w_c''\,dx
\efo
und wegen
\begin{align}
\delta A_i^c(w_c,G) &=  w_c(l) + \int_{x_a}^{\,x_b} \Delta EI\,G''\,w_c''\,dx \nn \\
&= \delta A_a^c(w_c,G) = \delta A_a(w,G) = \delta A_i(w,G) = w(l)\nn
\end{align}
schlie{\ss}lich das zentrale Ergebnis
\bfo
w_c(l) - w(l) = - \int_{x_a}^{\,x_b} \Delta EI\,G''\,w_c''\,dx\,.
\efo
Nun soll noch eine Vereinfachung vorgenommen werden. Um diese Formel auszuwerten, muss man die beiden Funktionen $G$ und $w_c$ kennen. Die Biegelinie $w_c$ muss an dem modifizierten Tragwerk berechnet werden. Wenn man aber die Gleichungen f\"{u}r das modifizierte Tragwerk aufstellt, dann braucht man diese Formel aber nicht mehr, denn dann erh\"{a}lt man alle interessierenden Ergebnisse automatisch.

Wir ersetzen daher $w_c$ im Abschnitt $[x_a, x_b]$ durch die urspr\"{u}ngliche Biegelinie $w$ und kommen so zu der N\"{a}herungsformel
\bfo
w_c(l) - w(l) \simeq - \int_{x_a}^{\,x_b} \Delta EI\,G''\,w''\,dx\,.
\efo
Setzt man noch
\bfo
M =  - EI\,w'' \qquad M_G = - EI\,G''\,,
\efo
so erh\"{a}lt man schlie{\ss}lich das Ergebnis
\bfo
w_c(l) - w(l) \simeq - \frac{\Delta EI}{EI}\int_{x_a}^{\,x_b} \frac{M\,M_G}{EI}\,dx\,.
\efo

{\bf Anhang\/}\\

'Wackelt' man mit der Biegelinie $w_c$ am System mit der Einzellast $P = 1$, dann muss dabei die \"{a}u{\ss}ere virtuelle Arbeit $\delta A_a(G,w_c) = P \times w_c(l)$ gleich der inneren virtuellen Arbeit sein, also
\bfo
\delta A_a(G,w_c) = 1 \times  w_c(l) = \int_0^{\,l} EI\,G''\,w_c''\,dx = \delta A_i(G,w_c)\,.
\efo

Was wir hier am Durchlauftr\"{a}ger erl\"{a}utert haben, gilt sinngem\"{a}{\ss} f\"{u}r alle Tragwerke.
Drei Faktoren bestimmen, welche Auswirkung eine \"{A}nderung $\Delta EI$ der Steifigkeit $EI$ auf eine Gr\"{o}{\ss}e $O$ hat
\bfo
O_c - O = \frac{\Delta EI}{EI}\int_{x_a}^{\,x_b} \frac{M \, M_G}{EI} \,dx = \frac{\Delta EI}{EI} \cdot \delta A_i\,.
\efo
Man \"{u}berlagert eine {\em  Wichtungsfunktion\/}, n\"{a}mlich $M_G$, mit dem Moment $M$ aus der Belastung und wichtet das Ganze dann noch einmal mit dem Faktor $\Delta EI/EI$, der ein Ma{\ss} f\"{u}r die relative \"{A}nderung der Steifigkeit ist.

Noch allgemeiner gesagt ist die Formeln von der Bauart
\bfo
O_c - O = \mbox{rel. Steifigkeits\"{a}nderung} \times \mbox{virt. innere Energie}\,.
\efo

%%%%%%%%%%%%%%%%%%%%%%%%%%%%%%%%%%%%%%%%%%%%%%%%%%%%%%%%%%%%%%%%%%%%%%%%%%%%%%%%%%%%%%%%%%%%%%%%%%%
\textcolor{blau2}{\section{Sensitivit\"{a}tsanalyse}}
Es gibt nun noch ein nah verwandtes Thema, bei dem die Berechnung von Einflussfunktionen nach {\em Mohr\/} eine gro{\ss}e Rolle spielt. In einem Rahmen \"{a}ndere sich die Biegesteifigkeit $EI$ in einem Stiel oder Riegel, $EI \to EI + \Delta EI$, dann kann man zeigen, dass die \"{A}nderung einer interessierenden Weg- oder Schnittgr\"{o}{\ss}e, z.B. des Moments $M + \Delta M$ an einer Stelle $x$, gleich dem Integral
\begin{align}
\Delta M(x) = \frac{\Delta EI}{EI} \int_0^{\,l} \frac{\bar{M}\,M_c}{EI}\,dx
\end{align}
ist. Hier ist $\bar{M}$ das Moment, das zur Einflussfunktion geh\"{o}rt und $M_c$ ist das Lastmoment in dem betroffenen Bauteil---nach der Steifigkeits\"{a}nderung---und es gilt:\\

\hspace*{-12pt}\colorbox{hellgrau}{\parbox{0.98\textwidth}{Bei der Auswertung wird nur \"{u}ber das betroffene Element integriert!}}\\

Was bedeutet das? Angenommen im 3. Stock reduziert sich in einem Riegel die Biegesteifigkeit um die H\"{a}lfte, $\Delta EI = 0.5 \cdot EI$. Welche Auswirkungen hat das auf das Biegemoment $M(x)$ in einem Riegel im 5. Stock? Theoretisch geht man wie folgt vor:
\begin{enumerate}
  \item Man ermittelt das Biegemoment aus der Last in dem Riegel im 3. Stock. Das ist der Verlauf $M_c(x)$. Der Index $c$ soll darauf hindeuten, dass es das Moment {\em nach\/} der Reduktion der Steifigkeit in dem Riegel ist.
  \item Man stellt die Einflussfunktion f\"{u}r das Biegemoment $M(x)$ im 5. Stock auf, $x$ sei die Mitte des interessierenden Riegels, und bestimmt den Verlauf, den die Momente $\bar{M}_M(x)$ aus dieser Einflussfunktion in dem Riegel im 3. Stock haben.
  \item Im letzten Schritt \"{u}berlagert man diese beiden Momente
\begin{align}
\Delta M(x) = 0.5 \int_{Riegel\,\,im\,\,3.\,\,Stock} \frac{\bar{M}_M\,M_c}{EI}\,dx\,.
\end{align}
   Das Ergebnis ist die \"{A}nderung $\Delta M(x)$ des Biegemomentes im Riegel im 5. Stock.
\end{enumerate}

Wenn man wissen will, wie sich die Querkraft $V(x)$ in der Riegelmitte im 5. Stock \"{a}ndert, dann bestimmt man die Einflussfunktion f\"{u}r $V(x)$ und \"{u}berlagert deren Moment $\bar{M}_V(x)$ im Riegel im 3. Stock mit dem Lastmoment
\begin{align}
\Delta V(x) = 0.5 \int_{Riegel\,\,im\,\,3.\,\,Stock} \frac{\bar{M}_V\,M_c}{EI}\,dx\,,
\end{align}
und so durch alle interessierenden Gr\"{o}{\ss}en.

Wenn in einem betroffenen Bauteil $\bar{M}$ und $M_c$ beide gro{\ss} sind, dann haben \"{A}nderungen $EI \to EI + \Delta EI$ sp\"{u}rbaren Einfluss, wenn dagegen eine oder beide Gr\"{o}{\ss}en klein sind, dann ist der Einfluss eher vernachl\"{a}ssigbar.

Das ganze ist nat\"{u}rlich theoretisch, weil man das Biegemoment $M_c(x)$ in dem Riegel im 3. Stock nach der Steifigkeits\"{a}nderung braucht und um das zu bestimmen, muss man den Rahmen mit der ge\"{a}nderten Steifigkeit, $EI \to 0.5 \cdot EI$ im Riegel im 3. Stock, neu durchrechnen. Dann wei{\ss} man aber nat\"{u}rlich auch, wie gro{\ss} die \"{A}nderung im 5. Stock ist.

Um daraus ein praktisches Werkzeug zu machen, muss man den Verlauf von $M_c(x)$ durch eine Sch\"{a}tzung ersetzen
\begin{align}
M_c(x) \simeq \alpha \cdot M(x)\,,
\end{align}
wobei die Wahl des Parameters $\alpha$ dem Geschick und der Erfahrung des Ingenieurs \"{u}berlassen bleibt
\begin{align}
\Delta M(x) \simeq \frac{\Delta EI}{EI} \int_0^{\,l} \frac{\bar{M}\,\alpha @ M}{EI}\,dx \qquad \alpha @ M \simeq M_c\,.
\end{align}
F\"{u}r Hinweise darauf, wie man den Faktor $\alpha$ geeignet absch\"{a}tzt, s. \cite{Ha5} Kapitel 3.8 {\em sensitivity analysis\/}.

Bei dem Rahmen in Bild   \ref{U186} soll der Einfluss einer Steifigkeits\"{a}nderung in einem der Riegel und Stiele des Rahmens auf das Fusspunktsmoment im linken Stiel \"{u}berschl\"{a}gig erfasst werden.


In Bild \ref{U186} a sieht man die Momente $M$ aus dem Wind und in Bild \ref{U186} b die Momente $\bar{M}$ der Einflussfunktion f\"{u}r das Fu{\ss}punktsmoment. In Gedanken kann man nun von Bauteil zu Bauteil gehen und so abw\"{a}gen, welchen Effekt eine \"{A}nderung $EI \to EI + \Delta EI$ in dem Stiel oder Riegel auf das Fu{\ss}punktsmoment haben wird. Es sind vor allem, wie man sieht, die beiden unteren Stiele, deren Steifigkeiten $EI$ am wichtigsten f\"{u}r die Gr\"{o}{\ss}e des Fu{\ss}punktsmoments sind.



%----------------------------------------------------------------------------------------------------------
\begin{figure}[tbp]
\includegraphics[width=1.0\textwidth]{\Fpath/VERYSIMPLE2}
\caption{Windlast auf einen Rahmen, {\bf a} Momente aus $p$, {\bf b}
Einflussfunktion f\"{u}r das Moment im Fusspunkt des rechten Stiels, {\bf c}
Momente aus der Einflussfunktion} \label{VerySimple2}
\end{figure}%
%----------------------------------------------------------------------------------------------------------

\begin{remark}
Diese 'Schaukellogik'\index{Schaukellogik} macht auch deutlich, dass der naive Gleichgewichtsbegriff, der von der Gleichheit der Kr\"{a}fte ausgeht, die links und rechts am Seil ziehen, s. Bild \ref{U220}, nur ein Spezialfall eines allgemeineren und weiter reichenden Gleichgewichtsbegriffs ist: Ein Getriebe, eine 'Schaukel' an der nicht notwendig gleich gro{\ss}e Kr\"{a}fte, $\sum H \neq 0, \sum V \neq 0$, angreifen, ist im Gleichgewicht, wenn die Kr\"{a}fte bei einer virtuellen Verr\"{u}ckung die gleiche Arbeit leisten. Dem \"{U}bersetzungsverh\"{a}ltnis des 'Getriebes', man denke an einen Flaschenzug, kommt also eine ma{\ss}gebende Rolle.

Weil man jede Scheibe durch Entfernen der Lager zu einem 'Ein-St\"{u}ck-Getriebe' machen kann, gilt das auch f\"{u}r starre K\"{o}rper. Nur ist es so, dass es bei diesen nur die Starrk\"{o}rperbewegungen $\vek u_0 = \vek a + \vek b \times \vek x$ als m\"{o}gliche Bewegungen gibt, woraus dann die bekannten Gleichgewichtsbedingungen  $\sum H = 0, \sum V = 0, \sum M = 0$ folgen.
\end{remark}


Es gibt aber noch eine weitere Methode der Mittelbildung und zwar das Umrechnen der Resultate in \"{a}quivalente Knotenkr\"{a}fte, wie es die FE-Programme in Punktlagern machen. Das ist so \"{a}hnlich wie beim Gl\"{a}tten einer Funktion $f(x) \to \bar{f}(x)$ mittels einer Gewichtsfunktion $\Np(x)$
\begin{align}
\bar{f}(x) = \int_0^{\,l} f(x)\,\Np(x)\,dx\,.
\end{align}
Ein Punktlager wird von einem FE-Programm ja nicht als Punktlager gerechnet, in dem Sinne, dass dort eine echte Einzelkraft die Scheibe st\"{u}tzt, sondern die Haltekr\"{a}fte sind vielmehr Fl\"{a}chen- und Linienkr\"{a}fte (letztere zwischen den Elementen), die prim\"{a}r \"{u}ber die Elemente verteilt sind, die den Lagerknoten enthalten. Aus der Ferne m\"{o}gen diese Kr\"{a}fte den Eindruck einer einzelnen Haltekraft vermitteln, aber statisch gelingt die Umwandlung in \"{a}quivalente Knotenkr\"{a}fte (horizontal und vertikal) nur, wenn man den Knoten um eine L\"{a}ngeneinheit in die entsprechende Richtung verr\"{u}ckt und die Arbeit z\"{a}hlt, die die Haltekr\"{a}fte dabei leisten.\\

Jetzt kann man fragen, wozu braucht man dann noch die Greenschen Identit\"{a}ten? Nun die Identit\"{a}ten bilden die Grundlage der Statik der Kontinua. An den Identit\"{a}ten kann man ablesen, was als \"{a}u{\ss}ere Arbeit z\"{a}hlt, ablesen, dass $M$ und $\tan \Np$ zueinander konjugiert oder dass die Querkraft $ V = - EI\,w'''$ lautet, und dass sie zu $w$ konjugiert ist, etc. Der Ingenieur macht das automatisch richtig, weil man ihm das so beigebracht hat, aber die Regeln daf\"{u}r, die stehen in den Greenschen Identit\"{a}ten.
\\

%%%%%%%%%%%%%%%%%%%%%%%%%%%%%%%%%%%%%%%%%%%%%%%%%%%%%%%%%%%%%%%%%%%%%%%%%%%%%%%%%%%%%%%%%%%%%%%%%%%
{\textcolor{blau2}{\section{Die Elemente $k_{ij}$ einer Steifigkeitsmatrix}}}\index{Elemente einer Steifigkeitsmatrix}
Wir wollen uns kurz die beiden Bedeutungen klar machen, die das Elemente $k_{ij}$ einer Steifigkeitsmatrix haben kann. Wegen $\delta A_i = \delta A_a$ kann $k_{ij}$ als (virtuelle) innere Arbeit gelesen werden oder als (virtuelle) \"{a}u{\ss}ere Arbeit.

%---------------------------------------------------------------------------------
\begin{figure}
\centering
{\includegraphics[width=0.8\textwidth]{\Fpath/U70}}
  \caption{Stab, Situation in der N\"{a}he des Knotens $i$, \textbf{ a)} Einheitsverformungen (L\"{a}ngsverschiebungen, hier als Funktionen nach oben abgetragen)
  \textbf{ b)} Kr\"{a}fte, die die Einheitsverformung $\Np_i(x)$ bewirken}
  \label{U70}
%
\end{figure}
%---------------------------------------------------------------------------------
Beginnen wir mit der inneren Energie. Das Element $k_{ij}$ ist zun\"{a}chst definiert als die Wechselwirkungsenergie zwischen den beiden Einheitsverformungen $\Np_i(x)$ und $\Np_j(x)$, s. Bild \ref{U70},
\begin{align}
k_{ij} = a(\Np_i,\Np_j) = \int_0^{\,l} EA\,\Np_i'(x)\,\Np_j'(x)\,dx \qquad \text{(Stab)}\,.
\end{align}
F\"{u}r das weitere nehmen wir nun die erste Greensche Identit\"{a}t zwischen $\Np_i(x)$ und $\Np_j(x)$ zu Hilfe und beachten, dass wir die Formulierung in Teilen vornehmen m\"{u}ssen, weil in dem Knoten  $x_i$  eine Einzelkraft $A$ angreift. Das ist die Kraft, die die Einheitsverschiebung $u_i = 1$ des Knotens bewirkt. Ferner wissen wir, dass die $\Np_i(x)$ homogene L\"{o}sungen der Stabgleichung sind, $- EA\,\Np_i'' = 0$. Es ergibt sich somit
\begin{align}
\text{\normalfont\calligra G\,\,}(\Np_i,\Np_j) &= \text{\normalfont\calligra G\,\,}(\Np_i,\Np_j)_{(0,x_i)} +
\text{\normalfont\calligra G\,\,}(\Np_i,\Np_j)_{(x_i,l)} \nn \\
&= \underbrace{B \cdot\Np_j(x_{i-1}) + A\cdot \Np_j(x_i) + C\cdot\Np_j(x_{i+1})}_{\delta A_a} -  k_{ij} = 0\,.
\end{align}
Die innere Arbeit $k_{ij}$ ist also gleich der \"{a}u{\ss}eren Arbeit, die die Kr\"{a}fte $A, B, C$ auf den Knotenbewegungen
der Verr\"{u}ckung $\Np_j(x)$ leisten.

Insbesondere ist
\begin{alignat}{3}
k_{i,\,i-1} &= B \cdot \Np_{i-1}(x_{i-1}) &&= B \cdot 1 = - \frac{EA}{l_e} \\
k_{i\,i} &= A \cdot \Np_{i}(x_{i}) &&= A \cdot 1 = 2\, \frac{EA}{l_e} \\
k_{i,\,i+1} &= C \cdot \Np_{i+1}(x_{i+1}) &&= C \cdot 1 = - \frac{EA}{l_e}\,.
\end{alignat}
Alle anderen $k_{ij}$ in der Zeile $i$ sind Null, weil die weiter abliegenden Funktionen $\Np_j(x)$ nicht dazu kommen, die drei Kr\"{a}fte $A, B, C$ zu verschieben.

Diese Resultate kann man nun wie folgt zusammenfassen:

\begin{itemize}
\item $k_{ij}$ ist eine innere Energie
\begin{align}
\delta A_i = \int_0^{\,l} EA\,\Np_i'(x)\,\Np_j'(x)\,dx = \int_0^{\,l} N_i(x)\,d\,\Np_j\,.
\end{align}
Es ist die Arbeit, die die Normalkraft $N_i(x) = EA\,\Np_i'(x)$ auf den Wegen $d\,\Np_j$ leistet.

\item Wegen $\delta A_i = \delta A_a$ ist $k_{ij}$ aber auch gleich der virtuellen \"{a}u{\ss}eren Arbeit, die die Knotenkr\"{a}fte $A, B, C$ bei der Verschiebung $\Np_j$ der Knoten leisten.
\item Die Elemente $k_{ii}$ auf der Hauptdiagonalen sind die Kr\"{a}fte, die man f\"{u}r
\begin{align}
u_i = 1 \qquad u_j = 0 \qquad j \neq i\,.
\end{align}
braucht.
\item Die Elemente $k_{ij}$ (in derselben Zeile) sind die 'Haltekr\"{a}fte', also die Kr\"{a}fte, die n\"{o}tig sind, um die durch $u_i = 1$ ausgel\"{o}ste Bewegung an den n\"{a}chsten Knoten zum Stillstand zu bringen, $u_j = 0$.
\end{itemize}
\begin{remark}
Bei 'echten' finiten Elementen, also bei $2-D$ und $3-D$ Problemen sind die treibenden und die haltenden Kr\"{a}fte keine Einzelkr\"{a}fte, sondern 'Wolken' von Fl\"{a}chenkr\"{a}ften und Kantenkr\"{a}ften auf den Kanten zwischen dem Element, auf dem der Knoten $\vek x_i$ liegt und den Nachbarelementen.

Wir werden weiter unten im Text die treibenden und haltenden Kr\"{a}fte, die zur Einheitsverformung $\Np_i(x)$ geh\"{o}ren {\em shape forces\/} nennen und mit $\vek p_i$ bezeichnen. Das k\"{o}nnen Fl\"{a}chenkr\"{a}fte, Linienkr\"{a}fte,  Einzelkr\"{a}fte etc. sein, alles kann in $\vek p_i$ vertreten sein. Der Ausdruck
\begin{align}
f_{ij} = \delta A_a(\vek p_i,\vek \Np_j)
\end{align}
soll dann die Arbeit sein, die die Kr\"{a}fte $\vek p_i$ auf den Wegen $\vek \Np_j$ leisten. Es ist nat\"{u}rlich $k_{ij} = f_{ij}$ gem\"{a}{\ss} dem Motto 'innen = au{\ss}en'. Die $f_j$ sind hier doppelt-indiziert, weil sie aus verschiedenen Lastf\"{a}llen $\vek p_i$ kommen.
In dieser Notation gilt f\"{u}r die Kr\"{a}fte $A, B, C$ in Bild \ref{U70}
\begin{align}
A &= f_{ii} = \delta A_a(\vek p_i,\vek \Np_i) \quad B = f_{i,i-1} =\delta A_a(\vek p_i,\vek \Np_{i-1})\nn \\
C &= f_{i,i-1} =\delta A_a(\vek p_i,\vek \Np_{i+1})\,.
\end{align}
  \end{remark}

%%%%%%%%%%%%%%%%%%%%%%%%%%%%%%%%%%%%%%%%%%%%%%%%%%%%%%%%%%%%%%%%%%%%%%%%%%%%%%%%%%%%%%%%%%%%%%%%%%%
{\textcolor{blau2}{\section{Reduktion in die Knoten}}}\index{Reduktion in die Knoten}
Die Knoteneinheitsverformungen $\Np_i$ bestimmen die \"{a}quivalenten Knotenkr\"{a}fte $f_i$
\begin{align} \label{Eq59}
f_i = \int_0^{\,l} p(x)\,\Np_i(x)\,dx\,.
\end{align}
Mit den $\Np_i$ kann man also die Belastung, wie man sagt, in die Knoten reduzieren.

Insbesondere erh\"{a}lt man also die Festhaltekr\"{a}fte $\times (-1)$ am beidseitig eingespannten Balkenelement durch \"{U}berlagerung der Belastung mit den Elementeinheitsverformungen $\Np_i^e(x)$, s. (\ref{Eq219}),
\begin{align}\label{Eq60}
f_i^e = \int_0^{\,l} p(x)\,\Np_i^e(x)\,dx\,.
\end{align}
 Die so berechneten $f_i^e$ sind die 'aktiven', die 'treibenden' Knotenkr\"{a}fte, w\"{a}hrend ja die Festhaltekr\"{a}fte, wie ihr Name schon sagt, diesen Kr\"{a}ften das Gleichgewicht halten und deswegen der Faktor $(-1)$.

 Die $f_i$ in (\ref{Eq59}) (ohne den Index $e$) enthalten die Beitr\"{a}ge von dem Element links und rechts von dem Knoten, weil sich ja die $\Np_i(x)$ \"{u}ber
 die ganze Umgebung des Knotens erstrecken, w\"{a}hrend die $f_i^e$ nur auf ein Element schauen, deswegen der Index $e$. Die $f_i^e$ findet man in den Handb\"{u}chern f\"{u}r die gebr\"{a}uchlichsten Lastf\"{a}lle tabelliert, aber im Grunde deckt (\ref{Eq60}) die ganze Vielfalt an m\"{o}glichen Lastf\"{a}llen ab.

 Wirkt eine Einzelkraft $P$ bzw. ein Moment $M$ in einem Punkt $x$, dann erh\"{a}lt man die zugeh\"{o}rigen $f_i^e$ einfach durch Auswertung im Punkt
 \begin{align}
 f_i^e = \Np_i^e(x) \cdot P \qquad  f_i^e = {\Np_i^e} '(x) \cdot M\,.
 \end{align}

\begin{align}
\vek K^{(-1)}\,\vek \Delta \vek K = \left[ \barr {r @{\hspace{2mm}}r }
      2 & -1  \\
      -1 & 2
     \earr \right]\left \,\left[ \barr {r @{\hspace{2mm}}r }
      1 & -1  \\
      -1 & 1
     \earr \right] = \left[ \barr {r @{\hspace{2mm}}r }
      0.33 & -0.33  \\
      -0.33 & 0.33
     \earr \right]
\end{align}

%----------------------------------------------------------------------------------------------------------
\begin{figure}[tbp]
\centering
\if \bild 2 \sidecaption \fi
\includegraphics[width=0.7\textwidth]{\Fpath/U267}
\caption{Moment einer Kraft um einen Punkt, $|\vek M|$ = Arbeit von $\vek P$ bei einer (Pseudo)-Drehung} \label{U267}
%
\end{figure}%
%----------------------------------------------------------------------------------------------------------

\begin{remark}
Das Operieren mit Pseudodrehungen ist keine Eigenheit der Stabstatik, sondern sie steckt schon von Anfang an in der Mechanik. Das Moment $\vek M = \vek r \times \vek P$ einer Kraft $\vek P$ um einen Punkt $\vek x$ ist dem Betrage nach, $
|\vek M| =  | \vek r | \cdot | \vek P| \cdot \sin\,\Np
$,
gleich der Arbeit, die die Kraft $\vek P$ bei einer Pseudodrehung leistet, s. Bild \ref{U267}.
\end{remark}

%---------------------------------------------------------------------------------
\begin{figure}
\centering
{\includegraphics[width=.7\textwidth]{\Fpath/U157}}
\caption{Einflussfunktion f\"{u}r die Querkraft $V$ in einem Seil. Auf der Strecke zwischen den beiden Kr\"{a}ften ist die Energie sehr gro{\ss}. In der Grenze, $\Delta x \to 0$, bleibt davon nichts \"{u}brig, weil alle Energie dazu benutzt wird, dass Seil auseinander zu rei{\ss}en (Versatz in der Biegelinie). Die Vorspannkraft $H$ spielt die Rolle der Steifigkeit beim Seil }
\label{U157}%
%
\end{figure}%
%---------------------------------------------------------------------------------

Wie ist das nun bei Einflussfunktionen? Bei Einflussfunktionen f\"{u}r Weggr\"{o}{\ss}en sind die 'treibenden Kr\"{a}fte' Einzelkr\"{a}fte $P = 1$ oder Momente $M = 1$. Es ist nicht viel Energie, die in das Tragwerk flie{\ss}t. Aber bei Einflussfunktionen f\"{u}r Schnittgr\"{o}{\ss}en scheint das anders. Die Kr\"{a}fte und Momente, die einen Knick oder Verschiebungssprung hervorrufen, sind unendlich gro{\ss}, aber trotzdem ist die Energie in dem Tragwerk endlich.

Das R\"{a}tsel l\"{o}st sich, wenn man sich das Bild \ref{U157} anschaut. Die unendlich gro{\ss}e Energie der Kr\"{a}fte $1/\Delta x$ steckt am Schluss in dem Versatz $[[w]] = 1$, ist sozusagen 'eingefroren', w\"{a}hrend die restliche Biegelinie eine endliche Energie hat
\begin{align}
A_i = \frac{1}{2}\,\int_0^{\,x} H\,(w')^2\,dx + \frac{1}{2}\,\int_x^{\,l} H\,(w')^2\,dx\,.
\end{align}

Das waren nun alles Ableitungen des Kerns $G_0(y,x)$ nach $x$. Es gibt aber auch Ableitungen des Kerns $G_0(y,x)$ nach $y$ und zwar, wenn
z.B. ein Einzelmoment $M$ angreift. Dann muss man die Ableitung von $G_0(y,x)$ nach $y$ mit $M$ multiplizieren
\begin{align}
w(x) = \frac{d}{dy}\,G_0(y,x)\,M\,,
\end{align}
denn ein Moment $M$ gleicht zwei dicht nebeneinander stehenden Einzelkr\"{a}ften
\begin{align}
\pm P = \frac{1}{\Delta y} \cdot M\,,
\end{align}
deren Abstand $\Delta y$ gegen Null geht
\begin{align}
w(x) = \lim_{\Delta y \to 0}\, (G_0(y + 0.5\,\Delta y,x)  -  G_0(y - 0.5\,\Delta y,x)) \cdot \frac{1}{\Delta y} \cdot M\,.
\end{align}\\

Man kann es auch so verstehen: Die klassische Statik kennt Einflussfunktionen f\"{u}r Lagerkr\"{a}fte oder Einspannmomente bei Stabtragwerken, aber sie kennt keine Einflussfunktionen f\"{u}r \"{a}quivalente Knotenkr\"{a}fte, weil dieser Begriff in der klassischen Statik direkt nicht vorkommt und solche Knotenkr\"{a}fte nur 'Rechengr\"{o}{\ss}en' sind.
Zur Illustration dieses Ph\"{a}nomens ist in Bild \ref{U113} die Situation dargestellt, dass sich in einem finiten Element einer Wandscheibe die Steifigkeit \"{a}ndert und somit in den vier Knoten des Elementes Zusatzkr\"{a}fte $f_i^+$ angreifen. Diese Kr\"{a}fte m\"{u}sste man also zur rechten Seite $\vek K\,\vek u = \vek f$ dazu addieren, um mit der urspr\"{u}nglichen Steifigkeitsmatrix den Verschiebungsvektor $\vek u_c$ des ge\"{a}nderten Modells zu berechnen, $\vek K\,\vek u_c = \vek f + \vek f^+$.
\\

Wenn wir das formalisieren, dann setzt der Ingenieur die L\"{o}sung aus $\vek K\,\vek u = \vek f$ und $\vek K_c\,\vek u_x = \vek f_x$ zusammen, wobei $\vek f_x$ der Lastvektor ist, der nur die umgedrehte St\"{u}tzenkraft enth\"{a}lt und $\vek u_x$ ist die zugeh\"{o}rige L\"{o}sung am modifizierten System, Matrix $\vek K_c$.

Zur Berechnung von $\vek u_c$
\begin{align}
\vek K\,\vek u_c = \vek f + \vek f^+
\end{align}
ben\"{o}tigt man den Vektor $\vek u_c$ {\em nach\/} der Steifigkeits\"{a}nderung, denn $\vek f^+ = \vek \Delta \,\vek K\,\vek u_c$.
%-----------------------------------------------------------------
\begin{figure}[tbp]
\centering
\includegraphics[width=0.9\textwidth]{\Fpath/U276}
\caption{Riegel mit Streckenlast \textbf{ a)} Verformungen  \textbf{ b)} nach Ausfall der St\"{u}tze \textbf{ c)} Berechnung von $\vek u_c$ am urspr\"{u}nglichen System \textbf{ d)} Ergebnis des Ingenieurs}
\label{U276}
%
\end{figure}%
%-----------------------------------------------------------------

Die Idee liegt nahe f\"{u}r $\vek u_c$ den alten Vektor $\vek u$ zu setzen, also mit einem Vektor
\begin{align}
\vek f_x = \vek \Delta \,\vek K\,\vek u
\end{align}
zu rechnen. Das ist im Grunde die Vorgehensweise des Ingenieurs. Wenn eine St\"{u}tze ausf\"{a}llt, dann dreht er die St\"{u}tzkraft um und bringt sie als Zusatzbelastung auf das Tragwerk auf. Das ist die Kraft $\vek f_x$. Der Ingenieur ermittelt also $\vek u_c$ n\"{a}herungsweise aus der Gleichung
\begin{align}
\vek K\,\vek u_c \simeq \vek f + \vek f_x\,.
\end{align}
Genauer gesagt, er l\"{o}st $\vek K\,\vek u = \vek f$ und addiert zu $\vek u$ die L\"{o}sung des Systems $\vek K\vek u_x = \vek f_x$, was in der Summe
$\vek  u_c \simeq \vek u + \vek u_x$ entspricht.
%-----------------------------------------------------------------
\begin{figure}[tbp]
\centering
\includegraphics[width=0.9\textwidth]{\Fpath/U186}
\caption{Momente $M$ \textbf{ a)} aus Wind und \textbf{ b)} Momente $\bar{M}$ der Verdrehung des Fu{\ss}punktes, Einflussfunktion f\"{u}r Fu{\ss}punktsmoment. Stabweise wird das Integral von $M$ und $\bar{M}$ mit dem Quotienten $\Delta EI/EI$ gewichtet und bestimmt so den Einfluss, den eine \"{A}nderung $\Delta EI$ in dem Stab auf das Fu{\ss}punktsmoment hat. Genau genommen m\"{u}sste man $M$ durch $M_c$ ersetzen, aber dies kann durch einen Korrekturfaktor, $M_c \simeq \alpha M$ n\"{a}herungsweise ausgeglichen werden}
\label{U186}
%
\end{figure}%
%-----------------------------------------------------------------

Betrachten wir ein Beispiel! In dem Rahmen in Bild \ref{U276} f\"{a}llt die Mittelst\"{u}tze aus und die Absenkung in der Riegelmitte betr\"{a}gt danach $u_c = 750$\,mm, was rechnerisch die Kraft $f^+$ ergibt\footnote{Das zweite Minus ist dem Verlust der Steifigkeit geschuldet.}
\begin{align}
f^+ = - (- \frac{EA}{l}) u_c = \frac{EA}{l}\,u_c = \frac{1.07\cdot 10^6}{4} \,0.75\,\text{m} = 200\,000\,\text{kN}\,.
\end{align}
Diese wird als Zusatzkraft auf den Riegel des alten Systems aufgebracht
\begin{align}
\vek K\,\vek u_c = \vek f + \vek f^+
\end{align}
und dabei ergibt sich genau der richtige Wert $u_c$ f\"{u}r die Durchbiegung des Riegels am modifizierten System.

Der Ingenieur setzt einfach die St\"{u}tzkraft als Riegellast auf den Rahmen, $f_x = 225$ kN, und kommt so auf den Wert $u_c \simeq u + u_x = 0 + 674$\,mm statt exakt $u_c = 750$\,mm. F\"{u}r eine Absch\"{a}tzung von Effekten, die durch Steifigkeits\"{a}nderungen hervorgerufen werden, ist diese Methode also ausreichend, man bringt einfach die St\"{u}tzenkraft auf das modifizierte System auf.
\\

%%%%%%%%%%%%%%%%%%%%%%%%%%%%%%%%%%%%%%%%%%%%%%%%%%%%%%%%%%%%%%%%%%%%%%%%%%%%%%%%%%%%%%%%%%%%%%%%%%%
{\textcolor{blau2}{\section{Zusammenfassung}}}
In diesem Kapitel haben wir die folgenden Punkte behandelt:
\begin{itemize}
  \item Differentialgleichungen der Stabstatik
  \item Die zugeh\"{o}rigen Greenschen Identit\"{a}ten
  \item Die Herleitung des Prinzips der virtuellen Verr\"{u}ckungen
\end{itemize}

%---------------------------------------------------------------------------------
\begin{figure}
\centering
{\includegraphics[width=0.8\textwidth]{\Fpath/U278}}
  \caption{In diesem Bild kommt die 'Doppelb\"{o}digkeit' der Statik mit finiten Elementen sehr gut zum Ausdruck: der Ingenieur hat kein arges, die \"{a}quivalenten Knotenkr\"{a}ften in den Lagerknoten f\"{u}r real zu nehmen}
  \label{U278}
%
\end{figure}
%---------------------------------------------------------------------------------

Auch bei jeder statisch bestimmt gelagerten Scheibe kann man diesen \"{U}bergang in den Lagerknoten erleben, s. Bild \ref{U278}. Wir wissen ja, was an Lagerkraft herauskommen muss, und so haben wir gar kein Problem damit, die $f_i$ in den Lagern f\"{u}r real zu nehmen. Im Innern der Scheibe, speziell in den Elementen, die dem Lagerknoten gegen\"{u}ber liegen, wirkt dagegen ein konfuses Durcheinander von Fl\"{a}chen- und Linienkr\"{a}ften. \\

Wenn man die Belastung auf einem Kragtr\"{a}ger in den Endknoten reduziert, dann mag das einem Laien sehr ungenau erscheinen, aber der Ingenieur wei{\ss}, dass die Momente des Knotenlastfalls eine ganz gute N\"{a}herung sind.

Es hat sich doch ein gro{\ss}er Ballast an S\"{a}tzen und Prinzipien angeh\"{a}uft, der oft mehr verdunkelt als erhellt. Wir brauchen den spitzen Bleistift um ohne Umwege die Ergebnisse direkt aus den Regeln der partiellen Integration (das ist die einzige Mathematik, die wir in diesem Buch ben\"{o}tigen) herleiten zu k\"{o}nnen. Wir haben uns aber bem\"{u}ht dem legitimen Wunsch nach anschaulichen Beispielen gerecht zu werden.

Wir haben mehrere e-Mails mit dem Autor gewechselt, weil wir der Meinung sind, dass man mathematische Ergebnisse (und Einflussfunktionen sind nun einmal Formeln) nur auf mathematischem Wege herleiten kann, w\"{a}hrend der Autor, im Glauben seinen Studenten damit einen Gefallen zu tun, die Anschauung zu Hilfe nahm, aber Anschauung und Balkenstatik nicht zur Deckung bringen konnte (Stichwort: Starrk\"{o}rperdrehungen).


Was umso mehr Wunder nimmt, weil ja gerade in der Statik Anschauung und das statische Gef\"{u}hl Hand in Hand gehen. \\

%---------------------------------------------------------------------------------
\begin{figure}
\centering
{\includegraphics[width=1.0\textwidth]{\Fpath/U286}}
\caption{Fachwerk \textbf{ a)} Minimall\"{o}sung \textbf{ b)} stark ausgefacht}
\label{U286}%
%
\end{figure}%
%---------------------------------------------------------------------------------

Anders gesagt, wenn die Zusatzbelastung gro{\ss}e Wege geht, ihre Eigenarbeit gro{\ss} ist, dann muss man genau hinschauen, w\"{a}hrend man in
allen anderen F\"{a}llen davon ausgehen kann, dass die Effekte 'versickern'.
\end{remark}

Im Grunde herrscht ein delikates Gleichgewicht zwischen innerer und \"{a}u{\ss}erer Energie bzw. Arbeit, $A_i = A_a$. Bei dem Fachwerk in Bild Bild \ref{U286} a ist die Bilanz einfach, weil nur \"{u}ber drei St\"{a}be integriert wird
\begin{align}
A_i = \sum_{j = 1}^3\,EA\int_0^{\,l_j} (u_j')^2\,dx = P\,u\,.
\end{align}
Je mehr St\"{a}be hinzukommen, s. Bild \ref{U286} b, \"{u}ber um so mehr St\"{a}be wird integriert und so k\"{o}nnte man meinen, dass wegen $A_i = A_a$ auch die Auslenkung w\"{a}chst. Aber das Gegenteil ist nat\"{u}rlich der Fall, $A_i$ und $A_a$ nehmen ab. Das Mehr an St\"{a}ben \"{u}ber die integriert wird, wird dadurch kompensiert, dass die Dehnungen $\varepsilon_j = u_j'$ der St\"{a}be kleiner werden. Die L\"{a}ngsverschiebungen $u_j(x)$ sind ja linear (keine \"{a}u{\ss}eren Kr\"{a}fte zwischen den Knoten) und so h\"{a}ngen die Dehnungen $u_j'$ nur von den Knotendifferenzen ab, aber diese Differenzen werden mit zunehmender Zahl von St\"{a}ben 'schneller' kleiner als die L\"{a}nge der Strecke, \"{u}ber die integriert wird, zunimmt.
\\

\begin{remark}
Wir haben es nun schon \"{o}fter erw\"{a}hnt. Betti und Mohr sind zwei Seiten einer Medaille. Mittels partieller Integration kommt man von Mohr
\begin{align}
\text{(Mohr)} \qquad m_{xx}^h(\vek x) = a(G_h,w_h) = G_h(\vek x,\vek x) \cdot P \qquad \text{(Betti)}\,.
\end{align}
zu Betti und umgekehrt. Der Punktwert $G_h(\vek x,\vek x)$ ist so 'falsch', wie $a(G_h,w_h)$ 'falsch' ist. Ein Nadel\"{o}hr zu treffen klingt dramatischer als ein Integral wie $a(G_h,w_h)$ auszuwerten, aber es ist kein Unterschied.
\end{remark}
\\

Bei der sogenannten 3-D Statik ist das anders, aber diese leidet darunter, dass die Ergebnisse leicht un\"{u}bersichtlich werden, man nicht mehr wei{\ss}, 'wo die Kr\"{a}fte hinflie{\ss}en', weil die sogenannte 'Knopfdruckstatik' den Tragwerksplaner viel weniger zum Mitdenken anregt, als die Positionsstatik.

Wir wollen das F\"{u}r und Wider hier nicht wiederholen. Der angenehme Nebeneffekt der 3-D Statik ist auf jeden Fall, dass Steifigkeits\"{a}nderungen oder allgemeiner \"{A}nderungen im Entwurf sich leichter nachvollziehen lassen, als bei der Positionsstatik.\\

%%%%%%%%%%%%%%%%%%%%%%%%%%%%%%%%%%%%%%%%%%%%%%%%%%%%%%%%%%%%%%%%%%%%%%%%%%%%%%%%%%%%%%%%%%%%%%%%%%%%%%%
\textcolor{blau2}{\section{Konstruktive Fragen}}
Bevor es finite Elemente (und Randelemente) gab, gab es die {\em Positionsstatik\/}, wo man ein Tragwerk in einzelne Positionen einteilte und diese einzeln berechnete. Die Ber\"{u}cksichtigung der Steifigkeiten der angrenzenden Bauteile ist dabei nur n\"{a}herungsweise m\"{o}glich.

Das Thema dieses Abschnitts soll daher die Frage sein: Wie kann man bei der Positionsstatik nachtr\"{a}gliche Steifigkeits\"{a}nderungen ber\"{u}cksichtigen, oder, was eine \"{a}hnliche Fragestellung ist, wie kann man den Effekt von zu groben Annahmen bez\"{u}glich der Steifigkeiten absch\"{a}tzen?

Das grunds\"{a}tzliche Werkzeug haben wir schon in Abschnitt XX diskutiert. Zu jedem interessierenden Wert $J(w)$, einer Schnittgr\"{o}{\ss}e, einer Lagerkraft, einer Verformung, gibt es eine Einflussfunktion
\begin{align}
J(w) = \int_0^{\,l} G(y,x)\,p(y)\,dy
\end{align}
und die entscheidende Frage ist daher, wie sich der Kern $G(y,x)$ dieser Einflussfunktion mit den Steifigkeiten \"{a}ndert.

Man muss diese Frage nat\"{u}rlich auch unter dem Aspekt betrachten, wieweit strahlen denn eigentlich die Effekte, die die Belastung verursacht aus? Wir haben oben festgehalten, dass die Einflussfunktionen je nach Ordnung der Ableitung der Zielgr\"{o}{\ss}e im Aufpunkt unterschiedliches Abklingenverhalten haben.

Es kann sich dabei auch nur mehr um eine qualitative Untersuchung handeln, als eine quantitative. Es gilt also die Effekte abzusch\"{a}tzen, abzusch\"{a}tzen ob eine Neuberechnung n\"{o}tig oder sinnvoll ist.


Die FE-Einflussfunktion $G_2(\vek y,\vek x)$ ist ja nicht exakt, sondern eine N\"{a}herung, der die Spitze $G_2(\vek x,\vek x) = \infty$ von dem FE-Programm abgeschnitten wird. Das verdeutlicht, wie verletzlich der Wert $m_{xx}$ \"{u}ber der St\"{u}tze ist.

Wechseln wir von gew\"{o}hnlichen Differentialgleichungen zu partiellen Differentialgleichungen, dann gilt all dies nat\"{u}rlich analog. Jede hinreichend regul\"{a}re Funktion, $u \in C^2(\Omega)$, gen\"{u}gt der Bilanz
\begin{align}\label{Eq90}
\text{\normalfont\calligra G\,\,}(u, u) = \int_{\Omega} - \Delta u\,u\,\,d\Omega + \int_{\Gamma} \frac{\partial u}{\partial n}\,u\,ds - \int_{\Omega} \nabla u \dotprod \nabla u\,d\Omega = 0\,,
\end{align}
gleichg\"{u}ltig wie $u$ aussieht oder wie kurvenreich der Rand $\Gamma$ ist. Das ist schon ein beeindruckendes Resultat und zeigt, welche analytische Kraft in der partiellen Integration steckt.\\

Wenn eine Str\"{o}mung divergenzfrei ist (keine Quellen oder Senken im Innern), dann muss zu jedem Zeitpunkt die Menge an Fl\"{u}ssigkeit, die in ein Kontrollvolumen $\Omega$ hineinflie{\ss}t, in gleicher Gr\"{o}{\ss}e auch wieder herausflie{\ss}en. Dieses intuitiv evidente Resultat beruht auf dieser Identit\"{a}t, $\text{\normalfont\calligra G\,\,}(u, 1) = 0$. In der Statik sind die Lasten die Quellen und wenn zwischen zwei Stabenden keine Lasten vorhanden sind, dann m\"{u}ssen die Schnittkr\"{a}fte an den beiden Enden des Stabes in der Summe null sein. \\

\hspace*{-12pt}\colorbox{hellgrau}{\parbox{0.98\textwidth}{Der Ingenieur stellt die Differentialgleichung $EI\,w^{IV}(x) = p(x)$ auf, aber wie die dazu passenden Gleichgewichtsbedingungen aussehen, steht nicht in seinem Belieben. Das entscheidet allein die Mathematik.}}\\

Man kann daher nicht mitten im Galopp pl\"{o}tzlich wieder auf richtige Drehungen umschwenken und so die Statik in die N\"{a}he der N\"{a}herungsrechnung r\"{u}cken. {\em Die Statik rechnet exakt, sie l\"{o}st die gestellten Aufgaben (im Rahmen der zu Grunde gelegten Annahmen) exakt!\/}\\

 genannt werden\footnote{Das $L$ steht f\"{u}r den franz\"{o}sischen Mathematiker Lebesgue\index{Lebesgue-Integral}},

 \begin{enumerate}
\item Das Ergebnis ist exakt, wenn die Einflussfunktion $G(y,x)$ mit den Ansatzfunktionen dargestellt werden kann.
\item Wenn das nicht m\"{o}glich ist, dann benutzt das FE-Programm eine N\"{a}herung, aber dann ist das Ergebnis im allgemeinen auch nur eine N\"{a}herung.
\end{enumerate}
Wenn der Benutzer die Durchbiegung des Seils in einem Punkt $x$ wissen, will, dann geht das FE-Programm wie folgt vor: Es ermittelt die Einflussfunktion $G(y,x)$ f\"{u}r diesen Punkt und \"{u}berlagert $G(y,x)$ mit der Belastung $p$ wie in (\ref{Eq108}). Wenn $G(y,x)$ exakt ist, dann ist auch das Ergebnis $w(x)$ exakt. Wenn sich jedoch die Einflussfunktion nicht mit den $\Np_i(x)$ darstellen l\"{a}sst, dann benutzt das FE-Programm eine N\"{a}herung $G_h(y,x) \sim G(y,x)$ und erh\"{a}lt aber nat\"{u}rlich dann auch nur einen gen\"{a}herten Wert $w_h(x) \sim w(x)$
\begin{align}
w_h(x) = \int_0^{\,l} G_h(y,x)\,p(y)\,dy\,.
\end{align}
Das ist die Logik der finiten Elemente.
\\
%----------------------------------------------------------------------------------------------------------
\begin{figure}[tbp]
\centering
\if \bild 2 \sidecaption \fi
\includegraphics[width=1.0\textwidth]{\Fpath/U74}
\caption{Vergleich einer FE-L\"{o}sung mit der exakten L\"{o}sung} \label{U74}
\end{figure}%
%----------------------------------------------------------------------------------------------------------
Die FE-L\"{o}sung eines Stabes, s. Bild \ref{U74}, mag einen solchen Vergleich zwischen $p$ und $p_h$ verdeutlichen. Weil die finiten Elemente die Lasten in den Knoten konzentrieren, ist der Fehler auf der Lastseite gro{\ss}, wo im Original Linienkr\"{a}fte $p = 10$ kN/m wirken, sind im FE-Lastfall keine Kr\"{a}fte vorhanden. Dagegen ist der Fehler in den Normalkr\"{a}ften deutlich kleiner, s. Bild \ref{U74} d, und wenn man $u$ und $u_h$ vergleicht, dann wird der Fehler noch mal kleiner,  s. Bild \ref{U74} b. \\

Es ist diese Ambivalenz, die die Diskussion von FE-Ergebnissen so schwierig macht. Der Ingenieur ist  ein Meister in diesem 'Seitenwechsel'. Er hat kein Problem damit, die $f_i$ einmal als nur gedacht und einmal als real zu nehmen. Es ist nur eine der unz\"{a}hligen Unsch\"{a}rfen und N\"{a}herungen mit denen er tagt\"{a}glich in seiner Arbeit konfrontiert ist.
\\
In der Mathematik nennt man ein Integral von zwei Funktionen ein $L_2$-Skalarprodukt\index{$L_2$-Skalarprodukt}, was aus der Sicht der Statik eine sinnvolle Bezeichnung ist, weil eigentlich alle Integrale in der Statik die \"{U}berlagerung einer Kraftgr\"{o}{\ss}e mit einer Weggr\"{o}{\ss}e sind, sie also Arbeitsintegrale sind.

%----------------------------------------------------------------------------------------------------------
\begin{figure}[tbp]
\centering
\if \bild 2 \sidecaption \fi
\includegraphics[width=1.0\textwidth]{\Fpath/U41}
\caption{Zwei Sichtweisen: Lagersenkung oder Kragarmbelastung}
\label{U41}
\end{figure}%
%----------------------------------------------------------------------------------------------------------

%%%%%%%%%%%%%%%%%%%%%%%%%%%%%%%%%%%%%%%%%%%%%%%%%%%%%%%%%%%%%%%%%%%%%%%%%%%%%%%%%%%%%%%%%%%%%%%%%%%
{\textcolor{blau2}{\subsection{Zwei Sichten auf dieselbe Sache}}}
Bei der Lagersenkung des Balkens in Bild \ref{U41} a berechnen wir f\"{u}r die potentielle Energie den Wert
\begin{align}
\Pi(w)= \frac{1}{2}\,\int_0^{\,l} \frac{M^2}{EI}\,dx = \frac{1}{2}\,V(l)\,w_\Delta\,,
\end{align}
wenn wir die erste Greensche Identit\"{a}t
\begin{align}
\frac{1}{2}\,\text{\normalfont\calligra G\,\,}(w,w) = \frac{1}{2}\,V(l)\,w_\Delta - \frac{1}{2}\,\int_0^{\,l} \frac{M^2}{EI}\,dx = 0
\end{align}
zu Hilfe nehmen.

Wenn man einen gleich langen Kragtr\"{a}ger an seinem freien Ende gerade so stark belastet, dass eine Kraft $P$ dieselbe Durchbiegung $w_\Delta$ generiert, dann gilt
\begin{align}
\Pi(w) = \frac{1}{2}\int_0^{\,l} \frac{M^2}{EI}\,dx - P\,w_\Delta = - \frac{1}{2}\, P\,w_\Delta \,.
\end{align}
Nat\"{u}rlich ist $P = V(l)$ und so stimmen die beiden Ausdr\"{u}cke, abgesehen von dem Vorzeichen, zahlenm\"{a}{\ss}ig \"{u}berein.

Wir sehen an diesem Beispiel aber auch, dass die potentielle Energie kein 'absoluter Wert' ist, sondern dass sie, je nach Sichtweise, einmal positiv und einmal negativ sein kann, wenn nat\"{u}rlich auch der Betrag sich nicht \"{a}ndert.


Und noch eine Bemerkung. Es ist gar nicht hilfreich, wenn behauptet wird $\delta w$ und $\delta \Np$ seien infinitesimal klein, aber ihr Verh\"{a}ltnis sei endlich. Das $\delta w$  in Bild \ref{U13} ist gesch\"{a}tzte 0.3 m und $\tan \Np_l$ und $\tan \Np_r$ kann man sicherlich auch nicht infinitesimal klein nennen.
\\

\begin{remark}
Dass in (\ref{Eq101}) auf der rechten Seite eine 1 steht, wo in Abschnitt \ref{Chap3}.\ref{General} an entsprechender Stelle eine -1 stand, ist dem Umstand geschuldet, dass die Lagerkraft $R$ (= Federkraft) hier nach oben zeigt, w\"{a}hrend sie in Abschnitt \ref{Chap3}.\ref{General} konsequent in die Richtung $f_i$ zeigt.

Da dann die Einflussfunktion wie 'auf den Kopf' gestellt aussieht, haben wir hier $R$ nach oben zeigen lassen.\\
\end{remark}


Wenn der Knoten ein starres Lager ist, dann muss man, wie oben, zur Lagerkraft der FE-L\"{o}sung noch den Anteil $R_d$ hinzu addieren, der direkt in den Lagerknoten reduziert wurde.

%%%%%%%%%%%%%%%%%%%%%%%%%%%%%%%%%%%%%%%%%%%%%%%%%%%%%%%%%%%%%%%%%%%%%%%%%%%%%%%%%%%%%%%%%%%%%%%%%%%
{\textcolor{blau2}{\section{Zusammenfassung}}\label{Zusammenfassung}
Es sei $f_i$ die Lagerkraft in einem festgehaltenen Knoten. Den Wert von $f_i$ erh\"{a}lt man nach der L\"{o}sung des Systems $\vek K\,\vek u = \vek f$ aus dem nicht-reduzierten System\footnote{bevor also die Zeilen und Spalten gestrichen werden, die zu Lagerknoten $u_i = 0$ geh\"{o}ren}\index{$\vek K_G$}\index{nicht-reduzierte Steifigkeitsmatrix}
\begin{align}
\vek K_G\,\vek u_G = \vek f_G\,.
\end{align}
Zu $f_i$ muss man noch den Anteil hinzu addieren, der direkt in das Lager reduziert wurde.

F\"{u}r jedes $f_i$ gibt es eine Einflussfunktion
\begin{align}
G_h(\vek y,\vek x) = \sum_j\,g_j(\vek x)\,\Np_j(\vek y)\,,
\end{align}
deren Knotenwerte $g_j$ die L\"{o}sung des Systems
\begin{align}
\vek K\,\vek g = \vek j
\end{align}
sind. Hierbei ist der Vektor $\vek j$ identisch mit der Spalte $i$ der Steifigkeitsmatrix $\vek K_G$, allerdings auf die L\"{a}nge $n$ gek\"{u}rzt, d.h. die zu gesperrten Freiheitsgraden geh\"{o}rigen Zeilen werden gestrichen.

Die vollst\"{a}ndige Einflussfunktion, die den Anteil mit umfasst, der direkt in den Knoten flie{\ss}t, lautet
\begin{align} \label{Eq104}
G_h(\vek y,\vek x) = \sum_j\,g_j(\vek x)\,\Np_j(\vek y) + \Np_i(\vek y)\,.
\end{align}
Ist der Knoten ein elastisches Lager, dann erh\"{a}lt man die Einflussfunktion f\"{u}r $f_i$, indem man eine Kraft $P = 1$ auf den Knoten setzt, und die dabei entstehende Biegefl\"{a}che mit der Steifigkeit $k$ des Lagers multipliziert. Eine Korrektur wie in (\ref{Eq104}) ist nicht notwendig.
\\

%%%%%%%%%%%%%%%%%%%%%%%%%%%%%%%%%%%%%%%%%%%%%%%%%%%%%%%%%%%%%%%%%%%%%%%%%%%%%%%%%%%%%%%%%%%%%%%%%%%
{\textcolor{blau2}{\section{Generalisierung}}\label{General}

Nehmen wir an, ein Ansatz besteht aus $n $ Ansatzfunktionen $\Np_i$ und irgendwo sei ein festes Lager, das wir uns als einzelnen Knoten vorstellen k\"{o}nnen. Hinter jeder Ansatzfunktion $\Np_i$ steht ein gewisser Satz von Kr\"{a}ften, die {\em shape forces\/} $p_i$, die dem Tragwerk die Verformung $\Np_i$ aufzwingen. Das Lager  h\"{a}lt dagegen und so geh\"{o}rt zu jedem $\Np_i$ eine \"{a}quivalente Lagerkraft $j_i$  in dem Lagerknoten, die gleich der Wechselwirkungsenergie zwischen $\Np_i$  und der Einheitsverformung $\Np_X$ ($X$ ist ein Index) des Lagerknotens ist
\begin{align}\label{Eq99}
a(\Np_X,\Np_i) = 1 \cdot j_i\,.\nn
\end{align}

\hspace*{-12pt}\colorbox{hellgrau}{\parbox{0.98\textwidth}{Die $j_i$ sind wegen $a(\Np_X,\Np_i) = k_{X i}$ identisch mit den Elementen der Zeile $X$ der nicht-reduzierten Steifigkeitsmatrix $\vek K_{G}$.}}\\

Hier ergibt sich eine kleine Schwierigkeit mit den Indices. Die Zeile $X$ von $\vek K_{G}$ enth\"{a}lt $N$ Eintr\"{a}ge, wenn $N \times N$ die Gr\"{o}{\ss}e von $\vek K_{G}$ ist. Der Vektor $\vek j$, mit dem wir im folgenden operieren, hat aber nur $n$ Eintr\"{a}ge, weil die Eintr\"{a}ge, die zu gesperrten Freiheitsgraden geh\"{o}ren, gestrichen wurden.

Zu einer $n$-gliedrigen FE-L\"{o}sung $u_h = u_1\,\Np_1(x) + u_2\,\Np_2(x) + \ldots$ geh\"{o}rt demnach die Lagerkraft
\begin{align}
R = \vek u^T\,\vek j = u_1\,j_1 + u_2\,j_2 + \ldots + u_n\,j_n\,.
\end{align}
Diese Lagerkraft muss nun aber auch gleich dem Skalarprodukt $\vek f^T\,\vek g$ sein, also dem Produkt aus den \"{a}quivalenten Knotenkr\"{a}ften $\vek f$ des Lastfalls und den Knotenverschiebungen $\vek g$ der Einflussfunktion f\"{u}r die Lagerkraft,
\begin{align}
R = \vek u^T\,\vek j = \vek f^T\,\vek g\,,
\end{align}
und damit folgt, dass das einzelne $g_i$ die Lagerkraft im Lastfall $\vek f = \vek e_i$ ($i$-ter Einheitsvektor) sein muss.
\begin{align}
R (\text{im LF $\vek e_i$}) = \vek u^T (\text{im LF $\vek e_i$})\,\vek j = \vek e_i^T \vek g = g_i\,.
\end{align}

Die Knotenverschiebungen in den Lastf\"{a}llen $\vek f = \vek e_i$ sind aber nun gerade die Zeilen (= Spalten) der Inversen $\vek K^{-1}$ und so folgt weiter, dass
\begin{align}
\vek g = \vek K^{-1}\,\vek j\,
\end{align}
der Vektor der Knotenwerte der FE-Einflussfunktion ist
\begin{align}
G_h(y,x) = \sum_{i = 1}^n g_i(x)\,\Np_i(y)\,.
\end{align}
Das $ x$ ist hier die Koordinate des Knotens.

Die vollst\"{a}ndige Einflussfunktion, die den Anteil mit umfasst, der direkt in einen Knoten $i$ flie{\ss}t, lautet
\begin{align} \label{Eq104}
G_h(y,x) = \sum_j\,g_j(x)\,\Np_j(y) + \Np_i(y)\,.
\end{align}
Ist der Knoten ein elastisches Lager, dann erh\"{a}lt man die Einflussfunktion f\"{u}r $f_i$, indem man eine Kraft $P = 1$ auf den Knoten setzt, und die dabei entstehende Biegefl\"{a}che mit der Steifigkeit $k$ des Lagers multipliziert. Eine Korrektur wie in (\ref{Eq104}) ist nicht notwendig.

%%%%%%%%%%%%%%%%%%%%%%%%%%%%%%%%%%%%%%%%%%%%%%%%%%%%%%%%%%%%%%%%%%%%%%%%%%%%%%%%%%%%
{\textcolor{blau2}{\section{Positionsstatik und 3-D Berechnung}}
Bei der sogenannten {\em Positionsstatik\/} wird jeder Unterzug, jede Deckenplatte f\"{u}r sich allein berechnet. Unter Umst\"{a}nden m\"{o}chte man aber die Werte $f_i$ in den Lagerknoten einer Platte mit den Ergebnissen einer 3-D Berechnung vergleichen.

Nun ist es aber so, dass bei einer 3-D Berechnung nicht die Anschnittkr\"{a}fte---in der Stabstatik w\"{a}ren das die Balkenendkr\"{a}fte---ausgegeben werden, sondern nur die Knotenkr\"{a}fte $f_i$, also die Summe \"{u}ber alle Anschlusskr\"{a}fte. Die Frage ist daher, wie man die  Anschlusskr\"{a}fte berechnen kann.

Das geht im Grunde wie in der Stabstatik. Wenn das Gleichungssystem
\begin{align} \label{Eq105}
\vek K\vek u = \vek f + \vek p
\end{align}
gel\"{o}st ist, dann kennt man die $u_i$ an jedem Knoten. Diese kann man nun stabweise in das lokale Koordinatensystem der angeschlossenen St\"{a}be umrechnen $u_i \to u_i^e$ und dann an Hand der Beziehung
\begin{align}\label{Eq106}
\vek K^e\,\vek u^e = \vek f^e + \vek p^e
\end{align}
die Balkenendkr\"{a}fte $f_i^e$ an jedem einzelnen Stab berechnen. Die $p_i^e$ sind die Auflagerdr\"{u}cke (= Festhaltekr\"{a}fte $\times (-1)$) am Stab aus der Belastung.

Bei einer Platte macht man sinngem\"{a}{\ss} dasselbe. Man multipliziert die Steifigkeitsmatrix $\vek K^p$ der Platte (nur der Platte!) mit den zur Platte geh\"{o}rigen Anteilen $\vek u^p$ des globalen Vektors $\vek u_G$ und erh\"{a}lt so die $f_i^p$ am Rand der Platte
\begin{align}
\vek K^p\,\vek u^p = \vek f^p
\end{align}
und diese kann man dann mit den Ergebnissen aus der Positionsstatik vergleichen.

Wenn man einmal annimmt, dass die Steifigkeitsmatrix $\vek K^p$ der Platte aus dem 3-D Modell und die Matrix $\vek K^{pos}$ aus der Positionsstatik nicht allzusehr voneinander abweichen, dann sind die Unterschiede in den $f_i$ auf die Unterschiede in den Knotenverformungen zwischen dem 3-D Modell und der Positionsstatik zur\"{u}ckzuf\"{u}hren.\\

\begin{remark}
Die $f_i$ in (\ref{Eq105}) sind die Kr\"{a}fte, die direkt in den Knoten des Rahmens angreifen und die $p_i$ sind die in jedem Knoten aufsummierten Auflagerdr\"{u}cke aus der verteilten Belastung links und rechts vom Knoten.

Die $f_i^e$ in (\ref{Eq106}) dagegen sind Balkenendkr\"{a}fte und keine Knotenkr\"{a}fte. In der Literatur wird leider derselbe Buchstabe f\"{u}r diese unterschiedlichen Kr\"{a}fte benutzt---einmal ist man am Balkenende und einmal im Knoten.
\end{remark}

%%%%%%%%%%%%%%%%%%%%%%%%%%%%%%%%%%%%%%%%%%%%%%%%%%%%%%%%%%%%%%%%%%%%%%%%%%%%%%%%%%%%%%%%%%%%%%%%%%%
\textcolor{blau2}{\section{Durchlauftr\"{a}ger}}

Wir wollen noch einige Dinge erg\"{a}nzen, die speziell die Stabstatik betreffen.
%-----------------------------------------------------------------
\begin{figure}[tbp]
\centering
\includegraphics[width=0.8\textwidth]{\Fpath/U232}
\caption{Anpassung eines Balkenelements an die Biegelinie des Tr\"{a}gers. Die Absenkung auf das Niveau des Tr\"{a}gers erfordert keine Kr\"{a}fte, nur die Verdrehung der Endtangenten des Elements ($- - - $) erfordert Momente $f^+$}
\label{U232}
\end{figure}%
%-----------------------------------------------------------------

Wenn sich die Biegesteifigkeit $EI$ eines Elementes \"{a}ndert, dann addieren wir ein Element $\Omega_e$ zu dem System
und koppeln es mit den Kr\"{a}ften/Momenten
\begin{align}\label{Eq72}
f_1^+ = - V_a^+,\qquad f_2^+ = - M_a^+, \qquad f_3^+ = V_b^+, \qquad f_4^+ = M_b^+\nn
\end{align}
an die Struktur. Wir wissen, dass diese Gr\"{o}{\ss}en im Gleichgewicht sind
\begin{align}
(\Sigma V = 0) \qquad f_1^+ + f_3^+ = 0\,\qquad  f_2^+ + f_4^+ - f_3^+ \cdot l_e = 0 \qquad (\Sigma M = 0)\,,
\end{align}
anders gesagt, wenn man die Einflussfunktion, die wir hier der Einfachheit halber $g(y)$ nennen, auf dem Element $(a,b)$ linear interpoliert,
\begin{align}
g(y) = g(a) + (g(b) - g(a)) \cdot \frac{y - a}{b - a} = g(a) + m\,(y - a)\,,
\end{align}
dann sind die Gr\"{o}{\ss}en (\ref{Eq72}) orthogonal zu dieser Interpolierenden, leisten keine Arbeit auf den Knotenverschiebungen/-verdrehungen des Weges $g(y)$. Bei der Auswertung der Einflussfunktion verbleibt also nur der Term (die Momente und $g'$ drehen entgegengesetzt)
\begin{align}\label{Eq73}
- f_2^+ \cdot (g'(a) - m) - f_4^+ \cdot (g'(b) - m)\,,
\end{align}
dessen Gr\"{o}{\ss}e davon abh\"{a}ngt, wie stark die Neigung der Einflussfunktion $g(y)$ an den Balkenenden von der Neigung $m$ der Interpolierenden abweicht.
%-----------------------------------------------------------------
\begin{figure}[tbp]
\centering
\includegraphics[width=0.9\textwidth]{\Fpath/U112}
\caption{Durchlauftr\"{a}ger, \textbf{ a)} gleiches $EI$ in allen Feldern, \textbf{ b)} Verdopplung von $EI$ im zweiten Feld, \textbf{ c)} Erzeugung von
$M_c$ am Original mit Hilfe von Koppelmomenten $f_i^+$, \textbf{ d)} Einflussfunktion f\"{u}r $M$ (Mitte 3. Feld)}
\label{U112}
\end{figure}%
%-----------------------------------------------------------------

Wenn der Aufpunkt $ x$ weit genug weg liegt, dann d\"{u}rfen wir annehmen, dass die Differenzen $g' - m$ klein sind, und dann k\"{o}nnen wir den Effekt der Steifigkeits\"{a}nderung  vernachl\"{a}ssigen. Bei Durchlauftr\"{a}gern ist die lineare Interpolierende der Einflussfunktionen null,  weil die Einflussfunktionen in allen Lagerknoten null sind, s. Bild \ref{U112} d, und somit ist auch $m = 0$.

Gerade bei Durchlauftr\"{a}gern klingen Einflussfunktionen sehr rasch ab, s. Bild \ref{U112} d, und \"{A}nderungen von $EI$ im zweitn\"{a}chsten oder drittn\"{a}chsten Feld d\"{u}rften vernachl\"{a}ssigbar sein.

Bei dem Beispiel in Bild \ref{U112}, $EI = 3.56\cdot 10^{3}$ kNm$^2$ findet die Steifigkeits\"{a}nderung, $EI \to 2\,EI$, in dem Feld direkt neben dem Aufpunkt statt, aber trotz dieser N\"{a}he \"{a}ndert sich das Feldmoment wenig, $M = 34.6 \to M_c = 32.3$ kNm.

Die beiden Momente $f_i^+$ am Element 2 ergeben sich aus der Gleichung $- \vek \Delta\,\vek K\,\vek u_c = \vek f^+$,
\begin{align} \label{Eq74}
 - \frac{EI}{l^3} \left[
\begin{array}{r r r r}
 12 & -6l & -12 &-6l \\
 -6l & 4l^2 & 6l &2l^2 \\
 -12 & 6l & 12 & 6l \\
 -6l &2l^2 &6l &4l^2
 \end{array}
  \right]\,\left [\barr{c} \phantom{-} 0 \\ 1.32 \cdot 10^{-3} \\ \phantom{-} 0 \\ -3.62 \cdot 10^{-3} \earr \right ] = \left [\barr{c}  * \\ 1.74 \\ * \\ 12.84 \earr \right ]
\end{align}
wobei in diesem Fall $\vek \Delta \vek K = \vek K$ die $4 \times 4$-Matrix des Standardelementes ($EI$) ist (wir haben ja $EI$ verdoppelt).


Die Momente $f_i^+$ alleine erzeugen also das Zusatzmoment ($m = 0$)
\begin{align}
-f_2^+ &\cdot (g'(a) - m) - f_4^+ \cdot (g'(b) - m)\nn \\
 &= (-1.74) \cdot (-0.054) - 12.89 \cdot 0.188  = -2.3\,\text{kNm}
\end{align}
und dieses zu dem urspr\"{u}nglichen Feldmoment addiert, ergibt das neue Feldmoment
\begin{align}
M_c = M - 2.3 = 34.6 \,\text{kNm} - 2.3\,\text{kNm} = 32.3\,\text{kNm}\,.
\end{align}
Das Beispiel ist nat\"{u}rlich rein theoretisch, weil wir am modifizierten System $(\vek K + \vek \Delta\,\vek K)\,\vek u_c = \vek f$ erst die Verformungen $u_i^c$ berechnen m\"{u}ssen, die wir in (\ref{Eq74}) benutzen, um die Momente $f_i^+$ zu berechnen, mit denen wir dann am System $\vek K\,\vek u_c = \vek f + \vek f^+$ den Vektor $\vek u_c$ berechnen...

Aber hier geht es prim\"{a}r darum, zu demonstrieren, wie gro{\ss} die Verdrehungen $g'(a)$ und $g'(b)$ sind, wenn der Aufpunkt gleich im n\"{a}chsten Element liegt und wie gro{\ss} die $f_i^+$ sind. Es geht um Absch\"{a}tzungen, um das statische Verst\"{a}ndnis der Situation, nicht darum, den Computer zu schlagen.

Wir halten also fest: \\

\hspace*{-12pt}\colorbox{hellgrau}{\parbox{0.98\textwidth}{\"{A}nderungen der Biegesteifigkeit, $EI + \Delta EI $, in einem Element, f\"{u}hren zu Zusatzkr\"{a}ften und -momenten $f_i^+$ an den Elementenden. F\"{u}r die dadurch ausgel\"{o}sten Effekte sind aber nur die Momente verantwortlich, s. (\ref{Eq73}). }}\\

Dass die Querkr\"{a}fte keine Rollen spielen, versteht man beim Blick auf das Bild \ref{U232}.

%-----------------------------------------------------------------
\begin{figure}[tbp]
\centering
\includegraphics[width=0.9\textwidth]{\Fpath/U101}
\caption{Der Einfluss, den das linke bzw. rechte Moment $X_i$ auf die Durchbiegung im Punkt $x$ haben, ist nicht gleich}
\label{U101}
\end{figure}%
%-----------------------------------------------------------------

%---------------------------------------------------------------------------------
\begin{figure}
\centering
\if \bild 2 \sidecaption \fi
\includegraphics[width=.75\textwidth]{\Fpath/U151}
\caption{Einflussfunktion f\"{u}r die Normalkraft links bzw. rechts vom Knoten 3. Die beiden Einflussfunktionen sind nach Addition der lokalen L\"{o}sung gleich}
\label{U151}%
\end{figure}%
%---------------------------------------------------------------------------------

%%%%%%%%%%%%%%%%%%%%%%%%%%%%%%%%%%%%%%%%%%%%%%%%%%%%%%%%%%%%%%%%%%%%%%%%%%%%%%%%%%%%%%%%%%%%%%%%%%%
{\textcolor{blau2}{\subsection{Netzlinien und Schnittgr\"{o}{\ss}en}}}\index{Netzlinien und Schnittgr\"{o}{\ss}en}
Wir hatten in Kapitel 3, S. \pageref{Jumps}, \"{u}ber die Spr\"{u}nge in den Spannungen an den Netzlinien gesprochen. Hierzu noch die folgende Erg\"{a}nzung.

Schnittgr\"{o}{\ss}en auf einer Netzlinie auszuwerten, ist nicht ratsam. Zum einen springen die ersten Ableitungen der Ansatzfunktionen auf den Linien, so wie ein Dach im First zwei Neigungen hat, man h\"{a}tte also zwei Spannungen in demselben Punkt und zum andern k\"{o}nnen die Elemente links und rechts von der Linie unterschiedliche Moduli $E_i$ haben.

Nat\"{u}rlich kann man den Netzlinien beliebig nahe kommen, so dass das keine echte Einschr\"{a}nkung ist, und wenn wie bei Stabtragwerken, die Schnittgr\"{o}{\ss}e $N, M$ oder $V$ am \"{U}bergang zweier Elemente stetig ist, dann m\"{u}ssen auch die Einflussfunktionen f\"{u}r die Schnittgr\"{o}{\ss}e links bzw. rechts vom Knoten dieselbe Gestalt haben und dann ist es egal, ob man den Punkt links oder rechts vom Knoten als Aufpunkt w\"{a}hlt, wie Bild \ref{U151} zeigt.
%----------------------------------------------------------------------------
\begin{figure}
\centering
{\includegraphics[width=0.9\textwidth]{\Fpath/U254}}
  \caption{Einflussfunktion f\"{u}r die Normalkraft links von der Mitte, \textbf{ a)} System, \textbf{ b)} $\Np_1(x)$, \textbf{ c)} FE-Einflussfunktion $j_1 = EA_1$, \textbf{ d)} lokale L\"{o}sung am Stab 1, \textbf{ e)} EL = Summe aus \textbf{ c)} + \textbf{ d)}}
  \label{U254}
\end{figure}%%

%----------------------------------------------------------

Das Gleichungssystem f\"{u}r die Knotenverschiebungen $g_i$ der Einflussfunktion f\"{u}r $N(x)$ links
\begin{align} \label{Eq79}
\left[\barr{r r r r} 2 & - 1 & 0 & 0 \\ - 1 & 2 & -1 & 0\\ 0 & -1 & 3 &-2 \\ 0 & 0 & -2 &4\earr\right]
\,\left[\barr{c} g_1 \\g_2 \\ g_3 \\ g_4 \earr \right] = \left[\barr{r} 0 \\ -1  \\
1  \\ 0 \earr \right] \qquad \ldots =  \left[\barr{r} 0 \\0  \\
-2  \\ 2 \earr \right]
\end{align}
hat die L\"{o}sung
\begin{align}
g_1 = -0.25, \,\,g_2 = -0.5, \,\,g_3 = 0.25,\,\, g_4 = 0.125\,,
\end{align}
und f\"{u}r $N(x)$ rechts, mit dem zweiten Vektor in (\ref{Eq79}) auf der rechten Seite, hat die L\"{o}sung
\begin{align}
g_1 = -0.25, \,\,g_2 = -0.5, \,\,g_3 = -0.75, \,\,g_4 = 0.125\,.
\end{align}
Addiert man zu diesen Funktionen noch die lokale L\"{o}sung, also die L\"{a}ngsverschiebungen am ein-elementrigen Stab aus der Spreizung $[[u]] = 1$ am rechten bzw. linken Ende, dann erh\"{a}lt man die exakten Einflussfunktionen und die stimmen nat\"{u}rlich \"{u}berein.

Dasselbe gilt im \"{U}brigen auch f\"{u}r den Stab in Bild \ref{U254}, wo die Einflussfunktion f\"{u}r die Normalkraft $N(x)$ links vom Steifigkeitssprung berechnet wird. W\"{u}rde man den Aufpunkt rechts davon legen, dann w\"{u}rde $j_1 = - 5 \cdot 1.07 \cdot 10^6$ nach links dr\"{u}cken und die lokale L\"{o}sung im rechten Element w\"{u}rde von +1 im mittleren Knoten auf 0 im rechten Knoten abfallen. Der Gesamteffekt w\"{a}re aber derselbe wie vorher, d.h. die Einflussfunktion f\"{u}r $N(x)$ links und rechts von der Mitte w\"{a}re dieselbe.

All dies ist keine \"{U}berraschung. Wir zitieren dieses Beispiel auch nur, um noch einmal deutlich zu machen, wie man durch Addition der lokalen L\"{o}sung zur exakten Einflussfunktion kommt, auch dann, wenn die Steifigkeiten unterschiedlich sind. Au{\ss}erhalb des
Elements, auf dem der Aufpunkt liegt, sind FE-Einflussfunktionen bei (nicht gevouteten) Stabtragwerken sowieso exakt.


Bei Fl\"{a}chentragwerken sind die FE-Einflussfunktionen immer nur N\"{a}herungen und deswegen stimmen die Einflussfunktionen f\"{u}r, z.B. $\sigma_{xx}$, in zwei nur durch eine Netzlinie getrennten Punkten nicht \"{u}berein, auch wenn theoretisch $\sigma_{xx}^L = \sigma_{xx}^R$ sein muss. Den Grund haben wir oben erl\"{a}utert.

%---------------------------------------------------------------------------------
\begin{figure}
\centering
{\includegraphics[width=0.9\textwidth]{\Fpath/U118}}
  \caption{Finite Elemente und finite Differenzen}
  \label{U118}
\end{figure}
%---------------------------------------------------------------------------------



%%%%%%%%%%%%%%%%%%%%%%%%%%%%%%%%%%%%%%%%%%%%%%%%%%%%%%%%%%%%%%%%%%%%%%%%%%%%%%%%%%%%%%%%%%%%%%%%%%%
{\textcolor{blau2}{\subsection{Betti und finite Differenzen}}\index{Betti und finite Differenzen}
Der Vollst\"{a}ndigkeit halber wollen wir noch erw\"{a}hnen, dass man finite Differenzen als Anwendung des Satzes von Betti verstehen kann.

Es sei $u(x)$ die L\"{o}sung des Randwertproblems
\begin{align}
- u''(x) = p(x) \qquad u(0) = u(l) = 0\,,
\end{align}
und $G(y,x_i)$ bzw. $G(y,x_{i +1}) $ seien die L\"{o}sungen, die zu Punktlasten $\pm 1/h$ in den Knoten $i$ und $i+1$ geh\"{o}ren, dann gilt nach dem {\em Satz von Betti\/}, s. Bild \ref{U118},
\begin{align}
A_{1,2} = \frac{1}{h} \,u_{i+1} - \frac{1}{h}\,u_i = \int_0^{\,l} \frac{G(y,x_{i +1}) - G(y,x_i)}{h}\, p(y)\,dy = A_{2,1}
\end{align}
und somit in der Grenze, $h \to 0$,
\begin{align}
u'(x) = \int_0^{\,l} G_1(y,x)\,p(y)\,dy \qquad G_1(y,x) = \frac{d}{dx}\,G_0(y,x)\,.
\end{align}
Die zweiten Differenzen lauten
\begin{align} \label{Eq75}
\frac{u_{i+1} - 2\,u_i + u_{i+1}}{h^2}\,.
\end{align}
Das Integral einer H\"{u}tchenfunktion $\Np_i(x)$ vom Knoten $i-1$ bis zum Knoten $i+1$ ist $h$ und somit folgt, wenn wir $u_h''$ auf dem Interval $(-h,h)$ gleich dem Ausdruck (\ref{Eq75})  setzen
\begin{align}
\int_{-h}^{\,h} - u_h''\,\Np_i(x)\,dx = \frac{-u_{i+1} + 2\,u_i - u_{i-1}}{h}\,,
\end{align}
was genau die Zeile $i$ der Steifigkeitsmatrix ist. In der Gleichung $\vek K\,\vek u = \vek f$ ist die linke Seite ja gleich dem Vektor $\vek f_h$, also der Arbeit, die der FE-Lastfall auf den Wegen $\Np_i$ leistet und das erkl\"{a}rt, warum $k_{ij}$ ein Integral ist, die \"{U}berlagerung von $p_h$ mit $\Np_i$.
%---------------------------------------------------------------------------------
\begin{figure}
\centering
\if \bild 2 \sidecaption[t] \fi
{\includegraphics[width=0.9\textwidth]{\Fpath/U119}}
\caption{Lineare Interpolation = lineare finite Elemente, zu allen Kurven mit denselben Knotenwerten geh\"{o}ren dieselben Knotenkr\"{a}fte $f_i$}
\label{U119}%
\end{figure}%
%---------------------------------------------------------------------------------

Aber auch, wenn $k_{ij}$ eine Arbeit ist, also $p_h$ einmal integriert wird
\begin{align}
k_{ij} = \int_0^{\,l} p_h\,\Np_i\,dx
\end{align}
so bleibt $\vek K$ doch eine Differenzenmatrix und damit ist---im Umkehrschluss---ihre Inverse $\vek K^{-1}$ (die Flexibilit\"{a}tsmatrix $\vek F$) ein Integraloperator.

Eine Flexibilit\"{a}tsmatrix\index{Flexibilit\"{a}tsmatrix} berechnet aus Kr\"{a}ften Verformungen, $\vek F\,\vek f = \vek u$, sie integriert die Kr\"{a}fte, w\"{a}hrend eine Steifigkeitsmatrix aus Verformungen Kr\"{a}fte berechnet, $\vek K\,\vek u = \vek f$, sie differenziert die Verformungen.

%%%%%%%%%%%%%%%%%%%%%%%%%%%%%%%%%%%%%%%%%%%%%%%%%%%%%%%%%%%%%%%%%%%%%%%%%%%%%%%%%%%%%%%%%%%%%%%%%%%
{\textcolor{blau2}{\subsection{Interpolation und finite Elemente}}\index{Interpolation}
Wenn man eine Kurve $w(x)$ st\"{u}ckweise linear interpoliert, dann ist
der Polygonzug  identisch mit der FE-L\"{o}sung $w_h$, die zu dem 'Seil' $w(x)$ geh\"{o}rt, das in den H\"{o}hen $w(0)$ und $w(l)$ aufgeh\"{a}ngt ist, und die Belastung $- w''(x)$ tr\"{a}gt, s. Bild \ref{U119}.

Man kann also einer solchen linearen Interpolation \"{a}quivalente Knotenkr\"{a}fte $f_i$ zuschreiben
\begin{align}
f_i = \int_0^{\,l} - w''(x)\,\Np_i(x)\,dx= \tan\,\Np_i^L - \tan\,\Np_i^R \,,
\end{align}
die in den Knoten das konzentrieren, was im Nahfeld, 'auf der Strecke', an Kr\"{u}mmung vorhanden ist, und die so abrupt den Richtungswechsel
 bewirken, den die (technische) Kr\"{u}mmung $-w''$ gleitend vollzieht.

Auffallend ist dabei, dass die Steuerung der linearen Interpolation \"{u}ber die zweiten Ableitungen geschieht. Man k\"{o}nnte jetzt auf die Idee kommen, dass man, wenn man Messpunkte mit einem Lineal verbindet, auf diesem Weg Klarheit \"{u}ber den Verlauf der Kurve dazwischen bekommt. Aber  unterschiedliche Belastungen auf einem Seil k\"{o}nnen bekanntlich dieselben Knotenkr\"{a}fte $f_i$ erzeugen, was bedeutet, dass es nicht m\"{o}glich ist, aus den $f_i$ auf die Kr\"{u}mmung $-w''$ dazwischen zu schlie{\ss}en.

Bei der {\em Hermite-Interpolation\/}\index{Hermite-Interpolation} werden auch die Ableitungen in den Knoten interpoliert. Sie kann man nat\"{u}rlich in demselben Sinn als die FE-L\"{o}sung eines Balkenproblems lesen, weil ja die Element-Ansatzfunktionen $\Np_i^e(x)$ mit den Hermite-Polynomen\index{Hermite-Polynome} \"{u}bereinstimmen.



%----------------------------------------------------------------------------------------------------------
\begin{figure}[tbp]
\centering
\if \bild 2 \sidecaption \fi
\includegraphics[width=0.7\textwidth]{\Fpath/U150}
\caption{Einflussfunktionen f\"{u}r \textbf{ a)} die Durchbiegung im ersten Gelenk, \textbf{ b)} f\"{u}r die Verdrehung $w'$ \"{u}ber der Innenst\"{u}tze} \label{U150}
\end{figure}%
%----------------------------------------------------------------------------------------------------------

%-----------------------------------------------------------------
\begin{figure}[tbp]
\centering
\includegraphics[width=0.9\textwidth]{\Fpath/U316}
\caption{Schnittkr\"{a}fte, die bei der Generierung der Einflussfunktion f\"{u}r das Momente $M$ in der Mitte des Riegels 10 entstehen, \textbf{ a)} Momente, \textbf{ b)} Normalkr\"{a}fte}
\label{U316}
\end{figure}%
%-----------------------------------------------------------------
%%%%%%%%%%%%%%%%%%%%%%%%%%%%%%%%%%%%%%%%%%%%%%%%%%%%%%%%%%%%%%%%%%%%%%%%%%%%%%%%%%%%%%%%%%%%%%%%%%%
\textcolor{blau2}{\section{\"{A}nderungen der L\"{a}ngs- und Biegesteifigkeit}}
In Bild \ref{U316} a und b sind die Momente bzw. die Normalkr\"{a}fte angetragen, die entstehen, wenn man in der Mitte des Riegels 10 eine Spreizung der Gr\"{o}{\ss}e eins erzeugt, also die Einflussfunktion f\"{u}r das Moment $M$ generiert.

Man sieht deutlich, dass die Momente sehr schnell verebben, was bedeutet, dass die \"{A}nderung von $M$ auf Grund einer Steifigkeits\"{a}nderung $\Delta EI$ in irgendeinem Riegel, etwa dem Riegel 3,
\begin{align}
M_c - M =  \frac{\Delta EI}{EI}\int_0^{\,l_e} \frac{M_G \cdot M_c}{EI}\,dx
\end{align}
praktisch vernachl\"{a}ssigbar ist, weil $M_G$ relativ klein sein wird. Man muss gar nicht $M_c$ in dem betreffenden Riegel kennen, um diesen Schluss ziehen zu k\"{o}nnen.

Dagegen ist die \"{A}nderung der L\"{a}ngssteifigkeit $EA \to EA + \Delta EA$ kritischer, weil die Normalkr\"{a}fte, die durch die Generierung der Spreizung entstehen, viel weniger ged\"{a}mpft werden und somit der Einfluss auf das Moment im Riegel
\begin{align}
M_c - M = \frac{\Delta EA}{EA}\int_0^{\,l_e} \frac{N_G \cdot N_c}{EA}\,dx
\end{align}
eher bemerkbar sein wird.

Uns scheint, dass dieses ein generelles Merkmal von Stockwerkrahmen ist: \"{A}nderungen in der Biegesteifigkeit $EI \to EI + \Delta EI$ eines Riegels oder Stils sind weniger dramatisch als \"{A}nderungen in der L\"{a}ngssteifigkeit $EA \to EA + \Delta EA$.


%%%%%%%%%%%%%%%%%%%%%%%%%%%%%%%%%%%%%%%%%%%%%%%%%%%%%%%%%%%%%%%%%%%%%%%%%%%%%%%%%%%%%%%%%%%%%%%%%%%
\textcolor{blau2}{\subsection{Lagersenkung}}\label{Lagersenkung}
Hier noch ein Nachtrag zu dem Lastfall Lagersenkung. Wie auf S. \pageref{Eq36} angedeutet, spaltet man bei einer Berechnung von Hand die Biegelinie
\begin{align}\label{Eq116}
EI\,w^{IV} = 0 \qquad w(0) = w'(0) = 0 \qquad M(l) = 0\quad w(l) = w_{\Delta}\,,
\end{align}
in zwei Teile auf, $w(x) = w_1(x) + w_2(x)$. Der erste Teil weist am Balkenende die richtige Durchbiegung, $w_1(x) = w_\Delta $ auf, und der zweite Teil ist die L\"{o}sung der Differentialgleichung
\begin{align} \label{Eq117}
EI\,w_2^{IV}(x)= - EI\,w_1^{IV}(x) \quad w_2(0) = w_2'(0) = w_2(l) = 0 \quad M_2(l) = 0\,.
\end{align}
Ist $\delta w$ eine zul\"{a}ssige virtuelle Verr\"{u}ckung (dieselben geometrischen Lagerbedingungen wie $w_2$), dann ist
\begin{align}
\text{\normalfont\calligra G\,\,}(w_2,\delta w) = \int_0^{\,l} (- EI\,w_1^{IV})\,\delta w\,dx - \int_0^{\,l} \frac{M_2\,\delta M}{EI}\,dx = 0
\end{align}
und wegen
\begin{align}
\text{\normalfont\calligra G\,\,}(w_1,\delta w) = \int_0^{\,l} ( EI\,w_1^{IV})\,\delta w\,dx - \int_0^{\,l} \frac{M_1\,\delta M}{EI}\,dx = 0
\end{align}
folgt mit $w = w_1 + w_2$ in der Summe
\begin{align}
\text{\normalfont\calligra G\,\,}(w,\delta w) = 0 - \int_0^{\,l} \frac{(M_1 + M_2)\,\delta M}{EI}\,dx = 0\,.
\end{align}
Das erkl\"{a}rt das Ergebnis in Glg. (\ref{Eq120})
\begin{align}
\text{\normalfont\calligra G\,\,}(w,\delta w) &= -\int_0^{\,l} \frac{M \delta M}{EI}\,dx =  -\delta A_i = 0\,.
\end{align}
Mit finiten Elementen macht man f\"{u}r die L\"{o}sung des Randwertproblems (\ref{Eq117}) den Ansatz
\begin{align}
w_h(x) = 0 \cdot \Np_1(x) + 0 \cdot \Np_2(x) + w_\Delta \cdot \Np_3(x) + u_4 \cdot \Np_4(x)
\end{align}
und hat damit schon die Bedingung $w_h(l) = w_\Delta$ erf\"{u}llt. Das $u_4$ braucht man, um sp\"{a}ter die Forderung $M(l) = 0 $ zu erf\"{u}llen. F\"{u}r diesen Ansatz---wie f\"{u}r alle Funktionen---muss gelten
\begin{align}
\text{\normalfont\calligra G\,\,}(w_h,\Np_i) = \delta A_a(p_h,\Np_i) - \delta A_i(w_h,\Np_i) = 0 \qquad i = 1,2,3,4\,.
\end{align}
Auf Grund der {\em Galerkin-Orthogonalit\"{a}t\/} (hier in den \"{a}u{\ss}eren Arbeiten geschrieben, s. S. \pageref{Eq123})
\begin{align}
\delta A_a(p,\Np_i) -  \delta A_a(p_h\,\Np_i) = f_i - f_i^h =  0\qquad i = 1,2,3,4
\end{align}
ist das dasselbe wie
\begin{align}
\text{\normalfont\calligra G\,\,}(w_h,\Np_i) = f_i - \delta A_i(w_h,\Np_i) = 0 \qquad i = 1,2,3,4\,.
\end{align}
Wegen
\begin{align}
\delta A_i(w_h,\Np_i) = \sum_j\,\delta A_i(\Np_j,\Np_i)\,u_j =  \sum_j\,k_{ij}\,u_j
\end{align}
sind diese vier Gleichungen schlie{\ss}lich identisch mit dem System $\vek K\,\vek u = \vek f$
\begin{align}
 \frac{EI}{l^3} \left[
\begin{array}{r r r r}
 12 & -6l & -12 &-6l \\
 -6l & 4l^2 & 6l &2l^2 \\
 -12 & 6l & 12 & 6l \\
 -6l &2l^2 &6l &4l^2
 \end{array}
  \right]\,\left [\barr{c}0 \\ 0 \\ w_\Delta \\ u_4 \earr \right ] = \left [\barr{c}  f_1 \\ f_2 \\ f_3\\ f_4 \earr \right ]\,.
\end{align}
Man bringt nun die dritte Spalte auf die rechte Seite und streicht auch die dritte Zeile, so dass
die einzige unbekannte Weggr\"{o}{\ss}e $u_4$, der Tangens der Verdrehung des rechten Lagers, an Hand des Gleichungssystems
\begin{align}
 \frac{EI}{l^3} \left[
\begin{array}{r r  r}
 12 & -6l &-6l \\
 -6l & 4l^2  &2l^2 \\
  -6l &2l^2  &4l^2
 \end{array}
  \right]\,\left [\barr{c} 0 \\  0 \\ u_4 \earr \right ] =  \left [\barr{r}  f_1 \\ f_2 \\  f_4 = 0 \earr \right ] -\frac{EI}{l^3} \left [\barr{r}  -12 \\ 6l \\  6l \earr \right ] \cdot  w_\Delta
\end{align}
bestimmt werden kann,
\begin{align}
u_4 = -1.5 \cdot \frac{w_\Delta}{l}\,.
\end{align}
Das Ergebnis hat die Form eines Differenzenquotienten, was ja auch zur Dimension $[L/L] = [\,]$ der ersten Ableitung $u_4 = w'(l)$ als $\tan\,\Np$ passt.

%%%%%%%%%%%%%%%%%%%%%%%%%%%%%%%%%%%%%%%%%%%%%%%%%%%%%%%%%%%%%%%%%%%%%%%%%%%%%%%%%%%%%%%%%%%%%%%%%%%
{\textcolor{blau2}{\section{Die Rolle der finiten Elemente}}}%\index{Grenzen der Numerik}
Vielleicht passt an diese Stelle auch ein Wort \"{u}ber die Rolle der finiten Elemente im Bauwesen. Der Mathematiker versteht unter finiten Elementen Funktionen, w\"{a}hrend f\"{u}r den Ingenieur finite Elemente reale Bauteile sind mit denen er ein Tragwerk nachbildet und so interessiert den Ingenieur  nicht nur der numerische Fehler, sondern auch der Modellfehler.

Beide Fehler sind aber miteinander verschr\"{a}nkt und so bleibt Modell bleibt immer in der Schwebe, weil der numerische Fehler und der Modellfehler verschr\"{a}nkt sind.

Und diese Flexibilit\"{a}t, die Mischung aus mathematischer N\"{a}herung und Variation im Modell ist es, die den Ingenieur eigentlich an den finiten Elementen fasziniert.


 sind wichtig sein, aber sie sind nur eine Seite des Problems. , die den Ingenieur eigentlich und prim\"{a}r interessieren.  Das Thema Modellfehler hat die Mathematik erst seit relativ kurzer Zeit, Stichworte {\em Verification and Validation\/}, entdeckt. auf diese Problematik aufmerksam geworden ist.

 F\"{u}r ihn ist Modell exakt und der Fehler liegt bei den finiten Elementen. F\"{u}r den Ingenieur ist aber schon das Modell eine N\"{a}herung.

Mathematische Fehlersch\"{a}tzer sind eine gro{\ss}e Hilfe, um die Numerik in den Griff zu bekommen, aber gleichgewichtig muss daneben die Analyse des Modellfehlers stehen und hier ist der Sachverstand des Ingenieurs gefragt.

Punkte und Linien haben unterschiedliche 'Kapazit\"{a}t'. Ein Integral wie
\begin{align}
\int_0^{\,l} \sin x\,dx
\end{align}
\"{a}ndert sich nicht, wenn man einen Punkt wegl\"{a}sst, aber man darf kein Intervall $(a,b)$ \"{u}berspringen. Vielleicht ist das der mathematische Hintergrund f\"{u}r die beobachteten Ph\"{a}nomene.

%---------------------------------------------------------------------------------
\begin{figure}
\centering
\includegraphics[width=1.0\textwidth]{\Fpath/U307}
\caption{Wandscheibe unter Eigengewicht \textbf{ a)} Einflussfunktion f\"{u}r $\sigma_{yy}$, \textbf{ b)} Hauptspannungen }
\label{U307}%
\end{figure}%
%---------------------------------------------------------------------------------
%---------------------------------------------------------------------------------
\begin{figure}
\centering
\includegraphics[width=1.0\textwidth]{\Fpath/U310}
\caption{Platte, Einflussfunktion f\"{u}r $m_{xx}$}
\label{U310}%
\end{figure}%
%---------------------------------------------------------------------------------

Ein weiteres Beispiel ist die Scheibe in Bild \ref{U307}. Dort, wo der starre Rand in den freien Rand \"{u}bergeht, liegt ein singul\"{a}rer Punkt. Das Verh\"{a}ltnis der Kr\"{a}fte, die die Einflussfunktion f\"{u}r $\sigma_{yy}$ generieren, betr\"{a}gt wieder 2:1, so dass die nach oben gerichteten Knotenkr\"{a}fte die \"{U}berhand haben und die Einflussfunktion f\"{u}r $\sigma_{yy}$ am Schluss bis in den Punkt $\infty$ ausschwingt.

Zum Schluss noch ein Blick auf die Einflussfunktion f\"{u}r das Moment $m_{xx}$ in einer gelenkig gelagerten Quadratplatte, Bild \ref{U310}. In der Stabstatik werden die Einflussfunktionen f\"{u}r ein Moment $M(x)$ mit einem {\em Quadropol\/} erzeugt, s. S. \pageref{U303}. Wenn man einen solchen Quadropol (in einem k\"{u}hnen Schritt---die Mathematiker m\"{u}ssen wegschauen) auf die Platte \"{u}bertr\"{a}gt, dann steht es zwar 2:2 zwischen den ab- und aufw\"{a}rts gerichteten Kr\"{a}ften, aber die Kr\"{a}fte, die nach unten zeigen, liegen in einer Spur, w\"{a}hrend die anderen beiden Kr\"{a}fte einen Abstand voneinander haben, und dieses handicap f\"{u}hrt wohl dazu, dass die abw\"{a}rts gerichteten Kr\"{a}fte 'gewinnen', und die Platte nach unten geht.

Wir hatten in Kapitel 3, S. \pageref{Punktlager}, die Singularit\"{a}t in den Punktlagern auf das ungleiche Verh\"{a}ltnis der Kr\"{a}fte, 2:1, zur\"{u}ckgef\"{u}hrt, die das Element mit dem Lagerknoten auseinandertreiben, um die Einflussfunktion f\"{u}r $\sigma_{yy}$ zu generieren. Vielleicht l\"{a}sst sich auch so  die Singularit\"{a}t in einspringenden Ecken, wie in Bild \ref{U134} erkl\"{a}ren.

Das Verh\"{a}ltnis der Kr\"{a}fte, die die Einflussfunktion f\"{u}r $\sigma_{yy}$ generieren, betr\"{a}gt zwar 2:2, aber die Kr\"{a}fte, die nach oben dr\"{u}cken, treffen auf einen gr\"{o}{\ss}eren Widerstand als die beiden Kr\"{a}fte, die die offene Flanke, den Rand der \"{O}ffnung, nach unten dr\"{u}cken, s. Bild \ref{U306} b. Vielleicht erkl\"{a}rt dies die Singularit\"{a}t in den Ecken der \"{O}ffnungen.


%---------------------------------------------------------------------------------
\begin{figure}
\centering
\includegraphics[width=0.6\textwidth]{\Fpath/U306}
\caption{Knotenkr\"{a}fte f\"{u}r Einflussfunktionen, Spannung $\sigma_{yy} $ in einer einspringenden Ecke}
\label{U306}%
\end{figure}%
%---------------------------------------------------------------------------------


%---------------------------------------------------------------------------------
\begin{figure}
\centering
\includegraphics[width=0.8\textwidth]{\Fpath/U136}
\caption{Die Einflussfunktion f\"{u}r die Durchbiegung in Plattenmitte \"{a}ndert sich praktisch nicht, wenn man die Umgebung des Aufpunktes verfeinert}
\label{U136}%
\end{figure}%
%---------------------------------------------------------------------------------

%%%%%%%%%%%%%%%%%%%%%%%%%%%%%%%%%%%%%%%%%%%%%%%%%%%%%%%%%%%%%%%%%%%%%%%%%%%%%%%%%%%%%%%%%%%%%%%%%%%
\textcolor{blau2}{\section{'Explodierende' Einflussfunktionen}}
Mit den Singularit\"{a}ten h\"{a}ngt ein bemerkenswerter Effekt zusammen, n\"{a}mlich, dass ihre Einflussfunktionen die Punkte eines Tragwerks in den Punkt $\infty$ verschieben. Wenn man die Einflussfunktion f\"{u}r die Spannung $\sigma_{yy}$ im Punktlager einer Scheibe berechnet, s. Bild \ref{U137}, und die Scheibe adaptiv verfeinert, dann werden die Werte der Einflussfunktion am oberen Rand der Scheibe immer gr\"{o}{\ss}er und bei unendlich feinem Netz werden sie wohl unendlich gro{\ss}, w\"{a}hrend Einflussfunktionen, die zu beschr\"{a}nkten, 'normalen' Werten geh\"{o}ren, in der Ferne abklingen, s. Bild \ref{U136}, auch dann, wenn man die Umgebung des Aufpunktes adaptiv verfeinert. Einflussfunktionen, die zu singul\"{a}ren Werten geh\"{o}ren, lassen dagegen eine Scheibe 'explodieren'.

Es scheint nun theoretisch denkbar, dass es Singularit\"{a}ten gibt, die nur bei vertikal oder horizontal gerichteten Lasten auftreten. Aber dann w\"{u}rde der winzigste Drift der Belastung in  die singul\"{a}re Richtung sofort zu unendlich gro{\ss}en Spannungen f\"{u}hren, und diese m\"{u}ssten ebenso abrupt wieder verschwinden, wenn man die Last wieder zur\"{u}ckstellt. Wir haben keine Idee, wie so etwas aussehen k\"{o}nnte, wie ein Tragwerk so 'auf Kipp' stehen k\"{o}nnte.


%---------------------------------------------------------------------------------
\begin{figure}
\centering
\includegraphics[width=0.8\textwidth]{\Fpath/U137}
\caption{Einflussfunktion f\"{u}r die Spannung $\sigma_{yy}$ im Lagerknoten. Mit einer adaptiven Verfeinerung \"{a}ndern sich die Ergebnisse, wie z.B. die Verschiebung des oberen Randes, merkbar}
\label{U137}%
\end{figure}%
%---------------------------------------------------------------------------------


\footnote{Auch die Fachwerkmodelle die beim Durchstanznachweis das Tragverhalten des Betons nachbilden, sind finite Elemente!}

%%%%%%%%%%%%%%%%%%%%%%%%%%%%%%%%%%%%%%%%%%%%%%%%%%%%%%%%%%%%%%%%%%%%%%%%%%%%%%%%%%%%%%%%%%%%%%%%%%%
\textcolor{blau2}{\subsection{Die Dimension der $j_i$}}\label{Dimji}
Wenn man Einflussfunktionen, $\vek K\vek g = \vek j$, berechnet, dann haben die $j_i$ die Dimension einer Arbeit. Wir wollen das am Beispiel der Einflussfunktion f\"{u}r die Spannung $\sigma_{xx}$ in einer Scheibe verifizieren.

Wie passt dazu z.B. das in dem Beispiel auf S. XXX
\begin{align}
j_i =
\end{align} 
\textcolor{chapterTitleBlue}{\chapter{Grundlagen}}
%%%%%%%%%%%%%%%%%%%%%%%%%%%%%%%%%%%%%%%%%%%%%%%%%%%%%%%%%%%%%%%%%%%%%%%%%%%%%%%%%%%%%%%%%%%%%%%%%%%
{\textcolor{sectionTitleBlue}{\section{Einf\"{u}hrung}}
Zur Einleitung wollen wir kurz die Arbeits- und Energieprinzipe der Statik\\

\begin{itemize}
  \item das Prinzip der virtuellen Verr\"{u}ckungen
  \item den Energieerhaltungssatz
  \item das Prinzip der virtuellen Kr\"{a}fte
  \item den Satz von Betti
\end{itemize}

in moderner Form herleiten, um das Thema Einflussfunktionen ausf\"{u}hrlich und pr\"{a}zise behandeln zu k\"{o}nnen.
%----------------------------------------------------------------------------------------------------------
\begin{figure}[tbp]
\centering
\if \bild 2 \sidecaption \fi
\includegraphics[width=0.85\textwidth]{\Fpath/U536}
\caption{ Virtuelle Verr\"{u}ckung \textbf{ a)} eines starren Stabes und \textbf{ b)} eines beidseitig festgehaltenen elastischen Stabes. Die beiden Integrale sind f\"{u}r jedes zul\"{a}ssige $\delta u $ gleich} \label{U536}
%
\end{figure}%
%----------------------------------------------------------------------------------------------------------

%%%%%%%%%%%%%%%%%%%%%%%%%%%%%%%%%%%%%%%%%%%%%%%%%%%%%%%%%%%%%%%%%%%%%%%%%%%%%%%%%%%%%%%%%%%%%%%%%%%
{\textcolor{sectionTitleBlue}{\subsection{Partielle Integration }}
Wir beginnen nicht gleich mit der Statik, sondern wir wollen zuvor noch an die partielle Integration erinnern
\begin{align}
\int_{0}^{l} u'\,\delta u\,dx = [u\,\delta u]_0^l - \int_{0}^{l} u\,\delta u'\,dx\,.
\end{align}
Schreiben wir das als \glq Null-Summe\grq{},
\begin{align}\label{Eq183}
\text{\normalfont\calligra I\,\,}(u, \delta u) = \int_0^{\,l} u'\,\delta u\,dx - [u\,\delta u]_{@0}^{@l} + \int_0^{\,l} u\,\delta u'\,dx = 0\,,
\end{align}
dann haben wir einen Ausdruck vor uns, der f\"{u}r {\em alle\/} Paare von Funktionen aus $C^1(0,l)$, wie $u = \sin(x)$ und $\delta u = \cos(x)$ null ist. Ein wie m\"{a}chtiges Resultat das ist, wird  verst\"{a}ndlich, wenn wir es mit Zahlen wiederholen. Eine Identit\"{a}t wie
\begin{align}
\text{\normalfont\calligra I\,\,}(a, b) = a \cdot b - b \cdot a = 0 \qquad \text{f\"{u}r alle Zahlen $a, b$}
\end{align}
scheint uns trivial, aber in (\ref{Eq183}) sind $u $ und $\delta u $  Funktionen, die ja unendlich viele Freiheitsgrade haben, und dann ist das Resultat, f\"{u}r uns zumindest, nicht mehr evident -- f\"{u}r Gau{\ss} war es das wahrscheinlich schon.

So, wie alle geraden Zahlen durch zwei teilbar sind, so gen\"{u}gen {\em alle\/} $C^1$-Funktionen $u $ und $\delta u $ der \glq quecksilbergleichen\grq{} \footnote{weil es zu jedem $u$ unendlich viele $\delta u$ gibt f\"{u}r die die Identit\"{a}t gilt}  Identit\"{a}t (\ref{Eq183}).
Mit dieser Gleichung kommt das \glq {\em f\"{u}r alle $u$\/}\grq{} und \glq {\em f\"{u}r alle $\delta u$\/}\/}\grq{} in die Welt, auf dem die Arbeits- und Energieprinzipe der Mechanik und Statik beruhen und daher haben wir gemeint, mit der partiellen Integration beginnen zu m\"{u}ssen.

%%%%%%%%%%%%%%%%%%%%%%%%%%%%%%%%%%%%%%%%%%%%%%%%%%%%%%%%%%%%%%%%%%%%%%%%%%%%%%%%%%%%%%%%%%%%%%%%%%%
{\textcolor{sectionTitleBlue}{\subsection{Das Prinzip der virtuellen Verr\"{u}ckungen }}
Wenn an einem Stab zwei gegengleiche Kr\"{a}fte $\pm f$ ziehen wie in Abb. \ref{U536} a, also
\begin{align}
-f + f = 0\,,
\end{align}
dann kann man die Gleichung mit einer beliebigen Zahl $\textcolor{red}{\delta u}$ multiplizieren, ohne etwas an dem Ergebnis zu \"{a}ndern
\begin{align}
\textcolor{red}{\delta u} \cdot (-f + f) = - \textcolor{red}{\delta u}\cdot f + \textcolor{red}{\delta u} \cdot f = 0\,.
\end{align}
Statisch bedeutet dies, dass man den Stab beliebig verschieben kann $(\textcolor{red}{\delta u})$ und dass jedesmal die Arbeit der beiden Stabendkr\"{a}fte in der Summe null ist. Das ist das einfachste Beispiel des {\em Prinzips der virtuellen Verr\"{u}ckungen\/}.

Formal beruht das Prinzip auf der einfachen Tatsache, dass, wenn eine Gleichung null ist
\begin{align}
Eq = 0\,,
\end{align}
dass dann auch das Produkt der Gleichung mit beliebigen Zahlen $\delta u$ null ist
\begin{align}
\textcolor{red}{\delta u} \cdot Eq = 0\,.
\end{align}
Was nat\"{u}rlich auch dann gilt, wenn $u$ und $\delta u$ Funktionen sind, s. Abb. \ref{U536} b. Gen\"{u}gt also z.B. die Funktion $u(x)$ der Differentialgleichung
\begin{align}
- EA\,u''(x) - p(x) = 0 \qquad 0 < x < l \,,
\end{align}
dann folgt
\begin{align}
\int_0^{\,l} (-EA\,u'' - p)\,\textcolor{red}{\delta u}\,dx = 0\,,
\end{align}
oder nach partieller Integration, wenn die Randwerte $\textcolor{red}{\delta u(0)} = \textcolor{red}{\delta u(l)} = 0$ null sind,
\begin{align}
\int_0^{\,l} \frac{N\,\textcolor{red}{\delta N}}{EA}\,dx = \int_0^{\,l} p\,\textcolor{red}{\delta u}\,dx\,.
\end{align}
%----------------------------------------------------------------------------------------------------------
\begin{figure}[tbp]
\centering
\if \bild 2 \sidecaption \fi
\includegraphics[width=0.5\textwidth]{\Fpath/U538}
\caption{ Zwei Federn und der Satz von Betti, $k = 3$} \label{U538}
%
\end{figure}%
%----------------------------------------------------------------------------------------------------------

%%%%%%%%%%%%%%%%%%%%%%%%%%%%%%%%%%%%%%%%%%%%%%%%%%%%%%%%%%%%%%%%%%%%%%%%%%%%%%%%%%%%%%%%%%%%%%%%%%%
{\textcolor{sectionTitleBlue}{\subsection{{Der Satz von Betti}}}
Wenn zwei Zahlen $u_1 $ und $u_2 $ die beiden Zwillings-Gleichungen
\begin{align} \label{Eq44}
3\cdot u_1 = 12 \qquad 3\cdot u_2 = 18
\end{align}
l\"{o}sen, (die 3 macht sie zu Zwillingen) und man multipliziert die beiden Gleichungen jeweils mit der anderen Zahl \glq \"{u}ber Kreuz\grq,
\begin{align}
u_2 \cdot 3\cdot u_1 = 12\cdot u_2 \qquad u_1\cdot 3\cdot u_2 = 18\cdot u_1\,,
\end{align}
dann sind die linken Seiten gleich und daher m\"{u}ssen auch die rechten Seiten gleich sein, s. Abb. \ref{U538},
\begin{align}
A_{12} = 12 \cdot x_2= 18 \cdot x_1 = A_{21}\,.
\end{align}
Das ist der {\em Satz von Betti\/} in seiner elementarsten Form:  {\em Die reziproken \"{a}u{\ss}eren Arbeiten zweier Systeme, die im Gleichgewicht sind, sind gleich gro{\ss}\/}. Dahinter steckt einfache Algebra -- wie auch in dem n\"{a}chsten Beispiel.

Multipliziert man die Knotenverschiebungen $\vek u_1$ und $\vek u_2$ eines Fachwerks aus zwei unterschiedlichen Lastf\"{a}llen
\begin{align}
\vek K\,\vek u_1 = \vek f_1 \qquad \vek K\,\vek u_2 = \vek f_2\,,
\end{align}
skalar \glq \"{u}ber Kreuz\grq{}, dann ergibt das das Resultat
\begin{align}
\vek u_2^T\,\vek K\,\vek u_1 = \vek u_2^T\,\vek f_1  \qquad \vek u_1^T\,\vek K\,\vek u_2  = \vek u_1^T\,\vek f_2
\end{align}
und weil die linken Seiten gleich sind, m\"{u}ssen auch die rechten Seiten gleich sein
\begin{align}
\vek u_2^T\,\vek f_1   = \vek u_1^T\,\vek f_2\,,
\end{align}
sind also die reziproken Arbeiten der Knotenkr\"{a}fte gleich gro{\ss}.

%----------------------------------------------------------------------------------------------------------
\begin{figure}[tbp]
\centering
\if \bild 2 \sidecaption \fi
\includegraphics[width=0.6\textwidth]{\Fpath/U537}
\caption{Wenn sich eine Feder unter einer Kraft $f = 1 $ um $u = 1/k$ verl\"{a}ngert, dann verl\"{a}ngert sie sich  unter einer Kraft $f $ um $u = f \cdot g$} \label{U537}
%
\end{figure}%
%----------------------------------------------------------------------------------------------------------

%%%%%%%%%%%%%%%%%%%%%%%%%%%%%%%%%%%%%%%%%%%%%%%%%%%%%%%%%%%%%%%%%%%%%%%%%%%%%%%%%%%%%%%%%%%%%%%%%%%
{\textcolor{sectionTitleBlue}{\subsection{Einflussfunktionen}}}
Um die Gleichung
\begin{align}\label{Eq45}
3\cdot x = 12
\end{align}
zu l\"{o}sen, dividieren wir die rechte Seite durch die Zahl 3, was man auch als Multiplikation der rechten Seite mit dem Faktor $g = 1/3$ lesen kann. {\em \glq The magic number\grq{}\/} $\textcolor{chapterTitleBlue}{g}$ ist die L\"{o}sung der Gleichung
\begin{align}
3\cdot \textcolor{chapterTitleBlue}{g} = 1\,,
\end{align}
wenn also rechts eine 1, eine \glq Punktlast\grq{} steht. Wie nat\"{u}rlich muss dann
die Zahl
\begin{align}
x = \textcolor{chapterTitleBlue}{g} \cdot 12 = \textcolor{chapterTitleBlue}{\frac{1}{3}}\cdot 12 = 4
\end{align}
die L\"{o}sung von (\ref{Eq45}) sein, s. Abb. \ref{U537}. Das ist die Technik der {\em Einflussfunktionen\/} oder {\em Greenschen Funktionen\/} (daher der Buchstabe $\textcolor{chapterTitleBlue}{g}$).

%----------------------------------------------------------------------------------------------------------
\begin{figure}[tbp]
\centering
\if \bild 2 \sidecaption \fi
\includegraphics[width=0.89\textwidth]{\Fpath/U539}
\caption{Biegebalken und Einflussfunktion f\"{u}r die Durchbiegung in Feldmitte} \label{U539}
%
\end{figure}%
%----------------------------------------------------------------------------------------------------------

Soll etwa die Verschiebung $u_i$ eines Fachwerkknotens berechnet werden, so
setzen wir in den Knoten eine Kraft $f_i = 1$,  bestimmen die dazu geh\"{o}rigen Knotenverschiebungen des Fachwerks, den Vektor $\textcolor{chapterTitleBlue}{\vek g_i}$,
\begin{align}
\vek K\,\textcolor{chapterTitleBlue}{\vek g_i} = \vek  e_i \qquad \text{($i$-ter Einheitsvektor)}\,,
\end{align}
und bilden das Skalarprodukt zwischen den Vektoren $\textcolor{chapterTitleBlue}{\vek g_i}$ und $\vek f$
\begin{align}
u_i = \vek e_i^T\,\vek u = \vek e_i^T\,\vek  K^{-1}\,\vek f = \textcolor{chapterTitleBlue}{\vek g_i^T}\,\vek f  \,.
\end{align}
Bei einem Balken setzen wir in analoger Weise eine Einzelkraft $P = 1$ in den Aufpunkt $x$, bestimmen die zugeh\"{o}rige Biegelinie $G(y,x)$, s. Abb. \ref{U539}, \"{u}berlagern die Belastung mit dieser Biegelinie, und erhalten so die Durchbiegung $w(x)$ in dem Punkt $x$
\begin{align}
w(x) = \int_0^{\,l} G(y,x)\,p(y)\,dy\,.
\end{align}
%%%%%%%%%%%%%%%%%%%%%%%%%%%%%%%%%%%%%%%%%%%%%%%%%%%%%%%%%%%%%%%%%%%%%%%%%%%%%%%%%%%%%%%%%%%%%%%%%%%
{\textcolor{sectionTitleBlue}{\subsection{Identit\"{a}ten}}}

Beim Rechnen in der Statik geht es in der Regel um das L\"{o}sen von einzelnen Gleichungen
\begin{align}
k\,u = f\,,
\end{align}
oder ganzen Systemen von Gleichungen wie
\begin{align}
\vek K\,\vek u = \vek f\,,
\end{align}
oder das L\"{o}sen von Differentialgleichungen wie
\begin{align}
EI@w^{IV}(x) = p(x)\,.
\end{align}
Zu jedem der Operatoren auf der linken Seite geh\"{o}rt eine einfache Identit\"{a}t
\begin{align}
\text{\normalfont\calligra B\,\,}(u,\delta u) = \delta u\,k\,u - u\,k\,\delta u = 0
\end{align}
%----------------------------------------------------------------------------------------------------------
\begin{figure}[tbp]
\centering
\if \bild 2 \sidecaption \fi
\includegraphics[width=0.7\textwidth]{\Fpath/U364}
\caption{Die Kontrolle des Gleichgewichts der Kr\"{a}fte an einem Balken beruht auf einer dualen Formulierung, $\text{\normalfont\calligra G\,\,}(w,1) = p \cdot l + V(l) - V(0) = 0$} \label{U364}%
\end{figure}%
%----------------------------------------------------------------------------------------------------------
\begin{align}
\text{\normalfont\calligra B\,\,}(\vek u,\vek \delta \vek u) = \vek \delta \vek u^T\,\vek K\,\vek u - \vek u^T\,\vek K\,\vek \delta \vek u = 0
\end{align}
\begin{align}\label{Eq55}
\text{\normalfont\calligra G\,\,}(w,\delta w) = \int_0^{\,l} EI@w^{IV}\,\delta w\,dx + [V@\delta w - M@\delta w']_{@0}^{@l} - \int_0^{\,l} \frac{M@\delta M}{EI}\,dx = 0\,.
\end{align}
Nur diese letzte Identit\"{a}t ist nicht ganz so evident, weil sie auf partieller Integration beruht und die Funktionen $w$ und $\delta w$ aus $C^4(0,l)$ bzw. $C^2(0,l)$ sein m\"{u}ssen, damit sie richtig ist.\\

\hspace*{-12pt}\colorbox{highlightBlue}{\parbox{0.98\textwidth}{ Die Arbeits- und Energieprinzipe der Statik sind verbale Umschreibungen der Greenschen Identit\"{a}ten}}\\

Die zentrale Rolle des Arbeitsbegriffes (= Skalarprodukt) basiert auf diesen Identit\"{a}ten, denn die wesentlichen Formulierungen der Statik und Mechanik sind {\em duale Formulierungen\/}, sind {\bf \glq Stereo\grq{}}, nicht \glq  Mono\grq{}. Zwei Funktionen, die Biegelinie $w$ und die virtuelle Verr\"{u}ckung $\delta w$, sind in der Identit\"{a}t
\begin{align}\label{Eq182}
\text{\normalfont\calligra G\,\,}(w,\delta w) =  \delta A_a - \delta A_i = 0
\end{align}
miteinander verkn\"{u}pft und die null bedeutet, dass bei jeder Verr\"{u}ckung $\delta w $ die virtuelle \"{a}u{\ss}ere Arbeit gleich der virtuellen inneren Arbeit ist.

Und weil (\ref{Eq55}) f\"{u}r alle $\delta w \in C^2(0,l) $ richtig ist, muss es auch f\"{u}r $\delta w = 1$ gelten
\begin{align}
\text{\normalfont\calligra G\,\,}(w,1) = \int_0^{\,l} EI@w^{IV} \cdot 1 \,dx + V(l)\cdot 1 - V(0) \cdot 1= 0\,,
\end{align}
und damit ist das Gleichgewicht der vertikalen Kr\"{a}fte, die zu der Biegelinie $w $ geh\"{o}ren,  s. Abb. \ref{U364}, wir setzen $ EI@w^{IV} = p $, garantiert.

%----------------------------------------------------------------------------------------------------------
\begin{figure}[tbp]
\centering
\if \bild 2 \sidecaption \fi
\includegraphics[width=0.6\textwidth]{\Fpath/U53}
\caption{Beim Treppensteigen sp\"{u}ren wir den Hauptsatz der Differential- und Integralrechnung} \label{U53}
%
\end{figure}%
%----------------------------------------------------------------------------------------------------------

%%%%%%%%%%%%%%%%%%%%%%%%%%%%%%%%%%%%%%%%%%%%%%%%%%%%%%%%%%%%%%%%%%%%%%%%%%%%%%%%%%%%%%%%%%%%%%%%%%%
{\textcolor{sectionTitleBlue}{\section{Greensche Identit\"{a}ten}}\index{Greensche Identit\"{a}ten}
Wir stellen im Folgenden zun\"{a}chst in knapper Form die wesentlichen Differentialgleichungen der Stabstatik\index{Differentialgleichungen der Stabstatik} vor und notieren die zu ihnen geh\"{o}renden  Identit\"{a}ten, wie sie sich mit partieller Integration ergeben
\begin{align}
\int_0^{\,l} - EA\,u''\,\delta u\,dx = [(- EA\,u')\,\delta u]_0^l - \int_0^{\,l} - EA\,u'\,\delta u'\,dx
\end{align}
-- hier am Beispiel des Stabs, $- EA\,u'' = p$.

%----------------------------------------------------------------------------------------------------------
\begin{figure}[tbp]
\centering
\if \bild 2 \sidecaption \fi
\includegraphics[width=1.0\textwidth]{\Fpath/U155}
\caption{Das Integral der Normalkraft und des Biegemomentes ist null} \label{U155}
%
\end{figure}%
%----------------------------------------------------------------------------------------------------------


Die bekannteste Anwendung der partiellen Integration ist das Treppensteigen\index{Treppensteigen}
\begin{align}\label{Eq83}
\int_a^{\,b} f'(x)\,dx = f(b) - f(a)\,.
\end{align}
Wenn bei jedem Schritt $dx$ in der Horizontalen der Zuwachs an H\"{o}he $df = f'(x)\,dx$ betr\"{a}gt, dann steigt man insgesamt um das Ma{\ss} $f(b) - f(a)$ nach oben, s. Abb. \ref{U53}.

Die Treppenformel\index{Treppenformel} (\ref{Eq83}) ist der {\em Hauptsatz der Differential- und Integralrechnung\/}\index{Hauptsatz der Differential- und Integralrechnung}. Aus ihr folgt z.B., dass das Integral der Normalkraft $N(x) = EA\,u'(x)$ in einem beidseitig festgehaltenen Stab null ist, s. Abb. \ref{U155} a,
\begin{align}\label{Eq31}
\int_0^{\,l} EA\,u'(x)\,dx = [EA\,u]_{@0}^{@l} = EA\,(u(l) - u(0)) = 0
\end{align}
wie auch das Integral der Biegemomente $M(x) = - EI\,w''(x)$ in einem beidseitig eingespannten Balken, s. Abb. \ref{U155} b,
\begin{align}\label{Eq33}
\int_0^{\,l} - EI\,w''(x)\,dx = - EI\,(w'(l) - w'(0)) = 0\,.
\end{align}
Bei partiellen Ableitungen lautet die Regel der partiellen Integration
\begin{align}
\int_{\Omega} u,_i\,v\,d\Omega = \int_{\Gamma} u\,n_i\,v\,ds - \int_{\Omega} u\,v,_i\,d\Omega\,.
\end{align}
Hier ist $\Gamma$ der Rand der Scheibe, der Platte $\Omega$ \"{u}ber die integriert wird, $n_i$ ist die $i$-te Komponente des Normalenvektors $\vek n$ (L\"{a}nge $|\vek n| = 1$) auf $\Gamma$ und\index{$u,_i$} $u,_i = \partial u/\partial x_i$
ist eine abk\"{u}rzende Schreibweise f\"{u}r die Ableitung nach $x_i$.
%----------------------------------------------------------------------------------------------------------
\begin{figure}[tbp]
\centering
\if \bild 2 \sidecaption \fi
\includegraphics[width=1.0\textwidth]{\Fpath/U52}
\caption{Bauteile der Stabstatik} \label{U52}
%
\end{figure}%
%----------------------------------------------------------------------------------------------------------

Wenn eine Scheibe $\Omega$ an ihrem Rand $\Gamma$ festgehalten wird, $u_x = u_y = 0$, dann ist daher das Integral der Spannung
\begin{align}
\sigma_{xx} = E\,(\varepsilon_{xx} + \nu\,\varepsilon_{yy}) =  E\,(u_x,_{x} + \nu\,u_y,_{y})
\end{align}
\"{u}ber $\Omega$, der Mittelwert von $\sigma_{xx}$, null (und ebenso von $\sigma_{yy}$), denn
\beq
\int_{\Omega} E\,(u_x,_{x} + \nu\,u_y,_{y})\,d\Omega =\int_{\Gamma} E\,(u_x\,n_x + \nu\,u_y\,n_y)\,ds = 0\,.
\eeq

{\textcolor{sectionTitleBlue}{\subsection{L\"{a}ngsverschiebung $u(x)$ eines Stabes}}}\index{L\"{a}ngsverformung}
\vspace{-0.7cm}
\begin{align}
- EA\,u''(x) = p(x)
\end{align}
\begin{align}\label{Eq106}
\text{\normalfont\calligra G\,\,}(u,\textcolor{red}{\delta u}) = \underbrace{\int_0^{\,l} - EA\,u''(x)\,\textcolor{red}{\delta u(x)}\,dx + [N\,\textcolor{red}{\delta u}]_{@0}^{@l}}_{\text{\"{a}u{\ss}ere virt. Arbeit}} - \underbrace{\int_0^{\,l} \frac{N\,\textcolor{red}{\delta N}}{EA}\,dx}_{\text{innere virt. Arbeit}} = 0\,,
\end{align}
mit der Normalkraft $N = EA\,u'$, s. Abb. \ref{U52}.

Wenn $EA(x)$ ver\"{a}nderlich ist, dann lautet die Differentialgleichung des Stabes $- (EA(x)\,u')' = p(x)$ und partielle Integration
\begin{align}
\int_0^{\,l} - (EA(x)\,u')'\,\textcolor{red}{\delta u}\,dx = [(- EA(x)\,u')\,\textcolor{red}{\delta u}]_0^l - \int_0^{\,l} - EA(x)\,u'\,\textcolor{red}{\delta u'}\,dx
\end{align}
f\"{u}hrt sinngem\"{a}{\ss} auf das Ebenbild der Identit\"{a}t (\ref{Eq106}), denn die Definition von $N = EA(x)\,u'$ \"{a}ndert sich nicht
\begin{align}\label{Eq106}
\text{\normalfont\calligra G\,\,}(u,\textcolor{red}{\delta u}) = \int_0^{\,l} - (EA(x)\,u')'\,\textcolor{red}{\delta u(x)}\,dx + [N\,\textcolor{red}{\delta u}]_{@0}^{@l} - \int_0^{\,l} \frac{N\,\textcolor{red}{\delta N}}{EA}\,dx = 0\,.
\end{align}
Weil $- N' = p$ dasselbe ist wie $- (EA(x)\,u')' = p$, kann man auch schreiben
\begin{align}
\int_0^{\,l} - N'\,\textcolor{red}{\delta u}\,dx = [N\,\textcolor{red}{\delta u}]_0^l - \int_0^{\,l} -N\,\textcolor{red}{\delta u'}\,dx\,.
\end{align}
Wenn die Ausdehnung des Stabes durch Reibung ($c$) behindert wird,
\begin{align}
- EA\,u''(x) + c\,u(x) = p(x)\,,
\end{align}
dann lautet die Identit\"{a}t
\begin{align}
\text{\normalfont\calligra G\,\,}(u,\textcolor{red}{\delta u}) &= \underbrace{\int_0^{\,l}\!\!\! (- EA\,u''(x) + c\,u(x))\,\textcolor{red}{\delta u(x)}\,dx + [N\,\textcolor{red}{\delta u}]_{@0}^{@l}}_{\delta A_a}\nn \\
& - \underbrace{\int_0^{\,l} (\frac{N\,\textcolor{red}{\delta N}}{EA} + c\,u\,\textcolor{red}{\delta u})\,dx}_{\delta A_i} = 0\,.
\end{align}

{\textcolor{sectionTitleBlue}{\subsection{Schubverformung $w_S(x)$ eines Balkens}}}\index{Schubverformung}
\vspace{-0.7cm}
\begin{align}
- GA\,w_s''(x) = p(x)
\end{align}

\begin{align}
\text{\normalfont\calligra G\,\,}(w_s,\textcolor{red}{\delta w}_s) = \underbrace{\int_0^{\,l} - GA\,w_s''(x)\,\textcolor{red}{\delta w_s(x)}\,dx + [V\,\textcolor{red}{\delta w_s}]_{@0}^{@l}}_{\delta A_a} - \underbrace{\int_0^{\,l} \frac{V\,\textcolor{red}{\delta V}}{GA}\,dx}_{\delta A_i} = 0\,,
\end{align}
mit $V = GA\,w_s'$\,.

Wenn der Balken auf einer elastischen Grundlage ($c$) ruht,
\begin{align}
- GA\,w_s''(x) + c\,w_s(x) = p(x)\,,
\end{align}
dann hat die Identit\"{a}t die Gestalt
\begin{align}
\text{\normalfont\calligra G\,\,}(w_s,\textcolor{red}{\delta w}_s) &= \underbrace{\int_0^{\,l} (- GA\,w_s''(x) + c\,w(x))\,\textcolor{red}{\delta w_s(x)}\,dx + [V\,\textcolor{red}{\delta w_s}]_{@0}^{@l}}_{\delta A_a}\nn \\
 &- \underbrace{\int_0^{\,l} (\frac{V\,\textcolor{red}{\delta V}}{GA}\, +c\,w_s\,\textcolor{red}{\delta w_s}) dx}_{\delta A_i} = 0\,.
\end{align}

{\textcolor{sectionTitleBlue}{\subsection{Durchbiegung $w$ eines Seils}}}\index{Durchbiegung, Seil}
\vspace{-0.7cm}
\begin{align}
- H\,w''(x) = p(x) \qquad H = \text{Horizontalzug im Seil}
\end{align}
mit $V(x) = H\,w'(x)$ als der Querkraft in dem Seil
\begin{align}
\text{\normalfont\calligra G\,\,}(w,\textcolor{red}{\delta w}) = \underbrace{\int_0^{\,l} - H\,w''(x)\,\textcolor{red}{\delta w(x)}\,dx + [V\,\textcolor{red}{\delta w}]_{@0}^{@l}}_{\delta A_a}  - \underbrace{\int_0^{\,l} \frac{V\,\textcolor{red}{\delta V}}{H}\,dx}_{\delta A_i}  = 0\,.
\end{align}%
{\textcolor{sectionTitleBlue}{\subsection{Durchbiegung $w$ eines Balkens}}}\index{Durchbiegung, Balken Th. I. Ordg.}
\vspace{-0.7cm}
\begin{align}\label{Eq115}
EI\,w^{IV}(x) = p(x)
\end{align}
\begin{align}\label{Eq107}
\text{\normalfont\calligra G\,\,}(w,\textcolor{red}{\delta w}) = \underbrace{\int_0^{\,l} EI\,w^{IV}(x)\,\textcolor{red}{\delta w}\,dx + [V\,\textcolor{red}{\delta w} - M\,\textcolor{red}{\delta w'}]_{@0}^{@l}}_{\delta A_a}  - \underbrace{\int_0^{\,l} \frac{M\,\textcolor{red}{\delta M}}{EI}\,dx}_{\delta A_i} = 0\,,
\end{align}
mit $M(x) = - EI\,w''(x)$ und $V(x) = - EI\,w'''(x)$.

Wenn $EI(x)$ ver\"{a}nderlich ist, dann lautet die Differentialgleichung des Balkens $(EI(x)\,w'')'' = p(x)$ und zweimalige partielle Integration
\begin{align}
\int_0^{\,l} (EI(x)\,w'')''\,\textcolor{red}{\delta w}\,dx &= [(EI(x)\,w'')'\,\textcolor{red}{\delta w} - EI(x)\,w''\,\textcolor{red}{\delta w'}]_0^l \nn \\
&+ \int_0^{\,l} EI(x)\,w''\,\textcolor{red}{\delta w''}\,dx
\end{align}
f\"{u}hrt mit $M = -EI(x)\,w''$ und $V = -(EI(x)\,w'')'$ auf die zu (\ref{Eq107}) analoge Identit\"{a}t.
\begin{align}
\text{\normalfont\calligra G\,\,}(w,\textcolor{red}{\delta w}) =\int_0^{\,l} (EI(x)\,w'')''\,\textcolor{red}{\delta w}\,dx + [V\,\textcolor{red}{\delta w} - M\,\textcolor{red}{\delta w'}]_{@0}^{@l}  - \int_0^{\,l} \frac{M\,\textcolor{red}{\delta M}}{EI}\,dx = 0\,.
\end{align}
Auch hier kann man, weil $-M'' = p$ dasselbe ist wie $(EI(x) w'')'' = p$, schreiben
\begin{align}
\int_0^{\,l} -M''\,\textcolor{red}{\delta w}\,dx = - [M'\,\textcolor{red}{\delta w}]_0^l + \int_0^{\,l} M'\,\textcolor{red}{\delta w'}\,dx\,.
\end{align}
{\textcolor{sectionTitleBlue}{\subsection{Durchbiegung $w$ eines Balkens, Theorie II. Ordnung}}}\index{Durchbiegung, Balken Th. II. Ordg.}
\vspace{-0.7cm}
\begin{align}
EI\,w^{IV}(x) + (D(x)\,w'(x))' = p_z(x) \qquad D(x) = P + \int_0^{\,x} p_x(y)\,dy
\end{align}
\begin{align}
\text{\normalfont\calligra G\,\,}(w,\textcolor{red}{\delta w}) &= \underbrace{\int_0^{\,l} (EI\,w^{IV}(x) + (D(x)\,w'(x))') \,\textcolor{red}{\delta w}\,dx + [T\,\textcolor{red}{\delta w} - M\,\textcolor{red}{\delta w'}]_{@0}^{@l}}_{\delta A_a} \nn \\
&- \underbrace{\int_0^{\,l} (\frac{M\,\textcolor{red}{\delta M}}{EI} - D(x)\,w'(x)\,\textcolor{red}{\delta w'(x)})\,dx}_{\delta A_i} = 0
\end{align}
mit der {\em Transversalkraft\/}\index{Transversalkraft}
\begin{align}
 T(x) = - EI\,w'''(x) - D(x)\,w'(x) = V(x) - D(x)\,w'(x)\,,
 \end{align}
als der Erweiterung der Querkraft um den vertikalen Anteil aus der schr\"{a}g gerichteten $(w' = \tan\,\Np)$ Druckkraft $D$.

Die Konstante $P$ ist eine Druckkraft in dem Stab und $p_x(x)$ und $p_z(x)$ sind Linienkr\"{a}fte in Achsrichtung und senkrecht dazu.

{\textcolor{sectionTitleBlue}{\subsection{Elastisch gebetteter Tr\"{a}ger}}}\index{elastisch gebetteter Tr\"{a}ger}
\vspace{-0.7cm}
\begin{align}
EI\,w^{IV}(x) + c\,w(x) = p(x)
\end{align}
Hierzu geh\"{o}rt die Identit\"{a}t
\begin{align}
\text{\normalfont\calligra G\,\,}(w,\textcolor{red}{\delta w}) &= \underbrace{\int_0^{\,l} (EI\,w^{IV}(x) + c\,w(x))\,\textcolor{red}{\delta w(x)}\,dx + [V\,\textcolor{red}{\delta w} - M\,\textcolor{red}{\delta w'}]_{@0}^{@l}}_{\delta A_a} \nn  \\
&- \underbrace{\int_0^{\,l}(\frac{M\,\textcolor{red}{\delta M}}{EI} + c\,w(x)\,\textcolor{red}{\delta w(x)})\,dx}_{\delta A_i} = 0\,.
\end{align}

{\textcolor{sectionTitleBlue}{\subsection{Zugbandbr\"{u}cke}}}\index{Zugbandbr\"{u}cke}

Man stelle sich einen Balken vor, durch den ein vorgespanntes Seil gezogen wird, so dass Balken und Seil gemeinsam die Streckenlast $p$ tragen
\begin{align}
EI\,w^{IV}(x) - H\,w''(x) = p(x) \qquad H = \text{Vorspannkraft}
\end{align}
\begin{align}
\text{\normalfont\calligra G\,\,}(w,\textcolor{red}{\delta w}) &= \underbrace{\int_0^{\,l} (EI\,w^{IV}(x) - H\,w''(x))\,\textcolor{red}{\delta w(x)}\,dx + [V\,\textcolor{red}{\delta w} - M\,\textcolor{red}{\delta w'}]_{@0}^{@l}}_{\delta A_a} \nn  \\
&- \underbrace{\int_0^{\,l}(\frac{M\,\textcolor{red}{\delta M}}{EI} + H\,w'(x)\,\textcolor{red}{\delta w'(x)})\,dx}_{\delta A_i} = 0\,,
\end{align}
mit $V = - EI\,w'''(x) + H\,w'(x)$.

{\textcolor{sectionTitleBlue}{\subsection{Torsion}}}\index{Torsion}
Die Differentialgleichung der {\em St. Venantschen Torsion\/}\index{St. Venantsche Torsion}
\begin{align}
- G\,I_T\,\vartheta '' = m_x
\end{align}
\begin{align}
\text{\normalfont\calligra G\,\,}(\vartheta,\textcolor{red}{\delta \vartheta}) = \underbrace{\int_0^{\,l} -  G\,I_T\,\vartheta''(x)\,\textcolor{red}{\delta \vartheta(x)}\,dx + [M_T\,\textcolor{red}{\delta \vartheta}]_{@0}^{@l}}_{\delta A_a} - \underbrace{\int_0^{\,l} \frac{M_T\,\textcolor{red}{\delta M_T}}{G\,I_T}\,dx}_{\delta A_i} = 0\,,
\end{align}
und der {\em W\"{o}lbkrafttorsion\/}\index{W\"{o}lbkrafttorsion}
\begin{align}
EI_\omega\,\vartheta^{IV} - G\,I_T\,\vartheta'' = m_x
\end{align}
\begin{align}
\text{\normalfont\calligra G\,\,}(\vartheta,\textcolor{red}{\delta \vartheta}) &= \underbrace{\int_0^{\,l} (EI_\omega\,\vartheta^{IV}(x) - G\,I_T\,\vartheta''(x))\,\textcolor{red}{\delta \vartheta(x)}\,dx + [M_T\,\textcolor{red}{\delta \vartheta} - M_\omega\,\textcolor{red}{\delta \vartheta'}]_{@0}^{@l}}_{\delta A_a} \nn  \\
&- \underbrace{\int_0^{\,l}(\frac{M_\omega\,\textcolor{red}{\delta M_\omega}}{EI_\omega} + G\,I_T\,\vartheta'(x)\,\textcolor{red}{\delta \vartheta'(x)})\,dx}_{\delta A_i} = 0\,,
\end{align}
mit
\begin{align}
M_\omega = - EI_\omega\,\vartheta''(x) \qquad M_T = - EI_\omega\,\vartheta'''(x) + G\,I_T\,\vartheta'(x)
\end{align}
wiederholen die obigen Muster.

%%%%%%%%%%%%%%%%%%%%%%%%%%%%%%%%%%%%%%%%%%%%%%%%%%%%%%%%%%%%%%%%%%%%%%%%%%%%%%%%%%%%%%%%%%%%%%%%%%%
{\textcolor{sectionTitleBlue}{\section{Die Arbeitss\"{a}tze der Statik}}}\index{Arbeitss\"{a}tze der Statik}
In allen Identit\"{a}ten, wie z.B. der des Seils,
\begin{align}
\text{\normalfont\calligra G\,\,}(w,\textcolor{red}{\delta w}) = \int_0^{\,l} - H\,w''(x)\,\textcolor{red}{\delta w(x)}\,dx + [V\,\textcolor{red}{\delta w}]_{@0}^{@l} - \int_0^{\,l} \frac{V\,\textcolor{red}{\hat{V}}}{H}\,dx = 0\,,
\end{align}
werden Kr\"{a}fte $[F]$\index{[F]} und Wege $[L]$\index{[L]} \"{u}berlagert, werden Arbeiten = $[F \cdot L] $ gez\"{a}hlt
\begin{align}
\int_0^{\,l} - H\,w''(x)\,\delta w(x) \,dx &= [F  / L] \cdot [L] \cdot [L] = [F  \cdot L]\\
[V\,\delta w]_{@0}^{@l} = V(l)\, \delta w(l) - V(0) \,\delta w(0) &= [F  \cdot L] -[F  \cdot L]\\
\int_0^{\,l} \frac{V\,\hat{V}}{H}\,dx &= \frac{[F] \cdot [F]}{[F]}\,[L] = [F  \cdot L]\,,
\end{align}
und die Bilanz ergibt am Schluss null. Auf diesem \glq Null-Summen-Spiel\grq{} beruhen die Arbeits- und Energieprinzipe der Balkenstatik.
\pagebreak
{\textcolor{sectionTitleBlue}{\subsubsection*{Prinzip der virtuellen Verr\"{u}ckungen}}}\index{Prinzip der virtuellen Verr\"{u}ckungen}

\vspace{-0.7cm}
\begin{align}
\boxed{\text{\normalfont\calligra G\,\,}(w, \textcolor{red}{\delta w}) = \delta A_a - \delta A_i = 0\,.}
\end{align}
{\textcolor{sectionTitleBlue}{\subsubsection*{Energieerhaltungssatz}}}\index{Energieerhaltungssatz}

Ist das zweite Argument identisch mit dem ersten, dann formuliert die erste Greensche Identit\"{a}t den Energieerhaltungssatz
\begin{align}
\boxed{\frac{1}{2}\, \text{\normalfont\calligra G\,\,}(w,  w) =  A_a -  A_i = 0\,,}
\end{align}
der besagt, dass die \"{a}u{\ss}ere Eigenarbeit (deswegen der Faktor $1/2 $) als innere Energie gespeichert wird.

{\textcolor{sectionTitleBlue}{\subsubsection*{Prinzip der virtuellen Kr\"{a}fte}}}\index{Prinzip der virtuellen Kr\"{a}fte}

R\"{u}ckt man $w(x) $ an die zweite Stelle und \"{u}berl\"{a}sst den ersten Platz einer Testfunktion $\textcolor{red}{\delta w^*} $, die man, wie es Tradition ist, mit einem Asterisk schreibt, dann ist es das Prinzip der virtuellen Kr\"{a}fte
\begin{align}
\boxed{\text{\normalfont\calligra G\,\,}(\textcolor{red}{\delta w^*},w) = \delta A_a^* - \delta A_i^* = 0\,.}
\end{align}
{\textcolor{sectionTitleBlue}{\subsubsection*{Satz von Betti}}}\index{Satz von Betti}
Auch der Satz von Betti geh\"{o}rt an diese Stelle, weil er durch Spiegelung aus der ersten Greenschen Identit\"{a}t entsteht
\begin{align}
\!\!\text{\normalfont\calligra B\,\,}(w,\,\textcolor{chapterTitleBlue}{\hat{w}}) &= \!\text{\normalfont\calligra G\,\,}(w,\textcolor{chapterTitleBlue}{\hat{w}}) - \!\!\! \text{\normalfont\calligra G\,\,}(\textcolor{chapterTitleBlue}{\hat{w}}, w)  = \!\!\underbrace{\int_0^{\,l} EI\,w^{IV}(x)\,\textcolor{chapterTitleBlue}{\hat{w}(x)}\,dx + [V\,\textcolor{chapterTitleBlue}{\hat{w}} - M\,\textcolor{chapterTitleBlue}{\hat{w}'}]_{@0}^{@l}}_{A_{1,2}}\nn \\
& - \underbrace{[w\,\textcolor{chapterTitleBlue}{\hat{V}}- w' \textcolor{chapterTitleBlue}{\hat{M}}]_{@0}^{@l}  - \int_0^{\,l} w(x)\,\textcolor{chapterTitleBlue}{EI\,\hat{w}^{IV}(x)}\,dx}_{A_{2,1}}= 0\,,
\end{align}
was bedeutet, dass die reziproken \"{a}u{\ss}eren Arbeiten zweier Biegelinien $w $ und $\hat{w} $ gleich gro{\ss} sind, $$\boxed{\text{\normalfont\calligra B\,\,}(w,\hat{w}) = A_{1,2} - A_{2,1} = 0\,.}$$

{\textcolor{sectionTitleBlue}{\subsubsection*{Prinzip vom Minimum der potentiellen Energie}}}\index{Prinzip vom Minimum der potentiellen Energie}
Die potentielle Energie eines gelenkig gelagerten Einfeldtr\"{a}gers ist -- in klassischer und moderner Notation nebeneinander -- der Ausdruck
\begin{align}
\Pi(w) &=\frac{1}{2}\, \int_0^{\,l} \frac{M^2}{EI}\,dx - \int_0^{\,l} p(x)\,w(x)\,dx = \frac{1}{2}\, a(w,w) - (p,w)\nn\\
 &=  \frac{1}{2}\, a(w,w) - \frac{1}{2}\,(p,w) - \frac{1}{2}\,(p,w)\,.
\end{align}
Ist $w$ die Biegelinie des Tr\"{a}gers, $EI\,w^{IV} = p$, dann ist $a(w,w) - (p,w) = \,0$  und dann verk\"{u}rzt sich das auf
\begin{align}
\Pi(w) &=- \frac{1}{2}\,(p,w) = - \frac{1}{2} \int_0^{\,l} p(x)\,w(x)\,dx\,,
\end{align}
woraus folgt, dass die potentielle Energie in der Gleichgewichtslage negativ ist, weil die {\em Eigenarbeit\/} $(p,w)$ immer positiv ist.

Addiert man zur tiefsten Lage $w$ eine zul\"{a}ssige virtuelle Verr\"{u}ckung $\delta w$, also $\delta w(0) = \delta w(l) = 0$, dann wird die potentielle Energie gr\"{o}{\ss}er
\begin{align}
\Pi(w + \delta w) = \Pi(w) + \underbrace{\text{\normalfont\calligra G\,\,}(w,\delta w)}_{=\, 0} + \underbrace{a(\delta w,\delta w)}_{> \,0}\,,
\end{align}
was belegt, dass $\Pi(w)$ wirklich der tiefste Punkt ist. Ferner gilt:\\

\hspace*{-12pt}\colorbox{highlightBlue}{\parbox{0.98\textwidth}{Die erste Variation $\delta \Pi$ der potentiellen Energie ist identisch mit der ersten Greenschen Identit\"{a}t,}}
\begin{align}
\delta \Pi(w, \delta w) = a(w, \delta w) - (p, \delta w) = \text{\normalfont\calligra G\,\,}(w,\delta w) = 0\,.
\end{align}
Deswegen bilden die Greenschen Identit\"{a}ten die Vorlage f\"{u}r die finiten Elemente. Man konstruiert eine L\"{o}sung $w_h = \sum_j w_j\,\Np_j(x)$ so, dass
\begin{align}\label{Eq190}
\boxed{ a(w_h, \Np_i) - (p, \Np_i)= 0 \quad i = 1,2,\ldots, n \quad \text{oder}\quad \vek K @\vek w - \vek f = \vek 0\,.}
\end{align}

{\textcolor{sectionTitleBlue}{\subsubsection*{Free body diagram}}}\index{free body diagram}
Es sollte klar sein, dass die Greenschen Identit\"{a}ten am frei geschnittenen System ({\em free body diagram\/}) formuliert werden, denn ohne Randarbeiten $[\ldots ]$ w\"{a}ren die Ausdr\"{u}cke nicht komplett. Sind nur starre Lager vorhanden, kann man auf das Freischneiden verzichten, wenn $\delta w$ eine {\em zul\"{a}ssige\/} virtuelle Verr\"{u}ckung\index{zul\"{a}ssige virtuelle Verr\"{u}ckung} ist.

%----------------------------------------------------------------------------------------------------------
\begin{figure}[tbp]
\centering
\if \bild 2 \sidecaption \fi
\includegraphics[width=0.4\textwidth]{\Fpath/U265A}
\caption{Tumbleweed} \label{U265}
\end{figure}%
%----------------------------------------------------------------------------------------------------------

%%%%%%%%%%%%%%%%%%%%%%%%%%%%%%%%%%%%%%%%%%%%%%%%%%%%%%%%%%%%%%%%%%%%%%%%%%%%%%%%%%%%%%%%%%%%%%%%%%%
{\textcolor{sectionTitleBlue}{\section{Ein Null-Summen-Spiel}}}\index{Null-Summen-Spiel}

Die erste Greensche Identit\"{a}t gleicht dem Spiel, das der W\"{u}stenwind $(= \delta u)$ mit dem ausgetrockneten {\em tumbleweed\/} $(= u)$\index{tumbleweed} treibt, s. Abb. \ref{U265}. Egal wie stark der Wind bl\"{a}st, und wie gro{\ss} die Kapriolen sind, am Schluss ist die Bilanz immer null,  $\text{\normalfont\calligra G\,\,}(u,\delta u) = 0$.

%----------------------------------------------------------------------------------------------------------
\begin{figure}[tbp]
\centering
\if \bild 2 \sidecaption \fi
\includegraphics[width=0.6\textwidth]{\Fpath/U324A}
\caption{Geschlossener Pfad} \label{U324}
\end{figure}%
%----------------------------------------------------------------------------------------------------------

Lesen wir die Identit\"{a}t wie eine Variationsaussage
\begin{align}
\text{\normalfont\calligra G\,\,}(u,\delta u) = 0 \qquad \text{f\"{u}r alle $\delta u$}\,,
\end{align}
dann erinnert sie an die Weg-Unabh\"{a}ngigkeit des Arbeitsintegrals einer Punktmasse $m$ im Schwerefeld der Erde, s. Abb. \ref{U324}. Nahe der Erdoberfl\"{a}che hat die potentielle Energie den Wert $\Pi = m\cdot g \cdot y$ und wenn sich die Punktmasse $m$ auf einem geschlossenen Pfad $\mathcal{ C} = \{x(s), y(s)\}^T$ bewegt, dann ist die Gesamtarbeit null\footnote{Part. Int. und $y(0) = y(L)$ mit $L = $ L\"{a}nge des Pfades}
\begin{align}
\int_{\mathcal{ C}} \cdot \nabla \Pi \dotprod  \vek d\vek s &= m \cdot g \int_{0}^L \left [\barr{c}  0 \\  1\earr \right ] \dotprod  \left [\barr{c}  x' \\  y'\earr \right ]\,ds =  m \cdot g \int_{0}^L  y'\,ds \nn \\
&= m \cdot g \cdot (y(L) - y(0)) = 0
\end{align}
unabh\"{a}ngig von der Gestalt der Kurve $\mathcal {C}$ -- dem Pfad $\delta u$ so zu sagen.

Mit der partiellen Integration kommt die {\em Dualit\"{a}t\/} in die Mechanik hinein, also das Wechselspiel von Kraft und Weg. Die fundamentale Bedeutung des Arbeitsbegriffs f\"{u}r die Mechanik beruht auf den Greenschen Identit\"{a}ten.

Am Anfang steht immer das Skalarprodukt\footnote{Die \"{U}berlagerung zweier Funktionen nennt man ein  $L_2$-Skalarprodukt.}\index{Ueberlagerung}\index{$L_2$-Skalarprodukt}  von zwei konjugierten Gr\"{o}{\ss}en, von einer Kraft und einem Weg,
\begin{align}
\int_0^{\,l}  - EA\,u''\,u\,dx = \text{{\em Kraft\/}} \times \text{{\em Weg\/}}
\end{align}
und wie nat\"{u}rlich entstehen so aus dem Ausgangsintegral durch partielle Integration die Arbeits- und Energieprinzipe der Mechanik und Statik.

Es gibt eben nicht nur die klassische Formel der partiellen Integration
\begin{align}
\text{\normalfont\calligra I\,\,}(u, v) = \int_0^{\,l} u'\,v\,dx - [u\,v]_{@0}^{@l} + \int_0^{\,l} u\,v'\,dx = 0\,,
\end{align}
sondern viele weitere M\"{o}glichkeiten, $\infty$ viele Paare von Funktionen $u$ und $\delta u$ in einem \glq Null-Summen-Spiel\grq{} miteinander zu verkn\"{u}pfen
\begin{align}
\text{\normalfont\calligra G\,\,}(u, \delta u) =  \left \{ \begin{array}{l } \displaystyle{ \int_0^{\,l} - EA\,u''\,\delta u\,dx  + \ldots}   \vspace{0.2cm} \\
 \displaystyle{\int_0^{\,l} EI\,u^{IV}\,\delta u\,dx   + \ldots} \vspace{0.2cm} \\
  \displaystyle{\int_{\Omega} - \Delta u\,\delta u\,d\Omega + \ldots} \vspace{0.2cm} \\
\ldots
\end{array} \right \} = 0\,.
\end{align}

\hspace*{-12pt}\colorbox{highlightBlue}{\parbox{0.98\textwidth}{Und dass es Null-Summen sind, also {\bf \em Invarianten\/} -- man denke an den W\"{u}stenwind -- darauf beruht der Erfolg der Arbeits- und Energieprinzipe.}}


%----------------------------------------------------------------------------------------------------------
\begin{figure}[tbp]
\centering
\if \bild 2 \sidecaption \fi
\includegraphics[width=0.9\textwidth]{\Fpath/U54}
\caption{Kragtr\"{a}ger, aux = Hilfssystem} \label{U54}
\end{figure}%
%----------------------------------------------------------------------------------------------------------

%%%%%%%%%%%%%%%%%%%%%%%%%%%%%%%%%%%%%%%%%%%%%%%%%%%%%%%%%%%%%%%%%%%%%%%%%%%%%%%%%%%%%%%%%%%%%%%%%%%
{\textcolor{sectionTitleBlue}{\section{Beispiele}}}
Nach dieser doch etwas knappen, schlagwortartigen Auflistung sollen nun Beispiele den Inhalt  veranschaulichen.

%%%%%%%%%%%%%%%%%%%%%%%%%%%%%%%%%%%%%%%%%%%%%%%%%%%%%%%%%%%%%%%%%%%%%%%%%%%%%%%%%%%%%%%%%%%%%%%%%%%
{\textcolor{sectionTitleBlue}{\subsection{Das Prinzip der virtuellen Verr\"{u}ckungen}}}\index{Prinzip der virtuellen Verr\"{u}ckungen}
Die Biegelinie des Kragtr\"{a}ger in Abb. \ref{U54}
\begin{align} \label{Eq27}
EI\,w^{IV}(x) = 10 \qquad w(0) = w'(0) = 0 \qquad M(l) = V(l) = 0
\end{align}
hat die Gestalt
\begin{align}
w(x) = \frac{1}{EI}\,(\frac{10}{24}\,x^4 - \frac{50}{6}\,x^3 + \frac{125}{2} \,x^2 )
\end{align}
und die Schnittkr\"{a}fte lauten
\begin{align}
M(x) = -5\,x^2 + 50\,x - 125 \qquad V(x) = -10\,x + 50\,.
\end{align}
Die erste Greensche Identit\"{a}t des Balkens
\begin{align}
\text{\normalfont\calligra G\,\,}(w,\textcolor{red}{ \delta w})  = \int_0^{\,l} EI\,w^{IV}(x)\,\textcolor{red}{\delta w}\,dx + [V\,\textcolor{red}{\delta w} - M\,\textcolor{red}{\delta w'}]_{@0}^{@l} - \int_0^{\,l} \frac{M\,\textcolor{red}{\delta M}}{EI}\,dx = 0\,,
\end{align}
reduziert sich unter Beachtung von (\ref{Eq27}), und der Annahme, dass $\textcolor{red}{\delta w(x)}$ eine zul\"{a}ssige virtuelle Verr\"{u}ckung ist,
\begin{align}\label{Eq28}
\textcolor{red}{\delta w(0)} = 0 \qquad \textcolor{red}{\delta w'(0)} = 0\,,
\end{align}
auf den Ausdruck
\begin{align}
\text{\normalfont\calligra G\,\,}(w, \textcolor{red}{\delta w})  = \int_0^{\,l} 10 \cdot \textcolor{red}{\delta w}\,dx - \int_0^{\,l} \frac{M\,\textcolor{red}{\delta M}}{EI}\,dx = 0\,,
\end{align}
der mit der Bilanz $\delta A_a - \delta A_i = 0$ identisch ist.

W\"{a}hlen wir z.B. als zul\"{a}ssige virtuelle Verr\"{u}ckung die Funktion $\textcolor{red}{\delta w(x) = x^2}$, so finden wir in der Tat, dass die Bilanz null ergibt
\begin{align}
\text{\normalfont\calligra G\,\,}(w, \textcolor{red}{x^2})  &= \int_0^{\,5} 10 \cdot\textcolor{red}{ x^2}\,dx - \int_0^{\,5} (-5\,x^2 + 50\,x - 125)\cdot \textcolor{red}{(-2)}\,dx \nn \\
&= \frac{1250}{3} - \frac{1250}{3} = \delta A_a - \delta A_i = 0\,.
\end{align}
Die virtuelle Verr\"{u}ckung
\begin{align}
\textcolor{red}{\delta w(x) = \cos x}
\end{align}
 ist dagegen keine zul\"{a}ssige virtuelle Verr\"{u}ckung, denn bei dieser Bewegung wird das eigentlich feste linke Lager verr\"{u}ckt, $\textcolor{red}{\delta w(0) = \cos 0 = 1}$. Das setzt aber die G\"{u}ltigkeit von $\text{\normalfont\calligra G\,\,}(w, \textcolor{red}{\delta w}) = 0 $ nicht au{\ss}er Kraft. Man muss jetzt nur richtig z\"{a}hlen und beachten, dass nun auch die Querkraft $V(0) = 50$ eine Arbeit leistet, und so ergibt sich mit
\begin{align}
\textcolor{red}{\delta M(x) = - EI\,\delta w''(x) = EI\,\cos\,x}
\end{align}
auch das richtige Resultat (es ist $-50 \cdot \textcolor{red}{1} = -V(0) \cdot \textcolor{red}{\cos\,0}$)
\begin{align}
\text{\normalfont\calligra G\,\,}(w,\textcolor{red}{\cos x}) &= \int_0^{\,5} 10\,\textcolor{red}{\cos x }\,dx - 50 \cdot \textcolor{red}{1}\, - \int_0^{\,5} (-5\,x^2 + 50\,x - 125)\, \textcolor{red}{\cos\,x}\,dx \nn \\
&= \underbrace{\phantom{[}-9.59 - 50}_{\delta A_a} + \underbrace{\phantom{[}59.59}_{\delta A_i} = 0\,.
\end{align}
Auch die Starrk\"{o}rperbewegungen $\textcolor{red}{\delta w = a + b\,x }$ sind keine zul\"{a}ssigen virtuellen Verr\"{u}ckungen, aber trotzdem ist ihre Anwendung erlaubt und sogar geboten, denn zwei spezielle Starrk\"{o}rperbewegungen, $\textcolor{red}{\delta w(x) = 1}$ und $\textcolor{red}{\delta w(x) = x}$, kontrollieren das Gleichgewicht, also die Summe der vertikalen Kr\"{a}fte und die Summe der Momente um das linke Lager
\begin{alignat}{2}
\text{\normalfont\calligra G\,\,}(w,\textcolor{red}{1}) &= \int_0^{\,5} 10 \cdot \textcolor{red}{1}\, dx - V(0) \cdot \textcolor{red}{1} = 50 - 50 = 0\qquad &&\textcolor{red}{\delta w = 1}\,,\\
\text{\normalfont\calligra G\,\,}(w,\textcolor{red}{x}) &= \int_0^{\,5} 10\cdot\textcolor{red}{x}\,dx - M(0)\cdot \textcolor{red}{1} = 125 - 125 = 0 \qquad &&\textcolor{red}{\delta w = x}\,.
\end{alignat}
($M(0) \cdot \textcolor{red}{1} = M(0) \cdot \textcolor{red}{x'})$.
Nur wenn $w$ orthogonal zu diesen beiden Verr\"{u}ckungen ist, herrscht Gleichgewicht.

%%%%%%%%%%%%%%%%%%%%%%%%%%%%%%%%%%%%%%%%%%%%%%%%%%%%%%%%%%%%%%%%%%%%%%%%%%%%%%%%%%%%%%%%%%%%%%%%%%%
{\textcolor{sectionTitleBlue}{\subsection{Energieerhaltungssatz}}}\index{Energieerhaltungssatz}
Man \"{u}berzeugt sich auch leicht, dass die Biegelinie des Kragtr\"{a}gers dem Energieerhaltungssatz gen\"{u}gt
\begin{align}
\frac{1}{2}\, \text{\normalfont\calligra G\,\,}(w,w) &= \frac{1}{2}\,\int_0^{\,l} p(x)\,w(x)\,dx - \frac{1}{2}\,\int_0^{\,l} \frac{M^2}{EI}\,dx = A_a - A_i \nn \\
& = \frac{1}{2}\,\frac{1}{EI} ( 1562.5 - 1562.5) = 0\,,
\end{align}
dass also die \"{a}u{\ss}ere Eigenarbeit $A_a$  gleich der inneren Energie $A_i$ ist.
%----------------------------------------------------------------------------------------------------------
\begin{figure}[tbp]
\centering
\if \bild 2 \sidecaption \fi
\includegraphics[width=1.0\textwidth]{\Fpath/U18}
\caption{Kragtr\"{a}ger} \label{U18}
\end{figure}%
%----------------------------------------------------------------------------------------------------------

%%%%%%%%%%%%%%%%%%%%%%%%%%%%%%%%%%%%%%%%%%%%%%%%%%%%%%%%%%%%%%%%%%%%%%%%%%%%%%%%%%%%%%%%%%%%%%%%%%%
{\textcolor{sectionTitleBlue}{\subsection{Das Prinzip der virtuellen Kr\"{a}fte}}}\index{Prinzip der virtuellen Kr\"{a}fte}
Bei diesem Prinzip ist die Reihenfolge von $w$ und $\textcolor{red}{\delta w}$ vertauscht und man schreibt dann \"{u}blicherweise $\textcolor{red}{\delta w^*}$ statt $\textcolor{red}{\delta w} $
\begin{align} \label{Eq29}
\text{\normalfont\calligra G\,\,}(\textcolor{red}{\delta w^*},w) &= \int_0^{\,l} EI\,\textcolor{red}{\delta w^{*IV}(x)}\,w(x)\,dx + [\textcolor{red}{\delta V^*}\, w - \textcolor{red}{\delta M^*}\,w']_{@0}^{@l} \nn \\
&- \int_0^{\,l} \frac{\textcolor{red}{\delta M^*}\,M}{EI}\,dx = 0\,.
\end{align}
Die { Mohrsche Arbeitsgleichung} basiert auf dieser Gleichung. Dort schreibt man $\textcolor{red}{\delta w^*} = \bar{w}$

Um auf diesem Weg die Durchbiegung am Kragarmende des Tr\"{a}gers in Abb. \ref{U18} a zu berechnen, belasten wir den Tr\"{a}ger in einem zweiten Lastfall mit einer Einzelkraft $\textcolor{red}{P^* = 1}$, zu der die Biegelinie $\textcolor{red}{\delta w^*(x)}$ geh\"{o}rt
\begin{align}
\textcolor{red}{EI\,\delta w^{*\,IV} = 0 \qquad \delta V^*(l) = 1\qquad \delta M^*(l) = 0}\,.
\end{align}
Mit $w(0) = w'(0) = 0 $ folgt dann
\begin{align}
\text{\normalfont\calligra G\,\,}(\textcolor{red}{\delta w^*},w) = \textcolor{red}{P^*} \cdot  w(l) - \int_0^{\,l} \frac{\textcolor{red}{\delta M^*}\,M}{EI}\,dx = 0\,,
\end{align}
oder
\begin{align}
\textcolor{red}{1}\cdot w(l) = \int_0^{\,l} \frac{\textcolor{red}{\delta M^*}\,M}{EI}\,dx\,,
\end{align}
was die Mohrsche Arbeitsgleichung\index{Mohrsche Arbeitsgleichung} ist.

Nach diesem ersten Probest\"{u}ck wollen wir das {\em Prinzip der virtuellen Kr\"{a}fte\/} nun systematischer fassen. Weil die Pl\"{a}tze vertauscht sind, $\textcolor{red}{\delta w^*(x)}$ an erster Stelle steht,  liefert $\textcolor{red}{\delta w^*(x)} $ die Kraftgr\"{o}{\ss}en, also die Streckenlast
\begin{align}\label{Eq17}
\textcolor{red}{ EI\, \delta\,w^{*IV}(x) = \delta \,p^*}
\end{align}
und ebenso die Momente und Querkr\"{a}fte an den Balkenenden
 \begin{align}\label{Eq18}
\textcolor{red}{\delta V^*(0)} &= \textcolor{red}{- EI\,\delta w^{*'''}(0)} \qquad  \textcolor{red}{\delta V^*(l) = - EI\,\delta w^{*'''}(l)} \nn \\
\textcolor{red}{\delta M^*(0)} &= \textcolor{red}{- EI\,\delta w^{*''}(0)} \qquad  \textcolor{red}{\delta M^*(l) = - EI\,\delta w^{*''}(l)}\,.
 \end{align}
Wir nennen die Gesamtheit der \"{a}u{\ss}eren Kr\"{a}fte, die zu $\textcolor{red}{\delta w^*(x)} $ geh\"{o}ren, $\textcolor{red}{\delta K^*}$.

Weil die Weggr\"{o}{\ss}en  von $\textcolor{red}{\delta w^*} $ an den Balkenenden in der ersten Greenschen Identit\"{a}t nicht abgefragt werden,  muss $\textcolor{red}{\delta w^* }$ keine R\"{u}cksicht auf die Lagerbedingungen des Tr\"{a}gers nehmen.

Die Identit\"{a}t $\text{\normalfont\calligra G\,\,}(\textcolor{red}{\delta w^*},w) = \delta A_a^* - \delta A_i^* = 0 $ ist dann die Bilanz der \"{a}u{\ss}eren Arbeiten $\delta A_a^*$, die die Kr\"{a}fte $\textcolor{red}{\delta K^*}$ auf den Wegen $w(x) $ leisten, minus der virtuellen inneren Energie $\delta A_i^*$, also der \"{U}berlagerung von $\textcolor{red}{\delta M^*}$ und $M$.

In der Literatur wird das {\em Prinzip der virtuellen Kr\"{a}fte\/} wie folgt ausgesprochen: \\

{\textcolor{sectionTitleBlue}{\subsubsection*{Prinzip der virtuellen Kr\"{a}fte}}}
{\em Ist ein System von \"{a}u{\ss}eren Kr\"{a}ften $\textcolor{red}{\delta K^*}$ im Gleichgewicht, dann ist die \"{a}ussere Arbeit $\delta A^*$ dieser Kr\"{a}fte auf den Wegen der Verformung $w$ des Systems
\begin{align}
\delta A_a^* = \int_0^{\,l} \textcolor{red}{EI\,\delta \,w^{*IV}}\,w(x)\,dx + [\textcolor{red}{\delta \,V^*}  w - \textcolor{red}{\delta \,M^*} w']_{@0}^{@l}\,,
\end{align}
gleich der virtuellen inneren Energie $\delta A_i^*$, also dem Integral\/}
\begin{align}
\delta A_i^* = \int_0^{\,l} \frac{\textcolor{red}{\delta \,M^*} M}{EI}\,dx\,.
\end{align}
In der Summe also
\begin{align}
\delta A_a^* - \delta A_i^* = 0\,,
\end{align}
was mit $\text{\normalfont\calligra G\,\,}(\textcolor{red}{\delta w^*},w) = 0$ identisch ist.

Manchmal wird verlangt, dass die  Kr\"{a}fte $\textcolor{red}{\delta K^*}$ {\em infinitesimal klein\/} sein m\"{u}ssen, aber daf\"{u}r gibt es keinen sachlichen Grund, denn partielle Integration macht keinen Unterschied zwischen gro{\ss} und klein.

Das Gleichgewicht der virtuellen Kr\"{a}fte ist garantiert, weil wir die Kr\"{a}fte aus der Funktion $\textcolor{red}{\delta w^*}$ (dem \glq Mutterschiff\grq{}) durch Differentiation abgeleitet haben und jede Funktion $\textcolor{red}{\delta w^*} \in C^4(0,l)$ die Gleich\-gewichtsbedingungen erf\"{u}llt
\begin{align}\label{Eq4}
\text{\normalfont\calligra G\,\,}(\textcolor{red}{\delta w^*},\delta w) =  0  \qquad \delta w = a + b\,x\,.
\end{align}
Anders w\"{a}re es, wenn $\textcolor{red}{EI\,\delta \,w^{*IV}}$ und die Balkenendkr\"{a}fte $\textcolor{red}{\delta\,V^*}$ und $\textcolor{red}{\delta M^*}$ nicht zueinander passen w\"{u}rden, wenn sie \glq gew\"{u}rfelt\grq{} w\"{a}ren, dann w\"{a}re die Bilanz $\delta A_a^* - \delta A_i^*$ wahrscheinlich nicht  null.
%----------------------------------------------------------------------------------------------------------
\begin{figure}[tbp]
\centering
\if \bild 2 \sidecaption \fi
\includegraphics[width=1.0\textwidth]{\Fpath/U2}
\caption{Stockwerkrahmen, Belastung und Verformung} \label{U2}
\end{figure}%
%----------------------------------------------------------------------------------------------------------

%%%%%%%%%%%%%%%%%%%%%%%%%%%%%%%%%%%%%%%%%%%%%%%%%%%%%%%%%%%%%%%%%%%%%%%%%%%%%%%%%%%%%%%%%%%%%%%%%%%
{\textcolor{sectionTitleBlue}{\section{Rahmen}}}\index{Rahmen}
Die Erweiterung der Identit\"{a}ten auf rahmenartige Tragwerke wie in Abb. \ref{U2} ist einfach, denn
$0 + 0 = 0$.

Der Rahmen m\"{o}ge aus $n$ Stielen und Riegeln mit entsprechenden L\"{a}ngs- und Biegeverformungen $u_i$ und $w_i$ bestehen. F\"{u}r jedes
$u_i$ bzw. $w_i $ formulieren wir die zugeh\"{o}rige erste Greensche Identit\"{a}t und dann addieren wir all diese Identit\"{a}ten
 \begin{align}
 0 + 0 + \ldots + 0 = 0\,.
 \end{align}
Im n\"{a}chsten Schritt trennen wir diesen Ausdruck nach \"{a}u{\ss}erer und innerer Arbeit auf. Was in den Identit\"{a}ten \"{a}u{\ss}ere Arbeit ist, bleibt auf der linken Seite und was innere Arbeit ist, kommt auf die rechte Seite, womit wir am Schluss einen Ausdruck wie
\begin{align} \label{Eq13}
\delta A_a = \delta A_i
\end{align}
vor uns haben.

Der Term $\delta A_a $ l\"{a}sst sich in der Regel weiter vereinfachen. Die beiden zu $u_i$ und $w_i$ geh\"{o}rigen Identit\"{a}ten eines Riegels oder Stieles,
\begin{align}
\text{\normalfont\calligra G\,\,}(u_i, \textcolor{red}{\delta u_i}) = 0\quad \text{(l\"{a}ngs)}\qquad  \text{\normalfont\calligra G\,\,}(w_i, \textcolor{red}{\delta w_i}) = 0 \quad \text{(quer)}
\end{align}
tragen in der Summe zu $\delta A_a$ einen Ausdruck wie
\begin{align}
\int_0^{\,l_i} p_x\,\textcolor{red}{\delta u_i}\,dx  + \int_0^{\,l_i} p_z\,\textcolor{red}{\delta w_i}\,dx  + \underbrace{[N_i\,\textcolor{red}{\delta u_i}]_0^{l_i} + [V_i\,\textcolor{red}{\delta w_i} - M_i\,\textcolor{red}{\delta w_i'}]_0^{l_i}}_{Randarbeiten}
\end{align}
bei. Das sind also die virtuellen \"{a}u{\ss}eren Arbeiten der Streckenlasten $p_x$ (l\"{a}ngs) und $p_z$ (quer) zwischen den Knoten, und die Randarbeiten, die die Balken\-endkr\"{a}fte, $ N_i, V_i$ und $M_i$ auf den zu ihnen konjugierten virtuellen Verr\"{u}ckungen leisten.

Wenn in den Knoten des Rahmens keine Kr\"{a}fte oder Momente angreifen, dann sind die Anschlusskr\"{a}fte der Balken in den Knoten unter sich im Gleichgewicht. Was als Normalkraft $N$ ankommt, wird als Querkraft $V$ weitergeleitet, etc. Ferner sind die Verformungen und auch die virtuellen Verr\"{u}ckungen der Balken in den Knoten alle gleich gro{\ss}.

Aus dem Gleichgewicht an den Knoten und dem Gleichklang der
virtuellen Verr\"{u}ckungen folgt, dass die Summe der Randarbeiten, also die Summe \"{u}ber die eckigen Klammern in jedem Knoten null sind, und damit reduziert sich die Bilanz auf
\begin{align}\label{Eq32}
\delta A_a &= \sum_i\,[\int_0^{\,l_i} \,p_z\,\textcolor{red}{\delta w_i}\,dx + \int_0^{\,l_i} \,p_x\, \textcolor{red}{\delta u_i}\,dx]\nn \\
 &= \sum_i\, [\int_0^{\,l_i} \frac{N_i\,\textcolor{red}{\delta N_i}}{EA_i}\,dx + \int_0^{\,l_i} \frac{M_i\,\textcolor{red}{\delta M_i}}{EI_i}\,dx ] = \delta A_i\,.
\end{align}
Wenn Punktlasten in den Knoten angreifen, dann springen die beteiligten Balken\-endkr\"{a}fte um die H\"{o}he dieser Punktlasten, d.h. die Summe \"{u}ber die Randarbeiten (die eckigen Klammern) ergibt dann in dem Knoten einen  Beitrag wie $P\cdot\textcolor{red}{\delta w(x)}$.
%----------------------------------------------------------------------------------------------------------
\begin{figure}[tbp]
\centering
\if \bild 2 \sidecaption \fi
\includegraphics[width=0.7\textwidth]{\Fpath/U62}
\caption{Einzelkr\"{a}fte erfordern eine Zweiteilung des Feldes} \label{U62}
\end{figure}%
%----------------------------------------------------------------------------------------------------------



Den Ausdruck (\ref{Eq32}) kann man nun weiter vereinfachen, indem man auf das Anschreiben der Integrationsgrenzen verzichtet und ebenso die Indices an $u_i$ und $w_i$ und $EA_i$ und $EI_i$ etc. wegl\"{a}sst, denn jeder wei{\ss} ja, welcher Teil des Rahmens gerade gemeint ist. Man schreibt also einfacher
\begin{align}
\delta A_a = \int \,p_z\,\textcolor{red}{\delta w}\,dx + \int \,p_x\, \textcolor{red}{\delta u}\,dx
\end{align}
und analog
\begin{align}
\int \frac{N\,\textcolor{red}{\delta N}}{EA} \,dx + \int \frac{M\,\textcolor{red}{\delta M}}{EI}\,dx = \delta A_i\,,
\end{align}
so dass aus den vielen Identit\"{a}ten am Ende schlie{\ss}lich der bequeme Ausdruck
\begin{align}
\delta A_a = \int \,p_z\,\textcolor{red}{\delta w}\,dx + \int \,p_x\, \textcolor{red}{\delta u}\,dx = \int \frac{N\,\textcolor{red}{\delta N}}{EA} \,dx + \int \frac{M\,\textcolor{red}{\delta M}}{EI}\,dx = \delta A_i
\end{align}
wird.

%%%%%%%%%%%%%%%%%%%%%%%%%%%%%%%%%%%%%%%%%%%%%%%%%%%%%%%%%%%%%%%%%%%%%%%%%%%%%%%%%%%%%%%%%%%%%%%%%%%
{\textcolor{sectionTitleBlue}{\section{Einzelkr\"{a}fte und Einzelmomente}}}\index{Einzelkr\"{a}fte und Einzelmomente}

Es ist noch zu kl\"{a}ren, wie die Einzelkr\"{a}fte und Einzelmomente in die Arbeitsgleichung hineinkommen, also Terme wie $P\cdot\textcolor{red}{\delta w(x)}$.

Diese Terme r\"{u}hren von den eckigen Klammern, den Randarbeiten, denn Einzelkr\"{a}fte und Einzelmomente auf freier Strecke machen eine Zwei\-teilung der Biegelinie in $w_L(x)$ und $w_R(x)$ notwendig, weil man, anschaulich gesagt, nicht einfach \"{u}ber eine Einzelkraft hinweg integrieren kann. Der Rand entsteht dort, wo die beiden H\"{a}lften zusammensto{\ss}en.


Man integriert vom linken Lager bis zum Fu{\ss}punkt $\bar{x}$ der Kraft, stoppt dort, und setzt hinter dem Lastangriffspunkt die Integration fort
\begin{align}
\text{\normalfont\calligra G\,\,}(w, \textcolor{red}{\delta w}) = \text{\normalfont\calligra G\,\,}(w_L, \textcolor{red}{\delta w})_{(0,\bar{x})}+  \text{\normalfont\calligra G\,\,}(w_R, \textcolor{red}{\delta w})_{(\bar{x},l)} = 0 + 0 = 0\,.
\end{align}
Die beiden Teile der Biegelinie, $w_L(x)$ und $w_R(x)$, sind jeweils homogene L\"{o}sungen der Balkengleichung, weil wir hier der Einfachheit halber annehmen d\"{u}rfen, dass keine Streckenlasten vorhanden sind
\begin{align}
EI\,w_L(x) = 0 \qquad 0 < x < \bar{x} \qquad EI\,w_R(x) = 0 \qquad \bar{x}  < x < l\,,
\end{align}
und an der Stelle $\bar{x}$ gehen die beiden L\"{o}sungen stetig ineinander \"{u}ber, bis auf die Querkr\"{a}fte $V_L$ und $V_R$, die um den Wert der Einzelkraft springen, s. Abb. \ref{U62},
\begin{align}
M_R(\bar{x}) - M_L(\bar{x}) = 0 \qquad V_L(\bar{x})  - V_R(\bar{x})  = P\,.
\end{align}
Bei der Addition der Randarbeiten, also der eckigen Klammern an der \"{U}bergangssstelle, bleibt allein die virtuelle Arbeit der Einzelkraft \"{u}brig
%----------------------------------------------------------------------------------------------------------
\begin{figure}[tbp]
\centering
\if \bild 2 \sidecaption \fi
\includegraphics[width=0.9\textwidth]{\Fpath/U204}
\caption{Der Tr\"{a}ger muss in f\"{u}nf Integrationsintervalle unterteilt werden} \label{U204}
\end{figure}%
%----------------------------------------------------------------------------------------------------------
\begin{align}
[V_L\,\textcolor{red}{\delta w} - M_L\,\textcolor{red}{\delta w'}]_0^{\bar{x}} + [V_R\,\textcolor{red}{\delta w} - M_R\,\textcolor{red}{\delta w'}]_{\bar{x}}^l = P\cdot\textcolor{red}{\delta w(\bar{x})}
\end{align}
und somit lautet die Bilanz bei einer zul\"{a}ssigen virtuellen Verr\"{u}ckung
\begin{align}
\delta A_a = P\cdot\textcolor{red}{\delta w(\bar{x})} = \int_0^{\,l} \frac{M\,\textcolor{red}{\delta M}}{EI}\,dx = \delta A_i\,.
\end{align}
Mit Einzelmomenten verf\"{a}hrt man sinngem\"{a}{\ss}.

Gegebenenfalls muss man, s. Abb. \ref{U204}, die Integration mehrmals unterbrechen
\begin{align}
\text{\normalfont\calligra G\,\,}(w,\textcolor{red}{\delta w}) &:= \!\!\!\text{\normalfont\calligra G\,\,}(w,\textcolor{red}{\delta w})_{(x_1,x_2)} + \!\!\!\text{\normalfont\calligra G\,\,}(w,\textcolor{red}{\delta w})_{(x_2, x_3)} + \ldots + \!\!\!\text{\normalfont\calligra G\,\,}(w,\textcolor{red}{\delta w})_{(x_5,x_6)} \nn \\
&= 0 + 0 \ldots + 0 = 0\,.
\end{align}
All dies gilt nat\"{u}rlich auch f\"{u}r Lagerkr\"{a}fte, die ja auch Punktkr\"{a}fte sind. Damit sie in der Bilanz auftauchen, muss man allerdings virtuelle Verr\"{u}ckungen w\"{a}hlen, die offiziell nicht zul\"{a}ssig sind, die die \glq Ruhepflicht\grq{}, die Festhaltung der Lager, ignorieren, was mathematisch ja vollkommen legitim ist.

Ist $\textcolor{red}{\delta w} $ eine solche virtuelle Verr\"{u}ckung des Durchlauftr\"{a}gers in Abb. \ref{U204}, die auch die Lager verschiebt, dann stehen in der ersten Greensche Identit\"{a}t des Gesamtsystems auch die Arbeiten der Lagerkr\"{a}fte
\begin{align}
\text{\normalfont\calligra G\,\,}(w,\textcolor{red}{\delta w}) &=  M_A\,\textcolor{red}{\delta w'(x_1)}+ A\,\textcolor{red}{\delta w(x_1)} + P\,\textcolor{red}{\delta w(x_2)} + M\,\textcolor{red}{\delta w'(x_3)} + B\,\textcolor{red}{\delta w(x_4)}\nn \\
&+ \int_{x_5}^{\,x_6} p\,\textcolor{red}{\delta w}\,dx + C\,\textcolor{red}{\delta w(x_6)} - \int_0^{\,l} \frac{M\,\textcolor{red}{\delta M }}{EI}\,dx = 0\,.
\end{align}
Wir k\"{o}nnen gleich den umgekehrten Schluss ziehen:\\

\hspace*{-12pt}\colorbox{highlightBlue}{\parbox{0.98\textwidth}{Wenn man nur mit zul\"{a}ssigen virtuellen Verr\"{u}ckungen $\textcolor{red}{\delta w }$ an einem Trag\-werk \glq wackelt\grq{}, dann sind die Randarbeiten in den Lagern  null.}}\\

%----------------------------------------------------------------------------------------------------------
\begin{figure}[tbp]
\centering
\if \bild 2 \sidecaption \fi
\includegraphics[width=0.6\textwidth]{\Fpath/U205}
\caption{Lagersenkung und Lagerverdrehung} \label{U205}
\end{figure}%
%----------------------------------------------------------------------------------------------------------

%%%%%%%%%%%%%%%%%%%%%%%%%%%%%%%%%%%%%%%%%%%%%%%%%%%%%%%%%%%%%%%%%%%%%%%%%%%%%%%%%%%%%%%%%%%%%%%
{\textcolor{sectionTitleBlue}{\section{Lagersenkung}}}\index{Lagersenkung}
Im Zusammenhang mit einer Lagersenkung interessieren uns drei Themen:\\

\begin{itemize}
  \item Der Energieerhaltungssatz
  \item Das Prinzip der virtuellen Verr\"{u}ckungen
  \item Die Anwendung des Prinzips der virtuellen Kr\"{a}fte zur Berechnung von Verformungen
\end{itemize}

Das rechte Lager des Tr\"{a}gers in Abb. \ref{U205} senkt sich um ein Ma{\ss} $w_{\Delta}$. Die Biegelinie des Tr\"{a}gers
\begin{align}\label{Eq36}
EI\,w^{IV} = 0 \qquad w(0) = w'(0) = 0 \qquad M(l) = 0\quad w(l) = w_{\Delta}\,,
\end{align}
besteht aus zwei Funktionen, einer Biegelinie $w_1(x)$ mit den korrekten Randwerten
\begin{align}
w_1(0) = w_1'(0) = 0 \qquad w_1(l) = w_{\Delta}
\end{align}
und einer zweiten Biegelinie $w_2(x)$, die die (eventuellen) Fehler von $w_1$, dass n\"{a}mlich $EI\,w_1^{IV}$ nicht null ist und $M_1(l) \neq 0$, korrigiert, d.h.
\begin{align}
EI\,w_2^{IV}(x) = - EI\,w_1^{IV}(x) \quad w_2(0) = w_2'(0) = w_2(l) = 0 \quad M_1(l) + M_2(l) = 0\,,
\end{align}
so dass die Summe $w(x) = w_1(x) + w_2(x)$ den Gleichungen (\ref{Eq36}) gen\"{u}gt.

Mit finiten Elementen setzt man  $w_1(x) = w_{\Delta}\cdot \Np_3(x)$, bringt die Spalte $\vek f_3$ von $\vek K$ auf die rechte Seite, $\vek K\vek u = - w_{\Delta}\,\vek f_3$ und bestimmt aus diesen $n-1$ Gleichungen (die Zeile $3$ wird gestrichen), die \"{u}brigen $u_i$.

\vspace{-0.5cm}
{\textcolor{sectionTitleBlue}{\subsubsection*{Energieerhaltungssatz}}}

Zur Formulierung des Energieerhaltungssatzes gehen wir auf die Diagonale und \"{u}berlagern $w$ mit sich selbst
\begin{align}
\text{\normalfont\calligra G\,\,}(w,w) &= \int_0^{\,l} EI\,w^{IV}\,w\,dx + [V\,w - M\,w']_{@0}^{@l} - \int_0^{\,l} \frac{M^2}{EI}\,dx \nn \\
&= V(l) \cdot w_{\Delta} - \int_0^{\,l} \frac{M^2}{EI}\,dx = 0\,,
\end{align}
was nach Multiplikation mit $1/2 $
\begin{align}
\frac{1}{2}\, \text{\normalfont\calligra G\,\,}(w,w) = \frac{1}{2}\,  V(l)\cdot w_{\Delta} - \frac{1}{2}\,\int_0^{\,l} \frac{M^2}{EI}\,dx = 0\,,
\end{align}
der Energieerhaltungssatz ist.

{\textcolor{sectionTitleBlue}{\subsubsection*{Prinzip der virtuellen Verr\"{u}ckungen}}}

Nun gehen wir auf die Nebendiagonale, $\textcolor{red}{\delta w}$ sei eine zul\"{a}ssige virtuelle Verr\"{u}ckung, also $\textcolor{red}{\delta w(0) = \delta w'(0) = \delta w(l) = 0}$, und weil auch die Streckenlast null ist, $EI\,w^{IV} = 0$, ist $\delta A_a = 0$, und somit muss auch $\delta A_i = 0$ sein
\begin{align}\label{Eq120}
\text{\normalfont\calligra G\,\,}(w, \textcolor{red}{\delta w}) &= - \int_0^{\,l} \frac{M \textcolor{red}{\delta M}}{EI}\,dx =  - \delta A_i = 0\,.
\end{align}
%----------------------------------------------------------------------------------------------------------
\begin{figure}[tbp]
\centering
\if \bild 2 \sidecaption \fi
\includegraphics[width=0.7\textwidth]{\Fpath/U63}
\caption{Virtuelle Verr\"{u}ckung, $\delta A_a = \delta A_i = 0$} \label{U63}
\end{figure}%
%----------------------------------------------------------------------------------------------------------
Das mag \"{u}berraschen, aber man versteht es, wenn man an die Mohrsche Arbeitsgleichung denkt: Wir berechnen in (\ref{Eq120}) mit Hilfe der Einzelkraft $V(l)$ im rechten Lager, um wieviel die virtuelle Verr\"{u}ckung dort nach unten geht, aber $\textcolor{red}{\delta w(l) = 0}$. Bei Lagersenkungen orientieren sich die virtuellen Verr\"{u}ckungen $\textcolor{red}{\delta w}$ am urspr\"{u}nglichen System, sind also in den (urspr\"{u}nglich) festen Lagern null.


{\textcolor{sectionTitleBlue}{\subsubsection*{Prinzip der virtuellen Kr\"{a}fte}}}

Jetzt vertauschen wir die Pl\"{a}tze von $w $ und $\textcolor{red}{\delta w}$, das wir nun $\textcolor{red}{\delta w^*}$ nennen, wir formulieren also das {\em Prinzip der virtuellen Kr\"{a}fte\/}
\begin{align}
\text{\normalfont\calligra G\,\,}(\textcolor{red}{\delta w^*}, w) &= \delta A_a^* - \delta A_i^* = 0\,,
\end{align}
und wir berechnen mit diesem Prinzip beispielhaft die Durchbiegung in Balkenmitte, s. Abb. \ref{U205}. Traditionsgem\"{a}{\ss} hei{\ss}t die Biegelinie $\textcolor{red}{\delta w^*}$ bei Mohr $\bar{w}$.

Wir lassen also eine Einzelkraft $\textcolor{red}{\bar{P} = 1}$ in Richtung der gesuchten Verschiebung wirken und formulieren mit den beiden Teilen der Biegelinie
\begin{align}
\textcolor{red}{\bar{w} = \bar{w}_L + \bar{w}_R }
\end{align}
und $w$ die erste Greensche Identit\"{a}t und erhalten so, wir \"{u}berspringen die Zwischenschritte, das Ergebnis
\begin{align}
\text{\normalfont\calligra G\,\,}(\textcolor{red}{\bar{w}_L},w)_{(0,0.5\,l)}  &+ \text{\normalfont\calligra G\,\,}(\textcolor{red}{\bar{w}_R},w)_{(0.5\,l, l)} \nn \\
&= \textcolor{red}{\bar{1}} \cdot w(0.5\,l) + \textcolor{red}{\bar{V}(l)} \cdot w_{\Delta} - \int_0^{\,l} \frac{\textcolor{red}{\bar{M}}\,M}{EI}\,dx = 0\,,
\end{align}
oder aufgel\"{o}st nach der gesuchten Durchbiegung
\begin{align}
w(0.5\,l) = \int_0^{\,l} \frac{\textcolor{red}{\bar{M}}\,M}{EI}\,dx - \textcolor{red}{\bar{V}(l)} \cdot w_{\Delta}\,.
\end{align}

\hspace*{-12pt}\colorbox{highlightBlue}{\parbox{0.98\textwidth}{Bei einer Lagersenkung ist also die Mohrsche Arbeitsgleichung um den Beitrag $ -\textcolor{red}{\bar{V}(l)} \cdot w_{\Delta} $ zu erweitern, wobei $ \textcolor{red}{\bar{V}(l)}$ die Lagerkraft aus $\bar{P} = 1 $ ist. Der Beitrag ist negativ, weil er eigentlich auf die linke Seite geh\"{o}rt, zu den virtuellen \"{a}u{\ss}eren Arbeiten.}}\\

Wenn sich die Einspannung um einen Winkel $\Np_\Delta $ verdreht,
\begin{align}
EI\,w^{IV} = 0 \qquad w'(0) = \tan \Np_\Delta \qquad w(0) = w(l) = M(l) = 0\,,
\end{align}
dann erh\"{a}lt man auf analoge Weise
\begin{align}
\textcolor{red}{\bar{1}}\cdot w(0.5\,l) = - \textcolor{red}{\bar{M}}(0) \cdot \tan \Np_\Delta + \int_0^{\,l} \frac{\textcolor{red}{\bar{M}}  M}{EI}\,dx\,.
\end{align}
Das Moment $\textcolor{red}{\bar{M}(0)} $ ist das Einspannmoment aus der Einzelkraft $\textcolor{red}{\bar{P} = \bar{1}}$. Eigentlich geh\"{o}rt es auf die linke Seite, weil es virtuelle \"{a}u{\ss}ere Arbeit ist, und so taucht es rechts mit dem Faktor $(-1) $ auf.
%----------------------------------------------------------------------------------------------------------
\begin{figure}[tbp]
\centering
\if \bild 2 \sidecaption \fi
\includegraphics[width=0.5\textwidth]{\Fpath/U55}
\caption{Schraubenfeder} \label{U55}
\end{figure}%
%----------------------------------------------------------------------------------------------------------


%%%%%%%%%%%%%%%%%%%%%%%%%%%%%%%%%%%%%%%%%%%%%%%%%%%%%%%%%%%%%%%%%%%%%%%%%%%%%%%%%%%%%%%%%%%%%%%
{\textcolor{sectionTitleBlue}{\section{Federn}}}\index{Federn}
In matrizieller Schreibweise lautet das Federgesetz, s. Abb. \ref{U55},
\begin{align}
\left[ \barr {r @{\hspace{4mm}}r @{\hspace{4mm}}r
@{\hspace{4mm}}r @{\hspace{4mm}}r}
      k & -k  \\
      -k & k \\
     \earr \right]\left [\barr{c}  u_1 \\  u_2\earr \right ]
=  \left [\barr{c}  f_1 \\  f_2\earr \right ]
\end{align}
oder, k\"{u}rzer, $\vek K\,\vek u = \vek f$.

Zu diesem System geh\"{o}rt die Identit\"{a}t
\begin{align}
\text{\normalfont\calligra G\,\,}(\vek u, \textcolor{red}{\vek  \delta \vek  u}) = \textcolor{red}{\vek \delta \vek u^T}\,\vek K\,\vek u - \vek u^T\,\vek K\,\textcolor{red}{\vek \delta \,\vek  u} = 0\,.
\end{align}
Ist $\vek u $ die Gleichgewichtslage der Feder, $\vek K\,\vek  u = \vek f$, dann ergibt sich daraus das {\em Prinzip der virtuellen Verr\"{u}ckungen\/} f\"{u}r die Feder
\begin{align}
\text{\normalfont\calligra G\,\,}(\vek u, \textcolor{red}{\vek  \delta \vek  u}) = \textcolor{red}{\vek \delta \vek u^T}\,\vek f - \vek u^T\,\vek K\,\textcolor{red}{\vek \delta \,\vek  u} =\delta A_a - \delta A_i = 0
\end{align}
und analog das {\em Prinzip der virtuellen Kr\"{a}fte\/}
\begin{align}
\text{\normalfont\calligra G\,\,}(\textcolor{red}{\vek  \delta \vek  u^*},\vek  u) = \vek u^T \textcolor{red}{\vek f^{*}}- \textcolor{red}{\vek u^{*T}}\,\vek K\,\vek  u = \delta A_a^* - \delta A_i^* = 0\,.
\end{align}
%----------------------------------------------------------------------------------------------------------
\begin{figure}[tbp]
\centering
\if \bild 2 \sidecaption \fi
\includegraphics[width=0.9\textwidth]{\Fpath/U57}
\caption{Temperaturverformungen} \label{U57}
\end{figure}%
%----------------------------------------------------------------------------------------------------------

%%%%%%%%%%%%%%%%%%%%%%%%%%%%%%%%%%%%%%%%%%%%%%%%%%%%%%%%%%%%%%%%%%%%%%%%%%%%%%%%%%%%%%%%%%%%%%%
{\textcolor{sectionTitleBlue}{\section{Temperatur}}}\index{Temperatur}
In der linearen Statik darf man die Ergebnisse superponieren und so kann man den Lastfall Temperatur wie einen zus\"{a}tzlichen Lastfall behandeln
\begin{align}
w(x) = w_{LF\,1} + w_{LF\,2} + \ldots + w_{T}\,.
\end{align}
Wir d\"{u}rfen immer annehmen, dass $w_T$ am statisch bestimmten Tragwerk berechnet wird, also die Form
\begin{align}
w_T(x) = \alpha_T \frac{\Delta T}{h}\,x^2 + a\,x + b\,, \qquad (a, b\,\, \text{sind Konstante})
\end{align}
hat, weil eventuell n\"{o}tige Korrekturen in den vorangehenden Lastf\"{a}llen behandelt werden.

Das {\em Prinzip der virtuellen Kr\"{a}fte\/} f\"{u}r eine Biegelinie $w_T$, wir setzen eine Punktlast $P^* = 1$ in den Aufpunkt $x$, lautet dann
\begin{align}
\text{\normalfont\calligra G\,\,}(\textcolor{red}{\delta w^*}, w_T) = \textcolor{red}{1} \cdot w_T(x) - \int_0^{\,l}\textcolor{red}{ EI\,\delta w^*{''}}\,w_T''\,dx = 0
\end{align}
oder mit $w_T'' = \alpha_T \Delta T/h$
\begin{align}
w_T(x) = \int_0^{\,l}\textcolor{red}{ \delta M^*}\,\alpha_T \frac{\Delta T}{h}\,dx \,.
\end{align}
Hierbei ist $\alpha_T \sim 10^{-5}$ (Stahl, Beton) der Temperaturkoeffizient des Materials, $\Delta T$ ist die Temperaturdifferenz zwischen Ober- und Unterkante des Tr\"{a}gers und $h$ ist die Tr\"{a}gerh\"{o}he, s. Abb. \ref{U57}.

Genauso leitet man die Formel f\"{u}r die L\"{a}ngsverschiebung aus Temperatur ab
\begin{align}
u_T(x) = \int_0^{\,l} \textcolor{red}{\delta N^*}\,\alpha_T\,T\,dx\,,
\end{align}
wobei $T$ die \"{A}nderung gegen\"{u}ber der Ausgangstemperatur ist.

%%%%%%%%%%%%%%%%%%%%%%%%%%%%%%%%%%%%%%%%%%%%%%%%%%%%%%%%%%%%%%%%%%%%%%%%%%%%%%%%%%%%%%%%%%%%%%%
{\textcolor{sectionTitleBlue}{\section{Die vollst\"{a}ndige Arbeitsgleichung}}}\index{vollst\"{a}ndige Arbeitsgleichung}
Wir haben nun alle Teile zusammen, um die vollst\"{a}ndige Arbeitsgleichung, die {\em Mohrsche Arbeitsgleichung\/},\index{Mohrsche Arbeitsgleichung} zu formulieren
  \begin{align}\label{Eq101}
    \textcolor{red}{\bar{1}} \cdot \delta
    = &
    \int\frac{\textcolor{red}{\bar{M}}\,M}{EI}\,dx
    + \int\frac{\textcolor{red}{\bar{N}}\,N}{EA}\,dx + \int \textcolor{red}{\bar{M}}\,\alpha_T\,\frac{\Delta T}{h}\,dx + \int \textcolor{red}{\bar{N}}\,\alpha_T\,T\,dx \nn \\
     &+ \underbrace{\sum_i \frac{\textcolor{red}{\bar{F}_i}\,F_i}{k_{i}}}_{\text{Normalkraftfedern}}
    + \underbrace{\sum_j \frac{\textcolor{red}{\bar{M}_j}\,M_j}{k_{\varphi j}}}_{\text{Biegemomentenfedern}}\nn \\
    &
%    + \underbrace{\int N_i\,\alpha_T\,T\;}_{\text{Gleichm. Erw\"{a}rmung}}
%    + \underbrace{\int M_i\,\alpha_T\,\frac{\Delta T}{h}\;}_{\text{Ungleichm. Erw\"{a}rmung}}
    - \underbrace{\sum_k \textcolor{red}{\bar{F}_{k}}\,w_{\Delta\,k}}_{\text{Lagerverschiebungen}}
    - \underbrace{\sum_l \textcolor{red}{\bar{M}_{l}}\,\tan \varphi_{\Delta\,l}}_{\text{Lagerverdrehungen}}\,.
  \end{align}
Das $\delta$ auf der linken Seite steht, wie es in der Statik-Literatur Tradition ist, sowohl f\"{u}r Verschiebungen als auch Verdrehungen.

Wenn es eine Verdrehung ist, dann ist es der Tangens des Drehwinkels, weil in der ersten Greenschen Identit\"{a}t -- auf der die Arbeitsgleichung ja beruht -- das Moment mit dem Tangens gepaart ist
\begin{align}
\ldots + [V\,w - M\,w'] + \ldots
\end{align}
und nicht mit dem Drehwinkel.

Nur so wird die Arbeitsgleichung auch ihrem Namen gerecht, stehen links wie rechts wirklich Arbeiten
\begin{align}
\bar{M} \cdot  \delta = 1 \,\text{kNm}\cdot \tan\,\Np = [F \cdot L] \cdot [\,\,\,  ] = \int_0^{\,l} \ldots
\end{align}

%%%%%%%%%%%%%%%%%%%%%%%%%%%%%%%%%%%%%%%%%%%%%%%%%%%%%%%%%%%%%%%%%%%%%%%%%%%%%%%%%%%%%%%%%%%%%%%
{\textcolor{sectionTitleBlue}{\section{Kurzform}}}\index{Kurzform}
Es ist nun sicherlich m\"{u}hsam, f\"{u}r ein gegebenes System die Bilanz
\begin{align}
\delta A_a = \delta A_i
\end{align}
aus den Greenschen Identit\"{a}ten der einzelnen Tragglieder zu entwickeln. Das macht kein Ingenieur so, sondern der Ingenieur wei{\ss} mit ein wenig \"{U}bung automatisch, welche Beitr\"{a}ge er $\delta A_a$ zuschlagen muss. Das sind die Arbeiten der Streckenlasten
\begin{align}
\int_0^{\,l} p_z\,\textcolor{red}{\delta w}\,dx \qquad \int_0^{\,l} p_z\,\textcolor{red}{\delta u}\,dx
\end{align}
und die Arbeiten der Punktlasten
\begin{align}
P_z\,\textcolor{red}{\delta w(x)}  \qquad P_x\,\textcolor{red}{\delta u(x)} \qquad M\,\textcolor{red}{\delta w'(x)} \qquad \text{etc.}
\end{align}
und die Beitr\"{a}ge zu $\delta A_i$ sind auch bekannt
  \begin{align}
   \delta A_i = &
    \int \frac{M\,\textcolor{red}{\delta M}}{EI}\,dx  + \int \frac{N\,\textcolor{red}{\delta N}}{EA}\,dx
    +  \int \textcolor{red}{\delta M}\,\alpha_T\,\frac{\Delta T}{h}\,dx + \int \textcolor{red}{\delta N}\,\alpha_T\,T\,dx \nn \\
     &+ \sum_i \frac{\textcolor{red}{\delta F_i}\,F_i}{k_{i}}
    + \sum_j \frac{\textcolor{red}{\delta M_j}\,M_j}{k_{\varphi j}} - \sum_k \textcolor{red}{\delta F_k}\,w_{\Delta\,k}
    - \sum_l  \textcolor{red}{\delta M_l}\,\tan \varphi_{\Delta\,l}\,,
\end{align}
und sie verk\"{u}rzen sich meist auf
\begin{align}
\delta A_i = \int_0^{\,l} \frac{M\,\textcolor{red}{\delta M}}{EI}\,dx  + \int_0^{\,l} \frac{N\,\textcolor{red}{\delta N}}{EA}\,dx
\end{align}
oder oft noch einfacher auf
\begin{align}
\delta A_i = \int_0^{\,l} \frac{M\,\textcolor{red}{\delta M}}{EI}\,dx\,.
\end{align}

%%%%%%%%%%%%%%%%%%%%%%%%%%%%%%%%%%%%%%%%%%%%%%%%%%%%%%%%%%%%%%%%%%%%%%%%%%%%%%%%%%%%%%%%%%%%%%%
{\textcolor{sectionTitleBlue}{\section{Dualit\"{a}t}}}\index{Dualit\"{a}t}
Die Arbeits- und Energieprinzipe des Balkens entwickeln sich spielerisch aus dem Arbeitsintegral
\begin{align}
\int_0^{\,l} EI\,w^{IV}(x) \,\textcolor{red}{\delta w(x)}\,dx\,,
\end{align}
also der \"{U}berlagerung, dem $L_2$-Skalarprodukt, von Kraft und Weg. Wie dies geschieht, ist in der ersten Greenschen Identit\"{a}t detailliert dargelegt. Und was f\"{u}r den Balken gilt, gilt f\"{u}r die anderen Bauteile ebenso.\\

\hspace*{-12pt}\colorbox{highlightBlue}{\parbox{0.98\textwidth}{ Kraft und Weg sind die beiden Pole, um die sich die Statik dreht. Der Arbeitsbegriff ist der zentrale Begriff der Statik und die fundamentale Rechenoperation der Statik ist das Skalarprodukt}}\\

Die Kunst im Umgang mit der ersten Greenschen Identit\"{a}t besteht nun einfach darin, die virtuelle Verr\"{u}ckung so zu w\"{a}hlen, dass man an die Information kommt, die man sucht. Drei Techniken stehen zur Verf\"{u}gung:
\begin{align}
&\text{{\em Prinzip der virtuellen Verr\"{u}ckungen\/}} \quad &\!\!\!\!\text{\normalfont\calligra G\,\,}(w, \textcolor{red}{\delta w}) = 0 \quad &\text{Kr\"{a}fte}\nn \\
&\text{{\em Prinzip der virtuellen Kr\"{a}fte\/}} \quad &\!\!\!\!\text{\normalfont\calligra G\,\,}(\textcolor{red}{\delta w^*}, w) = 0 \quad &\text{Wege}\nn \\
&\text{{\em Satz von Betti\/}} \quad &\!\!\!\!\text{\normalfont\calligra B\,\,}(w_1, w_2) = 0\quad &\text{Wege und Kr\"{a}fte}\nn
\end{align}
Mit dem {\em Satz von Betti\/} kann man Weg- und Kraftgr\"{o}{\ss}en berechnen. Mit dem {\em Prinzip der virtuellen Kr\"{a}fte\/} Weggr\"{o}{\ss}en und mit dem {\em Prinzip der virtuellen Verr\"{u}ckungen\/} Kraftgr\"{o}{\ss}en, \"{u}blicherweise sind das Lagerkr\"{a}fte.

%----------------------------------------------------------------------------------------------------------
\begin{figure}[tbp]
\centering
\if \bild 2 \sidecaption \fi
\includegraphics[width=1.0\textwidth]{\Fpath/U148B}
\caption{Einflussfunktionen bei einem Einfeldtr\"{a}ger. Man l\"{o}st ein Lager oder baut ein Gelenk ein, zeichnet das System noch einmal an und verr\"{u}ckt das rechte System. Es gilt der Satz von Betti $A_{1,2} = A_{2,1}$. Die Summe der Arbeiten der Null-Kr\"{a}fte rechts auf den (nicht gezeichneten) Wegen links ist null, $A_{2,1} = 0$, und daher sind auch die Arbeiten der Kr\"{a}fte links auf den Wegen rechts null, $A_{1,2} = 0$. Zur Illustration wurden die Kr\"{a}fte links rechts angetragen, um ihre Wege zu verfolgen, aber die rechten Systeme selbst sind kr\"{a}ftefrei, weil sie kinematisch sind und sich daher die Wege rechts ohne Kraftaufwand erzeugen lassen} \label{U148}
\end{figure}%
%----------------------------------------------------------------------------------------------------------
Im {\em Prinzip der virtuellen Verr\"{u}ckungen\/} formuliert man\footnote{Die Funktionen $\delta w $ und  $\delta w^* $ sind nat\"{u}rlich genauso real wie $w $. Es sind einfach geschickt gew\"{a}hlte Hilfsfunktionen, nur hat man sie mit dem Attribut \glq virtuell\grq{} versehen.}
\begin{align}\label{Eq61}
\text{\normalfont\calligra G\,\,}(w, \textcolor{red}{\delta w}) = \text{Reale Kr\"{a}fte} \,\times\, \textcolor{red}{\text{virtuelle Wege}} - a(w,\textcolor{red}{\delta w}) = 0
\end{align}
und im {\em Prinzip der virtuellen Kr\"{a}fte\/} dagegen
\begin{align}\label{Eq62}
\text{\normalfont\calligra G\,\,}(\textcolor{red}{\delta w^*}, w) = \textcolor{red}{\text{Virtuelle Kr\"{a}fte}} \,\times \,\text{reale Wege} - a(\textcolor{red}{\delta w^*},w) = 0\,.
\end{align}
%----------------------------------------------------------------------------------------------------------
\begin{figure}[tbp]
\centering
%\if \bild 2 \sidecaption \fi
\includegraphics[width=1.0\textwidth]{\Fpath/U149A}
\caption{Einflussfunktion f\"{u}r Moment in einem Sparren {\bf a)} \glq alles in einem\grq{} Zeichnung {\bf b)} die Bewegung nach dem L\"{o}sen des Gelenks als separates Bild, $\tan \Np_l + \tan \Np_r = 1$ am Gelenk} \label{U149}
\end{figure}%
%--------------------------------------------------------------------------------------------------------
Will man zum Beispiel die Lagerkraft $A$ an dem Einfeldtr\"{a}ger in Abb. \ref{U148} berechnen, so kann man eine Drehung um das rechte Lager, $\textcolor{red}{\delta w(x) = 1 - x/l}$, als virtuelle Verr\"{u}ckung w\"{a}hlen
\begin{align}
\text{\normalfont\calligra G\,\,}(w, \textcolor{red}{\delta w}) = \int_0^{\,l} p(x)\,\textcolor{red}{\delta w(x)}\,dx - V(0) \,\textcolor{red}{\delta w(0)} = 0\,,
\end{align}
und die Identit\"{a}t dann nach $A = V(0)$ aufl\"{o}sen
\begin{align}
A\cdot \textcolor{red}{1} = \int_0^{\,l} p\cdot\textcolor{red}{(1 - \frac{x}{l})}\,dx\,.
\end{align}
Der Vollst\"{a}ndigkeit halber wollen wir auch zeigen, wie man mit dem {\em Prinzip der virtuellen Verr\"{u}ckungen\/} Einflussfunktionen f\"{u}r Kraftgr\"{o}{\ss}en an statisch bestimmten Tragwerken berechnen kann, obwohl das Vorgehen eigentlich mit dem Satz von Betti identisch ist\footnote{Das sieht man besser, wenn man zwei Zeichnungen macht, denn Betti ist ja eigentlich \glq zwei\grq{}; einmal das Originalsystem (mit dem eingebauten Gelenk, im Gleichgewicht gehalten durch die zuvor inneren Momente $M_l$ und $M_r$ am Gelenk) und dann die Verr\"{u}ckung des Originalsystems mit dem gel\"{o}sten Gelenk. Die Verr\"{u}ckung ist gerade die Einflussfunktion. }.\\

\hspace*{-12pt}\colorbox{highlightBlue}{\parbox{0.98\textwidth}{Die Anwendung des Prinzips der virtuellen Verr\"{u}ckungen zur Berechnung von Kraftgr\"{o}{\ss}en an statisch bestimmten Tragwerken ist eigentlich eine Anwendung des Satzes von Betti.}}\\

Weil das Tragwerk nach Einbau des entsprechenden Gelenkes kinematisch ist, ist die virtuelle innere Arbeit $\delta A_i = 0$ und somit muss auch $\delta A_a = 0$ sein
\begin{align}
\text{\normalfont\calligra G\,\,}(w,\textcolor{red}{\delta w}) = \delta A_a - 0 = 0\,,
\end{align}
w\"{a}hrend beim Satz von Betti
\begin{align}
\text{\normalfont\calligra B\,\,}(w,\textcolor{red}{\delta w}) =  A_{1,2} - A_{2,1} = 0
\end{align}
die Arbeit $A_{2,1}$ null ist, s. S. \pageref{ImmerSo}, und daher muss auch $A_{1,2}$ null sein. Mathematisch sind aber $\delta A_a$ und $A_{1,2}$ bei den folgenden Beispielen gleich, nur werden sie anders benannt, und daher ist kein Unterschied zwischen den beiden Verfahren an dieser Stelle.

Um die Einflussfunktion f\"{u}r das Moment in Balkenmitte zu bestimmen,  bauen wir ein Gelenk in Balkenmitte ein, und bringen das zuvor innere Moment $M$ auf beiden Seiten des Gelenks als \"{a}u{\ss}eres Momentenpaar auf, damit der Balken weiterhin die Wanderlast abtragen kann, s. Abb. \ref{U148} {\em c\/} und {\em d\/}.

Dann erteilen wir dem Balken eine virtuelle Verr\"{u}ckung $\textcolor{red}{\delta w(x)}$ derart, dass sich die Balkenmitte um einen noch n\"{a}her zu bestimmenden Wert  $\textcolor{red}{\Delta\,w}$ absenkt. Der Stabdrehwinkel der linken H\"{a}lfte ist dabei $\textcolor{red}{\delta w_L'} = \textcolor{red}{\Delta w/(l/2)}$ und der rechten H\"{a}lfte ist $\textcolor{red}{\delta w_R'} = -\textcolor{red}{\Delta w/(l/2)}$. Gem\"{a}{\ss} dem {\em Prinzip der virtuellen Verr\"{u}ckungen\/} gilt
\begin{align}
\text{\normalfont\calligra G\,\,}(w, \textcolor{red}{\delta w}) = \delta A_a - \delta A_i = 0\,.
\end{align}
Nun ist die virtuelle innere Arbeit bei dieser Bewegung null (wie bei einer Marionette) und daher  muss auch die Summe der virtuellen \"{a}u{\ss}eren Arbeiten null sein
\begin{align}
\delta A_a = 1 \cdot \textcolor{red}{\delta w(x)} - M\cdot \textcolor{red}{\delta w_L'} + M \cdot \textcolor{red}{\delta w_R'}= 0\,.
\end{align}
Wenn man nun $\textcolor{red}{\Delta w}$ so gro{\ss} w\"{a}hlt, dass
\begin{align}
\textcolor{red}{\delta w_L'} - \textcolor{red}{\delta w_R'} = 1
\end{align}
ist, also $\textcolor{red}{\Delta w} = l/4$, dann folgt
\begin{align}
M(x) = 1 \cdot \textcolor{red}{\delta w(x)}\,.
\end{align}
Zur Bestimmung der Einflussfunktion f\"{u}r die Querkraft in Balkenmitte bauen wir in die Mitte des Balkens ein Querkraftgelenk ein, s. Abb. \ref{U148} {\em e\/}, und korrigieren diesen Verlust an Querkrafttragf\"{a}higkeit dadurch, dass wir links und rechts von dem Gelenk die zuvor innere Kraft $V$ als \"{a}u{\ss}ere St\"{u}tzkraft wirken lassen. Dann spreizen wir dieses Gelenk so, dass die beiden Seiten des Gelenks jeweils um $\pm 0.5$ m nach oben bzw. nach unten ausweichen und berechnen die virtuelle \"{a}u{\ss}ere Arbeit, die dabei von der Wanderlast und den beiden Kr\"{a}ften $V$ geleistet wird
\begin{align}
\delta A_a = 1 \cdot\textcolor{red}{\delta w(x)} - V(x) \cdot 0.5 - V \cdot 0.5 = 1 \cdot \textcolor{red}{\delta w(x)} - V(x) \cdot 1 = 0
\end{align}
oder
\begin{align}
V(x) = \textcolor{red}{\delta w(x)}\,.
\end{align}
Die Logik l\"{a}sst sich auch auf Wandermomente anwenden, wenn also ein Moment \"{u}ber den Tr\"{a}ger l\"{a}uft.
Ein Wandermoment $M = 1$ leistet Arbeit, wenn man es verdreht. Das Ma{\ss} f\"{u}r diese Arbeit ist das Produkt aus dem Moment und dem Tangens des Drehwinkels, also
\begin{align}
M \cdot\textcolor{red}{\delta w'}
\end{align}
und daher sind die Einflussfunktionen f\"{u}r $A(x), M(x)$ und $V(x)$ in diesem Fall identisch mit den Ableitungen von $\textcolor{red}{\delta w(x)}$.
%----------------------------------------------------------------------------------------------------------
\begin{figure}[tbp]
\centering
\if \bild 2 \sidecaption \fi
\includegraphics[width=1.0\textwidth]{\Fpath/U372}
\caption{Durchlauftr\"{a}ger und das Prinzip der virtuellen Verr\"{u}ckungen  \textbf{ a)} Momente  \textbf{b)} virtuelle Verr\"{u}ckung nach Einbau von Zwischengelenken } \label{U372}
\end{figure}%
%----------------------------------------------------------------------------------------------------------
Sind Streckenlasten vorhanden, dann geschieht die Auswertung der Einflussfunktionen durch Integration
\begin{align}
A(x) = \int_a^{\,b} p(x)\,\textcolor{red}{\delta w(x)}\,dx\,.
\end{align}
Bei schr\"{a}gen St\"{a}ben, wie dem Sparren in Abb. \ref{U149}, muss man darauf achten, dass nur der Anteil von $\textcolor{red}{\delta w(x)}$, der in Richtung der Wanderlast f\"{a}llt, gez\"{a}hlt wird.

%%%%%%%%%%%%%%%%%%%%%%%%%%%%%%%%%%%%%%%%%%%%%%%%%%%%%%%%%%%%%%%%%%%%%%%%%%%%%%%%%%%%%%%%%%%%%%%
{\textcolor{sectionTitleBlue}{\section{Ganze Tragwerke}}}
Der Formulierung des Prinzips der virtuellen Verr\"{u}ckungen an ganzen Tragwerken ist ein Summieren der einzelnen Identit\"{a}ten, l\"{a}ngs ($u_i$) und quer ($w_i$),
\begin{align}
\sum_i \text{\normalfont\calligra G\,\,}(u_i,\textcolor{red}{\delta u_i}) + \sum_i\ \text{\normalfont\calligra G\,\,}(w_i,\textcolor{red}{\delta w_i}) = 0 + 0 + \ldots + 0 = 0\,.
\end{align}
So k\"{o}nnte man z.B. in den Durchlauftr\"{a}ger in Abb. \ref{U372} drei Gelenke einbauen und diese Gelenkkette so auslenken, dass die Einzelkraft den Weg \textcolor{red}{Eins} geht und man h\"{a}tte das Resultat ($\delta A_i = 0$ weil die $\textcolor{red}{\delta w_i}$ Starrk\"{o}rperbewegungen sind)
\begin{align}\label{Eq73}
\sum_i\ \text{\normalfont\calligra G\,\,}(w_i,\textcolor{red}{\delta w_i}) &= \delta A_a + \delta A_i = \sum_{k = 1}^3 M_k \cdot \textcolor{red}{\Delta \Np_k} + P \cdot \textcolor{red}{1} = 0 \\
 \textcolor{red}{\Delta \Np_k} &= \textcolor{red}{\tan \Np_k^r - \tan \Np_k^l}\,,
\end{align}
was einem wahrscheinlich nicht viel weiterhilft, weil man drei Schnittmomente $M_k$ in einer Gleichung hat.
%----------------------------------------------------------------------------------------------------------
\begin{figure}[tbp]
\centering
\if \bild 2 \sidecaption \fi
\includegraphics[width=1.0\textwidth]{\Fpath/U384MitGitter}
\caption{Dreigelenkrahmen, Bestimmung des Moments $M_i$ mittels eines Polplans \textbf{ a)} System und Belastung  \textbf{b)} virtuelle Verr\"{u}ckung nach Einbau des Gelenks  \textbf{c)} vertikaler Anteil der Einflussfunktion im Lastgurt} \label{U384}
\end{figure}%
%----------------------------------------------------------------------------------------------------------

In einer anderen Situation k\"{o}nnte gleichwohl die Gleichung (\ref{Eq73}) als Kontrolle dienen. Setzt man die Feldl\"{a}nge $l = 4$ und $P = 10$, dann ergibt sich mit
$\textcolor{red}{\Delta \Np_1 = 0.5, \Delta \Np_2 = -0.5, \Delta \Np_3 = 0.5}$ und $M_1 = -1.31, M_2 = 3.98, M_3 = -14.71$ tats\"{a}chlich das richtige Ergebnis, n\"{a}mlich null
\begin{align}
\delta A_a = \sum_{k = 1}^3 M_k\,\textcolor{red}{\Delta \Np_k} + P \cdot \textcolor{red}{1 }= \textcolor{red}{0.5} \cdot (- 1.31 - 3.98 - 14.71) + 10 \cdot \textcolor{red}{1} = 0\,.
\end{align}
Wichtig ist, dass die Integrationsgrenzen $[\ldots]$ zum einen die Lager und Gelenke und zum anderen m\"{o}gliche Einzelkr\"{a}fte und Einzelmomente respektieren m\"{u}ssen. Mit jedem Einbau von zus\"{a}tzlichen Gelenken in ein Tragwerk \"{a}ndert sich die Zahl der Abschnitte, \"{u}ber die die Identit\"{a}ten integrieren. Bei dem Tr\"{a}ger in Abb. \ref{U372} sind es erst vier Identit\"{a}ten, entsprechend den vier Feldern, und danach sind es sieben.
%----------------------------------------------------------------------------------------------------------
\begin{figure}[tbp]
\centering
\if \bild 2 \sidecaption \fi
\includegraphics[width=1.0\textwidth]{\Fpath/U188A}
\caption{Mohr und der Satz von Betti, es ist einfacher $\bar{M}$ zu bestimmen, als die Biegelinie $G_0(y,x)$} \label{U188}
\end{figure}%
%----------------------------------------------------------------------------------------------------------

Der erfahrene Ingenieur geht nat\"{u}rlich nicht \"{u}ber die Identit\"{a}ten, sondern er wei{\ss} automatisch,  was er mitzunehmen hat, und was nicht. Das macht die Identit\"{a}ten aber nicht \"{u}berfl\"{u}ssig, denn sie legen ja eigentlich erst fest, was zu z\"{a}hlen ist, und was nicht, wie die innere Energie aussieht und die \"{a}u{\ss}ere Arbeit und wie sich Starrk\"{o}rperbewegungen \"{u}ber ein Tragwerk fortpflanzen -- n\"{a}mlich als Pseudodrehungen -- und die Identit\"{a}ten garantieren schlie{\ss}lich erst das Endresultat $\delta A_a - \delta A_i = 0$.

Das Prinzip der virtuellen Verr\"{u}ckungen ist immer dann gut anwendbar, wenn das Tragwerk statisch bestimmt ist, weil man dann durch das geschickte Wegnehmen nur eines Lagers oder den Einbau eines Gelenks ein kinematisches System erh\"{a}lt, an dem man mit geeigneten Starrk\"{o}rperbewegungen eine unbekannte Lagerkraft oder ein inneres Moment bestimmen kann.

Bei Rahmen, wie in Abb. \ref{U384}, muss man sich allerdings mit Hilfe von Polpl\"{a}nen Klarheit \"{u}ber die Wege verschaffen, die die Kr\"{a}fte gehen. Wenn ein Programm m\"{o}gliche Beweglichkeiten in einem Rahmen anzeigt, weil man ein Lager vergessen hat, dann kann man das Programm dazu benutzen, die Starrk\"{o}rperbewegungen darzustellen, die nach dem Einbau eines Gelenks oder der Entfernung eines Lagers m\"{o}glich sind.

%----------------------------------------------------------------------------------------------------------
\begin{figure}[tbp]
\centering
\if \bild 2 \sidecaption \fi
\includegraphics[width=0.7\textwidth]{\Fpath/U183}
\caption{Berechnung der Horizontalverschiebung eines Knotens mit Mohr und mit dem Satz von Betti, \textbf{ a)} Momente aus Last, \textbf{ b)} Momente aus $\bar{P} = 1$, \textbf{ c)} Verschiebung aus $\bar{P} = 1$ (= Einflussfunktion f\"{u}r die Horizontalverschiebung)} \label{U183}
\end{figure}%
%----------------------------------------------------------------------------------------------------------

%%%%%%%%%%%%%%%%%%%%%%%%%%%%%%%%%%%%%%%%%%%%%%%%%%%%%%%%%%%%%%%%%%%%%%%%%%%%%%%%%%%%%%%%%%%%%%%
{\textcolor{sectionTitleBlue}{\section{Mohr contra Betti}}}
Man kann also die Durchbiegung eines Balkens mit der Mohrschen Arbeitsgleichung (dem {\em Prinzip der virtuellen Kr\"{a}fte\/}) berechnen
\begin{align}\label{Eq132}
w(x) = \int_0^{\,l} \frac{M(y)\,\bar{M}(y,x)}{EI} \,dy \qquad \text{\normalfont\calligra G\,\,}(G_0, w) = 0
\end{align}
oder mit dem {\em Satz von Betti\/}
\begin{align}
w(x) = \int_0^{\,l} G_0(y,x)\,p(y)\,dy \qquad \text{\normalfont\calligra B\,\,}(G_0, w) = 0\,.
\end{align}
Die letztere Gleichung benutzt aber kein Ingenieur, weil er dazu erst die Biegelinie $G_0(y,x) $ bestimmen m\"{u}sste, die die Einzelkraft $\bar{P} = 1 $ an dem Tr\"{a}ger erzeugt, s. Abb. \ref{U188}. Die Berechnung der Momente $\bar{M} = - EI\,G_0'' $ f\"{a}llt dem Ingenieur dagegen viel leichter, und das ist der Grund, warum Verformungen an Tragwerken mit der Mohrschen Arbeitsgleichung berechnet werden und nicht mit dem Satz von Betti.

Allerdings muss man bei Mohr mehr tun, um zum Ergebnis zu kommen, wie man in Abb. \ref{U183} sieht, denn man muss die Momente $M$ und $\bar{M}$ \"{u}ber den ganzen Rahmen integrieren, w\"{a}hrend sich dasselbe Ergebnis nach dem Satz von Betti durch eine Auswertung in einem Punkt ergibt.

Mit Blick auf die finiten Elemente scheint es so zu sein, dass Mohr die genaueren Ergebnisse liefert, weil sich die mittleren Fehler in $M_h$ und $\bar{M}_h$ (den FE-N\"{a}herungen) besser ausgleichen, w\"{a}hrend Betti ja genau den richtigen Wert $G_0(y,x)$ am Ort $y$ von $P$ treffen muss. {\em Aber Mohr und Betti sind zwei Seiten einer Medaille!\/} Wenn man mit finiten Elementen rechnet, dann sind die Ergebnisse gleich genau (oder gleich ungenau), weil man Mohr mittels partieller Integration in Betti umformen kann und umgekehrt.

\begin{remark}
In der Statikliteratur schreibt man f\"{u}r das Integral (\ref{Eq132}) k\"{u}rzer
\begin{align}
\delta = \int_0^{\,l} \frac{M\,\bar{M}}{EI}\,dx\,,
\end{align}
taucht der Aufpunkt $x$ nicht auf, und deswegen kann dann die Integrationsvariable $x$ hei{\ss}en. Auch wir haben diese kurze Form bei der Arbeitsgleichung benutzt und werden sie gelegentlich weiter benutzen.
\end{remark}

%----------------------------------------------------------------------------------------------------------
\begin{figure}[tbp]
\centering
\if \bild 2 \sidecaption \fi
\includegraphics[width=0.9\textwidth]{\Fpath/U156}
\caption{Der Satz von Betti, System 1 der reale Balken, System 2 derselbe Balken, frei schwebend, ohne Belastung, ohne Lager, sich frei um sein rechtes Ende drehend } \label{U156}
\end{figure}%
%----------------------------------------------------------------------------------------------------------

%%%%%%%%%%%%%%%%%%%%%%%%%%%%%%%%%%%%%%%%%%%%%%%%%%%%%%%%%%%%%%%%%%%%%%%%%%%%%%%%%%%%%%%%%%%%%%%
{\textcolor{sectionTitleBlue}{\section{Schwache und starke Einflussfunktionen}}}
Die Gleichung (die Mohrsche Arbeitsgleichung)
\begin{align}\label{Eq81}
w(x) = \int_0^{\,l} \frac{M(y)\,\bar{M}(y,x)}{EI} \,dy
\end{align}
nennen wir eine {\em schwache Einflussfunktion\/}\index{schwache Einflussfunktion} und die Gleichung
\begin{align}
w(x) = \int_0^{\,l} G_0(y,x)\,p(y)\,dy
\end{align}
eine {\em starke Einflussfunktion\/}\index{starke Einflussfunktion}. Man kann also $w(x)$ aus den Momenten, der zweiten Ableitung von $w$, berechnen oder aus der Streckenlast, der vierten Ableitung von $w$.
%----------------------------------------------------------------------------------------------------------
\begin{figure}[tbp]
\centering
\if \bild 2 \sidecaption \fi
\includegraphics[width=0.9\textwidth]{\Fpath/U279}
\caption{Mohr und Kraftgr\"{o}{\ss}en, \textbf{ a)} Einflussfunktion f\"{u}r $V(x)$, \textbf{ b)} am statisch bestimmten Tr\"{a}ger, \textbf{ c)} Korrektur mit $X_1$, \textbf{ d) } Moment aus $X_1 = 1$ } \label{U279}
\end{figure}%
%----------------------------------------------------------------------------------------------------------

Schwache Einflussfunktionen basieren auf der ersten Greenschen Identit\"{a}t, in der Formulierung als {\em Prinzip der virtuellen Kr\"{a}fte\/},
\begin{align}
\text{\normalfont\calligra G\,\,}(G,w) = 0
\end{align}
und starke Einflussfunktionen auf der zweiten Greenschen Identit\"{a}t, dem {\em Satz von Betti\/}
\begin{align}
\text{\normalfont\calligra B\,\,}(G,w) = 0\,.
\end{align}
Das Standardbeispiel f\"{u}r eine schwache Einflussfunktion ist die Mohrsche Arbeitsgleichung (\ref{Eq81}).
%----------------------------------------------------------------------------------------------------------
\begin{figure}[tbp]
\centering
\if \bild 2 \sidecaption \fi
\includegraphics[width=0.8\textwidth]{\Fpath/U280}
\caption{Einflussfunktionen beim Seil f\"{u}r \textbf{ a)} die Durchbiegung und \textbf{ b)} die Querkraft} \label{U280}
\end{figure}%
%----------------------------------------------------------------------------------------------------------

Mit der Mohrschen Arbeitsgleichung kann man aber keine Kraftgr\"{o}{\ss}en wie etwa die Querkraft
\begin{align}\label{Eq113}
V(x)  \overset{?}{=} \int_0^{\,l} \frac{M\,\bar{M}}{EI}\,dy
\end{align}
berechnen. Was klar scheint, denn welche virtuelle Kraft will man anwenden, um die Querkraft $V(x)$ in einem Punkt zu berechnen? Das {\em Prinzip der virtuellen Kr\"{a}fte\/} taugt also nur zur Berechnung von Weggr\"{o}{\ss}en.

Auch rechnerisch ist das Ergebnis null, wie das folgende Beispiel zeigen soll. Betti kann mit der Einflussfunktion f\"{u}r $V(x)$, s. Abb. \ref{U279} a, die Querkraft in Balkenmitte exakt voraussagen
\begin{align}
V(x) = \int_0^{\,l} G_3(y,x)\,p(y)\,dy\,.
\end{align}
Probiert man dasselbe mit Mohr, dann muss man gem\"{a}{\ss} (\ref{Eq113}) das Moment $\bar{M}$ der Einflussfunktion mit dem Moment $M$ aus der Belastung \"{u}berlagern. Die Einflussfunktion $G_3(y,x)$ setzt sich aus zwei Teilen zusammen, der Scherbewegung in Abb.  \ref{U279} b, und einem Zusatzterm, der den Fehler in der Einspannung korrigiert. Entsprechend hat $\bar{M}$ die Gestalt
\begin{align}
\bar{M} = X_1 \cdot M_1 + 0\,,
\end{align}
wenn wir das Moment der st\"{u}ckweise linearen Scherbewegung null setzen, was nur im Aufpunkt etwas problematisch ist\footnote{Setzt man $\bar{M} = X_1 \cdot M_1 + \delta_1$ kann man die Situation retten, denn es ergibt sich   (wir d\"{u}rfen $EI = 1$ setzen) $(M,\bar{M}) = X_1 \cdot (M,M_1) + (M,\delta_1) = X_1 \cdot 0 + V(x) = V(x)$}. Die \"{U}berlagerung ergibt jedoch null und nicht $V(x)$
\begin{align}
V(x) \overset{?}{=} \int_0^{\,l} \frac{M\,\bar{M}}{EI}\,dy = X_1 \int_0^{\,l} \frac{M\,M_1}{EI}\,dy = 0\,,
\end{align}
weil die \"{U}berlagerung von $M$ und $M_1$ die Kontrolle ist ({\em Reduktionssatz\/}),\index{Reduktionssatz} ob die Verdrehung des Balkens in der Einspannfuge null ist -- was sie ist.

Man kann sich das auch analytisch zurechtlegen. Am einfachsten geht das an einem Seil mit den beiden Einflussfunktionen $G_0(y,x)$ f\"{u}r die Durchbiegung und $G_1(y,x)$ f\"{u}r die Querkraft $V(x) = H\,w'(x)$ in Seilmitte, s. Abb. \ref{U280}.

Formal geschieht die Herleitung so, dass man die Strecke $(0,l)$ in der Mitte zweiteilt, dabei einen Spalt $(x-\varepsilon,x+\varepsilon)$ l\"{a}sst, die Identit\"{a}t
\begin{align}
\text{\normalfont\calligra G\,\,}(w,\hat{w}) = \int_0^{\,l} - H\,w''\,\hat{w}\,dx + [V\,\hat{w}]_{@0}^{@l} - a(w,\hat{w}) = 0
\end{align}
f\"{u}r jeden Teil getrennt formuliert und dann den Spalt gegen null gehen l\"{a}sst
\begin{align}
\text{\normalfont\calligra G\,\,}(G_0,w) = \lim_{\varepsilon \to 0}\,\{\text{\normalfont\calligra G\,\,}(G_0,w)_{(0,x-\varepsilon)} + \text{\normalfont\calligra G\,\,}(G_0,w)_{(x+\varepsilon,l)}\} = 0 + 0\,.
\end{align}
Wir erhalten so
\begin{align}
\text{\normalfont\calligra G\,\,}(G_0,w) = \int_0^{\,l} 0 \cdot w\,dx + 1 \cdot w(x) - a(G_0,w) = 0 \,,
\end{align}
was die Gleichung von Mohr (am Seil) ist
\begin{align}
w(x) = a(G_0,w) = \int_0^{\,l} \frac{V_0\,V}{H}\,dy\,,
\end{align}
w\"{a}hrend aus der Formulierung
\begin{align}
\text{\normalfont\calligra G\,\,}(G_1,w) &= \int_0^{\,l} 0 \cdot w\,dx + (V_1(x_{-}) - V_1(x_{+})) \cdot w(x) - a(G_1,w) \nn \\
&= 0 + 0 \cdot w(x) - a(G_1,u) = 0
\end{align}
folgt, dass $a(G_1,w) = 0$ ist, was sich auch aus
\begin{align}
\text{\normalfont\calligra G\,\,}(w,G_1) = \underbrace{\int_0^{\,l} p\,G_1\,dy - V(x) \cdot 1}_{=\, 0\,(Betti)} - a(w,G_1) = 0
\end{align}
ergibt.

\begin{remark}
Um den Unterschied zwischen den beiden Typen von Einflussfunktionen deutlich zu machen, nehmen wir an, dass die Belastung nur aus einer einzelnen Kraft $P$ besteht.

Im {\em Satz von Betti\/}
\begin{align}
w(x) = \int_0^{\,l} G_0(y,x)\,p(y)\,dy = G_0(y,x) \cdot P
\end{align}
lassen wir -- anschaulich gesprochen -- im Aufpunkt $x$ einen Stein (ein Dirac Delta) in das Wasser fallen, und wir beobachten, wie sich die Welle $G_0(y,x)$ \"{u}ber den Balken ausbreitet, um wieviel die Punktlast in der Ferne von der Welle gehoben wird.

Bei dem {\em Prinzip der virtuellen Kr\"{a}fte (Mohr)\/}
\begin{align}
w(x) = \int_0^{\,l} \frac{M(y)\,M^*(y,x)}{EI}\,dy
\end{align}
beobachten wir dagegen die Interaktion von zwei Wellen. Die erste Welle, $M(y)$, ist das Biegemoment, das von der Punktlast erzeugt wird, und die zweite Welle $M^*(y,x)$ ist das Biegemoment, das von dem Dirac Delta erzeugt wird. Nur die Teile des Balkens, wo beide Momente gro{\ss} sind (und nicht orthogonal zueinander), sind relevant. \\
\end{remark}

%----------------------------------------------------------------------------------------------------------
\begin{figure}[tbp]
\centering
\if \bild 2 \sidecaption \fi
\includegraphics[width=0.7\textwidth]{\Fpath/U172}
\caption{Die kanonischen Randwerte von Stab und Balken} \label{U172}
\end{figure}%
%----------------------------------------------------------------------------------------------------------


%%%%%%%%%%%%%%%%%%%%%%%%%%%%%%%%%%%%%%%%%%%%%%%%%%%%%%%%%%%%%%%%%%%%%%%%%%%%%%%%%%%%%%%%%%%%%%%
{\textcolor{sectionTitleBlue}{\section{Die kanonischen Randwerte}}}\index{kanonische Randwerte}

Die erste Greensche Identit\"{a}t besteht aus lauter Skalarprodukten konjugierter Gr\"{o}{\ss}en. Dabei kommt den Weg- und Kraftgr\"{o}{\ss}en
in den eckigen Klammern, den Randarbeiten,
\begin{align}
[N\,u]_{@0}^{@l} &= N(l)\,u(l) - N(0)\,u(0) = f_2\,u_2 + f_1\,u_1 \\
[V\,w - M\,w']_{@0}^{@l} &= V(l)\,w(l) - M(l)\,w'(l) - V(0)\,w(0) + M(0)\,w'(0)\nn \\
 &= f_3\,u_3 + f_4\,u_4 + f_1\,u_1 + f_2\,u_2\,,
\end{align}
eine spezielle Bedeutung zu. Sie kann man als die {\em kanonischen Werte\/} eines Stabes bzw. eines Balkens bezeichnen, s. Abb. \ref{U172}. Das kommt am besten in den Steifigkeitsmatrizen des Stabes
\begin{align}
\frac{EA}{l}\,\left[ \barr {r @{\hspace{4mm}}r }
      1 & -1  \\
      -1 & 1 \\
     \earr \right]\left [\barr{c}  u_1 \\  u_2\earr \right ]
=  \left [\barr{c}  f_1 \\  f_2 \earr \right ]
\end{align}
und des Balkens
\begin{align}
 \frac{EI}{l^3} \left[
\begin{array}{r r r r}
 12 & -6l & -12 &-6l \\
 -6l & 4l^2 & 6l &2l^2 \\
 -12 & 6l & 12 & 6l \\
 -6l &2l^2 &6l &4l^2
 \end{array}
  \right]\,\left [\barr{c}u_1 \\ u_2 \\ u_3 \\ u_4 \earr \right ] = \left [\barr{c}  f_1 \\ f_2 \\ f_3\\ f_4 \earr \right ]
\end{align}
zum Ausdruck.

Die Steifigkeitsmatrizen formulieren eine {\em Kopplung \/} zwischen den $2 + 2$ Weg- und Kraftgr\"{o}{\ss}en eines Stabes bzw. den $4 + 4$ Gr\"{o}{\ss}en eines Balkens. Sind Streckenlasten vorhanden, dann sind die Gleichungen um den Vektor $\vek d$ der \"{a}quivalenten Knotenkr\"{a}fte aus der {\em domain load\/} zu erweitern,
\begin{align}
\vek K\,\vek u = \vek f + \vek d\,,
\end{align}
der beim Stab zwei Komponenten und beim Balken vier Komponenten hat
\begin{align}\label{Eq87}
d_i &= \int_0^{\,l} p(x)\,\Np_i^e(x)\,dx \quad && i = 1,2 && \text{(Stab)}\\
d_i &= \int_0^{\,l} p(x)\,\Np_i^e(x)\,dx \quad \qquad && i = 1,2,3,4 &&\text{(Balken)}\,.
\end{align}
Die $\Np_i^e(x)$ sind die zwei bzw. vier Einheitsverformungen\index{Einheitsverformungen} des Stabes bzw. Balkens, s. Abb. \ref{U89} auf S. \pageref{U89}. Die $d_i$ sind die Kr\"{a}fte, mit denen die Lasten auf die eingespannten Balken/Stabenden dr\"{u}cken oder ziehen, um sie zum Nachgeben zu zwingen. Die {\em Festhaltekr\"{a}fte\/}\index{Festhaltekr\"{a}fte} versuchen dies zu verhindern. Sie haben daher das entgegengesetzte Vorzeichen\\

\hspace*{-12pt}\colorbox{highlightBlue}{\parbox{0.98\textwidth}{\"{A}quivalente Knotenkr\"{a}fte = Festhaltekr\"{a}fte $\times (-1)$ }}\\

Bei den finiten Elementen operiert man meist mit nur einem Vektor $\vek f \equiv \vek f + \vek d$, der beide Anteile enth\"{a}lt, also die Kr\"{a}fte $f_i$, die direkt in den Knoten angreifen, und die Kr\"{a}fte $d_i$ aus der verteilten Belastung. Wenn der Ingenieur Lasten in die Knoten reduziert, dann sind  das die $d_i$. {\em Eine Knotenkraft hei{\ss}t \"{a}quivalent, wenn sie bei einer Einheitsverformung des Knotens dieselbe Arbeit leistet, wie die Last im Feld\/}.\index{\"{a}quivalente Knotenkraft}

Auch das System $\vek K\,\vek u = \vek f + \vek d$ basiert auf der ersten Greenschen Identit\"{a}t. Um dies zu sehen, spalten wir die L\"{a}ngsverschiebung $u(x) = u_h(x) + u_p(x)$ eines Stabes in eine homogene L\"{o}sung
\begin{align}
u_h(x) = u_1\,\Np_1(x) + u_2\,\Np_2(x)
\end{align}
und eine partikul\"{a}re L\"{o}sung $u_p(x)$ auf
\begin{align}
- EA\,u_p''(x) = p(x) \qquad u_p(0) = u_p(l) = 0\,,
\end{align}
denn dann folgt
\begin{align}
\text{\normalfont\calligra G\,\,}(u_h + u_p,\Np_i^e) &= \int_0^{\,l} p\,\Np_i^e \,dx + [N\,\Np_i^e]_{@0}^{@l} - a(u_h + u_p,\Np_i^e) \nn \\
&= d_i + f_i - a(u_h,\Np_i^e) - \underbrace{a(u_p,\Np_i^e)}_{ = 0} = d_i + f_i - \sum_{j = 1}^2 k_{ij}\,u_j = 0\,,
\end{align}
wobei wir das Resultat
\begin{align}
\text{\normalfont\calligra G\,\,}(\Np_i^e,u_p) = \int_0^{\,l} 0 \cdot u_p\,dx - a(\Np_i^e, u_p) = a(\Np_i^e, u_p) = 0
\end{align}
benutzt haben. Dies folgt aus der Tatsache, dass die Randwerte von $u_p$ null sind.

In einem Balken ergibt dieselbe Aufspaltung
\begin{align}\label{Eq159}
\text{\normalfont\calligra G\,\,}(w_h + w_p,\Np_i^e) &= \int_0^{\,l} p\,\Np_i^e\,dx + [V\,\Np_i^e - M\,\Np_i^{e'}]_{@0}^{@l} - a(w_h,\Np_i^e) -  \underbrace{a(w_p,\Np_i^e)}_{ = 0}  \nn \\
&= d_i + f_i - \sum_{j = 1}^4 k_{ij}\,u_j = 0
\end{align}
wobei die $\Np_i^e(x)$ die Einheitsverformung der Balkenenden sind.

Ganz wesentlich ist es, dass man mit dem System $\vek K\,\vek u = \vek f + \vek d$ die Kontrolle \"{u}ber die Randwerte $u_i$ und $f_i$ hat.\\

\begin{itemize}
  \item Man kann immer nur einen Teil der $u_i$ und $f_i$ frei w\"{a}hlen, der andere Teil ist dann durch  $\vek K\,\vek u = \vek f + \vek d$ bestimmt.
  \item In jeder Gleichung darf es immer nur eine Unbekannte geben, wie etwa im Fall eines gelenkig gelagerten Tr\"{a}gers
  \begin{align}
 \frac{EI}{l^3} \left[
\begin{array}{r r r r}
 12 & -6l & -12 &-6l \\
 -6l & 4l^2 & 6l &2l^2 \\
 -12 & 6l & 12 & 6l \\
 -6l &2l^2 &6l &4l^2
 \end{array}
  \right]\,\left [\barr{c}  0 \\ u_2 \,?\\ 0 \\ u_4 \,?\earr \right ] = \left [\barr{c}  f_1 \,?\\ 0 \\ f_3\,? \\ 0 \earr \right ] + \left [\barr{c}  d_1 \\ d_2 \\ d_3 \\d_4 \earr \right ]\,.
\end{align}
  \item Ebenso kann man nicht zwei konjugierte Gr\"{o}{\ss}en {\em gleichzeitig \/} vorschreiben, also an einem Stab mit einer Kraft von 10 kN ziehen und gleichzeitig ver\-langen, dass die L\"{a}ngsverschiebung $u(l)$ dabei 1 cm betragen soll.
\end{itemize}

Die eigentlich wichtige Botschaft, auf die wir gleich eingehen, ist jedoch:

\begin{itemize}
  \item Wenn man die $u_i$ und $f_i$ am Balkenende kennt, dann kann man mit ihnen (zusammen mit der Belastung $p(x)$) die Verformungen und Schnittgr\"{o}{\ss}en in allen Punkten dazwischen berechnen.
\end{itemize}

Dieser Schluss ist so wichtig, dass wir ihn in einen gr\"{o}{\ss}eren Kontext stellen wollen.


%%%%%%%%%%%%%%%%%%%%%%%%%%%%%%%%%%%%%%%%%%%%%%%%%%%%%%%%%%%%%%%%%%%%%%%%%%%%%%%%%%%%%%%%%%%%%%%%%%%
{\textcolor{sectionTitleBlue}{\section{Die Reduktion der Dimension}}\index{Reduktion der Dimension}}

Beim Drehwinkelverfahren sprechen wir vom {\em Grad der kinematischen Unbestimmtheit\/} und meinen damit die Zahl der unbekannten Knotenverschiebungen und Knotenverdrehungen. Nachdem die Verformungen der Knoten berechnet wurden, nennen wir das Tragwerk kinematisch bestimmt, und wir k\"{o}nnen uns dann daran machen, aus den Knotenwerten die Verformungen und die Schnittgr\"{o}{\ss}en zwischen den Knoten zu berechnen.
%----------------------------------------------------------------------------------------------------------
\begin{figure}[tbp]
\centering
\if \bild 2 \sidecaption \fi
\includegraphics[width=0.99\textwidth]{\Fpath/U252}  %U552
\caption{Die Knoten sind die \glq R\"{a}nder\grq{} eines Rahmens. Wei{\ss} man, wie sich die Knoten verformen, dann kennt man auch die Verl\"{a}ufe von $N, M$ und $V$ zwischen den Knoten} \label{U252}
\end{figure}%
%----------------------------------------------------------------------------------------------------------

In der Stabstatik reicht es also offenbar aus, die Weg- und Kraftgr\"{o}{\ss}en auf dem \glq Rand\grq{} zu kennen -- in den Knoten, s. Abb. \ref{U252} -- denn nur so ist es m\"{o}glich, dass sich die Statik eines Rahmens auf zwei Vektoren, $\vek u$ und $\vek f$, die dem Gleichungssystem
\begin{align}
\vek K\,\vek u = \vek f_K + \vek d = \vek f
\end{align}
gen\"{u}gen, reduzieren l\"{a}sst. Das bedeutet:\\

\hspace*{-12pt}\colorbox{highlightBlue}{\parbox{0.98\textwidth}{{\em Endlich viele\/} Knotenwerte bestimmen unendlich viele Zahlen, die Verformungen $u(x), w(x), w'(x) $ und Schnittgr\"{o}{\ss}en  $ N(x), M(x), V(x)$ in den unendlich vielen Punkten der Stiele und Riegel.}}
\\

Das ist aber doch eine Reduktion um Eins. Die $n = 1$ dimensionalen Tragglieder schrumpfen auf eine $n - 1 = 0$ dimensionale Menge von Punkten, von Knoten, zusammen. {\em Erst diese Reduktion macht das  Drehwinkelverfahren\index{Drehwinkelverfahren} m\"{o}glich: Es reicht, sich mit den Knoten zu besch\"{a}ftigen!\/}
%----------------------------------------------------------------------------------------------------------
\begin{figure}[tbp]
\centering
\if \bild 2 \sidecaption \fi
\includegraphics[width=0.7\textwidth]{\Fpath/U253A}
\caption{Staumauer, auf der Wasser- und Luftseite kennt man den Spannungsvektor $\vek t = \vek S\,\vek n$ und im Fels den Verschiebungsvektor $\vek u = \vek 0$. Die fehlenden Werte, den Spannungsvektor $\vek t$ im Fels und den Verschiebungsvektor $\vek u$ des oberen Teils kann man durch L\"{o}sen einer Integralgleichung (\glq Knotenausgleich auf der Oberfl\"{a}che der Staumauer\grq{}) berechnen. Anschlie{\ss}end kann man aus den Randwerten alle interessierenden Werte im Innern der Staumauer berechnen} \label{U253}
\end{figure}%
%----------------------------------------------------------------------------------------------------------

Alle linearen, selbstadjungierten Differentialgleichungen gestatten eine solche Reduktion der Dimension eines Problems um Eins, $n \to (n-1)$. Der praktische Wert dieser Reduktion kann nicht hoch genug gesch\"{a}tzt werden.\\

\hspace*{-12pt}\colorbox{highlightBlue}{\parbox{0.98\textwidth}{Wenn man die Weg- und Kraftgr\"{o}{\ss}en auf dem Rande kennt, dann kann man die Verformungen und Schnittgr\"{o}{\ss}en im Innern mittels Einflussfunktionen aus den Randwerten berechnen.}}
\\

Zur Ermittlung der Spannungen in einer Staumauer ($n = 3$), s. Abb. \ref{U253}, reicht die Kenntnis der Verschiebungen und Spannungen auf der Oberfl\"{a}che der Staumauer ($n = 2$) aus. Um eine Platte ($n = 2$) zu berechnen, reicht die Kenntnis der Weg- und Schnittgr\"{o}{\ss}en l\"{a}ngs des Randes ($n = 1$) aus und bei einem Balken ($n = 1$) muss man nur die Knotenwerte kennen.

Die einfachste und elementarste Umsetzung dieser Idee ist das Lineal. Eine Gerade (die L\"{o}sung der Differentialgleichung $u'' = 0$) ist durch ihre beiden Randwerte eindeutig bestimmt und daher kann man die Gerade zeichnen, wenn man das Lineal an die Endpunkte anh\"{a}lt. {\em Das Lineal ist die universelle Einflussfunktion der Geraden\/}.

Au{\ss}enraumprobleme werden gerne durch solche Methoden gel\"{o}st. Allein durch das Diskretisieren der Oberfl\"{a}che eines Motorblocks kann man den L\"{a}rm -- den Schalldruck -- in 3 m, 30 m oder 300 m Entfernung berechnen.

%----------------------------------------------------------------------------------------------------------
\begin{figure}[tbp]
\centering
\if \bild 2 \sidecaption \fi
\includegraphics[width=1.0\textwidth]{\Fpath/U93}
\caption{Deckenplatte,  \textbf{ a)} die Knoten der Randelemente und  \textbf{ b)} die Hauptmomente im LF $g$; sie wurden mittels Randintegralen berechnet} \label{U93}
\end{figure}
%----------------------------------------------------------------------------------------------------------

%%%%%%%%%%%%%%%%%%%%%%%%%%%%%%%%%%%%%%%%%%%%%%%%%%%%%%%%%%%%%%%%%%%%%%%%%%%%%%%%%%%%%%%%%%%%%%%%%%%
{\textcolor{sectionTitleBlue}{\section{Methode der Randelemente}}}\index{Methode der Randelemente}
Die  Methode der Randelemente ist die Anwendung dieser Idee auf Fl\"{a}chentragwerke (Scheiben und Platten) oder ganze Volumina, wie Staumauern. Sie hat ihren Namen von den kurzen Geradenst\"{u}cken (Randelementen), in die der Rand der Platte oder Scheibe unterteilt wird. Eine Unterteilung des Innern wie bei den finiten Elementen ist nicht n\"{o}tig, so wie ja noch nie ein Ingenieur einen Knotenausgleich \glq im Feld\grq{} gef\"{u}hrt hat.
%----------------------------------------------------------------------------------------------------------
\begin{figure}[tbp]
\centering
\if \bild 2 \sidecaption \fi
\includegraphics[width=1.0\textwidth]{\Fpath/U94}
\caption{Wandscheibe,  \textbf{ a)} Knoten der Randelemente, \textbf{ b)} Lagerkr\"{a}fte, \textbf{ c)} Hauptspannungen. Alle Spannungen im Innern wurden durch Integration \"{u}ber den Rand berechnet.} \label{U94}
\end{figure}%
%----------------------------------------------------------------------------------------------------------

Man kann sich die Methode der Randelemente als eine Mischung aus dem Drehwinkelverfahren und Einflussfunktionen vorstellen. Der Rand der Platte oder Scheibe wird in Randelemente unterteilt, s. Abb. \ref{U93} und \ref{U94}, um die Randverformungen und Randkr\"{a}fte (= Funktionen) l\"{a}ngs des Randes mit Polygonz\"{u}gen darstellen zu k\"{o}nnen. Dann wird, wie beim Drehwinkelverfahren, ein Knotenausgleich in den Randknoten durchgef\"{u}hrt -- allerdings nicht iterativ, sondern in einem Schritt.

Anschlie{\ss}end werden darauf mit Hilfe von Einflussfunktionen aus den Verformungen der R\"{a}nder und den Lagerkr\"{a}ften die Schnittgr\"{o}{\ss}en im Innern der Platte oder Scheibe berechnet.

Im \"{u}brigen gilt: {\em Auch das Drehwinkelverfahren ist eine Randelementmethode\/} bei der \"{U}bertragungsmatrizen die Rolle der Einflussfunktionen \"{u}bernehmen, die Randwerte nach Innen fortsetzen.

Und was \"{u}berraschen mag: {\em Auch die Methode der finiten Elemente ist in einem gewissen Sinn eine
\glq Randelementmethode'\/}, denn die Kerne $G_h(\vek y,\vek x)$ in den Einflussfunktionen
\begin{align}\label{Eq171}
u_h(\vek x) = \int_{\Omega} G_h(\vek y,\vek x)\,p(\vek y)\,d\Omega_{\vek y}
\end{align}
werden in der FEM \glq unsichtbar\grq{} aus {\em Randintegralen\/} (denselben wie in der BEM) und {\em Gebietsintegralen\/} erzeugt, s. S. \pageref{SingInf}, (ohne dass die Anwender und wohl auch die meisten Programmautoren sich dessen bewusst sind)
\begin{align}
G_h(\vek y,\vek x) = \int_{\Gamma} \ldots ds_{\vek y} + \int_{\Omega} \ldots \,d\Omega_{\vek y}\,,
\end{align}
Die Randintegrale propagieren die Singularit\"{a}ten auf dem Rand ins Innere, machen, dass die FE-Einflussfunkti\-on (\ref{Eq171}) und damit die FE-L\"{o}sung -- {\bf im ganzen Gebiet} -- an Genauigkeit verliert!

%----------------------------------------------------------
\begin{figure}[tbp]
\centering
\if \bild 2 \sidecaption \fi
\includegraphics[width=0.9\textwidth]{\Fpath/U96}
\caption{Stab,  \textbf{ a)} System und Belastung,  \textbf{ b)} gen\"{a}herte Einflussfunktion -- ihr fehlt der Knick,  \textbf{ c)} Fundamentall\"{o}sung, sie hat den Knick an der richtigen Stelle, aber ihre Randwerte sind nicht null; die horizontalen Verschiebungen sind, um sie sichtbar zu machen, nach unten abgetragen}
\label{U96}
\end{figure}%%
%----------------------------------------------------------

%%%%%%%%%%%%%%%%%%%%%%%%%%%%%%%%%%%%%%%%%%%%%%%%%%%%%%%%%%%%%%%%%%%%%%%%%%%%%%%%%%%%%%%%%%%%%%%%%%%
{\textcolor{sectionTitleBlue}{\section{Finite Elemente und Randelemente}}}
Einflussfunktionen sind das wesentliche Werkzeug von finiten Elementen wie von Randelementen. Ein FE-Programm berechnet die Verschiebung in einem Stab -- ganz klassisch -- mit einer Einflussfunktion
\begin{align}
u_h(x) = \int_0^{\,l} G_h(y,x)\,p(y)\,dy\,,
\end{align}
nur dass die Einflussfunktion $G_h(y,x)$ eine N\"{a}herung ist, s. Abb. \ref{U96} b, weil sie den Knick unter der Einzelkraft nicht darstellen kann (zumindest, wenn der Aufpunkt $x$ zwischen den Knoten liegt).

Die Methode der Randelemente geht im Grunde genauso vor, aber sie benutzt eine sogenannte {\em Fundamentall\"{o}sung\/}\index{Fundamentall\"{o}sung} $g(y,x)$. Das ist eine Funktion, die zwar den richtigen Knick unter der Einzelkraft aufweist, die aber an den Enden des Stabes nicht null ist, die also die Lagerbedingungen verletzt.

Die Folge ist, dass bei der Formulierung des {\em Satzes von Betti\/}
\begin{align}
\lim_{\varepsilon \to 0}\text{\normalfont\calligra B\,\,}(g,u)_{\Omega_e} = 0
\end{align}
nun auch die Normalkr\"{a}fte $N(0)$ und $N(l)$ an den Endes des Stabes, s. Abb. \ref{U96} a, von den \glq nicht-null\grq{} Verschiebungen $g$ verschoben werden und zur Dirac Energie beitragen, die Einflussfunktion wird \glq l\"{a}nger\grq{}
\begin{align}\label{Eq84}
1 \cdot u(x) = \underbrace{\int_0^{\,l} g(y,x)\,p(y)\,dy + N(l)\,g(l,x) - N(0)\,g(0,x)}_{Dirac\,\, Energie}
\end{align}
verglichen mit der urspr\"{u}nglichen Formulierung mit $G(y,x)$
\begin{align}
1 \cdot u(x) = \underbrace{\int_0^{\,l} G(y,x)\,p(y)\,dy}_{Dirac\,\, Energie} \,.
\end{align}
Man beachte, dass beide Formeln denselben Wert f\"{u}r die Dirac Energie liefern. Aber der $g(y,x)$-Zugang muss auch die Arbeit an den Stabenden, den R\"{a}ndern, mitz\"{a}hlen, das Arbeitsintegral $(p,g)$ allein ist \glq zu wenig\grq{}. Das ist der Unterschied.

Der Vorteil von Fundamentall\"{o}sungen ist, dass sie ein {\em Universalschl\"{u}ssel\/} sind, der \"{u}berall passt. Mit ein und derselben Fundamentall\"{o}sung kann man alle Platten berechnen. In BE-Programmen sind die Fundamentall\"{o}sungen \glq fest verdrahtet\grq{}. {\em One solution suffices to rule them all\/}.

Der Nachteil ist, dass auch die Randkr\"{a}fte und eventuell auch die Randverformungen mit zur {\em Dirac Energie\/} beitragen, so dass diese Randwerte, wenn sie unbekannt sind, durch das L\"{o}sen eines linearen Gleichungssystems bestimmt werden m\"{u}ssen. Technisch bedeutet dies kein Problem. Es ist nur so, dass in zwei und drei Dimensionen diese Hilfsprobleme nur n\"{a}herungsweise gel\"{o}st werden k\"{o}nnen und so ist auch bei Randelementen -- wie bei den finiten Elementen -- die {\em Dirac Energie\/} nur eine N\"{a}herung.

Das Operieren mit Fundamentall\"{o}sungen hat den weiteren Vorteil, dass eine Approximation der Einflussfunktionen f\"{u}r die Schnittgr\"{o}{\ss}en, wie bei den finiten Elementen, nicht notwendig ist.
BE-Programme benutzen exakte Einflussfunktionen f\"{u}r {\em alle\/} Weg- und Schnittgr\"{o}{\ss}en, nur die Marken auf dem Rand, an die wir sozusagen die Kurvenlineale, die Einflussfunktionen, halten, sind leicht verrutscht.


%----------------------------------------------------------------------------------------------------------
\begin{figure}[tbp]
\centering
\if \bild 2 \sidecaption \fi
\includegraphics[width=0.4\textwidth]{\Fpath/U64}
\caption{Test eines Keilriemens} \label{U64}
\end{figure}%
%----------------------------------------------------------------------------------------------------------

%%%%%%%%%%%%%%%%%%%%%%%%%%%%%%%%%%%%%%%%%%%%%%%%%%%%%%%%%%%%%%%%%%%%%%%%%%%%%%%%%%%%%%%%%%%%%%%%%%%
{\textcolor{sectionTitleBlue}{\section{Testfunktionen}}\index{Testfunktionen}
Die Arbeits- und Energieprinzipe der Statik beruhen also auf der ersten Greenschen Identit\"{a}t
\begin{align}\label{Eq133}
\text{\normalfont\calligra G\,\,}(w,\textcolor{red}{\delta w}) = 0\,,
\end{align}
und deren \glq Spiegelung\grq{}, dem Satz von Betti. Die Schreibweise $\text{\normalfont\calligra G\,\,}(w,\textcolor{red}{\delta w})$ ist geeignet deutlich zu machen, dass nichts besonders geheimnisvolles an einer virtuellen Verr\"{u}ckung $\textcolor{red}{\delta w}$ ist. Mathematisch ist es einfach der Gegenpart zu $w$. Und wie $w$ ist $\textcolor{red}{\delta w}$ eine Funktion -- und mehr nicht!

Leider ist der Begriff der virtuellen Verr\"{u}ckungen jedoch historisch so belastet, dass man manchmal versucht ist, ihn durch einen harmloseren Begriff wie den der {\em Testfunktion\/} zu ersetzen. Der Begriff der virtuellen Verr\"{u}ckung, des Testens ist ja auch nicht auf die Statik beschr\"{a}nkt, sondern ganz fest im Alltag verankert und wird st\"{a}ndig angewandt.

Um das Gewicht eines Koffers zu bestimmen, heben wir den Koffer hoch. Gem\"{a}{\ss}
der Formel {\em Kraft = Masse $\times$ Beschleunigung\/} k\"{o}nnen wir aus der Beschleunigung $a$ und der Kraft im Arm, auf die Masse $M$ des Koffers schlie{\ss}en.

Um die Spannung in einem Keilriemen zu ermitteln, dr\"{u}cken wir mit dem Daumen dagegen, s. Abb. \ref{U64}. Bei einen Fu{\ss}ball reicht ebenfalls ein Daumendruck.

Dualit\"{a}t findet mathematisch ihren Ausdruck in der ersten Greenschen Identit\"{a}t. Diese Invariante gleicht einer nie versiegenden Quelle, aus der wir durch geschickte Wahl der Testfunktion $\textcolor{red}{\delta w} $ (fast) jede gew\"{u}nschte Information \"{u}ber $w$ ziehen k\"{o}nnen.

%----------------------------------------------------------------------------------------------------------
\begin{figure}[tbp]
\centering
\if \bild 2 \sidecaption \fi
\includegraphics[width=0.69\textwidth]{\Fpath/U535}
\caption{Gleichg\"{u}ltig, wie gro{\ss} $\delta w$ ist, es ist immer $\delta A_i = \delta A_a$} \label{U535}
\end{figure}%
%----------------------------------------------------------------------------------------------------------
%%%%%%%%%%%%%%%%%%%%%%%%%%%%%%%%%%%%%%%%%%%%%%%%%%%%%%%%%%%%%%%%%%%%%%%%%%%%%%%%%%%%%%%%%%%%%%%%%%%
{\textcolor{sectionTitleBlue}{\section{M\"{u}ssen virtuelle Verr\"{u}ckungen klein sein?}}}
Nein. Virtuelle Verr\"{u}ckungen m\"{u}ssen nicht klein sein, s. Abb. \ref{U535}. Die Gleichung
\begin{align} \label{Eq145}
\delta A_a = \int_0^{\,l} p\,\textcolor{red}{\delta w}\,dx = \int_0^{\,l} \frac{M\,\textcolor{red}{\delta M}}{EI} \,dx = \delta A_i,
\end{align}
ist nicht deswegen wahr, weil $\delta A_a = \delta A_i$ ist. {\em Labels\/} sind kein Beweis! Man kann nicht einen Term $A$ mit dem {\em label\/} $\delta A_a$ versehen und einen zweiten Term $B$ mit dem {\em label\/} $\delta A_i$ und dann behaupten, dass $A = B$ ist, weil ja doch $\delta A_a = \delta A_i$ ist. Wir d\"{u}rfen Mathematik nicht mit Mechanik verwechseln.

Die Energieprinzipe bilden den Kern der Mechanik und an keiner Stelle ist man der Mechanik so nahe, wie bei Formulierungen wie $\delta A_a = \delta A_i$. Es ist auch richtig, dass die statische Interpretation einer Gleichung -- definitiv -- die beste Kontrolle ist. Aber das verf\"{u}hrt Ingenieure dazu Mathematik mit Mechanik zu beweisen, was nicht geht -- Kontrolle ja, aber Beweis nein.

Um nicht missverstanden zu werden: Wir halten die Energieprinzipe der Mechanik f\"{u}r ein sehr gelungenes Konzept. Vom didaktischen Standpunkt aus gibt es kaum einen besseren Zugang zur Statik --  Generationen von Ingenieuren haben so erfolgreich Statik gelernt.

Es ist nicht unsere Absicht, Statik in ein rigoros System von Axiomen und Theoremen zu verwandeln. Ein solcher Versuch w\"{u}rde mehr Unheil anrichten als dass er Gutes bewirkt. Statik kann nicht und sollte nicht im Sinne eines mathematischen Lehrbuches gelehrt werden\footnote{Auch wenn die  Lufthoheit, die die Mathematik (scheinbar) garantiert, viele Menschen anzieht. Aber die Statik lebt von lebendiger Anschauung und nicht von rigider Axiomatik. {\em Babu\v{s}ka\/} hat sogar einmal einen Doktoranden vor den Mathematikern gewarnt: \glq {\em Mathematicians are very clever\/}\grq{}. Anders gesagt: {\em Don't fall into their trap\/}, \cite{Babuska6}. }. Wir glauben nur, dass wir an einem Punkt des Studiums den Studenten beibringen sollten, warum die doch so zentrale Gleichung (\ref{Eq145}) richtig ist.

Die Gleichung ist wahr, weil
\begin{itemize}
  \item $w \in C^4(0,l)$ (wie wir annehmen) eine L\"{o}sung des Randwertproblems ist
\begin{align}\label{Eq146}
EI\,w^{IV} = p \qquad w(0)= w(l) = M(0) = M(l) = 0
\end{align}
  \item  $\textcolor{red}{\delta w} \in C^2(0,l)$  (wie wir annehmen) eine zul\"{a}ssige virtuelle Verr\"{u}ckung ist,  $\textcolor{red}{\delta w(0)} = \textcolor{red}{\delta w(l)} = 0$
  \item und wir die Regeln der partiellen Integration
\begin{align}
\int_0^{\,l} w'(x)\,\textcolor{red}{\delta w(x)}\,dx = [w\,\textcolor{red}{\delta w}]_{@0}^{@l} - \int_0^{\,l} w(x)\,\textcolor{red}{\delta w'(x)}\,dx
\end{align}
zweimal anwenden.
\end{itemize}
Da partielle Integration keinen Unterschied zwischen \glq gro{\ss}\grq{} und \glq klein\grq{} macht, kann $\textcolor{red}{\delta w}$ von beliebiger Gr\"{o}{\ss}e sein. Auf der linken Seite steht eine Zahl und auf der rechten Seite steht eine Zahl
\begin{align}
0.56@789\ldots = 0.56@789\ldots
\end{align}
die in allen Ziffern gleich sind. Welches mechanische Prinzip kann dies garantieren? Oder wenn wir das Argument auf den Kopf stellen, welches {\em mathematische Gesetz\/} w\"{u}rde missachtet werden, wenn $\textcolor{red}{\delta w}$ gro{\ss} w\"{a}re? Hat je ein Mathematiker eine Gleichung dadurch bewiesen, dass er sich auf ein Naturgesetz bezogen hat?

Die Balkenkr\"{u}mmung auf $\kappa \simeq w''$ zu reduzieren, wenn $ w' \ll 1$ ist, ist ein legitimes Argument, um die Balkengleichung zu linearisieren, aber man ist dann doch erstaunt, wenn einem ein Ingenieur erkl\"{a}rt -- wie uns das mehrfach passiert ist -- dass die Gleichung (\ref{Eq145}) nur solange richtig ist, solange $\textcolor{red}{\delta w}$ klein ist\footnote{Bei solchen Gespr\"{a}chen w\"{u}nscht man sich die Mathematiker als Zuh\"{o}rer, w\"{u}nscht sich, dass sie einmal ihre {\em splendid isolation\/} verlassen und verstehen, was Ingenieur-Mathematik ist. {\em \glq Epsilontik\grq{}\/} ist einfach, das schwere ist die Bedeutung! Viele Missverst\"{a}ndnisse beruhen auf dieser {\em Zwi-Natur\/} der Ingenieur-Mathematik.}

{\em Rechnen\/} ist Mathematik und die Arbeits- und Energieprinzipe, die der Ingenieur dabei geschickt zu seinem Vorteil nutzt, sind im Grunde mathematische Identit\"{a}ten, die der Ingenieur sich in seine Sprache \"{u}bersetzt hat.

\vspace{-0.5cm}
%%%%%%%%%%%%%%%%%%%%%%%%%%%%%%%%%%%%%%%%%%%%%%%%%%%%%%%%%%%%%%%%%%%%%%%%%%%%%%%%%%%%%%%%%%%%%%%%%%%
{\textcolor{sectionTitleBlue}{\section{Nur, wenn Gleichgewicht herrscht?}}}
Die Identit\"{a}ten beruhen auf einer Kette von Umformungen mittels partieller Integration und daher ist das Ergebnis $\text{\normalfont\calligra G\,\,}(w, \textcolor{red}{\delta w}) = 0 $ immer richtig.

Nun wird aber bei der Formulierung der Arbeitsprinzipien der Statik immer davon gesprochen, dass die Systeme im Gleichgewicht sein m\"{u}ssen. Warum diese Einschr\"{a}nkung? Das liegt an den Abk\"{u}rzungen, die in der Literatur an dieser Stelle genommen werden.

Um nachzuweisen, dass die Biegelinie eines Balkens,
\begin{align}\label{Eq147}
EI\,w^{IV}(x) = p(x) \qquad w(0) = w(l) = 0 \qquad M(0) = M(l) = 0\,,
\end{align}
dem {\em Prinzip der virtuellen Verr\"{u}ckungen\/} gen\"{u}gt, setzen wir -- bei unserem Ansatz -- die Biegelinie $w$ und eine zul\"{a}ssige virtuelle Verr\"{u}ckung $\textcolor{red}{\delta w }$ in die erste Greensche Identit\"{a}t ein
\begin{align} \label{Eq30}
\text{\normalfont\calligra G\,\,}(w, \textcolor{red}{\delta w})  = \int_0^{\,l} EI\,w^{IV}(x)\,\textcolor{red}{\delta w}\,dx + [V\,\textcolor{red}{\delta w} - M\,\textcolor{red}{\delta w'}]_{@0}^{@l} - \int_0^{\,l} \frac{M\,\textcolor{red}{\delta M}}{EI}\,dx = 0\,,
\end{align}
und wir erhalten so unter Ber\"{u}cksichtigung von
\begin{align} \label{Eq16}
EI\,w^{IV}(x) = p(x) \qquad M(0) = M(l) = 0 \qquad \textcolor{red}{\delta w(0) = \delta w(l) = 0}
\end{align}
das bekannte Ergebnis
\begin{align}\label{Eq15}
\text{\normalfont\calligra G\,\,}(w,\textcolor{red}{\delta w}) = \int_0^{\,l} p(x)\,\textcolor{red}{\delta w(x)}\,dx - \int_0^{\,l} \frac{M\,\textcolor{red}{\delta M}}{EI}\,dx = {\delta A_a - \delta A_i = 0}\,.
\end{align}
Anders in der Literatur: Dort wird die zu Grunde liegende Identit\"{a}t (\ref{Eq30}) gar nicht angeschrieben, sondern die Autoren formulieren direkt die verk\"{u}rzte Identit\"{a}t (\ref{Eq15}). Das verpflichtet die Autoren aber dann zu dem Hinweis, dass das ganze nur gilt, wenn der Balken im Gleichgewicht ist ($w$ gen\"{u}gt (\ref{Eq147})) und $\textcolor{red}{\delta w} $ eine zul\"{a}ssige virtuelle Verr\"{u}ckung ist.

Es gibt also sozusagen zwei Formulierungen der ersten Greenschen Identit\"{a}t: Eine \glq blanke\grq{} Formulierung, (\ref{Eq30}), bei der man nichts ver\"{a}ndert, und die garantiert null ist, weil ja alles nur auf partieller Integration beruht.

Bei der zweiten Formulierung substituiert man dagegen f\"{u}r \glq interne\grq{} Terme \glq externe\grq{} Daten, nutzt aus, dass die Biegelinie $w(x) $ das Randwertproblem l\"{o}st und $\textcolor{red}{\delta w} $ eine zul\"{a}ssige virtuelle Verr\"{u}ckung ist, s. (\ref{Eq16}). Diese Ersetzungen bewahren jedoch nur dann die Balance, wenn $w(x) $ wirklich die L\"{o}sung des Randwertproblems ist. Es ist ein {\em Stellvertreter-Problem\/}.

Das steckt hinter der Bemerkung, dass die Arbeitsprinzipe nur gelten, wenn das System im Gleichgewicht ist.

Man kann es auch anders ausdr\"{u}cken:  Die erste Greensche Identit\"{a}t ist $\infty \times \infty$,  unendlich viele $w $ und unendlich viele $\delta w $ gen\"{u}gen der Identit\"{a}t. Wenn man aber $w$ durch Forderungen wie $EI\,w^{IV} =  p$ und $w(0) = 0$ etc. festlegt, fixiert,  dann wird daraus ein $1 \times \infty$; so kommt es zu den Bemerkungen in der Literatur.



%%%%%%%%%%%%%%%%%%%%%%%%%%%%%%%%%%%%%%%%%%%%%%%%%%%%%%%%%%%%%%%%%%%%%%%%%%%%%%%%%%%%%%%%%%%%%%%%%%%
{\textcolor{sectionTitleBlue}{\section{Was ist Weg und was ist Kraft?}}}\index{Weg und Kraft}
Bei der Umformung des Arbeitsintegrals
\begin{align}
\int_0^{\,l} EI\,w^{IV}(x)\, w(x)\,dx \qquad \text{(Eigenarbeit)}
\end{align}
mittels partieller Integration erscheinen wie von selbst die Weg- und Kraftgr\"{o}{\ss}en, die zur Differentialgleichung geh\"{o}ren. Sie bilden paarweise die Rand\-arbeiten
\begin{align}
[V\, w - M\,w']_{@0}^{@l} = V(l)\,w(l) - M(l)\,w'(l) - V(0)\,w(0) + M(0)\,w'(0)\,,
\end{align}
woran man ablesen kann, dass die Querkraft $V$ zu $w$ konjugiert ist und das Moment $M$ zu $w'$. Das scheint uns selbstverst\"{a}ndlich, weil wir es nicht anders kennen, aber hier ist die Stelle, wo das amtlich gemacht wird.

Der Versuch eines Kollegen etwa die Gr\"{o}{\ss}e $w + 0.5\,w'$ als die \glq wahre\grq{} Weggr\"{o}{\ss}e auf dem Rand zu definieren, die zu $V$ konjugiert ist, muss scheitern, weil es in der ersten Greenschen Identit\"{a}t anders steht. Sie h\"{a}lt uns auf dem rechten Weg.

Bei der (zweimaligen) partiellen Integration der Arbeitsgleichung des Balkens nach Theorie II. Ordnung
\begin{align}
\int_0^{\,l} (EI\,w^{IV}(x) + P\,w''(x))\,w(x)\,dx &= [\underbrace{(EI\,w'''(x) + P\,w'(x))}_{- T(x)}\, w + M\,w']_{@0}^{@l}\nn \\ &+ \int_0^{\,l} (\frac{M^2}{EI} - P\,(w'(x))^2)\,dx
\end{align}
lernen wir z.B., dass nun der Ausdruck
\begin{align}
 T(x) = - EI\,w'''(x) - P\,w'(x) = V(x) - P\,w'(x)\,,
 \end{align}
die Transversalkraft, zu $w(x)$ konjugiert ist und wir lesen aber auch ab, dass $M(x) $ dasselbe Moment ist, wie bei der Theorie erster Ordnung und dass $M $ weiterhin zu $w' $ konjugiert ist.

Die Unterscheidung zwischen Weg und Kraft ist auch f\"{u}r die finiten Elemente wichtig. {\em Shape functions\/} sind konform, wenn ihre Weggr\"{o}{\ss}en stetig sind. Das ist der Grund, warum man einen Balken nicht mit H\"{u}tchenfunktionen $\Np_i(x)$ berechnen kann. Die Spr\"{u}nge in der ersten Ableitung $\Np_i'(x)$ disqualifizieren solche Funktionen ($w'$ z\"{a}hlt als Weggr\"{o}{\ss}e beim Balken). \index{konforme Elemente}

%%%%%%%%%%%%%%%%%%%%%%%%%%%%%%%%%%%%%%%%%%%%%%%%%%%%%%%%%%%%%%%%%%%%%%%%%%%%%%%%%%%%%%%%%%%%%%%%%%%
{\textcolor{sectionTitleBlue}{\section{Die Zahl der Weg- und Kraftgr\"{o}{\ss}en}}}\index{Zahl der Weg- und Kraftgr\"{o}{\ss}en}
Zu Differentialgleichungen zweiter Ordnung geh\"{o}ren eine Weg- und eine Kraftgr\"{o}{\ss}e. Bei einem Stab, $-EA\,u''(x)$, sind dies
\begin{align}
u(x) \quad\text{0-te Ableitung} \qquad N(x) = EA\,u'(x)\quad\text{1. Ableitung}
\end{align}
bei einem Seil, $- H\,w''(x)$,
\begin{align}
w(x) \quad\text{0-te Ableitung} \qquad V(x) = H\,w'(x)\quad\text{1. Ableitung}
\end{align}
oder einem Schubtr\"{a}ger, $- GA\,w_s''(x)$,
\begin{align}
w(x)  \quad\text{0-te Ableitung} \qquad V(x) = GA\,w'(x)\quad\text{1. Ableitung}\,.
\end{align}
Zu Differentialgleichungen vierter Ordnung geh\"{o}ren dagegen je zwei Weg- und Kraftgr\"{o}{\ss}en
\begin{alignat}{2}
&w(x), \, w'(x) \quad&&\text{0-te und 1. Ableitung} \\
&M(x) = - EI \,w''(x), V(x) = - EI\,w'''(x) \quad&&\text{2. und 3. Ableitung}\,.
\end{alignat}
Diese Unterscheidung ist nicht ganz unwichtig, weil Einflussfunktionen f\"{u}r Weggr\"{o}{\ss}en sowohl mit dem {\em Prinzip der virtuellen Kr\"{a}fte\/} als auch dem {\em Satz von Betti\/} berechnet werden k\"{o}nnen, w\"{a}hrend Einflussfunktionen f\"{u}r Kraftgr\"{o}{\ss}en in der Regel nur mit dem {\em Satz von Betti\/} berechnet werden k\"{o}nnen.

%----------------------------------------------------------------------------------------------------------
\begin{figure}[tbp]
\centering
\if \bild 2 \sidecaption \fi
\includegraphics[width=1.0\textwidth]{\Fpath/U51}
\caption{Oszillierende Belastung auf einem Seil und das getreue Echo im Seil} \label{U51}
\end{figure}%
%----------------------------------------------------------------------------------------------------------

%%%%%%%%%%%%%%%%%%%%%%%%%%%%%%%%%%%%%%%%%%%%%%%%%%%%%%%%%%%%%%%%%%%%%%%%%%%%%%%%%%%%%%%%%%%%%%%%%%%
{\textcolor{sectionTitleBlue}{\section{Warum das Minus in $-H\,w'' = p$?}}}
{\em Warum beginnen eigentlich alle Differentialgleichungen zweiter Ordnung mit einem Minus?\/}

Der Mathematiker w\"{u}rde antworten: Das h\"{a}ngt mit den trigonometrischen Funktionen zusammen. Eine wellenf\"{o}rmige Belastung $ p(x) = \sin (x)$ zwingt dem Seil ein entsprechendes Echo auf, $w(x) = 1/H \,\sin \,(x)$, s. Abb. \ref{U51}. Weil nun aber die zweite Ableitung des Sinus negativ ist, muss ein Minus vor der Differentialgleichung dies korrigieren
\begin{align}
- H\,w''(x) = - H\,(- \frac{1}{H}\,\sin(x)) = \sin(x)\,.
\end{align}
Statisch kommt das Minus aus dem Gleichgewicht $- V + V + dV + p\,dx = 0$ am infinitesimalen Element, aber was wissen $\sin (x)$ und $\cos(x)$ vom Gleichgewicht -- und wie kommt es, dass der {\em switch\/} $(1) \cdot (-1)$ in  $\sin(x)$ eingebaut ist, aber nicht in Polynome? Sind sie nicht identisch\footnote{Die Reihenentwicklung von $\sin(x)$ und das alternierende Vorzeichen sind anscheinend der Grund...}
\begin{align}
\sin(x) = \frac{x}{1!} - \frac{x^3}{3!} + \frac{x^5}{5!} + \ldots
\end{align}
Bei der Balkengleichung, $ EI w^{IV}(x)$, korrigiert sich der \glq Fehler\grq{} im \"{u}brigen durch die  viermalige Differentiation, $(-1) \cdot (1) \cdot (-1) \cdot (1) = 1$, von selbst.


%%%%%%%%%%%%%%%%%%%%%%%%%%%%%%%%%%%%%%%%%%%%%%%%%%%%%%%%%%%%%%%%%%%%%%%%%%%%%%%%%%%%%%%%%%%%%%%%%%%
{\textcolor{sectionTitleBlue}{\section{Die virtuelle innere Energie}}}\index{virtuelle innere Energie}
Das symmetrische Gebietsintegral in den  Greenschen Identit\"{a}ten ist die virtuelle innere Energie, die oft auch knapper
\begin{align}
a(w,\textcolor{red}{\delta w}) := \int_0^{\,l} \frac{M\,\textcolor{red}{\delta M}} {EI}\,dx
\end{align}
geschrieben wird. Wir bezeichnen sie auch als {\em Wechselwirkungsenergie\/}\index{Wechselwirkungsenergie} oder {\em strain energy product\/}\index{strain energy product}, weil das besser die Gleichwertigkeit der beiden Funktionen  $w$ und $\delta w$ zum Ausdruck bringt.

Auf der Diagonalen, $\delta w = w$, ist die Wechselwirkungsenergie gleich der inneren Energie
\begin{align}
\frac{1}{2}\, a(w,w) = \frac{1}{2}\, \int_0^{\,l} \frac{M^2}{EI}\,dx\,.
\end{align}
Sind $ w(x) $ und $ \delta w(x)$ zusammengesetzte Funktionen, $c_i$ und $d_i$ m\"{o}gen beliebige Zahlen sein,
\begin{align}
w(x)= c_1\,w_1(x) + c_2\,w_2(x) \qquad \delta w(x) = d_1\,\delta w_1(x) + d_2\,\delta w(x)\,,
\end{align}
dann kann man das Gesamtergebnis auf die \"{U}berlagerung der einzelnen Bie\-ge\-linien zur\"{u}ckf\"{u}hren
\begin{align}
a(w,\delta w) = c_1\,d_1\,a(w_1,\delta w_1) &+ c_1\,d_2\,a(w_1,\delta w_2)\nn \\
 &+ c_2\,d_1\,a(w_2,\delta w_1) + c_2\,d_2\,a(w_2,\delta w_2)\,.
\end{align}
Deswegen nennt man $ a(w,\delta w) $ eine {\em symmetrische Bilinearform\/}\index{Bilinearform}\index{symmetrische Bilinearform}.

%----------------------------------------------------------------------------------------------------------
\begin{figure}[tbp]
\centering
\if \bild 2 \sidecaption \fi
\includegraphics[width=0.75\textwidth]{\Fpath/U65}
\caption{Auslenkung $y'$ bei einer echten Drehung und Auslenkung $y$ bei einer Pseudodrehung} \label{U65}
\end{figure}
%----------------------------------------------------------------------------------------------------------

%%%%%%%%%%%%%%%%%%%%%%%%%%%%%%%%%%%%%%%%%%%%%%%%%%%%%%%%%%%%%%%%%%%%%%%%%%%%%%%%%%%%%%%%%%%%%%%%%%%
{\textcolor{sectionTitleBlue}{\section{Gleichgewicht}}}\index{Gleichgewicht}

Am freigeschnittenen Balken herrscht Gleichgewicht, wenn die \"{a}u{\ss}eren Kr\"{a}fte ($p$ + Lagerkr\"{a}fte) orthogonal sind zu allen  Funktionen $\textcolor{red}{\delta w}$, deren Momente null sind, denn dann ist
\begin{align}
a(w,\textcolor{red}{\delta w}) = \int_0^{\,l} \frac{M\,\textcolor{red}{\delta M}}{EI}\,dx = 0\,,
\end{align}
und in der ersten Greenschen Identit\"{a}t bleiben nur die Arbeiten der \"{a}u{\ss}eren Kr\"{a}fte \"{u}brig und die sind null
\begin{align}\label{Eq114}
\text{\normalfont\calligra G\,\,}(w,\textcolor{red}{\delta w}) = \int_0^{l} p\,\textcolor{red}{\delta w}\,dx + [V\,\textcolor{red}{\delta w} - M\,\textcolor{red}{\delta w'}]_{@0}^{@l} = 0\,.
\end{align}
Beim Balken sind die \glq Null-Energie\grq{}-Funktionen die Starrk\"{o}rperbewegungen
\begin{align}
\textcolor{red}{\delta w(x) = a + b\,x}\,.
\end{align}
Hier ist die Stelle, wo die {\em Pseudodrehungen\/}\index{Pseudodrehungen} in die Statik hinein kommen. Im Unterschied zu echten Drehungen bleiben die Punkte nicht auf dem Drehkreis, sondern sie folgen der Tangente an den Drehkreis, s. Abb. \ref{U65},
\begin{align}
\tan\,\Np = \frac{y}{x}\,.
\end{align}
Das ist kein \glq Defekt\grq{}, sondern es liegt in der Natur der Balkengleichung. Die Mathematik bestimmt an Hand von $a(w,\delta w) = 0$, dass Drehungen bei Balken so aussehen m\"{u}ssen. Alle Einflussfunktionen statisch bestimmter Tragwerke sind ja kinematische Ketten und als solche basieren sie auf Pseudodrehungen. {\em Echte Drehungen w\"{u}rden zu falschen Ergebnissen f\"{u}hren\/}.

Die Balkenendkr\"{a}fte $V$ und -momente $M$ eines Balkens stehen mit der verteilten Belastung  $p$ also genau dann im Gleichgewicht, wenn (\ref{Eq114}) f\"{u}r alle $\textcolor{red}{\delta w = a\,x + b }$ gilt. Ein erfolgreicher Test mit $\textcolor{red}{\delta w = 1}$ bedeutet, dass die Summe der vertikalen Kr\"{a}fte null ist, und mit $\textcolor{red}{\delta w = x}$, dass die Summe der Momente um das linke Lager  null ist. F\"{u}r die Kontrolle von $M = 0$ in anderen Punkten w\"{a}hle man $a, b$ geeignet!

Jede Funktion $w $ aus $C^4(0,l)$ ist im \"{u}brigen im Gleichgewicht, denn ihre Kraftgr\"{o}{\ss}en
\begin{align}
EI\,w^{IV}(x) \qquad M(x) = - EI w''(x) \qquad V(x)= - EI\,w'''(x)
\end{align}
gen\"{u}gen der Gleichung
\begin{align}
\text{\normalfont\calligra G\,\,}(w,\textcolor{red}{\delta w}) = \int_0^{\,l} EI\,w^{IV}(x)\,\textcolor{red}{\delta w(x)}\,dx + [V\,\textcolor{red}{\delta w} - M\,\textcolor{red}{\delta w'}]_{@0}^{@l} = 0\,,
\end{align}
wie immer die Starrk\"{o}rperbewegung $\textcolor{red}{\delta w(x) = a + b\,x}$ aussieht -- garantiert!

So ist eine Sinus-Welle $w(x) = \sin\, (x)$ im Gleichgewicht
\begin{align}
\text{\normalfont\calligra G\,\,}(\sin\,(x),\textcolor{red}{1}) &= \int_0^{\,l}\!\!EI\,\sin\,(x)\cdot \textcolor{red}{1}\,dx + [EI\,\cos\,(x) \cdot \textcolor{red}{1}]_{@0}^{@l} = 0\\
\text{\normalfont\calligra G\,\,}(\sin\,(x),\textcolor{red}{x}) &= \int_0^{\,l}\!\!EI\,\sin\,(x)\,\cdot \textcolor{red}{x} \,dx + [\underbrace{EI\,\cos\,(x)}_{V} \cdot \textcolor{red}{x} - \underbrace{( EI\,\sin\,(x))}_{M}\cdot \textcolor{red}{1}]_{@0}^{@l} = 0\,.
\end{align}
Bei einem Fundamentbalken ist das anders. Zur Differentialgleichung des elas\-tisch gebetteten Balkens
\begin{align}
EI\,w^{IV}(x) + c\,w(x) = p(x)\,,
\end{align}
geh\"{o}rt die Identit\"{a}t
\begin{align}
\text{\normalfont\calligra G\,\,}(w,\textcolor{red}{\delta w}) &= \int_0^{\,l} p(x)\,\textcolor{red}{\delta w}\,dx + [V\,\textcolor{red}{\delta w} - M\,\textcolor{red}{\delta w}']_{@0}^{@l} \nn  \\
&- \int_0^{\,l}(\frac{M\,\textcolor{red}{\delta M}}{EI} + c\,w(x)\,\textcolor{red}{\delta w(x)}
)\,dx = 0\,.
\end{align}
Jetzt gibt es keine Funktion $\textcolor{red}{\delta w} $, au{\ss}er $\textcolor{red}{\delta w = 0} $, die die virtuelle innere Energie
\begin{align}
a(w,\textcolor{red}{\delta w}) := \int_0^{\,l}(\frac{M\,\textcolor{red}{\delta M}}{EI} + c\,w(x)\,\textcolor{red}{\delta w(x)})\,dx
\end{align}
zu null macht. Gibt es also keine Gleichgewichtsbedingungen? Nun die Terme in der ersten Greenschen Identit\"{a}t m\"{u}ssen zumindest orthogonal sein zu allen Funktionen $\delta w \in C^2 $. W\"{a}hlen wir die Funktion $\textcolor{red}{\delta w(x) = 1} $, dann folgt
\begin{align}
\text{\normalfont\calligra G\,\,}(w,\textcolor{red}{1}) = \int_0^{\,l} p(x)\,dx + V(l) - V(0) - \int_0^{\,l} c\,w(x)\,dx = 0\,,
\end{align}
oder wegen $p(x) = EI\,^{IV}(x) + c\,w(x)$
\begin{align}
\text{\normalfont\calligra G\,\,}(w,\textcolor{red}{1}) = \int_0^{\,l} EI\,w^{IV}\,dx + V(l) - V(0) = 0\,,
\end{align}
was uns vertraut vorkommt. Analog resultiert die Drehung $\textcolor{red}{\delta w(x) = x}$ in der Momentenbedingung um den linken Anfangspunkt
\begin{align}
\text{\normalfont\calligra G\,\,}(w,\textcolor{red}{x}) = \int_0^{\,l} EI\,w^{IV}\,\textcolor{red}{x}\,dx + V(l)\,\textcolor{red}{x}  - M(l) \cdot \textcolor{red}{1} + M(0)\cdot \textcolor{red}{1} = 0\,.
\end{align}
Allerdings wird hier immer nur der Anteil $EI\,w^{IV} $ der Gesamtbelastung $p(x) = EI\,w^{IV}(x) + c\,w(x) $ bilanziert.


Wenn am Ende eines Fundamentbalkens eine Einzelkraft $P = V(l)$ steht, dann muss folglich gelten
\begin{align}
\int_0^{\,l} EI\,w^{IV}(x)\, dx + P = 0\,.
\end{align}
Wegen $EI\,w^{IV}(x) + c\,w(x) = 0$ (keine Streckenlast) kann man $EI\,w^{IV}(x) $ mit $-c\,w(x) $ vertauschen, und daher muss auch gelten
\begin{align}
 P =  \int_0^{\,l} c\,w(x)\,dx\,,
\end{align}
woraus man abliest, dass das Integral des Bodendrucks gleich der Kraft $P$ ist.


%%%%%%%%%%%%%%%%%%%%%%%%%%%%%%%%%%%%%%%%%%%%%%%%%%%%%%%%%%%%%%%%%%%%%%%%%%%%%%%%%%%%%%%%%%%%%%%%%%%
{\textcolor{sectionTitleBlue}{\section{Wie der Mathematiker das Gleichgewicht entdeckt}}}
Der Mathematiker wei{\ss}, dass jede Biegelinie $w(x)$ der Identit\"{a}t
\begin{align}
\text{\normalfont\calligra G\,\,}(w,\textcolor{red}{\delta w}) = 0
\end{align}
gen\"{u}gt, $w$ also orthogonal zu allen Biegelinien $\textcolor{red}{\delta w(x) = a + b\, x}$ sein muss. Weil in diesen F\"{a}llen aber $\delta A_i = 0$ ist, m\"{u}ssen notwendig die \"{a}u{\ss}eren Kr\"{a}fte ($p$ + Lagerkr\"{a}fte) zu jedem solchen $\textcolor{red}{\delta w}$ orthogonal sein
\begin{align}
\text{\normalfont\calligra G\,\,}(w,\textcolor{red}{\delta w}) = \delta A_a - \delta A_i = \delta A_a - 0 = 0\,,
\end{align}
und so entdeckt der Mathematiker die Forderung
\begin{align}
\sum V = 0 \qquad \sum M = 0\,,
\end{align}
ohne etwas von Statik zu wissen.
\pagebreak
%%%%%%%%%%%%%%%%%%%%%%%%%%%%%%%%%%%%%%%%%%%%%%%%%%%%%%%%%%%%%%%%%%%%%%%%%%%%%%%%%%%%%%%%%%%%%%%%%%%
{\textcolor{sectionTitleBlue}{\section{Die Mathematik hinter dem Gleichgewicht}}}
Das Gleichgewicht basiert im Grunde auf dem {\em Hauptsatz der Differential- und Integralrechnung\/}
\begin{align}
\int_0^{\,l} f'(x)\,dx = f(l) - f(0)\,,
\end{align}
denn weil $EI\,w^{IV}(x) = - V'(x) = p(x)$ die Ableitung der Querkraft $V(x) $ ist, gilt
\begin{align}
\int_0^{\,l} -V'(x)\,dx = - V(l) + V(0)\quad \Rightarrow \quad \int_0^{\,l} p(x)\,dx + V(l) - V(0) = 0\,.
\end{align}
Um zu zeigen, dass z.B. das Moment und das linke Lager null ist, $M = 0$, starten wir mit dem Moment der Streckenlast $p(x) = - V'(x)$ um das linke Lager und formulieren es mittels partieller Integration um
\begin{align}
\int_0^{\,l} - V'(x)\,x\,dx = [-V\,x]_{@0}^{@l} - \int_0^{\,l} -V(x)\cdot 1\,dx\,.
\end{align}
Wegen $V(x) = M'(x)$ ergibt dann der Hauptsatz den Ausdruck
\begin{align}
\int_0^{\,l} - V'(x)\,x\,dx = [-V\,x]_{@0}^{@l} + M(l) - M(0)\,,
\end{align}
oder
\begin{align}
\int_0^{\,l} p(x)\,x\,dx + V(l)\cdot l -  M(l) + M(0) = 0\,.
\end{align}
%----------------------------------------------------------------------------------------------------------
\begin{figure}[tbp]
\centering
\if \bild 2 \sidecaption \fi
\centering
\includegraphics[width=.6\textwidth]{\Fpath/U206}
\caption{Theorie II. Ordnung}
\label{U206}%
\end{figure}%
%----------------------------------------------------------------------------------------------------------

%%%%%%%%%%%%%%%%%%%%%%%%%%%%%%%%%%%%%%%%%%%%%%%%%%%%%%%%%%%%%%%%%%%%%%%%%%%%%%%%%%%%%%%%%%%%%%%%%%%
{\textcolor{sectionTitleBlue}{\section{Gleichgewicht am verformten Tragwerk?}}}\index{Gleichgewicht am verformten Tragwerk}
In der Theorie erster Ordnung wird das Gleichgewicht am unverformten Tragwerk aufgestellt und in der Theorie zweiter Ordnung am verformten Tragwerk -- so hei{\ss}t es zumindest. Aber das ist nicht ganz  richtig. Die Theorie zweiter Ordnung ist in Wirklichkeit eine Mischung aus beiden Theorien, s. Abb. \ref{U206}.

Die seitliche Auslenkung des Balkens, also die Vergr\"{o}{\ss}erung der Durchbiegung geht in die Gleichgewichtsbedingung ein, aber die Verk\"{u}rzung der Stabachse in L\"{a}ngsrichtung nicht.

Es ist daher theoretisch auch nicht m\"{o}glich, das Gleichgewicht eines Rahmens, der nach Theorie zweiter Ordnung berechnet wurde, zu \"{u}berpr\"{u}fen. Denn die Knotenverformungen enthalten ja Beitr\"{a}ge aus  Theorie I. wie II. Ordnung. Dass das in der Praxis nicht auff\"{a}llt, liegt daran, dass die Verk\"{u}rzungen der St\"{a}be sehr klein sind, so dass man bei einer \"{U}berpr\"{u}fung der Gleichgewichtsbedingungen geneigt ist, Abweichungen auf Rundungsfehler zu schieben, \cite{HaM2}.


%%%%%%%%%%%%%%%%%%%%%%%%%%%%%%%%%%%%%%%%%%%%%%%%%%%%%%%%%%%%%%%%%%%%%%%%%%%%%%%%%%%%%%%%%%%%%%%%%%%
{\textcolor{sectionTitleBlue}{\section{Quellen und Senken}}}\index{Quellen und Senken}
Aus der Sicht der Physik sind die Gleichgewichtsbedingungen Erhaltungss\"{a}tze. Das, was aus einer Platte $\Omega$ am Rande herausflie{\ss}t, also die Lagerkr\"{a}fte auf dem Rand $\Gamma$, das ist der Kirchhoffschub $v_n$, muss in der Summe gleich der aufgebrachten Belastung sein\footnote{Die Eckkr\"{a}fte haben wir weggelassen.}
\begin{align}
\text{\normalfont\calligra G\,\,}(w,1) = \int_{\Omega} p \,d\Omega + \int_{\Gamma} v_n\,ds = 0\,.
\end{align}
Die Temperaturverteilung $T(\vek x)$ in einem Zimmer, das von einer Fu{\ss}bodenheizung $p$ (= Quellen) erw\"{a}rmt wird und dessen W\"{a}nde konstant auf null Grad gehalten werden, gen\"{u}gt der Gleichung
\begin{align}
- \Delta T = p \qquad T = 0 \qquad \text{am Rand $\Gamma$}\,.
\end{align}
Aus der ersten Greenschen Identit\"{a}t des Laplace-Operators,
\begin{align}\label{Eq90}
\text{\normalfont\calligra G\,\,}(u, v) = \int_{\Omega} - \Delta u\,v\,\,d\Omega + \int_{\Gamma} \frac{\partial u}{\partial n}\,v\,ds - \int_{\Omega} \nabla u \dotprod \nabla v\,d\Omega = 0\,,
\end{align}
in der Formulierung
\begin{align}
\text{\normalfont\calligra G\,\,}(T,1) = \int_{\Omega} p \,d\Omega + \int_{\Gamma} \frac{\partial T}{\partial n}\,ds = 0\,,
\end{align}
folgt: Was an W\"{a}rme am Rand $\Gamma$ wegflie{\ss}t, das ist das Integral des Flu{\ss} $\partial T/\partial n$, muss gleich der im Zimmer produzierten W\"{a}rme sein.

Fachwerkst\"{a}be sind frei von Quellen (keine Streckenlast zwischen den Stabenden) und daher m\"{u}ssen sich die Normalkr\"{a}fte am Anfang und am Ende in der Summe aufheben, was an Kraft hineinflie{\ss}t muss auch wieder herausflie{\ss}en
\begin{align}
\text{\normalfont\calligra G\,\,}(u,1) = [N\cdot 1]_{@0}^{@l} = N(l) - N(0) = 0\,.
\end{align}
Summarisch hat also die erste Greensche Identit\"{a}t, wenn man $\delta u = 1$ setzt, die Struktur
\begin{align}
\text{\normalfont\calligra G\,\,}(u,1) = \int_{\Omega} \text{{\em Quellen im Gebiet\/}}\,\,d\Omega + \int_{\Gamma} \text{{\em Fluss am Rand\/}}\,ds = 0\,.
\end{align}
%%%%%%%%%%%%%%%%%%%%%%%%%%%%%%%%%%%%%%%%%%%%%%%%%%%%%%%%%%%%%%%%%%%%%%%%%%%%%%%%%%%%%%%%%%%%%%%%%%%
{\textcolor{sectionTitleBlue}{\section{Das Prinzip vom Minimum der potentiellen Energie}}}\index{Prinzip vom Minimum der potentiellen Energie}

Gem\"{a}{\ss} dem Federgesetz
\begin{align}
k\,u = f
\end{align}
ist die Auslenkung $ u $ einer Feder proportional zur aufgebrachten Kraft $ f $, s. Abb. \ref{U208}.
%----------------------------------------------------------------------------------------------------------
\begin{figure}[tbp]
\if \bild 2 \sidecaption \fi
\centering
\includegraphics[width=0.9\textwidth]{\Fpath/U208}
\caption{Dort, wo $A_i = A_a$ ist, liegt der Gleichgewichtspunkt $u$ der Feder. Weil die
innere Energie $A_i$ quadratisch mit $u$ w\"{a}chst, die \"{a}u{\ss}ere Arbeit $A_a$ aber nur
linear, holt $A_i$ immer $A_a$ ein, gibt es immer eine Gleichgewichtslage}
\label{U208}
\end{figure}%
%----------------------------------------------------------------------------------------------------------

Die Kraft $ f $, die die Feder nach unten zieht, leistet dabei eine Arbeit (weil es Eigenarbeit ist, steht hier der Faktor $1/2$),
\begin{align}
A_a = \frac{1}{2}\,f\,u\,,
\end{align}
und diese Arbeit wird als innere Energie in der Feder gespeichert
\begin{align}\label{Eq5}
A_i = \frac{1}{2}\, k\,u^2\,.
\end{align}
Wir erwarten nat\"{u}rlich, dass in der Gleichgewichtslage die \"{a}u{\ss}ere Arbeit und die innere Energie gleich gro{\ss} sind
\begin{align}
A_a = \frac{1}{2}\, f\,u =  \frac{1}{2}\, k\,u^2 = A_i\,,
\end{align}
was aber  durch die Identit\"{a}t
\begin{align}
\text{\normalfont\calligra G\,\,}(u,\delta u) = \delta u\,k\,u - u\,k\,\delta u = 0
\end{align}
garantiert ist, denn
\begin{align}
\frac{1}{2}\,\text{\normalfont\calligra G\,\,}(u,u) = \frac{1}{2}\, u\,k\,u - \frac{1}{2}\, u\,k\,u = \frac{1}{2}\, u\,f - \frac{1}{2}\, k\,u^2 = A_a - A_i = 0\,.
\end{align}
Tr\"{a}gt man den Verlauf der Funktion $1/2\,f\,u $ und der Funktion $1/2\,k\,u^2 $ als Funktion der Auslenkung $u$ auf, dann ist die Auslenkung $u$ der Feder  unter der Wirkung der Kraft $ f $ genau der Punkt $u$, in dem sich die beiden Kurven schneiden, siehe Abb. \ref{U208}.

Die dritte Kurve in Abb. \ref{U208} ist die potentielle Energie $\Pi$ der Feder
\begin{align}
\Pi(u) = \frac{1}{2}\, k\,u^2 - f\,u\,.
\end{align}
Der Faktor $1/2$ macht, dass sich bei der Bildung der Ableitung die 2 wegk\"{u}rzt
\begin{align}
\Pi'(u) = k\,u - f
\end{align}
und so, weil die Auslenkung $ u $ der Feder dem Federgesetz $k\,u = f $ gen\"{u}gt, die potentielle Energie im Gleichgewichtspunkt $u$ eine horizontale Tangente hat, $\Pi'(u) = 0$.

Die interessante Beobachtung ist nun, siehe Abb. \ref{U208}, dass der Punkt $u$, in dem sich die \"{a}u{\ss}ere und innere Arbeit schneiden,  auch genau der Punkt $u$ ist, in dem die potentielle Energie ihr Minimum hat.

Wie man im Abb. \ref{U208} sieht, steigt am Anfang die \"{a}u{\ss}ere Arbeit schneller als die innere Energie, aber dann passieren die beiden Kurven einen Punkt, von dem ab die innere Energie schneller w\"{a}chst als die \"{a}u{\ss}ere Arbeit. {\em Dieser Schnittpunkt ist der Gleichgewichtspunkt}. Nur in diesem Punkt gilt $A_a = A_i$.

W\"{u}rde von Anfang an die innere Energie schneller steigen als die \"{a}u{\ss}ere Arbeit, dann w\"{u}rde sich die Feder \"{u}berhaupt nicht bewegen, dann w\"{a}re schon im Nullpunkt der Wettlauf zu Ende.

Setzen wir alle Werte eins, also $ k = 1$ und $ f = 1 $, dann liegt der Gleich\-gewichtspunkt genau bei $ u = 1$. Woraus folgt, dass die ganze Mechanik und Statik im Grunde auf der Tatsache beruht, dass im Intervall $(0,1)$
die Ungleichung $u > u^2$ gilt und danach das Umgekehrte, $u^2 > u$.

%----------------------------------------------------------------------------------------------------------
\begin{figure}[tbp]
\centering
\if \bild 2 \sidecaption \fi
\includegraphics[width=.8\textwidth]{\Fpath/U209C}
\caption{Die potentielle Energie $\Pi(w_h)$ der FE-L\"{o}sung liegt
immer rechts von der exakten potentiellen Energie $\Pi(w)$, aber in beiden F\"{a}llen ist $\Pi$ eine nach oben offene Parabel, muss man Energie zuf\"{u}hren, um die Gleichgewichtslage zu verlassen} \label{U209}
\end{figure}%
%----------------------------------------------------------------------------------------------------------

%%%%%%%%%%%%%%%%%%%%%%%%%%%%%%%%%%%%%%%%%%%%%%%%%%%%%%%%%%%%%%%%%%%%%%%%%%%%%%%%%%%%%%%%%%%%%%%%%%%
{\textcolor{sectionTitleBlue}{\subsection{Minimum oder Maximum?}}}
Man kann die Lastf\"{a}lle (LF) in der Statik in zwei Typen, $p$ und $\Delta$, einteilen:\\

\begin{itemize}
  \item In einem LF $p$ \index{LF $p$} werden Kr\"{a}fte aufgebracht
  \item In einem LF $\Delta$\index{LF $\Delta$} werden Lagerverschiebungen/-verdrehungen aufgebracht.
\end{itemize}

Wir werden sehen, dass der Typ des Lastfalls bestimmt, ob die potentielle Energie in der Gleichgewichtslage positiv oder negativ ist.

In einem LF $p$ ist die potentielle Energie in der tiefsten Lage negativ, wie man durch Einsetzen ($ k\,u = f$) direkt verifiziert
\begin{align}
\Pi(u) = \frac{1}{2}\,k\,u^2 - f\,u = \frac{1}{2}\, f\,u - f\,u = - \frac{1}{2}\, f\,u\,.
\end{align}
Nun ist aber die Auslenkung $ u $ der Sieger in dem Wettbewerb, die potentielle Energie m\"{o}glichst klein zu machen, und das hei{\ss}t doch anschaulich, dass $ u $ den {\em gr\"{o}{\ss}tm\"{o}glichen Abstand\/} $|\Pi(u)|$ vom Nullpunkt hat.\\

\hspace*{-12pt}\colorbox{highlightBlue}{\parbox{0.98\textwidth}{Das Prinzip vom Minimum der potentiellen Energie ist in einem LF $p$ eigentlich ein Maximumsprinzip: Der Betrag $|\Pi(u)|$ wird maximiert. Nur weil die potentielle Energie in der Gleichgewichtslage negativ ist, ist das dasselbe, wie das Minimum der potentiellen Energie zu finden.}}
\\

%----------------------------------------------------------------------------------------------------------
\begin{figure}[tbp]
\centering
\if \bild 2 \sidecaption \fi
\includegraphics[width=1.0\textwidth]{\Fpath/U367}
\caption{Je weniger Festhaltungen desto gr\"{o}{\ss}er ist $\mathcal{V}$ und umso gr\"{o}{\ss}er wird der Betrag der potentiellen Energie, $|\Pi|$, in der Gleichgewichtslage; alle Werte $\times EI^{-1}$}
\label{U367}
\end{figure}%
%----------------------------------------------------------------------------------------------------------

Wir interpretieren das Prinzip in der Regel so, wie es die Wortwahl (anscheinend)  suggeriert, mit m\"{o}glichst wenig Anstrengung zum Ziel kommen, die potentielle Energie m\"{o}glichst klein machen, m\"{o}glichst nahe an null zu r\"{u}cken, w\"{a}hrend die wahre Bedeutung genau das Gegenteil ist, s. Abb. \ref{U209}. Je gr\"{o}{\ss}er $\mathcal{V}$ ist, je weniger Fesseln es gibt, desto negativer wird die potentielle Energie bei demselbe $p$, desto weiter r\"{u}ckt $|\Pi(w)|$ von null ab, s. Abb. \ref{U367}.
%-----------------------------------------------------------------
\begin{figure}[tbp]
\centering
\if \bild 2 \sidecaption \fi
\includegraphics[width=0.71\textwidth]{\Fpath/1GreenF208A}
\caption{Unter allen auf Eins normierten Biegelinien $w/\|w\|$ in $\mathcal{V}$, ist die Biegelinie $w$ des Balkens die Funktion, die die gr\"{o}{\ss}te Wirkung aus $p$ ziehen kann, \cite{Ha6} }
\label{Supremum}
\end{figure}%%
%-----------------------------------------------------------------


{\em Das Bestreben der Kraft $ f $ ist es, m\"{o}glichst viel Energie aus der Feder herauszuholen, $|\Pi(u)|$ m\"{o}glichst gro{\ss} zu machen\/}.

Abb. \ref{Supremum} illustriert dieses versteckte Maximumprinzip an einem Balken. Man kann zeigen, dass die Biegelinie $w$ des Balkens unter allen auf Eins normierten Biegelinien $w/\|w\|$ aus $\mathcal{V}$ die ist, die die gr\"{o}{\ss}te Wirkung, {\em the most mileage\/}, aus dem Arbeitsintegral
\begin{align}
J(w) = \int_0^{\,l} p\,w\,dx
\end{align}
zieht. Die Norm ist hier die Wurzel aus dem Integral von $M^2/EI$, sie ist also mit der Energienorm $||w|| = \sqrt{a(w,w)}$ identisch.

Betrachten wir nun dagegen einen LF $\Delta$ wie in Abb. \ref{U305}. Wegen der fehlenden \"{a}u{\ss}eren Lasten reduziert sich die potentielle Energie auf den positiven Ausdruck
\begin{align}
\Pi(w)= \frac{1}{2}\,a(w,w) = \frac{1}{2}\, \int_0^{\,l} \frac{M^2}{EI}\,dx\,.
\end{align}
In einem LF $\Delta$ ist die potentielle Energie also positiv, {\em liegt sie rechts vom Nullpunkt\/}, und wenn man jetzt die potentielle Energie minimiert, dann sucht man die Verformung $ u $ oder $w$, die die potentielle Energie m\"{o}glichst nahe an null r\"{u}ckt, s. Abb. \ref{U209}. In dieser Situation hat das Prinzip vom Minimum der potentiellen Energie die Bedeutung, die wir ihm normalerweise unterlegen. Das Tragwerk versucht mit m\"{o}glichst wenig Widerstand, sprich mit m\"{o}glichst wenig innerer Energie, die Verformungen zu ertragen, die ihm aufgezwungen werden.
%----------------------------------------------------------------------------------------------------------
\begin{figure}[tbp]
\centering
\if \bild 2 \sidecaption \fi
\includegraphics[width=0.6\textwidth]{\Fpath/U305}
\caption{Lagersenkung }
\label{U305}
\end{figure}%
%----------------------------------------------------------------------------------------------------------

Bei der Interpolation mit {\em splines\/}\index{splines} nutzt man diese Intelligenz des Materials aus, indem man es dem Material \"{u}berl\"{a}sst die optimale Kurve $w$ durch den Slalom der Pfl\"{o}cke zu finden; optimal in dem Sinn, dass die Biegeenergie
\begin{align}
\|w\| = \sqrt{a(w,w)} = \left[ \int_{0}^{l}\frac{M^2}{EI}\,dx \right]^{\frac{1}{2}}
\end{align}
m\"{o}glichst klein wird, m\"{o}glichst nahe an null r\"{u}ckt, s. Abb. \ref{U521}.

%-----------------------------------------------------------------
\begin{figure}[tbp]
\centering
\if \bild 2 \sidecaption \fi
\includegraphics[width=1.0\textwidth]{\Fpath/U521}
\caption{Spline Interpolation als LF Lagersenkung }
\label{U521}
\end{figure}%%
%-----------------------------------------------------------------

%----------------------------------------------------------------------------------------------------------
\begin{figure}[tbp]
\centering
\if \bild 2 \sidecaption \fi
\includegraphics[width=0.9\textwidth]{\Fpath/U5}
\caption{In einem LF $p$ (Kr\"{a}fte) nehmen die Spannungen zu, wenn das Material rei{\ss}t, $\vek u_1 \to \vek u_2$, w\"{a}hrend sie in einem LF $\Delta$ (Wege) sinken}
\label{U5}
\end{figure}%
%----------------------------------------------------------------------------------------------------------
%----------------------------------------------------------------------------------------------------------
\begin{figure}[tbp]
\centering
\if \bild 2 \sidecaption \fi
\includegraphics[width=1.0\textwidth]{\Fpath/U287}
\caption{Je mehr Lager vorhanden sind, um so kleiner wird der Betrag der potentiellen Energie, weil der Ansatzraum $\mathcal{V}$ schrumpft}
\label{U287}
\end{figure}%
%----------------------------------------------------------------------------------------------------------

%%%%%%%%%%%%%%%%%%%%%%%%%%%%%%%%%%%%%%%%%%%%%%%%%%%%%%%%%%%%%%%%%%%%%%%%%%%%%%%%%%%%%%%%%%%%%%%%%%%
{\textcolor{sectionTitleBlue}{\subsection{Wenn das Material rei{\ss}t}}}
Das Prinzip vom Minimum der potentiellen Energie kann auch erkl\"{a}ren, warum die Spannungen in einem Bauteil wachsen, wenn das Bauteil Risse bekommt, s. Abb. \ref{U5}.

Wenn das Bauteil rei{\ss}t, m\"{u}ssen die Verschiebungen auf den Flanken des Risses nicht mehr gleich sein, wie das der Fall war, solange die Flanke noch im Inneren des ungerissenen Bauteils lag. {\em Risse bewirken also, dass der Ansatzraum $\mathcal{V}$ gr\"{o}{\ss}er wird\/}, weil mehr Funktionen an der Konkurrenz um das Minimum teilnehmen k\"{o}nnen. Damit rutscht das Minimum aber noch weiter weg vom Nullpunkt, wird es dem Betrage nach gr\"{o}{\ss}er, und das bedeutet, dass die potentielle Energie und damit die Spannungen in dem Bauteil steigen.

Wenn sich ein Lager senkt, haben Risse den gegenteiligen Effekt, die Spannungen sinken, weil $\Pi(\vek u) > 0$ jetzt n\"{a}her an null rutschen kann; wieder weil der Ansatzraum $\mathcal{V}$ durch die Risse gr\"{o}{\ss}er wird.
%----------------------------------------------------------------------------------------------------------
\begin{figure}[tbp]
\centering
\if \bild 2 \sidecaption \fi
\includegraphics[width=1.0\textwidth]{\Fpath/U40}
\caption{Wenn man die Zahl der Lager erh\"{o}ht, aber die Absenkung unver\"{a}ndert beibeh\"{a}lt, dann nimmt die potentiellen Energie zu}
\label{U40}
\end{figure}%
%----------------------------------------------------------------------------------------------------------

%%%%%%%%%%%%%%%%%%%%%%%%%%%%%%%%%%%%%%%%%%%%%%%%%%%%%%%%%%%%%%%%%%%%%%%%%%%%%%%%%%%%%%%%%%%%%%%%%%%
{\textcolor{sectionTitleBlue}{\subsection{Wenn Lager entfallen}}}
Dieselbe Logik gilt auch bei Durchlauftr\"{a}gern, bei denen die Zahl der Lager sozusagen der Gr\"{o}{\ss}e des Ansatzraums $\mathcal{V}$ entspricht, auf dem das Minimum der potentiellen Energie gesucht wird, s. Abb. \ref{U287}. Je mehr Lager vorhanden sind, um so kleiner ist der Ansatzraum, weil die wachsende Anzahl von Zwangspunkten, $w(x) = 0$, die Zahl der Konkurrenten immer kleiner werden l\"{a}sst, $\mathcal{V}$ also schrumpft, und das bedeutet, dass die potentielle Energie in einem LF $p$ dem Betrage nach kleiner wird.

Gerade bei Einzelkr\"{a}ften kann man das direkt an den Verformungen ablesen, denn bei diesen ist die potentielle Energie proportional zur Durchbiegung unter der Einzelkraft
\begin{align}
\Pi(w) = \frac{1}{2}\,\int_0^{\,l} \frac{M^2}{EI}\,dx - P\cdot w(l) = - \frac{1}{2}\, P \cdot w(l)
\end{align}
und wegen $|\Pi(w_2)| = 0.5 \,P\,w_2(l) < |\Pi(w_1)| = 0.5 \,P\,w_1(l)$ muss daher gelten, s. Abb. \ref{U287}, dass die Kragarmdurchbiegung kleiner wird, wenn man ein Zwischenlager einbaut.

Die umgekehrte Tendenz stellt sich ein, wenn man Lager wegnimmt, denn dann wird der Ansatzraum $\mathcal{V}$ gr\"{o}{\ss}er, weil weniger Forderungen an die Biegelinien gestellt werden, die an der Konkurrenz um das Minimum teilnehmen, und das bedeutet, dass der Betrag der potentiellen Energie w\"{a}chst.

Das gegenteilige Ph\"{a}nomen hat man, wenn man einen Durchlauftr\"{a}ger, dessen Ende man um einen vorgegebenen Betrag $w_\Delta$ nach unten dr\"{u}ckt (Lagersenkung), auf zus\"{a}tzliche Lager stellt, s. Abb. \ref{U40}. Wieder wird der Ansatzraum $\mathcal{V}$ kleiner, aber weil in einem LF $\Delta$ die potentielle Energie positiv ist, bedeutet dies, dass die potentielle Energie steigt. Es macht mehr M\"{u}he, einem Tr\"{a}ger mit $n + 1$ Lagern eine Verformung aufzuzwingen, als einem Tr\"{a}ger mit $n$ Lagern.
%----------------------------------------------------------------------------------------------------------
\begin{figure}[tbp]
\centering
\if \bild 2 \sidecaption \fi
\includegraphics[width=0.6\textwidth]{\Fpath/UE351}
\caption{Hauptsystem}
\label{UE351}
\end{figure}%
%----------------------------------------------------------------------------------------------------------

Auf den ersten Blick scheint es so zu sein, dass der Raum $\mathcal{V}$ nur schrumpft, wenn zus\"{a}tzlich \glq harte\grq{} Lagerbedingungen wie $w(x) = 0$ dazu kommen, aber auch \glq weiche\grq{} Nebenbedingungen lassen $\mathcal{V}$ schrumpfen. Eine St\"{u}tze, die man zum Beispiel unter das Ende eines Tr\"{a}gers stellt, s. Abb. \ref{UE351}, formuliert einen solchen {\em soft constraint\/} und $\mathcal{V}$ schrumpft.

Um das zu verstehen, argumentieren wir mit dem Kraftgr\"{o}{\ss}enverfahren. Der Balken mit der St\"{u}tze ist das statisch unbestimmte Hauptsystem und wir w\"{a}hlen die Normalkraft $N$ in der St\"{u}tze als statisch \"{U}berz\"{a}hlige $X_1$. Nach dem Einbau des Normalkraftgelenkes k\"{o}nnen sich der obere und untere Teil unabh\"{a}ngig voneinander bewegen.

Der Raum $\mathcal{V}$ des statisch bestimmten Hauptsystems besteht aus allen Biegelinien $w$, die die Lagerbedingungen des Balkens erf\"{u}llen, $w(0) = w'(0) = 0$, und aus allen L\"{a}ngsverschiebungen $u_1(x)$ und $u_2(x)$ der zweigeteilten St\"{u}tze.  Dieses System ist unserem urspr\"{u}nglichen System \"{a}quivalent, weil die zweigeteilte St\"{u}tze keine Lasten tr\"{a}gt.

Die Kraft $X_1$ unterliegt der Bedingung, dass der obere und untere Teil der St\"{u}tze in der Mitte die gleiche Verschiebung aufweisen, $u_1(h/2) = u_2(h/2)$, und das bedeutet das die Zahl der m\"{o}glichen Funktionen $u_i(x)$ kleiner wird, $\mathcal{V}$  schrumpft.

\hspace*{-12pt}\colorbox{highlightBlue}{\parbox{0.98\textwidth}{Wenn man Riegel oder Stiele zu einem Rahmen addiert, schrumpft der Raum $\mathcal{V}$, die Steifigkeit des Tragwerks nimmt zu.}}\\



%----------------------------------------------------------------------------------------------------------
\begin{figure}[tbp]
\centering
\if \bild 2 \sidecaption \fi
\includegraphics[width=1.0\textwidth]{\Fpath/U240}
\caption{Membran und zentrische Punktlast}
\label{U240}
\end{figure}%
%----------------------------------------------------------------------------------------------------------

%%%%%%%%%%%%%%%%%%%%%%%%%%%%%%%%%%%%%%%%%%%%%%%%%%%%%%%%%%%%%%%%%%%%%%%%%%%%%%%%%%%%%%%%%%%%%%%%%%%
{\textcolor{sectionTitleBlue}{\section{Unendliche Energie}}}

Die Erzeugung von Einflussfunktionen bedeutet f\"{u}r einen Rahmen eine gro{\ss}e Strapaze, denn dabei werden konzentrierte Punktlasten aufgebracht oder ein Riegel wird geknickt ($EF\!-\!M$) oder gar auseinander gerissen, (Verschiebungssprung, $EF\!-\!V$). Daher haben viele Einflussfunktionen unendlich gro{\ss}e Energie. Was das bedeutet, wollen wir im Folgenden diskutieren.

Von allen Fl\"{a}chentragwerken ist die Membran das einfachst m\"{o}gliche und daher beginnen wir mit einer kreisf\"{o}rmigen Membran, Radius $R = 1$, die eine \"{O}ffnung $\Omega$ \"{u}berdeckt und die am ringf\"{o}rmigen Rand $\Gamma$ gehalten wird. Unter einem Druck $p$ bildet sich eine Biegefl\"{a}che $u(\vek x)$ aus, die die L\"{o}sung des Randwertproblems
\begin{align}
 - H\,\Delta u = p \quad \text{in $\Omega$} \qquad u = 0 \quad \text{auf $\Gamma$}
\end{align}
ist. Die Konstante $H$ ist die in $x_1$- und $x_2$-Richtung, (also $x$ und $y$) gleich gro{\ss}e Vorspannkraft in der Membran. Wir werden  sp\"{a}ter $H = 1$ setzen.

Beim Seil ist die Schnittkraft die Querkraft $V = H\,w'$ und bei einer Membran gibt es nun zwei Querkr\"{a}fte $v_{x_1}, v_{x_2}$ (Kr\"{a}fte/lfd. m)
\begin{align}
\left[ \barr {c }
      v_{x_1}  \\
      v_{x_2}
     \earr \right]= H \left[ \barr {c }
      u,_{x_1}  \\
      u,_{x_2}
     \earr \right]= H\,\nabla u\,,
\end{align}
die proportional den Neigungen der Biegefl\"{a}che in die beiden Richtungen $x_1$ bzw. $x_2$ sind.

Die Arbeits- und Energieprinzipe der Membran basieren auf der ersten Greenschen Identit\"{a}t des Laplace-Operators,  also dem Ausdruck, (wir setzen $H = 1$),
\begin{align}
\text{\normalfont\calligra G\,\,}(u, \delta u) &= \underbrace{\int_{\Omega} -\Delta u\,\delta u\,d\Omega + \int_{\Gamma} \nabla u \dotprod \vek n\,\delta u\,ds}_{\delta A_a} - \underbrace{\int_{\Omega} \nabla u \dotprod \nabla \delta u \,d\Omega}_{\delta A_i} = 0\,.
\end{align}
Das Skalarprodukt zwischen dem Gradienten und dem Normalenvektor
\begin{align}
\nabla u \dotprod \vek n = u,_{x_1}\,n_1 + u,_{x_2}\,n_2 = \frac{\partial u}{\partial n}
\end{align}
ist die Normalableitung der Biegefl\"{a}che, also die Neigung der Membran am Rand in Richtung des nach Au{\ss}en zeigenden Normalenvektors. Wenn die Durchbiegung zum Rande hin kleiner wird, was die Regel ist, ist die Normalableitung negativ und wir haben Gleichgewicht zwischen der abw\"{a}rts gerichteten Fl\"{a}chenkraft $p\,\downarrow$ und den aufw\"{a}rts gerichteten Haltekr\"{a}ften $\partial u/\partial n\,\uparrow$ am Rand
\begin{align}
\text{\normalfont\calligra G\,\,}(u, 1) &= \int_{\Omega} p \cdot 1\,d\Omega + \int_{\Gamma} \nabla u \dotprod \vek n\,\cdot 1 \,ds = 0\,.
\end{align}
Wenn wir die Membran in ihrer Mitte $\vek x = \vek 0$ mit einer Einzelkraft $P=1$ belasten, s. Abb. \ref{U240}, dann bildet sich ein Trichter aus, \begin{align} \label{Eq103}
G(\vek y,\vek x) = -\frac{1}{2\,\pi}\,\ln\,r \qquad r = |\vek y - \vek x|\,,
\end{align}
der in der Tiefe bis zum Punkt $\infty$ reicht, d.h. die Membran kann die Einzelkraft nicht festhalten.

Nun wollen wir den {\em Energieerhaltungssatz\/} in diesem Lastfall formulieren, also die Gleichung
\begin{align}
\frac{1}{2}\, \text{\normalfont\calligra G\,\,}(G, G) &= A_a - A_i = 0\,,
\end{align}
anschreiben, aber ohne den Faktor $1/2$, weil er f\"{u}r die Argumentation nicht wesentlich ist.

Der Gradient der schlauchartigen Biegefl\"{a}che $G$ verh\"{a}lt sich wie $1/r$
\begin{align}
\nabla G = \frac{1}{2\,\pi}\,\frac{1}{r} \left[ \barr {c }
      \cos \Np  \\
      \sin \Np
     \earr \right]
\end{align}
und wegen
\begin{align}
\nabla G \dotprod  \nabla G = \frac{1}{4\,\pi^2}\frac{1}{r^2} (\cos^2 \Np + \sin^2 \Np) = \frac{1}{4\,\pi^2}\frac{1}{r^2}
\end{align}
ist daher die innere Energie unendlich gro{\ss},
\begin{align}
A_i = \int_{\Omega} \nabla G \dotprod \nabla G \,d\Omega = \int_0^{\,2\,\pi} \int_0^{\,1} \frac{1}{4\,\pi^2}\frac{1}{r^2}\,r\,dr\,d\Np = \infty\,,
\end{align}
denn das Integral
\begin{align}
\int_0^{\,1} \frac{1}{r}\,dr = \infty
\end{align}
ist unbeschr\"{a}nkt. Dazu passend ist auch die \"{a}u{\ss}ere Arbeit unendlich gro{\ss}
\begin{align}
A_a = P \cdot \infty\,,
\end{align}
weil $P$ unendlich tief absinkt. In sich ist das Resultat zwar stimmig
\begin{align}
A_a - A_i = \infty - \infty = 0\,,
\end{align}
aber mit unendlich kann man leider nicht rechnen, unendlich ist einfach nur \glq unz\"{a}hlbar viel\grq{}.

Der Grund f\"{u}r die unendliche Energie ist, dass wir zur Formulierung von $A_i$ auf die Diagonale
der ersten Greenschen Identit\"{a}t
\begin{align}
\frac{1}{2}\, \text{\normalfont\calligra G\,\,}(G, G) &= A_a - A_i = 0
\end{align}
gehen m\"{u}ssen und sich so die Singularit\"{a}t im Integral verdoppelt. Aus $1/r$ wird $1/r^2$ und das ist nicht mehr integrierbar.

Dies wiederholt sich in der Elastizit\"{a}tstheorie. Wenn man eine Scheibe, die auch wieder kreisf\"{o}rmig sei, $R = 1$, mit einer Einzelkraft belastet, dann verhalten sich die Dehnungen und Spannungen wie $1/r$ und durch das Verdoppeln der Singularit\"{a}t auf der Diagonalen wird $A_i$ unendlich gro{\ss},
\begin{align}
A_i = \int_{\Omega} \sigma_{ij}\,\varepsilon_{ij}\,\,d\Omega \simeq \int_0^{\,2\,\pi} \int_0^{\,1} \frac{1}{r^2}\,r\,dr\,d\Np\,.
\end{align}
Bei dreidimensionalen Problemen verhalten sich die $\sigma_{ij}$ und $\varepsilon_{ij}$ wie $1/r^2$ und das Volumenelement $d\Omega = r^2\,dr\,d\Np\,\sin\,\theta\,d\theta$ kann der verdoppelten Singularit\"{a}t $1/r^4$ nicht paroli bieten, d.h. die innere Energie ist dann ebenfalls unendlich gro{\ss}.

Belastet man dagegen eine Platte (Kirchhoff) mit einer Einzelkraft, dann hat die Biegefl\"{a}che die Gestalt ($c $ ist eine Konstante)
\begin{align}
w(\vek x) = c \cdot \frac{1}{8\,\pi\,K}\,r^2\,\ln\,r + \text{regul\"{a}re Terme}\,.
\end{align}
und die Momente $m_{ij}$ bzw. Kr\"{u}mmungen $\kappa_{ij}$ gehen daher \glq nur\grq{} wie
\begin{align}
m_{ij} \sim \ln\,r
\end{align}
gegen Unendlich und so ist die innere Energie in einer kreisf\"{o}rmigen Platte, $R = 1$,
\begin{align}
A_i = \int_{\Omega} m_{ij}\,\kappa_{ij} \,d\Omega  \sim \int_0^{\,2\,\pi} \int_0^{\,1} \ln^2\,r\,\underbrace{r\,dr\,d\Np}_{d \Omega} + \,\text{endliches Integral}
\end{align}
beschr\"{a}nkt, ist $A_i$ endlich, weil der Integrand in der Grenze gegen null geht, $r \to 0$ gewinnt gegen\"{u}ber $\ln^2 r \to \infty$,
\begin{align}
\lim_{r \to 0}\,r\,\ln^2\,r = 0\,.
\end{align}
Wie bei der Scheibe verursacht die Punktlast $P = 1$ eine $1/r$ Singularit\"{a}t, hier des Kirchhoffschubs $v_n$ (der \glq dritten\grq{} Ableitung),
\begin{align}
\lim_{\varepsilon \to 0}\,\int_{\Gamma_{N_\varepsilon}} \frac{1}{2\,\pi\,r}\,ds_{\vek y} = 1\,,
\end{align}
weil aber die Plattengleichung von vierter Ordnung ist, braucht es drei Schritte von $v_n$ zu $w$
\begin{align}
w = \int \int \int v_n (d\Omega)^3 \simeq r^2\,\ln\,r\qquad\text{(drei Schritte)}
\end{align}
und das reicht, um die Singularit\"{a}t zu d\"{a}mpfen. Bei Problemen zweiter Ordnung, wie der Scheibe, trennt nur eine Integrationsstufe $\sigma \simeq 1/r$ von $u$ und das ist nicht genug
\begin{align}
u \simeq \int \frac{1}{r}\,dr = \ln\,r \qquad\text{(ein Schritt)}\,.
\end{align}
%%%%%%%%%%%%%%%%%%%%%%%%%%%%%%%%%%%%%%%%%%%%%%%%%%%%%%%%%%%%%%%%%%%%%%%%%%%%%%%%%%%%%%%%%%%%%%%%%%%
{\textcolor{sectionTitleBlue}{\section{Sobolevscher Einbettungssatz}}}
In dieses scheinbare Durcheinander von endlicher und unendlicher Energie kann man nun mit dem {\em Sobolevschen Einbettungssatz\/}\index{Sobolevscher Einbettungssatz} eine gewisse Systematik hineinbringen. Er erlaubt genaue Voraussagen, wann die innere Energie endlich ist und wann nicht, \cite{Ha6}.

Eine Greensche Funktion hat eine endliche Energie, wenn die Ungleichung
\begin{align}\label{Eq23}
\boxed{m - i > \frac{n}{2}}
\end{align}
erf\"{u}llt ist\footnote{Die Tabelle auf S. \pageref{TabelleSobolev} enth\"{a}lt eine systematische Auswertung der Ungleichung im Zusammenhang mit den Gleichungen der Statik}. Hier ist $2\,m$ die Ordnung des Differentialoperators, $i$ ist die Ordnung der Singularit\"{a}t, die in unserer Notation identisch ist mit dem Index  $i$ an dem Dirac Delta $\delta_i$, und $n$ ist die Dimension des Raums. Bei Scheiben ist $n = 2$, das Differentialgleichungssystem hat die Ordnung $2 m = 2$ und daher hat das Verschiebungsfeld, das von einer Einzelkraft, $i = 0$, erzeugt wird unendliche Energie, weil die Ungleichung
\begin{align}
1 - 0 > \frac{2}{2} = 1\,\,\,?
\end{align}
nicht gilt, w\"{a}hrend sie im Fall der Kirchhoffplatte, $2m = 4$, erf\"{u}llt ist
\begin{align}
2 - 0 > \frac{2}{2} = 1\,.
\end{align}
Wir k\"{o}nnen diese Ergebnisse wie folgt zusammenfassen: \\

\hspace*{-12pt}\colorbox{highlightBlue}{\parbox{0.98\textwidth}{Die innere Energie $A_i$ ist unendlich, wenn die \"{a}u{\ss}ere Arbeit $A_a$ unendlich ist und das ist genau dann der Fall, wenn in dem Ausdruck
\begin{align}
A_a = \text{{\em Kraft\/}}\, \times\, \text{{\em Weg\/}}
\end{align}
einer der beiden Terme unendlich gro{\ss} ist.}}\\

Bei der Membran ist die Kraft zwar endlich, $P = 1$, aber der Weg, den die Kraft geht, die Durchbiegung im Aufpunkt, ist $\infty$. Ebenso ist es bei der Scheibe. Bei der Platte ist hingegen die Durchbiegung, die die Kraft $P = 1$ erzeugt, endlich und somit auch die Arbeit $A_a = \text{{\em Kraft\/}} \times \text{{\em Weg\/}} < \infty$.

Die Energie ist immer unendlich, wenn das Material \"{u}ber die Flie{\ss}grenze hinaus deformiert wird, wie das zur Erzeugung von Einflussfunktionen f\"{u}r Kraftgr\"{o}{\ss}en n\"{o}tig ist. Die Einflussfunktion f\"{u}r eine Normalkraft $N$ in einem Stab entsteht durch ein Verschiebungssprung, man muss den Stab also buchst\"{a}blich zerrei{\ss}en, und die Einflussfunktion f\"{u}r ein Moment $M$ verlangt einen Knick, einen pl\"{o}tzlichen Richtungswechsel der Tangente im Aufpunkt und das geht nur, wenn man das Material vorher zum Flie{\ss}en bringt.

Bei Fl\"{a}chentragwerken haben eigentlich alle Einflussfunktionen, auch die f\"{u}r Weggr\"{o}{\ss}en, unendlich gro{\ss}e Energie. Die Ausnahme ist die Einflussfunktion f\"{u}r die Durchbiegung $w(\vek x)$ einer Kirchhoffplatte\index{Kirchhoffplatte}\footnote{Die Kirchhoffplatte, auch schubstarre Platte genannt, ist die Erweiterung des Biegebalkens $EI\,w^{IV}$ auf zwei Dimensionen. Im Unterschied hierzu ist die Mindlin-Reissner Platte\index{Reissner-Mindlin Platte} eine schubweiche Platte, s. Kapitel 7. Im Regelfall meint der Ingenieur die Kirchhoffplatte, wenn er von Platten spricht. }.

Nun kann man fragen: \glq Wenn die Energie der Einflussfunktionen unendlich ist, wieso kann man dann mit ihnen rechnen?' Der Unterschied ist, dass man bei der Anwendung des {\em Prinzips der virtuellen Verr\"{u}ckungen\/} oder des {\em Prinzips der virtuellen Kr\"{a}fte\/} auf der { Nebendiagonale} ist
\begin{align}
\text{\normalfont\calligra G\,\,}(G,u) &= \delta A_a - \delta A_i = 0\,,
\end{align}
und sich daher die Singularit\"{a}t nicht verdoppelt, die virtuelle innere Energie $\delta A_i$ bleibt endlich
\begin{align}
\delta A_a = 1 \cdot u(\vek x) = \int_{\Omega}  \nabla G \dotprod  \nabla u \,d\Omega = \delta A_i\,,
\end{align}
weil das $1/r$ des Gradienten $\nabla G$ durch das $r$ in dem Fl\"{a}chenelement $d\Omega = r\,dr\,d\Np$ ausgeglichen wird. Genau genommen m\"{u}ssen wir auch noch fordern, dass der Gradient von $u$ beschr\"{a}nkt ist, $|\nabla u| \leq \infty$.

Der {\em Satz von Betti\/} ist von den Singularit\"{a}ten der Einflussfunktionen auch betroffen, aber weil man am Schluss nur das Endergebnis sieht
\begin{align}
\lim_{\varepsilon \to 0} \text{\normalfont\calligra B\,\,}(G,u)_{\Omega_\varepsilon} = u(\vek x) -\int_{\Omega} G(\vek y,\vek x)\,p(\vek y)\,d\Omega_{\vek y} = 0\,,
\end{align}
sieht alles glatt aus. Das, was singul\"{a}r war, hat zu dem Term $u(\vek x)$ gef\"{u}hrt und der Rest sind alles Integrale, deren Berechnung man dem Computer \"{u}berlassen kann (numerische Quadratur).\\


%----------------------------------------------------------------------------------------------------------
\begin{figure}[tbp]
\centering
\if \bild 2 \sidecaption \fi
\includegraphics[width=0.6\textwidth]{\Fpath/U314}
\caption{Balkenbiegelinien,  \textbf{ a)} mit unendlich gro{\ss}er Energie, \textbf{ b)} mit endlicher Energie, hier sogar null}
\label{U314}
\end{figure}%
%----------------------------------------------------------------------------------------------------------

\begin{remark}
Die charakteristischen Singularit\"{a}ten der verschiedenen Einflussfunktionen $G_i$ f\"{u}r  {\em solids\/}, Scheiben und Platten, bis hinunter zu den Einflussfunktionen f\"{u}r Momente und Querkr\"{a}fte, findet der interessierte Leser in \cite{Ha2} und \cite{Ha3}. Dort werden auch die Grenzprozesse
\begin{align}
\lim_{\varepsilon \to 0} \text{\normalfont\calligra G\,\,}(G_i,u)_{\Omega_\varepsilon} = 0 \qquad \lim_{\varepsilon \to 0} \text{\normalfont\calligra B\,\,}(G_i,u)_{\Omega_\varepsilon} = 0\,,
\end{align}
die ja den Einflussfunktionen zu Grunde liegen, detailliert diskutiert.
\end{remark}

%----------------------------------------------------------------------------------------------------------
\begin{figure}[tbp]
\centering
\if \bild 2 \sidecaption \fi
\includegraphics[width=0.6\textwidth]{\Fpath/UE358}
\caption{Definition der Winkel $\Np_l$ und $\Np_r$}
\label{UE358}
\end{figure}%
%----------------------------------------------------------------------------------------------------------


\begin{remark} {\em Ingenieur versus Mathematiker\/}\label{Fourierreihe}
Ein Mathematiker wird darauf hinweisen, dass unendlich viel Energie n\"{o}tig ist, um einen Knick in einem Balken zu erzeugen, und zur Bekr\"{a}ftigung seiner Behauptung wird er die Biegelinie $w$, s. Abb. \ref{U314} a, des Balkens in eine Fourier-Reihe entwickeln
\begin{align}
w(x)= \frac{\pi}{2} - \frac{4}{\pi}(\cos x + \frac{1}{3^2}\,\cos 3x + \frac{1}{5^2}\,\cos 5x + \ldots )
\end{align}
und dann die Energie berechnen
\begin{align}
\frac{1}{2} \cdot  EI\int_0^{\,2\,\pi} (w''(x))^2\,dx = \frac{1}{2}\cdot \frac{16}{\pi} (1 + 1 + 1 \ldots ) = \infty\,.
\end{align}
Aber ein Ingenieur geht anders vor: Er installiert ein Gelenk und verdreht die beiden Seiten des Gelenkes so, dass $\tan \Np_l + \tan \Np_r = 1$. Die Biegeenergie in dem Balken ist dann einfach die Energie, die notwendig ist, um die Balkenenden zu verdrehen und diese Energie ist endlich. Im Falle des statisch bestimmten Balkens in Abb. \ref{U314} b ist sie sogar null, denn $w'' = 0$.
\end{remark}

\begin{remark}
In Abb. \ref{UE358} haben wir angetragen, wie wir die Winkel $\Np_l$ und $\Np_r$ z\"{a}hlen. Die Arbeit, die die beiden Momente, links und rechts vom Gelenk, leisten, ist
\begin{align}
\text{\normalfont\calligra G\,\,}(w, \textcolor{red}{\delta w}) &= \text{\normalfont\calligra G\,\,}(w_l, \textcolor{red}{\delta w})_{(0,{x})}+  \text{\normalfont\calligra G\,\,}(w_r, \textcolor{red}{\delta w})_{({x},l)} \nn\\
&= \ldots - M_l \cdot \delta w'_l + M_r \cdot \delta w'_r + \ldots = 0\,.
\end{align}
Am besten w\"{a}re es, es bei dieser Notation zu belassen, aber als Ingenieure wollen wir mit Winkeln rechnen und nicht mit \glq abstrakten\grq\ Steigungen $\delta w'$. Mit den positiven Richtungen in Abb.  \ref{UE358} geht dieser Ausdruck \"{u}ber in
\begin{align}
- M_l \cdot \delta w'_l + M_r \cdot \delta w'_r &= - (M_l \cdot \tan\,\Np_l + M_r\,\tan\,\Np_r)\nn \\
&= - M \,(\tan\,\Np_l + \tan\,\Np_r)\,,
\end{align}
weil $\delta w_r' = - \tan\,\Np_r$. Wenn wir diesen Term auf die linke Seite bringen, dann verschwindet das Minuszeichen
\begin{align}
M \cdot (\tan\,\Np_l + \tan\,\Np_r) = M \cdot 1 = \ldots
\end{align}
In den B\"{u}chern wird das meist geschrieben als
\begin{align}
M \cdot \Delta\,\Np = M \cdot 1 = \ldots\,,
\end{align}
weil f\"{u}r kleine  Winkel $\tan \Np \simeq \Np$, aber einige Autoren gehen so weit $\Delta \Np$ mit der Dimension {\em rad\/} oder {\em degree\/} zu schreiben, was nicht korrekt ist. $M \cdot \Delta \Np$ ist eine \"{a}u{\ss}ere Arbeit. $ \Delta \Np$ steht f\"{u}r $\tan\,\Np_l + \tan\,\Np_r$. Wenn man ein Balkenende um $45^\circ$ verdreht, dann leistet das Moment dabei die Arbeit
\begin{align}
M \cdot \tan\,45^\circ = M\, [\text{kNm}] \cdot 1 = M  \,[\text{kNm}]
\end{align}
und nicht die Arbeit $M \cdot 45^\circ$
\begin{align}
M \cdot 45^\circ =  M\, [\text{kNm}] \cdot 45^\circ [\text{rad}] \qquad \text{(?)}
\end{align}
Der letzte Ausdruck hat nicht die Dimension einer Arbeit.
\end{remark}

\begin{remark}
F\"{u}r weitere, erg\"{a}nzende Angaben zu dem Sobolevschen Einbettungssatz siehe Kapitel 7, S. \pageref{Eq71}.
\end{remark}

%%%%%%%%%%%%%%%%%%%%%%%%%%%%%%%%%%%%%%%%%%%%%%%%%%%%%%%%%%%%%%%%%%%%%%%%%%%%%%%%%%%%%%%%%%%%%%%%%%%
\textcolor{sectionTitleBlue}{\section{Der Reduktionssatz}}\label{RedSatz}
Der {\em Reduktionssatz\/}\index{Reduktionssatz} ist eine spezielle Variante der Mohrschen Arbeitsgleichung.
%----------------------------------------------------------
\begin{figure}[tbp]
\centering
\if \bild 2 \sidecaption[t] \fi
\includegraphics[width=1.0\textwidth]{\Fpath/UE363}
\caption{Anwendung des Reduktionssatzes, \textbf{ a)} und \textbf{ b)} Momentenverlauf und Mohrsche Formulierung, \textbf{ c)} auf unterschiedlichen Pfaden ist das Ergebnis dasselbe, \textbf{ d)}  Reduktionssatz} \label{UE363}
\end{figure}%%
%----------------------------------------------------------
Zur Berechnung der horizontalen Verschiebung $u_i$ des Rahmens in Abb. \ref{UE363} w\"{u}rde Mohr eine Kraft $X_1 = 1$ in Richtung der gesuchten Verschiebung $u_i$ wirken lassen, und das Integral
\begin{align}
1 \cdot u_i = \sum_e \int_0^{\,l_e} (\frac{M\,M_1}{EI} + \frac{N\,N_1}{EA})\,dx
\end{align}
auswerten, und dabei \"{u}ber alle St\"{a}be des Rahmens integrieren.

Gem\"{a}{\ss} dem Reduktionssatz reicht es jedoch aus, die Einzelkraft $X_1 = 1$ an einem statisch bestimmten Teilsystem des urspr\"{u}nglichen Tragwerkes wirken zu lassen.

Wir verstehen das besser, wenn wir uns klarmachen, dass der Knoten, der die Last tr\"{a}gt von vier verschiedenen Startpunkten aus angesteuert werden kann, s. Abb. \ref{UE363} c, und dass auf jedem Pfad die Summe der horizontalen Verschiebungen $u_i$ sein muss.
%----------------------------------------------------------
\begin{figure}[tbp]
\centering
\if \bild 2 \sidecaption[t] \fi
\includegraphics[width=1.0\textwidth]{\Fpath/UE333}
\caption{Kopplung zweier Spannungszust\"{a}nde auf einem Teilnetz} \label{UE333}
\end{figure}%%
%----------------------------------------------------------

Wir k\"{o}nnen also $u_i$ berechnen, indem wir zum Beispiel einfach nur \"{u}ber den Pfosten in Abb. \ref{UE363} d integrieren
\begin{align}\label{Eq158}
1 \cdot u_i = \int_0^{\,l} (\frac{M\,M_1}{EI} + \frac{N\,(N_1 = 0)}{EA})\,dx\,.
\end{align}


Der Reduktionssatz sagt im wesentlichen, dass das Dirac Delta, das $X_1$, nicht auf den urspr\"{u}nglichen Rahmen aufgebracht werden muss, sondern dass es irgendein { Teilsystem} sein kann, das in dem urspr\"{u}nglichen System \glq enthalten\grq{} ist, s. Abb. \ref{UE363} d. Enthalten bedeutet, dass beim \"{U}bergang zum Teilsystem die Festhaltungen von Knoten gel\"{o}st werden k\"{o}nnen, aber das keine zus\"{a}tzlichen Festhaltungen eingebaut werden d\"{u}rfen. In der Sprache der Mathematik bedeutet dies, dass der { Dirichlet Rand} (die Lagerknoten) schrumpfen kann, aber dass er nicht wachsen darf, \cite{Ha6} p. 149. Die zul\"{a}ssigen Teilsysteme sind \"{u}blicherweise gerade die Teilsysteme, die man auch beim Kraftgr\"{o}{\ss}enverfahren w\"{a}hlen w\"{u}rde, wie der einzelne Pfosten.

Der Reduktionssatz ist eine geschickte Anwendung der ersten Greenschen Identit\"{a}t, denn weil diese Identit\"{a}t f\"{u}r jedes einzelne Stabelement null ist
\begin{align}
\sum_e \text{\normalfont\calligra G\,\,}(u,u_1)_{e} = 0\,,
\end{align}
ist sie auch f\"{u}r jedes Teilsystem null. Der \glq Trick\grq{} besteht nun darin, Teilsysteme zu w\"{a}hlen, die sich leichter analysieren lassen, weil sie statisch bestimmt sind.
%----------------------------------------------------------
\begin{figure}[tbp]
\centering
\if \bild 2 \sidecaption[t] \fi
\includegraphics[width=1.0\textwidth]{\Fpath/U405}
\caption{Virtuelle Verr\"{u}ckung und Lagersenkung} \label{U405}
\end{figure}%%
%----------------------------------------------------------

Nur muss man eben Systeme vermeiden, die bei der Formulierung der Identit\"{a}ten $\text{\normalfont\calligra G\,\,}(u, u_1) = 0$, nach unbekannten Knotenverschiebungen oder Knotenkr\"{a}ften fragen. Das ist die Essenz der obigen \glq Dirichlet Bedingung\grq{}\index{Dirichlet Bedingung}. Wenn man sich aber von dem Kraftgr\"{o}{\ss}enverfahren leiten l\"{a}sst, dann kommt man nicht in diese Verlegenheit.
%----------------------------------------------------------
\begin{figure}[tbp]
\centering
\if \bild 2 \sidecaption[t] \fi
\includegraphics[width=0.9\textwidth]{\Fpath/U374X8}
\caption{Kraftgr\"{o}{\ss}enverfahren} \label{U374}
\end{figure}%%
%----------------------------------------------------------

Die Idee der Teilsysteme l\"{a}sst sich nat\"{u}rlich auf jedes FE-Netz anwenden. Auf jedem Teilnetz kann man unterschiedliche Spannungszust\"{a}nde via der Greenschen Identit\"{a}t in einer Null-Summe koppeln, s.  Abb. \ref{UE333}
\begin{align}
\text{\normalfont\calligra G\,\,}(\vek u,\vek  \delta \vek u) = 0\,.
\end{align}
Mit dem Reduktionssatz kann man auch leicht zeigen, dass Biegelinien $w$ aus Lagersenkung orthogonal sind zu den virtuellen Verr\"{u}ckungen $\delta w$ des Systems, s. Abb. \ref{U405}. Die Absenkung kann man sich als Reaktion des Tr\"{a}gers ohne Zwischenlager (= Hauptsystem) auf eine Kraft $X_1 = 1 \cdot P$ vorstellen (auf den Faktor $P$ kommt es nicht an) und gem\"{a}{\ss} Reduktionssatz muss gelten
\begin{align}
\delta A_i(w, \delta w) = \int_0^{\,l} \frac{M\,\delta M}{EI}\,dx = 0 \qquad M = P \cdot M_1\,,
\end{align}
weil dieser Ausdruck gerade das $P$-fache der Durchbiegung der virtuellen Verr\"{u}ckung $\delta w$ im Zwischenlager ist, aber $\delta w$ ist null, s. Abb. \ref{U405} a.

%%%%%%%%%%%%%%%%%%%%%%%%%%%%%%%%%%%%%%%%%%%%%%%%%%%%%%%%%%%%%%%%%%%%%%%%%%%%%%%%%%%%%%%%%%%%%%%%%%%
{\textcolor{sectionTitleBlue}{\section{Das Kraftgr\"{o}{\ss}enverfahren}}}\index{Kraftgr\"{o}{\ss}enverfahren}
Beim Kraftgr\"{o}{\ss}enverfahren macht man ein Tragwerk durch den Einbau von $M$-, $V$- oder $N$-Gelenken statisch bestimmt, s. Abb. \ref{U374}.

Die statisch \"{U}berz\"{a}hligen $X_i$ werden so bestimmt, dass die Klaffungen in den Gelenken null sind
\begin{align}
\left[ \barr {r @{\hspace{4mm}}r} \delta_{11} & \delta_{12} \\ \delta_{21} & \delta_{22} \earr \right] \left[ \barr {r } X_1 \\ X_2 \earr \right] -  \left[ \barr {r } \delta_{10} \\ \delta_{20} \earr \right] = \left[ \barr {r } 0 \\ 0 \earr \right] \,.
\end{align}
Die $\delta_{ij}$ sind die Relativverdrehungen/Verschiebungen zwischen den $X_i$ (links und rechts vom Gelenk) und die $\delta_{i0}$ sind dieselben Gr\"{o}{\ss}en aus der Belastung. Ihre Berechnung beruht auf der Mohrschen Arbeitsgleichung
\begin{align}
\delta_{ij} = \int_0^{\,l} \frac{M_i\,M_j}{EI}\,dx\,,
\end{align}
die ja wiederum mit der ersten Greenschen Identit\"{a}t identisch ist
\begin{align}
\text{\normalfont\calligra G\,\,}(w_i,w_j) = \delta_{ij} - a(w_i,w_j) = \delta_{ij}  - \int_0^{\,l}\frac{M_i\,M_j}{EI}\,dx = 0\,.
\end{align}
Die Biegelinien $w_i$ und $w_j$ sind die Biegelinien, die zu $X_i$ und $X_j$ geh\"{o}ren (sie werden zum Gl\"{u}ck nicht gebraucht -- nur ihre Momente) und der einzelne Term
\begin{align}
\delta_{ij} = [V_i\,w_j - M_i\,w'_j]_0^l - [w_i,\,V_j - w_i'\,M_j]_0^l
\end{align}
ist sozusagen der \glq Rest\grq{}, das was von den eckigen Klammern, den Randarbeiten, in der Summe \"{u}ber alle St\"{a}be \"{u}brig bleibt. Der Rest $\delta_{ij} = Spreizung \cdot 1$ hat immer die Dimension einer Arbeit.

Zur Berechnung der $\delta_{i0}$ ersetzt man in den obigen Formeln das zweite Momente $M_j$ durch das Moment $M_0$ am statisch bestimmten Hauptsystem aus der Belastung.

%%%%%%%%%%%%%%%%%%%%%%%%%%%%%%%%%%%%%%%%%%%%%%%%%%%%%%%%%%%%%%%%%%%%%%%%%%%%%%%%%%%%%%%%%%%%%%%%%%%
\textcolor{sectionTitleBlue}{\section{Wo l\"{a}uft es hin?}}
Wir haben oben von dem Null-Summen-Spiel der ersten Greenschen Identit\"{a}t gesprochen. Dieser Begriff hat auch eine direkte statische Relevanz, wie die beiden folgenden Beispielen erl\"{a}utern sollen.

Ein fester Punkt reicht Archimedes nicht aus, um die Welt aus den Angeln zu heben. Er ben\"{o}tigt auch einen Hebel mit einer unendlich gro{\ss}en Biegesteifigkeit $EI = \infty$, s. Abb. \ref{Hebel2A} a, denn sonst geht seine ganze Kraft nur in die Verkr\"{u}mmung des Hebels.
%-----------------------------------------------------------------
\begin{figure}[tbp]
\centering
\if \bild 2 \sidecaption \fi
\includegraphics[width=.9\textwidth]{\Fpath/HEBEL2A}
\caption{Archimedes' Dilemma: Aller Aufwand flie{\ss}t in die Biegeenergie {\bf a)} die Erde wird sich keinen Millimeter bewegen {\bf b)} und das Gummiband wird lang und l\"{a}nger...}\label{Hebel2A}
\end{figure}%
%-----------------------------------------------------------------

Wegen $\text{\normalfont\calligra G\,\,}(w,w) = A_a - A_i = 0$ sind zu jedem Zeitpunkt die \"{a}u{\ss}ere Arbeit und die Biegeenergie in dem Balken gleich gro{\ss} (wir lassen den Faktor $1/2$ weg)
\bfo
 A_a = P_r \cdot w_r - P_l \cdot w_l = a(w,w) = \int_0^{\,l} \frac{M^2}{EI}\,dx = A_i
\efo
oder aufgel\"{o}st nach dem angestrebten Effekt
\bfo
P_l \cdot w_l = a(w,w) - P_r \cdot w_r \simeq 0\,,
\efo
der aber praktisch null ist, weil der ganze Aufwand von Archimedes, $P_r \cdot w_r$, in die Verkr\"{u}mmung des Balkens flie{\ss}t, also in die Biegeenergie $a(w,w)$, und praktisch nichts auf der linken Seite der Gleichung ankommt, nichts \"{u}brigbleibt, um die Erde anzuheben.

Dieselbe Situation liegt vor, wenn man ein schweres Gewicht \"{u}ber den nassen Sand am Strand zieht, s. Abb. \ref{Hebel2A} b, \cite{Ha5},
\bfo
A_a = \underbrace{P_r \cdot u_r}_{Aufwand} - P_l \cdot u_l = a(u,u) =
\int_0^{\,l} \frac{N^2}{EA}\,dx = A_i\,.
\efo
Das Gummiband ($EA$) wird sich l\"{a}ngen, die Verzerrungsenergie $a(u,u)$ wird immer weiter anwachsen, aber das Gewicht wird sich kaum bewegen, $u_l \simeq 0$,
\begin{align}
P_l \cdot u_l = a(u,u) - P_r \cdot u_r \simeq 0\,.
\end{align}
Das sind Situationen, die dem Steckenbleiben im Treibsand \"{a}hneln, wo man sich mit eigener Kraft nicht aus dem Sand ziehen  kann. Hier ist es die erste Greensche Identit\"{a}t, das Null-Summen-Spiel, die das verhindert, die es dem Balken, bzw. dem Seil, erlaubt auszuweichen. Der Anwender hat keine Kontrolle dar\"{u}ber, wo sein Effort, sein Input, seine Energie hinflie{\ss}t. Das ist, wenn man so will,  die \glq Unsch\"{a}rferelation der Statik\grq{}\index{Unsch\"{a}rferelation der Statik}.

Der Anwender wei{\ss} das erst, wir wechseln jetzt zu einem kompletten Tragwerk, wenn er das System $\vek K\,\vek u = \vek f$ gel\"{o}st hat, denn dann ist klar, wie sich die Knoten verformen, um wieviel sich die Elementenden verschieben, $\vek u_e$, und dann kann er f\"{u}r jedes einzelne Element die Energie berechnen. %Wie das geht, erl\"{a}utern wir im n\"{a}chsten Abschnitt.



%%%%%%%%%%%%%%%%%%%%%%%%%%%%%%%%%%%%%%%%%%%%%%%%%%%%%%%%%%%%%%%%%%%%%%%%%%%%%%%%%%%%%%%%%%%%%%%%%%%
\textcolor{sectionTitleBlue}{\section{Finite Elemente und die erste Greensche Identit\"{a}t}}
Wir wollen zum Schluss dieses Kapitels noch vor einem m\"{o}glichen Missverst\"{a}ndnis warnen. Ist $u$ die L\"{a}ngsverschiebung eines links festgehaltenen Stabes mit einem freien Ende
\begin{align}
- EA u'' = p \qquad u(0) = 0\,, \quad  N(l) = 0
\end{align}
und $\delta u$ eine virtuelle Verr\"{u}ckung, $\delta u(0) = 0$, dann ist die Gleichung
\begin{align}
\text{\normalfont\calligra G\,\,}(u,\delta u) = (p, \delta u)  - a(u, \delta u) = 0
\end{align}
die Vorlage zur Bestimmung der FE-L\"{o}sung $u_h(x) = \sum_j u_j\,\Np_j(x)$
\begin{align}
(p, \Np_i)  - a(u_h, \Np_i)= 0 \qquad i = 1,2,\ldots, n\,.
\end{align}
Man ist versucht, das mit
\begin{align}
\text{\normalfont\calligra G\,\,}(u_h,\Np_i) &= (p_h,\Np_i) - a(u_h,\Np_i)= 0
\end{align}
gleichzusetzen, also die FEM auf die bequeme Formulierung
\begin{align}
\text{\normalfont\calligra G\,\,}(u_h,\Np_i) &= 0\qquad i = 1,2,\ldots, n
\end{align}
zu reduzieren, aber $p_h = - EA\,u_h''$ ist nicht $p$ (bei linearen Elementen best\"{u}nde $p_h$ aus lauter Knotenkr\"{a}ften $f_i$)
\begin{align}
\text{FEM} = (p,\Np_i) - a(u_h,\Np_i) \neq (p_h,\Np_i) - a(u_h,\Np_i) = \text{\normalfont\calligra G\,\,}(u_h,\Np_i)\,.
\end{align}
%Versuchsweise k\"{o}nnte man die FE-Gleichungen als
%\begin{align}
% \text{\normalfont\calligra G\,\,}^{\,ex}(u_h,\Np_i) = (p,\Np_i) - a(u_h,\Np_i) = 0
%\end{align}
%mit $ex = exchange$ schreiben, d.h. ersetze in $\text{\normalfont\calligra G\,\,}(u_h,\Np_i)$ alle %Kraftgr\"{o}{\ss}en von $u_h$ durch die Kraftgr\"{o}{\ss}en der exakten L\"{o}sung.



%%%%%%%%%%%%%%%%%%%%%%%%%%%%%%%%%%%%%%%%%%%%%%%%%%%%%%%%%%%%%%%%%%%%%%%%%%%%%%%%%%%%%%%%%%%%%%%%%%%
\textcolor{chapterTitleBlue}{\chapter{Der Satz von Betti}}}\index{Satz von Betti}
Das Thema dieses Kapitels ist die Berechnung von Einflussfunktionen mit dem Satz von Betti.


%%%%%%%%%%%%%%%%%%%%%%%%%%%%%%%%%%%%%%%%%%%%%%%%%%%%%%%%%%%%%%%%%%%%%%%%%%%%%%%%%%%%%%%%%%%%%%%%%%%
{\textcolor{sectionTitleBlue}{\section{Grundlagen}}}
Der {\em Satz von Betti\/} besagt, dass die reziproken \"{a}u{\ss}eren Arbeiten zweier Systeme, die jedes f\"{u}r sich im Gleichgewicht ist, gleich gro{\ss} sind
\begin{align}
A_{1,2} = A_{2,1}\,.
\end{align}
{\em Die Arbeiten, die die Lasten des Systems 1 auf den Wegen des Systems 2 leisten, $A_{1,2} $, sind genauso gro{\ss} wie die Arbeiten, die die Lasten des Systems 2 auf den Wegen des Systems 1 leisten, $A_{2,1} $\/}.

Dieser Satz beruht auf der zweiten Greenschen Identit\"{a}t $\text{\normalfont\calligra B\,\,}(w,\hat{w})$. Sie erh\"{a}lt man durch Spiegelung der ersten Greenschen Identit\"{a}t an sich selbst und Vertauschung der Reihenfolge von $w$ und $\hat{w}$
\begin{align}
\text{\normalfont\calligra B\,\,}(w,\textcolor{chapterTitleBlue}{\hat{w}}) = \text{\normalfont\calligra G\,\,}(w,\textcolor{chapterTitleBlue}{\hat{w}}) - \text{\normalfont\calligra G\,\,}(\textcolor{chapterTitleBlue}{\hat{w}}, w) = 0 - 0 = 0\,,
\end{align}
was im Falle des Balkens zu dem Ergebnis
\begin{align}\label{Eq57}
\text{\normalfont\calligra B\,\,}(w_1,\textcolor{chapterTitleBlue}{w_2})= &\underbrace{\int_0^{\,l} EI\,w_1^{IV}\,\textcolor{chapterTitleBlue}{w_2}\,dx + [V_1\,\textcolor{chapterTitleBlue}{w_2} - M_1\,\textcolor{chapterTitleBlue}{w_2}']_{@0}^{@l}}_{A_{1,2}} \nn \\
&- \underbrace{[\textcolor{chapterTitleBlue}{V_2}\,w_1 - \textcolor{chapterTitleBlue}{M_2}\,w_1']_{@0}^{@l} - \int_0^{\,l} w_1\,\textcolor{chapterTitleBlue}{EI\,w_2^{IV}}\,dx}_{A_{2,1}} = 0
\end{align}
f\"{u}hrt.
%----------------------------------------------------------------------------------------------------------
\begin{figure}[tbp]
\centering
\if \bild 2 \sidecaption \fi
\includegraphics[width=1.0\textwidth]{\Fpath/U210}
\caption{{\em Satz von Betti\/}} \label{U210}
%
\end{figure}%
%----------------------------------------------------------------------------------------------------------

Man kann sich das auch so vorstellen, dass man mit dem Arbeitsintegral
\begin{align}
\int_0^{\,l} EI\,w_1^{IV}(x)\,\textcolor{chapterTitleBlue}{w_2(x)}\,dx
\end{align}
beginnt, und dann mittels partieller Integration die Ableitungen von $w_1$ vollst\"{a}ndig auf $\textcolor{chapterTitleBlue}{w_2} $ \"{u}berw\"{a}lzt und so am Schluss das Spiegelbild des Ausgangsintegrals erh\"{a}lt.

Differentialgleichungen, bei denen auf diesem Weg das Spiegelbild entsteht, hei{\ss}en {\em selbstadjungiert\/}. Alle linearen Differentialgleichungen gerader Ordnung sind selbstadjungiert.

Differentialgleichungen ungerader Ordnung, wie $u' = p$ nennt man {\em schiefsymmetrisch\/},  weil sie nur nach Multiplikation mit $(-1)$ mit dem Ausgangsintegral zur Deckung zu bringen sind
\begin{align}\label{Eq58}
\int_0^{\,l} u'\,\hat{u}\,dx = [u\,\hat{u}]_{@0}^{@l} - \int_0^{\,l} u\,\hat{u}'\,dx\,.
\end{align}
Partielle Integration ist daher, wenn man so will, eine \glq schiefsymmetrische\grq{} Operation.

%%%%%%%%%%%%%%%%%%%%%%%%%%%%%%%%%%%%%%%%%%%%%%%%%%%%%%%%%%%%%%%%%%%%%%%%%%%%%%%%%%%%%%%%%%%%%%%%%%%
{\textcolor{sectionTitleBlue}{\subsection*{Beispiel}}}
Die beiden Balken in Abb. \ref{U210} tragen verschiedene Streckenlasten
\begin{align}
p_1 &= 10 \qquad w_1(x) = \frac{10\cdot 5^4}{24\,EI}\,(\xi - 2\,\xi^3 + \xi^4) \qquad \xi = \frac{x}{l}\\
\textcolor{chapterTitleBlue}{p_2} &= 7\,\xi \qquad \textcolor{chapterTitleBlue}{w_2(x)} = \frac{7\cdot 5^3\,x}{360\,EI}\,(7 - 10\,\xi^2 + 3\,\xi^4)\,,
\end{align}
aber ihre reziproken \"{a}u{\ss}eren Arbeiten, also die Arbeiten \glq \"{u}ber Kreuz\grq{}, sind dennoch gleich gro{\ss}
\begin{align}\label{Eq35}
\text{\normalfont\calligra B\,\,}(w_1,\textcolor{chapterTitleBlue}{w_2}) &= \int_0^{\,l} p_1(x)\,\textcolor{chapterTitleBlue}{w_2(x)}\,dx - \int_0^{\,l} \textcolor{chapterTitleBlue}{p_2(x)}\,w_1(x)\,dx\nn \\
 &= \frac{1}{EI}\cdot 911.46 - \frac{1}{EI}\cdot 911.46 = 0\,.
\end{align}

%----------------------------------------------------------------------------------------------------------
\begin{figure}[tbp]
\centering
\if \bild 2 \sidecaption \fi
\includegraphics[width=1.0\textwidth]{\Fpath/U211}
\caption{{\em Satz von Betti\/} bei zwei unterschiedlich gelagerten Systemen} \label{U211}
%
\end{figure}%
%----------------------------------------------------------------------------------------------------------

In den Statikb\"{u}chern wird der {\em Satz von Betti\/} auf Systeme beschr\"{a}nkt, die im Gleichgewicht sind. Dieser Hinweis ist notwendig, weil die Autoren nicht mit der voll ausgeschriebenen zweiten Greenschen Identit\"{a}t beginnen, Glg. (\ref{Eq57}), und daraus alles weitere ableiten, sondern sie  beginnen den {\em Satz von Betti\/} gleich mit Glg. (\ref{Eq35}). Dann muss man aber dazu sagen, dass die beiden Biegelinien den Differentialgleichungen $EI\,w_1^{IV} = p_1$ bzw. $EI\,w_2^{IV} = p_2$ und den Randbedingungen gen\"{u}gen m\"{u}ssen, damit (\ref{Eq35}) richtig ist.

Der {\em Satz von Betti\/} gilt im \"{u}brigen auch dann, wenn die beiden Balken unterschiedlich gelagert sind, wie in Abb. \ref{U211}, denn
\begin{align}
A_{1,2} = M_1(0)\,\textcolor{chapterTitleBlue}{w_2'(0)} = - 50 \cdot \frac{17}{EI}= - 850\cdot\frac{1}{EI}
\end{align}
ist dasselbe, wie
\begin{align}
A_{2,1} &=   \int_0^{\,l} \textcolor{chapterTitleBlue}{p_2}\,w_1(x)\,dx - \textcolor{chapterTitleBlue}{B_2}\,w_1(l)\nn \\
&= \int_0^{\,5} 7 \cdot \frac{x}{5} \cdot ( 25\,x^2 - \frac{10}{6}\,x^3)\,\frac{1}{EI}\,dx  - \frac{35}{3} \cdot \frac{1250}{3\,EI}\nn \\
&=  4010.42  \cdot\frac{1}{EI} - 4861.1\cdot\frac{1}{EI} = - 850\cdot\frac{1}{EI}\,.
\end{align}
Die unterschiedliche Lagerung bedeutet nur, dass jetzt auch die Lagerkr\"{a}fte Arbeiten leisten und die m\"{u}ssen mitgez\"{a}hlt werden.

%%%%%%%%%%%%%%%%%%%%%%%%%%%%%%%%%%%%%%%%%%%%%%%%%%%%%%%%%%%%%%%%%%%%%%%%%%%%%%%%%%%%%%%%%%%%%%%%%%%
{\textcolor{sectionTitleBlue}{\section{Einflussfunktionen f\"{u}r Weggr\"{o}{\ss}en}}}\index{Einflussfunktionen f\"{u}r Weggr\"{o}{\ss}en}

Die Einflussfunktion $G_0(y,x)$ f\"{u}r die Verschiebung eines Punktes $x$ ist identisch mit der Verformung des Tragwerks, wenn eine Kraft $P = 1$ den Aufpunkt $x$ in Richtung der gesuchten Verformung dr\"{u}ckt. Die zweite Gr\"{o}{\ss}e $y$ ist die Laufvariable, also die Orte $y$, an denen wir die Verschiebung beobachten, die die Einzelkraft bewirkt.

Die Einflussfunktion ist symmetrisch, $G_0(y,x) = G_0(x,y)$, man kann also jederzeit $x$ mit $y$ vertauschen. Ob die Kraft im Punkte $x$ steht und wir beobachten die Verformung im Punkt $y $, oder ob die Kraft im Punkt $y$ steht und wir beobachten die Verformung im Punkt $x$, ist numerisch dasselbe ({\em Satz von Maxwell\/}).

F\"{u}r unsere Zwecke wird es sich als sinnvoll erweisen, mit $x$ den Aufpunkt zu bezeichnen und mit $y$ die Punkte, in denen die Belastung steht.

Eine Streckenlast $p$ kann man als eine Serie von kleinen Einzelkr\"{a}ften
\begin{align}
dP(y) = p(y)\,dy
\end{align}
ansehen, die jede f\"{u}r sich die Durchbiegung im Aufpunkt $x$ um das Ma{\ss}
\begin{align}
dw = G_0(y,x)\,dP(y)
\end{align}
erh\"{o}hen. Die gesamte Durchbiegung ist daher die Summe \"{u}ber die $dw$, also die \"{U}berlagerung der Einflussfunktion mit der Belastung
\begin{align} \label{Eq37}
w(x) = \int_0^{\,l} dw = \int_0^{\,l} G_0(y,x)\,p(y)\,dy\,.
\end{align}
Besteht die Belastung nur aus einer einzelnen Kraft $P$ in einem Punkt $y$, dann reduziert sich das nat\"{u}rlich auf den Ausdruck
\begin{align}
w(x) = G_0(y,x) \cdot P\,.
\end{align}
Und so, wie die Weg- und Kraftgr\"{o}{\ss}en eines Balkens aus der Biegelinie durch Differentiation hervorgehen,
\begin{align}
w'(x) = \frac{d}{dx}\,w(x) \qquad M(x) = - EI\,\frac{d^2}{dx^2}\,w(x) \qquad V(x) = - EI\,\frac{d^3}{dx^3}\,w(x)\,,
\end{align}
so gehen die zugeh\"{o}rigen Einflussfunktionen aus $G_0(y,x)$ durch Differentiation nach dem Aufpunkt $x$ hervor
\begin{subequations}
\begin{alignat}{3}
G_0(y,x) & \phantom{\frac{d}{dx} G_0(y,x)} &&= \text{Einflussfunktion f\"{u}r $w(x)$}\\
G_1(y,x) &= \frac{d}{dx} G_0(y,x) &&= \text{Einflussfunktion f\"{u}r $w'(x)$}\\
G_2(y,x) &= - EI\,\frac{d^2}{dx^2} G_0(y,x) &&= \text{Einflussfunktion f\"{u}r $M(x)$}\\
G_3(y,x) &= - EI\,\frac{d^3}{dx^3} G_0(y,x) &&= \text{Einflussfunktion f\"{u}r $V(x)$}
\end{alignat}
\end{subequations}\index{$G_0$}\index{$G_1$}\index{$G_2$}\index{$G_3$}
Wir bezeichnen die Einflussfunktionen mit dem Buchstaben $G$, weil in der Mathematik Einflussfunktionen {\em Greensche Funktionen\/}\index{Greensche Funktion} hei{\ss}en, und weil die Durchbiegung die nullte Ableitung ist, schreiben wir ihre Einflussfunktion $G_0 $ mit einem Index $0$.

Eigentlich m\"{u}sste man immer sauber trennen zwischen {\em Kern\/}\index{Kern einer Einflussfunktion} und Einflussfunktion. Die Greenschen Funktionen $G_i(y,x)$ sind die Kerne und die Integrale wie (\ref{Eq37}) sind die Einflussfunktionen, aber f\"{u}r Ingenieure sind die Kerne die Einflussfunktionen.

%----------------------------------------------------------------------------------------------------------
\begin{figure}[tbp]
\centering
\if \bild 2 \sidecaption \fi
\includegraphics[width=1.0\textwidth]{\Fpath/U212}
\caption{Anwendung des Satzes von Betti bei einem Balken} \label{U212}
%
\end{figure}%
%----------------------------------------------------------------------------------------------------------

%----------------------------------------------------------------------------------------------------------
\begin{figure}[tbp]
\centering
\if \bild 2 \sidecaption \fi
\includegraphics[width=1.0\textwidth]{\Fpath/U213}
\caption{Anwendung des Satzes von Betti bei einem Stab} \label{U213}
%
\end{figure}%
%----------------------------------------------------------------------------------------------------------
%%%%%%%%%%%%%%%%%%%%%%%%%%%%%%%%%%%%%%%%%%%%%%%%%%%%%%%%%%%%%%%%%%%%%%%%%%%%%%%%%%%%%%%%%%%%%%%%%%%
{\textcolor{sectionTitleBlue}{\subsection{Herleitung}}}

Technisch gesehen geschieht bei der Herleitung der Einflussfunktion (\ref{Eq37}) das folgende: Wir belasten den Tr\"{a}ger im Aufpunkt $x$ mit einer Einzelkraft $\textcolor{chapterTitleBlue}{P = 1} $, s. Abb. \ref{U212} a, ermitteln die zugeh\"{o}rige Biegelinie $\textcolor{chapterTitleBlue}{G_0(y,x)}$ und
formulieren dann mit den beiden Biegelinien $\textcolor{chapterTitleBlue}{G_0(y,x)}$ und $w(y)$ den {\em Satz von Betti\/}, d.h. die zweite Greensche Identit\"{a}t.

Das geht nicht in einem St\"{u}ck, weil die dritte Ableitung (die Querkraft) der Einflussfunktion im Aufpunkt springt. Wir integrieren also vom linken Lager bis zum Aufpunkt $x$, unterbrechen dort, und setzen die Integration hinter dem Aufpunkt fort
\begin{align}
\text{\normalfont\calligra B\,\,}(\textcolor{chapterTitleBlue}{G_0},w) &= \text{\normalfont\calligra B\,\,}(\textcolor{chapterTitleBlue}{G_{@0}^{@l}},w)_{(0,x)} + \text{\normalfont\calligra B\,\,}(\textcolor{chapterTitleBlue}{G_0^R},w)_{(x,l)}\,.
\end{align}
An den beiden Balkenenden sind $w$ und $M$ null und so verbleibt in der Summe
\begin{align}
\text{\normalfont\calligra B\,\,}(\textcolor{chapterTitleBlue}{G_0},w) &= \text{\normalfont\calligra B\,\,}(\textcolor{chapterTitleBlue}{G_{@0}^{@l}},w)_{(0,x)} + \text{\normalfont\calligra B\,\,}(\textcolor{chapterTitleBlue}{G_0^R},w)_{(x,l)}\nn \\
 & = \textcolor{chapterTitleBlue}{V_{@0}^{@l}(x)}\,w(x) - \textcolor{chapterTitleBlue}{M_{@0}^{@l}(x)}\,w'(x) - \int_0^{\,x} \textcolor{chapterTitleBlue}{G_{@0}^{@l}(y,x)}\,p(y)\,dy\nn \\
 &- \textcolor{chapterTitleBlue}{V_0^R(x)}\,w(x) + \textcolor{chapterTitleBlue}{M_0^R(x)}\,w'(x)- \int_x^{\,l} \textcolor{chapterTitleBlue}{G_0^R(y,x)}\,p(y)\,dy \nn \\
 &= \underbrace{(\textcolor{chapterTitleBlue}{V_{@0}^{@l}(x)} - \textcolor{chapterTitleBlue}{V_0^R(x)})}_{= 1}\,w(x) - \underbrace{(\textcolor{chapterTitleBlue}{M_{@0}^{@l}(x) - M_0^R(x)})}_{= 0}\,w'(x)\nn\\
  &- \int_0^{\,l} \textcolor{chapterTitleBlue}{G_0(y,x)}\,p(y)\,dy\nn \\
  &= \textcolor{chapterTitleBlue}{1}\cdot w(x) - \int_0^{\,l} \textcolor{chapterTitleBlue}{G_0(y,x)}\,p(y)\,dy = 0
\end{align}
oder
\begin{align}
\textcolor{chapterTitleBlue}{1}\cdot w(x) = \int_0^{\,l} \textcolor{chapterTitleBlue}{G_0(y,x)}\,p(y)\,dy\,,
\end{align}
was die Einflussfunktion f\"{u}r $w(x)$ ist.

%%%%%%%%%%%%%%%%%%%%%%%%%%%%%%%%%%%%%%%%%%%%%%%%%%%%%%%%%%%%%%%%%%%%%%%%%%%%%%%%%%%%%%%%%%%%%%%%%%%
{\textcolor{sectionTitleBlue}{\subsubsection{Einflussfunktion f\"{u}r $w'(x) $}}}
Zur Berechnung von $w'(x) $ belasten wir den Tr\"{a}ger im Aufpunkt mit einem Einzelmoment $\textcolor{chapterTitleBlue}{M = 1} $ und formulieren mit den beiden Teilen $\textcolor{chapterTitleBlue}{G_1^L}$ und $\textcolor{chapterTitleBlue}{G_1^R}$ den {\em Satz von Betti\/}, s. Abb. \ref{U212} c,
\begin{align}
\text{\normalfont\calligra B\,\,}(\textcolor{chapterTitleBlue}{G_1},w) = \text{\normalfont\calligra B\,\,}(\textcolor{chapterTitleBlue}{G_1^L},w)_{(0,x)} + \text{\normalfont\calligra B\,\,}(\textcolor{chapterTitleBlue}{G_1^R},w)_{(x,l)} =  0 + 0\,.
\end{align}
Der Sprung des Biegemomentes im Aufpunkt macht, dass bei der Addition der beiden Identit\"{a}ten im Aufpunkt die Arbeit
\begin{align}
(\textcolor{chapterTitleBlue}{M_L(x,x) - M_R(x,x)})\,w'(x) = \textcolor{chapterTitleBlue}{1} \cdot w'(x)
\end{align}
\"{u}brig bleibt und damit ergibt sich die Einflussfunktion f\"{u}r $w'(x)$
\begin{align}
 \textcolor{chapterTitleBlue}{1}\cdot w'(x) = \int_0^{\,l} \textcolor{chapterTitleBlue}{G_1(y,x)}\,p(y)\,dy\,.
\end{align}

%%%%%%%%%%%%%%%%%%%%%%%%%%%%%%%%%%%%%%%%%%%%%%%%%%%%%%%%%%%%%%%%%%%%%%%%%%%%%%%%%%%%%%%%%%%%%%%%%%%
{\textcolor{sectionTitleBlue}{\subsubsection{Einflussfunktion f\"{u}r die L\"{a}ngsverschiebung $u(x)$}}}
Die zweite Greensche Identit\"{a}t ({\em Satz von Betti\/}) der Differentialgleichung $-EA\,u''(x) = p(x)$ lautet
\begin{align}
\text{\normalfont\calligra B\,\,}(u,\textcolor{chapterTitleBlue}{\hat{u}}) &= \int_0^{\,l} - EA\,u''(x)\,\textcolor{chapterTitleBlue}{\hat{u}(x)}\,dx + [N\,\textcolor{chapterTitleBlue}{\hat{u}}]_{@0}^{@l}\nn \\
&- [u\textcolor{chapterTitleBlue}{\hat{N}}]_{@0}^{@l} - \int_0^{\,l} u(x)\,(\textcolor{chapterTitleBlue}{- EA\,\hat{u}''(x))}\,dx = 0\,,
\end{align}
und aus ihr erh\"{a}lt man die Einflussfunktion f\"{u}r $u(x) $, indem man eine Kraft $\textcolor{chapterTitleBlue}{P = 1} $ in Richtung der Stabachse wirken l\"{a}sst, s. Abb. \ref{U213},
\begin{align}
\text{\normalfont\calligra B\,\,}(\textcolor{chapterTitleBlue}{G_0},u) &= \text{\normalfont\calligra B\,\,}(\textcolor{chapterTitleBlue}{G_{@0}^{@l}},u)_{(0,x)} + \text{\normalfont\calligra B\,\,}(\textcolor{chapterTitleBlue}{G_0^R},u)_{(x,l)} = 0 + 0 \nn \\
&= \textcolor{chapterTitleBlue}{N_{@0}^{@l}(x)}\,u(x) - \int_0^{\,x} \textcolor{chapterTitleBlue}{G_{@0}^{@l}(y,x)}\,p(y)\,dy \nn \\
&- \textcolor{chapterTitleBlue}{N_0^R(x)}\,u(x) - \int_x^{\,l} \textcolor{chapterTitleBlue}{G_0^R(y,x)}\,p(y)\,dy\nn \\
&= \underbrace{(\textcolor{chapterTitleBlue}{N_{@0}^{@l}(x) - N_0^R(x)})}_{= \textcolor{chapterTitleBlue}{1}}\,u(x) - \int_0^{\,l} \textcolor{chapterTitleBlue}{G_0(y,x)}\,p(y)\,dy
\end{align}
oder
\begin{align}
1 \cdot u(x) = \int_0^{\,l} \textcolor{chapterTitleBlue}{G_0(y,x)}\,p(y)\,dy\,.
\end{align}

%-----------------------------------------------------------------
\begin{figure}[tbp]
\centering
\if \bild 2 \sidecaption \fi
\includegraphics[width=1.0\textwidth]{\Fpath/U164}
\caption{Eine Einflussfunktion gleicht einer Schaukel} \label{U164A}
\end{figure}%%
%-----------------------------------------------------------------

%%%%%%%%%%%%%%%%%%%%%%%%%%%%%%%%%%%%%%%%%%%%%%%%%%%%%%%%%%%%%%%%%%%%%%%%%%%%%%%%%%%%%%%%%%%%%%%%%%%
{\textcolor{sectionTitleBlue}{\section{Einflussfunktionen f\"{u}r Kraftgr\"{o}{\ss}en}}}\index{Einflussfunktionen f\"{u}r Kraftgr\"{o}{\ss}en}

Bei der Berechnung von Einflussfunktionen f\"{u}r Kraftgr\"{o}{\ss}en geht man -- kurz gesagt -- in zwei Schritten vor:\\

\colorbox{highlightBlue}{\parbox{0.5\textwidth}{
\begin{description}
  \item[$\bullet$] Kraftgr\"{o}{\ss}e sichtbar machen
  \item[$\bullet$] Schaukeln
\end{description}}}\\

Erst macht man die Schnittgr\"{o}{\ss}e durch den Einbau eines entsprechenden Gelenkes zu einer \"{a}u{\ss}eren Kraftgr\"{o}{\ss}e, s. Abb. \ref{U164A}, und dann bewegt man die beiden Gelenkh\"{a}lften so, dass die beiden Schnittgr\"{o}{\ss}en, links und rechts  vom Gelenk, insgesamt den Weg $-1$ zur\"{u}cklegen.

In der Statik hei{\ss}t dies der {\em Satz von Land\/}\index{Satz von Land}, der aber im Grunde doch nur das wiederholt, was im {\em Satz von Betti\/} seht. Man sucht in der zweiten Greenschen Identit\"{a}t nach der Kraftgr\"{o}{\ss}e, schaut auf ihren Partner, also die Weggr\"{o}{\ss}e, mit der die Kraftgr\"{o}{\ss}e gepaart ist, und wei{\ss} dann, dass man diese Weggr\"{o}{\ss}e um Eins springen lassen muss, damit bei der Addition, s. (\ref{Eq6}), die Kraftgr\"{o}{\ss}e \glq ans Licht kommt\grq{}.

Man unterscheidet, s. Abb. \ref{U30}, zwischen $M-$, $N-$ und $V-$ Gelenken. Ist das Tragwerk statisch bestimmt, dann wird aus dem Tragwerk durch den Einbau des Gelenkes ein {\em Getriebe\/} und dann sind keine Kr\"{a}fte n\"{o}tig, um die beiden Gelenkh\"{a}lften zu spreizen.

Ist das Tragwerk statisch unbestimmt, dann ben\"{o}tigt man daf\"{u}r Kr\"{a}fte. Praktisch geht man dabei so vor, dass  man zun\"{a}chst auf beiden Seiten des Gelenkes eine Kraftgr\"{o}{\ss}e $X = \pm 1$ wirken l\"{a}sst, die dadurch verursachte Spreizung des Gelenks ausrechnet, und dann das Paar $\pm X$ so normiert, dass die Spreizung sich genau zu Eins ergibt.


%----------------------------------------------------------------------------------------------------------
\begin{figure}[tbp]
\centering
\if \bild 2 \sidecaption \fi
\includegraphics[width=0.7\textwidth]{\Fpath/U30}
\caption{Der Einbau von Gelenken erm\"{o}glicht die Berechnung von Einflussfunktionen, \textbf{ a)} $M$-Gelenk, \textbf{ b)} $N$-Gelenk, \textbf{ c)} $V$-Gelenk } \label{U30}
\end{figure}%

%----------------------------------------------------------------------------------------------------------
\begin{figure}[tbp]
\centering
\if \bild 2 \sidecaption \fi
\includegraphics[width=0.8\textwidth]{\Fpath/U214}
\caption{Der Hebel des Archimedes} \label{U214}
%
\end{figure}%
%--------------------------------------------------------------------------------------------------------

%----------------------------------------------------------------------------------------------------------

Archimedes wusste, wenn er das linke Lager an dem Hebel in Abb. \ref{U214} wegnimmt, und den Hebel dort um eine L\"{a}ngeneinheit nach unten dr\"{u}ckt, dass dann die Arbeit der Lagerkraft $A$ und die Arbeit der Kraft $P$ in der Summe null sein m\"{u}ssen
\begin{align}
A_{1,2} = A \cdot 1 - P\,h_2\,\tan\,\Np = 0\,,
\end{align}
und er fand so f\"{u}r den Wert von $A$ das Resultat
\begin{align}
A = P\,h_2\,\tan\,\Np = P \cdot (\uparrow)\,.
\end{align}
Alle Einflussfunktionen f\"{u}r Kraftgr\"{o}{\ss}en sind im Grunde solche \glq Schaukeln\grq{}, s. Abb. \ref{U164A}, denn das Spiel von Kr\"{a}ften und Bewegungen ist der Grundpfeiler der Statik.\\

\hspace*{-12pt}\colorbox{highlightBlue}{\parbox{0.98\textwidth}{Statik ist nicht statisch, sondern Statik ist \glq kinematisch\grq{}.}}\\

Zu jeder Schnittkraft geh\"{o}rt ein Gelenk und die Bewegung, die \"{u}ber das Tragwerk l\"{a}uft, wenn man das Gelenk spreizt, {\em das Echo\/}, entscheidet dar\"{u}ber, wie gro{\ss} die Schnittkraft in dem Aufpunkt ist.


%----------------------------------------------------------------------------------------------------------
\begin{figure}[tbp]
\centering
\if \bild 2 \sidecaption \fi
\includegraphics[width=1.0\textwidth]{\Fpath/U215}
\caption{Berechnung der Einflussfunktion f\"{u}r die Normalkraft $N(x)$} \label{U215}
%
\end{figure}%
%---------------------------------------------------------------------------------------------------------
%%%%%%%%%%%%%%%%%%%%%%%%%%%%%%%%%%%%%%%%%%%%%%%%%%%%%%%%%%%%%%%%%%%%%%%%%%%%%%%%%%%%%%%%%%%%%%%%%%%
{\textcolor{sectionTitleBlue}{\subsection{Einflussfunktion f\"{u}r $N(x)$}}}
Die Einflussfunktion $\textcolor{chapterTitleBlue}{G_1(y,x)} $ f\"{u}r eine Normalkraft $N(x)$ weist im Aufpunkt $x$ einen Verschiebungssprung der Gr\"{o}{\ss}e Eins auf, s. Abb. \ref{U215} b,
\begin{align}
\textcolor{chapterTitleBlue}{G_1(x_{-}) - G_1(x_+) = 1}\,.
\end{align}
Die Randarbeiten an den Enden des Stabes im {\em Satz von Betti\/}
\begin{align}
\text{\normalfont\calligra B\,\,}(\textcolor{chapterTitleBlue}{G_1},u ) &= \text{\normalfont\calligra B\,\,}(\textcolor{chapterTitleBlue}{G_1^L},u)_{(0,x)} + \text{\normalfont\calligra B\,\,}(\textcolor{chapterTitleBlue}{G_1^R},u)_{(x,l)} = 0 + 0\nn \\
&= [\ldots]_0^x - \int_0^{\,x} \textcolor{chapterTitleBlue}{G_1^L(y,x)}\,p(y)\,dy + [\ldots]_x^l -  \int_x^{\,l} \textcolor{chapterTitleBlue}{G_1^R(y,x)}\,p(y)\,dy \,,
\end{align}
sind, wegen $u(0) = u(l) = 0$ und $\textcolor{chapterTitleBlue}{G_1(0,x) = G_1(l,x) = 0}$, null, und so verbleiben nur die Randarbeiten links und rechts vom Aufpunkt $x$.

Die zu $\textcolor{chapterTitleBlue}{G_1}$ geh\"{o}rige Normalkraft $\textcolor{chapterTitleBlue}{N_1}$ ist im Aufpunkt stetig, weil $\textcolor{chapterTitleBlue}{G_1} $ links und rechts vom Aufpunkt dieselbe Steigung hat, s. Abb. \ref{U215} b, und  auch $u(x)$ ist dort stetig, so dass die Arbeit der beiden Normalkr\"{a}fte $\pm \textcolor{chapterTitleBlue}{N_1(x)}$, links und rechts vom Gelenk, in der Summe null ist
\begin{align}
\underbrace{\textcolor{chapterTitleBlue}{N_1(x_{-})}\,u(x)}_{links} - \underbrace{\textcolor{chapterTitleBlue}{N_1(x_{+})}\,u(x)}_{rechts} = (\textcolor{chapterTitleBlue}{N_1(x_{-}) - N_1(x_{+})})\,u(x) = 0\,,
\end{align}
und sich somit alles auf
\begin{align}\label{Eq6}
\text{\normalfont\calligra B\,\,}(\textcolor{chapterTitleBlue}{G_1},u ) &= N(x) (\textcolor{chapterTitleBlue}{G_1(x_{-}) - G_1(x_+)}) - \int_0^{\,l} \textcolor{chapterTitleBlue}{G_1(y,x)}\,p(y)\,dy\nn \\
&= N(x) \cdot \textcolor{chapterTitleBlue}{1} - \int_0^{\,l}\textcolor{chapterTitleBlue}{ G_1(y,x)}\,p(y)\,dy = 0
\end{align}
reduziert, oder
\begin{align}
\textcolor{chapterTitleBlue}{1} \cdot N(x) = \int_0^{\,l} \textcolor{chapterTitleBlue}{G_1(y,x)}\,p(y)\,dy\,.
\end{align}
%----------------------------------------------------------
\begin{figure}[tbp]
\centering
\if \bild 2 \sidecaption \fi
\includegraphics[width=1.0\textwidth]{\Fpath/U216}
\caption{Einflussfunktionen f\"{u}r $M(x)$ und $V(x)$} \label{U216}
\end{figure}%%
%-----------------------------------------------------------------

%%%%%%%%%%%%%%%%%%%%%%%%%%%%%%%%%%%%%%%%%%%%%%%%%%%%%%%%%%%%%%%%%%%%%%%%%%%%%%%%%%%%%%%%%%%%%%%%%%%
{\textcolor{sectionTitleBlue}{\subsection{Einflussfunktion f\"{u}r $M(x)$}}}
Im Aufpunkt $x$ wird ein Momentengelenk eingebaut und dieses wird so bewegt, dass eine Spreizung von Eins entsteht
\begin{align}
\textcolor{chapterTitleBlue}{G_2'(x_{-}) - G_2'(x_+) = 1}\,.
\end{align}
Bei der Formulierung des Satzes von Betti mit den beiden Teilen der Einflussfunktion, $\textcolor{chapterTitleBlue}{G_2^L}$ und $\textcolor{chapterTitleBlue}{G_2^R}$,
\begin{align}
\text{\normalfont\calligra B\,\,}(\textcolor{chapterTitleBlue}{G_2},w) = \text{\normalfont\calligra B\,\,}(\textcolor{chapterTitleBlue}{G_2^L},w)_{(0,x)} + \text{\normalfont\calligra B\,\,}(\textcolor{chapterTitleBlue}{G_2^L},w)_{(x,l)} = 0 + 0
\end{align}
sind die Randarbeiten an den Balkenenden null und die Randarbeiten an der \"{U}bergangsstelle, im Aufpunkt $x$, heben sich gegenseitig weg bis auf den Term
\begin{align}
\textcolor{chapterTitleBlue}{G_2'(x_{-})}\,M(x) - \textcolor{chapterTitleBlue}{G_2'(x_+)}\,M(x) = \textcolor{chapterTitleBlue}{1} \cdot M(x)
\end{align}
und so folgt
\begin{align}
\textcolor{chapterTitleBlue}{1} \cdot M(x) = \int_0^{\,l} \textcolor{chapterTitleBlue}{G_2(y,x)}\,p(y)\,dy\,.
\end{align}
%----------------------------------------------------------
\begin{figure}[tbp]
\centering
\if \bild 2 \sidecaption \fi
\includegraphics[width=1.0\textwidth]{\Fpath/U260}
\caption{Herleitung von Einflussfunktionen durch Differentiation} \label{U260}
\end{figure}%%
%-----------------------------------------------------------------


%%%%%%%%%%%%%%%%%%%%%%%%%%%%%%%%%%%%%%%%%%%%%%%%%%%%%%%%%%%%%%%%%%%%%%%%%%%%%%%%%%%%%%%%%%%%%%%%%%%
{\textcolor{sectionTitleBlue}{\subsection{Einflussfunktion f\"{u}r $V(x)$}}}
Die Querkraft machen wir durch den Einbau eines Querkraftgelenks sichtbar, s. Abb. \ref{U216} e, und spreizen es dann derart, dass die beiden Querkr\"{a}fte in der Summe den Weg $(-1)$ zur\"{u}cklegen. Das bedeutet, dass die Einflussfunktion $\textcolor{chapterTitleBlue}{G_3}$ im Aufpunkt einen Versatz der Gr\"{o}{\ss}e Eins aufweist
\begin{align}
\textcolor{chapterTitleBlue}{G_3(x_{-}) - G_3(x_+) = 1}\,.
\end{align}
Entsprechend besteht die Biegelinie $\textcolor{chapterTitleBlue}{G_3}$ aus zwei Teilen, $\textcolor{chapterTitleBlue}{G_3^L}$ und $\textcolor{chapterTitleBlue}{G_3^R}$, und so m\"{u}ssen wir auch den {\em Satz von Betti\/} zweiteilen
\begin{align}
\text{\normalfont\calligra B\,\,}(\textcolor{chapterTitleBlue}{G_3},w) = \text{\normalfont\calligra B\,\,}(\textcolor{chapterTitleBlue}{G_3^L},w)_{(0,x)} + \text{\normalfont\calligra B\,\,}(\textcolor{chapterTitleBlue}{G_3^R},w)_{(x,l)} = 0 + 0\,.
\end{align}
An der \"{U}bergangsstelle, im Aufpunkt $x$, heben sich die Randarbeiten gegenseitig weg bis auf
\begin{align}
\textcolor{chapterTitleBlue}{G_3(x_{-})}\,V(x) - \textcolor{chapterTitleBlue}{G_3(x_+)}\,V(x) = \textcolor{chapterTitleBlue}{1} \cdot V(x)
\end{align}
und so ergibt sich aus $A_{1,2} = 0$ das Resultat
\begin{align}
\textcolor{chapterTitleBlue}{1} \cdot V(x) = \int_0^{\,l} \textcolor{chapterTitleBlue}{G_3(y,x)}\,p(y)\,dy\,.
\end{align}
%----------------------------------------------------------
\begin{figure}[tbp]
\centering
\if \bild 2 \sidecaption \fi
\includegraphics[width=1.0\textwidth]{\Fpath/U407}
\caption{Auswertung einer Einflussfunktion bei Lagersenkung, \textbf{ a)} Lagersenkung \textbf{ b)} Einflussfunktion $G_2(y,x)$ f\"{u}r $M(x)$} \label{U407}
\end{figure}%%
%-----------------------------------------------------------------

%%%%%%%%%%%%%%%%%%%%%%%%%%%%%%%%%%%%%%%%%%%%%%%%%%%%%%%%%%%%%%%%%%%%%%%%%%%%%%%%%%%%%%%%%%%%%%%%%%%
{\textcolor{sectionTitleBlue}{\subsection{Lagersenkung }}}\label{Korrektur18}\label{LagerWeg}\index{Lagersenkung}
Wenn sich ein Lager senkt, dann hat man kein $p$. Wie werden dann die Einflussfunktionen ausgewertet? Antwort: Indem man \"{u}ber die Lagerkr\"{a}fte geht. Wie das genau geht, soll ein Beispiel erl\"{a}utern.

Abb. \ref{U407}b zeigt die Einflussfunktion f\"{u}r das Biegemoment in Feldmitte des Tr\"{a}gers und die zugeh\"{o}rige vertikale Lagerkraft von 425 kN im Lager rechts. Abb. \ref{U407}a zeigt die eigentliche Lagersenkung.

Die Biegelinie $w$ ist eine homogene L\"{o}sung, $EI\,w^{IV} = 0$, und die Einflussfunktion ist eine L\"{o}sung der Gleichung $EI\,G_2^{IV}(y,x) = \delta_2(y-x)$. Das Dirac Delta ist die Spreizung in Feldmitte, die die Einflussfunktion f\"{u}r $M(x)$ erzeugt.

Mit diesen beiden Biegelinien, $w(y)$ aus der Lagersenkung und der Einflussfunktion $G_2(y,x)$, formulieren wir den Satz von Betti, und erhalten (wir \"{u}berspringen die Zwischenschritte)
\begin{align}
\text{\normalfont\calligra B\,\,}(G_2,w) = - M(x) - V_2(l)\,w(l) = 0\,,
\end{align}
oder
\begin{align}
M(x) = - V_2(l)\,w(l) = -\text{Lagerkraft aus Einflussfunktion} \times \text{Lagersenkung}
\end{align}
Das Minus in $- V_2(l)\,w(l) $ kommt aus dem Minus, das vor dem zweiten Teil von Betti steht
\begin{align}
\text{\normalfont\calligra B\,\,}(G_2,w) = \int_0^{\,l} \ldots\,dy + [\ldots]_0^l - [V_2\,w + \ldots]_0^l  - \int_0^{\,l} \ldots dy = 0\,.
\end{align}
Fassen wir das als Regel:\\

\hspace*{-12pt}\colorbox{highlightBlue}{\parbox{0.98\textwidth}{Bei einer Lagerbewegung geschieht die Auswertung einer Einflussfunktion durch Multiplikation der zur Einflussfunktion geh\"{o}rigen Lagerkraft mit dem Lagerweg $\times (-1)$}}\\

%%%%%%%%%%%%%%%%%%%%%%%%%%%%%%%%%%%%%%%%%%%%%%%%%%%%%%%%%%%%%%%%%%%%%%%%%%%%%%%%%%%%%%%%%%%%%%%%%%%
{\textcolor{sectionTitleBlue}{\subsection{Temperatur\"{a}nderungen}}}\label{Korrektur39}\index{Temperatur\"{a}nderungen}
Auch bei Temperatur\"{a}nderungen $\Delta T$ hat man kein $p$. Hier benutzt man die Mohrsche Arbeitsgleichung
\begin{align}\label{Eq131}
    \textcolor{red}{\bar{1}} \cdot \delta
    = &\ldots  + \int \textcolor{red}{\bar{M}}\,\alpha_T\,\frac{\Delta T}{h}\,dx + \int \textcolor{red}{\bar{N}}\,\alpha_T\,T\,dx\,.
\end{align}
Diese Terme kommen \"{u}brigens aus einer starken Einflussfunktion. Man k\"{o}nnte mit ihnen deshalb auch die \"{A}nderungen von Schnittgr\"{o}{\ss}en aus Temperatur verfolgen, wenn $\bar{M}$ und $\bar{N}$ die Momente und Normalkr\"{a}fte der Einflussfunktion f\"{u}r die Schnittkraft sind, s. S. \pageref{TempIdentit}.

%%%%%%%%%%%%%%%%%%%%%%%%%%%%%%%%%%%%%%%%%%%%%%%%%%%%%%%%%%%%%%%%%%%%%%%%%%%%%%%%%%%%%%%%%%%%%%%%%%%
{\textcolor{sectionTitleBlue}{\subsection{Die Kette der Einflussfunktionen }}}
Es beginnt mit der Einflussfunktion $G_0(y,x)$ f\"{u}r die Verschiebung bzw. die Durchbiegung in einem Punkt $x$ und die Ableitung nach dem Aufpunkt $x$ f\"{u}hrt dann zu den anderen Einflussfunktionen, wie in Abb. \ref{U260} gezeigt.
%----------------------------------------------------------
\begin{figure}[tbp]
\centering
\if \bild 2 \sidecaption \fi
\includegraphics[width=1.0\textwidth]{\Fpath/U301}
\caption{Der Einfluss eines Momentes h\"{a}ngt von der Neigung der Tangente an die Einflussfunktion im Quellpunkt, am Ort von $M$, ab, \textbf{ a)} Biegelinie aus dem Moment \"{u}ber dem Lager, \textbf{ b)} Einflussfunktion f\"{u}r Durchbiegung des Kragarmendes} \label{U301}
%
\end{figure}%%
%-----------------------------------------------------------------

Man kann die Einflussfunktion auch direkt differenzieren
\begin{align}
u(x) = \int_0^{\,l} G_0(y,x)\,p(y)\,dy \qquad \rightarrow \qquad N(x) = \int_0^{\,l} EA\,\frac{d}{dx}\,G_0(y,x)\,p(y)\,dy\,,
\end{align}
was aber als Differentiation eines Integrals nach einem Parameter gilt und da muss man aufpassen.

Theoretisch muss man in zwei Schritten vorgehen, wie wir am Beispiel einer Platte und der Einflussfunktion f\"{u}r das Moment $m_{xx}(\vek x)$ erl\"{a}utern wollen:
\begin{enumerate}
  \item Zun\"{a}chst muss man den Kern $G_2(\vek y,\vek x)$ der Einflussfunktion f\"{u}r das Moment $m_{xx}$ durch Ableitung aus $G_0$ berechnen, $G_2 = m_{xx} (G_0(\vek y,\vek x))$\,.
  \item Dann muss man den Grenzprozess
  \begin{align}
  \text{\normalfont\calligra B\,\,}(G_2,w) = \lim_{\varepsilon \to 0} \text{\normalfont\calligra B\,\,}(G_2,w)_{\Omega_\varepsilon} = 0
    \end{align}
    ausf\"{u}hren und das Ergebnis nach $m_{xx}(\vek x)$ aufl\"{o}sen
    \begin{align}
    m_{xx}(\vek x) = \int_{\Omega} G_2(\vek y,\vek x)\,p(\vek y)\,d\Omega_{\vek y}\,.
    \end{align}
\end{enumerate}
Der Praktiker wird nat\"{u}rlich sagen, \glq ich wei{\ss}, was heraus kommt\grq{}, und das Ergebnis direkt hinschreiben.

%%%%%%%%%%%%%%%%%%%%%%%%%%%%%%%%%%%%%%%%%%%%%%%%%%%%%%%%%%%%%%%%%%%%%%%%%%%%%%%%%%%%%%%%%%%%%%%%%%%
{\textcolor{sectionTitleBlue}{\subsection{Lastmomente differenzieren die Einflussfunktionen}}}
Betrachten wir den Balken in Abb. \ref{U301}. Das Moment \"{u}ber dem Lager im Punkt $y$ kann man in zwei Einzelkr\"{a}fte $ P = \pm  M/\Delta y$ aufl\"{o}sen, die untereinander den Abstand $\Delta y$ haben. Ist $G_0(y,l)$ die Einflussfunktion f\"{u}r die Durchbiegung am Kragarmende $x = l$, dann ist
\begin{align}
w(l) &= \lim_{\Delta y \to 0} (G_0(y + 0.5\,\Delta y,l) - G_0(y - 0.5\,\Delta y,l)) \cdot \frac{M}{\Delta y} = \frac{d}{dy}\,G_0(y,l)\cdot M
\end{align}
die Durchbiegung am Kragarmende aus dem Moment $M$ \"{u}ber dem Lager. Entscheidend f\"{u}r die Wirkung des Moments auf $w(l)$ ist also die Steigung der Einflussfunktion am Ort von $M$, im Quellpunkt.

Bei Rahmen haben also Einzelmomente $M$ in Feldmitte geringen Einfluss, weil dort die Neigung der Einflussfunktionen ann\"{a}hernd null ist,  $G' \sim 0$, und Momente $M$ in den Knoten den maximal m\"{o}glichen Einfluss, weil dort die Neigung $G'$ der Einflussfunktionen meist am gr\"{o}{\ss}ten ist.

%----------------------------------------------------------
\begin{figure}[tbp]
\centering
\if \bild 2 \sidecaption \fi
\includegraphics[width=0.9\textwidth]{\Fpath/U13}
\caption{Einflussfunktion f\"{u}r ein Moment} \label{U13}
%
\end{figure}%%
%-----------------------------------------------------------------

%%%%%%%%%%%%%%%%%%%%%%%%%%%%%%%%%%%%%%%%%%%%%%%%%%%%%%%%%%%%%%%%%%%%%%%%%%%%%%%%%%%%%%%%%%%%%%%%%%%
{\textcolor{sectionTitleBlue}{\subsection{Ein R\"{a}tsel}}}
Bei der Herleitung der Einflussfunktion f\"{u}r Biegemomente wird oft die Spreizung des Gelenks, wie in Abb. \ref{U13} gezeigt, mit $\delta \Np = 1$ angegeben, wo man dann r\"{a}tselt, was das denn genau bedeutet. Betr\"{a}gt der Winkel $45^\circ$ und meint $\delta \Np = 1$ also den Tangens dieses Winkels?
%----------------------------------------------------------
\begin{figure}[tbp]
\centering
\if \bild 2 \sidecaption \fi
\includegraphics[width=1.0\textwidth]{\Fpath/U217}
\caption{Gelenke machen die Schnittgr\"{o}{\ss}en sichtbar} \label{U217}
\end{figure}%%
%-----------------------------------------------------------------

Was eigentlich gemeint ist, sieht man in Abb. \ref{U13} auch. Der linke Teil des Tr\"{a}gers wird um einen Winkel $\Np_l$ verdreht und der rechte um einen Winkel $\Np_r$ und zwar so, dass die Summe
\begin{align}
\tan\,\Np_l +\tan\,\Np_r  = 1
\end{align}
gleich $1$ ist, denn dann erh\"{a}lt man prompt das gew\"{u}nschte Resultat
\begin{align}
- M \cdot \tan\,\Np_l  - M \cdot\tan\,\Np_r + P \cdot \delta w = - M \cdot 1 + P \cdot \delta w = 0\,,
\end{align}
also $M = P \cdot\delta w$.

In den Statikb\"{u}chern wird oft nicht sauber getrennt zwischen dem Tangens, $\tan\,\Np$, und dem Winkel $\Np$ selbst. Wenn die Autoren $\Np$ schreiben, dann meinen sie eigentlich immer den Tangens, und so auch hier
\begin{align}
\delta\,\Np = \Np_l + \Np_r = \tan\,\Np_l +\tan\,\Np_r  = 1\,.
\end{align}
Um dem Leser aber einen Gefallen (?) zu tun, wird der Tangens\index{Tangens} oft als der Drehwinkel selbst genommen, $ \tan\,\Np \simeq \Np $, und tritt dann prompt mit der Dimension {\em Rad \/} auf. Damit sind aber allen Missverst\"{a}ndnissen Tor und T\"{u}r ge\"{o}ffnet.

In diesem Buch schreiben wir $\tan\,\Np$, wenn wir den Tangens meinen. Wir erlauben uns nur die eine Unsch\"{a}rfe, dass wir gelegentlich im Text von Verdrehungen reden, wenn rechnerisch der Tangens gemeint ist.

Dass der Tangens eine so dominante Rolle spielt, liegt daran, dass er die Weggr\"{o}{\ss}e ist, die  zu $M$ konjugiert ist (erste Greensche Identit\"{a}t), w\"{a}hrend der Winkel keinen \glq Partner\grq{} hat und daher nicht in den Grundgleichungen der Statik vorkommt, die ja praktisch alle Arbeitsgleichungen sind.

%----------------------------------------------------------------------------------------------------------
\begin{figure}[tbp]
\centering
\if \bild 2 \sidecaption \fi
\includegraphics[width=0.6\textwidth]{\Fpath/U302}
\caption{{\em Satz von Betti\/}---Einflussfunktion f\"{u}r ein Moment, \textbf{ a)} Tr\"{a}ger mit Belastung, \textbf{ b)} dasselbe System unbelastet aber mit einer Spreizung $\tan \Np_l + \tan \Np_r = 1$ des Gelenks} \label{U302}
\end{figure}%
%----------------------------------------------------------------------------------------------------------

%%%%%%%%%%%%%%%%%%%%%%%%%%%%%%%%%%%%%%%%%%%%%%%%%%%%%%%%%%%%%%%%%%%%%%%%%%%%%%%%%%%%%%%%%%%%%%%%%%%
{\textcolor{sectionTitleBlue}{\section{Statisch bestimmte Tragwerke}}}
Die Einflussfunktionen f\"{u}r Kraftgr\"{o}{\ss}en an statisch bestimmten Tragwerken sind kinematische Ketten, weil durch den Einbau des Zwischengelenks der Grad der statischen Bestimmtheit sich von $n = 0$ auf $n = -1$ reduziert.

Die Schritte sind immer dieselben. Man baut ein $M$- bzw. $V$-Gelenk in das Tragwerk ein, um die innere Schnittgr\"{o}{\ss}e sichtbar zu machen, \glq sie ans Licht zu zwingen\grq{}, s. Abb. \ref{U217}. Dann zeichnet man das so modifizierte Tragwerk noch einmal an und verformt es nun so, dass die beiden Kraftgr\"{o}{\ss}en links und rechts vom Gelenk zusammen den Weg $-1$ gehen.

Ein Beispiel soll dies erl\"{a}utern. In Abb. \ref{U302} wird die Einflussfunktion f\"{u}r ein Moment hergeleitet. Zun\"{a}chst wird in den Tr\"{a}ger ein Gelenk eingebaut, um das innere Moment $M(x)$ \glq sichtbar\grq{} zu machen, zu einem \"{a}u{\ss}eren Momentenpaar zu machen. Dann wird der so modifizierte Tr\"{a}ger noch einmal angezeichnet, aber ohne Belastung. Statt dessen wird er so verschoben, dass die Spreizung im Gelenk genau $\tan \Np_l + \tan \Np_r = 1$ betr\"{a}gt. Weil der modifizierte Tr\"{a}ger kinematisch ist, sind dazu keine Kr\"{a}fte n\"{o}tig.
%----------------------------------------------------------
\begin{figure}[tbp]
\centering
\if \bild 2 \sidecaption \fi
\includegraphics[width=1.0\textwidth]{\Fpath/U165}
\caption{Regeln f\"{u}r die Polplankonstruktion} \label{U165}
\end{figure}%%
%-----------------------------------------------------------------

Nach dem {\em Satz von Betti\/} gilt
\begin{align}
\text{\normalfont\calligra B\,\,}(w_1,w_2) = A_{1,2} - A_{2,1} = 0\,.
\end{align}
Nun ist $A_{2,1} = 0$, weil die nicht vorhandenen Kr\"{a}fte am Tr\"{a}ger 2 keine Arbeit auf den Wegen $w_1(x)$ leisten. Die Arbeit der Kr\"{a}fte am Tr\"{a}ger 1 auf den Wegen $w_2(x)$ ist somit ebenso null
\begin{align}
A_{1,2} = -M_L\,\tan\,\Np_l - M_R\,\tan\,\Np_r + P\,w_2(x) = - M \cdot 1 + P\,w_2(x) = 0
\end{align}
oder
\begin{align}
1 \cdot M = P\,w_2(x)\,,
\end{align}
was beweist, dass $w_2(x)$ die Einflussfunktion f\"{u}r $M(x)$ ist.\\

\begin{remark}
Bekanntlich \"{a}ndern sich die Kraftgr\"{o}{\ss}en in einem statisch bestimmten Tragwerk nicht, wenn sich die Steifigkeiten \"{a}ndern. Ein \glq akademischer\grq{} Beweis dieses Prinzips l\"{a}sst sich wie folgt f\"{u}hren: wenn sich $EI$ oder $EA$ in einem Stab \"{a}ndert, dann sind die zugeh\"{o}rigen $f^+$ Gleichgewichtskr\"{a}fte, s. Kapitel 5, und weil Gleichgewichtskr\"{a}fte orthogonal sind zu allen Starrk\"{o}rperbewegungen, also allen kinematischen Ketten (den Einflussfunktionen f\"{u}r $N, M, V$), \"{a}ndern sich die  Schnittkr\"{a}fte nicht.\\
\end{remark}


%%%%%%%%%%%%%%%%%%%%%%%%%%%%%%%%%%%%%%%%%%%%%%%%%%%%%%%%%%%%%%%%%%%%%%%%%%%%%%%%%%%%%%%%%%%%%%%%%%%
{\textcolor{sectionTitleBlue}{\subsection{Polpl\"{a}ne}}}

Bei der Konstruktion der Verschiebungsfiguren, die durch das Spreizen der Gelenke entstehen, hilft die Kenntnis der {\em Drehpole\/} der einzelnen Scheiben. Als Scheiben bezeichnet man einzelne St\"{a}be und Balken, oder biegesteife Verbindungen und unverschiebliche Konstruktionen aus diesen.

Hierzu muss man jedoch anmerken, dass diese lediglich Repr\"{a}sentanten der entsprechenden Scheiben sind. Scheiben sind vielmehr unendlich gro{\ss}e Mengen von Punkten, deren Verdrehung um den zugeh\"{o}rigen Hauptpol so erfolgt, dass sie sich dabei auf Geraden und nicht auf Kreisbahnen bewegen. \\

Die Regeln f\"{u}r die Konstruktion der Polpl\"{a}ne lauten, s. Abb. \ref{U165}:\\

\begin{enumerate}
  \item Jedes feste Gelenklager ist Hauptpol der angeschlossenen Scheibe.
  \item Jedes Biegemomentengelenk bildet den Nebenpol der von diesem verbundenen Scheiben.
  \item Die Senkrechte zur Bewegungsrichtung eines verschieblichen Gelenklagers bildet den geometrischen Ort des Hauptpols der angeschlossenen Scheibe.
  \item Der Nebenpol zweier, durch einen verschieblichen Anschluss (Normalkraft- oder Querkraftgelenk) verbundenen Scheiben liegt auf jeder Senkrechten zur Bewegungsrichtung im Unendlichen.
  \item Die Hauptpole zweier Scheiben und ihr gemeinsamer Nebenpol liegen auf einer Geraden:
$(i)-(i, j)-(j)$, z.B.: $(1)-(1, 2)-(2)$.

  \item Die Nebenpole $(i, j), (j, k), (i, k)$ dreier Scheiben $I, J, K$ liegen auf einer Geraden: $(i, j)-(j, k)-(i, k)$, z.B.: $(1, 3)-(1, 4)-(3, 4)$.
  \end{enumerate}

%%%%%%%%%%%%%%%%%%%%%%%%%%%%%%%%%%%%%%%%%%%%%%%%%%%%%%%%%%%%%%%%%%%%%%%%%%%%%%%%%%%%%%%%%%%%%%%%%%%
{\textcolor{sectionTitleBlue}{\subsection{Konstruktion von Polpl\"{a}nen und Verschiebungsfiguren}}}


Am einfachsten beginnt man mit den festen Gelenklagern, denn diese sind, s. Regel 1, der Hauptpol der angeschlossenen Scheibe. Momentengelenke bilden den Nebenpol der angeschlossenen Scheiben, s. Regel 2.

Alle \"{u}brigen Pole bestimmt man nun mit Hilfe sogenannter {\em Ortslinien\/}. Unter einer Ortslinie versteht man die Gerade, auf der sich gem\"{a}{\ss} den Regeln 3 bis 7 der Pol befinden muss.\\

\begin{itemize}
  \item Der Schnittpunkt zweier Ortslinien f\"{u}r ein und denselben Pol ist der exakte geometrische Ort des Pols.
  \item Laufen verschiedene Ortslinien f\"{u}r ein und denselben Pol parallel, so liegt dieser als Schnittpunkt aller dieser Linien im Unendlichen.
  \item Liegt der Hauptpol einer Scheibe im Unendlichen bedeutet dies, dass sich die Scheibe nur parallel verschieben kann, ihre Verdrehung ist null.
  \item  Liegt der Nebenpol zweier Scheiben im Unendlichen bedeutet dies, dass sich beide Scheiben um ihre jeweiligen Hauptpole um exakt denselben Winkel verdrehen. Also sind zum Beispiel St\"{a}be dieser beiden Scheiben, die vor der Verdrehung parallel zueinander waren, es auch danach.
\end{itemize}

Mit diesen Regeln kann man die Verformungsfigur bestimmen, die durch das \glq normierte\grq{} Spreizen des $M$-, $V$- oder $N$-Gelenks entstehen. Normiert meint, dass das Gelenk so gespreizt wird, dass am Gelenk negative Arbeit auf einem Weg von 1 m geleistet wird. Der Teil der Verformungsfigur, der in Richtung der Wanderlast f\"{a}llt, ist dann die gesuchte Einflusslinie. Wir sagen dazu auch, dass die Einflussfunktion die \glq Projektion\grq{} der Verformungsfigur in Richtung der Wanderlast ist.

%----------------------------------------------------------
\begin{figure}[tbp]
\centering
\if \bild 2 \sidecaption \fi
\includegraphics[width=0.95\textwidth]{\Fpath/U366}
\caption{Verschiebungsberechnungen \textbf{ a)} Berechnung der Verschiebungen $u$ und $v$ einer Kraft \textbf{ b)} Berechnung der Verdrehungen zweier Scheiben zueinander } \label{U366}
\end{figure}%%
%-----------------------------------------------------------------

%%%%%%%%%%%%%%%%%%%%%%%%%%%%%%%%%%%%%%%%%%%%%%%%%%%%%%%%%%%%%%%%%%%%%%%%%%%%%%%%%%%%%%%%%%%%%%%%%%%
{\textcolor{sectionTitleBlue}{\subsection{Berechnung der Verdrehungen}}}}

An verschiedenen Stellen ben\"{o}tigt man ferner die Stabdrehwinkel von kinematischen Ketten und ihre Abh\"{a}ngigkeiten untereinander. Diese Aufgabe ist sehr leicht und elegant zu l\"{o}sen, wenn man sich den Zusammenhang zwischen der Verdrehung zweier Scheiben $(i)$ und $(k)$, die \"{u}ber den Nebenpol $(i,k)$ miteinander verbunden sind, klar macht. Das Vorgehen wollen wir an Abb. \ref{U366} b illustrieren.

Die Verdrehung $\Np_i$ des Stabes $i$ (bzw. der Scheibe $(i)$) ist gegeben und die Verdrehung Stabes $k$ in Abh\"{a}ngigkeit von $\varphi_i$ ist gesucht.

%----------------------------------------------------------
\begin{figure}[tbp]
\centering
\if \bild 2 \sidecaption \fi
\includegraphics[width=0.9\textwidth]{\Fpath/U290}
\caption{Berechnung der Verschiebungen der Scheiben bzw. St\"{a}be $2$ und $3$ bei vorgegebener Verdrehung von Scheibe 1} \label{U290}\label{Korrektur29}
\end{figure}%%
%-----------------------------------------------------------------

Es bezeichne $x_i$ den horizontalen Abstand des Hauptpols $(i)$ vom Nebenpol $(i,k)$ bzw. $x_k$ den horizontalen Abstand des Hauptpols $(k)$ vom Nebenpol $(i,k)$. Entsprechend bezeichnen $y_i$ und $y_k$ die vertikalen Abst\"{a}nde und $l_i$ und $l_k$ die Abst\"{a}nde der Hauptpole $(i)$ bzw. $(k)$ vom zugeh\"{o}rigen Nebenpol $(i,k)$.

Damit gilt
\begin{align}
\tan\,\Np_i = \frac{\eta}{l_i} \qquad \tan\,\Np_k = \frac{\eta}{l_k}
\end{align}}
also
\begin{align} \label{Eq115c}
\boxed{l_i \cdot \tan\,\Np_i = l_k \cdot \tan\,\Np_k }
\end{align}
Die Hauptpole der beiden Scheiben und ihr gemeinsamer Nebenpol liegen -- wie immer -- auf einer Geraden, die hier unter dem Winkel $\alpha$ geneigt ist, und daher gilt
\begin{align}
\sin\,\alpha = \frac{y_i}{l_i} = \frac{y_k}{l_k}
\end{align}
oder aufgel\"{o}st nach den L\"{a}ngen
\begin{align}
l_i = \frac{y_i}{\sin\,\alpha}\qquad l_k = \frac{y_k}{\sin\,\alpha}
\end{align}
und mit (\ref{Eq115c}) folgt also
\begin{align}\label{Eq116A}
\boxed{y_i \cdot \tan\,\Np_i = y_k \cdot \tan\,\Np_k}
\end{align}
Ebenso ergibt sich aus
\begin{align}
\cos\,\alpha = \frac{x_i}{l_i} = \frac{x_k}{l_k}
\end{align}
das Ergebnis
\begin{align}\label{Eq117A}
\boxed{x_i \cdot \tan\,\Np_i = x_k \cdot \tan\,\Np_k}
\end{align}
An einem System aus drei Scheiben, s. Abb. \ref{U290},  wollen wir die Anwendung dieser Beziehungen erl\"{a}utern. Gegeben ist der Winkel $\Np_1$ mit dem Wert $\tan\varphi_1 = 1/3$. Gesucht sind die anderen beiden Drehwinkel $\Np_2$ und $\Np_3$. Die Verdrehung der Scheibe 2 und damit des Stabes 2 ergibt sich \"{u}ber (\ref{Eq117A}) zu
\begin{align}
\tan\varphi_2=\frac{x_1\cdot\tan\varphi_1}{x_2}=\frac{3\cdot
1/3}{1}=1\,.
\end{align}
Die Verdrehung der Scheibe 3 erh\"{a}lt man nun z.B. aus (\ref{Eq116A})
\begin{align}
\tan\varphi_3=\frac{y_2\cdot\tan \varphi_2}{y_3}=\frac{2\cdot 1}{2}=1
\end{align}
oder mit (\ref{Eq117A}) \"{u}ber die Verdrehung der Scheibe 1
\begin{align}
\tan\varphi_3=\frac{\bar x_1\cdot\tan\varphi_1}{\bar x_3}=\frac{9\cdot 1/3}{3}=1\,.
\end{align}
Damit ergibt sich insgesamt f\"{u}r die Verdrehungen der Scheiben in Abh\"{a}ngigkeit von der Verdrehung der Scheibe 1 das Resultat
\begin{align}
\left [\barr{c}   \tan \Np_1 \\  \tan \Np_2 \\  \tan \Np_3\earr \right ]
 = \left [\barr{c}  1 \\  3 \\  3\earr \right ]\cdot \tan\varphi_1\,.
\end{align}

%%%%%%%%%%%%%%%%%%%%%%%%%%%%%%%%%%%%%%%%%%%%%%%%%%%%%%%%%%%%%%%%%%%%%%%%%%%%%%%%%%%%%%%%%%%%%%%%%%%
{\textcolor{sectionTitleBlue}{\subsection{Berechnung der Verschiebung eines Punktes}}}}
Die Arbeit, die eine Last $P$ auf der zugeh\"{o}rigen Verschiebung $v$ leistet, ist $P \cdot v$, s. Abb. \ref{U366} a. Deshalb ist es h\"{a}ufig notwendig den Anteil $v$ von $\delta P$ zu ermitteln, der in Richtung der Last f\"{a}llt. Das ist jedoch einfach, denn weil sich der Stab um seinen Hauptpol dreht, besteht zwischen der Auslenkung $\delta P$ in senkrechter Richtung zum Stab und den \"{u}brigen Gr\"{o}{\ss}en die Beziehung
\begin{align}
\frac{v}{\delta P} = \frac{x_p}{l_p}\,,
\end{align}
woraus schon das Ergebnis folgt
\begin{align}
v = x_p \cdot \frac{\delta P}{l_p} = x_p \cdot \tan \Np\,.
\end{align}
Analog l\"{a}sst sich auch die horizontale Verschiebung ermitteln, denn man erh\"{a}lt sofort
\begin{align}
u = y_p \cdot \tan\,\Np\,.
\end{align}

%----------------------------------------------------------
\begin{figure}[tbp]
\centering
\if \bild 2 \sidecaption \fi
\includegraphics[width=0.9\textwidth]{\Fpath/U201}
\caption{Vertikale Wanderlast und Einflussfunktion f\"{u}r eine Querkraft, $(1), (2), (3)$ sind die Hauptpole der Scheiben 1, 2 und 3 und $(1,2), (2,3)$ sind die Nebenpole von Scheibe 1 und 2 bzw. Scheibe 2 und 3} \label{U201}
\end{figure}%%
%-----------------------------------------------------------------

%----------------------------------------------------------
\begin{figure}[tbp]
\centering
\if \bild 2 \sidecaption \fi
\includegraphics[width=0.9\textwidth]{\Fpath/U202}
\caption{Einflussfunktion f\"{u}r eine Normalkraft bei vertikaler Wanderlast} \label{U202}
\end{figure}%%
%----------------------------------------------------------

%%%%%%%%%%%%%%%%%%%%%%%%%%%%%%%%%%%%%%%%%%%%%%%%%%%%%%%%%%%%%%%%%%%%%%%%%%%%%%%%%%%%%%%%%%%%%%%%%%%
{\textcolor{sectionTitleBlue}{\subsection{Einflussfunktion f\"{u}r eine Querkraft, Abb. \ref{U201}}}}

Im Abb. \ref{U201} ist die Einflusslinie f\"{u}r die Querkraft in dem rechten, schr\"{a}g verlaufenden Stab gesucht. Bei der Konstruktion des Polplans ist zu beachten, dass die Ortslinie des Nebenpols (2,3) auf jeder Senkrechten zur Bewegungsrichtung des Querkraftgelenkes im Unendlichen liegt, also auch auf derjenigen durch den Hauptpol (3). Diese Gerade durch den Hauptpol (3) ist gleichzeitig Ortslinie f\"{u}r den Hauptpol (2), genauso wie die Verbindungsgerade von (1) und (1,2). In dem Schnittpunkt der beiden Ortslinien liegt der Hauptpol (2).

Zur Generierung der Einflusslinie wird zwischen den beiden Ufern des Querkraftgelenkes eine Spreizung von Eins erzeugt. Vertikal, also in Lastrichtung, bedeutet dies in der Projektion eine relative Verschiebung der Scheiben zueinander im Gelenk und auch \"{u}ber den Hauptpolen von $0.8$ m.

Die relative Verschiebung \"{u}ber den Hauptpolen l\"{a}sst sich zur Konstruktion der Bewegung der Scheiben in der Projektion nutzen, da wegen der Unverschieblichkeit der Hauptpole die Verschiebung der Scheibe $2$ \"{u}ber dem Hauptpol (3) betragsm\"{a}{\ss}ig $0.8$ m ist und umgekehrt die Verschiebung der Scheibe $3$ \"{u}ber dem Hauptpol (2) betragsm\"{a}{\ss}ig ebenso $0.8$ m.

Die vertikalen Anteile der Bewegungen des Lastgurtes bilden die gesuchte Einfluss\-linie.

%----------------------------------------------------------
\begin{figure}[tbp]
\centering
\if \bild 2 \sidecaption \fi
\includegraphics[width=0.9\textwidth]{\Fpath/U248}
\caption{Einflussfunktion f\"{u}r ein Moment bei vertikaler Wanderlast} \label{U248}
\end{figure}%%
%-----------------------------------------------------------------


%%%%%%%%%%%%%%%%%%%%%%%%%%%%%%%%%%%%%%%%%%%%%%%%%%%%%%%%%%%%%%%%%%%%%%%%%%%%%%%%%%%%%%%%%%%%%%%%%%%
{\textcolor{sectionTitleBlue}{\subsection{Einflussfunktion f\"{u}r eine Normalkraft, Abb. \ref{U202}}}}

In Abb. \ref{U202} ist die Einflusslinie f\"{u}r die Normalkraft in dem schr\"{a}g verlaufenden Stab gesucht. Bei der Konstruktion des Polplanes ist zu beachten, dass die Ortslinie des Nebenpols (2,3) auf jeder Senkrechten zur Bewegungsrichtung des Normalkraftgelenkes im Unendlichen liegt, also auch auf derjenigen durch den Hauptpol (3). Diese Gerade durch den Hauptpol (3) ist auch Ortslinie f\"{u}r den Hauptpol (2) genauso wie die Verbindungsgerade von (1) und (1,2). Beide Ortslinien liefern in ihrem Schnittpunkt den Hauptpol (2).

Zur Konstruktion der Einflusslinie wird im Normalkraftgelenk eine Spreizung von Eins erzeugt. Vertikal, also in Lastrichtung, bedeutet dies in der Projektion eine relative Verschiebung der Scheiben zueinander im Gelenk und auch \"{u}ber den Hauptpolen von $0.5\cdot\sqrt{2}$ m. Die relative Verschiebung \"{u}ber den Hauptpolen l\"{a}sst sich zur Konstruktion der Scheiben in der Projektion nutzen, da hier wegen der Unverschieblichkeit der Hauptpole die Verschiebung der Scheibe $2$ \"{u}ber dem Hauptpol (3)  $0.5 \cdot \sqrt{2}$ m betr\"{a}gt und umgekehrt die Verschiebung der Scheibe $3$ \"{u}ber dem Hauptpol (2) absolut genommen $0.5 \cdot \sqrt{2}$ m betr\"{a}gt.


Die vertikalen Anteile der Bewegungen des Lastgurtes bilden wieder die gesuchte Einfluss\-linie.


%Bsp: 2.17
%%%%%%%%%%%%%%%%%%%%%%%%%%%%%%%%%%%%%%%%%%%%%%%%%%%%%%%%%%%%%%%%%%%%%%%%%%%%%%%%%%%%%%%%%%%%%%%%%%%
{\textcolor{sectionTitleBlue}{\subsection{Einflussfunktion f\"{u}r ein Moment, Abb. \ref{U248}}}}

In Abb. \ref{U248} ist die Einflusslinie f\"{u}r das Biegemoment im Punkt $i$ gesucht, und so wird an der Stelle $i$ zun\"{a}chst ein Momentengelenk eingef\"{u}gt. Das vormals statisch bestimmte System ist nun verschieblich. Die normierte Verschiebungsfigur ergibt sich dann \"{u}ber die Bedingung $\textcolor{chapterTitleBlue}{\tan\,\Np_r +\tan\,\Np_l=1}$. Diese ist genau dann erf\"{u}llt, wenn die vertikale Verschiebung im Aufpunkt $i$ den Wert
\begin{align}
\eta=\frac{ x_1\cdot x_2 }{ x_1+x_2}=\frac{ 3\cdot 4 }{ 3+4}=\frac{ 12 }{7}
\end{align}
hat.
%----------------------------------------------------------
\begin{figure}[tbp]
\centering
\if \bild 2 \sidecaption \fi
\includegraphics[width=0.9\textwidth]{\Fpath/U249}
\caption{Einflussfunktion (-linie) f\"{u}r ein Moment bei vertikaler Wanderlast} \label{U249}
\end{figure}%%
%-----------------------------------------------------------------

Zum Schluss muss man noch die Verschiebungsfigur in die Lastrichtung projizieren. Die Verdrehungen der drei Scheiben in der Projektion stimmen mit den Verdrehungen in der Verschiebungsfigur \"{u}berein. In den Hauptpolen ist die Verschiebung null und somit auch in der Projektion. Unter Beachtung dieser Zusammenh\"{a}nge ist es im Allgemeinen m\"{o}glich, sofort die Projektion des Lastgurtes in der Verschiebungsfigur zu zeichnen, ohne vorher die komplette Verschiebungsfigur am verschieblichen System zu bestimmen.

%%%%%%%%%%%%%%%%%%%%%%%%%%%%%%%%%%%%%%%%%%%%%%%%%%%%%%%%%%%%%%%%%%%%%%%%%%%%%%%%%%%%%%%%%%%%%%%%%%%
{\textcolor{sectionTitleBlue}{\subsection{Einflussfunktion f\"{u}r ein Moment, Abb. \ref{U249}}}}
%----------------------------------------------------------
\begin{figure}[tbp]
\centering
\if \bild 2 \sidecaption \fi
\includegraphics[width=0.8\textwidth]{\Fpath/U250}
\caption{Einflussfunktion f\"{u}r eine Querkraft bei vertikaler Wanderlast} \label{U250}
\end{figure}%%
%-----------------------------------------------------------------
In Abb. \ref{U249} ist ebenfalls die Einflusslinie f\"{u}r ein Biegemoment in einem Punkt $i$ gesucht. An der Stelle $i$ wird zun\"{a}chst wieder ein Momentengelenk eingef\"{u}gt, wodurch das ehemals statisch bestimmte System verschieblich wird. Im Unterschied zum Beispiel in Abb. \ref{U248} liegen beide Hauptpole der im Gelenk $i$, dem Nebenpol, miteinander verbundenen Scheiben, auf der rechten Seite des Gelenkes.

Die normierte Verschiebungsfigur ergibt sich nun \"{u}ber die Bedingung $\textcolor{chapterTitleBlue}{\tan\,\Np_4 -\tan\,\Np_2=1}$. Diese ist erf\"{u}llt, wenn die relative Verdrehung zwischen den beiden Scheiben $2$ und $4$ gleich eins ist, was genau dann der Fall ist, wenn die vertikale Verschiebung im Aufpunkt $i$ den Wert
\begin{align}
\eta=\frac{x_2\cdot x_4 }{x_2-x_4}=\frac{3\cdot 1 }{3-1}=\frac{3 }{2}
\end{align}
hat.

%%%%%%%%%%%%%%%%%%%%%%%%%%%%%%%%%%%%%%%%%%%%%%%%%%%%%%%%%%%%%%%%%%%%%%%%%%%%%%%%%%%%%%%%%%%%%%%%%%%
{\textcolor{sectionTitleBlue}{\subsection{Einflussfunktion f\"{u}r eine Querkraft, Abb. \ref{U250}}}}

In Abb. \ref{U250} ist die Einflusslinie f\"{u}r die Querkraft im Punkt $i$ gesucht und so wird an der Stelle $i$ zun\"{a}chst ein Querkraftgelenk eingef\"{u}gt. Analog zum Beispiel in Abb. \ref{U249} liegen beide Hauptpole der in diesem Gelenk verbundenen Scheiben rechts vom Aufpunkt $i$.

Zur Konstruktion der Einflusslinie wird zwischen den beiden Ufern des Querkraftgelenkes eine Spreizung von Eins erzeugt. Negative Arbeit wird dann  geleistet, wenn sich die Scheibe $2$ bzw. die Scheibe $3$ im Uhrzeigersinn bzw. entgegen dem Uhrzeigersinn um die zugeh\"{o}rigen Hauptpole drehen. Diese Drehrichtung findet man so auch in der Projektion wieder.
%----------------------------------------------------------
\begin{figure}[tbp]
\centering
\if \bild 2 \sidecaption \fi
\includegraphics[width=0.8\textwidth]{\Fpath/U251}
\caption{Einflussfunktionen f\"{u}r zwei Lagerkr\"{a}fte} \label{U251}
\end{figure}%%
%-----------------------------------------------------------------
%----------------------------------------------------------
\begin{figure}[tbp]
\centering
\if \bild 2 \sidecaption \fi
\includegraphics[width=1.0\textwidth]{\Fpath/U318}
\caption{Spreizung der Bogenmitte um 1 Meter, eine f\"{u}r die Baustatik fundamentale Figur, die zudem verdeutlicht, wie eng Statik und Kinematik zusammenh\"{a}ngen} \label{U318}
\end{figure}%%
%---------------------------------------------------------------

Vertikal, also in Lastrichtung, findet man in der Projektion eine relative Verschiebung der Scheiben zueinander im Gelenk und auch \"{u}ber den Hauptpolen von $1$ m. Die relative Verschiebung \"{u}ber den Hauptpolen l\"{a}sst sich zur Konstruktion der Scheiben in der Projektion nutzen, da hier wegen der Unverschieblichkeit der Hauptpole die Verschiebung der Scheibe $2$ \"{u}ber dem Hauptpol (3) absolut $1$ m ist und umgekehrt. Die Teile des Lastgurtes auf den Projektionen der Scheiben geh\"{o}ren zur gesuchten Einflusslinie.

%Bsp: 2.20
%%%%%%%%%%%%%%%%%%%%%%%%%%%%%%%%%%%%%%%%%%%%%%%%%%%%%%%%%%%%%%%%%%%%%%%%%%%%%%%%%%%%%%%%%%%%%%%%%%%
{\textcolor{sectionTitleBlue}{\subsection{Einflussfunktion f\"{u}r zwei Lagerkr\"{a}fte, Abb. \ref{U251}}}}

In Abb. \ref{U251} sind die Einflusslinien f\"{u}r die Auflagerkr\"{a}fte in $A$ und $B$ gesucht. Nach dem L\"{o}sen der jeweiligen Fessel wird, da die Auflagerkr\"{a}fte genau in Belastungs- und Projektionsrichtung liegen, der Punkt $A$ und $B$ um den Wert $1$ entgegengesetzt zur positiven Richtung der Auflagerkraft, also nach unten, verschoben (negative Arbeit!).

Die Einflusslinie f\"{u}r die Auflagerkraft $A$ l\"{a}sst sich sofort ablesen, da der abgesenkte Punkt der Scheibe $1$ zum Lastgurt und damit zur Einflusslinie geh\"{o}rt. Im Fall der Einflussfunktion f\"{u}r die Auflagerkraft in $B$ geh\"{o}rt dieser abgesenkte Punkt jedoch zu den Scheiben $3$ und $4$, deren Hauptpole im Unendlichen liegen. Damit verschieben sich diese beiden Scheiben parallel ebenfalls um eins nach unten.

Auf diesen beiden Scheiben findet man nun die Bilder der zu den Scheiben $1$ bzw. $2$ geh\"{o}renden Nebenpole (1,3) bzw. (2,4). Die Verbindung von (1) mit (1,3) und von (2) mit (2,4) in der Projektion liefert uns die zum Lastgurt geh\"{o}renden Scheiben $1$ und $2$ und damit die Einflusslinie.

%%%%%%%%%%%%%%%%%%%%%%%%%%%%%%%%%%%%%%%%%%%%%%%%%%%%%%%%%%%%%%%%%%%%%%%%%%%%%%%%%%%%%%%%%%%%%%%%%%%
{\textcolor{sectionTitleBlue}{\subsection{K\"{a}mpferdruck am Bogen, Abb. \ref{U318}}}}}
Der K\"{a}mpferdruck $H$ eines Bogens, s. Abb. \ref{U318},
\begin{align}
H = \frac{M}{f}\,,
\end{align}
ist gleich dem Feldmoment $M$ am gleichlangen Einfeldtr\"{a}ger dividiert durch den Stich $f$. Die richtige Balance zwischen $H$ und $f$ zu finden ist das Kernproblem bei der Konstruktion von H\"{a}ngebr\"{u}cken.

Eigentlich ist ein gelenkig gelagerter Bogen einfach statisch unbestimmt. Wenn aber die Belastung symmetrisch ist, dann ist im Scheitel des Bogens, die Querkraft null und daher kann man dann dort ein Querkraftgelenk einbauen und den Bogen somit statisch bestimmt machen.

In Abb. \ref{U318} wird die Normalkraft $N = H$ im Zenith des Bogens durch den Einbau eines $N$-Gelenks \glq sichtbar gemacht\grq{}. Bei einer Spreizung der Bogenmitte um einen Meter dreht sich die linke Seite des Bogens um den Pol $1$ und alle Punkte, die dieselbe H\"{o}he $f$ \"{u}ber dem Pol haben, schwenken um 0.5 m nach links, wie der Punkt $A$ in Abb. \ref{U318}. Daran kann man $\tan\,\Np = 0.5/f$ ablesen und alle anderen Punkte, die in der Horizontalen den Abstand $\ell/2$ vom Drehpol haben, schwenken um $v = \ell/2 \cdot \tan\,\Np$ nach oben und somit ist $H = P \cdot v = P \cdot l/(4 \cdot f) = M/f$. Bei einer Gleichlast $p$ gilt
\begin{align}
H = 2 \cdot p\int_0^{\,\ell/2} x \cdot \tan\,\Np\,dx = \frac{p \cdot \ell^2}{8 \cdot f } =  \frac{M}{f}\,.
\end{align}

%%%%%%%%%%%%%%%%%%%%%%%%%%%%%%%%%%%%%%%%%%%%%%%%%%%%%%%%%%%%%%%%%%%%%%%%%%%%%%%%%%%%%%%%%%%%%%%%%%%
{\textcolor{sectionTitleBlue}{\section{Statisch unbestimmte Tragwerke}}}
Wenn das Tragwerk statisch unbestimmt ist, dann sind Kr\"{a}fte n\"{o}tig, um die beiden Gelenkh\"{a}lften zu spreizen, aber auch dann ist $A_{2,1} = 0$ und der {\em Satz von Betti\/} reduziert sich wie oben auf
\begin{align}
\text{\normalfont\calligra B\,\,}(w_1,w_2) = A_{1,2} = 0\,.
\end{align}
%----------------------------------------------------------
\begin{figure}[tbp]
\centering
\if \bild 2 \sidecaption \fi
\includegraphics[width=0.9\textwidth]{\Fpath/U218}
\caption{Gelenke machen die Schnittgr\"{o}{\ss}en sichtbar} \label{U218}
\end{figure}%%
%-----------------------------------------------------------------
Betrachten wir den Balken in Abb. \ref{U218}. Um die Einflussfunktion f\"{u}r das Biegemoment in Feldmitte zu erzeugen, wird ein Momentengelenk eingebaut und die beiden H\"{a}lften so gegeneinander verdreht, dass sich eine Spreizung
\begin{align}
\textcolor{chapterTitleBlue}{\tan\,\Np_r + \tan\,\Np_l = 1}
\end{align}
einstellt. Dann integriert man von $0$ bis $x$ und von $x$ bis zum Tr\"{a}gerende $l$
\begin{align}
\text{\normalfont\calligra B\,\,}(\textcolor{chapterTitleBlue}{G_2},w) = \text{\normalfont\calligra B\,\,}(\textcolor{chapterTitleBlue}{G_2},w)_{(0,x)} + \text{\normalfont\calligra B\,\,}(\textcolor{chapterTitleBlue}{G_2},w)_{(x,l)}\,.
\end{align}
Die Randarbeiten an den Tr\"{a}gerenden sind null und an der \"{U}bergangsstelle, im Aufpunkt $x$, heben sich alle Randarbeiten weg, bis auf den Term
\begin{align}
\underbrace{-M(x)\,\textcolor{chapterTitleBlue}{w'(x_{-})}}_{\text{von links}} + \underbrace{M(x)\,\textcolor{chapterTitleBlue}{w'(x_+)}}_{\text{von rechts}} = - M(x) \cdot (\textcolor{chapterTitleBlue}{\tan\,\Np_r +\tan\,\Np_l})\,,
\end{align}
und somit ergibt sich in der Summe
\begin{align}
\text{\normalfont\calligra B\,\,}(\textcolor{chapterTitleBlue}{G_2},w) = -M(x) \cdot \underbrace{(\textcolor{chapterTitleBlue}{\tan\,\Np_r + \tan\,\Np_l})}_{= \textcolor{chapterTitleBlue}{ 1}} + \int_0^{\,l} \textcolor{chapterTitleBlue}{G_2(y,x)}\,p(y)\,dy = 0
\end{align}
oder
\begin{align}
M(x)  =  \int_0^{\,l} \textcolor{chapterTitleBlue}{G_2(y,x)}\,p(y)\,dy\,.
\end{align}
Um die Spreizung zu erzeugen, m\"{u}ssen an dem rechten Tr\"{a}ger links und rechts von dem Gelenk zwei gegengleiche Momente $\pm X$ wirken. In der Praxis macht man das so, dass man zun\"{a}chst ein Momentenpaar $\pm X = 1$ aufbringt, die Relativverdrehung berechnet und dann das $X$ so normiert, dass sich die gew\"{u}nschte Spreizung von eins einstellt.
%----------------------------------------------------------
\begin{figure}[tbp]
\centering
\if \bild 2 \sidecaption \fi
\includegraphics[width=1.0\textwidth]{\Fpath/U169}
\caption{Einflussfunktion f\"{u}r die Normalkraft in einer St\"{u}tze} \label{U169}
\end{figure}%%
%-----------------------------------------------------------------
%----------------------------------------------------------
\begin{figure}[tbp]
\centering
\if \bild 2 \sidecaption \fi
\includegraphics[width=1.0\textwidth]{\Fpath/U173}
\caption{Einflussfunktion f\"{u}r das Moment in einem Unterzug} \label{U173}
\end{figure}%%
%-----------------------------------------------------------------

Das ergibt die folgende Bilanz. Die Arbeit der \"{a}u{\ss}eren Kr\"{a}fte am Original auf den Wegen $G_2(y,x)$ aus der Spreizung ist
\begin{align}
A_{1,2} = \int_0^{\,l} \textcolor{chapterTitleBlue}{G_2(y,x)}\,p(y)\,dy - M \cdot (\textcolor{chapterTitleBlue}{\tan\,\Np_l + \tan\,\Np_r})\,,
\end{align}
aber die Arbeit der Kr\"{a}fte rechts auf den Wegen links ist null
\begin{align}\label{ImmerSo}
A_{2,1} = - \textcolor{chapterTitleBlue}{X} \cdot w'(x) + \textcolor{chapterTitleBlue}{X}\cdot w'(x) = (\textcolor{chapterTitleBlue}{-X + X})\,w'(x) = 0\,,
\end{align}
was immer so ist. Die Kr\"{a}fte $\textcolor{chapterTitleBlue}{\pm X} $, also hier die Momente, die die Spreizung des Gelenks bewirken, sind gegengleich und weil die zu den beiden $\textcolor{chapterTitleBlue}{\pm X}$ konjugierte Weggr\"{o}{\ss}e des Originals im Aufpunkt stetig ist ($w'(x)$ springt nicht bei diesem Beispiel), ist die Arbeit der Kr\"{a}fte $\textcolor{chapterTitleBlue}{\pm X}$ in der Summe null.\\
%----------------------------------------------------------------------------------------------------------
\begin{figure}[tbp]
\centering
\if \bild 2 \sidecaption \fi
\includegraphics[width=0.9\textwidth]{\Fpath/U219}
\caption{Eine weiche Feder f\"{a}ngt viel von der Fusspunktsbewegung auf, w\"{a}hrend eine harte Feder fast die ganze Bewegung an den Tr\"{a}ger weitergibt und somit die Einflussfunktion f\"{u}r die Lagerkraft weiter ausschl\"{a}gt als bei einer weichen Feder} \label{U219}
\end{figure}%
%----------------------------------------------------------------------------------------------------------

\hspace*{-12pt}\colorbox{highlightBlue}{\parbox{0.98\textwidth}{Bei der Berechnung von Einflussfunktionen f\"{u}r Kraftgr\"{o}{\ss}en reduziert sich der {\em Satz von Betti\/}  auf die Gleichung
\begin{align}
A_{1,2} =  0\,.
\end{align}}}\\

%%%%%%%%%%%%%%%%%%%%%%%%%%%%%%%%%%%%%%%%%%%%%%%%%%%%%%%%%%%%%%%%%%%%%%%%%%%%%%%%%%%%%%%%%%%%%%%%%%%
{\textcolor{sectionTitleBlue}{\section{Einflussfunktionen f\"{u}r Lagerkr\"{a}fte}}}\index{Einflussfunktionen f\"{u}r Lagerkr\"{a}fte}
Lagerkr\"{a}fte k\"{o}nnen, wie andere Schnittkr\"{a}fte auch, durch den Einbau eines entsprechenden Gelenks sichtbar gemacht werden.
Die Einflussfunktion entsteht dann wie gewohnt durch die Spreizung des Lagers.  Wenn der Boden starr ist, dann kann sich nur eine Seite des Lagers bewegen, die somit allein den vollen Weg $1$ gehen muss und die $1$ geht in voller H\"{o}he in das Tragwerk, wie etwa in Abb. \ref{U169}.

%----------------------------------------------------------------------------------------------------------
\begin{figure}[tbp]
\centering
\if \bild 2 \sidecaption \fi
\includegraphics[width=0.9\textwidth]{\Fpath/U15}
\caption{Einflussfunktion f\"{u}r eine elastische Einspannung} \label{U15}
\end{figure}%
%----------------------------------------------------------------------------------------------------------

Wenn der Boden elastisch ist, dann muss man durch eine lokale Analyse untersuchen, wieviel
von der $1$ der Boden beitr\"{a}gt und wieviel das Tragwerk.

Die Kraft $X$, die n\"{o}tig ist, um das Lager um 1 m auseinander zu dr\"{u}cken, betr\"{a}gt
\begin{align}
X = \frac{k_S\,k_B}{k_S + k_B} \qquad k_B = \text{$k$-Boden} \qquad k_S = \text{$k$-Struktur}\,.
\end{align}
Die Steifigkeit $k_S$ der Struktur im Lager ermittelt man, indem man die Verbindung des Tragwerks mit dem Boden l\"{o}st, und mit einer Kraft $X = 1$ gegen das Tragwerk dr\"{u}ckt. Der Kehrwert der Verformung ist $k_S$.
%----------------------------------------------------------------------------------------------------------
\begin{figure}[tbp]
\centering
\if \bild 2 \sidecaption \fi
\includegraphics[width=1.0\textwidth]{\Fpath/U365}
\caption{Einflussfunktionen f\"{u}r Querkr\"{a}fte, \textbf{ a)} $V_l$, \textbf{ b)} $V_r$ und \textbf{ c)} die Lagerkraft $B = V_r - V_l$} \label{U365}
\end{figure}%
%----------------------------------------------------------
\begin{figure}[tbp]
\centering
\includegraphics[width=1.0\textwidth]{\Fpath/1GREENF74D}
\caption{Wie die Einflussfunktion f\"{u}r die Querkraft \"{u}ber den Tr\"{a}ger wandert und dabei im Grunde immer gleich bleibt, \cite{Ha6}. }
\label{1GreenF74}%
%
\end{figure}%%
%----------------------------------------------------------
%----------------------------------------------------------
\begin{figure}[tbp]
\centering
\includegraphics[width=1.0\textwidth]{\Fpath/1GREENF73D}
\caption{Wie die Einflussfunktion f\"{u}r das Biegemoment \"{u}ber den Tr\"{a}ger wandert und dabei im Grunde immer gleich bleibt, \cite{Ha6}.}
\label{1GreenF73}%
%
\end{figure}%%

%----------------------------------------------------------------------------------------------------------
Ein verwandtes Problem stellen nachgiebige St\"{u}tzen (= Federn) dar, s. Abb. \ref{U219}.
Wenn die Feder sehr weich ist, dann wird der Weg $1 $, den der Fusspunkt der Feder geht, zu einem gro{\ss}en Teil von der Feder verschluckt und der Tr\"{a}ger sp\"{u}rt wenig von der Spreizung, d.h. die Einflussfunktion verl\"{a}uft sehr flach in dem Tr\"{a}ger. Umgekehrt, wenn die Feder sehr hart ist, dann teilt sich der Weg $1 $ am Fuss der Feder dem Tr\"{a}ger deutlich mit, d.h. die Feder nimmt relativ viel Last auf, weil die Einflussfunktion jetzt weit ausschwingt.
%----------------------------------------------------------------------------------------------------------
\begin{figure}[tbp]
\centering
\if \bild 2 \sidecaption \fi
\includegraphics[width=1.0\textwidth]{\Fpath/U282}
\caption{Durchlauftr\"{a}ger unter Gleichlast, \textbf{ a)} die Einflussfunktion f\"{u}r die Lagerkraft, \textbf{ b)} die Nullstellen der Querkraft, \textbf{ c)} Momentenverlauf } \label{U282}
\end{figure}%
%----------------------------------------------------------------------------------------------------------

Eine elastische Einspannung, s. Abb. \ref{U15}, kann durch eine Drehfeder mit der Steifigkeit $k_\Np$ simuliert werden. Bei einer Drehfeder betr\"{a}gt der Zusammenhang zwischen Drehwinkel $\Np$ und dem Moment $M$
\begin{align}
M = k_{\Np}\,\tan\,\Np\,.
\end{align}
Um die Drehfedersteifigkeit des Tragwerks zu ermitteln, denken wir uns das Tragwerk frei drehbar durch ein Gelenk mit der Einspannstelle verbunden und wir lassen ein Moment $X = 1 $ wirken. Sei $\tan\,\Np$ der Tangens des Drehwinkels, der sich dabei einstellt, dann betr\"{a}gt die Drehsteifigkeit der Struktur (S)
\begin{align}
k_{\Np}^S = \frac{1}{\tan\,\Np}\,,
\end{align}
und das Moment $X$, das f\"{u}r eine Spreizung $1$ n\"{o}tig ist, hat die Gr\"{o}{\ss}e
\begin{align}
X = \frac{k_\Np^S\,k_\Np}{k_\Np^S + k_\Np} \,.
\end{align}

%%%%%%%%%%%%%%%%%%%%%%%%%%%%%%%%%%%%%%%%%%%%%%%%%%%%%%%%%%%%%%%%%%%%%%%%%%%%%%%%%%%%%%%%%%%%%%%%%%%
{\textcolor{sectionTitleBlue}{\section{Spr\"{u}nge in Schnittgr\"{o}{\ss}en}}}\index{Spr\"{u}nge in Schnittgr\"{o}{\ss}en}
Momente $M$ oder Querkr\"{a}fte $V$ k\"{o}nnen springen. Es macht daher keinen Sinn einen Aufpunkt genau in einen solchen Sprung, wie das Zwischenlager eines Balkens, zu legen, s. Abb. \ref{U365}. Man kann nur eine Einflussfunktion f\"{u}r die linke Querkraft aufstellen und eine Einflussfunktion f\"{u}r die rechte Querkraft. Die Einflussfunktion f\"{u}r den Sprung $V_r - V_l$ ist identisch mit der Einflussfunktion f\"{u}r die Lagerkraft, die sich ja aus beiden Teilen zusammensetzt.

Im Angriffspunkt einer Einzelkraft $P = 1$ springt die Querkraft um eins und deswegen weisen Einflussfunktionen den typischen Sprung auf, s. Abb. \ref{1GreenF74}. Rechnerisch kommt er durch die Spreizung des Querkraftgelenks in das System hinein.

Die Einflussfunktionen f\"{u}r Schnittmomente sind immer stetig, springen nicht, weil sie ja die Wirkung von vertikalen Wanderlasten (keinen Momenten) erfassen, s. Abb. \ref{1GreenF73}. Wollte man Spr\"{u}nge sehen, dann m\"{u}sste man die Ableitung der Einflussfunktion antragen. Diese Kurve w\"{a}re dann die Einflussfunktion f\"{u}r Wandermomente.

%%%%%%%%%%%%%%%%%%%%%%%%%%%%%%%%%%%%%%%%%%%%%%%%%%%%%%%%%%%%%%%%%%%%%%%%%%%%%%%%%%%%%%%%%%%%%%%%%%%
{\textcolor{sectionTitleBlue}{\section{Die Nullstellen der Querkraft}}}\index{Nullstellen der Querkraft}
Praktiker sch\"{a}tzen die Gr\"{o}{\ss}e einer Lagerkraft \"{u}ber das Querkraftdiagramm ab. Je weiter die Nullstellen der Querkraft auseinander liegen, um so gr\"{o}{\ss}er ist die Lagerkraft, s. Abb. \ref{U282}.

Diese Absch\"{a}tzung beruht auf der Formel $V'(x) = - p(x)$. Links vom Lager (Koordinate $x_s$) gilt
\begin{align}
\int_{x_a}^{\,x_s}\,V'(x)\,dx = V_l - V(x_a) = V_l
\end{align}
und rechts vom Lager
\begin{align}
\int_{x_s}^{\,x_b}\,V'(x)\,dx = V(x_b) - V_r = - V_r
\end{align}
und somit ergibt sich die Lagerkraft zu
\begin{align}
R = V_r - V_l = \int_{x_s}^{\,x_b}\,p(x)\,dx + \int_{x_a}^{\,x_s}\,p(x)\,dx = \int_{x_a}^{\,x_b} \,p(x)\,dx\,.
\end{align}
In allen Lastf\"{a}llen $p = c$ (konstante Streckenlast) ist bei einem Durchlauftr\"{a}ger die Lage und der Abstand der Nullpunkte gleich und der Abstand ist gleich der Fl\"{a}che der Einflussfunktion
\begin{align}
R = \int_0^{\,l}  G(y,x_s) \cdot c\,dy = (x_b - x_a) \cdot c\,.
\end{align}
\"{A}hnliches gilt f\"{u}r das St\"{u}tzmoment $M = M_l = M_r$. Aus $M'(x) = V(x)$ folgt
\begin{align}
M = M_l = \int_{x_a}^{\,x_s} V\,dx \qquad M = M_r = -\int_{x_s}^{\,x_b} V\,dx\,,
\end{align}
wenn jetzt $x_a$ und $x_b$ die Nullstellen im $M$-Verlauf bezeichnen, s. Abb. \ref{U282} c. Das St\"{u}tzmoment ist also gleich dem Fl\"{a}cheninhalt von $V$ auf der linken Seite bzw. von $-V$ auf der rechten Seite.

%%%%%%%%%%%%%%%%%%%%%%%%%%%%%%%%%%%%%%%%%%%%%%%%%%%%%%%%%%%%%%%%%%%%%%%%%%%%%%%%%%%%%%%%%%%%%%%%%%%
{\textcolor{sectionTitleBlue}{\section{Dirac Deltas}}}\index{Dirac Delta}
All diese Ergebnisse, die wir oben doch relativ m\"{u}hsam durch Aufspalten des Integrationsbereichs in zwei Teile und dem genauen Verfolgen der einzelnen Terme hergeleitet haben, kann man mit dem
Dirac Delta viel schneller hinschreiben.

Das Dirac Delta wurde eingef\"{u}hrt, um mit Einzelkr\"{a}ften wie mit anderen Funktionen auch rechnen zu k\"{o}nnen. Das Dirac Delta ist eine Linienlast, die in allen Punkten $y$ au{\ss}er dem Aufpunkt $x$ null ist
\begin{align}
\delta_0(y-x) =  0 \qquad y \neq x\,,
\end{align}
und die bei einer virtuellen Verr\"{u}ckung $w$ gerade die Arbeit $w(x) \cdot 1$ leistet
\begin{align}
\int_0^{\,l} \delta_0(y-x)\, w(y)\,dy = w(x) \qquad x \in (0,l)\,,
\end{align}
ganz so, wie man sich das von einer echten Einzelkraft vorstellt.
%----------------------------------------------------------
\begin{figure}[tbp]
\centering
\if \bild 2 \sidecaption \fi
\includegraphics[width=1.0\textwidth]{\Fpath/U117}
\caption{In der obersten Reihe sind die vier Einflussfunktionen eines Balkens f\"{u}r \textbf{ a)} $w$, \textbf{ b)} $w'$, \textbf{ c)} $M $ und \textbf{ d)} $V$, jeweils in der Balkenmitte, dargestellt. In der zweiten Reihe sieht man die Einflussfunktionen f\"{u}r einen Stab, \textbf{ e)} $u$, \textbf{ f)} $N$. Die Einflussfunktionen integrieren, $+ $,  bzw. differenzieren, $-$, die Belastung} \label{U117}
%
\end{figure}%%
%-----------------------------------------------------------------

Die Biegelinie, die zu der Einzelkraft geh\"{o}rt, ist dann die L\"{o}sung der Differentialgleichung
\begin{align}
\textcolor{chapterTitleBlue}{EI \frac{d^4}{dy^4}\,G_0(y,x) = \delta_0(y-x)}
\end{align}
und mit dieser Definition und aufgrund der obigen Eigenschaften des Dirac Deltas ergibt sich die Einflussfunktion f\"{u}r $w(x) $ sozusagen automatisch
\begin{align}
\text{\normalfont\calligra B\,\,}(\textcolor{chapterTitleBlue}{G_0},w) &= \int_0^{\,l} \textcolor{chapterTitleBlue}{\delta_0(y-x)}\,w(y)\,dy - \int_0^{\,l} \textcolor{chapterTitleBlue}{G_0(y,x)}\,p(y)\,dy \nn \\
&= w(x) - \int_0^{\,l} \textcolor{chapterTitleBlue}{G_0(y,x)}\,p(y)\,dy  = 0\,.
\end{align}
Die Einflussfunktionen f\"{u}r die zweite Weggr\"{o}{\ss}e, $w'(x) $, und die beiden Kraftgr\"{o}{\ss}en, $M(x) $ und $V(x) $, ergeben sich analog durch Einf\"{u}hrung weiterer Dirac Deltas\index{$\delta_0$}\index{$\delta_1$}\index{$\delta_2$}\index{$\delta_3$}
\begin{alignat}{2}
&\delta_0(y-x) \qquad &&\text{Kraft $P = 1$}\nn \\
&\delta_1(y-x) \qquad &&\text{Moment $M = 1$}\nn \\
&\delta_2(y-x) \qquad &&\text{Knick $\Delta w' = 1$}\nn\\
&\delta_3(y-x) \qquad &&\text{Versatz $\Delta w = 1$}\nn
\end{alignat}
mit entsprechenden Eigenschaften, s. Abb. \ref{U117},
\begin{subequations}
\begin{alignat}{2}
&\int_0^{\,l} \delta_0(y-x)\,w(y)\,dy = w(x) \\
&\int_0^{\,l} \delta_1(y-x)\,w(y)\,dy = w'(x) \\
&\int_0^{\,l} \delta_2(y-x)\,w(y)\,dy = M(x) \\
&\int_0^{\,l} \delta_3(y-x)\,w(y)\,dy = V(x)\,.
\end{alignat}
\end{subequations}
Die Dirac Deltas sind sozusagen die Akteure, die aus der Biegelinie $w$ die interessierende Gr\"{o}{\ss}e herauspr\"{a}parieren.

Das Operieren mit Dirac Deltas is ein sehr eleganter Kalk\"{u}l, mit dem man sehr einfach die vielen Schritte, die zur Herleitung einer Einflussfunktion n\"{o}tig sind, wie die Zweiteilung des Intervalls, die genaue Verfolgung des Sprungs in der Querkraft, etc., umgehen kann, aber auf der anderen Seite darf man nicht vergessen, dass man nur auf diesem analytischen Weg
\begin{align}
\text{\normalfont\calligra B\,\,}(G_0,w) = \text{\normalfont\calligra B\,\,}(G_{@0}^{@l},w)_{(0,x)} + \text{\normalfont\calligra B\,\,}(G_0^R,w)_{(x,l)} = 0 + 0
\end{align}
die Ergebnisse wirklich herleiten kann. Wenn man danach wei{\ss}, was herauskommt, kann man die Abk\"{u}rzung nehmen, aber vorher muss man wissen, was eigentlich herauskommt...

Und noch eine Anmerkung: Die Punkt\-werte $w(x)$ etc. entspringen gar nicht dem Gebietsintegral, wie es das Dirac Delta glauben machen will, sondern es ist die Differenz zweier Randarbeiten, (Querkraftsprung), die den Punkt\-wert $w(x) $ liefert
\begin{align}
 \underbrace{(\textcolor{chapterTitleBlue}{V_{@0}^{@l}(x) - V_0^R(x))}}_{= 1}\,w(x) = \textcolor{chapterTitleBlue}{1}\cdot w(x)\,.
\end{align}
Das ist auch bei 2-D und 3-D Problemen so. Dann sind die Punktwerte die Grenzwerte von Randintegralen l\"{a}ngs der kreisf\"{o}rmigen \"{O}ffnung, die den Aufpunkt umgibt, und die sich dann zu einem Punkt zusammenschn\"{u}rt.

%-----------------------------------------------------------------
\begin{figure}[tbp]
\centering
\if \bild 2 \sidecaption \fi
\includegraphics[width=0.9\textwidth]{\Fpath/U220}
\caption{\textbf{ a)} Gleichgewicht der Kr\"{a}fte am Stab: Ut tenso sic vis, \textbf{ b)} Gleichheit der Arbeiten beim Flaschenzug} \label{U220}
\end{figure}%%
%-----------------------------------------------------------------

Die f\"{u}r uns wichtigste Eigenschaft von Dirac Deltas ist, dass man sie integrieren kann. Genauer gesagt, dass man feste Regeln daf\"{u}r hat, was
\begin{align}
\int_0^{\,l} \delta(y-x)\,\Np_i(y)\,dy
\end{align}
bedeuten soll, denn so kann man die Dirac Deltas nahtlos in die Methode der finiten Elemente einf\"{u}gen, kann man jedem Dirac Delta \"{a}quivalente Knotenkr\"{a}fte zuordnen\footnote{In Kapitel 1, $\vek K\,\vek u = \vek f + \vek p$, haben wir zwischen echten Knotenkr\"{a}ften $\vek f $ und \"{a}quivalenten Knotenkr\"{a}ften $\vek p$ aus Lasten im Feld unterschieden, um uns aber nicht zu weit vom {\em mainstream\/} zu entfernen, bezeichnen wir von nun ab, alle Kr\"{a}fte mit $\vek f $.}
\begin{subequations}
\begin{align}
f_i &= \int_0^{\,l} \delta_0(y-x)\,\Np_i(y)\,dy = \Np_i(x) \\
f_i &= \int_0^{\,l} \delta_1(y-x)\,\Np_i(y)\,dy = \Np_i'(x)\\
f_i &= \int_0^{\,l} \delta_2(y-x)\,\Np_i(y)\,dy = M(\Np_i)(x) \\
f_i &= \int_0^{\,l} \delta_3(y-x)\,\Np_i(y)\,dy = V(\Np_i)(x)\,.
\end{align}
\end{subequations}
Hier bedeutet $M(\Np_i)(x)$ das Moment der Ansatzfunktion $\Np_i$ im Aufpunkt $x$ und analog ist $ V(\Np_i)(x)$ die Querkraft von $\Np_i$ im Aufpunkt $x$.

%%%%%%%%%%%%%%%%%%%%%%%%%%%%%%%%%%%%%%%%%%%%%%%%%%%%%%%%%%%%%%%%%%%%%%%%%%%%%%%%%%%%%%%%%%%%%%%%%%%
{\textcolor{sectionTitleBlue}{\section{Dirac Energie}}}\index{Dirac Energie}
Das Geheimnis des Flaschenzuges ist, dass die Kraft, die am Seilende zieht, und das Gewicht dieselbe Arbeit verrichten , s. Abb. \ref{U220}.

Auch Einflussfunktionen dr\"{u}cken eine solche Balance aus, eine Energiebalance. Die Arbeit, die eine Einzelkraft $P = 1$
auf dem Weg $w(x)$ verrichtet
\beq
1 \cdot w(x) = \int_0^{\,l}  G_0(y,x)\,p(y)\,dy\,,
\eeq
ist dieselbe, die die verteilte Belastung $p$ auf dem Weg $G_0(y,x)$, der Durchbiegung des Balkens unter der Einzelkraft, leistet.

Der Faktor 1 ist wesentlich, weil ohne diesen Faktor die Dimensionen nicht \"{u}bereinstimmen
\begin{align}
\!\!\!\!\!Kraft \cdot Weg =  1 \cdot u(x) &= \int_0^{\,l} G_0(y, x)\,p(y)\,dy \nn \\
&= Weg \cdot Kraft/L\ddot{a}nge \cdot L\ddot{a}nge\,.
\end{align}
Die Auswertung einer Einflussfunktion ergibt daher eine {\em Energie\/}. Wir nennen dieses Energiequantum die {\em Dirac Energie\/}
\index{Dirac Energie}.\\

\hspace*{-12pt}\colorbox{highlightBlue}{\parbox{0.98\textwidth}{Die Dirac Energie ist die Arbeit, die die Belastung auf dem Wege der Einflussfunktion leistet.}}\\

%-----------------------------------------------------------------
\begin{figure}[tbp]
\centering
\if \bild 2 \sidecaption \fi
\includegraphics[width=0.9\textwidth]{\Fpath/U32}
\caption{Schaukel} \label{U32}
\end{figure}%%
%-----------------------------------------------------------------

Das einfachste Beispiel f\"{u}r diesen Gedanken ist die Schaukel, siehe Abb. \ref{U32}. Die Arbeit der beiden Gewichte ist bei jeder Drehung $\Np$ der Schaukel null
\begin{align}
P_l \, u_l - P_r \, u_r = P_l \,\tan \Np \, h_l - P_r \, \tan \Np \, h_r
= (P_l \, h_l - P_r \, h_r) \,\tan \Np  = 0\,,
\end{align}
weil die beiden Kr\"{a}fte dem Hebelgesetz\index{Hebelgesetz} gehorchen, $P_l \, h_l = P_r \, h_r$.
%-----------------------------------------------------------------
\begin{figure}[tbp]
\centering
\if \bild 2 \sidecaption \fi
\includegraphics[width=0.7\textwidth]{\Fpath/U258}
\caption{Die Kinematik eines Tragwerks bestimmt den Abtrag der Kr\"{a}fte} \label{U258}
\end{figure}%%
%-----------------------------------------------------------------

{\em In diesem Sinne gleicht jede Einflussfunktion einer Schaukel\/}. Um die Querkraft $V(x)$ eines Tr\"{a}gers in einem Punkt $x$ wie in Abb.  \ref{U164A}  zu berechnen, installieren wir im Punkt $x$ ein Querkraftgelenk und spreizen das Gelenk so, dass die beiden Querkr\"{a}fte dabei insgesamt die Arbeit $- V(x) \cdot 1$ leisten
\beq
-V(x) \, w(x_{-} ) - V(x) \, w(x_{+}) = -V(x) \, (w(x_{-} )
 + w(x_{+})) = -V(x) \cdot  1\,.
\eeq
Die Arbeit der Punktlast $P$ auf der Verschiebung $w$, die durch die Spreizung des Gelenks ausgel\"{o}st wird, muss genau das Gegenteil davon sein
\beq
\underbrace{- V(x) \cdot 1 + P \, w}_{A_{1,2}} = 0 \,,
\eeq
wie aus dem {\em Satz von Betti\/}, $A_{1,2} = A_{2,1}$ folgt. (Die Arbeit $A_{2,1}$ ist null, siehe die folgende Bemerkung).


Zu jeder Schnittgr\"{o}{\ss}e $V(x), N(x), M(x)$ etc., geh\"{o}rt also ein gewisser Mechanismus, eine gewisse Schaukel, s. Abb. \ref{U258}, und wenn wir das Gelenk l\"{o}sen und die Arbeit berechnen, die die Belastung auf den Wegen leistet, die durch die Spreizung des Gelenks verursacht werden, dann lernen wir, wie gro{\ss} die Schnittgr\"{o}{\ss}e in dem Gelenk sein muss, damit sie die Arbeit der \"{a}u{\ss}eren Belastung ins Gleiche setzt.

Bei einer FE-Berechnung behindern wir die freie Bewegung eines Tragwerks, wir legen dem Tragwerk sozusagen Fesseln an, weil die {\em shape functions\/} $\Np_i(x)$ zu \glq ungelenk\grq{} sind, und daher bekommt das Gelenk das falsche Signal. Die Verschiebung im Fu{\ss}punkt von $P$ ist $w_h$
\beq
-V_h(x) \cdot 1 + P \, w_h = 0
\eeq
und nicht der exakte Wert $w$
\beq
-V(x) \cdot 1  + P \, w = 0\,,
\eeq
und so ist $V_h(x) \neq V(x)$. {\em Ein FE-Programm versch\"{a}tzt sich bei den Dirac Energien\/}\footnote{Vor allem bei Fl\"{a}chentragwerken. Bei Stabtragwerken ist die Kinematik meist exakt, es sei denn $EA$ oder $EI$ sind nicht konstant.}.

Wir ziehen also den Schluss, dass die Kinematik eines FE-Netzes, die Feinheit der Details, die Genauigkeit der FE-L\"{o}sung bestimmt.\\

\hspace*{-12pt}\colorbox{highlightBlue}{\parbox{0.98\textwidth}{Netz = Kinematik = Pr\"{a}zision der Einflussfunktionen = G\"{u}te der Ergebnisse}}\\

%-----------------------------------------------------------------
\begin{figure}[tbp]
\centering
\if \bild 2 \sidecaption \fi
\includegraphics[width=0.9\textwidth]{\Fpath/U166}
\caption{Wenn eine Punktlast an der Turmspitze angreift, dann ist die Normalkraft in der Strebe proportional zur Auslenkung der Turmspitze, die durch die Spreizung der Strebe in Achsrichtung verursacht wird} \label{U166}
\end{figure}%%
%-----------------------------------------------------------------
Wir k\"{o}nnen jetzt auch sagen, was ein guter Entwurf\index{guter Entwurf} ist. Die Schaukellogik
\beq\label{Eq185}
V(x) = \frac{P \cdot w}{1} = P \cdot \frac{w}{1} \qquad \leftarrow \qquad\text{mache $w$ klein!}
\eeq
signalisiert, dass ein Entwurf dann gut ist, wenn das, was von der ausl\"{o}senden Bewegung, also hier der Spreizung 1 des Querkraftgelenks (im Nenner), im Fu{\ss}punkt der Punktlast $P$ ankommt, das ist das $w$ im Z\"{a}hler, so klein wie m\"{o}glich ist, weil dann $V(x)$ nur ein Bruchteil der Belastung $P$ sein wird.

{\em Wirf einen Stein ins Wasser und schau den Wellen zu!\/} Je kleiner die Wellen sind, die die Last erreichen, um so besser.
Der Hebel des Archimedes ist (ganz bewusst) das Gegenteil eines guten Entwurfs. Dr\"{u}ckt man das kurze linke Ende um eins nach unten, dann stellt sich am rechten Ende eine sehr gro{\ss}e Verschiebung $w$ ein, weswegen Archimedes nur eine kleine Kraft braucht, um die Welt aus den Angeln zu heben. Umgekehrt bedeutet dies aber auch, dass Archimedes lange, lange Wege gehen muss, um die Welt nur ein Iota zu heben.\\

%-----------------------------------------------------------------
\begin{figure}[tbp]
\centering
\if \bild 2 \sidecaption \fi
\includegraphics[width=0.9\textwidth]{\Fpath/U445}
\caption{Fachwerk mit biegesteifen Knoten: Einflussfunktion f\"{u}r ein Moment im Zuggurt. Die Spreizung erzeugt keine Verschiebung in den Knoten und das Bild best\"{a}tigt damit die Zul\"{a}ssigkeit der Fachwerktheorie} \label{U445}
\end{figure}\label{Korrektur38}
%-----------------------------------------------------------------
\begin{remark}
Glg. (\ref{Eq185}) macht noch einmal deutlich, dass Einfluss ein Verh\"{a}ltnis $w/1$ von zwei Verschiebungen ist, und daher ist es gleichg\"{u}ltig, ob die ausl\"{o}sende Spreizung 1 mm, 1 cm, 1 m oder 1 km ist. Es kommt nur auf das Verh\"{a}ltnis von gesp\"{u}rter Bewegung zu ausl\"{o}sender Bewegung an.
\end{remark}

Die Abb. \ref{U166} soll zeigen, dass der Normalkraftanteil einer Strebe an einer Last $\vek P = \{P_x, P_y, P_z\}^T$ an der Spitze des Eiffelturms, davon bestimmt wird, wie gro{\ss} die Auslenkungen $g_x, g_y, g_z$ sind, die die Spreizung der Strebe an der Spitze des Turms verursacht
\begin{align}
1 \cdot N = P_x \cdot g_x + P_y \cdot g_y + P_z \cdot g_z\,.
\end{align}
Die Abb. \ref{U445} demonstriert, dass die Kinematik die Fachwerktheorie best\"{a}tigt, mit der der Eiffelturm ja berechnet wurde (+ grafischer Statik(!)). Die Knoten verschieben sich bei der Spreizung des Aufpunktes nicht und so k\"{o}nnen Lasten in den Knoten keine Momente in dem Gurt erzeugen, d.h. die Knoten k\"{o}nnen in solchen Lastf\"{a}llen gelenkig gerechnet werden.

Der Schadensfall in Abb. \ref{U531} belegt eindr\"{u}cklich, welche gro{\ss}e und wichtige Rolle die Kinematik in der Statik spielt. Wir haben dieses Bild auch zu Ehren von Prof. C. Petersen eingef\"{u}gt, der \"{u}ber Zylinderschalen mit ver\"{a}nderlicher Kr\"{u}mmung promoviert hat, \cite{Petersen0}.

%-----------------------------------------------------------------
\begin{figure}[tbp]
\centering
\if \bild 2 \sidecaption \fi
\includegraphics[width=0.9\textwidth]{\Fpath/U531}
\caption{Die Kinematik \glq ist das Schicksal\grq{} -- sie bestimmt die Kr\"{a}fte. Nach 200 Jahren wurden Sicherungsma{\ss}nahmen an einer als Korbbogen ausgebildeten Br\"{u}cke n\"{o}tig, die man damals nachtr\"{a}glich \"{u}berbaut hat, {\em Kassel Schloss Wilhelmsh\"{o}he\/}} \label{U531}
\end{figure}
%-----------------------------------------------------------------
\vspace{-0.5cm}
%%%%%%%%%%%%%%%%%%%%%%%%%%%%%%%%%%%%%%%%%%%%%%%%%%%%%%%%%%%%%%%%%%%%%%%%%%%%%%%%%%%%%%%%%%%%%%%%%%%
{\textcolor{sectionTitleBlue}{\section{Punktwerte bei Fl\"{a}chentragwerken}}}
Punktwerte, wie etwa die Durchbiegung $w(x)$ eines Balkens, kommen direkt in der zweiten Greenschen Identit\"{a}t der Balkengleichung vor und daher ist es ein einfaches, eine Einflussfunktion f\"{u}r $w(x)$ herzuleiten, man muss nur den Balken in zwei Teile teilen, denn dann springt wie von selbst an der Intervallgrenze mit Hilfe des dualen Lastfalls, der Einzelkraft $P = 1$, der Wert $w(x)$ heraus
\begin{align}
1 \cdot w(x) = \int_0^{\,l} G_0(y,x)\,p(y)\,dy\,.
\end{align}
Bei Fl\"{a}chentragwerken ist das anders. Die Biegefl\"{a}che $w(\vek x)$ einer Membran ist die L\"{o}sung des Randwertproblems
\begin{align}
- \Delta w = p \qquad  w = 0 \,\,\,\text{auf dem Rand $\Gamma$}\,,
\end{align}
und in der zugeh\"{o}rigen zweiten Greensche Identit\"{a}t,
\begin{align}
\text{\normalfont\calligra B\,\,}(w,\hat{w}) &= \int_{\Omega} - \Delta\,w\,\hat{w}\,d\Omega + \int_{\Gamma} \frac{\partial w}{\partial n}\,\hat{w}\,ds \nn \\
&- \int_{\Gamma} w\,\frac{\partial \hat{w}}{\partial n}\,ds - \int_{\Omega} w\,(- \Delta \hat{w})\,d\Omega = 0\,,
\end{align}
stehen nur Integrale, aber keine Punktwerte.

Der \"{U}bergang zum Punkt gelingt, weil die Biegefl\"{a}che $G_0(\vek y,\vek x)$, die zu einer Punktlast $P = 1$ geh\"{o}rt, die Eigenschaft hat, dass das Integral der Querkr\"{a}fte $\partial G_0/\partial n$ der Membran \"{u}ber immer enger gezogene Kreise $\Gamma_\varepsilon$ um den Aufpunkt in der Grenze den Wert $1$ hat
\begin{align}\label{Eq88}
\lim_{\varepsilon \to 0} \int_{\Gamma_\varepsilon} Querkraft\,\,ds = \lim_{\varepsilon \to 0} \int_{\Gamma_\varepsilon} \frac{\partial G_0}{\partial n}\,ds = 1\,.
\end{align}
Dieser Grenzwert macht den \"{U}bergang vom Integral zum Punkt m\"{o}glich. Es ist eine sehr wichtige Eigenschaft von $G_0$.

Man formuliert daher den {\em Satz von Betti\/} zun\"{a}chst auf dem gelochten Gebiet $\Omega_\varepsilon = \Omega - N_\varepsilon$, spart also einen kleinen Kreis $N_\varepsilon$ um den Aufpunkt aus, und l\"{a}sst dann den Radius $\varepsilon \to 0$ gegen null gehen. In der Grenze
\begin{align}
\text{\normalfont\calligra B\,\,}(G_0,w) := \lim_{\varepsilon \to 0} \int_{\Gamma_\varepsilon} \frac{\!\!\partial G_0(\vek y,\vek x)}{\partial n}\,w(\vek y)\,\,ds_{\vek y} - \lim_{\varepsilon \to 0} \int_{\Omega_\varepsilon} \!\!G_0(\vek y,\vek x)\,p(\vek y)\,d\Omega_{\vek y} = 0
\end{align}
 erh\"{a}lt man so die Einflussfunktion f\"{u}r $w(\vek x)$ in dem Aufpunkt $\vek x$,
\begin{align}
 1 \cdot w(\vek x) = \int_{\Omega} G_0(\vek y,\vek x)\,p(\vek y)\,d\Omega_{\vek y}\,.
\end{align}
Die Herleitung von Einflussfunktionen bei Fl\"{a}chentragwerken ist sehr technisch und nicht immer einfach, siehe \cite{Ha2} und \cite{Ha3}. Zum Gl\"{u}ck geht das ganze mit finiten Elementen aber viel einfacher, s. Kapitel 3.\\

\begin{remark}
Bei einem Seil ist $V = H\,w'$ die Querkraft, bei einer Membran ist die Querkraft $v_n$ in einem Schnitt mit der Schnittnormalen $\vek n$ die Normalableitung der Biegefl\"{a}che in Richtung von $\vek n$
\begin{align}
v_n = H\,\frac{\partial w}{\partial n} = H\,\nabla w \dotprod \vek n = H\,(w,_x\,n_x + w,_y\,n_y)\qquad \text{[kN/m]}\,.
\end{align}
$H$ ist die Vorspannkraft, die wir oben Eins gesetzt haben.
\end{remark}

\vspace{-0.5cm}
%%%%%%%%%%%%%%%%%%%%%%%%%%%%%%%%%%%%%%%%%%%%%%%%%%%%%%%%%%%%%%%%%%%%%%%%%%%%%%%%%%%%%%%%%%%%%%%%%%%
{\textcolor{sectionTitleBlue}{\section{Dualit\"{a}t}}}\index{Dualit\"{a}t}
Hinter dem {\em Satz von Betti\/} steckt ein Begriff, der f\"{u}r das Rechnen in der Statik sehr wichtig ist, der Begriff der {\em Dualit\"{a}t\/}.

Den einfachsten Zugang zu diesem Thema bietet eine Steifigkeitsmatrix $\vek K$. Wenn man die Matrix mit einem Vektor $\vek u$ multipliziert, $\vek K\,\vek u$, und diesen
Vektor dann skalar mit einem zweiten Vektor $\textcolor{chapterTitleBlue}{\vek \delta \vek u} $, ist das Ergebnis eine Zahl $\textcolor{chapterTitleBlue}{\vek \delta \vek u^T}\,\vek K\,\vek u$.

Weil eine  reelle Zahl wie $\pi$ sich nicht \"{a}ndert, wenn man sie transponiert, $\pi^T = \pi$, gilt
\begin{align}
\textcolor{chapterTitleBlue}{\vek \delta \vek u^T}\,\vek K\,\vek u = \vek u^T\,\vek K\,\textcolor{chapterTitleBlue}{\vek \delta \vek u}
\end{align}
oder
\begin{align}\label{Eq39}
\text{\normalfont\calligra B\,\,}(\vek u,\textcolor{chapterTitleBlue}{\vek \delta \vek u})= \textcolor{chapterTitleBlue}{\vek \delta \vek u^T}\,\vek K\,\vek u - \vek u^T\,\vek K\,\textcolor{chapterTitleBlue}{\vek \delta \vek u} = 0\,.
\end{align}
Das ist der {\em Satz von Betti\/} f\"{u}r symmetrische, d.h. selbstadjungierte Matrizen\index{selbstadjungierte Matrizen}.  Symmetrie bei Matrizen ist dasselbe wie selbstadjungiert bei Differentialgleichungen.

Eine Steifigkeitsmatrix kann man bekanntlich als die Abbildung eines Vektors $\vek u$ auf einen Vektor $\vek f$ lesen
\begin{align}
\vek K\,\vek  u = \vek f\,.
\end{align}
Nun stellen wir uns vor, wir kennen den Vektor $\vek f$, der etwa die Knotenkr\"{a}fte eines Fachwerks darstellt, und wir wollen die Komponente $u_1$ des Vektors $\vek u$ im LF $\vek f$ wissen.

Um $u_1$ zu bestimmen, l\"{o}sen wir das Gleichungssystem
\begin{align}\label{Eq38}
\vek K\,\textcolor{chapterTitleBlue}{\vek g_1} = \textcolor{chapterTitleBlue}{\vek e_1}\,,
\end{align}
wir vertauschen also $\vek f$ mit dem ersten Einheitsvektor $\textcolor{chapterTitleBlue}{\vek e_1^T = \{1,0,0,\ldots,0\}}$.
Mit der L\"{o}sung $\textcolor{chapterTitleBlue}{\vek g_1}$ und dem Vektor $\vek u$ gehen wir dann in
die Identit\"{a}t (\ref{Eq39})
\begin{align}
\text{\normalfont\calligra B\,\,}(\textcolor{chapterTitleBlue}{\vek g_1},\vek u) = \textcolor{chapterTitleBlue}{\vek g_1^T}\,\vek f - \vek u^T\,\textcolor{chapterTitleBlue}{\vek e_1} = \textcolor{chapterTitleBlue}{\vek g_1^T}\,\vek f - u_1 = 0
\end{align}
und erhalten so das gew\"{u}nschte Resultat
\begin{align}
u_1 = \textcolor{chapterTitleBlue}{\vek g_1^T}\,\vek f\,.
\end{align}
Zu jeder Komponente $u_i$ gibt es einen solchen Vektor $\textcolor{chapterTitleBlue}{\vek g_i}$, der die L\"{o}sung von
\begin{align}
\vek K\,\textcolor{chapterTitleBlue}{\vek g_i }= \textcolor{chapterTitleBlue}{\vek e_i}\,,
\end{align}
ist und mit dem man $u_i$ aus der rechten Seite $\vek f$ berechnen kann, s. Abb. \ref{U42},
\begin{align}
u_i = \textcolor{chapterTitleBlue}{\vek g_i^T}\,\vek f\,.
\end{align}
Indem man also den Vektor $\vek f$ auf die $n$ Vektoren $\textcolor{chapterTitleBlue}{\vek g_i, i = 1, 2 \ldots n}$ projiziert, kann man die L\"{o}sung $\vek u = u_1\,\vek e_1 + \ldots + u_n\,\vek e_n$
bestimmen, denn die Projektion von $\vek f$ auf die Vektoren $\textcolor{chapterTitleBlue}{\vek g_i}$ ist dasselbe, wie die Projektion von $\vek u$ auf die Einheitsvektoren $\textcolor{chapterTitleBlue}{\vek e_i}$
\beq\label{Eq40}
u_i = \textcolor{chapterTitleBlue}{\vek g_i^T} \vek f = \vek u^T\,\textcolor{chapterTitleBlue}{\vek e_i}\,.
\eeq
%----------------------------------------------------------------------------------------------------------
\begin{figure}[tbp]
\centering
\includegraphics[width=.65\textwidth]{\Fpath/U42}
\caption{Dualit\"{a}t am Beispiel der linearen Algebra, Dualit\"{a}t = \glq \"{u}ber Kreuz\grq{}}
\label{U42}%
\end{figure}%%
%--------------------------------------------------------------------------------------------------
Genau das passiert, (f\"{u}r alle $u_i$ gleichzeitig), wenn wir den Vektor $\vek f$ mit der inversen Matrix $\vek K^{-1}$ multiplizieren
\beq
\vek u = \vek K^{-1}\,\vek f\,,
\eeq
denn die Zeilen (und Spalten) der symmetrischen Matrix $\vek K^{-1}$ sind gerade die Vektoren $\textcolor{chapterTitleBlue}{\vek g_i}$ und daher folgt
\beq
\vek u = (\textcolor{chapterTitleBlue}{\vek g_1^T}\,\vek f) \, \vek e_1 + (\textcolor{chapterTitleBlue}{\vek g_2^T}\,\vek f) \,\vek e_2 + \ldots + (\textcolor{chapterTitleBlue}{\vek g_n^T}\,\vek f)\,\vek e_n\,.
\eeq
Geht es um Funktionen, also die L\"{o}sungen von Differentialgleichungen, wie zum Beispiel
\beq
- EA\,u''(x) = p(x) \qquad u(0) = u(l) = 0\,,
\eeq
dann hat die Matrix $\vek K$  unendlich viele Spalten, und die Einheitsvektoren gehen in Dirac Deltas \"{u}ber
\beq
- EA\frac{d^2}{dy^2} \,\textcolor{chapterTitleBlue}{G(y,x)} = \textcolor{chapterTitleBlue}{\delta(y- x)}\,,
\eeq
aber der Formalismus ist derselbe. Indem wir die rechte Seite $p$ auf die L\"{o}sungen $\textcolor{chapterTitleBlue}{G(y,x)}$ projizieren, also das $L_2$-Skalarprodukt (Integral) der beiden Funktionen bilden, k\"{o}nnen wir den Wert der L\"{o}sung an jeder Stelle $x$ berechnen
\beq\label{Eq41}
u(x) = \underbrace{\int_0^{\,l} \textcolor{chapterTitleBlue}{G(y,x)}\,p(y)\,dy}_{\textcolor{chapterTitleBlue}{\vek g_i^T} \vek f} = \underbrace{\int_0^{\,l} \textcolor{chapterTitleBlue}{\delta(y-x)}\,u(y)\,dy}_{\vek u^T\,\textcolor{chapterTitleBlue}{\vek e_i}}\,.
\eeq
%-----------------------------------------------------------------
\begin{figure}[tbp]
\centering
\if \bild 2 \sidecaption \fi
\includegraphics[width=1.0\textwidth]{\Fpath/U44A}
\caption{Einflussfunktionen werden von Monopolen (linke Seite) bzw. Dipolen (rechte Seite) erzeugt,  Einflussfunktion f\"{u}r \textbf{ a)} Durchbiegung,  \textbf{ b)} Verdrehung $w,_x$, \textbf{ c)} Moment $m_{xx}$,  \textbf{ d)} Querkraft $q_x$ }\label{U44A}
\end{figure}%%
%-----------------------------------------------------------------

%%%%%%%%%%%%%%%%%%%%%%%%%%%%%%%%%%%%%%%%%%%%%%%%%%%%%%%%%%%%%%%%%%%%%%%%%%%%%%%%%%%%%%%%%%%%%%%%%%%
{\textcolor{sectionTitleBlue}{\section{Monopole und Dipole}}}\index{Monopole}\index{Dipole}
Die Einflussfunktion f\"{u}r die Verdrehung $w'$ eines Balkens wird durch ein Einzelmoment $M = 1 $ erzeugt
\beq
M = \lim_{\Delta x \to 0} \,\,\frac{1}{\Delta x}  \, \Delta x = 1\,,
\eeq
das man sich durch zwei gegengleiche Kr\"{a}fte, $P = \pm 1/\Delta x$, erzeugt denken kann, deren Abstand $\Delta x $ gegen null geht, w\"{a}hrend gleichzeitig die Kr\"{a}fte gegen unendlich gehen. In der Physik nennt man dies einen {\em Dipol\/}.

Die Einflussfunktion f\"{u}r eine Durchbiegung $w(x)$ hingegen wird von einem {\em Monopol\/}, einer Einzelkraft, erzeugt.

Einflussfunktionen, die von Monopolen erzeugt werden, summieren. Solche Einflussfunktionen gleichen Dellen oder Senken, s. Abb. \ref{U44A} und \ref{U77} a. Alles was in die Delle hineinf\"{a}llt, vergr\"{o}{\ss}ert die Durchbiegung der Platte.

Dipole hingegen erzeugen Scherbewegungen, die auf Ungleichgewichte reagieren, sie differenzieren, s. Abb. \ref{U44A} und \ref{U77} b.\\

\hspace*{-12pt}\colorbox{highlightBlue}{\parbox{0.98\textwidth}{Monopole integrieren und Dipole differenzieren.}}\\


%----------------------------------------------------------
\begin{figure}[tbp]
\centering
\includegraphics[width=1.0\textwidth]{\Fpath/U177}
\caption{Oberste Reihe Einflussfunktionen f\"{u}r \textbf{ a)} das Biegemoment und \textbf{ b)} die Querkraft in der Mitte des Balkens, \textbf{ c)} und \textbf{ d)} Momente und Querkr\"{a}fte unter symmetrischer Last und antimetrischer Last, \textbf{ e)} und \textbf{ f)}}
\label{U177}%
%
\end{figure}%%
%----------------------------------------------------------
%----------------------------------------------------------------------------------------------------------
\begin{figure}[tbp]
\centering
\includegraphics[width=1.0\textwidth]{\Fpath/U268}
\caption{Die Steigerung der Komplexit\"{a}t, \textbf{ a)} Durchbiegung $w$, \textbf{ b)} Momente $m_{yy}$, \textbf{ c)} Querkr\"{a}fte $q_y$ }
\label{U268}%
\end{figure}%%
%--------------------------------------------------------------------------------------------------
%----------------------------------------------------------
\begin{figure}[tbp]
\centering
\includegraphics[width=0.85\textwidth]{\Fpath/U77}
\caption{Deckenplatte Einflussfunktionen \textbf{ a)} f\"{u}r eine Durchbiegung ($G_0 = O(r^2\ln r)$), \textbf{ b)} f\"{u}r eine Querkraft ($G_3 = O(r^{-1})$), \textbf{ c)} f\"{u}r ein Moment ($G_2 = O(\ln r)$), s. (\ref{Eq150}) S. \pageref{Eq150} letzte Spalte der Matrix}
\label{U77}%
\end{figure}%%
%----------------------------------------------------------

Jede der vier Einflussfunktionen in Abb. \ref{U44A} geh\"{o}rt sinngem\"{a}{\ss} zu einem der beiden Typen:\\

\begin{itemize}
  \item E.F. f\"{u}r Durchbiegungen und Momente {\em summieren\/}.
  \item E.F. f\"{u}r Verdrehungen, Spannungen und Querkr\"{a}fte {\em  differenzieren\/}
\end{itemize}

Die Einflussfunktion f\"{u}r die Querkraft $V$ wird von einem Dipol erzeugt, w\"{a}hrend die Einflussfunktion f\"{u}r das Biegemoment $M$ von zwei entgegengesetzt drehenden Momenten $M = \pm 1/\Delta x$ erzeugt wird, die nach Innen drehen und so eine symmetrische Biegefigur aber mit einem scharfen Knick im Aufpunkt generieren\footnote{Genau genommen lautet die Folge: Monopol -- Dipol -- Quadropol -- Octopol, entsprechend den finiten Differenzen f\"{u}r $w, w', M, V$, s. Abb. \ref{U303} S. \pageref{U303}, aber f\"{u}r unsere Zwecke reicht das einfache Raster: Monopol -- Dipol oder symmetrisch-antimetrisch aus.}.\label{Korrektur6}
 \label{Footnote1}

Das maximale Ergebnis ergibt sich, wenn die Belastung und die Einflussfunktion vom selben Typ sind ({\em symmetrisch -- symmetrisch\/} oder {\em anti\-metrisch -- anti\-metrisch\/}) und der minimale Effekt, wenn sie vom entgegengesetzten Typ sind, siehe Abb. \ref{U177}.
\\

\hspace*{-12pt}\colorbox{highlightBlue}{\parbox{0.98\textwidth}{Der Unterschied zwischen Monopolen und Dipolen ist der Grund, warum es einfacher ist, Verschiebungen und Biegemomente anzun\"{a}hern, als Spannungen und Querkr\"{a}fte. Es ist der Unterschied zwischen  numerischer Integration und numerischer Differentiation, s. Abb. \ref{U268}.}}\\

\begin{remark} Alle Einflussfunktionen f\"{u}r Lagerreaktionen integrieren, obwohl die Lagerkr\"{a}fte ja Normalkr\"{a}fte (Spannungen) oder Querkr\"{a}fte sind und daher w\"{u}rden wir erwarten, dass die Einflussfunktionen differenzieren. Aber in einem festen Lager wird der eine Teil der Scherbewegung durch den Baugrund behindert, so dass der andere Teil den ganzen Weg allein gehen muss, um die vorgeschriebene Versetzung $[[u]] = 1$ zu realisieren und daher wird aus der Einflussfunktion eine einseitige Integration.
\end{remark}
%----------------------------------------------------------
\begin{figure}[tbp]
\centering
\includegraphics[width=0.8\textwidth]{\Fpath/U178}
\caption{\textbf{ a)} Gerbertr\"{a}ger, \textbf{ b)} Einflussfunktion f\"{u}r ein Moment $M$. Nicht alle Einflussfunktionen klingen ab! }
\label{U178}%
%
\end{figure}%%
%----------------------------------------------------------

\begin{remark}
Nicht alle Einflussfunktionen tendieren gegen null. Wenn Teile des Tragwerks (nach dem Einbau eines $N$-, $V$- oder $M$-Gelenkes) Starrk\"{o}rperbewegungen ausf\"{u}hren k\"{o}nnen, dann kann es passieren, dass sich die Einflussfunktionen aufschaukeln, siehe Abb. \ref{U178} b.
\end{remark}
%-------------%----------------------------------------------------------
\begin{figure}[tbp]
\centering
\includegraphics[width=1.0\textwidth]{\Fpath/U78}
\caption{Kragplatte, \textbf{ a)} Einflussfunktion f\"{u}r die Querkraft $q_x$ und \textbf{ b)} f\"{u}r das Moment $m_{xx}$; es ist erstaunlich, wie es mit einer \glq numerischen\grq{} Spreizung bzw. einem \glq numerischen\grq{} Knick (Randelemente) m\"{o}glich ist, einen fast konstanten Versatz bzw. eine Rotation von genau 45$^\circ$ zu erreichen. Frage: Wie nahe kann man dem Aufpunkt kommen, bevor die Singularit\"{a}t, $O(1/r)$ in Bild a und $O(\ln r)$ in Bild b, durchschl\"{a}gt?   }
\label{U78}%
\end{figure}%%
%------------------------------------------------------------------------------------------------------

%-------------%----------------------------------------------------------
\begin{figure}[tbp]
\centering
\includegraphics[width=0.9\textwidth]{\Fpath/U292}
\caption{Plattenbr\"{u}cke, \textbf{ a)} Einflussfunktion f\"{u}r das Moment $m_{xx}$ und \textbf{ b)} f\"{u}r die Querkraft $q_{x}$ in der Plattenmitte (also in einem Punkt); die Einflussfunktion f\"{u}r das Integral von $q_x$ quer durch die Mitte d\"{u}rfte mit der Balkenl\"{o}sung identisch sein.}
\label{U292}%
\end{figure}%%
%------------------------------------------------------------------------------------------------------
\begin{remark}
Das Abklingverhalten von Einflussfunktionen h\"{a}ngt von der Ordnung $d^n w/dx^n$ der Zielgr\"{o}{\ss}e ab. Beim Balken haben die Zielgr\"{o}{\ss}en
\begin{align}
w(x), \quad w'(x), \quad M(x) = - EI\,w''(x), \quad V(x) = - EI\,w'''(x)
\end{align}
die Ordnung $0, 1, 2, 3$. Je niedriger die Ordnung ist, um so weiter schwingt eine Einflussfunktion aus und um so langsamer klingt sie ab, wie man an der Einflussfunktion f\"{u}r die Durchbiegung $w(\vek x)$ der Platte sieht, s. Abb. \ref{U77} a, w\"{a}hrend die Einflussfunktion f\"{u}r die Querkraft $q_x$ sehr eng gefasst ist, s. Abb. \ref{U77} b. Es sind praktisch zwei gegengleiche Spitzen $\pm \infty$, die aus der Platte herausragen, die dann aber sehr rasch auf null abfallen.

Nat\"{u}rlich sind das nur \glq Trendmeldungen\grq{} und das genaue Verhalten h\"{a}ngt auch von der Art der Lagerung ab, s. Abb. \ref{U78} und Abb. \ref{U292}, denn gerade Kragtr\"{a}ger und Kragplatten spielen diesbez\"{u}glich eine Sonderrolle, weil sie freie Enden haben.
\end{remark}

Eine Sonderrolle spielen auch Einflussfunktionen f\"{u}r Kraftgr\"{o}{\ss}en an statisch bestimmten Systemen. Weil nach dem Einbau des Gelenks das System kinematisch ist, k\"{o}nnen sich die Verformungen frei ausbilden, denn es wird keine Energie verbraucht. Nichts kann die Einflussfunktion f\"{u}r das Moment in einem Kragtr\"{a}ger daran hindern den Schenkel rechts vom Aufpunkt unter $45^\circ$ bis \glq in den Himmel\grq{} laufen zu lassen, denn es kostet ja nichts. Deswegen st\"{u}rzen kinematische Strukturen auch so leicht ein, denn es ist keine Energie n\"{o}tig, um den Einsturz auszul\"{o}sen.

Statisch unbestimmte Systeme d\"{a}mpfen also die Ausbreitung der Einflussfunktionen f\"{u}r Kraftgr\"{o}{\ss}en, w\"{a}hrend bei statisch bestimmten Systemen eine solche Sperre fehlt.

%-----------------------------------------------------------------
\begin{figure}[tbp]
\centering
\if \bild 2 \sidecaption \fi
\includegraphics[width=.9\textwidth]{\Fpath/U82}
\caption{Scheibe, {\bf a)} Einflussfunktion f\"{u}r $N_y$ (exakt nach der gedehnten 1. Elementreihe), {\bf b)} Einflussfunktion f\"{u}r $\sigma_{yy}$, Kr\"{a}fte in kNm, Verschiebungen in m} \label{U82}
\end{figure}%
%-----------------------------------------------------------------
\vspace{-0.5cm}
%%%%%%%%%%%%%%%%%%%%%%%%%%%%%%%%%%%%%%%%%%%%%%%%%%%%%%%%%%%%%%%%%%%%%%%%%%%%%%%%%%%%%%%%%%%%%%%%%%%
{\textcolor{sectionTitleBlue}{\section{Einflussfunktionen f\"{u}r integrale Werte}}}\index{Einflussfunktionen f\"{u}r integrale Werte}

In einem Punkt fokussiert man den Blick auf einen einzelnen Wert des Moments, der Durchbiegung, der Querkraft, etc. Manchmal ist es jedoch sinnvoller, die Ergebnisse \"{u}ber eine k\"{u}rzere oder l\"{a}ngere Linie aufzuintegrieren, also zu mitteln, weil die Punktwerte zu stark schwanken.

Warum eine solche Mittelung bessere Ergebnisse liefert, versteht man, wenn man sich die unterschiedlichen Einflussfunktionen anschaut. Die Einflussfunktion f\"{u}r die Spannung $\sigma_{yy}$ in einem Punkt ist eine Spreizung des Aufpunktes in vertikaler Richtung, s. Abb. \ref{U82} b. Erweitern wir den Punkt zu einer kurzen Linie $\ell$ und entschlie{\ss}en uns mit dem Mittelwert der Spannungen l\"{a}ngs dieser Linie zu rechnen
\begin{align}
\bar{\sigma}_{yy} = \frac{1}{\ell } \int_0^{\,\ell} \sigma_{yy}\,ds \,,
\end{align}
dann ist die Einflussfunktion eine linienhafte Versetzung der Punkte auf der Linie und eine solche Bewegung ist einfacher mit finiten Elementen anzun\"{a}hern als eine Punktversetzung. Das ist der Grund, warum eine Mittelung in der Regel bessere Werte liefert.

Wenn, wie in Abb. \ref{U82} a, der Schnitt ganz durch die Scheibe geht, ist das Integral der Spannungen
\begin{align}
N_y = \int_0^{\,l} \sigma_{yy}\,dx
\end{align}
sogar exakt, weil der {\em lift\/}  in $\mathcal{V}_h^+$ (= $\mathcal{V}_h$ + Starrk\"{o}rperbewegungen) liegt. Dagegen d\"{u}rfte die Einflussfunktion f\"{u}r den Punktwert $\sigma_{yy}$ nur eine N\"{a}herung sein, denn so eckig sieht keine Einflussfunktion aus.

Einflussfunktionen f\"{u}r integrale Werte ordnen sich dem globalen Schema unter. Bei einem Punktfunktional wie $J(w) = w(x)$ sind die $j_i$ die Durchbiegungen der Ansatzfunktionen $\Np_i(x)$ im Aufpunkt
\begin{align}
\vek K\,\vek g = \vek j\,.
\end{align}
Ist $J(w)$ hingegen ein Integral, etwa der Mittelwert der Durchbiegung auf einer Strecke $(x_a, x_b)$,
\begin{align}
J(w) = \frac{1}{(x_b - x_a)}\int_{x_b}^{\, x_b} w(x)\,dx\,,
\end{align}
dann sind die \"{a}quivalenten Knotenkr\"{a}fte die Mittelwerte der $\Np_i$
\begin{align}
j_i = \frac{1}{(x_b - x_a)} \int_{x_b}^{\, x_b}  \Np_i(x)\,dx\,.
\end{align}
%----------------------------------------------------------------------------------------------------------
\begin{figure}[tbp]
\centering
\if \bild 2 \sidecaption \fi
\includegraphics[width=0.9\textwidth]{\Fpath/U408}
\caption{Einflussfunktion f\"{u}r den Mittelwert von $\sigma_{xx}$ in dem Element {\bf a)\/} das \lqq Dirac Delta\rqq \, besteht aus horizontalen Linienkr\"{a}ften auf dem vertikalen Rand und (kleinen, $\nu$-fachen) vertikalen Linienkr\"{a}ften auf dem horizontalen Rand von $\Omega_e$, {\bf b)\/}
horizontale Verschiebungen, nach oben und unten in $z$-Richtung abgetragen. Bei bilinearen Elementen sind die Einflussfunktionen f\"{u}r den Mittelwert im Element und von $\sigma_{xx}$ im Mittelpunkt des Elements identisch, \cite{Ha5}} \label{U408}
\end{figure}%%
%----------------------------------------------------------------------------------------------------------
Die mittlere Spannung $\sigma_{xx}^\varnothing$ in einem Element $\Omega_e$ ist das Integral
\bfo
\sigma_{xx}^\varnothing = \frac{1}{\Omega_e}\int_{\Omega_e} \sigma_{xx}\,d\Omega =
\frac{E}{\Omega_e}\int_{\Omega_e} (\varepsilon_{xx} + \nu\,\varepsilon_{yy})\,d\Omega\,.
\efo
Wegen $\varepsilon_{xx} = u_x,_x$ und $\varepsilon_{yy} = u_y,_y$, kann das Gebietsintegral durch ein Rand\-integral \"{u}ber den Rand $\Gamma_e$ des Elements ersetzt werden
\bfo
\sigma_{xx}^\varnothing = \frac{E}{\Omega_e}\int_{\Omega_e}  (\varepsilon_{xx} +
\nu\,\varepsilon_{yy})\,d\Omega = \frac{E}{\Omega_e}\int_{\Gamma_e} (u_x\,n_x +
\nu\,u_y\,n_y) \,ds\,.
\efo
Die Einflussfunktion f\"{u}r die Verschiebung $u_x$ bzw. $u_y$ eines Randpunktes $\vek x$ ist die Verschiebung, die durch eine Einzelkraft $P_x = 1$ bzw. $P_y = 1$ ausgel\"{o}st wird, die im Punkt $\vek x$ angreift. Daher ist die Einflussfunktion f\"{u}r das Integral
\bfo
\frac{E}{\Omega_e}\int_{\Gamma_e} (u_x\,n_x + \nu\,u_y\,n_y) ds
\efo
das Verschiebungsfeld, das durch horizontale bzw. vertikale Linienkr\"{a}fte $E/\Omega_e
\cdot n_x$ bzw. $E/\Omega_e \cdot n_y$ l\"{a}ngs des Elementrandes $\Gamma_e$ erzeugt wird, s. Abb. \ref{U408}. (Das $\Omega_e$ im Nenner ist nat\"{u}rlich die Fl\"{a}che des Elements.)

Daraus folgt, dass die mittleren Spannungen in einer Scheibe, die am Rand festgehalten wird, null sind, weil die Randkr\"{a}fte der Einflussfunktion die Scheibe nicht deformieren k\"{o}nnen.
%-----------------------------------------------------------------
\begin{figure}[tbp]
\centering
\if \bild 2 \sidecaption \fi
\includegraphics[width=0.6\textwidth]{\Fpath/U551}
\caption{Integralbeziehungen an einem Balken} \label{U551}
\end{figure}%
%-----------------------------------------------------------------
Sinngem\"{a}{\ss} dasselbe gilt f\"{u}r Platten: Die Mittelwerte der Momente einer allseits eingespannten Platte sind null. Im eindimensionalen Fall hatten wir das schon in Kapitel 1, Glg. (\ref{Eq31}) und Glg. (\ref{Eq33}), festgestellt.

Erfahrungsgem\"{a}{\ss} sind die Spannungen in der Mitte eines Elements am genauesten. Zum einen, weil man in der Mitte von den R\"{a}ndern des Elements, wo die FE-Spannungen springen, am weitesten entfernt ist, zum andern liegt es aber auch daran, dass, wenn man bilineare Elemente benutzt, die FE-Einflussfunktionen f\"{u}r die Spannungen in der Elementmitte die gleichen sind, wie f\"{u}r die Mittelwerte der Spannungen im Element. Letztere sind aber einfacher zu erzeugen, weil sie ja keine Punktversetzung simulieren m\"{u}ssen. Im output stehen also eigentlich die Mittelwerte der Spannungen, \cite{Ha5}.

%-----------------------------------------------------------------
\begin{figure}[tbp]
\centering
\if \bild 2 \sidecaption \fi
\includegraphics[width=1.0\textwidth]{\Fpath/U304}
\caption{Durchbiegung am Kragarmende aus {\bf a)} Einzelkraft -- dreimal integrieren, {\bf b)} Moment -- zweimal integrieren -- und {\bf c)} Streckenlast -- viermal integrieren} \label{U304}
\end{figure}%
%-----------------------------------------------------------------

Die partielle Integration ist auch eine Aussage \"{u}ber den Mittelwert der Ableitung einer Funktion
\begin{align}
\frac{1}{l}\int_{0}^{l} u'(x)\,dx = \frac{1}{l} (u(l) - u(0))
\end{align}
und sinngem\"{a}{\ss} ist daher das Integral der Normalkraft $N = EA\,u'(x)$ in einem Stab (wir lassen die Division durch die L\"{a}nge weg) proportional zur Spreizung der Endpunkte
\begin{align}
\int_{0}^{l} N(x)\,dx = EA\,(u(l) - u(0))\,.
\end{align}
Das Integral des Moments $M(x) = - EI\,w''(x)$ ist proportional zur Differenz der Endtangenten, s. Abb. \ref{U551},
\begin{align}
\int_{0}^{l} M(x) \,dx = - EI\,(w'(l) - w'(0)),
\end{align}
bei der Querkraft
\begin{align}
\int_{0}^{l} V(x)\,dx = M(l) - M(0)
\end{align}
ist es die Differenz der Endmomente und die Resultierende der Belastung $p = -V' = EI\,w^{IV}$ ist nat\"{u}rlich gerade die Differenz der Querkr\"{a}fte
\begin{align}
\int_{0}^{l} p\,dx = - (V(l) - V(0)) \,.
\end{align}

%%%%%%%%%%%%%%%%%%%%%%%%%%%%%%%%%%%%%%%%%%%%%%%%%%%%%%%%%%%%%%%%%%%%%%%%%%%%%%%%%%%%%%%%%%%%%%%%%%%
{\textcolor{sectionTitleBlue}{\section{Einflussfunktionen rechnen r\"{u}ckw\"{a}rts}}}\index{Einflussfunktionen rechnen r\"{u}ckw\"{a}rts}

Wenn man differenziert, dann geht man \glq vorw\"{a}rts\grq{} und wenn man integriert, dann geht man \glq r\"{u}ckw\"{a}rts\grq{}. Einflussfunktionen rechnen r\"{u}ckw\"{a}rts. Aus $- EA\,u'' = p$  bzw.  $EI\,w^{IV} = p$ werden die Ableitungen niedrigerer Ordnung
\begin{align}
u,\, N = EA\,u' \qquad w,\,w',\,M, \,V
\end{align}
berechnet.

Die Einflussfunktion $G_1(y,x)$ f\"{u}r die Normalkraft in einem Stab integriert die Belastung einmal
\begin{align}
N(x) = \int_0^{\,l} G_1(y,x)\,p(y)\,dy \qquad ('') \to (')
\end{align}
und die Einflussfunktion $G_0(y,x)$ f\"{u}r die L\"{a}ngsverschiebung $u(x) $ integriert die Belastung zweimal
\begin{align}
u(x) = \int_0^{\,l} G_0(y,x)\,p(y)\,dy \qquad ('') \to (\,\,)\,.
\end{align}
Das R\"{u}ckw\"{a}rtsrechnen sieht man sehr sch\"{o}n an dem Kragtr\"{a}ger in Abb. \ref{U304}. Die Durchbiegung $w$ ist ja das dreifach unbestimmte Integral der Querkraft $V = - EI\,w'''$
\begin{align}
w = -\int\! \int\! \int\, V\,dx\,dx\,dx = - \int\! \int\! \int\, P\,dx\,dx\,dx
\end{align}
und prompt steht ein $\ell^3$ im Ergebnis
\begin{align}\label{Eq129}
w(\ell) = \frac{P\,\ell^3}{3\,EI}\,,
\end{align}
und wenn ein Moment $M = - EI\,w''$ angreift, dann steht dort ein  $\ell^2$
\begin{align}
w(\ell) = \frac{M\,\ell^2}{2\,EI}\,.
\end{align}
Die letzte \glq vern\"{u}nftige\grq{}, integrierbare Funktion in der Kette der Ableitungen ist $w'''$ (LF $P$) bzw. $w''$ (LF $M$) und deswegen wird $w$  aus $V$ bzw. $M$ berechnet. W\"{u}rde statt $P$ eine Streckenlast $p$ angreifen, dann w\"{a}re
\begin{align}
w(\ell) = \frac{p\,\ell^4}{8\,EI}
\end{align}
und das $\ell^4$ passt zu $ EI\,w^{IV} = p$.

%-----------------------------------------------------------------
\begin{figure}[tbp]
\centering
\if \bild 2 \sidecaption \fi
\includegraphics[width=0.6\textwidth]{\Fpath/UE344}
\caption{Balkenrost} \label{UE344}
\end{figure}%
%-----------------------------------------------------------------

Wir finden das $l^3$ der Glg. (\ref{Eq129}) auch in der Formel
\begin{align}
\frac{P_a}{P_b} = \frac{l_b^3}{l_a^3}\,,
\end{align}
die erkl\"{a}rt, wie sich eine Punktlast $P = P_a + P_b$ auf zwei Balken verteilt, s. Abb. \ref{UE344}.

Eine Steifigkeitsmatrix, $\vek K\,\vek u = \vek f$, dagegen differenziert und daher finden wir den \glq inversen\grq{} Faktor $EI/l^3$ vor einer Balkenmatrix bzw. den Faktor $EA/l$ vor einer Stabmatrix.

Bei dem Weg zur\"{u}ck ist es wichtig zu wissen, wie man auf das $p$ gekommen ist. Man nehme die Funktion $u(x) = \sin (\pi\,x/l)$ und differenziere sie zweimal bzw. viermal
\begin{align}
- u'' &= (\frac{\pi}{l})^2 \sin (\pi\,x/l)  \qquad = p(x)\,\\
EI\,u^{IV} &= (\frac{\pi}{l})^4 \sin (\pi\,x/l) \qquad =\bar{p}(x)\,.
\end{align}
Im ersten Fall ist sie die Durchbiegung eines vorgespannten Seils unter einer Streckenlast $p(x)$ und im zweiten Fall ist sie die Durchbiegung eines Balkens unter einer Streckenlast $\bar{p}(x)$.

Um $u$ im Punkt $x = l/2$ aus den rechten Seiten $p(x)$ und $\bar{p}(x)$ zu berechnen, sind verschiedene Einflussfunktionen n\"{o}tig, obwohl wir nach demselben Wert fragen, $u(l/2) = \sin (0.5 \cdot \pi)$. Wir m\"{u}ssen wissen, welcher Operator $p$ aus $u$ erzeugt hat. Wo kommen die Daten her?


%%%%%%%%%%%%%%%%%%%%%%%%%%%%%%%%%%%%%%%%%%%%%%%%%%%%%%%%%%%%%%%%%%%%%%%%%%%%%%%%%%%%%%%%%%%%%%%%%%%
{\textcolor{sectionTitleBlue}{\section{Prinzip von St. Venant}}}\index{Prinzip von St. Venant}
{\em \glq Wenn die auf einen kleinen Teil der Oberfl\"{a}che eines elastischen K\"{o}rpers wirkende Kraft durch ein \"{a}quivalentes Kr\"{a}ftesystem ersetzt wird, ruft diese Belastungsumverteilung wesentliche \"{A}nderungen nur bei den \"{o}rtlichen Spannungen hervor: nicht aber in Bereichen, die gro{\ss} sind im Vergleich zur belasteten Oberfl\"{a}che'\/}, \cite{Wiki1}.

Dieses Prinzip ist eine direkte Konsequenz der Tatsache, dass Wirkungen per Einflussfunktionen propagieren. Einflussfunktionen sind Skalarprodukte, sind Integrale, die die Belastung $p$ mit einem Kern $G(y,x)$ wichten und der Kern hat (gew\"{o}hnlich) die Eigenschaft, dass er mit wachsendem Abstand vom Aufpunkt gegen null tendiert. Wenn der Abstand nur gro{\ss} genug ist kann man eine Ein-Punkt-Quadratur benutzen, d.h. man kann die Belastung durch ihre Resultierende ersetzen. Weil nun \"{a}quivalente Kr\"{a}ftesysteme dieselbe Resultierende haben, wirkt sich ein Austausch in der Ferne nicht aus.

Daraus folgt im \"{u}brigen, dass die Wirkungen von antimetrischen Lasten, von Lasten mit null Resultierender, besonders schnell abklingen. Ja wenn die Einflussfunktionen im Bereich der Belastung \glq flach\grq{} verl\"{a}uft, keine Steigung hat, dann ist der Einfluss sofort null, {\em Symmetrie $\times$ Antimetrie = 0\/}. \\

\hspace*{-12pt}\colorbox{highlightBlue}{\parbox{0.98\textwidth}{Antimetrische Belastungen \glq differenzieren\grq{} die Einflussfunktionen.}}\\

Dieser Effekt spielt bei den Kr\"{a}ften $f^+ $ in Kapitel 5 eine gro{\ss}e Rolle.

%-----------------------------------------------------------------
\begin{figure}[tbp]
\centering
\if \bild 2 \sidecaption \fi
\includegraphics[width=1.0\textwidth]{\Fpath/UE342}
\caption{Theorie II. Ordnung {\bf a)} Druckkraft $P$, {\bf b)} Einflussfunktionen f\"{u}r $w'(l)$ und {\bf c)} f\"{u}r $w(l)$} \label{UE342}
\end{figure}%
%-----------------------------------------------------------------

%%%%%%%%%%%%%%%%%%%%%%%%%%%%%%%%%%%%%%%%%%%%%%%%%%%%%%%%%%%%%%%%%%%%%%%%%%%%%%%%%%%%%%%%%%%%%%%%%%%
{\textcolor{sectionTitleBlue}{\section{Theorie II. Ordnung}}}\index{Theorie II. Ordnung}
Die Differentialgleichung f\"{u}r den Balken nach Theorie zweiter Ordnung lautet
\begin{align}\label{Eq164}
EI\,w^{IV}(x) + P\,w''(x) = p(x)
\end{align}
hierbei ist  $P$ die Druckkraft in dem Balken und $p(x)$ die Streckenlast, s. Abb. \ref{UE342} a. Dies ist eine lineare, selbstadjungierte Differentialgleichung vierter Ordnung mit konstanten Koeffizienten, aber das Problem ist, dass der Koeffizient $P$ Lastfall abh\"{a}ngig ist und daher h\"{a}ngt auch die Einflussfunktion von $P$ ab.

Je mehr $P$ sich der Knicklast $P_{crit}$ n\"{a}hert, desto mehr w\"{o}lben sich die Einflussfunktionen f\"{u}r die Verdrehung am Balkenende, Abb. \ref{UE342} b,  bzw. f\"{u}r die Durchbiegung, Abb. \ref{UE342} c, auf.

Diese Abh\"{a}ngigkeit von der Normalkraft $N$ in den einzelnen Stielen ist der Grund, warum es nicht m\"{o}glich ist, Einflussfunktionen f\"{u}r z.B. Hochregallager anzugeben. Erst muss die Gleichgewichtslage des Regals nach Theorie erster Ordnung gefunden werden und dann kann man diese L\"{o}sung iterativ korrigieren.

Im Prinzip ist die Theorie zweiter Ordnung ein nichtlineares Problem, wo die L\"{a}ngsverschiebung $u(x)$ und die seitliche Auslenkung $w(x)$ gem\"{a}{\ss} dem System
\begin{subequations}
\begin{align}
- EA \left(u' + \frac{1}{2}\, (w')^2\right)' &= p_x \\
EI\,w^{IV} - \left(EA (u' + \frac{1}{2}\, (w')^2)\,w'\right)' &= p_z
\end{align}
\end{subequations}
miteinander verkn\"{u}pft sind.

Nur wenn die Normalkraft
\begin{align}
N = EA (u' + \frac{1}{2}\, (w')^2)
\end{align}
konstant ist und bekannt ist, kann dieses System auf die Gleichung (\ref{Eq164}) reduziert werden. Man beachte, dass ein negatives $N$ ein positives $P$ in (\ref{Eq164}) ist.



%%%%%%%%%%%%%%%%%%%%%%%%%%%%%%%%%%%%%%%%%%%%%%%%%%%%%%%%%%%%%%%%%%%%%%%%%%%%%%%%%%%%%%%%%%%%%%%%%%%
\textcolor{chapterTitleBlue}{\chapter{Finite Elemente}}\index{finite Elemente}\label{Chap3}
%%%%%%%%%%%%%%%%%%%%%%%%%%%%%%%%%%%%%%%%%%%%%%%%%%%%%%%%%%%%%%%%%%%%%%%%%%%%%%%%%%%%%%%%%%%%%%%%%%%
Zur Vorbereitung auf das Thema finite Elemente und Einflussfunktionen wollen wir kurz die Grundlagen der finiten Elemente rekapitulieren.

%%%%%%%%%%%%%%%%%%%%%%%%%%%%%%%%%%%%%%%%%%%%%%%%%%%%%%%%%%%%%%%%%%%%%%%%%%%%%%%%%%%%%%%%%%%%%%%%%%%
{\textcolor{sectionTitleBlue}{\section{Das Minimum}}}
Das {\em Prinzip vom Minimum der potentiellen Energie\/} besagt, dass die Gleichgewichtslage eines Tragwerks die potentielle Energie des Tragwerks zum Minimum macht. Zur Konkurrenz zugelassen sind bei diesem Wettbewerb alle Funktionen, die die geometrischen Lagerbedingungen des Tragwerks erf\"{u}llen. Man nennt diese Menge \"{u}blicherweise $\mathcal{V}$ (wie  \glq Vorrat\grq{}).

So ist die Biegelinie des Seils in Abb. \ref{U33}
\begin{align}\label{Eq88}
- H\,w''(x) = p(x) \qquad w(0) = w(l) = 0 \qquad \text{$H$ = Vorspannung in dem Seil}\,,
\end{align}
der Sieger, wenn es darum geht, die potentielle Energie des Seils
\begin{align}
\Pi(w) = \frac{1}{2}\,\int_0^{\,l} \frac{V^2}{H}\,dx - \int_0^{\,l} p(x)\,w(x)\,dx \qquad (V = H\,w')
\end{align}
auf der Menge aller Funktionen, deren Durchbiegungen in den Aufh\"{a}ngepunkten null sind, $w(0) = w(l) = 0$, zum Minimum zu machen.

%----------------------------------------------------------
\begin{figure}[tbp]
\centering
\if \bild 2 \sidecaption[t] \fi
\includegraphics[width=.85\textwidth]{\Fpath/U33A}
\caption{FE-Berechnung eines Seils, \textbf{a)} System und Belastung, \textbf{ b)} Dach- oder H\"{u}tchenfunktionen, \textbf{ c)} FE-L\"{o}sung $w_h(x)$ ,
\textbf{ d)} Vergleich $w(x)$ und $w_h(x)$} \label{U33}
\end{figure}%%
%----------------------------------------------------------

Nun ist es nicht m\"{o}glich, den ganzen Raum $\mathcal{V}$ zu durchsuchen, um $w(x)$ zu finden, dazu ist er zu gro{\ss}, und so beschr\"{a}nken wir die Suche auf einen {\em endlichdimensionalen\/} Teilraum $\mathcal{V}_h \subset \mathcal{V}$ und erkl\"{a}ren den Sieger  $w_h$ des Wettbewerbs um das Minimum auf diesem Teilraum als die beste N\"{a}herung. Dies nennt man das {\em Verfahren von Ritz\/}\index{Verfahren von Ritz}.

Der Wettbewerb beginnt damit, dass wir das Seil in mehrere finite Elemente unterteilen. Das ist einfach ein St\"{u}ck Seil, (so sieht es der Ingenieur), bzw. ein St\"{u}ck der $x$-Achse, (so sieht es der Mathematiker) auf dem zwei lineare Funktionen definiert sind, die sogenannten {\em Element-Einheitsverformungen\/}, die die Auslenkung des linken bzw. des rechten Knotens des Elements beschreiben.  Indem man nun diese Verformungen \"{u}ber die Elementgrenzen hinweg geeignet fortsetzt, kann man \glq H\"{u}tchenfunktionen\grq{}\index{H\"{u}tchenfunktionen} konstruieren. Das sind st\"{u}ckweise lineare Verl\"{a}ufe $\Np_i(x)$, die in dem Knoten $x_i$  den Wert 1 haben und zu den Nachbarknoten hin auf null abfallen, s. Abb. \ref{U33}. Sie stellen die {\em Einheitsverformungen der Knoten\/}\index{Einheitsverformungen} dar, sie sind die {\em shape functions\/}.

Die vier Einheitsverformungen der vier innenliegenden Knoten bilden also den FE-Ansatz
\begin{align}\label{Eq112}
w_h(x) = w_1\,\Np_1(x) + w_2\,\Np_2(x) + w_3\,\Np_3(x) + w_4\,\Np_4(x)\,,
\end{align}
und wir bestimmen die Knotenverformungen $w_i$ so, dass die FE-L\"{o}sung die potentielle Energie
\begin{align}\label{Eq46}
\Pi(w_h)= \frac{1}{2}\,\int_0^{\,l} H\,(w_h')^2\,dx - \int_0^{\,l} p\,w_h\,dx
\end{align}
auf $\mathcal{V}_h$, das ist die Menge aller Seilecke, die sich mit den $\Np_i(x)$ darstellen lassen, zum Minimum macht.
%----------------------------------------------------------
\begin{figure}[tbp]
\centering
\if \bild 2 \sidecaption[t] \fi
\includegraphics[width=.7\textwidth]{\Fpath/U122}
\caption{FE-Modell eines Seils, \textbf{a)} Ansatzfunktionen, \textbf{ b)} Einflussfunktion (EF) f\"{u}r $w(x_1)$ und \textbf{ c)} f\"{u}r die Durchbiegung $w(x)$ im Zwischenpunkt, \textbf{ d)} die exakte Einflussfunktion f\"{u}r $w(x)$} \label{U122}
\end{figure}%%
%----------------------------------------------------------

Der Ansatz (\ref{Eq112}) gewinnt den Wettbewerb genau dann, wenn der Vektor $\vek w$ der Knotenverformungen (die \glq Adresse\grq{} des Ansatzes auf $\mathcal{V}_h$) dem Gleichungssystem $\vek K\,\vek w = \vek f$, oder
\beq\label{Eq69}
\frac{H}{l_e} \,  \left[\barr{r r r r} 2 & - 1 & 0 & 0 \\ - 1 & 2 & -1 & 0\\ 0 & -1 & 2 &-1 \\ 0 & 0 & -1 &2\earr\right]
\,\left[\barr{c} w_1 \\w_2 \\ w_3 \\ w_4 \earr \right] = \left[\barr{c} 1 \\ 1  \\
1  \\ 1 \earr \right]
\eeq
gen\"{u}gt. Die Elemente $k_{ij}$ der Steifigkeitsmatrix $\vek K$ sind die Wechselwirkungsenergien zwischen den Ansatzfunktionen
\begin{align}
k_{ij} = a(\Np_i,\Np_j) = \int_0^{\,l} H\,\Np_i'(x)\,\Np_j'(x)\,dx = \int_0^{\,l} \frac{V_i\,V_j}{H}\,dx\,,
\end{align}
und die \"{a}quivalenten Knotenkr\"{a}fte auf der rechten Seite, $f_i = 1$, sind die Integrale
\begin{align}
f_i = \int_0^{\,l} p(x)\,\Np_i(x)\,dx\, .
\end{align}
Das System (\ref{Eq69}) hat, bei einer Vorspannung $H = 1$ und einer Elementl\"{a}nge $l_e = 1$, die L\"{o}sung
\beq
w_1 = w_4 = 2 \qquad w_2 = w_3 =  3\,,
\eeq
und daher ist das Seileck
\beq\label{A11Resultat}
w_h(x) = 2 \cdot  \Np_1(x) + 3 \cdot \Np_2(x) + 3 \cdot \Np_3(x) + 2 \cdot \Np_4(x)
\eeq
auf $\mathcal{V}_h$ die beste Ann\"{a}herung  an die wahre Biegelinie $w(x)$.


%%%%%%%%%%%%%%%%%%%%%%%%%%%%%%%%%%%%%%%%%%%%%%%%%%%%%%%%%%%%%%%%%%%%%%%%%%%%%%%%%%%%%%%%%%%%%%%%%%%
{\textcolor{sectionTitleBlue}{\section{Warum die Knotenwerte beim Seil exakt sind}}}
Wenn man die FE-L\"{o}sung mit der exakten L\"{o}sung
\begin{align}
w(x) = \frac{1}{2}\,\cdot (5\,x - x^2)
\end{align}
vergleicht, dann f\"{a}llt auf, dass die FE-L\"{o}sung mit der exakten L\"{o}sung in den Knoten \"{u}bereinstimmt, $w_i = w(x_i)$. Das  liegt daran, dass das FE-Programm Einflussfunktionen benutzt, also die Durchbiegung des Seils mit der Formel\footnote{Wir schreiben k\"{u}rzer $G(y,x)$ statt $G_0(y,x)$ und werden das bei Gelegenheit \"{o}fter tun.}
\begin{align}
w(x) = \int_0^{\,l} G(y,x)\,p(y)\,dy
\end{align}
berechnet und die Einflussfunktionen der Knoten in $\mathcal{V}_h$\index{$\mathcal{V}_h$} liegen. $\mathcal{V}_h$ ist die Menge aller Biegelinien, die mit den $\Np_i(x)$ darstellbar sind.

Die Einflussfunktion f\"{u}r die Durchbiegung $w(x_1)$ im ersten Innenknoten ist  das Seileck $G(y,x_1)$, das sich ausbildet, wenn in dem Knoten $x_1$ eine Kraft $P = 1$ angreift, s. Abb. \ref{U122} b, und
%----------------------------------------------------------
\begin{figure}[tbp]
\centering
\if \bild 2 \sidecaption[t] \fi
\includegraphics[width=1.0\textwidth]{\Fpath/U189}
\caption{Hochbauplatte, \textbf{a)} System,  \textbf{ b)} Biegefl\"{a}che im LF $g$, \textbf{ c)} Einflussfunktion f\"{u}r die Durchbiegung $w$ in einem Knoten $\vek x$} \label{U189}
\end{figure}%%
%----------------------------------------------------------
dieses Seileck k\"{o}nnen die vier Ansatzfunktionen darstellen.

Das ist der Grund, warum in diesem, aber {\em auch in jedem anderen Lastfall\/}, die FE-L\"{o}sung mit der exakten L\"{o}sung im Knoten $x_1$ \"{u}bereinstimmt
\begin{align}
 w_h(x_1) = \int_0^{\,l} G(y,x_1)\,p(y)\,dy = A \cdot 1.0 = 2.0 \cdot 1.0 = w(x_1)\,.
\end{align}
Wir machen die Probe. Es sei $p(x) = \sin(\pi x/5)$, dann ist
\begin{align}
\vek f = \{0.569, \,0.920, \,0.920, \,0.569\}^T
\end{align}
und das System $\vek K\,\vek w = \vek f$ hat die L\"{o}sung $\vek w = \{1.489, 2.409, 2.409, 1.489\}^T$, was die Knotenwerte der exakten L\"{o}sung $w(x) = 25/\pi^2 \cdot \sin(\pi  x/5)$ sind.

Wenn der Aufpunkt aber zwischen zwei Knoten liegt wie in Abb. \ref{U122} c, er liegt im Punkt $x = 1.5$, dann hat das zu dem Punkt geh\"{o}rige Seileck seine Spitze zwischen den beiden Knoten, und ein solches Dreieck kann man mit den vier Ansatzfunktionen nicht darstellen. Das FE-Programm verbindet daher die beiden Knoten links und rechts vom Aufpunkt mit einer geraden Linie und rechnet mit dieser N\"{a}herung $G_h(y,x)$
\begin{align}
w_h(x) = \int_0^{\,l} G_h(y,x)\,p(y)\,dy = A_h \cdot 1.0 = 2.5 \neq 2.75 = w(x)\,,
\end{align}
und so ist das Ergebnis nat\"{u}rlich auch nur eine N\"{a}herung, $w_h(x) = 2.5$ m, w\"{a}hrend die exakte Durchbiegung den Wert $w = 2.75$ m hat.

Nun wird man einwenden wollen: ein FE-Programm berechnet doch die Knotenwerte durch L\"{o}sen des Gleichungssystems $\vek K\,\vek w = \vek f$ und die Werte dazwischen findet es, indem es zwischen den Knoten interpoliert.

Das ist richtig, aber die Werte in dem Vektor $\vek w$ sind genauso gro{\ss}, {\em als ob\/} das FE-Programm sie mit den gen\"{a}herten Einflussfunktionen berechnet h\"{a}tte. Das ist der entscheidende Punkt. Von der klassischen Statik zu den finiten Elementen ist es ein ganz, ganz kurzer Weg.

Und diese Vorgehensweise ist nat\"{u}rlich nicht auf die Stabstatik beschr\"{a}nkt. So hat das FE-Programm die Biegefl\"{a}che der Platte in Abb. \ref{U189} (theoretisch) so berechnet, dass es in jeden Knoten $\vek x_i$ nacheinander eine Kraft $P = 1$ gestellt hat und die sich dabei ausbildende Biegefl\"{a}che $G_h(\vek y,\vek x_i)$ mit dem Eigengewicht $g$ \"{u}berlagert hat
\begin{align}\label{Eq78}
w_h(\vek x_i) = \int_{\Omega} G_h(\vek y, \vek x_i)\,g(\vek y)\,\,d\Omega_{\vek y} = \text{Volumen von $G_h$ $\times\,g$}\,.
\end{align}
Wir sagen theoretisch, weil nat\"{u}rlich das FE-Programm die Knotenwerte durch das L\"{o}sen von $\vek K\,\vek w = \vek f$ bestimmt hat, aber diese sind  genau so gro{\ss}, {\em als ob\/} das FE-Programm die Einflussfunktion (\ref{Eq78}) benutzt h\"{a}tte.

Das System $\vek K\,\vek w = \vek f$ ist der \glq kurze\grq{} Weg zu den $w_i$, die Formel (\ref{Eq78}) ist der \glq lange\grq{} Weg, aber die Ergebnisse sind dieselben\footnote{Wenn es der Leser nicht glaubt, kann er die rechte Seite von (\ref{Eq189}) partiell integrieren}
\begin{align}\label{Eq189}
w_h(\vek x_i) = w_i = \sum_j\,k_{ij}^{(-1)}\,f_j = \int_{\Omega} G_h(\vek y, \vek x_i)\,g(\vek y)\,\,d\Omega_{\vek y}\,.
\end{align}
Dies ist das geheime, wenig bekannte Gesetz hinter den finiten Elementen. \\

\hspace*{-12pt}\colorbox{highlightBlue}{\parbox{0.98\textwidth}{So gut, wie die Einflussfunktionen sind, so gut sind die FE-Ergebnisse.}}\\

%----------------------------------------------------------
\begin{figure}[tbp]
\centering
\if \bild 2 \sidecaption[t] \fi
\includegraphics[width=.8\textwidth]{\Fpath/U163}
\caption{Seilberechnung mit zwei Elementen,  \textbf{a)} Belastung und Biegelinie, \textbf{ b)} FE-L\"{o}sung + lokale L\"{o}sungen, \textbf{ c)} lokale L\"{o}sungen, \textbf{ d)} Einheitsverformung des Knotens. Bemerkenswert ist, dass die Tangente im Mittenknoten automatisch stetig ist (kein Knick!), kein Sprung in der Querkraft $V = H\,w'$} \label{U163}
\end{figure}%%
%----------------------------------------------------------

%%%%%%%%%%%%%%%%%%%%%%%%%%%%%%%%%%%%%%%%%%%%%%%%%%%%%%%%%%%%%%%%%%%%%%%%%%%%%%%%%%%%%%%%%%%%%%%%%%%
{\textcolor{sectionTitleBlue}{\section{Addition der lokalen L\"{o}sung}}}\index{lokale L\"{o}sung}
Wenn man die Durchbiegung des Seils mit einem FE-Programm berechnet, dann sieht man auf dem Bildschirm kein Seileck, sondern eine wohl geschwungene Parabel zweiten Grades, also die exakte Kurve. Wie macht das das FE-Programm? Das Programm geht genau so vor, wie wir das beschrieben haben:\\

\begin{itemize}
  \item Es unterteilt das Seil in kleine Elemente.
  \item Es reduziert die Belastung in die Knoten, es berechnet also die $f_i$\,.
  \item Es l\"{o}st das Gleichungssystem $\vek K\,\vek w = \vek f$\,.
\end{itemize}
Wenn es jetzt stehen bleiben w\"{u}rde, dann w\"{u}rde man auf dem Bildschirm ein Seileck sehen.

Es folgt nun aber noch ein weiterer Schritt. Das Programm berechnet f\"{u}r jedes Element die sogenannte {\em lokale L\"{o}sung\/}\index{lokale L\"{o}sung} $w_{loc}$. Das ist die Durchbiegung, die die Streckenlast an dem {\em beidseitig festgehaltenen Element\/} erzeugt, und diese wird elementweise zu dem Seileck addiert. So ist die exakte Seilkurve in Abb. \ref{U163} entstanden.

Im Grund ist das genau die Technik des Drehwinkelverfahrens. Das Drehwinkelverfahren reduziert alle Belastung in die Knoten, f\"{u}hrt dann einen Knotenausgleich durch und h\"{a}ngt zum Schluss feldweise die lokalen L\"{o}sungen ein.

Die finiten Elemente machen es nicht anders, denn das Gleichungssystem, das aufgestellt wird, lautet eigentlich
\begin{align}\label{Eq67}
\vek K\,\vek w = \vek f_K + \vek d\,.
\end{align}
Die $f_{K @i}$ sind die Einzelkr\"{a}fte, die direkt in den Knoten angreifen und die $d_i$ sind die in die Knoten reduzierten Lasten, die Lagerdr\"{u}cke aus der {\em domain load\/}.  Die Begriffe Lagerdruck ({\em actio\/})\index{Lagerdr\"{u}cke} und Festhaltekraft ({\em reactio\/}) sind spiegelbildlich. Die Lagerdr\"{u}cke $\times (-1)$ sind die Festhaltekr\"{a}fte.

In der FE-Literatur wird der Unterschied zwischen den $f_{K @i}$ und $d_i$ normalerweise verwischt, steht $f_i \equiv f_{K @i} + d_i$  f\"{u}r beide Anteile. Die $f_i$ in dem obigen Beispiel sind eigentlich die $d_i$, also die in die Knoten reduzierte Streckenlast,
\begin{align}\label{Eq93}
d_i^e = \int_0^{\,l_e} p\,\Np_i^e\,dx \qquad \Np_i^e = \text{Element-Einheitsverformungen}\,,
\end{align}
w\"{a}hrend die echten $f_{K @i}$ null sind, weil keine Einzelkr\"{a}fte in den Knoten angreifen.

Der Ingenieur wendet die Formel (\ref{Eq93}) links und rechts vom Knoten an, und addiert dann die beiden Beitr\"{a}ge $d_i^{L} + d_i^{R}$, w\"{a}hrend die finiten Elemente die Belastung $p$ mit den Knoteneinheitsverformungen $\Np_i = \Np_i^{L} + \Np_i^{R}$ gleich \glq in einem St\"{u}ck\grq{} \"{u}berlagern
\begin{align}
d_i = \int_0^{\,l} p\,\Np_i\,dx = d_i^{L} + d_i^{R}\,.
\end{align}
Die enge Verwandtschaft der finiten Elemente mit dem Drehwinkelverfahren beruht auf dieser Formel, denn die Einflussfunktionen f\"{u}r die \"{a}quivalenten Lagerkr\"{a}fte $d_i$ sind genau die Element-Einheits\-verformungen $\Np_i^e$. Ob man die Belastung in die Knoten reduziert (Drehwinkelverfahren) oder die \"{a}quivalenten Knotenkr\"{a}fte berechnet, ist dasselbe.\\

\hspace*{-12pt}\colorbox{highlightBlue}{\parbox{0.98\textwidth}{Bei Stabtragwerken ist die Methode der finiten Elemente mit dem Drehwinkelverfahren identisch. }}\\

Das L\"{o}sen des Gleichungssystems $\vek K\,\vek w = \vek  f_K + \vek d$ entspricht einem Knotenausgleich in einem Schritt. Elementweise werden dann nur noch die lokalen L\"{o}sungen dazu addiert.

So gelingt es also den finiten Elementen trotz ihrer beschr\"{a}nkten Kinematik, d.h. der Verwendung von
\begin{itemize}
  \item linearen Ans\"{a}tzen f\"{u}r die Element-L\"{a}ngsverschiebungen
  \item kubischen Polynomen f\"{u}r die Element-Durchbiegungen
\end{itemize}
die exakten Verformungen zu generieren; die lokalen L\"{o}sungen bringen den fehlenden \glq Schwung\grq{} in die Verformungsfigur. Die $u_{loc}$ bzw. $w_{loc}$ stehen in einer (aus der Statik-Literatur \"{u}bernommen) Bibliothek des FE-Programms und werden von dort bei Bedarf abgerufen.

%Bei Seilen und St\"{a}ben (L\"{a}ngsverschiebungen) sieht man die lokalen L\"{o}sungen, wenn man die Knotenwerte mit einem Lineal verbindet. Dann sind die Teile, die man noch zu den geraden Linien addieren  muss, die lokalen L\"{o}sungen.

All dies gilt genau genommen nur, wenn $EA$ bzw. $EI$ konstant sind, weil nur dann die Element-Einheitsverformungen $\Np_i^e$ homogene L\"{o}sungen der Stab- bzw. Balkendifferentialgleichung sind. Bei gevouteten Tr\"{a}gern liefern die finiten Elementen  also nur eine N\"{a}herung, was aber auch f\"{u}r das Drehwinkelverfahren gilt, denn die exakte Reduktion der Belastung in die Knoten bei gevouteten Tr\"{a}gern beherrscht auch das Drehwinkelverfahren nicht. Ganz zu schweigen von der Kenntnis der exakten Fortleitungszahlen in einem solchen Fall.

Die \"{A}quivalenz {\em Finite Elemente = Drehwinkelverfahren\/} bedeutet aber auch, dass es keinen Sinn macht, die einzelnen Stiele und Riegel eines Rahmens weiter in Elemente zu unterteilen. Es bringt nichts an Genauigkeit.


\begin{remark}
Die finiten Elemente werden gerne am Balken erkl\"{a}rt. Wir machen das auch. Damit die finiten Elemente aber finite Elemente bleiben, m\"{u}ssen wir uns darauf verst\"{a}ndigen, dass alle diese Demonstrationen sich auf den Zeitpunkt beziehen, {\em bevor\/} die lokale L\"{o}sung zur FE-L\"{o}sung addiert wird.
\end{remark}
%----------------------------------------------------------------------------------------------------------
\begin{figure}[tbp]
\centering
\if \bild 2 \sidecaption \fi
\includegraphics[width=1.0\textwidth]{\Fpath/UE338A}
\caption{Lokales und globales Koordinatensystem} \label{UE338}
\end{figure}%
%----------------------------------------------------------------------------------------------------------
\vspace{-1cm}
{\textcolor{blue}{\subsubsection*{Beispiel}}}
Da es wichtig ist, dieses Vorgehen zu verstehen, wollen wir die einzelnen Schritte an Hand des Systems in Abb.  \ref{UE338} erl\"{a}utern.

Im ersten Schritt stellt das FE-Programm die globale, nicht reduzierte Steifigkeitsmatrix $ \vek K_G$ der Gr\"{o}{\ss}e $9 \times 9$ auf. Es geht dabei (theoretisch) wie folgt vor:  Es schreibt das System zun\"{a}chst in entkoppelter Form (12 Gleichungen)
\begin{align}
\left[ \barr{c c} \vek K_1 & \vek 0 \\ \vek  0 & \vek K_2 \earr \right] \,\left[ \barr{c}\vek u_1 \\ \vek u_2 \earr \right] =\left[ \barr{c} \vek f_1 \\ \vek f_2\earr \right] + \left[ \barr{c} \vek d_1\\ \vek d_2\earr \right]
\end{align}
wobei $\vek K_i$ die Elementmatrizen sind, s. (\ref{Eq166}), und die Vektoren $\vek u_i, \vek f_i$ und $\vek d_i$ die zugeh\"{o}rigen Weg- und Kraftgr\"{o}{\ss}en an den Balkenenden sind. Die Kopplung der Balkenendverformungen $\vek u_i$ an die Knotenverformungen $\vek u$ kann man durch eine Matrix $\vek A$ beschreiben
\begin{align}
 \left[ \barr{c}\vek u_1 \\ \vek u_2 \earr \right]_{(12)} = \vek A_{(12 \times 9)}\,\vek u_{(9)}\,.
\end{align}
Gleichzeitig m\"{u}ssen die Balkenendkr\"{a}fte $\vek f_i + \vek d_i$ in jedem Knoten mit den \"{a}u{\ss}eren Knotenkr\"{a}ften $\vek f$ im Gleichgewicht sein, was gerade die Gleichung
\begin{align}
\vek A^T_{(9 \times 12)}\,(\left[ \barr{c} \vek f_1 \\ \vek f_2\earr \right]_{(12)} + \left[ \barr{c} \vek d_1\\ \vek d_2\earr \right]_{(12)}) = \vek f_{(9)}
\end{align}
ist und so kommt man auf die Beziehung
\begin{align}
\vek A^T \left[ \barr{c c} \vek K_1 & \vek 0 \\ \vek  0 & \vek K_2 \earr \right]\,\vek A\,\vek u = \vek K_G\,\vek u = \vek f
\end{align}
wobei die Matrix links die globale, nicht reduzierte Steifigkeitsmatrix $\vek K_G$ des Rahmens ist. Im Anschluss streicht das Programm die Spalten und Zeilen der Matrix $ \vek K_G$, die gesperrten Freiheitsgraden entsprechen (praktisch geht es anders vor), und reduziert so die Matrix $\vek K_G$ auf eine $5 \times 5$ Matrix $\vek K$.

Im zweiten Schritt berechnet das FE-Programm f\"{u}r jedes Element die \"{a}quivalenten Knotenkr\"{a}fte aus der verteilten Belastung\footnote{Ein oberer Index $e$ bedeutet, dass sich die Gr\"{o}{\ss}e auf das lokale KS des Elements bezieht}
\begin{align}
d_{i}^{@e} = \int_0^{\,l_e} p^e(x)\,\Np_i^{@e}(x)\,dx\,.
\end{align}
Die Notation ist symbolisch zu nehmen, weil $p^e(x)$ zwei Richtungen haben kann, in lokale $x_e$- oder lokale $z_e$-Richtung und die Einheitsverschiebungen $\Np_i^{@e}(x)$ sind entsprechend die korrespondierenden Verschiebungen, zwei in $x_e$-Richtung und vier in $z_e$-Richtung.

Eine $6 \times 6$ Matrix $\vek T_e$ ($\ldots \sin\,\alpha, \cos\,\alpha, \ldots$), s. (\ref{Eq72}), transformiert diese Vektoren $\vek d^e$ in das globale Koordinatensystem $\vek d_e = \vek T_e\,\vek d^e$, wo dann aus den $2 \times 6$ Komponenten die 5 Komponenten des Vektors
\begin{align}
\vek d\leftarrow \{\vek d_1,\, \vek d_2\}
\end{align}
zusammengestellt werden, der die \"{a}quivalenten Knotenkr\"{a}fte aus der verteilten Belastung in die 5 Richtungen $u_i$ enth\"{a}lt.

Im zentralen, dritten Schritt l\"{o}st das Programm das System
\begin{align}\label{Eq160}
\vek K\,\vek u = \vek f_K + \vek d
\end{align}
und bestimmt damit den Vektor $\vek u$ der Knotenverschiebungen. Der Vektor $\vek f_K$ enth\"{a}lt nur Nullen, au{\ss}er $f_{K @3} = P$.

Im vierten Schritt werden die Knotenverschiebungen $u_i$ aus dem globalen Koordinatensystem in die lokalen Koordinatensysteme der einzelnen Elemente transformiert,
\begin{align}
\vek u^e = \vek T_e^T \vek u \qquad (\vek  T_e^T = \vek T_e^{-1})\,.
\end{align}
Im f\"{u}nften Schritt bestimmt das Programm f\"{u}r jedes Element anhand des Systems (jetzt wird im lokalen KS gerechnet)
\begin{align}
\vek K^e\,\vek u^e = \vek f^e + \vek d^e
\end{align}
den Vektor der Balkenendkr\"{a}fte $\vek f^e$
\begin{align}
\vek f^e = \vek K^e\,\vek u^e - \vek d^e\,,
\end{align}
wobei $\vek K^e$ die Elementmatrix der Gr\"{o}{\ss}e $6 \times 6$ ist, (\ref{Eq167}). Man beachte, dass die $f_i^e$ Balkenendkr\"{a}fte sind, w\"{a}hrend die $f_{K @i}$ in (\ref{Eq160}) Knotenkr\"{a}fte sind.

Wenn man die Balkenendkr\"{a}fte $f_1^{@e}, f_2^{@e}, f_3^{@e}$ am Anfang des Elements kennt und die Streckenlast, dann kann man mit den Gleichgewichtsbedingungen die Schnittgr\"{o}{\ss}en $N(x), M(x)$ und $V(x)$ jedem Punkt $x$ des Elements berechnen.

Im letzten sechsten Schritt addiert das FE-Programm schlie{\ss}lich die lokale L\"{o}sung zur FE-L\"{o}sung und kommt so zur selben L\"{o}sung wie das Drehwinkelverfahren.

In einem einzelnen Balken entspricht die ganze Prozedur der Aufteilung der Biegelinie $w(x)$ in eine homogene und eine partikul\"{a}re L\"{o}sung
\begin{align}
w(x) = w_h(x) + w_p(x)\,,
\end{align}
wobei die homogene L\"{o}sung eine Entwicklung nach den Einheitsverformungen ist
\begin{align}
w_h(x) = \sum_i\,u_i\,\Np_i(x)\,,
\end{align}
und die partikul\"{a}re L\"{o}sung ist die Biegelinie am eingespannten Balken.
In Glg. (\ref{Eq159}), S. \pageref{Eq159}, haben wir diese Aufspaltung vorgenommen.

Nach diesem kurzen Ausflug in die Notation $\vek K\,\vek u = \vek f_K + \vek d$  wollen wir im Folgenden zur \"{u}blichen Notation $\vek K\,\vek  u = \vek f$ zur\"{u}ckkehren, bei der die rechte Seite f\"{u}r $\vek f \equiv \vek f_K + \vek d$ steht.

Wir werden auch meist die Knotenwerte mit $u_i$ bezeichnen, auch wenn es Ergebnisse aus einer Balkenberechnung sind, weil $\vek K\,\vek u = \vek f$ die Standardnotation ist.


%----------------------------------------------------------------------------------------------------------
\begin{figure}[tbp]
\centering
\if \bild 2 \sidecaption \fi
\includegraphics[width=.9\textwidth]{\Fpath/U161}
\caption{Der Fehlervektor $\vek e$ steht senkrecht auf der Ebene, auf die projiziert wird, und alle Vektoren, die sich in Projektionsrichtung von $\vek x$ nicht unterscheiden, haben dasselbe Bild} \label{U161}
\end{figure}%
%----------------------------------------------------------------------------------------------------------
%%%%%%%%%%%%%%%%%%%%%%%%%%%%%%%%%%%%%%%%%%%%%%%%%%%%%%%%%%%%%%%%%%%%%%%%%%%%%%%%%%%%%%%%%%%%%%%%%%%
{\textcolor{sectionTitleBlue}{\section{Projektion}}}
Wir hatten oben das System $\vek K\,\vek u = \vek f$ aus dem Prinzip vom Minimum der potentiellen Energie hergeleitet. Die Projektion der exakten L\"{o}sung auf den Ansatzraum $\mathcal{V}_h$, also das {\em Galerkin-Verfahren\/}, f\"{u}hrt jedoch, wie wir zeigen wollen, auf dasselbe System.

Das Bild bei der Projektion eines Vektors $\vek x =  \{x_1, x_2, x_3\}^T$ auf die $x\!-\!y$-Ebene ist sein Schatten $\vek x'$, s. Abb. \ref{U161}. Wir wissen nat\"{u}rlich, wo der Schatten hinf\"{a}llt, aber der Computer hat keine Augen, er rechnet. Er macht f\"{u}r den Schatten $\vek x'$ den Ansatz $\vek x' = c_1\,\vek e_1 + c_2\,\vek e_2$ und bestimmt $c_1$ und $c_2$ so, dass der Fehler senkrecht auf $\vek e_1$ und $\vek e_2$ steht
\begin{align}
(\vek x - \vek x')^T\, \vek e_i = 0 \qquad i = 1,2 \qquad \Rightarrow \qquad c_1 = x_1\,, c_2 = x_2\,,
\end{align}
was gleichbedeutend damit ist, dass der Schatten der Vektor in der Ebene ist, der den kleinstm\"{o}glichen Abstand
\begin{align}
|\vek e| = |\vek x - \vek x'| = \text{Minimum}
\end{align}
von $\vek x$ hat. Dies kann man auch schreiben als (wir vergessen einmal die Wurzel zu ziehen)
\begin{align}
|\vek e|^2 = (\vek x - \vek x')^T\,(\vek x - \vek x') = \text{Minimum}\,,
\end{align}
so sieht man besser die Verwandtschaft mit (\ref{Eq109}), denn auch bei den finiten Elementen handelt es sich, wenn man es als {\em Galerkin-Verfahren\/} interpretiert, um ein Projektionsverfahren. Nur dass die Metrik eine andere ist, nicht das Skalarprodukt zwischen zwei Vektoren, sondern die Wechselwirkungsenergie zwischen zwei Funktionen ist das Ma{\ss}.

Beim {\em Galerkin-Verfahren\/} w\"{a}hlt man als beste N\"{a}herung die Projektion der exakten L\"{o}sung $w$ auf den Ansatzraum $\mathcal{V}_h$, also die Funktion $w_h$ in $\mathcal{V}_h$, deren Fehler $w - w_h$ senkrecht auf allen $\Np_i$ steht
\begin{align}\label{Gortho}
 a(w - w_h,\Np_i) = 0 \qquad i = 1,2, \ldots, n\,.
\end{align}
Auf Grund der ersten Greenschen Identit\"{a}t, etwa des Seils,
\begin{align}\label{Eq89}
\text{\normalfont\calligra G\,\,}(w,\Np_i) = \int_0^{\,l} p\,\Np_i\,dx - a(w,\Np_i) = 0 \quad \Rightarrow \quad f_i = \int_0^{\,l} p\,\Np_i\,dx = a(w,\Np_i)
\end{align}
ist dies \"{a}quivalent mit
\begin{align}
 a(w_h,\Np_i) = f_i \qquad i = 1,2, \ldots, n
\end{align}
und diese $n$ Gleichungen entsprechen genau dem System $\vek K\,\vek u = \vek f$.

Diese Eigenschaft impliziert, dass $w_h$ auf $\mathcal{V}_h$ den kleinstm\"{o}glichen Abstand in der Energiemetrik,
\begin{align}\label{Eq109}
a(w-w_h,w-w_h) = \text{Minimum}
\end{align}
von $w$ hat, was praktisch bedeutet, dass $w_h$ im Sinne des Fehlerquadrats die kleinstm\"{o}gliche Abweichung in den Schnittgr\"{o}{\ss}en aufweist, \cite{Ha5}, S. 572.

Was man in Abb. \ref{U161} auch sieht ist, dass die nochmalige Projektion des Bildes $\vek x'$ nichts bringt, die Projektion bleibt stehen. Das ist auch der Grund, warum die Strategie, die Abweichung  $p - p_h$ in den Lasten im Sinne einer Korrektur nachtr\"{a}glich auf die Struktur aufzubringen, zu nichts f\"{u}hrt, die zu dem Lastfall $p - p_h$ geh\"{o}rigen $f_i$ sind null, weil der Schatten des Fehlers $\vek e$ der Nullvektor ist.

Glg. (\ref{Gortho}) ist die {\em Galerkin-Orthogonalit\"{a}t\/}\index{Galerkin-Orthogonalit\"{a}t}. Wegen $\delta A_i = \delta A_a$ kann man sie auch als Orthogonalit\"{a}t in den \"{a}u{\ss}eren Arbeiten schreiben
\begin{align}
a(w-w_h,\Np_i) = \int_0^{\,l} (p - p_h)\,\Np_i\,dx = 0\,.
\end{align}
Die Differenz zwischen dem Originallastfall $p$ und dem FE-Lastfall $p_h$ (dessen L\"{o}sung $w_h$ ist), ist also orthogonal zu allen $\Np_i$, d.h. die Fehlerkr\"{a}fte $p - p_h$ leisten keine Arbeit, wenn man an dem Seil mit den $\Np_i$ wackelt.


%%%%%%%%%%%%%%%%%%%%%%%%%%%%%%%%%%%%%%%%%%%%%%%%%%%%%%%%%%%%%%%%%%%%%%%%%%%%%%%%%%%%%%%%%%%%%%%%%%%
{\textcolor{sectionTitleBlue}{\section{\"{A}quivalente Knotenkr\"{a}fte}}}\index{aequivalente Knotenkr\"{a}fte}
Die $f_i $ auf der rechten Seite des Gleichungssystems $\vek K\,\vek u = \vek f $ sind keine Kr\"{a}fte, sondern Arbeiten\footnote{F\"{u}r eine alternative Interpretation in der Stabstatik s. S. \pageref{Dimensionsbetrachtung}}
\begin{align}
f_i = \int_0^{\,l} p(x)\,\Np_i(x)\,dx = [F/L] \cdot [L] \cdot [L] = [F \cdot L]\,,
\end{align}
und auch auf der linken Seite stehen Arbeiten, denn der einzelne Eintrag $k_{ij}$ in der Steifigkeitsmatrix beruht, wenn wir einen Stab als Beispiel nehmen, auf der Formel
\begin{align}
k_{ij } = \int_0^{\,l} EA\,\Np_i'\,\Np_j'\,dx = [F/L^2 \cdot L^2] \cdot [L/L] \cdot [L/L] \cdot [L] = [F \cdot L]\,,
\end{align}
und entsprechend auch beim Balken
\begin{align}
k_{ij } = \int_0^{\,l} EI\,\Np_i''\,\Np_j''\,dx = [F \cdot L^2] \cdot [1/L] \cdot [1/L] \cdot [L] = [F \cdot L]\,.
\end{align}
Bei jeder Ableitung wird mit $[L]^{-1}$ multipliziert
\begin{align}
\Np_i\,\,[L] \qquad \Np_i' = \frac{d\,\Np_i}{dx} = [\phantom{L}] \qquad \Np_i'' = \frac{d\,\Np_i'}{dx} = \frac{1}{[L]}\,.
\end{align}
Bei dreidimensionalen Problemen f\"{u}hrt die Dimensionsbetrachtung auf
\begin{align}
\int_{\Omega} \sigma_{ij}\,\varepsilon_{ij} \,dV = \frac{[F]}{[L^2]}\,\frac{[L]}{[L]} \,[L^3] = [F \cdot L]\,,
\end{align}
woraus bei Scheibenproblemen, $d = $ Dicke der Scheibe und $ dV = d\, d\Omega$, der Ausdruck
\begin{align}
\int_{\Omega} \sigma_{ij}\,\varepsilon_{ij}\, d \,d\Omega = \frac{[F]}{[L^2]}\,\frac{[L]}{[L]} \,[L \cdot L^2] = [F \cdot L]
\end{align}
wird.\\

\hspace*{-12pt}\colorbox{highlightBlue}{\parbox{0.98\textwidth}{Die Knotenverschiebungen $u_i$ sind also (intern) {\em dimensionslose\/} Gewichte an dem FE-Ansatz
\begin{align}
u_h = \sum_i\,u_i\,\Np_i(x) = [\phantom{L}] \cdot [L] = [L]\,.
\end{align}
Im Ausdruck haben Sie nat\"{u}rlich die Dimension einer L\"{a}nge, wie es der Ingenieur sehen will.
}}\\

%%%%%%%%%%%%%%%%%%%%%%%%%%%%%%%%%%%%%%%%%%%%%%%%%%%%%%%%%%%%%%%%%%%%%%%%%%%%%%%%%%%%%%%%%%%%%%%%%%%
{\textcolor{sectionTitleBlue}{\subsubsection*{Rechenpfennige}}}\label{Rechenpfennige}

F\"{u}r ein FE-Programm sind die \"{a}quivalenten Knotenkr\"{a}fte $f_i$ {\em Rechenpfennige\/}\index{Rechenpfennige} wie \glq eins im Sinn\grq{}. Es geht von Knoten zu Knoten, verschiebt den Knoten um einen Meter in horizontaler und vertikaler Richtung und notiert sich, wieviel Arbeit die Belastung dabei leistet. Das sind die $f_i$.

Hat ein horizontales $f_i$ in einem Knoten einer Scheibe den Wert 10 kNm, so bedeutet dies, dass in der N\"{a}he des Knotens Lasten so verteilt sind, dass sie bei einer horizontalen Auslenkung $\vek \Np_i$ des Knotens um 1 m die Arbeit 10 kNm leisten.

Alles, was ein FE-Programm macht, ist, dass es dann Ersatzlasten so \"{u}ber die Scheibe verteilt, dass diese bei einer Auslenkung der einzelnen Knoten um 1 Meter dieselbe Arbeit leisten, wie die Originalbelastung, was man kurz als $f_{h @i}  = f_i$ schreiben kann.

In der Notation des \"{u}bern\"{a}chsten Abschnitts ist $f_{h @i}$ die Arbeit, die die FE-Lasten
\begin{align}
\vek p_h = \sum_j\,u_j\,\vek p_j
\end{align}
auf dem Weg $\vek \Np_i$ leisten
\begin{align}
f_{h @i} = \sum_j\,u_j\,\delta A_a(\vek p_j,\vek \Np_i)\,.
\end{align}
Der FE-Lastfall $\vek p_h$ sind die Kr\"{a}fte, die n\"{o}tig sind, um dem Tragwerk die Gestalt $\vek u = \{u_1, u_2, \ldots, u_n\}^T$ zu geben.
%----------------------------------------------------------------------------------------------------------
\begin{figure}[tbp]
\centering
\if \bild 2 \sidecaption \fi
\includegraphics[width=1.0\textwidth]{\Fpath/U89}
\caption{Einheitsverformungen am Stab (l\"{a}ngs) und am Balken (quer)} \label{U89}
\end{figure}%
%----------------------------------------------------------------------------------------------------------

%%%%%%%%%%%%%%%%%%%%%%%%%%%%%%%%%%%%%%%%%%%%%%%%%%%%%%%%%%%%%%%%%%%%%%%%%%%%%%%%%%%%%%%%%%%%%%%%%%%
{\textcolor{sectionTitleBlue}{\section{Festhaltekr\"{a}fte}}}\index{Festhaltekr\"{a}fte}
Wenn man die \"{a}quivalenten Knotenkr\"{a}fte aus der Streckenlast berechnet,
\begin{align}\label{Eq85}
d_i = \int_0^{\,l} p(x)\,\Np_i(x)\,dx\,,
\end{align}
dann nennt man das die Reduktion der Belastung in die Knoten.

Wir erinnern daran, dass wir die \"{a}quivalenten Knotenkr\"{a}fte $\vek f = \vek f_K + \vek d$, in die Kr\"{a}fte $\vek f_K$ aufspalten, die direkt in den Knoten angreifen und die Kr\"{a}fte $\vek d$, die bei der Reduktion der Belastung im Feld, der {\em domain load\/}, in die Knoten entstehen. Hier behandeln wir die Kr\"{a}fte $\vek d$.

Die \"{a}quivalenten Knotenkr\"{a}fte sind die Kr\"{a}fte, die ({\em actio\/}), mit denen  die Last auf die beidseitig festgehaltenen bzw. eingespannten Enden des Stabes oder Balkens dr\"{u}ckt. Also gilt: {\em Die Einheits\-ver\-form\-ungen $\Np_i^e$ sind die Einflussfunktionen f\"{u}r die \"{a}quivalenten Knotenkr\"{a}fte\/}. Bei einem Stab sind das die Funktionen
\begin{align}
\barr{l l l} \Np_1^e(x) = {\displaystyle \frac{1 - x}{l_e}} \qquad &\Np_1^e(0) = 1\,, \qquad
&\Np_1^e(l_e) = 0\,, \\ [0.3cm] \Np_2^e(x) = {\displaystyle \frac{x}{l_e}}  \qquad &\Np_2^e(0) =
0\,, \qquad &\Np_2^e(l_e) = 1 \earr
\end{align}
und bei einem Balken sind es die kubischen Polynome, s. Abb. \ref{U89},
\bfo\label{Phi1Bis4}
\parbox{5cm}{
\bfo
\Np_1^e(x) &=& 1 - \frac{3x^2}{l_e^2} + \frac{2x^3}{l_e^3} \nn \\
\Np_2^e(x) &=& - x + \frac{2x^2}{l_e} - \frac{x^3}{l_e^2} \nn
\efo
}
%\hfill
\parbox{5cm}{
\bfo
\Np_3^e(x) &=& \frac{3x^2}{l_e^2} - \frac{2x^3}{l_e^3}\nn \\
\Np_4^e(x) &=& \frac{x^2}{l_e} - \frac{x^3}{l_e^2}\,.\nn  \label{Einheitsverformungen}
\efo
}
\efo
Die $d_i^e$ sind dann die Gr\"{o}{\ss}en
\begin{align}
d_i^e &= \int_0^{\,l_e} \,\underset{\rightarrow}{p(x)}\,\Np_i^e(x)\,dx \qquad i = 1,2 &&\qquad \text{Stab} \\
d_i^e &= \int_0^{\,l_e} \,\underset{\downarrow}{p(x)}\,\Np_i^e(x)\,dx \qquad i = 1,2, 3, 4 &&\qquad \text{Balken}\,.
\end{align}
Wirkt eine Einzelkraft $P$ bzw. ein Moment $M$ in einem Punkt $x$, dann erh\"{a}lt man die zugeh\"{o}rigen $d_i^e$ einfach durch Auswertung im Punkt
\begin{align}
d_i^e = \Np_i^e(x) \cdot P \qquad  d_i^e = {\Np_i^e} '(x) \cdot M\,.
\end{align}
\"{A}quivalente Knotenkr\"{a}fte und Festhaltekr\"{a}fte  verhalten sich wie {\em actio\/} und {\em reactio\/} zueinander. Die Festhaltekr\"{a}fte sind also die $d_i^e \times (-1)$.

Eventuell muss man auch noch die $d_i^e$ in das DIN-Koordinatensystem umrechnen
\begin{align}
d_1^e &= - N(0) \qquad d_2^e = N(l_e) \\
d_1^e &= - V(0) \qquad d_2^e = - M(0) \qquad d_3^e = V(l_e) \qquad d_4^e = M(l_e) \,,
\end{align}
wenn man anders mit ihnen weiterrechnen will.\\

\hspace*{-12pt}\colorbox{highlightBlue}{\parbox{0.98\textwidth}{Bei Stabtragwerken sind die Einheitsverformungen der Knoten die Einflussfunktionen f\"{u}r die Festhaltekr\"{a}fte $\times (-1)$.}}\\

All dies nat\"{u}rlich unter der Voraussetzung, dass die Elementeinheitsverformungen $\Np_i^e$ homogene L\"{o}sungen der Stab- und Balkenl\"{o}sung sind, was man in der Regel voraussetzen kann, wenn keine gevouteten oder andere exotische Profile vorkommen, wenn also $EA$ und $EI$ konstant sind.

%----------------------------------------------------------------------------------------------------------
\begin{figure}[tbp]
\centering
\if \bild 2 \sidecaption \fi
\includegraphics[width=0.7\textwidth]{\Fpath/U73}
\caption{Ausschnitt aus einem FE-Netz: Die Kr\"{a}fte, die den Mittenknoten um eine L\"{a}ngeneinheit nach rechts verschieben, und die Bewegung an den umliegenden Knoten abstoppen, sind die {\em shape forces\/}. Die Fl\"{a}chenkr\"{a}fte $p_x$ und $p_y$ sind nur als integrale Werte angegeben. Im \"{u}brigen Netz sind die {\em shape forces\/} null, weil sich dort nichts bewegt} \label{U73}
\end{figure}%
%----------------------------------------------------------------------------------------------------------

%----------------------------------------------------------------------------------------------------------
\begin{figure}[tbp]
\centering
\if \bild 2 \sidecaption \fi
\includegraphics[width=.9\textwidth]{\Fpath/U28NEW}
\caption{Belastung einer Scheibe mit einer Einzelkraft, {\bf a)} System {\bf b)} der FE-Lastfall $\vek p_h$; die Graustufen entsprechen der Intensit\"{a}t der aufintegrierten FE-Fl\"{a}chenlast, wie in Abb. \ref{U73}, in den bilinearen Elementen} \label{U28}
\end{figure}%
%----------------------------------------------------------------------------------------------------------\\

%%%%%%%%%%%%%%%%%%%%%%%%%%%%%%%%%%%%%%%%%%%%%%%%%%%%%%%%%%%%%%%%%%%%%%%%%%%%%%%%%%%%%%%%%%%%%%%%%%%
{\textcolor{sectionTitleBlue}{\section{Shape forces und der FE-Lastfall}}}\index{shape forces}
Um einen Knoten eines Netzes um einen Meter horizontal oder vertikal zu verschieben -- und dabei gleichzeitig alle anderen Knoten festzuhalten -- sind gewisse Kr\"{a}fte n\"{o}tig, s. Abb. \ref{U73}. Wir nennen diese Kr\"{a}fte, in Analogie zu dem Begriff {\em shape functions\/}\index{shape functions}, die {\em shape forces\/} $\vek p_j = \{p_x, p_y\}^T$, die zu dem Freiheitsgrad $u_j$ geh\"{o}ren. Es sind treibende wie haltende Kr\"{a}fte. Die treibenden Kr\"{a}fte lenken den Knoten aus, und die haltenden Kr\"{a}fte sorgen daf\"{u}r, dass die Bewegung an den umliegenden Knoten zum Stillstand kommen. Es sind immer Gleichgewichtskr\"{a}fte.
%----------------------------------------------------------------------------------------------------------
\begin{figure}[tbp]
\centering
\if \bild 2 \sidecaption \fi
\includegraphics[width=1.0\textwidth]{\Fpath/U127}
\caption{Lineare FE-L\"{o}sung eines Stabes ($EA = 1$) unter konstanter Streckenlast. Die Kr\"{a}fte an den $\Np_i$ sind die {\em shape forces\/} (s.f.)} \label{U127}
\end{figure}%
%----------------------------------------------------------------------------------------------------------\\

Die Summe dieser {\em shape-forces\/} -- mit den Knotenverschiebungen $u_j$ gewich\-tet -- stellt den FE-Lastfall\index{FE-Lastfall} dar, das sind die Kr\"{a}fte, die $\vek u$ erzeugen
\begin{align}\label{Eq70}
\vek p_h = \sum_j\,u_j\,\vek p_j\,.
\end{align}
Nun kann man fragen, wenn die einzelnen $\vek p_j$ alle Gleichgewichtskr\"{a}fte sind, was wandert dann eigentlich in die Lager? Das kommt zum Vorschein, wenn man einen Schnitt neben einem Lager f\"{u}hrt, wie in Abb. \ref{U127} b, wo der Schnitt durch die Mitte des ersten Elements geht. Durch diesen Schnitt wird das Gleichgewicht in dem ersten Element gest\"{o}rt und eine entsprechend gro{\ss}e Schnittkraft von 2.5 kN muss das Gleichgewicht wieder herstellen. Dass es nicht die volle Belastung von 3.0 kN ist, liegt daran, dass das FE-Programm ja die \"{a}quivalente Knotenkraft $f = 0.5$ in dem Lager ignoriert, weil sie nicht statisch wirksam ist.

W\"{a}hrend bei Fl\"{a}chentragwerken einiger Aufwand n\"{o}tig ist, um den FE-Lastfall $\vek p_h$ zu berechnen, s. Abb. \ref{U28}, muss man bei Stabtragwerken ($EA$ und $EI$ konstant) gar nichts tun, denn bei Stabtragwerken ist der FE-Lastfall mit den Knotenkr\"{a}ften $f_i = f_{K @ i} + d_i$ identisch.\\

\begin{remark}
Die Abb. \ref{U28} illustriert auch sehr gut, die \glq Doppelb\"{o}digkeit\grq{} der Statik im Umgang mit finiten Elementen. Gehalten wird die Scheibe in der N\"{a}he der Lagerknoten eigentlich von einem konzentrierten System von Fl\"{a}chen- und Linienkr\"{a}ften. Im Ausdruck stehen aber nur die \"{a}quivalenten Knotenkr\"{a}fte $f_i$, die diese Haltekr\"{a}fte in der \glq Summe\grq{} repr\"{a}sentieren, und der Ingenieur findet (zu Recht) gar nichts dabei mit diesen $f_i$ weiter zu rechnen, aus ihnen echte Kr\"{a}fte zu machen.
\end{remark}
\vspace{-0.5cm}
%%%%%%%%%%%%%%%%%%%%%%%%%%%%%%%%%%%%%%%%%%%%%%%%%%%%%%%%%%%%%%%%%%%%%%%%%%%%%%%%%%%%%%%%%%%%%%%%%%%
{\textcolor{sectionTitleBlue}{\subsubsection*{Die Rolle der $u_i$}}}\index{die Rolle der $u_i$}

Der FE-Lastfall $\vek p_h$ wird mit Hilfe der $u_i$ so austariert, dass die zugeh\"{o}rigen Knotenkr\"{a}fte $f_{h @i}$ mit den \"{a}quivalenten Knotenkr\"{a}ften aus der Belastung \"{u}bereinstimmen, also im Fall der Scheibe die Arbeiten $f_{h @i}$ genauso gro{\ss} sind, wie die Arbeiten $f_i$
\begin{align}\label{Eq123}
f_{h @i} = \int_{\Omega} \vek p_h^T \dotprod \vek \Np_i\,d\Omega = \int_{\Omega}\,\vek p^T \dotprod \vek \Np_i\,d\Omega = f_i\,.
\end{align}
Diese Forderung ist \"{a}quivalent mit dem System $\vek K\,\vek u = \vek f$, denn die linke Seite $\vek K\,\vek u$ (innere Arbeiten $\delta A_i$) ist wegen {\em \glq innen = au{\ss}en\grq{}\/} identisch mit dem Vektor $\vek f_h$ (\"{a}u{\ss}ere Arbeiten $\delta A_a$).

Man stelle sich die Scheibe einmal mit der Originalbelastung $\vek p$ vor und daneben mit der FE-Belastung $\vek p_h$. Nun gehe man von Knoten zu Knoten und verschiebe den Knoten probeweise um einen Meter in horizontaler und vertikaler Richtung. Dann wird man finden, dass die Arbeiten immer gleich sind, $f_{h @i} = f_i$. In diesem Sinne gilt:\\

\hspace*{-12pt}\colorbox{highlightBlue}{\parbox{0.98\textwidth}{Der FE-Lastfall ist \glq wackel\"{a}quivalent\grq{} zu dem Originallastfall.
}}\\

Das ist wie bei einer Waage, wo bei jeder Drehung des Waagebalkens die Arbeiten des linken und rechten Gewichts gleich sind, d.h. die beiden Gewichte sind \glq wackel\"{a}quivalent\grq{}.\index{wackel\"{a}quivalent}

Ob ein Tragwerk die Originallasten tr\"{a}gt oder die FE-Lasten, kann man durch Wackeln mit den $\vek \Np_i$ allein nicht entscheiden, weil die Arbeiten jedesmal gleich sind.

Der FE-Lastfall ist der Lastfall, f\"{u}r den ein Tragwerksplaner das Tragwerk eigentlich bemisst, denn die Verformungen und die Schnittkr\"{a}fte im Ausdruck geh\"{o}ren zu diesem Lastfall $\vek p_h$.

{\em Wenn man die Belastung $\vek p_h$ auf das Tragwerk aufbringen w\"{u}rde und ein Statiker w\"{u}rde die Verformungen und Schnittkr\"{a}fte von Hand berechnen, dann w\"{u}rde er genau die FE-Ergebnisse erhalten\/}.

%----------------------------------------------------------------------------------------------------------
\begin{figure}[tbp]
\centering
\if \bild 2 \sidecaption \fi
\includegraphics[width=0.95\textwidth]{\Fpath/U29}
\caption{Vergleich der Hauptspannungen, {\bf a)} grobes Netz, {\bf b)} sehr feines Netz} \label{U29}
\end{figure}%
%----------------------------------------------------------------------------------------------------------

In Abb. \ref{U28} b ist ein solcher Vergleich des Originallastfalls und des FE-Lastfalls einmal dargestellt. Zu berechnen waren die Wirkungen einer Einzelkraft von 10 kN, die an der rechten oberen Ecke zieht. Diesen doch einfachen und klar definierten Lastfall ersetzt nun das FE-Programm durch ein sehr konfus wirkendes System von Fl\"{a}chenkr\"{a}ften und Kantenkr\"{a}ften, die den Lastfall $\vek p_h$ darstellen. Weil diese Lasten so \glq merkw\"{u}rdig\grq{} aussehen, werden sie normalerweise von FE-Programmen nicht gezeigt, weil ein Anwender, der mit der Theorie der finiten Elemente nicht vertraut ist, irritiert w\"{a}re.

Nur darf man sich davon aber nicht abschrecken lassen, denn den Ingenieur interessieren prim\"{a}r die Schnittgr\"{o}{\ss}en, und es ist Spekulation aus der Differenz $\vek p - \vek p_h$ in den Lasten auf die Differenz in den Schnittgr\"{o}{\ss}en schlie{\ss}en zu wollen. Die Differenz in der Belastung kann man ausrechnen, die Differenz in den Schnittkr\"{a}ften aber leider nicht.

Dass die FE-Ergebnisse nicht so schlecht sein k\"{o}nnen, wie dies die Abb. \ref{U28} anscheinend suggeriert, macht der direkte Vergleich der Hauptspannungen in Abb. \ref{U29} klar. Diese wirken durchaus glaubhaft.

Wir wollen noch ein zweites, indirektes Argument anf\"{u}hren. Im Originallastfall sind alle $f_i = 0$, bis auf das $f_i = 10$ in der oberen rechten Ecke, und daher m\"{u}ssen auch alle $f_{h @i} = 0$ sein, bis auf den Knoten in der Ecke, $f_{h @i} = 10$
\begin{align}\label{Eq124}
\int_{\Omega} \vek p_h \dotprod \vek \Np_i\,\,d\Omega = f_{h @i} = f_i = 0\,.
\end{align}
Das FE-Programm muss also schon kr\"{a}ftig jonglieren, um diese Eigenschaft zu garantieren, und das mag das \glq Chaos\grq{} in Abb. \ref{U28} b erkl\"{a}ren, denn all die recht willk\"{u}rlich aussehenden Teile des Lastfalls $\vek p_h$ sind so ausbalanciert, dass sie keine Arbeit leisten, wenn man einen Knoten probeweise um einen Meter in horizontaler oder vertikaler Richtung verschiebt.

Der Gro{\ss}teil der FE-Lasten $\vek p_h$  ist f\"{u}r das FE-Programm im Sinne der Energiemetrik null, weil sie keine Knotenkr\"{a}fte $f_{h @i}$ generieren.\\

\begin{remark}
Die {\em shape forces\/} $\vek p_i$ sind bei einer Scheibe Fl\"{a}chenkr\"{a}fte und Linienkr\"{a}fte und die virtuelle Arbeit dieser Kr\"{a}fte ist daher eigentlich eine Summe aus Gebietsintegralen und Linienintegralen \"{u}ber die Kanten $\Gamma$ der Elemente, auf denen der Knoten liegt, zu dem $\Np_i$ geh\"{o}rt
\begin{align}
f_{h @i} = \int_{\Omega} ... \,d\Omega + \int_{\Gamma} ...\,ds\,.
\end{align}
Die Schreibweise (\ref{Eq124}) ist daher eine zusammenfassende Kurzschreibweise f\"{u}r all diese Integrale.
\end{remark}
\vspace{-0.5cm}
%%%%%%%%%%%%%%%%%%%%%%%%%%%%%%%%%%%%%%%%%%%%%%%%%%%%%%%%%%%%%%%%%%%%%%%%%%%%%%%%%%%%%%%%%%%%%%%%%%%
{\textcolor{sectionTitleBlue}{\section{Wie die Lawine ins Rollen kam}}}
Wenn wir die Gleichung $\vek f_h = \vek f $ als die Grundgleichung der finiten Elemente ansehen, dann stimmt das mit dem Zugang in der Originalarbeit \cite{Turner} von {\em Turner et alteri\/} \"{u}berein. Die Autoren betrachteten damals ein {\em CST-Element\/}\index{CST-Element} und sie leiteten eine Matrix $\vek S $ her, die die drei Spannungen in dem Element, $\vek \sigma = \{\sigma_{xx}, \sigma_{yy}, \sigma_{xy}\}^T$, mit den Knotenverschiebungen verkn\"{u}pft\footnote{Wir orientieren uns hier an der Darstellung in \cite{Kurrer} S. 882-883}
\begin{align}\label{Eq174}
\vek \sigma_{(3)} = \vek S_{(3 \times 6)}\,\vek u_{(6)} \,.
\end{align}
Das sind die Spannungen, die zu dem FE-Lastfall $\vek p_h$ geh\"{o}ren, also zu den Kr\"{a}ften, die die Knotenverschiebungen, Vektor $\vek u$, bewirken. Weil die Spannungen konstant sind, gibt es in einem solchen LF $\vek p_h$ keine Volumenkr\"{a}fte sondern nur Randkr\"{a}fte. Als n\"{a}chstes haben die Autoren die sechs \"{a}quivalenten Knotenkr\"{a}fte, Vektor $\vek f_{h}$, berechnet, die zu diesen Kr\"{a}ften geh\"{o}ren, sie haben also die Randkr\"{a}fte mit den Einheitsverformungen $\vek \Np_i(\vek x)$ der Knoten \"{u}berlagert, was auf eine Matrix $\vek T $ f\"{u}hrte
\begin{align}
\vek f_h = \vek T_{(6 \times 3)}\,\vek \sigma_{(3)}
\end{align}
oder mit (\ref{Eq174}) auf die Beziehung
\begin{align}
\vek f_h = \vek T_{(6 \times 3)}\,\vek S_{(3 \times 6)}\,\vek u_{(6)}
\end{align}
und das ist genau die Steifigkeitsmatrix $\vek K = \vek T\,\vek S$ des {\em CST-Element\/}. Dass hier die Steifigkeitsmatrix auftaucht, deren Eintr\"{a}ge virtuelle {\em innere\/} Energien sind, obwohl doch eigentlich nur \glq au{\ss}en\grq{} gerechnet wurde, liegt an dem \glq au{\ss}en = innen\grq{}, das die erste Greensche Identit\"{a}t garantiert
\begin{align}
\text{\normalfont\calligra G\,\,}(\vek \Np_j,\vek \Np_i) = \delta A_a(\vek p_j,\vek \Np_i) - \delta A_i(\vek \Np_j,\vek \Np_i) = \delta A_a(\vek p_j,\vek \Np_i)-  k_{ij} =0\,,
\end{align}
zeilenweise also
\begin{align}
f_{hi} &= \delta A_a(\vek p_h,\vek \Np_i) = \sum_j\,\delta A_a(\vek p_j,\vek \Np_i)\,u_j =  \sum_j\,\delta A_i(\vek \Np_j,\vek \Np_i)\,u_j \nn \\
&= \sum_j\,k_{ij}\,u_j = \text{Zeile $i$ von $\vek K$ $\dotprod\,\vek u$}\,.
\end{align}

\hspace*{-12pt}\colorbox{highlightBlue}{\parbox{0.98\textwidth}{Die Elemente $k_{ij}$ einer Steifigkeitsmatrix $\vek K$ kann man als virtuelle innere Arbeit wie als virtuelle \"{a}u{\ss}ere Arbeit lesen.}}\\

Das $k_{11} = \delta A_i(\Np_1,\Np_1)$ einer Balkenmatrix
\begin{align}
 k_{11} = \int_{0}^{l}\frac{M_1^2}{EI}\,dx = a(\Np_1,\Np_1) = \frac{12\,EI}{l^3} \cdot 1 = \delta A_a(\Np_1,\Np_1)
\end{align}
ist genauso gro{\ss}, wie die Arbeit, die die Kraft $12\,EI/l^3$ -- das ist die Kraft, die den Knoten nach unten dr\"{u}ckt -- auf dem Weg $\Np_1(0) = 1$ leistet.

{\em Die \glq erste\grq{} Steifigkeitsmatrix in der Geschichte der FEM war also eine Tabelle von {\bf \"{a}u{\ss}eren Arbeiten}\/}. Erst die \"{A}quivalenz $\delta A_i = \delta A_a$ f\"{u}hrt auf die Interpretation, wie wir Steifigkeitsmatrizen $\vek K$ heute lesen, $k_{ij} = a(\vek \Np_i,\vek \Np_j)$.

Bemerkenswert ist dabei, wie simpel die finiten Elemente begonnen haben -- kein Energieprinzip, kein Galerkin, \"{u}berhaupt keine h\"{o}here Mathematik, sondern ein uraltes Prinzip, das Prinzip Waage\index{Prinzip Waage}, die \glq Wackel\"{a}quivalenz\grq{}\,
\begin{align}
 \boxed{\delta A_a(\vek p, \vek \Np_i) = \delta A_a(\vek p_h,\vek \Np_i) \qquad \text{der Start der FEM}}
\end{align}
hat die Lawine ins Rollen gebracht\footnote{
Erst sp\"{a}ter, als die Mathematiker an Bord  kamen, hat man erkannt, dass man die Elementans\"{a}tze  als finite Funktionen deuten konnte und dass die Wackel\"{a}quivalenz $\vek f_h = \vek f$ dem $\delta \Pi = 0 $ der potentiellen Energie entspricht.}.




%---------------------------------------------------------------------------------
\begin{figure}
\centering
{\includegraphics[width=0.8\textwidth]{\Fpath/U192A}}
  \caption{Gelenkig gelagerte Platte mit Innenst\"{u}tze im LF $g$, \textbf{ a)} Hauptmomente, \textbf{ b)} Biegefl\"{a}che mit \"{a}quivalenter Knotenkraft in der St\"{u}tze,  \textbf{ c)} L\"{a}ngsschnitt durch die Platte mit den FE-Lasten (symbolische Darstellung). Die \"{a}quivalente Knotenkraft im mittleren Bild steht stellvertretend f\"{u}r die aufw\"{a}rts gerichteten FE-Lasten im Bereich der St\"{u}tze, die eigentlich die Platte halten}
  \label{U192}
\end{figure}
%---------------------------------------------------------------------------------
%---------------------------------------------------------------------------------
\begin{figure}
\centering
{\includegraphics[width=0.8\textwidth]{\Fpath/U168}}
  \caption{Der FE-Lastfall $g_h$, das ist der LF, den das FE-Programm f\"{u}r den LF $g$ setzt, \textbf{ a)} die Zahlen sind die mittleren Fl\"{a}chenkr\"{a}fte in den Elementen, im Bereich der Innenst\"{u}tze sind sie negativ, dort st\"{u}tzen die Fl\"{a}chenkr\"{a}fte also die Platte, \textbf{ b)} ebenso wie die Linienkr\"{a}fte l\"{a}ngs der Elementkanten im Bereich der St\"{u}tze (rote Farbe); die Linienmomente des LF $g_h$ l\"{a}ngs der Linien sind nicht dargestellt}
  \label{U168}
\end{figure}
%---------------------------------------------------------------------------------


%---------------------------------------------------------------------------------
\begin{figure}
\centering
\if \bild 2 \sidecaption[t] \fi
{\includegraphics[width=0.75\textwidth]{\Fpath/U175}}
  \caption{Einzelkraft und Platte}
  \label{U175}
\end{figure}
%---------------------------------------------------------------------------------

%%%%%%%%%%%%%%%%%%%%%%%%%%%%%%%%%%%%%%%%%%%%%%%%%%%%%%%%%%%%%%%%%%%%%%%%%%%%%%%%%%%%%%%%%%%%%%%%%%%
{\textcolor{sectionTitleBlue}{\section{Der FE-Lastfall bei Platten}}}
Der FE-Lastfall und seine Interpretation ist bei Platten \"{a}hnlich ambivalent wie bei Scheiben, wie die Bilder \ref{U192} und \ref{U168} zeigen. Die St\"{u}tzenkraft $f_i$ im Ausdruck suggeriert, dass die Platte in der Mitte punktgenau von einer Einzelkraft gehalten wird, aber wenn man sich die Verteilung der FE-Lasten in Abb. \ref{U168} ansieht, dann erkennt man, dass das symbolisch zu nehmen ist. Rechnerisch sind es aufw\"{a}rts gerichtete Fl\"{a}chen- und Linienkr\"{a}fte in der Umgebung der St\"{u}tze, die die Platte nach oben dr\"{u}cken, keine Einzelkraft.

Denn w\"{a}re die St\"{u}tzenkraft $f_i$ eine echte Einzelkraft, dann m\"{u}ssten in dem Knoten die Schubkr\"{a}fte $v_n$ bei Ann\"{a}herung an die St\"{u}tze unendlich gro{\ss} werden, $v_n = 1/(2\,\pi\,r)$, weil anders die Bilanz\footnote{s. S. \pageref{EinzelF}}
\begin{align}
\lim_{r \to 0} \int_0^{\,2\,\pi} v_n \,r\,d\Np = f_i
\end{align}
nicht einzuhalten ist, s. Abb. \ref{U175}. Aber die {\em shape functions\/} $\Np_i(\vek x)$ sind Polynome und sie k\"{o}nnen sich nicht wie $1/r$ in einem Knoten zusammenschn\"{u}ren.

Die Knotenkraft $f_i$ im Ausdruck ist daher eine \"{a}quivalente Knotenkraft, also ein Arbeits\"{a}quivalent im Sinne von \glq soviel Arbeit wie eine Einzelkraft $f_i$ auf dem Weg 1 Meter leisten w\"{u}rde\grq{}.

Zwar hat die die St\"{u}tzenkraft $f_i$ direkt nichts mit der Bemessung der Platte zu tun, weil die Momente $m_{xx}, m_{xy}$ und $m_{xy}$ und auch die Querkr\"{a}fte $q_x$ und $q_y$ zu dem FE-Lastfall $p_h$ geh\"{o}ren, s. Abb. \ref{U168}, bei dem eben Fl\"{a}chen- und Linienkr\"{a}fte die Platte st\"{u}tzen, aber der Ingenieur wird nat\"{u}rlich mit der Knotenkraft $f_i$ einen Durchstanznachweis f\"{u}hren und sie auch als St\"{u}tzenkraft weiterleiten.

Uns scheint, dass diese Ambivalenz der finiten Elemente in der Praxis zu wenig diskutiert wird. Ein FE-Modell ist ein {\em Ersatzmodell\/}\index{Ersatzmodell} und es ist gar nicht eindeutig gekl\"{a}rt, wie man die FE-Ergebnisse auf das reale Tragwerk \"{u}bertr\"{a}gt.

Wenn man sich \"{u}berlegt,  mit welcher Akkuratesse heute der Durchstanznachweis bei Platten gef\"{u}hrt wird, und wie \glq wacklig\grq{} die FE-Ergebnisse sind, dann fragt man sich manchmal, ob die Relationen noch stimmen. Nat\"{u}rlich kann man auf der Materialseite keine R\"{u}cksicht darauf nehmen, dass Kr\"{a}fte nur n\"{a}herungsweise bekannt sind, man muss so tun, als ob sie exakt w\"{a}ren, aber exakte Resultate gibt es in der Praxis eben leider nicht.

%Es w\"{a}re dringend notwendig, \"{u}ber den Umgang mit FE-Ergebnissen zu diskutieren und nicht l\"{a}nger die Materialseite und die Numerik getrennt zu behandeln.


%%%%%%%%%%%%%%%%%%%%%%%%%%%%%%%%%%%%%%%%%%%%%%%%%%%%%%%%%%%%%%%%%%%%%%%%%%%%%%%%%%%%%%%%%%%%%%%%%%%
\textcolor{sectionTitleBlue}{\section{Kopplung von Bauteilen}}
Wenn man einen Stab an zwei Knoten einer Scheibe festmacht, dann bedeutet $f_i^{Stab} = f_i^{Scheibe}$, dass bei einer Verschiebung des Lagerknotens $\delta u_i = 1$ (alle anderen Knoten sind dabei fest) die zugeh\"{o}rigen inneren Energien in den beiden Bauteilen gleich gro{\ss} sind. Es ist keine statische Kopplung, weil das $f_i$ auf der Seite der Scheibe keine echte Einzelkraft ist.

Nur bei eindimensionalen Problemen, $EI w^{IV} = p$, $-EA u'' =  p$ etc., und dann auch nur bei konstanten Steifigkeiten, sind die $f_i$ (dimensionsbereinigt) echte Knotenkr\"{a}fte.


Wie sich die einzelnen Elemente eines Netzes zu einem Ganzen f\"{u}gen, kann man mit Hilfe der {\em Inzidenzmatrizen\/}\index{Inzidenzmatrizen} verfolgen.

%-----------------------------------------------------------------
\begin{figure}[tbp]
\centering
\includegraphics[width=1.0\textwidth]{\Fpath/U387A}
\caption{Kopplung zweier Stabelemente }
\label{U387}
\end{figure}%
%-----------------------------------------------------------------

Die beiden Elementmatrizen des Stabes in Abb. \ref{U387} stehen auf der Diagonalen einer $4 \times 4$ Matrix $\vek K_E$\index{$\vek K_E$}
\begin{align}
\vek f_E = \left[ \barr{c} f_1^a \\ f_2^a \\ f_1^b \\ f_2^b \earr \right] =  \frac{EA}{\ell_e} \left [
\barr {r @{\hspace{2mm}} r @{\hspace{2mm}} r @{\hspace{2mm}} r} 1 & -1 & 0 & 0 \\ -1 & 1 & 0 & 0 \\ 0 & 0 & 1 & -1 \\ 0 & 0 & -1 & 1 \\ \earr \right] \left[ \barr{cc} u_1^a \\ u_2^a \\ u_1^b \\ u_2^b \earr \right] = \vek K_E\,\vek u_E\,.
\end{align}
Die Verschiebungen der Elementenden sind an die Knotenverschiebungen $u_1, u_2, u_3$ gekoppelt
\begin{align}
\vek u_E = \left[ \barr{c} u_1^a \\ u_2^a \\ u_1^b \\ u_2^b \earr \right] = \left [
\barr {r @{\hspace{2mm}} r @{\hspace{2mm}} r } 1 & 0 & 0  \\ 0 & 1 & 0 \\ 0 & 1 & 0 \\ 0 & 0 &  1 \\ \earr \right] \left[ \barr{cc} u_1 \\ u_2 \\ u_3 \earr \right] = \vek  A\,\vek u\,.
\end{align}
Die Kr\"{a}fte $\vek f_E$ und die Knotenkr\"{a}fte $\vek f$ m\"{u}ssen bei einer virtuellen Verr\"{u}ckung $\vek \delta \vek u$ bzw. $\vek \delta \vek u_E = \vek A\,\vek \delta \vek  u$ die gleiche Arbeit leisten
\begin{align}
\vek f_E^T\,\vek \delta \vek u_E = \vek f^T\,\vek \delta \vek u \qquad \text{oder} \qquad \vek f_E^T\,\vek A\,\vek  \delta\vek u = \vek f^T\,\vek  \delta\vek u\,,
\end{align}
was $\vek f = \vek A^T\,\vek f_E$ ergibt und das sind nat\"{u}rlich gerade die Gleichgewichtsbedingungen zwischen den Stabendkr\"{a}ften und den Knotenkr\"{a}ften $f_i$
\begin{align}
\vek f = \left[ \barr{c} f_1 \\ f_2 \\ f_3 \earr \right] = \left [
\barr {r @{\hspace{2mm}} r @{\hspace{2mm}} r @{\hspace{2mm}} r} 1 & 0 & 0 & 0 \\ 0 & 1 & 1 & 0 \\ 0 & 0 & 0 & 1 \earr \right] \left[ \barr{c} f_1^a \\ f_2^a \\ f_1^b \\ f_2^b \earr \right] = \vek A^T\,\vek f_E\,.
\end{align}
Entsprechend erh\"{a}lt man durch Multiplikation der Matrix $\vek K_E$ von links und rechts mit
$\vek A^T$ bzw. $\vek A$ die Gesamtsteifigkeitsmatrix
\begin{align}
\vek K = \vek A^T\,\vek K_E\,\vek A = \frac{EA}{\ell_e} \left [
\barr {r @{\hspace{2mm}} r @{\hspace{2mm}} r} 1 & -1 & 0  \\ -1 & 2 &-1 \\ 0 &-1 &1\earr \right]\,.
\end{align}

%-----------------------------------------------------------------
\begin{figure}[tbp]
\if \bild 2 \sidecaption[t] \fi
\centering
\includegraphics[width=0.99\textwidth]{\Fpath/U548Q}
\caption{Kopplung Scheibe -- Balken. Die drei Schnittkr\"{a}fte $F_{B,x}, F_{B,y}, M_B$ des Balkens werden in \"{a}quivalente Knotenkr\"{a}fte $F_{x,i}, F_{y,i}$ rechts umgerechnet, so dass sie zur Scheibe links passen \cite{Werkle3} }
\label{U548}
\end{figure}%
%-----------------------------------------------------------------
%%%%%%%%%%%%%%%%%%%%%%%%%%%%%%%%%%%%%%%%%%%%%%%%%%%%%%%%%%%%%%%%%%%%%%%%%%%%%%%%%%%%%%%%%%%%%%%%%%%
\textcolor{chapterTitleBlue}{\section{\"{A}quivalente Spannungs Transformation}}\label{AST}
Die Kopplung von gleichartigen Elementen untereinander stellt also kein Problem dar. Schwieriger ist es aber z.B. St\"{u}tzen (Balken) und Scheiben miteinander zu koppeln, weil Balkenenden Drehfreiheitsgrade haben, die den Knoten einer Scheibe fehlen.

Die  {\em \"{A}quivalente Spannungs Transformation\/} (EST) von Werkle, \cite{Werkle1}, l\"{o}st dieses Problem auf sehr elegante, ja nat\"{u}rliche Weise. Bei ihr z\"{a}umt man das Pferd sozusagen von hinten auf. Normalerweise geht man bei der Formulierung einer Steifigkeitsmatrix
\begin{align}
\vek K = \vek A^T\vek K^{\mathcal{D}} \vek A
\end{align}
ja so vor\footnote{Auf der Diagonalen von $\vek K^{\mathcal{D}} $ stehen die Elementmatrizen, s. (\ref{Eq24}) }, dass man erst die Kopplung zwischen den Weggr\"{o}{\ss}en beschreibt,
\begin{align} \label{Eq186}
\vek u_{loc} = \vek A\,\vek u_{Knoten}
\end{align}
und dann mit der Transponierten $\vek A^T$ die zugeh\"{o}rigen Gleichgewichtsbedingungen formuliert\index{Aequivalente Spannungs Transformation}
\begin{align} \label{Eq187}
\vek f_{Knoten} = \vek A^T\,\vek f_{loc}\,.
\end{align}
Bei der \"{a}quivalenten Spannungs Transformation ist es umgekehrt. Bei ihr wird zuerst die Abbildung (\ref{Eq187}) formuliert -- hier ist das statische Verst\"{a}ndnis und das Geschick des Ingenieurs gefragt -- und diese Matrix $\vek A^T$ wird dann in transponierter Form in (\ref{Eq186}) \"{u}bernommen.

An dieser Stelle zeigt sich -- und das war vorher nicht so deutlich -- dass es zwei Wege gibt, den Zusammenhang $\vek A$ der Elemente zu beschreiben, den geometrischen Pfad $\vek u_{Knoten} \to \vek u_{loc}$ oder den statischen Pfad $\vek f_{loc} \to \vek f_{Knoten}$.\\


\textcolor{chapterTitleBlue}{\subsubsection*{Beispiel}}

Wie man diese Technik nutzen kann um eine Steifigkeitsmatrix herzuleiten, die die Kopplung eines Balkens an eine Scheibe beschreibt, soll das folgende Beispiel zeigen.

Die Situation zeigt Abb. \ref{U548}; drei Knoten mit den Freiheitsgraden  $u_i, v_i$ liegen dem Balken gegen\"{u}ber. Beim geometrischen Pfad (\ref{Eq186}) macht man die Annahme, dass der Querschnitt des Balkens eben bleibt, also mit $a = d/2$, halbe Tr\"{a}gerh\"{o}he,
\begin{align}
u_1 = u_B + a\,\tan \Np_B, \quad u_2 = u_B, \quad u_3 = u_B - a\,\tan \Np_B, \quad v_1 = v_2 = v_3
\end{align}
und damit lautet (\ref{Eq186}) ausgeschrieben
\begin{align}
\left[\barr{c} u_1^{(1)} \\ v_1^{(1)}  \\u_2^{(1)} \\ v_2^{(1)} \\ u_2^{(2)} \\ v_2^{(2)} \\ u_3^{(2) } \\ v_3^{(2)}\earr \right] = \left[\barr{c @{\hspace{6mm}} c @{\hspace{2mm}} c} 1 &0 & a \\ 0 & 1 & 0 \\ 1 & 0 & 0 \\ 0 & 1 & 0  \\ 1 & 0 & 0 \\ 0 &1 & 0 \\ 1 & 0 &- a \\ 0 & 1 & 0 \earr \right] \,\left[\barr{c} u_B \\ v_B \\ \tan \Np_B \earr \right]\,.
\end{align}
Entscheidend ist, dass hier ein linearer Verlauf der Verschiebungen angenommen wurde -- eine Annahme, die nat\"{u}rlich so nicht richtig ist. Richtig im Sinne der Elastizit\"{a}tstheorie w\"{a}re ein  Verschiebungsverteilung, wie sie sich bei einer Berechnung des Anschlusses als Scheibe, d.h. der Modellierung des Balkens mit Scheibenelementen, einstellt.

Beim statischen Pfad  geht man dagegen \"{u}ber die Kr\"{a}fte. Die Schnittgr\"{o}{\ss}en $F_{Bx}, F_{By}$ und $M_B$, s. Abb. \ref{U548}, erzeugen, bei Ansatz der Biegebalkentheorie, die Spannungen\footnote{Vorzeichen gem\"{a}{\ss} Abb. \ref{U548}} (Rechteckquerschnitt, $t$ = Wandst\"{a}rke)
\begin{align}
p_x &= \frac{F_{B,x}}{A} - \frac{M_B}{I}\,y_B = \frac{F_{B,x}}{d\,t} - \frac{12\,M_B}{t\,d^3}\,y_B \\
p_y &= \frac{3}{2}\,\frac{1}{d\,t} (1 - 4\,\frac{y_B^2}{d^2})\,F_{B,y}\,,
\end{align}
in der Stirnfl\"{a}che des Balkens, die man in \"{a}quivalente Knotenkr\"{a}ften $\vek f$ auf der Seite der Scheibe umrechnen kann und so kommt man zu einer Beziehung zwischen den Kr\"{a}ften auf den beiden Seiten des Schnittufers.

Bei dieser Technik werden erst aus dem Vektor $\vek f_B = \{F_{B,x}, F_{B,y}, M_B\}^T$ die Knotenwerte $p_i$ der Spannungen $\sigma$ und $\tau$ ermittelt. Die Knotenwerte fassen wir zu einem  Vektor $\vek p $ zusammen und so kann der \"{U}bergang $\vek f_B \to \vek p$ mit einer Matrix $\vek P$ (wie Polynome) beschrieben werden.

Mit den $y$-Koordinaten der Punkte 1, 2 und 3 (Achse $y_B$ bezogen auf den Schwerpunkt des Balkens)
\begin{align}
y_{B,1} = - a, \qquad y_{B,2} = 0, \qquad y_{B,3} = a
\end{align}
erh\"{a}lt man mit den obigen Formeln die Knotenwerte der Spannungen zu
\begin{align}
\left[\barr{c} p_{x,1} \\ p_{y,1}  \\p_{y,m,1-2} \\ p_{x,2} \\ p_{y,2} \\ p_{y,m,2-3} \\  p_{x,3} \\  p_{y,3}\earr \right] = \frac{ 1}{16\,t\,a^2}\left[\barr{c @{\hspace{6mm}} c @{\hspace{2mm}} c} 8\,a &0 & 24 \\ 0 & 0 & 0 \\ 0 & 9\,a & 0 \\ 8\,a &0 & 0 \\ 0 &12\,a &0 \\ 0 & 9\,a & 0 \\ 8\,a &0 & -24 \\ 0 & 0 &0  \earr \right]\,\left[\barr{c} F_{B,x} \\ F_{B,y} \\ M_B \earr \right]
\end{align}
oder
\begin{align}\label{Eq26}
\vek p = \vek P\,\vek f_B\,.
\end{align}
Die Berechnung der \"{a}quivalenten Knotenkr\"{a}fte $\vek f_S$ aus den als Linienlasten aufgefassten Spannungen $p_x$ und $p_y$ geschieht, wie es die Regel ist, durch die \"{U}berlagerung der Spannungen mit den {\em shape functions\/}. Das Ergebnis hat formal die Gestalt
\begin{align}
\vek f_S = \vek Q\,\vek p\,.
\end{align}
mit einer Matrix, die wir $\vek Q$ (wie Quadratur) nennen.

Diese Beziehung wird zun\"{a}chst f\"{u}r jedes Element separat ermittelt und daraus dann die Gesamtmatrix $\vek Q $ gebildet. Nach Bild \ref{U548} sind hier zwei Scheibenelemente zu ber\"{u}cksichtigen. Man erh\"{a}lt\footnote{nach \cite{Werkle2} Glg. (4.78c, d), S. 259 und Glg. (4.86a, b), S. 262}, am Element 1
\begin{align}
\left[\barr{c} F_{x,1}^{(1)} \\ F_{x,2}^{(1)} \earr \right] =
\frac{a\,t}{6} \left[\barr{c @{\hspace{6mm}} c} 2 & 1 \\ 1 &2  \earr \right]\,\left[\barr{c} p_{x,1}\\ p_{x,2} \earr \right]
\end{align}
und
\begin{align}
\left[\barr{c} F_{y,1}^{(1)} \\ F_{y,2}^{(1)} \earr \right] =
\frac{a\,t}{12} \left[\barr{c @{\hspace{6mm}} c @{\hspace{6mm}} c} 3 & 4 &-1 \\ -1 &4 &3  \earr \right]\,\left[\barr{c} p_{y,1} \\ p_{y,m,1-2} \\ p_{y,2} \earr \right]\,.
\end{align}
Damit lautet am Element 1 die Beziehung
\begin{align}
\left[\barr{c} F_{x,1}^{(1)} \\ F_{y,1}^{(1)} \\ F_{x,2}^{(1)} \\ F_{y,2}^{(1)}\earr \right]
= \frac{ a\,t}{12}\,\left[\barr{c @{\hspace{3mm}} c @{\hspace{3mm}} c @{\hspace{3mm}} c @{\hspace{3mm}} c} 4 & 0 & 0 &2 & 0 \\ 0 & 3 & 4 &0 & -1\\  2 & 0 & 0 &4 & 0 \\
 0 & -1 & 4 &0 & 3  \earr \right]\,\left[\barr{c} p_{x,1} \\ p_{y,1} \\ p_{y,m,1-2} \\ p_{x,2} \\ p_{y,2} \earr \right]
\end{align}
und entsprechend am Element 2
\begin{align}
\left[\barr{c} F_{x,2}^{(2)} \\ F_{y,2}^{(2)} \\ F_{x,3}^{(2)} \\ F_{y,3}^{(2)}\earr \right]
= \frac{ a\,t}{12}\,\left[\barr{c @{\hspace{3mm}} c @{\hspace{3mm}} c @{\hspace{3mm}} c @{\hspace{3mm}} c} 4 & 0 & 0 &2 & 0 \\ 0 & 3 & 4 &0 & -1\\  2 & 0 & 0 &4 & 0 \\
 0 & -1 & 4 &0 & 3  \earr \right]\,\left[\barr{c} p_{x,2} \\ p_{y,2} \\ p_{y,m,2-3} \\ p_{x,3} \\ p_{y,3} \earr \right]\,.
\end{align}
Den Vektor der Knotenkr\"{a}fte, die auf die Scheibe an der Verbindung wirken, ergibt sich durch Addition der Elementkr\"{a}fte der einzelnen Elemente, s. Abb. \ref{U548}, und man erh\"{a}lt so die Matrix $\vek Q $ zu
\begin{align}
\left[\barr{c} F_{x,1} \\ F_{y,1} \\ F_{x,2} \\ F_{y,2} \\ F_{x,3} \\ F_{y,3}\earr \right] = \frac{a\,t}{12}\,\left[\barr{c @{\hspace{3mm}} c @{\hspace{3mm}} c @{\hspace{3mm}} c @{\hspace{3mm}} c @{\hspace{3mm}} c @{\hspace{3mm}} c @{\hspace{3mm}} c} 4 &0 &0 &2 &0 &0 &0 &0\\
0 &3 &4 &0 &-1 &0 &0 &0 \\
2 &0 &0 &8 &0 &0 &2 &0 \\ 0 &-1 &4 &0 &6 &4 &0 &-1 \\ 0 &0 & 0 &2 &0 &0 &4 &0 \\
0 & 0 &0 &0 &-1 &4 &0 &3\earr \right]
\left[\barr{c} p_{x,1} \\ p_{y,1} \\ p_{y,m,1-2} \\ p_{x,2} \\ p_{y,2} \\ p_{y,m,2-3} \\ p_{x,3} \\ p_{y,3}\earr \right]
\end{align}
oder
\begin{align}
\vek f_S = \vek Q\,\vek p\,.
\end{align}
Die Kr\"{a}fte $\vek f_S$ sind die Kr\"{a}fte rechts in Abb. \ref{U548}.
Mit (\ref{Eq26}) folgt weiter
\begin{align}
\vek f_S = \vek Q\,\vek P\,\vek f_B = \vek A^T\,\vek f_B
\end{align}
und somit lautet die Transformationsmatrix
\begin{align}\label{Eq27}
\vek A = \vek P^T\,\vek Q^T = \left[\barr{c @{\hspace{3mm}} c @{\hspace{3mm}} c @{\hspace{3mm}} c
@{\hspace{3mm}} c @{\hspace{3mm}} c} 1/4 & 0 &1/2 & 0 &1/4 & 0 \\
0 &1/8 & 0 &3/8 &0 &1/8\\
1/2\,a & 0 & 0& 0  &-1/2\,a & 0\earr \right]\,.
\end{align}
%-----------------------------------------------------------------
\begin{figure}[tbp]
\if \bild 2 \sidecaption[t] \fi
\centering
\includegraphics[width=0.5\textwidth]{\Fpath/U549}
\caption{Stabelement }
\label{U549}
\end{figure}%
%-----------------------------------------------------------------
Die Weg- und Kraftgr\"{o}{\ss}en am Stabende
\begin{align}
\vek f_B = \left[\barr{c} F_{B,x} \\ F_{B,y} \\ M_{B} \earr \right]\,, \quad
\vek u_B = \left[\barr{c} u_B \\ v_B \\ \Np_B \earr \right]\,,\quad
\vek f_S = \left[\barr{c} F_{x,1} \\ F_{y,1} \\ F_{x,2} \\ F_{y,2} \\ F_{x,3} \\ F_{y,3}\earr \right]\,,\quad
\vek u_S = \left[\barr{c} u_1 \\ v_1 \\ u_2 \\ v_2 \\ u_3 \\ v_3 \earr \right]
\end{align}
transformieren sich also wie
\begin{align} \label{Eq990}
\vek u_B = \vek A_{(3 \times 6)}\,\vek u_S \qquad \vek f_S = \vek A^T_{(6 \times 3)}\,\vek f_B\,.
\end{align}
Am Stab, s. Abb. \ref{U549}, lauten die Beziehungen zwischen den Weg- und Kraftgr\"{o}{\ss}en
\begin{align}
\left[\barr{c @{\hspace{3mm}} c @{\hspace{3mm}} c @{\hspace{3mm}} c @{\hspace{3mm}} c @{\hspace{3mm}} c @{\hspace{3mm}} c @{\hspace{3mm}} c} a_0 & 0 & 0 &- a_0 & 0 & 0 \\
0 & 12\,a_1/\ell^2 & 6\,a_1/\ell & 0 &-12\,a_1/\ell^2& 6\,a_1/\ell \\
0 &6\,a_1/\ell & 4\,a_1 & 0 &-6\,a_1/\ell & 2\,a_1 \\
-a_0 & 0 & 0 & a_0 & 0 & 0 \\
0 & -12\,a_1/\ell^2 & -6\,a_1/\ell & 0 &12\,a_1/\ell^2& -6\,a_1/\ell \\
0 &6\,a_1/\ell & 2\,a_1 & 0 &-6\,a_1/\ell & 4\,a_1
\earr \right]\,\left[\barr{c} u_a \\ v_a \\ \Np_a \\ u_b \\v_b \\ \Np_b \earr \right] =
\left[\barr{c} F_{x,a} \\ F_{y,a} \\ M_a \\ F_{x,b} \\ F_{y,b} \\ M_b\earr \right]
\end{align}
mit
\begin{align}
a_0 &= \frac{E A}{\ell}\,, \quad a_1 = \frac{E I}{\ell}\,,  \quad \ell = \text{L\"{a}nge des Stabes}.
\end{align}
Das Balkenelement besitzt die Knoten $a $ und $b $. Entsprechend wird die obige Steifigkeitsmatrix des Balkens nun in Untermatrizen, die sich auf die Knoten $a $ und $b $ beziehen, unterteilt
\begin{align} \label{Eq30}
\left[\barr{c @{\hspace{3mm}} c } \vek K_{aa} & \vek K_{ab} \\
\vek K_{ba} &\vek K_{bb}\earr \right]\,\left[\barr{c} \vek u_a \\ \vek u_b\earr \right] = \left[\barr{c} \vek f_a \\ \vek f_b\earr \right]\,.
\end{align}
Beim Anschluss des Knoten $a $ an zwei Scheibenelemente wie in Abb. \ref{U548} lauten die Weg- und Kraftgr\"{o}{\ss}en im Knoten $a$ in der Notation der EST
\begin{align}
\vek u_a = \vek u_B =\left[\barr{c} u_B \\ v_B \\ \Np_B\earr \right] \qquad \vek f_a = \vek f_B = \left[\barr{c}  F_{B,x} \\  F_{B,y} \\ M_B\earr \right]\,.
\end{align}
Wenn die Verschiebungen in den finiten Elementen linear verlaufen, transformieren sich die Gr\"{o}{\ss}en, s.o., gem\"{a}{\ss}
\begin{align}
\vek u_B = \vek A_{(3 \times 6)}\,\vek u_S
\end{align}
mit der Matrix $\vek A$ wie in (\ref{Eq27}). Setzen wir nun $\vek u_a = \vek u_B = \vek A\,\vek u_S$ in (\ref{Eq30}) ein, multiplizieren dann die erste Zeile von links mit $\vek A^T$, so ergibt sich mit
\begin{align}
\vek A^T_{(6 \times 3)}\,\vek f_a = \vek A^T_{(6 \times 3)}\,\vek f_B = \vek f_S
\end{align}
das Resultat
\begin{align}\label{Eq31}
\left[\barr{c @{\hspace{3mm}} c } \vek A^T\,\vek K_{aa}\,\vek A & \vek A^T\vek K_{ab} \\
\vek K_{ba}\,\vek A &\vek K_{bb}\earr \right]\,\left[\barr{c} \vek u_S \\ \vek u_b\earr \right] = \left[\barr{c} \vek f_S \\ \vek f_b\earr \right]\,.
\end{align}
Das ist eine $9 \times 9 $ Matrix, 6 {\em dofs\/} $\vek u_S$ an der Scheibe und 3 {\em dofs\/} $\vek u_b$ am Balken, und die $\vek f_S$ sind die sechs Knotenkr\"{a}fte an der Scheibe\footnote{{\em dofs\/} = {\em degrees of freedom\/}, Freiheitsgrade}.

Die Steifigkeitsmatrix (\ref{Eq31}) ist nun im Knoten $a $ auf die Freiheitsgrade des Scheibenmodells und im Knoten $b$ auf die Freiheitsgrade des Stabes bezogen. Knoten $b$ kann, falls er ebenfalls an ein Scheibenmodell angeschlossen ist, ebenfalls transformiert werden.

Zum Verst\"{a}ndnis sei gesagt, dass der statische Pfad hier nur zur Herleitung der Matrix (\ref{Eq31}) benutzt wird. Der  Zusammenbau aller Elementmatrizen zur Gesamtsteifigkeitsmatrix erfolgt dann wie sonst auch.\\

\begin{remark}
Der geometrische Pfad und der statische Pfad beruhen auf unterschiedlichen Annahmen, die zu unterschiedlichen Ergebnissen f\"{u}hren. Beim geometrischen Pfad werden die Verschiebungsverl\"{a}ufe vorgegeben, und die Spannungen der Scheibenelemente passen sich diesen Vorgaben an. Beim statischen Pfad werden dagegen die Spannungsverl\"{a}ufe vorgegeben und die Verschiebungen (hier die Punkte 1, 2 und 3) k\"{o}nnen sich anpassen und weichen dann aber von der linearen Verteilung des geometrischen Pfads ab.

Beim geometrischen Pfad ist die Verbindung zu steif, beim statischen Pfad ist sie zu weich. Der geometrische Pfad bedeutet jedoch -- insbesondere bei der Verbindung von St\"{u}tzen mit Platten wie bei einer Flachdecke -- einen starren Einschluss im FE-Modell. Die FEM kommt jedoch mit starren Einfl\"{u}ssen nicht gut klar, d.h. es ergeben sich im Verbindungsbereich Spannungssingularit\"{a}ten und damit stark fehlerhafte Elementspannungen. Dies ist beim statischen Pfad nicht der Fall und daher sollte man dem statischen Pfad den Vorzug geben. F\"{u}r weitere Details verweisen wir auf  \cite{Werkle2}.
\end{remark}

%----------------------------------------------------------
\begin{figure}[tbp]
\centering
\if \bild 2 \sidecaption[t] \fi
\includegraphics[width=.8\textwidth]{\Fpath/U72}
\caption{Berechnung der Einflussfunktion f\"{u}r eine Verschiebung $u(x)$, \textbf{a)} Dachfunktionen und Originalbelastung, \textbf{ b)} Ersatzkr\"{a}fte und daraus resultierende FE-Einflussfunktion} \label{U72}
\end{figure}%%
%----------------------------------------------------------



%%%%%%%%%%%%%%%%%%%%%%%%%%%%%%%%%%%%%%%%%%%%%%%%%%%%%%%%%%%%%%%%%%%%%%%%%%%%%%%%%%%%%%%%%%%%%%%%%%%
{\textcolor{sectionTitleBlue}{\section{Berechnung von Einflussfunktionen mit finiten Elementen}}}\label{InfFEM}
Wir kommen nun zu einem sehr wichtigen Thema, der Berechnung von Einflussfunktionen mit finiten Elementen.

Auch die Berechnung von Einflussfunktionen mit finiten Elementen f\"{u}hrt auf das Gleichungssystem $\vek K\,\vek u = \vek f$, nur dass wir die $u_i$ jetzt $g_i$ nennen und statt $f_i$ schreiben wir $j_i$
\begin{align}
\vek K\,\vek g = \vek j\,.
\end{align}
Dieser Namenswechsel erleichtert das Operieren mit FE-Einflussfunktionen.

Unser erstes Probest\"{u}ck ist die Berechnung der Einflussfunktion f\"{u}r die L\"{a}ngsverschiebung $u(x)$ des Stabes in Abb. \ref{U72} im Punkt $x = 2.5$. Die \"{a}quivalenten Knotenkr\"{a}fte sind in diesem Fall die Verschiebungen der Ansatzfunktionen $\Np_i(x)$ in dem Punkt $x = 2.5$
\begin{align}\label{Eq110}
\Np_1(2.5) = 0 \qquad \Np_2(2.5) = 0.5 \qquad \Np_3(2.5) = 0.5 \qquad \Np_4(2.5) = 0\,,
\end{align}
was auf  das Gleichungssystem
\begin{align}\label{Eq68}
\frac{EA}{l_e} \left[\barr{r r r r} 2 & - 1 & 0 & 0 \\ - 1 & 2 & -1 & 0\\ 0 & -1 & 2 &-1 \\ 0 & 0 & -1 &2\earr\right]
\,\left[\barr{c} g_1 \\g_2 \\ g_3 \\ g_4 \earr \right] = \left[\barr{c} 0 \\ 0.5  \\
0.5  \\ 0 \earr \right]
\end{align}
f\"{u}r die Knotenverschiebungen $g_i$ f\"{u}hrt. Dieses System  hat die L\"{o}sung
\begin{align}
g_1 = 1\qquad g_2 = 2\qquad g_3 = 2.5\qquad g_4 = 2.5\,,
\end{align}
und somit hat die Einflussfunktion die Gestalt
\begin{align}
G_0^h(y,x = 2.5) = \frac{l_e}{EA}\, [1 \cdot \Np_1(y) + 2 \cdot \Np_2(y) + 2.5 \cdot \Np_3(y) + 2.5 \cdot\Np_4(y)]\,.
\end{align}
Da $x$ schon den Aufpunkt bezeichnet, benutzen wir als Lauf- oder Summationsvariable den Buchstaben $y$.

Die FE-Einflussfunktion ist, bis auf das Element, in dem der Aufpunkt $x$ liegt, exakt.
Den Fehler in dem Element beheben die FE-Programme dadurch, dass sie zur FE-L\"{o}sung die lokale L\"{o}sung addieren
\begin{align}
G_0(y,x) = G_0^h(y,x) + \text{lokale L\"{o}sung}\,.
\end{align}
So gelingt es den FE-Programmen exakte Einflussfunktionen f\"{u}r Stabtragwerke zu generieren -- vorausgesetzt $EA$ und $EI$ sind konstant.\\

{\textcolor{sectionTitleBlue}{\subsubsection*{Der Schl\"{u}ssel zu den Knotenkr\"{a}ften $ j_i$}}}\index{Schl\"{u}ssel zu den  $j_i$}

Warum sind bei dem obigen Beispiel die \"{a}quivalenten Knotenkr\"{a}fte $j_i$ (= $f_i$) gleich den Werten der Ansatzfunktionen im Aufpunkt, $j_i = \Np_i(2.5)$? Der Schl\"{u}ssel hierzu liegt in der Definition der \"{a}quivalenten Knotenkr\"{a}fte $f_i$.

%----------------------------------------------------------
\begin{figure}[tbp]
\centering
\if \bild 2 \sidecaption[t] \fi
\includegraphics[width=.99\textwidth]{\Fpath/U451A}
\caption{Berechnung der vier Einflussfunktionen eines Balkens mit finiten Elementen. Die \"{a}quivalenten Knotenkr\"{a}fte (hier ohne Vorzeichen---das steckt in den Pfeilen) sind die Werte der Ansatzfunktionen im Aufpunkt $x = 0.5\,\ell$. Wenn zu den FE-L\"{o}sungen noch die lokalen L\"{o}sungen addiert werden, sind die Ergebnisse auch im Element mit dem Aufpunkt exakt---sonst nur au{\ss}erhalb von dem Element. Die $j_i$ findet man in (\ref{Eq219}) und (\ref{Eq219X}). } \label{U451}
\end{figure}%%
%----------------------------------------------------------
Die Knotenkraft $f_i$ ist eine Arbeit und zwar die Arbeit, die die Belastung $p(x)$  auf den Wegen der Ansatzfunktion $\Np_i(x)$ leistet
\begin{align}
f_i = \int_0^{\,l} p(x)\,\Np_i(x)\,dx\,.
\end{align}
Bei Einflussfunktionen steht auf der rechten Seite der Differentialgleichung ein Dirac Delta (eine in einem Punkt konzentrierte Linienlast)
\begin{align}
-EA\,\frac{d^2}{dy^2}\,G_0(y,x) = \delta_0(y-x) \qquad \leftarrow \,\,\text{[kN/m]}\,,
\end{align}
die eine Einzelkraft im Aufpunkt $x = 2.5$ repr\"{a}sentiert. Sie ist sozusagen das $p$, das zur Einflussfunktion geh\"{o}rt. (Wir differenzieren auf der linken Seite nach $y$, dies ist hier die Laufvariable).

Jetzt rechnen wir und finden, dass die \"{a}quivalenten Knotenkr\"{a}fte ($[\text{kN} \cdot \text{m}]$)
\begin{align}
j_i = \int_0^{\,l} \underbrace{\delta_0(y-x)}_{[\text{kN/m}]}\,\underbrace{\Np_i(y)}_{[\text{m}]}\,\underbrace{dy}_{\text{[m]}} = \Np_i(x)  \qquad x = 2.5
\end{align}
zahlenm\"{a}{\ss}ig einfach die Werte der vier Ansatzfunktionen $\Np_i$ im Aufpunkt $x = 2.5$ sind, so kommt die Liste (\ref{Eq110}) zustande, siehe auch Abb. \ref{U451}.

%%%%%%%%%%%%%%%%%%%%%%%%%%%%%%%%%%%%%%%%%%%%%%%%%%%%%%%%%%%%%%%%%%%%%%%%%%%%%%%%%%%%%%%%%%%%%%%%%%%
{\textcolor{sectionTitleBlue}{\section{Funktionale}}}
Um nun doch etwas systematischer vorzugehen, wollen wir den Begriff des Funktionals\index{Funktional} einf\"{u}hren.

Wenn wir die Komponente $u_i$ eines Vektors $\vek u$ abfragen, dann werten wir streng genommen das Funktional
\begin{align}
J_i(\vek u) = u_i = \vek e_i^T\,\vek u
\end{align}
aus, das dem Skalarprodukt zwischen dem Einheitsvektor $\vek e_i$, dem diskreten Dirac Delta, und $\vek u$ entspricht.

Geht es um die Auswertung von Funktionen, dann sind die Funktionale Integrale, wie etwa der Ausdruck
\begin{align} \label{Eq43}
J(u) = \int_0^{\,l} u(x)\,dx\,.
\end{align}
Der Wert dieses Funktionals ist gleich dem Integral der Funktion $ u(x) $ \"{u}ber das Intervall $(0,l)$.
Funktionale sind also im allgemeinen Funktionen von Funktionen.

Funktionale wie
\begin{align}
J(u) = u(0)
\end{align}
nennt man {\em Punktfunktionale\/}\index{Punktfunktionale}, weil sie einen Wert in einem Punkt zur\"{u}ckgeben
\begin{align}
J(\sin(x)) = \sin(0) = 0 \quad J(e^x) = e^0 = 1\,.
\end{align}
Lineare Funktionale\index{lineares Funktional} wie das Integral einer Funktion
\begin{align}
J(u_1 + u_2) &= \int_0^{\,l} (u_1(x) + u_2(x))\,dx \nn \\
& = \int_0^{\,l} u_1(x)\,dx + \int_0^{\,l} u_2(x)\,dx = J(u_1) + J(u_2)\,,
\end{align}
lassen sich \glq superponieren\grq{}.

Jede Lagerkraft, jede Durchbiegung, jedes Moment, etc., ist ein (nat\"{u}rlich jeweils anderes) Funktional $J(w)$
\begin{align}\label{Eq20}
J(w) = V(0) \qquad J(w) = w(x) \qquad J(w) = M(x)\,.
\end{align}
Der entscheidende Schritt ist nun, dass wir die Auswertung eines linearen Funktionals auf ein Arbeitsintegral, ein $L_2$-Skalarprodukt zur\"{u}ckf\"{u}hren und dabei das Dirac Delta als Vorbild nehmen.

Das Funktional $J(w) = w(x)$, die Durchbiegung des Tr\"{a}gers in einem Punkt $x$, ist im Ergebnis die \"{U}berlagerung von $w$ mit einem Dirac Delta $\delta(y-x)$ (= Punktlast)
\begin{align}\label{Eq125}
J(w) = 1 \cdot w(x) = \int_0^{\,l} \delta(y-x)\,w(y)\,dy \qquad [\text{kNm}]\,.
\end{align}
Das Dirac Delta spielt hier dieselbe Rolle, wie oben der Einheitsvektor $\vek e_i$.

Und diese Interpretation wenden wir jetzt konsequent an. Jedes Funktional ist f\"{u}r uns das Ergebnis der \"{U}berlagerung eines Dirac Deltas mit $w$
\begin{align}
J(w) = \int_0^{\,l} \delta(y-x)\,w(y)\,dy \,.
\end{align}
Wie wir gleich sehen werden, m\"{u}ssen wir gar nicht wissen, wie diese verschiedenen Dirac Deltas aussehen. Wir m\"{u}ssen nur wissen, was das Ergebnis $J(w)$ sein soll. Und weil wir jedes Funktional als Arbeitsintegral lesen, hat jedes so dargestellte Funktional die Dimension einer Arbeit
\begin{align}
J(w) = 1 \cdot \text{\glq irgendetwas\grq{}}
\end{align}
wobei die 1 immer eine solche Dimension hat, dass $1 \cdot\text{\glq irgendetwas\grq{}}$ die Dimension einer Arbeit hat, die 1 also konjugiert zu ihrem Begleiter ist. So lesen wir die Funktionale in (\ref{Eq20}) wie folgt
\begin{alignat}{3}
J(w) &= V(x) \cdot 1 &&= \text{kN}  \cdot \text{m}\, &&\quad\text{1 = Versetzung} \\
J(w) &= w(x) \cdot 1 &&= \text{m}   \cdot \text{kN}\,&&\quad\text{1 = Kraft}\\
J(w) &= M(x) \cdot 1 &&= \text{kNm} \cdot [\,] \,      &&\quad\text{1 = Knick }\,.
\end{alignat}
Der Knick ist ein Sprung in der Ableitung $w'$ also im Tangens, der ja als Quotient zweier L\"{a}ngen, $dw/dx$, keine Dimension hat. Im Folgenden schreiben wir die $1$ meist nicht mit an, aber wir denken sie dann immer mit.

Die Einf\"{u}hrung der Dirac Deltas hat den Vorteil, dass wir die zugeh\"{o}rige Einflussfunktion als die L\"{o}sung der Differentialgleichung (wir bleiben der Einfachheit halber beim Balken) mit dem Dirac Delta auf der rechten Seite interpretieren k\"{o}nnen
\begin{align}
EI\,\frac{d^4}{dy^4}\,G(y,x) = \delta (y-x)\,.
\end{align}
(Weil wir den Aufpunkt mit $x$ bezeichnen, benutzen wir $y$ als Laufvariable und deswegen stehen links Ableitungen nach $y$).

Die \"{a}quivalenten Knotenkr\"{a}fte, die die Einflussfunktion generieren, sind dann einfach die Zahlen
\begin{align}
j_i = \int_0^{\,l} \delta(y-x)\,\Np_i(y)\,dy = J(\Np_i)\,,
\end{align}
also die Werte $J(\Np_i)$ der Ansatzfunktionen. Einfacher geht es eigentlich nicht mehr.
Das bedeutet also:

 (1) Die Knotenkr\"{a}fte $j_i$, die die Einflussfunktion f\"{u}r die Durchbiegung eines Balkens in einem Punkt $x$ erzeugen, sind die Durchbiegungen der Ansatzfunktionen $\Np_i$ in diesem Punkt
\beq
j_i = \Np_i(x) \,.
\eeq
(2) Die Kr\"{a}fte $j_i$, die die Einflussfunktion f\"{u}r das Moment $M(x) $ in einem Punkt $x$ eines Balkens erzeugen,
\beq
j_i = - EI\,\Np_i''(x) = M(\Np_i)(x)\,,
\eeq
sind die Momente der Ansatzfunktionen in diesem Punkt $x$ -- etc.
%---------------------------------------------------------------------------------
\begin{figure}
\centering
{\includegraphics[width=0.9\textwidth]{\Fpath/U34}}
\caption{FE-Modell eines Seils, \textbf{ a)} Ansatzfunktionen,  \textbf{ b)} FE-Einflussfunktion f\"{u}r $w$ im Punkt $x = 1.25$ und exakter Wert (0.94),  \textbf{ c)} FE-Einflussfunktion f\"{u}r $w$ im ersten Knoten, die Funktion ist exakt, $G_h(y,x) = G(y,x)$}
\label{U34}
%
\end{figure}%%
%---------------------------------------------------------------------------------

Formulieren wir das als Regel:\\

\begin{theorem}[Knotenkr\"{a}fte f\"{u}r Einflussfunktionen]\index{Knotenkr\"{a}fte f\"{u}r Einflussfunktionen}
Die Einflussfunktion f\"{u}r ein lineares Funktional $J(u)$ wird durch die Knotenkr\"{a}fte $j_i = J(\Np_i)$ erzeugt. Die Knotenkr\"{a}fte sind also zahlenm\"{a}{\ss}ig einfach die Werte $J(\Np_i)$ der Ansatzfunktionen.
\end{theorem}

\hspace*{-12pt}\colorbox{highlightBlue}{\parbox{0.98\textwidth}{Die Knotenwerte der Einflussfunktionen nennen wir $g_i$ und die \"{a}quivalenten Knotenkr\"{a}fte $j_i$, so dass das System $\vek K\,\vek u = \vek f$ jetzt also
\begin{align}
\vek K\,\vek g = \vek j
\end{align}
hei{\ss}t. Die Bedeutung der $g_i \equiv u_i$ und $j_i \equiv f_i$ \"{a}ndert diese Umbenennung nat\"{u}rlich nicht.}}\\

%%%%%%%%%%%%%%%%%%%%%%%%%%%%%%%%%%%%%%%%%%%%%%%%%%%%%%%%%%%%%%%%%%%%%%%%%%%%%%%%%%%%%%%%%%%%%%%%%%%
{\textcolor{sectionTitleBlue}{\section{Schwache und starke Einflussfunktionen}}}
In Kapitel 1 haben wir \"{u}ber den Unterschied zwischen schwachen und starken Einflussfunktionen gesprochen und gezeigt, dass es nicht m\"{o}glich ist mit schwachen Einflussfunktionen Kraftgr\"{o}{\ss}en zu berechnen.

Es mag daher eine \"{U}berraschung sein, dass die finiten Elemente (scheinbar, s. S. \pageref{Eq155}) nicht zwischen schwachen und starken Einflussfunktionen unterscheiden. Eine FE-Einflussfunktion ist die L\"{o}sung des Variationsproblems
\begin{equation}\label{EE7Equationforz}
G_h \in \mathcal{V}_h: \qquad
a(G_h,\varphi_i ) = J(\varphi_i ) \qquad \text{f\"{u}r alle}\,\,\varphi_i  \in \mathcal{V}_h
\end{equation}
und diese Funktion $G_h$ kann in beide Formulierungen eingesetzt werden
\begin{equation}
J(u_h) = \underbrace{\int_0^{\,l} G_h(y,x)\,p(y)\,dy}_{stark} =
\underbrace{\vphantom{\int_0^{\,l} }a(G_h,u_h)}_{schwach}\,,
\end{equation}
weil auf $\mathcal{V}_h$ die beiden Formeln -- stark und schwach -- zusammenfallen
\begin{equation}
J(u_h) = \int_0^{\,l}
G_h(y,x)\,p(y)\,dy = \vek g^T\,\vek f = \vek g^T\,\vek K\,\vek
u = a(G_h,u_h)\,.
\end{equation}
In Matrizenschreibweise besteht der Unterschied nur darin, wie man die Gleichungen liest
\begin{equation}
J(u_h) = \underbrace{\vek g^T\,\vek f}_{stark} = \underbrace{\vek g^T\,\vek K\,\vek u}_{schwach}\,.
\end{equation}
In der schwachen Formulierung $\vek g^T\,\vek K\,\vek u$ summieren wir \"{u}ber alle Eintr\"{a}ge
\begin{align}
\sum_{i,j} g_i\cdot k_{ij}\cdot u_j = \sum_{i,j} g_i \cdot a(\Np_i,\Np_j) \cdot u_j\,,
\end{align}
was einem Gebietsintegral (wie \glq Mohr\grq{}) gleichkommt, w\"{a}hrend die starke Formulierung $\vek g^T\,\vek f$ im Gegensatz dazu die Knotenverschiebungen $g_i$ mit den Knotenkr\"{a}ften $f_i$ wichtet.

%%%%%%%%%%%%%%%%%%%%%%%%%%%%%%%%%%%%%%%%%%%%%%%%%%%%%%%%%%%%%%%%%%%%%%%%%%%%%%%%%%%%%%%%%%%%%%%%%%%
{\textcolor{sectionTitleBlue}{\section{Beispiele}}}
\begin{example}
Um die Einflussfunktion f\"{u}r die Durchbiegung $w(x)$ des Seils in Abb. \ref{U34} im Punkt $x = 1.25$ zu berechnen, $J(w) = w(1.25)$, werden die Durchbiegungen der $\Np_i$ im Punkt $x$ als Knotenkr\"{a}fte aufgebracht
\beq
j_1 = \Np_1(x) = 0.75 \quad j_2 = \Np_2(x) = 0.25 \quad j_3 = \Np_3(x) = 0 \quad j_4 = \Np_4(x) = 0\,.
\eeq
Das Gleichungssystem
\beq\label{Eq176}
 \left[\barr{r r r r} 2 & - 1 & 0 & 0 \\ - 1 & 2 & -1 & 0\\ 0 & -1 & 2 &-1 \\ 0 & 0 & -1 &2\earr\right]
\,\left[\barr{c} g_1 \\g_2 \\ g_3 \\ g_4 \earr \right] = \left[\barr{c} 0.75 \\ 0.25  \\
0  \\ 0 \earr \right]
\eeq
f\"{u}r die Knotenverschiebungen $g_i$ der Einflussfunktion hat die L\"{o}sung
\beq
g_1 = 0.75 \qquad g_2 = 0.75 \qquad g_3 = 0.5 \qquad g_4 = 0.25
\eeq
und so lautet die Einflussfunktion
\beq
G(y,x) = 0.75 \cdot \Np_1(y) + 0.75 \cdot \Np_2(y) + 0.5 \cdot \Np_3(y) + 0.25 \cdot \Np_4(y)\,.
\eeq
Ginge es um die Berechnung der Einflussfunktion f\"{u}r die Durchbiegung im ersten Knoten, $x = 1.0$, h\"{a}tten die Knotenkr\"{a}fte die Werte
\beq
j_1 = \Np_1(x) = 1.0 \quad j_2 = \Np_2(x) = 0 \quad j_3 = \Np_3(x) = 0 \quad j_4 = \Np_4(x) = 0\,,
\eeq
und dann w\"{a}re die FE-Einflussfunktion f\"{u}r $w(x)$ sogar exakt, weil die Zahlen
\beq
g_1 = 0.8 \qquad g_2 = 0.6 \qquad g_3 = 0.4 \qquad g_4 = 0.2
\eeq
genau die Knotenwerte der Einflussfunktion sind und sie dazwischen linear verl\"{a}uft, siehe Abb. \ref{U34} c.
\end{example}
%-----------------------------------------------------------------
\begin{figure}[tbp]
\centering
\if \bild 2 \sidecaption \fi
\includegraphics[width=0.9\textwidth]{\Fpath/U37}
\caption{FE-Einflussfunktion f\"{u}r \textbf{ a)} das Biegemoment $M$ und \textbf{ b)} die Querkraft $V$ im Punkt $0.5\,\ell_e$. Die Knotenkr\"{a}fte sind die $j_i = M(\Np_i)$ bzw. $j_i = V(\Np_i)$, s. (\ref{Eq219X})}
\label{U37}
\end{figure}%
%-----------------------------------------------------------------

\begin{remark}
Die $g_i$ \"{a}ndern sich mit der Lage $x$ des Aufpunktes, sie sind also  Funktionen von $x$, so dass eine FE-Einflussfunktion im allgemeinen eine separierte Gestalt hat
\beq
G(y,x) = g_1(x) \cdot \Np_2(y) + g_2(x) \cdot \Np_2(y) + g_3(x) \cdot \Np_3(y) + g_4(x) \cdot \Np_4(y)\,.
\eeq
\end{remark}

\begin{example}
Die Einflussfunktion f\"{u}r das Biegemoment $M = - EI\,w''$ des Durchlauftr\"{a}gers in Abb. \ref{U37} im Punkt $x = 4.5$  wird von den \"{a}quivalenten Knotenkr\"{a}ften, siehe Abb.  \ref{U22},
\beq
j_i = - EI\,\Np_i''(x) \cdot 1 \,\,\,\text{[kNm]} \qquad 1 = \text{Knick}
\eeq
erzeugt und die Einflussfunktion f\"{u}r die Querkraft $V(x) = - EI\,w'''(x)$ von den Kr\"{a}ften
\beq
j_i = - EI\,\Np_i'''(x) \cdot 1\,\,\,\text{[kNm]} \qquad 1 = \text{Versatz}\,,
\eeq
wobei die $\Np_i$ die Ansatzfunktionen ({\em shape functions\/}\index{shape functions}) sind, die auf einem einzelnen Element mit der L\"{a}nge $\ell$ die Gestalt
%-----------------------------------------------------------------
\begin{figure}[tbp]
\centering
\if \bild 2 \sidecaption \fi
\includegraphics[width=1.0\textwidth]{\Fpath/U22}
\caption{Balkenelement, \textbf{ a)} die vier Ansatzfunktionen $\Np_i$ und
\textbf{ b)} die dazu geh\"{o}rigen Biegemomente $M$ und \textbf{ c)}
Querkr\"{a}fte $V$, s. (\ref{Eq219X})}
\label{U22}
\end{figure}%
%-----------------------------------------------------------------



\begin{subequations}\label{Eq219}
\begin{alignat}{3}
\Np_1(x) &= 1 - \frac{3x^2}{\ell^2} + \frac{2x^3}{\ell^3} &\qquad \Np_2(x) &= - x + \frac{2x^2}{\ell} - \frac{x^3}{\ell^2} \\
\Np_3(x) &= \frac{3x^2}{\ell^2} - \frac{2x^3}{\ell^3} &\qquad \Np_4(x) &= \frac{x^2}{\ell} - \frac{x^3}{\ell^2}  \label{Einheitsverformungen}
\end{alignat}
\end{subequations}
haben. Ihre Schnittgr\"{o}{\ss}en $M_i = - EI\,\Np_i''$ und $V_i = - EI\,\Np_i'''$ sind
\begin{subequations}\label{Eq219X}
\allowdisplaybreaks{}
\begin{alignat}{3}
M_1(x) &=   (\frac{6}{\ell^2} - \frac{12x}{\ell^3}) \cdot EI  &\qquad  M_2(x) &=  (\frac{6 x}{\ell^2}-\frac{4}{\ell}) \cdot EI \\
M_3(x) &= (\frac{12x}{\ell^3}-\frac{6}{\ell^2})\cdot EI &\qquad M_4(x) &= (\frac{6 x}{\ell^2}-\frac{2}{\ell})\cdot EI \\
V_1(x) &=   - \frac{12}{\ell^3} \cdot EI &\qquad V_2(x) &=  \frac{6}{\ell^2} \cdot EI \\
V_3(x) &=  \frac{12}{\ell^3}\cdot EI &\qquad  V_4(x) &= \frac{6 }{\ell^2}\cdot EI\,.
\end{alignat}
\end{subequations}

Nur die Knoten des Elements, das den Aufpunkt $x$ enth\"{a}lt, tragen Knotenkr\"{a}fte $j_i$, weil die Ansatzfunktionen $\Np_i$ aller anderen Elemente, die weiter weg liegen, null Momente bzw. null Querkr\"{a}fte im Aufpunkt $x$ haben.
%----------------------------------------------------------------------------------------------------------
\begin{figure}[tbp]
\centering
\if \bild 2 \sidecaption[t] \fi
\includegraphics[width=1.0\textwidth]{\Fpath/U6}
\caption{CST-Element,  \textbf{ a)} Geometrie und Freiheitsgrade,  \textbf{ b)} diese Knotenkr\"{a}fte erzeugen die Einflussfunktion f\"{u}r die Spannung $\sigma_{xx}$. Sie ist konstant,  wie es bei einem CST-Element ({\em constant strain element\/}) \index{CST-Element} ja auch sein muss}
\label{U6}
\end{figure}%%
%----------------------------------------------------------------------------------------------------------

Die so erzeugten Einflussfunktionen sind au{\ss}erhalb des Elementes, auf dem der Aufpunkt liegt, exakt. Nur im Element m\"{u}ssen sie korrigiert werden. In der Praxis macht man das so, wie oben schon erl\"{a}utert, dass man zu der FE-Einflussfunktion die lokale L\"{o}sung addiert.

Mit den obigen Formeln kann man also Einflussfunktionen f\"{u}r die Momente bzw. Querkr\"{a}fte in einem beliebigen Punkt $0 \leq x \leq l$ eines Elements berechnen. Die Knotenkr\"{a}fte in den beiden Knoten des Elements sind die Momente $M_i(x)$ (EF f\"{u}r $M$) bzw. die Querkr\"{a}fte $V_i(x)$ (EF f\"{u}r $V$) der $\Np_i$ im Aufpunkt $x$. Alle anderen Knoten des Tragwerks sind lastfrei, $f_i = 0$.
\end{example}


\begin{example}
In einem CST-Element ({\em constant strain triangle\/}) wie in Abb. \ref{U6} sind die Spannungen konstant
\begin{align}
\left[\barr{r} \sigma_{xx} \\ \sigma_{yy} \\ \sigma_{xy}\earr \right]  = \frac{E}{2\,A} \left[ \barr {r @{\hspace{4mm}} r @{\hspace{4mm}} r
@{\hspace{4mm}} r  @{\hspace{4mm}} r @{\hspace{4mm}} r}
      y_{23} & 0 &y_{31} &0 & y_{12} & 0  \\
     0 & x_{23} & 0 &x_{13} &0 & x_{21}  \\
        x_{32} & y_{23} &x_{13} &y_{31} & x_{21} & y_{12}
    \earr \right]\,\left[\barr{r} u_1 \\ v_1  \\ u_2 \\ v_2 \\ u_3 \\ v_3\earr \right]\,.
\end{align}
Es bedeutet $x_{ij} = x_i - x_j$ und $y_{ij} = y_i - y_j$.
%----------------------------------------------------------------------------------------------------------
\begin{figure}[tbp]
\centering
\if \bild 2 \sidecaption[t] \fi
\includegraphics[width=0.6\textwidth]{\Fpath/U221}
\caption{Bilineares Element}
\label{U221}
\end{figure}%%
%----------------------------------------------------------------------------------------------------------

Um die Einflussfunktion f\"{u}r $\sigma_{xx}$ zu erzeugen, l\"{a}sst man in den Knoten des Elements die Spannungen $\sigma_{xx}$ aus den Einheitsverformungen $\vek \Np_i(\vek x)$ der Knoten wirken, $j_i = \sigma_{xx}(\vek \Np_i)(\vek x)$, also
\begin{align}
j_1 = \frac{E}{2\,A}\,y_{23} \cdot 1 \quad j_3 =  \frac{E}{2\,A}\,y_{31} \cdot 1 \quad j_5 =  \frac{E}{2\,A}\,y_{12} \cdot 1\quad j_2 = j_4 = j_6 = 0 \,.
\end{align}
Man beachte, dass die Summe der $j_i$ null ist
\begin{align}
j_1 + j_3 + j_5 =  \frac{E}{2\,A}( y_2 - y_3 + y_3 - y_1 + y_1 - y_2) = 0\,.
\end{align}
Das ist bei allen Einflussfunktionen f\"{u}r Kraftgr\"{o}{\ss}en so, weil man im Grunde das Element mit dem Aufpunkt durch gegengleiche Kr\"{a}fte $j_i$ spreizt.
\end{example}
\begin{example}
Bei den n\"{a}chsten beiden Beispielen rechnen wir mit bilinearen Scheibenelementen. Ein bilineares Element hat vier Knoten und $2 \cdot 4$ Freiheitsgrade, s. Abb. \ref{U38G}. Zu jedem Freiheitsgrad  (FG)\index{FG} geh\"{o}rt ein Verschiebungsfeld $\vek \Np_i(\vek x)$, das den Knoten in horizontaler oder vertikaler Richtung auslenkt
%----------------------------------------------------------------------------------------------------------
\begin{figure}[tbp]
\centering
\if \bild 2 \sidecaption \fi
\includegraphics[width=1.0\textwidth]{\Fpath/U38G}
\caption{Bilineare Elemente, \textbf{ a)} Einflussfunktion f\"{u}r $u_x(\vek x)$ und \textbf{ b)} f\"{u}r $\sigma_{yy}(\vek x)$}
\label{U38G}
\end{figure}%%
%----------------------------------------------------------------------------------------------------------
\beq
\vek \Np_1(\vek x) = \left[\barr{c} \psi_1(\vek x) \\ 0 \earr \right] \quad \vek \Np_2(\vek x) = \left[\barr{c} 0 \\ \psi_1(\vek x)  \earr \right] \quad\vek \Np_3(\vek x) = \left[\barr{c} \psi_2(\vek x) \\ 0 \earr \right] \quad \mbox{etc.}
\eeq
Die $\psi_i(\vek x)$ sind die vier Ansatzfunktionen der vier Eckpunkte, siehe Abb. \ref{U221},
\begin{alignat}{2}\label{Eq179}
\psi_1(\vek x) &= \frac{1}{4\,a\,b} \,(a - 2 x)(b - 2 y)\qquad &\psi_2(\vek x) &= \frac{1}{4\,a\,b} \,(a + 2 x)(b - 2 y) \\
\psi_3(\vek x) &= \frac{1}{4\,a\,b} \,(a + 2 x)(b + 2 y)\qquad &\psi_4(\vek x) &= \frac{1}{4\,a\,b}
\,(a - 2 x)(b + 2 y)\,.
\end{alignat}
Wenn also ein Verschiebungsfeld einen Knoten in horizontaler Richtung dr\"{u}ckt, dann sind alle vertikalen Verschiebungen null und umgekehrt. Solche Verschiebungsfelder machen es leicht, die Bewegungen der Knoten zu kontrollieren.

{\textcolor{sectionTitleBlue}{\subsubsection*{Einflussfunktion f\"{u}r $u_x$}}}\index{Einflussfunktion f\"{u}r $u_x$}

Um die Einflussfunktion f\"{u}r die horizontale Verschiebung in dem Viertelspunkt eines Elementes mit der L\"{a}nge $a = 2$ und H\"{o}he $b = 1$ zu generieren, l\"{a}sst man vier horizontale Kr\"{a}fte in den vier Ecken des Elementes wirken. Diese Kr\"{a}fte sind die Verschiebungen der vier horizontalen Verschiebungsfelder, Indices $1, 3, 5, 7$, im Aufpunkt $\vek x = (-0.5, -0.25)$ (Element-Koordinaten)
\beq
j_1 = 0.5625 \qquad j_3 = 0.1875 \qquad j_5 = 0.0625 \qquad j_7 = 0.1875 \,,
\eeq
und sie erzeugen die Verformung in Abb. \ref{U38G}. (Die vier vertikalen Verschiebungsfelder haben nat\"{u}rlich null Horizontalverschiebungen im Aufpunkt und daher sind auch die $j_i$ in vertikaler Richtung, $j_2, j_4, j_6, j_8$, alle null).

{\textcolor{sectionTitleBlue}{\subsubsection*{Einflussfunktion f\"{u}r  $\sigma_{xx}$}}}\index{Einflussfunktion f\"{u}r  $\sigma_{xx}$}

Die Einflussfunktion f\"{u}r die Spannung $\sigma_{yy}$ in dem Viertelspunkt entsteht, wenn man die Spannungen $\sigma_{yy}(\vek \Np_i)$ der $4 \times 2$ Verschiebungsfelder $\vek \Np_i$ als Knotenkr\"{a}fte $j_i$ aufbringt.
%-----------------------------------------------------------------
\begin{figure}[tbp]
\centering
\if \bild 2 \sidecaption \fi
\includegraphics[width=1.0\textwidth]{\Fpath/U75}
\caption{Einflussfunktion f\"{u}r das Integral von $\sigma_{xy}$ in einem senkrechten Schnitt, \textbf{ a)} \"{a}quivalente Knotenkr\"{a}fte, \textbf{ b)} Einflussfunktion, \cite{Ha6}}
\label{U75}
%
\end{figure}%
%-----------------------------------------------------------------

In einem bilinearen Element mit der L\"{a}nge $a$ und H\"{o}he $b$, wie in Abb. \ref{U221}, haben die Spannungen den Verlauf
\begin{align}\label{SigBilinear1}
\sigma_{xx}(x,y)&=\frac{E}{a\,
     b\,( -1 + \nu^2) }\cdot  \bigg[
     b\,( {u_1} - {u_3}
          )  + a\,\nu\,
        ( {u_2} - {u_8} )\,+\nn\\&\hphantom{=}
        + x\, \nu\,(-u_2+u_4 -u_6 +u_8) +y\,(-u_1 + u_3 -u_5+ u_7)\bigg]\\
\sigma_{yy}(x,y)&=\frac{E}{a\,b\,
     ( -1 + \nu^2)}\cdot  \bigg[
      b\,\nu\,
        ( {u_1} - {u_3} )  +
       a\,( {u_2} - {u_8} )\,+\nn \\&\hphantom{=}
        +x\,(-u_2 + u_4-u_6+ u_8) +y\,\nu\,(-u_1+u_3-u_5+u_7)  \bigg]
\end{align}
und
\begin{align}
\sigma_{xy}(x,y)&=\frac{- E}{2\,a\,b\,
     ( 1 + \nu )}\cdot  \bigg[
        b\,( {u_2} - {u_4}
             )  + a\,
          ( {u_1} - {u_7} )\,+ \nn \\&\hphantom{=}
        +x\,(-u_1 +u_3-u_5+ u_7)+ y\,(-u_2+ u_4-u_6+ u_8) \bigg]\,.
\end{align}
Setzen wir $u_1 = 1$ setzen und alle anderen $u_i = 0$, so erhalten wir die Spannungen, die zu dem Verschiebungsfeld $\vek \Np_1(\vek x)$ geh\"{o}ren. So betragen am ersten Knoten die \"{a}quivalenten Knotenkr\"{a}fte $j_i$ in horizontaler Richtung ($u_1 = 1$)
\begin{equation}
j_1 = \sigma_{yy}(x,y) = \frac{E}{a\,b\,
     ( -1 + \nu^2)}\cdot  \bigg[
      b\,\nu\,  {u_1} +y\,\nu\,(-u_1)  \bigg] = -3.07 \cdot 10^6\,\mbox{kNm}
\end{equation}
und in vertikaler Richtung ($u_2 = 1$)
\begin{equation}
j_2 = \sigma_{yy}(x,y) =\frac{E}{a\,b\, ( -1 + \nu^2)}\cdot  \bigg[
             a\,( {u_2} ) +x\,(-u_2) \bigg] = -3.85 \cdot 10^7\,\mbox{kNm}\,.
\end{equation}
Die anderen $j_i$ ergeben sich nach demselben Muster. Das Ergebnis und die Knotenkr\"{a}fte sind in Abb. \ref{U38G} b dargestellt.
%-----------------------------------------------------------------
\begin{figure}[tbp]
\centering
\if \bild 2 \sidecaption \fi
\includegraphics[width=0.85\textwidth]{\Fpath/U368}
\caption{Generierung der Einflussfunktionen in einem CST-Element, \textbf{ a)} f\"{u}r $\sigma_{xx}$, \textbf{ b)} f\"{u}r $\sigma_{yy}$; weil die Spannungen \"{u}berall gleich sind, gelten die Einflussfunktionen f\"{u}r jeden Punkt }
\label{U368}
%
\end{figure}%
%-----------------------------------------------------------------

Die $j_i$ verhalten sich wie \glq inverse\grq{} Gummib\"{a}nder\index{Gummib\"{a}nder}, was ein Charakteristikum der Einflussfunktionen f\"{u}r Spannungen ist, also f\"{u}r den Differenzenquotient des Verschiebungsfeldes im Aufpunkt.\label{rubberband}

Man kann die $j_i$, die das Element spreizen, auch {\em \glq geometrische Kr\"{a}fte\grq{}\/}\index{geometrische Kr\"{a}fte} nennen, weil ihre Gr\"{o}{\ss}e vom Zuschnitt des Elements abh\"{a}ngt.
%-----------------------------------------------------------------
\begin{figure}[tbp]
\centering
\if \bild 2 \sidecaption \fi
\includegraphics[width=0.9\textwidth]{\Fpath/U88}
\caption{Lokale L\"{o}sungen = ein-elementrige Einflussfunktionen am festgehaltenen Stab bzw. Balken}
\label{U88}
\end{figure}%
%-----------------------------------------------------------------

Bei einem langgezogenen Dreieck (CST-Element) mit dem Seitenverh\"{a}ltnis $l_x:l_y = 4:1$ braucht man z.Bsp. f\"{u}r die Spreizung in vertikaler Richtung das vierfache an Kraft gegen\"{u}ber einer Spreizung in horizontaler Richtung, s. Abb. \ref{U368}. Entsprechend unausgewogen sind auch die Elementbeitr\"{a}ge in der Steifigkeitsmatrix des Elements, weil da ja noch quadriert wird
\begin{align}
\int_{\Omega_e} \varepsilon_{yy}^2 \,d\Omega : \int_{\Omega_e} \varepsilon_{xx}^2 \,d\Omega = 16:1\,.
\end{align}
Bei der Berechnung einer Steifigkeitsmatrix wird zwar nichts gespreizt, sondern die Knoten werden um $u_i = 1$ ausgelenkt, aber das ist beim CST-Element dasselbe, weil die Spannungen konstant sind.

{\textcolor{sectionTitleBlue}{\subsubsection*{Einflussfunktion f\"{u}r $N_{xy}$}}}\index{Einflussfunktion f\"{u}r $N_{xy}$}

Nun soll die Einflussfunktion f\"{u}r das Integral der Schubspannungen
\beq
N_{xy} = \int_0^{\,l} \sigma_{xy}\,dy
\eeq
in einem vertikalen Schnitt, der durch einen vorgegebenen Punkt $\vek x = (x,y)$ l\"{a}uft, berechnet werden. Die \"{a}quivalenten Knotenkr\"{a}fte sind jetzt Integrale, siehe Abb. \ref{U75},
\beq
j_i = \int_0^{\,l} \sigma_{xy}(\vek \Np_i)\,dy\,,
\eeq
also die aufintegrierten Schubspannungen der Verschiebungsfelder, die zu den vier Ecken des Elementes geh\"{o}ren. In den vier Ecken jedes Elements, durch das der Schnitt f\"{u}hrt, werden die folgenden \"{a}quivalenten Knotenkr\"{a}fte aufgebracht
\begin{align}
j_i^e &= \int_0^{\,b} \sigma_{xy}(\vek \Np_i)\,dy = \frac{- E}{2\,a\,
     ( 1 + \nu )}\cdot  \bigg[
        b\,( {u_2} - {u_4}
             )  + a\,
          ( {u_1} - {u_7} )\,+ \nn \\
                &  +x\,(-u_1 +u_3-u_5+ u_7)+ \frac{b}{2}\,(-u_2+ u_4-u_6+ u_8) \bigg]\,.
\end{align}
mit $x $ als der $x$-Koordinate des Schnittes.

Um $j_1^e$ zu berechnen, setzen wir $u_1 = 1$ und alle anderen $u_i = 0$. F\"{u}r $j_2^e$ setzen wir $u_2 = 1$ und alle anderen $u_i = 0$, etc. Der Index $e$ an $j_i^e$ soll darauf hinweisen, dass dies Elementbeitr\"{a}ge sind. Die resultierende Knotenkraft ergibt sich durch die Summation \"{u}ber alle an den Knoten angeschlossenen Elemente.
\end{example}

%%%%%%%%%%%%%%%%%%%%%%%%%%%%%%%%%%%%%%%%%%%%%%%%%%%%%%%%%%%%%%%%%%%%%%%%%%%%%%%%%%%%%%%%%%%%%%%%%%%
{\textcolor{sectionTitleBlue}{\section{Die lokale L\"{o}sung}}}\index{lokale L\"{o}sung}\label{lokaleL\"{o}sung}
Die lokale L\"{o}sung ist die Einflussfunktion am beidseitig eingespannten Element, s. Abb. \ref{U88}. Diese L\"{o}sung muss zu der FE-Einflussfunk\-ti\-on in dem Element, in dem der Aufpunkt liegt, addiert werden.
%-----------------------------------------------------------------
\begin{figure}[tbp]
\centering
\if \bild 2 \sidecaption \fi
\includegraphics[width=1.0\textwidth]{\Fpath/U91A}
\caption{Berechnung der lokalen L\"{o}sung der Einflussfunktion f\"{u}r die Querkraft $V$ }
\label{U91}
\end{figure}%
%-----------------------------------------------------------------
%-----------------------------------------------------------------
\begin{figure}[tbp]
\centering
\if \bild 2 \sidecaption \fi
\includegraphics[width=.99\textwidth]{\Fpath/U471}
\caption{Einflussfunktion f\"{u}r das Moment bzw. die Querkraft im beidseitig eingespannten Stab, links und rechts vom Aufpunkt $x$ }
\label{U471}
\end{figure}%
%-----------------------------------------------------------------
Zum Exempel berechnen wir die Einflussfunktion f\"{u}r die Querkraft $V$, s. Abb. \ref{U91}, in einem beidseitig festgehaltenen Balken. Der linke Teil der Funktion, $w_L(x)$, ist im Punkt $x = 0$ eingespannt und gen\"{u}gt den statischen Randbedingungen\footnote{Das sind die Festhaltekr\"{a}fte aus der Spreizung des Aufpunkts}
\begin{align}
M(0) = - \frac{6}{3^2} EI  \qquad V(0) = - \frac{12}{3^3}\,EI\,,\
\end{align}
so dass $ w_L(x) = a\,\Np_3(x) + b\,\Np_4(x)$ der nat\"{u}rliche Ansatz f\"{u}r diese Funktion ist und die Koeffizienten $a = -1$ und $b = 0$ sind die L\"{o}sung des Systems
\begin{align}
\left[ \barr {r @{\hspace{4mm}}r @{\hspace{4mm}}r
@{\hspace{4mm}}r @{\hspace{4mm}}r}
      M_3(0) & M_4(0)  \\
      V_3(0) & V_4(0) \\
     \earr \right]\left [\barr{c}  a \\ b\earr \right ]
=  EI\,\left [\barr{c}  -6/3^2 \\  -12/3^3\earr \right ]
\end{align}
wobei $M_i$ und $V_i$ die Biegemomente bzw. Querkr\"{a}fte der {\em shape functions\/} $\Np_i(x)$ sind.

Rechts vom Aufpunkt w\"{a}hlen wir den Ansatz $w_R(x) = c\,\Np_1(x) + d\,\Np_2(x)$ und bestimmen $c = 1$ und $d = 0$ so, dass die statischen Randbedingungen
\begin{align}
M(l) = - \frac{6}{3^2} EI  \quad V(l) = - \frac{12}{3^3}\,EI\,,
\end{align}
erf\"{u}llt sind. Die Einflussfunktion hat somit die Gestalt
\begin{align}
w(x) = \left \{ \barr{r r} -\Np_3(x)  \qquad &0 < x < 1.5\\ \Np_1(x) \qquad &1.5 < x < 3.0 \earr \right.
\end{align}
Die lokale L\"{o}sung  basiert immer auf den $2 \times 2$ Einheitsverformungen des Elements; links auf den Funktionen $\Np_3(x)$ und $\Np_4(x)$ und rechts auf den Funktionen $\Np_1(x)$ und $\Np_2(x)$. Nat\"{u}rlich muss der Aufpunkt $x$ nicht in der Mitte des Elements liegen, s. Bild \ref{U471}.

%%%%%%%%%%%%%%%%%%%%%%%%%%%%%%%%%%%%%%%%%%%%%%%%%%%%%%%%%%%%%%%%%%%%%%%%%%%%%%%%%%%%%%%%%%%%%%%%%%%
{\textcolor{sectionTitleBlue}{\section{Die zentrale Gleichung}}}
FE-Einflussfunktionen liegen in separierter Form vor
\begin{align}
G_h(y,x) = g_1(x)\,\Np_1(y) + g_2(x)\,\Np_2(y) + \ldots + g_n(x)\,\Np_n(y)\,.
\end{align}
Das ist der Grund, warum bei der \"{U}berlagerung mit der Belastung $p(y)$ die \"{a}quivalenten Knotenkr\"{a}fte erscheinen
\begin{align}
w_h(x) = \int_0^{\,l} G_h(y,x)\,p(y)\,dy = \sum_{i = 1}^n\,g_i(x) \int_0^{\,l} p(y)\,\Np_i(y)\,dy = \sum_{i = 1}^n\,g_i(x)\,f_i\,,
\end{align}
und somit die Auswertung einer  Summation \"{u}ber die Knoten gleichkommt. F\"{u}r die Belastung werden die \"{a}quivalenten Knotenkr\"{a}fte gesetzt und der Einfluss einer Knotenkraft $f_i$ auf das Funktional $J(w_h) = w_h(x)$ ist gleich der Arbeit, die $f_i$ auf der zur Einflussfunktion geh\"{o}rigen Knotenverschiebung $g_i $ leistet.

Nun kann man $w_h(x)$ aber auch berechnen, indem man das Dirac Delta mit der Biegelinie $w_h$ \"{u}berlagert
\begin{align}
w_h(x) = \int_0^{\,l} \delta(y-x)\,w_h(y)\,dy\,.
\end{align}
Verkn\"{u}pfen wir diese beiden Darstellungen, dann haben wir die zentrale Gleichung zu dem Thema Einflussfunktionen und finite Elemente vor uns. \\


\begin{theorem}[Die zentrale Gleichung]\index{zentrale Gleichung}
\begin{align} \label{EE1LongChain}
w_h(x) &= \int_0^{\,l} G_h(y,x)\,p(y)\,dy = \int_0^{\,l} \sum_i\,g_i(x)\,\Np_i(y)\,p(y)\,dy = \sum_i\,g_i(x)\,f_i \nn\\
&= \vek g^T\,\vek f = \vek g^T\,\vek K\,\vek w = \vek g^T\,\vek K^T\,\vek w = \vek
j^T\,\vek w =  \sum_i\,j_i\,w_i \nn \\
&= \sum_i\,\Np_i(x)\,w_i =   \int_0^{\,l} \sum_i\, w_i\,\Np_i(y)\,\delta(y-x)\,dy
=\int_0^{\,l} w_h(y)\,\delta(y-x)\,dy\,. \nn \\
\end{align}
\end{theorem}

Die Durchbiegung $w_h(x)$ in einem Punkt $x$ ist, so lesen wir, das Skalarprodukt zwischen dem Vektor $\vek g$ der Knotenwerte der Einflussfunktion und dem Vektor der \"{a}quivalenten Knotenkr\"{a}fte $\vek f$ oder, umgekehrt, zwischen den Knotenwerten $w_i$ der FE-L\"{o}sung und den \"{a}quivalenten Knotenkr\"{a}ften $j_i = \Np_i(x)$ der Einflussfunktion
\beq\label{Eq84A}
w_h(x) = \left \{ \begin{array}{l } {\displaystyle  \int_0^{\,l} w_h(y)\,\delta(y-x)\,dy = \sum_i\,\Np_i(x)\,w_i = \vek j^T\,\vek w }          \\
{\displaystyle \int_0^{\,l} G_h(y,x)\,p(y)\,dy = \vek
g^T\,\vek f}\,.
\end{array} \right.
\eeq
Diese Darstellung gilt f\"{u}r alle linearen Funktionale
\beq
J(u) = \left \{ \begin{array}{l } {\displaystyle  \vek j^T\,\vek u }          \\
{\displaystyle \vek
g^T\,\vek f}\,,
\end{array} \right.
\eeq
und sie ist die denkbar knappste Darstellung eines linearen Funktionals.

Die erste Formel
\begin{align}
J(\vek u) = \vek j^T\,\vek u &= j_1\,u_1 + j_2\,u_2 + \ldots + j_n\,u_n \nn \\
&= J(\Np_1)\,u_1 + J(\Np_2)\,u_2 + \ldots + J(\Np_n)\,u_n
\end{align}
spielt die Berechnung von $J(\vek u)$ auf die Einzelwerte $j_i = J(\Np_i)$ zur\"{u}ck. Setzt man die $j_i$ als Kr\"{a}fte in die Knoten, wie in Abb. \ref{U37}, dann ist $J(\vek u)$ die Arbeit, die die $j_i$ auf den Wegen $u_i$ leisten.

In der zweiten Formel, $J(\vek u) = \vek g^T\,\vek f$, werden die \"{a}quivalenten Knotenkr\"{a}fte mit den Einflusskoeffizienten $g_i $ gewichtet, also den Knotenverschiebungen der FE-Einflussfunktion.

%%%%%%%%%%%%%%%%%%%%%%%%%%%%%%%%%%%%%%%%%%%%%%%%%%%%%%%%%%%%%%%%%%%%%%%%%%%%%%%%%%%%%%%%%%%%%%%%%%%
{\textcolor{sectionTitleBlue}{\section{Zustandsvektoren und Messungen}}}\label{Zustandsvektoren}

In einem \"{u}bertragenen Sinne repr\"{a}sentiert der Verschiebungsvektor $\vek u$ einen Zustandsvektor des Tragwerks und die Auswertung eines Funktionals
\begin{align}
J(\vek u) = \vek g^T\,\vek f = \vek g^T\,\vek K\,\vek u
\end{align}
kann man als eine Messung an dem Tragwerk betrachten. Die Frage ist nun, wie sich die Messwerte ver\"{a}ndern, wenn sich die Steifigkeitsmatrix ver\"{a}ndert, wenn das Tragwerk aus einem System $\vek K$ in ein System $\vek K \to \vek K + \vek \Delta \vek K$ \"{u}bergeht?

Die urspr\"{u}ngliche schwache Formulierung (Variationsformulierung)
\begin{align}
\vek \delta\,\vek u^T\,\vek K\,\vek  u= \vek \delta \vek u^T \,\vek f
\end{align}
und die ge\"{a}nderte Formulierung haben dieselbe rechte Seite
\begin{align}
\vek \delta\,\vek u^T\,(\vek K + \vek \Delta \vek K) \,\vek  u_c= \vek \delta \vek u^T \,\vek f\,,
\end{align}
so dass die Differenz der beiden Gleichungen, $\vek e = \vek u_c - \vek u$, den Ausdruck
\begin{align}
\vek \delta \vek u^T\,\vek K\,\vek e = - \vek \delta \,\vek u^T\,\vek \Delta\,\vek K\,\vek u_c
\end{align}
ergibt. Setzen wir f\"{u}r $\vek  \delta \vek u$ den Vektor $\vek g$ der Einflussfunktion dann folgt
\begin{align}
J(\vek e)= -  \vek g^T\,\vek \Delta\,\vek K\,\vek u_c\,,
\end{align}
was ein lokales Resultat ist, zumindest so lokal wie die Matrix $\vek \Delta \vek K$, weil wir zur Auswertung von $J(\vek e)$ nur \"{u}ber das ge\"{a}nderte Element $\Omega_e$ integrieren m\"{u}ssen (sinngem\"{a}{\ss} ist die rechte Seite ja so etwas wie das Mohrsche Arbeitsintegral beim Balken).

Man stelle sich einen gro{\ss}en ebenen Rahmen vor, in dem ein einzelner Riegel rei{\ss}t, $EI \to EI + \Delta EI$, $\Delta EI < 0$. Der Riss bedeutet einen \"{U}bergang von der Matrix $\vek K$ zu einer neuen Matrix $\vek K + \vek \Delta \vek K$, wobei die Zusatzmatrix $\vek \Delta \vek K = \Delta EI/EI \cdot \vek K_e$ die kleine, urspr\"{u}ngliche Steifigkeitsmatrix des gerissenen Elementes, gewichtet mit dem Faktor $\Delta EI/EI$ ist -- klein verglichen mit der Gr\"{o}{\ss}e von $\vek K$.

Wenn der Rahmen $2\,n$ Freiheitsgrade $u_i$ hat, dann k\"{o}nnten wir theoretisch $2\,n$ Messungen $J_i(\vek e) = u_{ic} - u_i$ mit den $2\,n$ Knotenvektoren $\vek g_i$ an dem gerissenen Element
\begin{align}
J_i(\vek e) = u_{ic} - u_i = -\vek g_i^T\,\vek \Delta\,\vek K\,\vek u_c
\end{align}
vornehmen und so die ge\"{a}nderten Knotenverschiebungen
\begin{align}
u_{ic} = u_i + J_i(\vek e)
\end{align}
-- in Matrizenschreibweise ist das
\begin{align}\label{Eq161}
\vek u_c = \vek u + \vek K^{(-1)}\,\vek \Delta\,\vek K\,\vek u_c
\end{align}
berechnen, weil die Spalten der Inversen gerade die Knotenverschiebungsvektoren $\vek g_i$ der Einflussfunktionen f\"{u}r die Knotenverschiebungen $u_i$ sind.

In Kapitel 5 schreiben wir das als
\begin{align}\label{Eq7}
\vek K\,\vek u_c = \vek K\,\vek u + \vek \Delta\,\vek K\,\vek u_c = \vek f + \vek f^+\,.
\end{align}
Das Problem dabei ist, dass wir nat\"{u}rlich den Vektor $\vek u_c$ nicht kennen, den wir brauchen um $J_i(\vek e)$ zu berechnen. In dieser Situation k\"{o}nnten wir f\"{u}r den Vektor $\vek u_c$ n\"{a}herungsweise den urspr\"{u}nglichen Vektor $\vek u$ setzen
\begin{align}
J_i(\vek e)  \simeq \vek g_i^T\,\vek \Delta\,\vek K\,\vek u\,,
\end{align}
oder $\vek u_c$ durch Iteration bzw. eine (stark verk\"{u}rzte) direkte Berechnung bestimmen, s. Kapitel 5, S. \pageref{Eq59}. \\

\begin{remark}
Das obige Resultat (\ref{Eq7}) impliziert, dass
\begin{align}\label{Eq162}
(\vek I + \vek K^{-1}\,\vek \Delta \,\vek K)\,\vek u_c = \vek u
\end{align}
gelten muss, was bedeutet, dass ein Tragwerk im \"{U}bergang von  $\vek K$ zu \vek K + \vek \Delta \vek K den Zuwachs $\vek \Delta \vek u = \vek u_c - \vek u$ nicht frei w\"{a}hlen kann, sondern dass der neue Zustand $\vek u_c$ mit dem vorherigen Zustand $\vek u$ kompatibel sein muss.

Theoretisch  k\"{o}nnte man mit einer ganz einfachen Steifigkeitsmatrix  $\vek K_0$ starten, die man dann mit weiteren Matrizen $\vek \Delta \vek K_i$ anreichert
\begin{align}
\vek K_i = \vek K_o + \vek \Delta \vek K_1 + \vek \Delta \vek K_2 + \ldots \vek \Delta \vek K_i
\end{align}
und so w\"{u}rde man eine Kette von L\"{o}sungen $\vek u_i$ produzieren, wo jeder neue Vektor $\vek u_{i + 1}$ mit dem vorhergehenden Vektor verschr\"{a}nkt ist
\begin{align}\label{Eq163}
(\vek I + \vek K_i^{-1}\,\vek \Delta \,\vek K_{i + 1})\,\vek u_{i + 1} = \vek u_i\,.
\end{align}
\end{remark}


%%%%%%%%%%%%%%%%%%%%%%%%%%%%%%%%%%%%%%%%%%%%%%%%%%%%%%%%%%%%%%%%%%%%%%%%%%%%%%%%%%%%%%%%%%%%%%%%%%%
{\textcolor{sectionTitleBlue}{\section{Der Satz von Maxwell}}}\label{Maxwell}
Eigentlich geh\"{o}rt der {\em Satz von Maxwell\/} in das Kapitel 2, aber wir mussten auf den Begriff des Funktionals warten.

%---------------------------------------------------------------------------------
\begin{figure}
\centering
\if \bild 2 \sidecaption \fi
\includegraphics[width=1.0\textwidth]{\Fpath/U354A}
\caption{Zwei Einflussfunktionen, \textbf{ a)} eine Einzelkraft generiert die Einflussfunktion $\vek G_1$ f\"{u}r $u_x(\vek x_a)$ und \textbf{ b)} ein Versatz erzeugt die Einflussfunktion $\vek G_2$ f\"{u}r die Spannung $\sigma_{xx}$ im Punkt $\vek x_b$; die beiden Kerne sind adjungiert, $J_2(\vek G_1) = J_1(\vek G_2)$}
\label{UE354}%
\end{figure}%
%---------------------------------------------------------------------------------
Den {\em Satz von Maxwell\/} kennt der Ingenieur als die Gleichung
\begin{align} \label{Eq126}
w_1(x_2) = w_2(x_1)\,.
\end{align}
{\em Die Durchbiegung, die eine Kraft $P = 1$ am Ort $x_1$ in einem abliegenden Punkt $x_2$ erzeugt, ist genauso gro{\ss}, wie die Durchbiegung, die eine Kraft $P = 1$ am Ort $x_2$ im Punkt $x_1$ erzeugt, s. Abb. \ref{U128} S. \pageref{U128}\/}.
%---------------------------------------------------------------------------------
\begin{figure}
\centering
\if \bild 2 \sidecaption \fi
\includegraphics[width=1.0\textwidth]{\Fpath/U317}
\caption{Zwei Einflussfunktionen und ihre Gleichheit $J_2(G_1) = J_1(G_2)$ \"{u}ber Kreuz}
\label{U317}%
\end{figure}%
%---------------------------------------------------------------------------------

Wir k\"{o}nnen dies dahingehend verallgemeinern, dass wir sagen, dass die Kerne zweier Funktionale \glq \"{u}ber Kreuz\grq{} gleich sind, was bedeuten soll
\begin{align}\label{Eq127}
\boxed{J_1(G_2) = J_2(G_1)}\,.
\end{align}
In Worten: {\em Was das erste Funktional $J_1$ angewandt auf die Einflussfunktion $G_2$ liefert, ist derselbe Wert, den das zweite Funktional $J_2$ angewandt auf die Einflussfunktion $G_1$ liefert\/}.

Die beiden Biegelinien $w_1$ bzw. $w_2$ in (\ref{Eq126}), Einzelkraft $P = 1$ in $x_1$ bzw. $x_2$, sind ja gerade die Einflussfunktionen $G_1$ und $G_2$ f\"{u}r die beiden Funktionale,
\begin{align}
J_1(w) = w(x_1) = \int_0^{\,l} G_1(y,x_1)\,p\,dy \qquad J_2(w) = w(x_2)= \int_0^{\,l} G_2(y,x_2)\,p\,dy
\end{align}
und so kommen wir auf den Ausdruck (\ref{Eq127}), der im \"{u}brigen f\"{u}r alle Paare von linearen Funktionalen und ihre Kerne gilt. Der {\em Satz von Maxwell\/} ist nicht auf Durchbiegungen begrenzt.

In einem gewissen Sinne ist das Resultat $w_1(x_2) = w_2(x_1)$ dasselbe, wie die Feststellung, dass die Entfernung von einem Punkt $A$ zu einem Punkt $B$ genauso gro{\ss} ist, wie die Entfernung von $B$ nach $A$\footnote{Diese Bemerkung ist nicht so trivial, wie sie klingt.}.

Glg. (\ref{Eq127}) ist die Grundgleichung. Sie ist der Satz von Betti auf den Punkt gebracht.

Um diese Gleichung in Aktion zu sehen, betrachten wir eine quadratische  Scheibe auf der zwei Punkte $\vek x_a$ und $\vek x_b$ markiert sind, s. Abb. \ref{UE354}. Im Punkt $\vek x_a$ messen wir die horizontale Verschiebung eines Verschiebungsfeldes $\vek u$
\begin{align}
J_1(\vek u) = u_x(\vek x_a)
\end{align}
und im Punkt $\vek x_b$ messen wir die Spannung $\sigma_{xx}$
\begin{align}
J_2(\vek u) = \sigma_{xx}(\vek u) (\vek x_b)
\end{align}
dieses Feldes.

Die Einflussfunktion f\"{u}r das Funktional $J_1$ ist das Verschiebungsfeld $\vek G_1$, das von einer horizontalen Einzelkraft $P = 1$ erzeugt wird, und die Einflussfunktion $\vek G_2$ f\"{u}r $J_2$ wird von einer horizontalen Versetzung im Punkt $\vek x_b$ erzeugt.

Gem\"{a}{\ss} Maxwell (= Satz von Betti) gilt
\begin{align}\label{Eq172}
J_1(\vek G_2) = J_2(\vek G_1)
\end{align}
oder\\

{\em Die Verschiebung im Punkt $\vek x_a$ verursacht durch die Versetzung im Punkt $\vek x_b$ ist gleich der Spannung im Punkt $\vek x_b$ infolge der Einzelkraft im Punkt $\vek x_a$.\/}\\

%---------------------------------------------------------------------------------
\begin{figure}
\centering
{\includegraphics[width=0.7\textwidth]{\Fpath/U370}}
  \caption{FE-Einflussfunktion f\"{u}r die vertikale Spannung $\sigma_{yy}$ in der Elementmitte, \"{a}quivalente Knotenkr\"{a}fte und zugeh\"{o}rige Lagerkr\"{a}fte aus der Spreizung des Aufpunkts.}
  \label{U370}\label{Korrektur19}
\end{figure}
%---------------------------------------------------------------------------------
In Abb. \ref{U317} ist das erste Funktional
\begin{align}
J_1(w) = M(x_c)
\end{align}
das Moment einer Biegelinie $w$ im Punkt $x_c$ und das zweite Funktional
\begin{align}
J_2(w) = B
\end{align}
ist die Lagerkraft, die dieselbe Biegelinie $w$ im Lager $B$ hat.

Zu $G_1$ (= Einflussfunktion f\"{u}r $J_1$) geh\"{o}rt die Lagerkraft $J_2(G_1) = -8\, 112$ kN$\times 1$\,m und zur Einflussfunktion $G_2$ (= Einflussfunktion f\"{u}r $B$) geh\"{o}rt ein Moment $J_1(G_2) = -8\,112$ kNm und beide Werte sind zahlenm\"{a}{\ss}ig gleich\footnote{Das Ergebnis von Einflussfunktionen hat immer die Dimension {\em Arbeit\/}.}
\begin{align}
J_1(G_2) = J_2(G_1)\,.
\end{align}
{\em Betti extended\/}, s. Kapitel 4, garantiert \"{u}brigens, dass dies auch f\"{u}r die FE-L\"{o}sungen gilt, d.h. in der Gleichung
\begin{align}
J_1(G_2) = \int_0^{\,l} \delta_1\, G_2\,dy = \int_0^{\,l} \delta_2\,G_1\,dy  = J_2(G_1)
\end{align}
darf man $G_1$ und $G_2$ durch die FE-L\"{o}sungen ersetzen, $J_1(G_2^h) = J_2(G_1^h)$.\\

\begin{remark}
Den klassischen Maxwell, $\delta_{12} = \delta_{21}$ unter Einzelkr\"{a}ften s. Abb. \ref{U128} S. \pageref{U128}, kann man auch aus der Mohrschen Arbeitsgleichung herleiten, wenn man zur Berechnung der Durchbiegungen $\delta_{ij}$ schwache Einflussfunktionen benutzt, denn dann ist die Symmetrie im Ergebnis eine einfache Konsequenz der Symmetrie der Wechselwirkungsenergie
\begin{align}
\delta_{12} = a(w_1,w_2) = a(w_2,w_1) = \delta_{21} \,.
\end{align}
Bei \glq h\"{o}heren\grq{} Dirac Deltas, $\delta_3, \delta_4$ (Balken), muss man allerdings \"{u}ber den Satz von Betti gehen, weil es f\"{u}r $M(x)$ und $V(x)$ keine schwachen Einflussfunktionen gibt.
\end{remark}

\vspace{-0.5cm}
{\textcolor{blue}{\subsubsection*{Lagersenkung}}}
Der Satz von Maxwell\index{Lagersenkung} beantwortet auch die Frage, wie man Einflussfunktionen auswertet, wenn sich ein Lager setzt, wie z.B. das mittlere Lager der Wandscheibe in Abb. \ref{U370}. Dargestellt ist die Einflussfunktion f\"{u}r $\sigma_{yy}$ in einem Element. Die Spreizung des Aufpunktes erzeugt in dem Lager eine vertikale Lagerkraft $R_y$ von $119\,259$ kN und damit ergibt sich die Spannung zu
\begin{align}
\sigma_{yy} = -119\,259 \cdot \Delta_y\,,
\end{align}
wenn $\Delta_y$ (in $y$-Richtung positiv) die Lagerbewegung ist. Das Minus haben wir auf S. \pageref{LagerWeg} erkl\"{a}rt.

%---------------------------------------------------------------------------------
\begin{figure}
\centering
{\includegraphics[width=0.8\textwidth]{\Fpath/U191A}}
  \caption{\textbf{ a)} Seil aus $n = 5$ Elementen, \textbf{ b-e)} die Durchbiegungen sind die Spalten der inversen Steifigkeitsmatrix (alle Werte mal $l_e/(5\,H)$), \textbf{ f)} wenn $n$ w\"{a}chst, werden die Spalten von $\vek K^{-1}$ immer \"{a}hnlicher, $\det(\vek K^{-1}) \to 0$, d.h. $\vek K^{-1}$ wird singul\"{a}r}\label{Korrektur6}
  \label{U191}
\end{figure}
%---------------------------------------------------------------------------------

%%%%%%%%%%%%%%%%%%%%%%%%%%%%%%%%%%%%%%%%%%%%%%%%%%%%%%%%%%%%%%%%%%%%%%%%%%%%%%%%%%%%%%%%%%%%%%%%%%%
{\textcolor{sectionTitleBlue}{\section{Die inverse Steifigkeitsmatrix}}}\index{inverse Steifigkeitsmatrix}
Die FE-Einflussfunktion f\"{u}r die Verschiebung $u(x)$ in einem Knoten $x_k$ hat die Form
\beq
G_h(y,x_k) = \sum_i g_i(x_k)\,\Np_i(y)\,.
\eeq
Der Vektor $\vek g = \{g_{1},g_{2}, \ldots, g_{n}\}^T $ ist die L\"{o}sung des $n\times n$ Systems
\beq
\vek K\,\vek g = \vek e_k \qquad \text{(Einheitsvektor $\vek
e_k$)}\,,
\eeq
was bedeutet, dass die Spalten $\vek g_k$ der inversen Steifigkeitsmatrix $\vek K^{-1}$
\beq
\vek g = \vek K^{-1} \vek e_k = \vek g_k
\eeq
die Knotenverschiebungen sind, die zu den $n$ Einflussfunktionen $G_h(y, x_k)$ der $n$ Knoten $x_k$ geh\"{o}ren
\beq
 G_h(y,x_k) = \sum_i g_{k @i}\,\Np_i(y)  = \vek g_k^T\,\vek \Phi (y)\,,
\eeq
mit $\vek \Phi(y) = \{\Np_1(y), \Np_2(y), \ldots, \Np_n(y)\}^T. $

Das erkl\"{a}rt, warum die Inverse einer tri-diagonalen Matrix voll besetzt ist.
Schon eine einzelne Punktlast $P = 1$ zwingt das Seil zu einer Ausgleichsbewegung, die alle Knoten erfasst.  Die Inverse einer Differenzenmatrix wie $\vek K$ (man denke an ein Seil ($\ldots 0\,\,-1\,\,\,2\,\,-1\,\,\,0\,\,\ldots$) ist also eine Summenmatrix.\\

\hspace*{-12pt}\colorbox{highlightBlue}{\parbox{0.98\textwidth}{Eine Steifigkeitsmatrix $\vek K$ \glq differenziert\grq{} und ihre Inverse $\vek K^{-1}$ \glq integriert\grq{}. Die Inverse ist {\em immer\/} voll besetzt und sie ist symmetrisch (wegen Maxwell).
}}\\

%%%%%%%%%%%%%%%%%%%%%%%%%%%%%%%%%%%%%%%%%%%%%%%%%%%%%%%%%%%%%%%%%%%%%%%%%%%%%%%%%%%%%%%%%%%%%%%%%%%
{\textcolor{sectionTitleBlue}{\section{Beispiele}}}
Das Seil in Abb. \ref{U191}\,a, das mit einer Kraft $H$ vorgespannt wird, ist in f\"{u}nf lineare Elemente unterteilt. Die Steifigkeitsmatrix
$\vek K$
\beq
    \vek K = \frac{H}{l_e}
    \left[ \barr {r @{\hspace{4mm}}r @{\hspace{4mm}}r
@{\hspace{4mm}}r}
      2 & -1 & 0 & 0  \\
      -1 & 2 & -1 & 0 \\
      0 & -1 & 2 & -1 \\
      0 & 0 & -1 & 2  \\
    \earr \right]\,
\eeq
ist tri-diagonal, aber die Inverse
\beq
   \vek K^{-1} = \frac{l_e}{5\,H}
    \left[ \barr {r @{\hspace{4mm}}r @{\hspace{4mm}}r
@{\hspace{4mm}}r}
      4 & \phantom{-}3 & \phantom{-}2 & \phantom{-}1  \\
      3 & 6 & 4 & 2 \\
      2 & 4 & 6 & 3 \\
      1 & 2 & 3 & 4  \\
    \earr \right]\,
\eeq
ist dagegen voll besetzt. Die Spalte $\vek g_k$ der Inversen, siehe die Bilder \ref{U191}\,b---f, sind die Durchbiegungen der Knoten, wenn im Knoten $x_k$ eine Einzelkraft $P = 1$ steht.

Die Zeilensumme der Inversen verr\"{a}t im \"{u}brigen, wie oben gezeigt, welche Knoten sich am st\"{a}rksten verformen, denn
\begin{align}
w_h(x_i) &= \int_0^{\,l} G_h(y,x_i)\,p(y)\,dy = \sum_{j = 1}^4 g_j(x_i) \int_0^{\,l} \Np_j(y)\,p(y)\,dy = \sum_{j = 1}^4 k_{ij}^{(-1)} f_j\nn \\
&\simeq  \sum_{j = 1}^4 k_{ij}^{(-1)} \qquad (\text{setze alle $f_j = 1$)}\,.
\end{align}
Das sind hier die Knoten 2 und 3.
%---------------------------------------------------------------------------------
\begin{figure}
\centering
{\includegraphics[width=1.0\textwidth]{\Fpath/U25}}
  \caption{\textbf{ a)} Unterteilung eines Balkens in drei Elemente,
  \textbf{ b)} Biegelinie aus $f_1 = 1$ (Spalte 1 von $\vek K^{-1}$),  \textbf{ c)} aus $f_2 = 1$ (Spalte 2 von $\vek K^{-1}$),   \textbf{ d)} aus $f_3 = 1$ (Spalte 3)}
  \label{U25}
\end{figure}
%---------------------------------------------------------------------------------
%---------------------------------------------------------------------------------
\begin{figure}
\centering
{\includegraphics[width=0.9\textwidth]{\Fpath/U360}}
  \caption{Die Knotenverschiebungen dieser drei Einflussfunktionen bilden die Spalten $\#13, \,\#14\,, \#15$ der Inversen $\vek K^{-1}$}
  \label{UE360}
\end{figure}
%---------------------------------------------------------------------------------

Zu dem Balken in Abb. \ref{U25}, es sei $EI = 1$ und $l_e = 1$, geh\"{o}rt die Steifigkeitsmatrix
\begin{align}
\vek K = \left[ \barr {r @{\hspace{4mm}}r @{\hspace{4mm}}r
@{\hspace{4mm}}r @{\hspace{4mm}}r @{\hspace{4mm}}r}
4    & 6    & 2    & 0   &  0     &0\\
     6    & 24    &  0   & -12    & -6     &0\\
     2    &  0    &  8   &   6   &   2     &0\\
     0   & -12    &  6   &  24   &   0    &-6\\
     0   &  -6    &  2   &  0    &  8     &2\\
     0   &   0    &  0   &  -6    &  2   &  4
    \earr \right]
\end{align}
und die Spalten der Inversen
\begin{align}
\vek K^{-1} = \left[ \barr {r @{\hspace{4mm}}r @{\hspace{4mm}}r
@{\hspace{4mm}}r @{\hspace{4mm}}r @{\hspace{4mm}}r}
 1.00  &-0.56   & 0.17  & -0.44  & -0.33  & -0.50\\
   -0.56  & 0.44   &-0.22  &  0.39   & 0.28   & 0.44\\
    0.17   &-0.22   & 0.33  & -0.28   &-0.17  & -0.33\\
   -0.44  & 0.39  & -0.28  &  0.44   & 0.22  &  0.56\\
   -0.33   & 0.28  & -0.17  &  0.22   & 0.33  &  0.17\\
   -0.50   & 0.44  & -0.33  &  0.56  &  0.17  &  1.00
    \earr \right]
    \end{align}
sind die Knotenbewegungen in den sechs Grundlastf\"{a}llen $\vek f = \vek e_i$, wenn also jeweils ein Knoten belastet wird, $i = 1,2,\ldots 6$.

Dies gilt f\"{u}r Rahmen beliebiger Gr\"{o}{\ss}e, siehe Abb. \ref{UE360}. Die einzelnen Spalten der Inversen $\vek K^{-1}$ sind immer die Knotenwerte der Einflussfunktionen f\"{u}r die Knotenverschiebungen.

Eine Einzelkraft $f_i = 1$ in Richtung eines Freiheitsgrades $u_i$ lenkt die Knoten aus. Die Energie, die dabei erzeugt wird, steht auf der Diagonalen $\{g_{ii}\}$ von $\vek K^{-1}$, denn
\begin{align}
a(\vek g_i, \vek g_i) = \vek g_i^T\,\vek K\,\vek g_i = \vek g_i^T\,\vek K\,\vek K^{-1}\,\vek e_i = g_{ii}
\end{align}
Je gr\"{o}{\ss}er $g_{ii}$ ist, desto weiter ist der Ausschlag, desto weicher ist das Tragwerk in Richtung von $u_i$, denn $2 \cdot A_a = 1 \cdot u_i = 2 \cdot A_i = g_{ii}$. \"{U}berraschend ist das nicht, denn $g_{ii} = f_{ii}$ ist ja das Diagonalelement der Flexibilit\"{a}tsmatrix $\vek F = \vek K^{-1}$.

%%%%%%%%%%%%%%%%%%%%%%%%%%%%%%%%%%%%%%%%%%%%%%%%%%%%%%%%%%%%%%%%%%%%%%%%%%%%%%%%%%%%%%%%%%%%%%%%%%%
{\textcolor{sectionTitleBlue}{\section{Allgemeine Form einer FE-Einflussfunktion}}}
Das folgende Theorem fasst die Ergebnisse zusammen.\\
%----------------------------------------------------------------------------
\begin{figure}
\centering
{\includegraphics[width=0.95\textwidth]{\Fpath/U83}}
  \caption{Die Biegelinie unter der Streckenlast $p$ ist die Einh\"{u}llende der Seilecke aus den Einzelkr\"{a}ften $dP$}
  \label{U83}
\end{figure}%%
%----------------------------------------------------------------------------

\begin{theorem}[Allgemeine Form einer FE-Einflussfunktion]
Es sei $\vek K$ die Steifigkeitsmatrix des Tragwerks.\\
 (i) Die FE-Einflussfunktion f\"{u}r $u_h(x)$, der Wert der FE-L\"{o}sung in einem Punkt $x$, ist
\beq
G_h(y,x) = \vek \phi(y)^T\, \vek K^{-1}\,\vek \phi(x)
\eeq
wobei
\begin{align}
\vek \phi(x) = \{\Np_1(x), \Np_2(x), \ldots, \Np_n(x)\}^T
\end{align}
die Liste mit den Werten der Ansatzfunktionen in dem Punkt $x$ ist, und $\vek \phi(y)$ ist dieselbe Liste, nur dass $x$ durch $y$ (die Integrationsvariable) ersetzt wird. \\
(ii) Die Einflussfunktion f\"{u}r ein lineares Punktfunktional $J$ ist
\beq
G_h(y,x) = \vek \phi(y)^T\, \vek K^{-1}\,\vek j(x)
\eeq
wobei der Vektor
\beq
\vek j(x) =  \{J(\Np_1), J(\Np_2), J(\Np_3), \ldots, J(\Np_n) \}^T
\eeq
die Liste der Werte $J(\Np_i)(x)$ ist.
\end{theorem}
Die Biegelinie eines Seils ist die Einh\"{u}llende der unendlich vielen Einflussfunktionen, die jede f\"{u}r sich den Einfluss eines infinitesimalen Teils $p(y)\,dy$ der Belastung auf die Durchbiegung $w(x)$ beschreiben, s. Abb. \ref{U83},
\beq
w(x) = \int_0^{\,l} G(y,x)\,p(y)\,dy\,.
\eeq
Im Unterschied hierzu ist die FE-L\"{o}sung $w_h(x) = \vek w^T\,\vek \phi(x) $ darstellbar als eine Summe von {\em endlich vielen\/} Einflussfunktionen, die einzeln mit den \"{a}quivalenten Knotenkr\"{a}ften $f_i$ der Knoten $x_i$ gewichtet werden
\beq
w_h(x) =  f_1\,G_h(x_1, x) + f_2\,G_h(x_2, x) + \ldots + f_n\,G_h(x_n, x)\,,
\eeq
denn der Vektor $\vek w$ der Knotenwerte ist
\begin{align}
\vek w &= \vek K^{-1} \vek f =  \vek K^{-1} (f_1\,\vek e_1 + f_2\,\vek e_2 + \ldots + f_n\,\vek e_n) \nn \\
&= f_1\,\vek g_1 + f_2\,\vek g_2 + \ldots + f_n\,\vek g_n = \sum_{k = 1}^n\,f_k \cdot \left [\barr{c} \phantom{.} \\  \vek g_k \\ \phantom{.} \earr \right ]
\end{align}
und die Spalte  $\vek g_k$ von $\vek K^{-1}$ entspricht $G_h(x_k,x)$.

%---------------------------------------------------------------------------------
\begin{figure}
\centering
\if \bild 2 \sidecaption \fi
\includegraphics[width=1.0\textwidth]{\Fpath/UE355}
\caption{Einflussfunktionen eine Stabes \textbf{ a)} f\"{u}r $u(x_i)$, \textbf{ b)} f\"{u}r $u(x_{i+1})$ und Reaktion des Stabes auf Punktlasten $\mp EA/l_e$ im \textbf{ c)} Knoten $x_i$ und \textbf{ d)} im Knoten $x_{i+1}$}
\label{UE355}%
\end{figure}%
%---------------------------------------------------------------------------------

%%%%%%%%%%%%%%%%%%%%%%%%%%%%%%%%%%%%%%%%%%%%%%%%%%%%%%%%%%%%%%%%%%%%%%%%%%%%%%%%%%%%%%%%%%%%%%%%%%%
{\textcolor{sectionTitleBlue}{\section{Finite Differenzen und Einflussfunktionen}}}\index{finite Differenzen}
Im Folgenden konzentrieren wir uns der Einfachheit halber auf einen Stab, aber die Ergebnisse lassen sich nat\"{u}rlich verallgemeinern.

Das erste, was wir notieren wollen, ist, dass die Knotenkr\"{a}fte $f_i$, die die Einflussfunktionen f\"{u}r die Verschiebung im Mittelpunkt eines Elementes erzeugen, immer gleich gro{\ss} sind, $f_i = 1/2$, unabh\"{a}ngig davon, wie lang, $l_e$, das Element ist. Bei der Einflussfunktionen f\"{u}r die Normalkraft $N = EA\,u'$ ist das (in \"{U}bereinstimmung mit dem umgekehrten Gummibandeffekt, S. \pageref{rubberband}) anders
\begin{align}
f_i = -\frac{EA}{l_e} \qquad f_{i+1} = \frac{EA}{l_e}\,.
\end{align}
Die Ansatzfunktion $\Np_i(x)$ (eine H\"{u}tchenfunktion) hat eine negative Steigung $-1/l_e$ im Punkt $x$ und $\Np_{i+1}$ hat eine positive Steigung $1/l_e$ im Punkt $x$.

Angenommen eine Kraft $P$ wirkt im Punkt $y$ des Stabes. Um die Normalkraft im Punkt $x$ zu berechnen, werten wir die FE-Einflussfunktion $G_1^h(y,x)$ f\"{u}r $N_h(x)$ im Punkt $y$ aus
\begin{align}
N_h(x) &= G_1^h(y,x) \cdot P = [G_0^h(y,x_i) \cdot (-\frac{EA}{l_e}) + G_0^h(y,x_{i+1}) \cdot (+\frac{EA}{l_e})] \cdot P\nn \\
 &= \frac{EA}{l_e}\,(G_0^h(y,x_{i+1}) - G_0^h(y,x_i)) \cdot P\,.
\end{align}
Und dies ist offenbar eine Finite-Differenzen-N\"{a}herung der wahren Gleichung
\begin{align}
N(x) = G_1(y,x)\cdot P = EA\,\frac{d}{dx}\,G_0(y,x) \cdot P\,.
\end{align}
\hspace*{0pt}\colorbox{highlightBlue}{\parbox{0.98\textwidth}{Bei der Verwendung von linearen Elementen n\"{a}hern wir die Einflussfunktionen f\"{u}r die Schnittgr\"{o}{\ss}en durch finite Differenzen der Einflussfunktionen der Knotenverschiebungen $G_0^h$ an.}}\\

Die $G_0^h$ der Knotenverschiebungen $u_i$ sind sozusagen die Universalschl\"{u}ssel. Es sei daran erinnert, dass ja die Knotenvektoren der Einflussfunktionen der $u_i$ die Spalten der Inversen $\vek K^{-1}$ bilden.

%---------------------------------------------------------------------------------
\begin{figure}
\centering
\if \bild 2 \sidecaption \fi
\includegraphics[width=1.0\textwidth]{\Fpath/UE356}
\caption{Die Knotenkr\"{a}fte $j_i$, die die FE-Einflussfunktion f\"{u}r $u(\vek x)$ auf einem bilinearen Netz (Poisson Gleichung) erzeugen}
\label{UE356}%
\end{figure}%
%---------------------------------------------------------------------------------


Einflussfunktionen f\"{u}r Verschiebungen interpolieren die Knotenwerte
\begin{align}
u(x) = \frac{1}{2}\,(G_0(y,x_i) + G_0(y,x_{i+1})) \cdot P\,.
\end{align}
Dasselbe gilt f\"{u}r die Einflussfunktionen selbst, s. Abb. \ref{UE356}. Die Knotenwerte $g_i(\vek x)$ der Einflussfunktion
\begin{align}
G_h(\vek y,\vek x) = \sum_i g_i(\vek x)\,\Np_i(\vek y)
\end{align}
f\"{u}r die Durchbiegung $u_h(\vek x)$ der Membran im Aufpunkt $\vek x$ ist der Vektor
\begin{align}
\vek g(\vek x) = j_a(\vek x) \cdot \left [\barr{c} \phantom{.} \\  \vek g_a \\ \phantom{.} \earr \right ] + j_b(\vek x) \cdot \left [\barr{c} \phantom{.} \\  \vek g_b \\ \phantom{.} \earr \right ]+  j_c(\vek x) \cdot \left [\barr{c} \phantom{.} \\  \vek g_c \\ \phantom{.} \earr \right ] + j_d(\vek x) \cdot \left [\barr{c} \phantom{.} \\  \vek g_d \\ \phantom{.} \earr \right ]
\end{align}
wobei die Vektoren $\vek g_a, \vek g_b, \vek g_c, \vek g_d$ die entsprechenden Spalten der Inversen $\vek K^{-1}$ sind und die Gewichte sind die Werte der vier {\em shape functions\/} $\psi_i(\vek x)$ in (\ref{Eq179}) im Aufpunkt $\vek x = (x_1,x_2)$
\begin{align}
j_a = \psi_1(\vek x) \qquad j_b = \psi_2(\vek x)   \qquad j_c = \psi_3(\vek x) \qquad j_d = \psi_4(\vek x) \,.
\end{align}
%---------------------------------------------------------------------------------
\begin{figure}
\centering
\if \bild 2 \sidecaption \fi
\includegraphics[width=0.8\textwidth]{\Fpath/U35}
\caption{FE-Einflussfunktion f\"{u}r $\sigma_{yy}$ in zwei benachbarten Punkten}
\label{U35}%
\end{figure}%
%---------------------------------------------------------------------------------

\vspace{-1cm}
%%%%%%%%%%%%%%%%%%%%%%%%%%%%%%%%%%%%%%%%%%%%%%%%%%%%%%%%%%%%%%%%%%%%%%%%%%%%%%%%%%%%%%%%%%%%%%%%%%%
{\textcolor{sectionTitleBlue}{\section{Die Natur macht keine Spr\"{u}nge, aber die finiten Elemente}}}\label{Jumps}
Wenn man die Trennlinie zwischen zwei Elementen \"{u}berschreitet, dann springen die Spannungen. Das bedeutet aber doch, dass auch die Einflussfunktionen springen m\"{u}ssen. Wie kommt das?

Den Grund sieht man in Abb. \ref{U35}. Die \"{a}quivalenten Knotenkr\"{a}fte, die die Einflussfunktion f\"{u}r $\sigma_{yy}$ in dem oberen Punkt $\vek x_1$ generieren, sind die Spannungen $\sigma_{yy}$ der Knotenverschiebungen $\vek \Np_i$ in diesem Punkt. Weil nur die Ansatzfunktionen des Elements, in dem $\vek x_1$ liegt, Spannungen in dem Punkt $\vek x_1$ erzeugen, werden nur die vier Knoten des Elementes belastet. Wenn der Punkt in das n\"{a}chste Element wandert, $\vek x_1 \to \vek x_2$, dann verschwinden diese Knotenkr\"{a}fte und tauchen an den vier Knoten des Nachbarelementes auf. Dieser pl\"{o}tzliche Sprung in den belasteten Knoten ist der Grund, warum die Spannungen springen: {\em Die Einflussfunktionen springen\/}.
%---------------------------------------------------------------------------------
\begin{figure}
\centering
\if \bild 2 \sidecaption \fi
\includegraphics[width=.75\textwidth]{\Fpath/U458}
\caption{Einflussfunktionen an einer Platte und einem Stab \textbf{ a)} Lage der Knotenkr\"{a}fte f\"{u}r die Durchbiegung der Platte in einem Knoten und in Elementmitte -- das punktgenau gelingt in einem Knoten am besten, \textbf{ b)} Einflussfunktion f\"{u}r eine Knotenverschiebung, \textbf{ c)} f\"{u}r eine Verschiebung in Elementmitte, \textbf{ d)} f\"{u}r die Normalkraft in Elementmitte und \textbf{ e)} f\"{u}r den Mittelwert der Normalkraft in einem Knoten, Mittel aus links und rechts}\label{U458}%
\end{figure}%
%---------------------------------------------------------------------------------


Verschiebungen springen beim \"{U}berschreiten der Elementlinie nicht, weil sich die Einflussfunktionen (im unmittelbarer N\"{a}he der Linie) nicht \"{a}ndern. W\"{a}re es anders, dann w\"{a}ren die Elemente nicht konform -- dann w\"{a}ren die {\em shape functions\/} unstetig.

Technisch ist es so, dass bei einer Verschiebungs-Einflussfunktion die beiden Knotenkr\"{a}fte, die in Abb. \ref{U35} springen, gleich bleiben und die anderen $f_i = \vek \Np_i(\vek x \pm 1 \text{mm}) = 0$ sind null, wenn $\vek x$ auf der Kante liegt.

An Hand der Einflussfunktionen, s. Abb. \ref{U458}, erkennt man im \"{u}brigen, dass Verschiebungen  in den Knoten am genauesten sind (bei 1-D Problemen sind sie dort sogar exakt, wenn $EA$ oder $EI$ konstant sind) und Spannungen sind es in der Mitte der Elemente. Wenn man Spannungen -- gezwungenerma{\ss}en -- an Knoten mittelt\index{Mittelung der Spannungen}, dann ist das so, als ob man bei der Berechnung der Einflussfunktionen f\"{u}r die Spannungen die Elementgr\"{o}{\ss}e in der Umgebung des Knotens verdoppelt h\"{a}tte.
%---------------------------------------------------------------------------------
\begin{figure}
\centering
\if \bild 2 \sidecaption \fi
\includegraphics[width=.75\textwidth]{\Fpath/U288}
\caption{Auf dem Weg vom Aufpunkt zum Fu{\ss}punkt der Einzelkraft m\"{u}ssen alle Steifig\-keiten richtig modelliert werden, denn nur dann werden die Einflusskoeffizienten (die Fortleitungszahlen) richtig erfasst}
\label{U288}%
\end{figure}%
%---------------------------------------------------------------------------------

Hierhin geh\"{o}rt auch das Thema {\em Gausspunkte\/}\index{Gausspunkte}, also die Beobachtung, dass in den Integrationspunkten die Ergebnisse genauer sind, als in den \"{u}brigen Punkten. Der Grund ist, dass der Fehler einer FE-L\"{o}sung eine \"{a}hnliche Verteilung aufweist, wie die partikul\"{a}re L\"{o}sung am allseits eingespannten Element -- bei eindimensionalen Problemen stimmt das sogar genau -- und die Nullstellen der Schnittkr\"{a}fte dieser partikul\"{a}ren L\"{o}sungen genau in den Gausspunkten liegen, \cite{Ha6}.
%----------------------------------------------------------
\begin{figure}[tbp]
\centering
\if \bild 2 \sidecaption[t] \fi
\includegraphics[width=0.99\textwidth]{\Fpath/U441}
\caption{Hochbaudecke \textbf{ a)} Unterkonstruktion \textbf{ b)} Biegemomente $m_{xx}$} \label{U441}
\end{figure}%%
%----------------------------------------------------------

Auch das ist ein Ergebnis, das nur auf partieller Integration beruht. Beim Balken ist die Abweichung in den Momenten zwischen der FE-L\"{o}sung und der exakten L\"{o}sung gleich dem Momentenverlauf $M_p(x)$ am eingespannten Balken
\begin{align}
M(x) - M_h(x) = M_p(x)\,.
\end{align}
Wegen der Einspannung, $w_p'(0) = w_p'(l) = 0$,  ist das Integral des Moments $M_p$ null, s. (\ref{Eq33}),
\begin{align}
\int_0^{\,l_e} M_p(x)\,dx = 0\,.
\end{align}
Nun ist $M_p(x)$ ein Polynom zweiten Grades, wenn wir annehmen, dass $p$ konstant ist, und das muss sich mit einer Gauss-Quadratur exakt berechnen lassen\footnote{$n = 2$ Punkte k\"{o}nnen Polynome bis zur Ordnung $2\,n - 1 = 3$ exakt integrieren.} und das geht nur so, dass das Moment $M_p$ in den beiden Integrationspunkten null ist. Das ist ein \"{u}berraschendes Resultat, aber es l\"{a}sst sich leicht verifizieren. (Kubische Momente, $p$ = linear, sind antimetrisch in den Gausspunkten, aber nicht null).

Bei Scheiben und Platten gilt das unter Umst\"{a}nden nur noch n\"{a}herungsweise, aber immer noch hinreichend deutlich, was den Gausspunkten ihren guten Ruf eingebracht hat.

%%%%%%%%%%%%%%%%%%%%%%%%%%%%%%%%%%%%%%%%%%%%%%%%%%%%%%%%%%%%%%%%%%%%%%%%%%%%%%%%%%%%%%%%%%%%%%%%%%%
\textcolor{sectionTitleBlue}{\section{Ein merkw\"{u}rdiges Ergebnis}}\label{Dimensionsbetrachtung}
An diese Stelle passt vielleicht auch die folgende Beobachtung: Die Platte in Bild \ref{U441} wurde mit finiten Elementen berechnet und im Nachlauf sollen nun die Biegemomente $m_{xx}$ in den Elementen ermittelt werden. Wir nehmen einmal an, dass sie elementweise linear sind, $m_{xx} = a + b\,x + c\,y$.

In jedem Element kann man den Momentenverlauf theoretisch mit der  Momenten-Einflussfunktion ermitteln
\begin{align}
m_{xx} = a + b\,x + c\,y = \sum_e \int_{\Omega_e} G_2(\vek y,\vek x) p_e(\vek y)\,d\Omega_{\vek y} + \ldots\,.
\end{align}
Die rechte Seite dieses Ausdrucks wird sich allerdings \"{u}ber viele, viele Seiten Papier erstrecken, weil wir \"{u}ber jedes der (gesch\"{a}tzt) 1000 Elemente $\Omega_e$ zu integrieren haben und danach auch noch \"{u}ber alle Kanten des Netzes, (Linienlasten und -momente aus $p_h$). Das merkw\"{u}rdige ist nun, dass all diese vielen Beitr\"{a}ge sich in der Summe auf ein lineares Polynom reduzieren, das durch drei Zahlen $a, b$  und $c$ charakterisiert ist.


%%%%%%%%%%%%%%%%%%%%%%%%%%%%%%%%%%%%%%%%%%%%%%%%%%%%%%%%%%%%%%%%%%%%%%%%%%%%%%%%%%%%%%%%%%%%%%%%%%%
{\textcolor{sectionTitleBlue}{\section{Der Weg vom Aufpunkt zur Belastung}}}\index{vom Aufpunkt zur Belastung}
F\"{u}r eine korrekte Kommunikation zwischen dem Aufpunkt und der Belas\-tung ist es wichtig, dass die Steifigkeiten auf dem Weg vom Aufpunkt zur Belastung richtig erfasst werden, weil davon sehr viel abh\"{a}ngt, s. Abb. \ref{U288}.
%---------------------------------------------------------------------------------
\begin{figure}
\centering
\if \bild 2 \sidecaption \fi
\includegraphics[width=.95\textwidth]{\Fpath/U174A}
\caption{Durchlauftr\"{a}ger mit Kragarmlast, \textbf{ a)} Einflusslinie f\"{u}r das Einspannmoment mit $2 \cdot EI$ im mittleren Feld  und \textbf{ b)} bei konstantem $EI$}
\label{U174}%
\end{figure}%
%---------------------------------------------------------------------------------

Die Abb. \ref{U174} demonstriert dies am Beispiel der Einflussfunktion f\"{u}r das Einspannmoment eines Durchlauftr\"{a}gers mit Kragarm. Eine Verdopplung von $EI$ im mittleren Feld f\"{u}hrt zu einer sp\"{u}rbaren Senkung des Einspannmoments im Verh\"{a}ltnis von 2:3 , wie man an den unterschiedlichen Durchbiegungen des Kragarms ablesen kann.
%---------------------------------------------------------------------------------
\begin{figure}
\centering
\if \bild 2 \sidecaption \fi
\includegraphics[width=.95\textwidth]{\Fpath/U176}
\caption{Eine Spreizung der St\"{u}tze erzeugt die Einflussfunktion f\"{u}r die St\"{u}tzenkraft. Die korrekte Propagierung \"{u}ber das Tragwerk h\"{a}ngt von der korrekten Modellierung der Steifigkeiten ab, \cite{Sof}}
\label{U176}%
\end{figure}%
%---------------------------------------------------------------------------------

Es ist anschaulich klar, dass diese Kommunikation um so \glq wackliger\grq{} wird, je weiter der Aufpunkt und die Last auseinander liegen, weil mit wachsender Entfernung immer mehr Bauteile zu passieren sind und sich so die Fehler aus nur n\"{a}herungsweise richtig erfassten Steifigkeiten, $EA \pm \Delta EA$, $EI \pm \Delta EI$, oder Einspanngraden $k_\Np \pm \Delta k_\Np$, kumulieren k\"{o}nnen, s. Abb. \ref{U176}. Zum Gl\"{u}ck ist es aber so, dass in der Regel mit der Entfernung die Einflusskoeffizienten abnehmen und damit auch die Auswirkungen von m\"{o}glichen Fehlern.

%%%%%%%%%%%%%%%%%%%%%%%%%%%%%%%%%%%%%%%%%%%%%%%%%%%%%%%%%%%%%%%%%%%%%%%%%%%%%%%%%%%%%%%%%%%%%%%%%%%
{\textcolor{sectionTitleBlue}{\section{Die Spalten von $\vek K$ und $\vek K^{-1}$}}

Nachdem wir jetzt auch Einflussfunktionen mit finiten Elementen berechnen k\"{o}nnen, wollen wir hier die wesentlichen Eigenschaften der Spalten der beiden Matrizen $\vek K$ und $\vek K^{-1}$ im Vergleich zusammenstellen.\index{Spalten von $\vek K$}\index{Spalten von $\vek K^{-1}$}

{\textcolor{sectionTitleBlue}{\subsubsection*{Die Spalten von $\vek K$}}}
Die Matrix $\vek K = [\vek f_1, \vek f_2, \ldots, \vek f_n]$ bildet den Vektor $\vek u$ auf einen Vektor $\vek f_h$ ab
\begin{align}
\vek K \vek u = u_1 \cdot \vek f_i + u_2 \cdot \vek f_2 + \ldots + u_n \cdot \vek f_n = \vek f_h\,.
\end{align}
Warum wir den Vektor $\vek f_h$ nennen, wird gleich deutlich.

%---------------------------------------------------------------------------------
\begin{figure}
\centering
\if \bild 2 \sidecaption \fi
\includegraphics[width=.90\textwidth]{\Fpath/U455}
\caption{Der Mittenknoten hat den FG $u_y = u_{7}$ \textbf{ a)} Spalte 7 der Steifigkeitsmatrix $\vek K$ in vektorieller Darstellung (je zwei Eintr\"{a}ge ($x$-, $y$-Komponenten)) in Spalte 7 bilden eine Kraft), die Kraft $k_{7, 7}$ in der Mitte lenkt den Knoten um $u_{7} = 1$ aus und die umliegenden Kr\"{a}fte $k_{i,7}$ stoppen die Bewegung an den n\"{a}chsten Knoten ab \textbf{ b)} Spalte 7 der Flexibilit\"{a}tsmatrix $\vek F = \vek K^{-1}$, die Eintr\"{a}ge in der Spalte $\vek f_{7}$ sind die Verschiebungen $u_i$ der Knoten verursacht durch die vertikale Kraft $f_{7} = 1$. Au{\ss}erhalb der Umgebung des belasteten Knotens in Bild a herrscht \glq Windstille\grq{}, die Matrix $\vek K$ ist schwach besetzt, w\"{a}hrend in Bild b die ganze Scheibe \glq durchweht wird\grq{}, die ganze Scheibe sp\"{u}rt die Punktlast, $\vek F$ ist voll besetzt }
\label{U447}%
\end{figure}%
%---------------------------------------------------------------------------------
Die Spalte $\vek f_i$ ist der Vektor der \"{a}quivalenten Knotenkr\"{a}fte, der zur Einheitsverformung $\Np_i$ geh\"{o}rt und die gewichtete ($u_i$) Summe  $\vek f_h$ sind daher die \"{a}quivalenten Knotenkr\"{a}fte, die zur FE-L\"{o}sung $\sum_i u_i\,\Np_i$ geh\"{o}ren, s. Abb. \ref{U447} und \ref{U448}.

Um die Bewegung $\Np_i$ zu erzeugen, sind Kr\"{a}fte n\"{o}tig, die wir den Lastfall $p_i$ nennen und die wir uns hier, der Einfachheit halber als eine Streckenlast $p_i$ vorstellen. Die zu dem Lastfall $p_i$ geh\"{o}rigen \"{a}quivalenten Knotenkr\"{a}fte sind also komponentenweise ($j$)
\begin{align}
f_{ij}    = \int_0^{\,l} p_i\,\Np_j\,dx \qquad (\delta A_a)\,.
\end{align}
Wegen der ersten Greenschen Identit\"{a}t, $\text{\normalfont\calligra G\,\,}(\Np_i,\Np_j) = \delta A_a - \delta A_i = 0$, sind diese \"{a}u{\ss}eren Arbeiten gleich der inneren Arbeit, also der Wechselwirkungsenergie $f_{ij} = a(\Np_i,\Np_j) = k_{ij}$, und so sind die Spalten von $\vek K$ gerade die \"{a}quivalenten Knotenkr\"{a}ften der Kr\"{a}fte $p_i$, die die $\Np_i$ erzeugen.

Die Grundgleichung $\vek K\,\vek u = \vek f$ der finiten Elemente bedeutet also $\vek f_h = \vek f$.  Das ist die \glq Wackel\"{a}quivalenz\grq{} zwischen dem Originallastfall und dem FE-Lastfall.

{\textcolor{sectionTitleBlue}{\subsubsection*{Die Spalten von $\vek K^{-1}$}}}
Die Matrix $\vek K^{-1} = [\vek g_1, \vek g_2, \ldots, \vek g_n]$ bildet den Vektor $\vek f$ der \"{a}quivalenten Knotenkr\"{a}fte auf den Vektor $\vek u$ ab
\begin{align}
\vek K^{-1} \vek f = f_1 \cdot \vek g_1 + f_2 \cdot \vek g_2 + \ldots + f_n \cdot \vek g_n = \vek u\,,
\end{align}
oder komponentenweise
\begin{align}
u_i = \vek g_i^T\,\vek f = \sum_j g_{ij} f_j  = \int_0^{\,l} \sum_j g_{ij}  \,\Np_j(y)\,p\,dy\,,
\end{align}
was belegt, dass
\begin{align}
G_h(y,x_i)= \sum_j\, g_{ij} \,\Np_i(y)
\end{align}
die Einflussfunktion f\"{u}r $u_i$ ist
\begin{align}
u_i = \int_0^{\,l} G_h(y,x_i)\,p(y)\,dy\,.
\end{align}
Die $g_{ij}$ sind die Elemente von $\vek K^{-1}$ und die Vektoren $\vek g_i$ sind die Spalten von $\vek K^{-1}$. Normalerweise w\"{u}rde man die Spalten mit $\vek u_i$ bezeichnen, weil es Verschiebungen sind, aber weil die $g_{ij}$ die Knotenwerte der Einflussfunktionen f\"{u}r die Knotenverschiebungen sind,  schreiben wir $\vek g_i$.
%---------------------------------------------------------------------------------
\begin{figure}
\centering
\if \bild 2 \sidecaption \fi
\includegraphics[width=.99\textwidth]{\Fpath/U448}
\caption{Der Drehfreiheitsgrad $u_{3}$ \textbf{ a)} die Spalte 3 der Steifigkeitsmatrix $\vek K$ enth\"{a}lt die \"{a}quivalenten Knotenkr\"{a}fte zum LF $u_{3} = 1$ und $u_i = 0$ sonst, $\vek K\,\vek e_{3} = \text{Spalte 3}$; das Moment $k_{3,3}$ ist der Ausl\"{o}ser der Bewegung, die von den $k_{i,3}$ an den Nachbarknoten gestoppt wird, \textbf{ b)} Spalte 3 = $\vek F\,\vek e_{3}$ der Flexibilit\"{a}tsmatrix $\vek F = \vek K^{-1}$, die Eintr\"{a}ge sind die Verschiebungen und Verdrehungen der Knoten verursacht durch das Moment $f_{3} = 1$ ($f_i = 0$ sonst). Hier wird klar, warum $\vek K$ schwach und $\vek F$ voll besetzt ist }
\label{U448}%
\end{figure}%
%---------------------------------------------------------------------------------
Wir notieren noch
\begin{align}
\vek e_i = \vek K^{-1}\,\vek f_i \qquad \vek g_i = \vek K^{-1}\,\vek e_i \qquad \vek K\,\vek K\,\vek g_i = \vek f_i\,.
\end{align}

Die Inverse hei{\ss}t wegen der Richtung $\vek K^{-1}\,\vek f = \vek u$ die Flexibilit\"{a}tsmatrix $\vek F = \vek K^{-1}$, aber man k\"{o}nnte sie auch {\em Greensche Matrix\/} $\vek G= \vek K^{-1}$\index{Greensche Matrix}  nennen.

Der Mittenknoten der Scheibe in Abb. \ref{U447} hat in horizontaler Richtung den FG 7, $u_x = u_{7}$\footnote{Realiter wird der FG nat\"{u}rlich eine zwei- oder dreistellige Zahl sein, aber um die Notation einfach zu halten, w\"{a}hlen wir 7 als FG und in Abb. \ref{U448} die Zahl 3}. In Abb. \ref{U447} a sieht man schematisch die Spalte $\vek f_{7}$ der Matrix $\vek K$ und in Bild b die Spalte $\vek g_{7}$ der Matrix $\vek F = \vek K^{-1}$. Die Spalte $\vek f_{7}$ enth\"{a}lt komponentenweise die Kr\"{a}fte in $x$- und $y$-Richtung, die die Auslenkung des FG $u_{7} = 1$ bewirken und zwar so, dass die Verschiebung in den Nachbarknoten zum Stillstand kommt.

Die Spalte $\vek g_{7}$ der Flexibilit\"{a}tsmatrix $\vek F = \vek K^{-1}$ enth\"{a}lt die $x$- und $y$-Verschiebungen, die durch eine horizontale Einzelkraft $f_{7} = 1$ in Richtung von $u_{7}$ verursacht wird. Was hier dicht beieinander liegt, kann nat\"{u}rlich in den beiden Spalten durch L\"{u}cken voneinander getrennt sein, weil die Nummerierung dar\"{u}ber entscheidet, was wo steht.

Die Einheitsverformung $\Np_3$ in Abb. \ref{U448} a ist die Einflusslinie f\"{u}r die \"{a}quivalente Knotenkraft $f_3$ (das Knotenmoment), also das Moment, mit dem die Belastung $p(x)$ den festgehaltenen Knoten zu verdrehen versucht
\begin{align}
f_3 = \int_0^{\,l} \Np_3(x)\,p(x)\,dx\,.
\end{align}
Die Verformungen in Abb. \ref{U448} b sind nat\"{u}rlich die Einflussfunktion f\"{u}r die Verdrehung des Knotens.

%%%%%%%%%%%%%%%%%%%%%%%%%%%%%%%%%%%%%%%%%%%%%%%%%%%%%%%%%%%%%%%%%%%%%%%%%%%%%%%%%%%%%%%%%%%%%%%%%%%
{\textcolor{sectionTitleBlue}{\section{Die inverse Steifigkeitsmatrix als Analysetool}}}
Es geh\"{o}rt zu den ehernen Regeln in der {\em computational mechanics\/}, dass man die Inverse $\vek K^{-1}$ nicht berechnet und nicht abspeichert, aber man verpasst so doch einfache M\"{o}glichkeiten ein Tragwerk zu analysieren.

%%%%%%%%%%%%%%%%%%%%%%%%%%%%%%%%%%%%%%%%%%%%%%%%%%%%%%%%%%%%%%%%%%%%%%%%%%%%%%%%%%%%%%%%%%%%%%%%%%%
{\textcolor{sectionTitleBlue}{\subsection{Maximale Verformungen}}}\index{maximale Verformungen}\label{Korrektur12}
Weil die Elemente $g_{ij}$ der Inversen die Knotenwerte der Einflussfunktionen sind, s. Abb. \ref{U275},
\begin{align}
u_i = \int_0^{\,l} G_h(y,x_i)\,p(y)\,dy = \sum_{j = 1}^n \int_0^{\,l} g_{ij}\,\Np_j(y)\,p(y)\,dy = \sum_{j = 1}^n\,g_{ij}\,f_j
 = \vek g_i^T\,\vek f\,.
\end{align}
%---------------------------------------------------------------------------------
\begin{figure}
\centering
\if \bild 2 \sidecaption \fi
\includegraphics[width=.95\textwidth]{\Fpath/U275}
\caption{Der Eintrag $g_{ij}$ in $\vek K^{-1}$ beschreibt den gegenseitigen Einfluss }
\label{U275}%
\end{figure}%
%---------------------------------------------------------------------------------
kann man die $g_{ij}$ dazu benutzen, um \"{u}berschl\"{a}gig die Laststellungen zu finden, die die maximale Verformung in einem Knoten $x_i$ erzeugen, denn angenommen alle $f_i$ sind eins (oder zumindest gleich gro{\ss}), dann ist die maximal auftretende Verformung in dem Knoten die Summe \"{u}ber die Werte $g_{ij} > 0$ in der Spalte $i$
\begin{align}
\text{max}\,\,u_i = \sum_{j = 1}^n g_{ij} \qquad g_{ij} > 0\,,
\end{align}
also die \glq Quersumme\grq{} in Zeile $i$ (Spalte = Zeile) \"{u}ber die positiven Werte. Diese Knoten sind demnach als Lastknoten zu w\"{a}hlen.
%---------------------------------------------------------------------------------
\begin{figure}
\centering
\if \bild 2 \sidecaption \fi
\includegraphics[width=1.0\textwidth]{\Fpath/U388}
\caption{Einflussfunktion $G_0$ f\"{u}r eine Durchbiegung, den FG $u_7$. Wo $G_0 > 0$ ist, sind die Eintr\"{a}ge $g_{7j}$ in $\vek K^{-1}$ (Zeile 7, Spalte $j$) positiv und im umgekehrten Fall negativ. So kann ein Programm automatisch, indem es die Eintr\"{a}ge in Zeile 7 der Inversen $\vek K^{-1}$ Revue passieren l\"{a}sst, die g\"{u}nstigen oder ung\"{u}nstigen Laststellungen finden}
\label{U388}%
\end{figure}%
%---------------------------------------------------------------------------------

Die typische Frage, ob denn eine Last  $f_j$ im 3. Stock einen Einfluss auf die Verformung $u_i$ eines Knotens im 1. Stock hat, kann man also einfach an Hand der Gr\"{o}{\ss}e des Elementes $g_{ij}$ der Inversen beantworten, wie in Abb. \ref{U388} am Beispiel eines Rahmens gezeigt wird.

Man k\"{o}nnte auch Karten (Plots) generieren, wie weit die Kr\"{a}fte $f_j$ maximal von einem fest gew\"{a}hlten Knoten  entfernt sein d\"{u}rfen, damit noch etwas in $u_i$ sp\"{u}rbar ist. Das w\"{a}ren dann alle Knoten mit $|g_{ij}| > \varepsilon$, also gr\"{o}{\ss}er als eine gewisse Schranke $\varepsilon$.


%%%%%%%%%%%%%%%%%%%%%%%%%%%%%%%%%%%%%%%%%%%%%%%%%%%%%%%%%%%%%%%%%%%%%%%%%%%%%%%%%%%%%%%%%%%%%%%%%%%
{\textcolor{sectionTitleBlue}{\subsection{Maximale Momente}}}\index{maximale Momente}
Die Einflussfunktion f\"{u}r das Feldmoment in einem Riegel hat die Knotenwerte
 \begin{align}
 \vek g = \vek K^{-1}\,\vek j
 \end{align}
wobei die $j_i = M(\Np_i)$ die Momente der {\em shape functions\/} in der Mitte des Riegels sind. Das werden nicht mehr als vier Gr\"{o}{\ss}en $j_i \neq 0$ sein und so ist der Vektor $\vek g$ einfach zu berechnen. Das Feldmoment $M(x) = \vek g^T\,\vek f$ wird dann maximal/minimal, wenn man alle $f_i \gtreqless 0$ mitnimmt, die zu positiven/negativen Eintr\"{a}gen $g_i \gtreqless 0$ passen. So kann man also zu jedem interessierenden Punkt $x$ einen Vektor $\vek g$ berechnen und mit diesem Vektor die $f_i$ nach $g_i \cdot f_i \gtreqless 0$ sortieren, um die Extremwerte der Momente zu finden.

%%%%%%%%%%%%%%%%%%%%%%%%%%%%%%%%%%%%%%%%%%%%%%%%%%%%%%%%%%%%%%%%%%%%%%%%%%%%%%%%%%%%%%%%%%%%%%%%%%%
{\textcolor{sectionTitleBlue}{\subsection{Beliebige Funktionale}}}
Das geht mit jeder beliebigen Gr\"{o}{\ss}e, sprich jedem Funktional $J(u)$. Man muss ja einfach nur rechnen
\begin{align}
\vek g = \vek K^{-1}\,\vek j \qquad j_i = J(\Np_i)(x)
\end{align}
und kann dann das \glq Zeilenkriterium\grq{} auf den Vektor $\vek g$ der Einflussfunktion $J(u) = \vek g^T\,\vek f$ anwenden, sprich die Komponenten $g_i$ nach positiven und negativen Werten sortieren
\begin{align}
g_i \gtrless 0
\end{align}
und wei{\ss} dann, welche $f_i\gtrless 0$ einen positiven bzw. negativen Beitrag zu $J(u)$ leisten.

\pagebreak
%%%%%%%%%%%%%%%%%%%%%%%%%%%%%%%%%%%%%%%%%%%%%%%%%%%%%%%%%%%%%%%%%%%%%%%%%%%%%%%%%%%%%%%%%%%%%%%%%%%
{\textcolor{sectionTitleBlue}{\section{Lokale \"{A}nderungen und die Inverse}}}
Das Thema Inverse gibt uns auch Gelegenheit darauf hinzuweisen, dass lokale Steifigkeits\"{a}nderungen die ganze Inverse $\vek K^{-1}$ \"{a}ndern, auch wenn sich nur eine Zahl in $\vek K$ \"{a}ndert.
%----------------------------------------------------------------------------
\begin{figure}
\centering
{\includegraphics[width=.99\textwidth]{\Fpath/U273}}
  \caption{Stab aus vier Elementen}
  \label{U273}
\end{figure}%%

%----------------------------------------------------------

Der Stab in Abb. \ref{U273} hat im letzten Element die L\"{a}ngssteifigkeit $EA = a$ im Unterschied zu den ersten drei Elementen, in denen $EA = 1$ ist, und so kommt $a$ nur einmal vor
\begin{align}
\vek K = \left[ \barr{r @{\hspace{4mm}}r @{\hspace{4mm}}r} 2 &-1 &0 \\ -1 &2 &-1 \\ 0 &-1 &1 + a\earr\right]\,,
\end{align}
aber das $a$ taucht in jedem Element der Inversen $\vek G = \vek K^{-1}$ auf
\begin{align}
\vek K^{-1} = \left[
\begin{array}{c@{\hspace{4mm}}c@{\hspace{4mm}}c}
 \displaystyle{\frac{2 a+1}{3 a+1}} & \displaystyle{\frac{a+1}{3 a+1}} & \displaystyle{\frac{1}{3 a+1} }\vspace{0.3cm}\\
 \displaystyle{\frac{a+1}{3 a+1}} & \displaystyle{\frac{2 a+2}{3 a+1}} & \displaystyle{\frac{2}{3 a+1}} \vspace{0.3cm}\\
 \displaystyle{\frac{1}{3 a+1}} & \displaystyle{\frac{2}{3 a+1}}& \displaystyle{\frac{3}{3 a+1}}
\end{array}
\right]\,.
\end{align}

\hspace*{-12pt}\colorbox{highlightBlue}{\parbox{0.98\textwidth}{Eine {\em lokale\/} Steifigkeits\"{a}nderung \"{a}ndert also den Verlauf der Einflussfunktionen im {\em ganzen\/} Tragwerk.}}\\

Wir werden aber in Kapitel 5 sehen, dass es trotzdem einen Weg gibt, wie man Effekte von Steifigkeits\"{a}nderungen nur durch Betrachtung des betroffenen Elements verfolgen kann -- zumindest n\"{a}herungsweise.
\vspace{-0.5cm}
%%%%%%%%%%%%%%%%%%%%%%%%%%%%%%%%%%%%%%%%%%%%%%%%%%%%%%%%%%%%%%%%%%%%%%%%%%%%%%%%%%%%%%%%%%%%%%%%%%%
{\textcolor{sectionTitleBlue}{\section{Das Weggr\"{o}{\ss}enverfahren}}% hardwired

Das Weggr\"{o}{\ss}enverfahren basiert auf der Beobachtung, dass die Kenntnis der Weggr\"{o}{\ss}en $u_i$ an den \glq R\"{a}ndern\grq{}, in den Knoten, ausreicht, um die Schnittkr\"{a}fte in den St\"{a}ben zu berechnen, s. Abschnitt 1.18.
 Die Zahl der unbekannten $u_i$ ist der {\em Grad der kinematischen Unbestimmtheit\/}\index{Grad der kinematischen Unbestimmtheit}. Das Ziel ist es daher, herauszufinden, wie sich die $u_i$ unter Last einstellen. Dies f\"{u}hrt auf dasselbe System, $\vek K\,\vek u = \vek f$, wie bei den finiten Elementen.

{\textcolor{blue}{\subsection{Wie kommt man auf $\vek K \vek u = \vek f$ ?}}}

Wir stellen uns einen Rahmen vor, der aus mehreren St\"{a}ben besteht, zu denen je eine horizontale Verschiebung $u(x)$ und vertikale Biegelinie $w(x)$ geh\"{o}rt, also ein \glq Sammelsurium\grq{} von Funktionen. Wir werden uns aber im folgenden kurz halten und in der Notation so tun, als h\"{a}tten wir es nur mit einem Stab zu tun und alles auf eine Funktion $u(x)$ reduzieren, die stellvertretend f\"{u}r all die anderen Funktionen in dem Rahmen stehen m\"{o}ge. Wir betreiben ja hier mehr Algebra als Mathematik.
%----------------------------------------------------------------------------
\begin{figure}
\centering
{\includegraphics[width=0.80\textwidth]{\Fpath/U382}}
  \caption{Weggr\"{o}{\ss}enverfahren \textbf{ a)} System, \textbf{ b)} \"{A}quivalente Knotenkr\"{a}fte am festgehaltenen System und Momente der lokalen L\"{o}sung, \textbf{ c)} Momente $M_R$ im Lastfall Knotenlasten = $\vek f_K + \vek p$, \textbf{ d)} Momente $M = M_R + M_{loc}$; $M_R$ sind die Momente der Biegelinie $w_R$, wenn also erst die Streckenlasten in die Knoten reduziert wurden und dann die Knoten gel\"{o}st wurden}
  \label{U382}
\end{figure}%%

%----------------------------------------------------------

Zuerst werden alle Knoten festgehalten und die Belastung in die Knoten reduziert, d.h. es werden die \"{a}quivalenten Knotenkr\"{a}fte, die {\em actio\/},
\begin{align}
d_i = \int_0^{\,l} p\,\Np_i\,dx
\end{align}
berechnet, s. Abb. \ref{U382} b. Die Festhaltekr\"{a}fte sind die {\em reactio\/}, $- d_i$. Wichtig ist, dass man die exakten $\Np_i(x)$ benutzt, weil die {\em shape functions\/} ja die exakten Einflussfunktionen f\"{u}r die \"{a}quivalenten Knotenkr\"{a}fte sind.

Wir erinnern daran, dass das $\vek f$ auf der rechten Seite von $\vek K\,\vek u = \vek f$ eine Summe von zwei Vektoren
\begin{align} \label{Eq95}
\vek f = \vek f_K + \vek d
\end{align}
ist. In dem Vektor $\vek f_K$ stehen die Kr\"{a}fte, die direkt in den Knoten angreifen und der Vektor $\vek d$ enth\"{a}lt die \"{a}quivalenten Knotenkr\"{a}fte aus der verteilten Belastung. Die Belastung in den Knoten ist also die Summe (\ref{Eq95}).

L\"{a}sst man dann die Knoten los, so stellt sich in dem Rahmen eine Verformungsfigur $u_R(x)$ ($R$ = reduziert) ein, die die Gleichgewichtslage des Rahmens unter der Wirkung der Knotenlasten $f_i = f_{K @i} + d_i$ ist, s. Abb. \ref{U382} c. \\

\hspace*{-12pt}\colorbox{highlightBlue}{\parbox{0.98\textwidth}{Das Weggr\"{o}{\ss}enverfahren berechnet nur die Figur $u_R(x)$. Was fehlt, wird am Schluss als lokale L\"{o}sung stabweise zu $u_R(x)$ addiert, $u(x) = u_R(x) + u_{loc}$.}}\\

Weil $u_R(x)$ eine homogene L\"{o}sung ist, kann man sie stabweise mit den {\em shape functions\/} $\Np_i(x)$ darstellen,
\begin{align}
u_R(x) = \sum_i\,u_i\,\Np_i(x)\,,
\end{align}
denn die $\Np_i(x)$ bilden in jedem Stab ein {\em vollst\"{a}ndiges\/} System von homogenen L\"{o}sungen.

Es fehlen noch die Knotenverschiebungen $u_i$. Diese bestimmen wir mit Hilfe der ersten Greenschen Identit\"{a}t. Die Identit\"{a}t $\text{\normalfont\calligra G\,\,}(u,v) = 0$ ist f\"{u}r alle Paare $(u,v)$ von hinreichend glatten Funktionen null, und daher muss sie auch f\"{u}r jedes Paar $(u_R,\Np_j), j = 1,2,\ldots n$ null sein
\begin{align}\label{Eq74}
\text{\normalfont\calligra G\,\,}(u_R,\Np_j) = f_j - a(u_R,\Np_j) = f_j - \sum_i\,u_i\,a(\Np_i,\Np_j) = 0\,,
\end{align}
was das System
\begin{align}\label{Eq75}
\vek K\,\vek u = \vek f
\end{align}
oder hier
\begin{align}
\frac{EI}{l} \left[\barr{c @{\hspace{4mm}}c @{\hspace{4mm}}c} 4 & 2 & 0 \\ 2 & 8 & 2 \\ 0 & 2 & 4 \earr \right]\left[\barr{c c c} u_2 \\ u_4 \\ u_6 \earr \right] = \left[\barr{c c c} 0 \\ 0 \\ 10 \earr \right] + \left[\barr{c c c} -16 \\ -9 \\ 25 \earr \right] = \vek f_K + \vek d
\end{align}
ergibt. So kommt die Steifigkeitsmatrix in das Weggr\"{o}{\ss}enverfahren hinein.

Wenn man das System (\ref{Eq75}) in Einzelschritten l\"{o}st (Verfahren von {\em Gauss-Seidel\/}), dann entspricht das dem Vorgehen des {\em Drehwinkelverfahren\/} \index{Drehwinkelverfahren} und wenn man in es einem Schritt l\"{o}st, dann macht man es wie die finiten Elemente heute.

\begin{remark}
Dass sich in (\ref{Eq74}) die \"{a}u{\ss}ere Arbeit $\delta A$ auf
\begin{align}
\delta A_a = f_j = [V_R\,\Np_j - M_R\,\Np_j']_0^l = \sum_{i = 1}^4\,f_i(u_R)\,u_i(\Np_j)
\end{align}
reduziert, ist hoffentlich evident, denn $u_i(\Np_j) = \delta_{ij}$ (Kronecker Delta). Hier steht $u_i(\Np_j) $ f\"{u}r die Weggr\"{o}{\ss}en $u_1(\Np_j) = \Np_j(0), u_2(\Np_j) = \Np_j'(0), u_3(\Np_j) = \Np_j(l), u_4(\Np_j) = \Np_j'(l)$.
\end{remark}
\vspace{-0.5cm}
{\textcolor{sectionTitleBlue}{\subsection{Handberechnung von $\vek K$}}}
Bei einer Handberechnung stellt man die Steifigkeitsmatrix $\vek K$ spaltenweise von Hand auf. Die Spalte $\vek f_i$ enth\"{a}lt ja die \"{a}quivalenten Knotenkr\"{a}fte, die zur Einheitsverformung $\Np_i(x)$ geh\"{o}ren. Also lenkt man einen Knoten aus, $u_i = 1$, h\"{a}lt alle anderen Knoten fest und berechnet, welche Kr\"{a}fte $f_{ij}$ dies in Richtung der $u_j$ ergibt. Das $f_{ii}$ ist die \"{a}quivalente Knotenkraft, die den Knoten auslenkt und die anderen $f_{ij}$ sind die \"{a}quivalenten Knotenkr\"{a}fte in Richtung der \"{u}brigen $u_j$, die die Bewegung abstoppen.

So kann man die Steifigkeitsmatrix $\vek K = [\vek f_1, \vek f_2, \ldots , \vek f_n]$ spaltenweise direkt berechnen, was den kleinen Vorteil hat, dass man unter Umst\"{a}nden Freiheitsgrade sparen kann, man nicht jeden Stab als Vollstab mit $6 = 3 \times 2$ Freiheitsgraden ansetzen muss, sondern z.B. einen Pendelstab als ein $2 \times 2$ Element behandeln kann.

Die Handberechnung wird erg\"{a}nzt von der Berechnung der \"{a}quivalenten Knotenkr\"{a}fte $\vek f$ und am Schluss wird, wie zuvor, das System $\vek K\,\vek u = \vek f$ gel\"{o}st.

{\textcolor{sectionTitleBlue}{\subsection{Finite Elemente}}}

Wiederholen wir noch einmal, zum Vergleich, wie die finiten Elemente bei Stabtragwerken vorgehen: Zuerst wird die Belastung in die Knoten reduziert. Wir bezeichnen die zugeh\"{o}rige L\"{o}sung, also die Gleichgewichtslage des Rahmens unter den Knotenkr\"{a}ften mit $u_R$.

Mit den {\em shape functions\/} wird dann die FE-L\"{o}sung $u_h $ dieses Lastfalls gebildet
\begin{align}
 u_h = \sum_i u_i\,\Np_i(x)
\end{align}
und so eingestellt, dass bei jeder virtuellen Verr\"{u}ckung $\Np_i(x)$ die beiden L\"{o}sungen dieselbe virtuelle \"{a}u{\ss}ere Arbeit leisten,
\begin{align}
f_i = \delta A_a(u_R,\Np_i) = \delta A_a(u_h,\Np_i) = f_{h @i}\,,
\end{align}
was wegen
\begin{align}
\text{\normalfont\calligra G\,\,}(u_h,\Np_i) = \delta A_a - \delta A_i = f_{h @i} - \sum_j k_{ij} \,u_j = 0
\end{align}
mit dem Gleichungssystem
\begin{align}\label{Eq79}
\sum_j k_{ij} \,u_j = f_i \qquad i = 1,2\ldots n
\end{align}
identisch ist. Weil die L\"{o}sung $u_R(x)$ nach Konstruktion eine homogene L\"{o}sung ist (keine Belastung im Feld) kann sie mit den {\em shape functions\/} $\Np_i(x)$ dargestellt werden. Die $u_j$ aus (\ref{Eq79}) liefern also die exakte L\"{o}sung
\begin{align}
u_h(x) = \sum_i\,u_i\,\Np_i(x) = u_R(x)\,.
\end{align}
Was noch fehlt, sind die lokalen L\"{o}sungen, die man in einem zweiten Schritt zu $u_R(x)$ addiert und so ist die \"{U}bereinstimmung mit dem Weggr\"{o}{\ss}enverfahren perfekt.

Ein FE-Programm berechnet also automatisch die exakte L\"{o}sung $u_R(x)$, wenn die Steifigkeiten abschnittsweise konstant sind, denn dann liegt $u_R(x)$ in $\mathcal{V}_h$ und es liefert eine N\"{a}herung, wenn die Tr\"{a}ger gevoutet sind oder andere exotische Profile haben, weil dann $u_R(x)$ nicht mehr in $\mathcal{V}_h$ liegt.

{\textcolor{blue}{\subsection{Drehwinkelverfahren}}}
Fr\"{u}her hat man das Aufstellen und L\"{o}sen des Systems $\vek K\,\vek u = \vek f$ vermieden, ist man iterativ vorgegangen, wie in den Verfahren von {\em Cross\/} und {\em Kani\/}, \cite{Hirschfeld}. Exemplarisch wollen wir die einzelnen Schritte am Drehwinkelverfahren erl\"{a}utern.\\

\begin{enumerate}
  \item Erst werden alle Knoten festgehalten und die Schnittkraftverl\"{a}ufe (= lokale L\"{o}sungen) und die Festhaltekr\"{a}fte in den Knoten berechnet.
  \item Knotenweise werden dann die Knoten ausgeglichen: Man l\"{o}st einen Knoten und stellt das Gleichgewicht her, erlaubt dem Knoten also sich so zu verdrehen, dass die inneren Momente dem Lastmoment das Gleichgewicht halten k\"{o}nnen. Der Ausgleich an einem Knoten f\"{u}hrt zur Weiterleitung von Momenten an die (weiterhin festgehaltenen) Nachbarknoten.
  \item Danach wird der Knoten wieder gesperrt und der n\"{a}chste Knoten ausgeglichen. Nach drei- bis viermaligen Durchl\"{a}ufen konvergiert das Verfahren in der Regel, weil die Gr\"{o}{\ss}e der weiterzuleitenden Momente relativ schnell abnimmt.
\end{enumerate}

Man beginnt also, wie bei den finiten Elementen, mit der Reduktion der Belastung in die Knoten. Aber w\"{a}hrend die finiten Elemente das System $\vek K\,\vek u = \vek f$ heute in einem Schritt l\"{o}sen, hat der Praktiker das System im Grunde fr\"{u}her iterativ gel\"{o}st, zeilenweise, was mathematisch dem Verfahren von {\em Gauss-Seidel\/} entspricht. Das ist, aus mathematischer Sicht, im Grunde der einzige Unterschied zwischen dem Drehwinkelverfahren und den finiten Elementen. Wenn sich nat\"{u}rlich der Ingenieur so auch das Aufstellen der Steifigkeitsmatrix $\vek K$ erspart hat.

\begin{remark}
Das Weggr\"{o}{\ss}enverfahren bildet traditionsgem\"{a}{\ss} den Gegenpol zum Kraftgr\"{o}{\ss}enverfahren. Die starke \glq Magnetwirkung\grq{} der finiten Elementen hat dazu gef\"{u}hrt, dass das Verfahren heute meist in einer Matrizenschreibweise dargestellt wird.

Wir geben daher zu Bedenken, ob man das Weggr\"{o}{\ss}enverfahren nicht gleich in FE-Schreibweise formulieren sollte, denn dann spart man Zeit, kann schon sehr fr\"{u}h die finiten Elemente (noch als exakte, homogene L\"{o}sungen) einf\"{u}hren und hat den ganzen Apparat zur Hand, um aus den Elementmatrizen die Steifigkeitsmatrix zu erzeugen.

Die \glq echten\grq{} finiten Elemente kann man ja dann als Galerkin-Verfahren einf\"{u}hren, als Projektion der exakten L\"{o}sung auf den Ansatzraum $\mathcal{V}_h$, mit derselben Gleichung $\vek K\,\vek u = \vek f$ wie beim Weggr\"{o}{\ss}enverfahren und der Student hat den Vorteil, dass er in dem Neuen das Vertraute wiedererkennt.
\end{remark}
\vspace{-0.5cm}
%%%%%%%%%%%%%%%%%%%%%%%%%%%%%%%%%%%%%%%%%%%%%%%%%%%%%%%%%%%%%%%%%%%%%%%%%%%%%%%%%%%%%%%%%%%%%%%%%%%
{\textcolor{sectionTitleBlue}{\section{Mohr und die Flexibilit\"{a}tsmatrix $\vek K^{-1}$}}
Es sei $\vek K$ die Steifigkeitsmatrix eines Fachwerks. Die Steifigkeit $k_i$ des Fachwerks in Richtung des Freiheitsgrades $u_i$ ist definiert als
\begin{align}
k_i = \frac{1}{u_i}\,,
\end{align}
wobei $u_i$ die Verschiebung ist, die aus einer Einzelkraft $f_i = 1$ resultiert.

Der Verschiebungsvektor in diesem Lastfall ist $\vek u = \vek K^{-1}\,\vek e_i$ und dies ist der Grund, warum die Verschiebung
\begin{align}\label{Eq168}
u_i = \vek e_i^T\,\vek u = \vek e_i^T\,\vek K^{-1}\,\vek e_i = g_{ii}
\end{align}
mit dem Eintrag $g_{ii}$ auf der Diagonalen der Flexibilit\"{a}tsmatrix $\vek F = \vek K^{-1}$ identisch ist.

Wir k\"{o}nnten $u_i$ aber auch mit dem Mohrschen Arbeitsintegral
\begin{align}
u_i = g_{ii} = \sum_e\,\frac{N_e^2\,l_e}{EA_e}
\end{align}
berechnen, wobei $N_e$ die Normalkraft in dem Stab $e$ im Lastfall $\vek f = \vek e_i$ ist.

Bei Stockwerkrahmen w\"{u}rde dieselbe Gleichung wie folgt lauten
\begin{align}
u_i = g_{ii} =  \sum_e\,\int_0^{\,l_e} (\frac{N_e^2}{EA_e} + \frac{M_e^2}{EI_e})\,dx\,,
\end{align}
wobei $N_e$ und $M_e$ die korrespondierende Bedeutung haben.

In moderner Notation k\"{o}nnten wir das schreiben als
\begin{align}
g_{ii} = a(u^{(i)}, u^{(i)})
\end{align}
wobei $u^{(i)}$ die FE-L\"{o}sung des Lastfalls $\vek f = \vek e_i$ ist. Die Verallgemeinerung auf die Nebendiagonalen ist offensichtlich
\begin{align}
g_{ij} =  \vek e_j^T\,\vek K^{-1}\,\vek e_i = \sum_e\,\frac{N_e^{(i)}\,N_e^{(j)}\,l_e}{EA_e} = a(u^{(i)}, u^{(j)})\,.
\end{align}
Auf den ersten Blick ist es erstaunlich, dass wir dasselbe Ergebnis einmal mit $\vek K$ und einmal mit $\vek K^{-1}$ schreiben k\"{o}nnen
\beq
u_i =  \left \{ \begin{array}{l } {\displaystyle \vek g^T\,\vek K\,\vek u }   \vspace{0.3 cm}       \\
{\displaystyle \vek e_i^T\,\vek K^{-1}\,\vek e_i}\,,
\end{array} \right.
\eeq
aber etwas lineare Algebra macht schnell klar, warum das geht.

Der Vektor $\vek g$ ist die L\"{o}sung des Gleichungssystems
\begin{align}
\vek K\,\vek g = \vek j = \vek e_i
\end{align}
und daher sind in der Tat die beiden Formen gleich
\begin{align}
J(\vek u) = u_i = \vek g^T\,\vek K\,\vek u = \vek e_i^T\,\vek u = \vek e_i^T\,\vek K^{-1}\,\vek e_i\,.
\end{align}
Es ist auch hilfreich sich daran zu erinnern, dass die Spalten $\vek g_i$ der Inversen $\vek K^{-1}$ die Knotenvektoren $\vek g$ der Einflussfunktionen der Knotenverschiebungen sind und daher gilt nat\"{u}rlich $u_i = g_i = g_{ii}$.

%----------------------------------------------------------------------------
\begin{figure}
\centering
{\includegraphics[width=0.9\textwidth]{\Fpath/UE255D}}
  \caption{Einflussfunktion f\"{u}r $ \sigma_{yy}$ im mittleren Element, \textbf{ a)} die Elementsteifigkeit hat sich verdoppelt und \textbf{ b)} hat ihren normalen Wert, \textbf{ c)} Einflussfunktion f\"{u}r die Normalkraft $N(x)$ in einem Stab---die Funktion \"{a}ndert sich nicht}
  \label{U255}
\end{figure}%%
%----------------------------------------------------------

%%%%%%%%%%%%%%%%%%%%%%%%%%%%%%%%%%%%%%%%%%%%%%%%%%%%%%%%%%%%%%%%%%%%%%%%%%%%%%%%%%%%%%%%%%%%%%%%%%%
{\textcolor{sectionTitleBlue}{\section{Querschnitts\"{a}nderungen}}}\index{Querschnitts\"{a}nderungen}
Wenn die St\"{a}rke einer Scheibe abschnittsweise unterschiedlich ist, dann \"{a}ndert sich an der Berechnung der Einflussfunktionen technisch nichts. Man kann nur sehr sch\"{o}n beobachten, wie steife Zonen Kr\"{a}fte anziehen und weiche Zonen vermieden werden.

Wenn, wie in Abb. \ref{U255}, das Material in der e\'{\i}ngebetteten Zone h\"{a}rter ist als die Umgebung, dann wird die Einflussfunktion f\"{u}r die Spannung $\sigma_{yy}$ in der Mitte der Zone weit ausstrahlen, d.h. ein relativ gro{\ss}er Anteil der Belastung flie{\ss}t durch die steife Zone. Umgekehrt, wenn die Zone sehr viel weicher ist als die Umgebung, s. Abb. \ref{UE324}, dann behindert die Umgebung die Ausbreitung der Spreizung des Aufpunktes, d.h. nur ein kleiner Anteil der Belastung wird durch den weichen Kern flie{\ss}en.
%----------------------------------------------------------------------------
\begin{figure}
\centering
{\includegraphics[width=0.85\textwidth]{\Fpath/U495}}
  \caption{Hochbauplatte mit zwei abgeminderten Bereichen, 20 cm statt 40 cm im LF $g$, {\bf a)} Hauptmomente, {\bf b)} Einflussfunktion f\"{u}r $m_{xx}$ im Aufpunkt 1 und {\bf c)} im Aufpunkt 2} \label{UE324}
\end{figure}
%----------------------------------------------------------------------------
%----------------------------------------------------------------------------
\begin{figure}
\centering
{\includegraphics[width=0.9\textwidth]{\Fpath/KOPFMOMENTED}}
  \caption{Platte mit St\"{u}tzenkopfverst\"{a}rkung, Verteilung der \textbf{a)} Momente $m_{xx}$ (as in $x$-Richtung) und  \textbf{b)} $m_{yy}$ (as in $y$-Richtung), \cite{Ha5}} \label{Kopfmomente}
\end{figure}
%----------------------------------------------------------------------------
Das erscheint logisch, aber es verbleibt doch eine Frage: Wenn der E-Modul des Elementes sich verdoppelt, dann kostet es doppelt so viel M\"{u}he das Element auseinander zu rei{\ss}en, um den Verschiebungssprung in vertikaler Richtung (= Einflussfunktion f\"{u}r $\sigma_{yy}$) n\"{a}herungsweise zu generieren. Es ist klar, dass die Kr\"{a}fte $\vek f$ dann doppelt so gro{\ss} sein m\"{u}ssen wie im einfachen Fall. Aber es muss anscheinend so sein, dass diese doppelt so gro{\ss}en Kr\"{a}fte sich nicht allein in dem Spreizen des Elements verbrauchen, sondern dass ein Teil \"{u}brig bleibt, die obere Kante der Scheibe \"{u}berproportional weit nach oben zu dr\"{u}cken. Die Kr\"{a}fte $2 \cdot \vek f$ kommen da anscheinend weiter als die Kr\"{a}fte $\vek f$.


Ein Blick auf einen Zugstab, Abb. \ref{U255} c, kann helfen. Wenn man in einem Element die Steifigkeit verdoppelt, dann \"{a}ndert das nicht die Einflussfunktion $G(y,x)$ f\"{u}r $N(x)$ am Stabende $y = l$, weil die Verdopplung von $EA$ die Steigung der Einflussfunktion $G_c$ in dem Element halbiert und wegen
\begin{align}
G_c' \cdot 2 \cdot f \cdot l_e = \frac{1}{2}\, G' \cdot 2 \cdot f \cdot l_e = (G(y_b,x) - G(y_a,x)) \cdot f
\end{align}
\"{a}ndert sich nichts, ($y_a$ und $y_b$ sind die Endpunkte des Elements mit der L\"{a}nge $l_e$).

Bei der Scheibe ist das anscheinend anders. Eine Verdopplung des E-Moduls halbiert nicht die Steigung der Einflussfunktion f\"{u}r $\sigma_{yy}$ (jetzt in vertikaler Richtung), sondern der Abfall der Steigung muss geringer sein, vielleicht weil sich die Steifigkeiten in den Elementen links und rechts von dem Element $2\cdot E$ ja nicht ge\"{a}ndert haben, und so kommt es zu dem \glq \"{U}berschuss\grq{} an der oberen Kante.

Eine St\"{u}tzenkopfverst\"{a}rkung verursacht Spr\"{u}nge in den Momenten $m_{yy}$, aber -- wie beim
Balken -- nicht in den Momenten $m_{xx}$, s. Abb. \ref{Kopfmomente}.


%----------------------------------------------------------------------------
\begin{figure}
\centering
{\includegraphics[width=0.99\textwidth]{\Fpath/U438}}
  \caption{Einflussfunktionen f\"{u}r Spannungen $\sigma_{yy}$ und $\sigma_{xx}$ in zwei Scheiben. Die Pfeile sind die Knotenverschiebungen $\vek g_i$ in den Knoten $\vek x_i$ aus der Spreizung des Aufpunkts. Knotenkr\"{a}fte $\vek f_i$ aus der Belastung, die in Richtung der $\vek g_i$ weisen, haben maximalen Einfluss und Knotenkr\"{a}fte, die senkrecht auf den $\vek g_i$ stehen, keinen Einfluss}  \label{U438}\label{Korrektur21}
\end{figure}
%----------------------------------------------------------------------------
%----------------------------------------------------------------------------
\begin{figure}
\centering
{\includegraphics[width=1.0\textwidth]{\Fpath/U11}}
  \caption{ Plot der Knotenvektoren $\vek g_i$ des Funktionals $J(u_h) = \sigma_{xx}$, der horizontalen Spannungen in der Scheibe nahe der \"{O}ffnung}
  \label{U11}
\end{figure}%%
%----------------------------------------------------------------------------
%----------------------------------------------------------------------------
\begin{figure}
\centering
{\includegraphics[width=0.8\textwidth]{\Fpath/U416}}  %% Pos. KURZ
  \caption{Scheibe aus vier bilinearen Elementen, links festgehalten, Einflussfunktion f\"{u}r $\sigma_{xx}$ im Knoten oben links. Knotenkr\"{a}fte, die auf den roten Linien liegen, senkrecht zu den Verschiebungslinien der Knoten, verursachen keine Spannungen $\sigma_{xx}$ in dem Knoten}
  \label{U416}
\end{figure}%%
%----------------------------------------------------------
%----------------------------------------------------------------------------
\begin{figure}
\centering
{\includegraphics[width=0.8\textwidth]{\Fpath/U222}}
  \caption{Plot der Knotenvektoren $\vek g_i$ des Funktionals $J(u_h) = \sigma_{yy}$, also der vertikalen Spannung im Rissgrund. Die {\em Lagrangepunkte\/} sind die Punkte, in denen der Einfluss der Knotenkr\"{a}fte $\vek f_i$ auf $\sigma_{yy}$ praktisch null ist}
  \label{U222}
\end{figure}%%
%----------------------------------------------------------

%%%%%%%%%%%%%%%%%%%%%%%%%%%%%%%%%%%%%%%%%%%%%%%%%%%%%%%%%%%%%%%%%%%%%%%%%%%%%%%%%%%%%%%%%%%%%%%%%%%
{\textcolor{sectionTitleBlue}{\section{Sensitivit\"{a}tsplots}}}\index{Sensitivit\"{a}tsplots}
Das Ergebnis $J(u_h) = \vek g^T\,\vek f$ ist das Skalarprodukt aus dem Vektor $\vek g$, also den Knotenwerten der Einflussfunktion, und dem Vektor $\vek f$ der \"{a}quivalenten Knotenkr\"{a}fte aus der Belastung. Dieses Skalarprodukt kann als eine Summe \"{u}ber die $N$ Knoten des FE-Netzes geschrieben werden
\beq
J(u_h) =  \sum_{i = 1}^N \vek g_i^T\,\vek f_i \qquad i = \text{Knoten}\,,
\eeq
wobei die Vektoren $\vek g_i$ und $\vek f_i$ die Anteile aus den gro{\ss}en Vektoren $\vek g$ and $\vek f$ sind, die sich auf den Knoten $i$ beziehen
\beq
\vek g = \{\underbrace{g_1, g_2}_{\vek g_1}, \underbrace{g_3, g_4}_{\vek g_2}, \ldots, g_{2N}\}^T \qquad 2-D\,.
\eeq
Wenn daher $\vek f_i$ in einem Knoten orthogonal zu $\vek g_i$ ist, dann ist der Beitrag des Knotens zu $J(u_h)$ null. Der Plot der Vektoren $\vek g_i$ gleicht somit einem {\em Sensitivit\"{a}tsplot\/} des Funktionals $J(u_h)$, siehe die Bilder \ref{U438}, \ref{U11} und \ref{U416}. Knotenkr\"{a}fte $\vek f_i$, die in dieselbe Richtung zeigen wie die $\vek g_i$, \"{u}ben einen maximal gro{\ss}en Einfluss auf $J(u_h)$ aus\footnote{Das Verformungsbild einer Scheibe gleicht -- auch in normalen Lastf\"{a}llen -- einer eingefrorenen Str\"{o}mung. Nur sind die FE-Netze meist zu grob, um das zu sehen. Die Bilder hier wurden mit Randelementen erzeugt. }.
%-----------------------------------------------------------------
\begin{figure}[tbp]
\centering
\includegraphics[width=.99\textwidth]{\Fpath/U434}
\caption{Fachwerk mit biegesteifen Knoten \textbf{ a)} Einflussfunktion f\"{u}r die vertikale Verschiebung im Knoten 2 \textbf{ b)} Sensitivit\"{a}tsplot der Verschiebung} \label{U434}
\end{figure}%
%-----------------------------------------------------------------

In Abb. \ref{U222} ist die Einflussfunktion f\"{u}r die Spannung $\sigma_{yy}$ im Rissgrund einer Zugscheibe dargestellt. Auff\"{a}llig ist, dass es zwei ruhige Zonen gibt, in denen der Einfluss der Knotenkr\"{a}fte auf $J(u_h)$ praktisch null ist. Wir nennen diese Punkte {\em  Lagrangepunkte\/}\index{Lagrangepunkt}. In der Astronomie sind die Lagrangepunkte die Punkte, in denen sich die Gravitationskr\"{a}fte der Sonne und des Mondes das Gleichgewicht halten, weil sie mit gegengleichen Kr\"{a}ften an einem Satelliten ziehen, der dort geparkt ist. Solche Lagrangepunkte findet man in fast allen diesen Plots.\\

\begin{remark}
Mit einem FE-Programm erzeugt man diese Bilder wie folgt:
\begin{enumerate}
  \item Man bringt die $j_i = J(\vek \Np_i)$ als \"{a}quivalente Knotenkr\"{a}fte auf und l\"{o}st das System $\vek K\,\vek g = \vek j$.
  \item Man plottet in jedem Knoten $k$ den Vektor $\vek g_k = \{g_x^{(k)}, g_y^{(k)}\}^T$, also die horizontale und vertikale Verschiebung des Knotens.
\end{enumerate}
\end{remark}
%-----------------------------------------------------------------
\begin{figure}[tbp]
\centering
\includegraphics[width=.99\textwidth]{\Fpath/U436}
\caption{Stockwerkrahmen mit Diagonalstreben \textbf{ a)} Einflussfunktion f\"{u}r die horizontale Lagerkraft im linken Lager \textbf{ b)} f\"{u}r die vertikale Lagerkraft im vorletzten Lagerknoten} \label{U436}
\end{figure}%
%-----------------------------------------------------------------

Bei eindimensionalen Problemen wie Rahmen entstehen die Sensitivit\"{a}tsplots, wenn man die Einflussfunktionen als ebene Verschiebungsfiguren antr\"{a}gt, wie in Abb. \ref{U434}, also in einer Figur beide Anteile, horizontal wie vertikal, zeichnet. Wanderlasten, die maximalen Effekt erzielen wollen, m\"{u}ssen in Richtung der roten Pfeile zeigen -- senkrecht dazu ist die Wirkung null.

Abb. \ref{U436} enth\"{a}lt die Sensitivit\"{a}tsplots f\"{u}r zwei Lagerkr\"{a}fte eines Stockwerkrahmens mit  Diagonalstreben.


%%%%%%%%%%%%%%%%%%%%%%%%%%%%%%%%%%%%%%%%%%%%%%%%%%%%%%%%%%%%%%%%%%%%%%%%%%%%%%%%%%%%%%%%%%%%%%%%%%%
{\textcolor{sectionTitleBlue}{\section{Die Lagerkr\"{a}fte der FE-L\"{o}sung}}}\label{Korrektur20}
Das Thema Lagerkr\"{a}fte und FE-L\"{o}sungen muss man mit Vorsicht angehen. Nichts liegt n\"{a}her, als die Knotenkr\"{a}fte direkt in Lagerkr\"{a}fte zu \"{u}bersetzen. Bei Stabtragwerken sind es echte Kr\"{a}fte, bei Fl\"{a}chentragwerken sind es jedoch in der Regel nur \"{a}quivalente Knotenkr\"{a}fte, also Kr\"{a}fte, die nicht wirklich im strengen Punktsinn vorhanden sind.

%-----------------------------------------------------------------
\begin{figure}[tbp]
\centering
\includegraphics[width=0.8\textwidth]{\Fpath/U142}
\caption{Starre St\"{u}tze, die gesamte St\"{u}tzenkraft ist die Summe aus der St\"{u}tzenkraft $R_{FE}$ der FE-L\"{o}sung plus dem direkt in die St\"{u}tze reduziertem Anteil aus der Last $p$} \label{U142}
\end{figure}%
%-----------------------------------------------------------------

Die $f_i$ in den Lagerknoten berechnet ein FE-Programm im Nachlauf, nachdem es das System $\vek K\,\vek u = \vek f$ gel\"{o}st hat, wie folgt: \\
\begin{itemize}
  \item Es erweitert den Vektor $\vek u$ zun\"{a}chst um die zuvor gestrichenen $u_i = 0$ in den Lagerknoten, $\vek u \to \vek u_{G}$,
  \item und multipliziert die nicht-reduzierte, globale Steifigkeitsmatrix $\vek K_{G}$\index{nicht-reduzierte Steifigkeitsmatrix} mit dem vollen Vektor $\vek u_{G}$,
  \item die Eintr\"{a}ge $f_i$ in dem Vektor $\vek f_{G} = \vek K_{G}\,\vek u_{G}$, die zu den gesperrten Freiheitsgraden geh\"{o}ren, sind die Knotenkr\"{a}fte in den Lagern {\em ohne\/} die Anteile der Last, die direkt in die Lager reduziert wurden. Zu diesen muss man also noch die Lagerkr\"{a}fte aus der direkten Reduktion addieren, die wir $R_{d}$ nennen, s. Abb. \ref{U142},
      \begin{align}
      f_i(komplett) = f_i +  R_{d} = R_{FE} + R_{d}\,.
      \end{align}
      \item Wenn allerdings die Lager nachgiebig gerechnet wurden, dann ist das letzte Man\"{o}ver nicht notwendig, dann beinhaltet $f_i = R_{FE}$ die volle Lagerkraft.
\end{itemize}

Die Summe der $f_i (komplett)$ in den Lagern muss gleich der aufgebrachten Belastung sein, weil die {\em shape functions\/} im Regelfall eine {\em partition of unity\/} bilden oder bilden sollten, die Starrk\"{o}rperbewegungen des frei geschnittenen Tragwerks also in $\mathcal{V}_h^+$ liegen, s. \cite{Ha5} Chapter 1.35 und 1.44. \index{Gleichgewicht}\index{$\mathcal{V}_h^+$}

Die Ansatzfunktionen $\Np_i$ eines Netz bilden den Raum $\mathcal{V}_h^+$. Wenn man die Ansatzfunktionen $\Np_i$ streicht, die zu gesperrten Freiheitsgraden geh\"{o}rt, dann erh\"{a}lt man den Unterraum $\mathcal{V}_h \subset \mathcal{V}_h^+$ auf dem wir die FE-L\"{o}sung suchen.

Lokal ist das Gleichgewicht, also der Vergleich der Belastung $p$ mit den Schnittkr\"{a}ften der FE-L\"{o}sung, nicht erf\"{u}llt, weil ja die FE-L\"{o}sung zu einem anderen Lastfall, dem FE-Lastfall $p_h$ geh\"{o}rt. Technisch ist der Grund der, dass die Starrk\"{o}rperbewegungen der lokalen {\em patchs\/} nicht in $\mathcal{V}_h$ liegen.

Was es mit der Zweiteilung der Lagerkr\"{a}fte in \glq echte\grq{} und \glq nur gedachte\grq{} auf sich hat, soll zun\"{a}chst an Hand der Stabstatik untersucht werden.

%%%%%%%%%%%%%%%%%%%%%%%%%%%%%%%%%%%%%%%%%%%%%%%%%%%%%%%%%%%%%%%%%%%%%%%%%%%%%%%%%%%%%%%%%%%%%%%%%%%
{\textcolor{sectionTitleBlue}{\section{Lagersenkung}}}\label{Korrektur17}\index{Lagersenkung}
Es sei $u_5$ der Freiheitsgrad, der zu dem abgesenkten ($\Delta w$) Lager geh\"{o}re. Das FE-Programm bringt die Spalte $\vek f_5$ von $\vek K$ auf die rechte Seite, streicht die Zeile 5 und Spalte 5 in dem System und bestimmt den Vektor $\vek u$ aus dem verk\"{u}rzten System $\vek K\,\vek u = - \Delta w\cdot \vek f_5$ und findet so die Biegelinie
\begin{align}
w(x) = \sum_{\stackrel{i \neq 5}{i = 1}}^n\, u_i\,\Np_i(x) + \Delta w \cdot \Np_5(x) =: w_V(x) + \Delta w \cdot \Np_5(x)\,.
\end{align}
Die Umstellung  $\vek K\,\vek u = - \Delta w \cdot \vek f_5$ ist nachvollziehbar und schl\"{u}ssig, aber auch sie hat einen mathematischen Hintergrund.

Es gilt n\"{a}mlich, dass Biegelinien $w$ aus Lagersenkung orthogonal sind zu den virtuellen Verr\"{u}ckungen des Systems, $\delta A_i(w,\delta w) = 0$, s. Glg. (\ref{Eq120}) S. \pageref{Eq120}. W\"{a}hlen wir als $\delta w$ die $\Np_i$ -- mit Ausnahme von $\Np_5$ selbst, weil man das abgesenkte aber feste Lager ja nicht verr\"{u}cken kann -- dann gilt also
\begin{align}
\delta A_i(w,\Np_i) = a(w_V + \Delta w \cdot \Np_5,\Np_i) = a(w_V,\Np_i) + \Delta w \cdot a(\Np_5,\Np_i) = 0
\end{align}
und somit
\begin{align}
a(w_V,\Np_i) = - \Delta w \cdot a(\Np_5,\Np_i) \qquad i = 1,2,\ldots, n \,\,\,\, i \neq 5
\end{align}
und das ist genau das verk\"{u}rzte System $\vek K\,\vek u = -\Delta w \,\vek f_5$, weil $w_V$ ja eine Entwicklung nach den {\em shape functions\/} $\Np_i$ ist (mit L\"{u}cke bei $\Np_5$).
%-----------------------------------------------------------------
\begin{figure}[tbp]
\centering
\includegraphics[width=.99\textwidth]{\Fpath/U443}
\caption{Lagersenkung} \label{U443}
\end{figure}%
%-----------------------------------------------------------------

Rechentechnisch einfacher ist vielleicht das folgende Vorgehen: Man entfernt den Knoten nicht, belastet wie zuvor die Knoten mit den Kr\"{a}ften
\begin{align}
f_j = -a(\Np_5,\Np_j) \cdot \Delta w= - k_{5j} \cdot \Delta w
\end{align}
und addiert am Schluss zu der Biegefigur die mit der Lagersenkung $\Delta w$ skalierte Einheitsverformung $\Np_5 \cdot \Delta w$ des Lagers.

Das kann man sich so zurechtlegen: In Spalte 5 stehen die Kr\"{a}fte $f_j = k_{5j}$, die n\"{o}tig sind, um den FG $u_5$ um eine Einheit auszulenken und gleichzeitig die Bewegung an den n\"{a}chsten Knoten zum Stillstand zu bringen. Indem man die Knoten mit den $-f_j$ belastet, hebt man die Sperre auf (alles nat\"{u}rlich mal $\Delta w$), und weil man das Lager nicht weggenommen hat, muss man noch $\Np_5 \cdot \Delta w$ zur Biegelinie addieren.

{\em Beispiel:\/} Das rechte Lager in Abb. \ref{U443} senkt sich um 1 m. An der nicht verk\"{u}rzten Steifigkeitsmatrix
\begin{align}
\left[ \barr{l  @{\hspace{4mm}}l   @{\hspace{4mm}}l} 8 &6\,\leftarrow &2 \\ 6 &12 &6 \\ 2 & 6\,\leftarrow &4 \earr \right] \left[ \barr{c} u_3 \\u_5 \\u_6 \earr\right] = \left[ \barr{c} f_3 \\f_5 \\f_6 \earr\right]
\end{align}
kann man ablesen, wie man das System (mit gesperrtem FG 5) zu belasten hat
\begin{align}
\left[ \barr{c  @{\hspace{4mm}}c} 8 &2 \\ 2  &4 \earr \right] \left[ \barr{c} u_3 \\u_6 \earr\right] = -\left[ \barr{c} 6 \\6 \earr\right]
\end{align}
Mit der L\"{o}sung $u_3 = -0.4286$, $u_6 = -1.2857$ und $u_5 = 1.0$ ergibt sich so die Biegelinie als die Summe der drei Einheitsverformungen
\begin{align}
w(x) = u_3\,\Np_3(x) + u_5\,\Np_5(x) + u_6\,\Np_6(x)\,.
\end{align}

%%%%%%%%%%%%%%%%%%%%%%%%%%%%%%%%%%%%%%%%%%%%%%%%%%%%%%%%%%%%%%%%%%%%%%%%%%%%%%%%%%%%%%%%%%%%%%%%%%%
{\textcolor{sectionTitleBlue}{\section{Einflussfunktion f\"{u}r ein starres Lager}}}
Vom Standpunkt der \glq Schulstatik\grq{} aus ist alles klar: Man entfernt das Lager und dr\"{u}ckt den Balken um einen Meter nach unten, s. den vorhergehenden Abschnitt. Die sich dabei einstellende Biegelinie ist die Einflussfunktion. Das bleibt nat\"{u}rlich weiterhin richtig.

Nun kann man sich dem Problem aber auch aus der Sicht der finiten Elemente n\"{a}hern. Eine Lagerkraft $R$ ist ein Funktional
\begin{align}
R = J(w)\,,
\end{align}
angewandt auf die Biegelinie $w$ und die FE-Einflussfunktion f\"{u}r $R$ erh\"{a}lt man, so lautet die Regel, wenn man als Knotenkr\"{a}fte $j_i$ die Lagerkr\"{a}fte der Ansatzfunktionen w\"{a}hlt
\begin{align}
j_i = R(\Np_i)\,.
\end{align}
Also die Lagerkraft, die zur Biegelinie $w = \Np_i$ geh\"{o}rt.

Diese Lagerkraft ist aber gleich der Wechselwirkungsenergie $\times (-1)$ zwischen $\Np_i $ und der Funktion $\Np_l$
\begin{align}\label{Eq53}
a(\Np_i,\Np_l) = \int_0^{\,l} EI\,\Np_i''\,\Np_l''\, dx = - R(\Np_i) \cdot 1\,,
\end{align}
wenn $\Np_l$ die Einheitsverformung des Lagerknotens ist, also die Funktion, bei der sich das Lager um eine L\"{a}ngeneinheit nach unten bewegt. Die Terme (\ref{Eq53}) stehen in der Spalte $l$ der nicht reduzierten globalen Steifigkeitsmatrix $\vek K_G$.

%-----------------------------------------------------------------
\begin{figure}[tbp]
\centering
\includegraphics[width=.99\textwidth]{\Fpath/U106}
\caption{Tr\"{a}ger, \textbf{ a)} Lagerreaktion $R$, \textbf{ b)} Einflussfunktion $G$ f\"{u}r $R$, \textbf{ c)} FE-Einflussfunktion $G_h$ auf $\mathcal{V}_h$; die Knotenkr\"{a}fte $j_i$, die $G_h$ erzeugen, sind die Lagerkr\"{a}fte $R$ der Ansatzfunktionen} \label{U106}
\end{figure}%
%-----------------------------------------------------------------

Dieses Ergebnis beruht auf der ersten Greenschen Identit\"{a}t, denn (\ref{Eq53}) ist die Identit\"{a}t in der Gestalt $\delta A_i = \delta A_a$
\begin{align}
\text{\normalfont\calligra G\,\,}(\Np_i,\Np_l) &= \text{\normalfont\calligra G\,\,}(\Np_i,\Np_l)_{(0,x)} + \text{\normalfont\calligra G\,\,}(\Np_i,\Np_l)_{(x,l)}\nn \\
&=\underbrace{V(\Np_i)(x_{-}) - V(\Np_i)(x_{+})}_{-R(\Np_i)} -\int_0^{\,l} EI\,\Np_i''\,\Np_l''\, dx = 0\,.
\end{align}
Die $\Np_i $ sind homogene L\"{o}sungen der Balkengleichung und $\Np_l$ ist in allen Knoten, bis auf den Lagerknoten, null. Dort ist $\Np_l = 1$ und $V$ springt. Das erkl\"{a}rt, warum die virtuelle \"{a}u{\ss}ere Arbeit sich auf den Ausdruck $ -R(\Np_i) \cdot 1$ verk\"{u}rzt.

\hspace*{-12pt}\colorbox{highlightBlue}{\parbox{0.98\textwidth}{Die Lagerkraft $R$ einer Ansatzfunktion $\Np_i(x)$ ist gleich der Wechselwirkungsenergie $a(\Np_i,\Np_l) \times (-1)$  zwischen $\Np_i(x)$ und $\Np_l(x)$. Wenn $u_l$ der Freiheitsgrad des Lagers ist, dann ist also $\vek j = -\vek K_G\,\vek e_l$.}}\\

Mit dem Vektor\footnote{Im Vektor $\vek K_G\,\vek e_l$ werden die Eintr\"{a}ge gestrichen, die zu gesperrten FG geh\"{o}ren.} $\vek g = \vek K^{-1}\,\vek j = -\vek K^{-1}\,\vek K_G\,\vek e_l$ lautet also die Einflussfunktion f\"{u}r die Lager\-kraft im Knoten $x$ (FG $u_l$)
\begin{align}
G_h(y,x) = \sum_i\,g_i\,\Np_i(y) = \vek \Phi(y)^T\,\vek g = \vek \Phi(y)^T\,\vek K^{-1}\,\vek j  \,.
\end{align}
Dieser Einflussfunktion fehlt aber offensichtlich, s. Abb. \ref{U106} c, das St\"{u}ck $\Np_l$ direkt unter  dem Lager. Das muss aber so sein, weil der Ansatzraum $\mathcal{V}_h$ ja die Funktion $\Np_l$ nicht enth\"{a}lt -- der Knoten wird ja festgehalten. Aber warum kommen die finiten Elemente dann trotzdem auf die richtige Lagerkraft? Das liegt daran, wie FE-Programme vorgehen.


Ist eine verteilte Last $p(x)$ gegeben, dann stellen sie in jeden Knoten die zugeh\"{o}rige \"{a}quivalente Knotenkraft und daher in den Lagerknoten die Kraft
\begin{align}
f_l = \int_0^{\,l} p(x)\,\Np_l(x)\,dx\,.
\end{align}
Diese Kraft wandert aber direkt in das Lager und beeinflusst somit die FE-Berechnung gar nicht. Alle Verformungen und alle Schnittkr\"{a}fte der FE-L\"{o}sung kommen aus dem Lastfall, bei dem dieser Anteil fehlt. Und die Einflussfunktion in Abb. \ref{U106} ist genau die Einflussfunktion  f\"{u}r die Lagerkraft in solchen \glq amputierten\grq{} Lastf\"{a}llen.
%-----------------------------------------------------------------
\begin{figure}[tbp]
\centering
\includegraphics[width=0.9\textwidth]{\Fpath/U194}
\caption{Seil, \textbf{ a)} System, \textbf{ b)} FE-L\"{o}sung \textbf{ c)} -   \textbf{ d)} lokale L\"{o}sungen, \textbf{ e)} exakte L\"{o}sung} \label{U194}
\end{figure}%
%-----------------------------------------------------------------

Die direkt in die St\"{u}tze flie{\ss}ende Knotenkraft $f_l$ erzeugt eine Gegenkraft in der St\"{u}tze, die wir $R_{d} = - f_l$ nennen.

Am Ende der Berechnung addiert das FE-Programm zu der Lagerkraft $R_{FE}$ des \glq amputierten\grq{} Lastfalls die Lagerkraft $R_{d}$ hinzu und so stimmt am Schluss wieder alles
\begin{align}
R = R_{FE} + R_{d}\,.
\end{align}
Man kann das ganze nat\"{u}rlich auch \glq von hinten\grq{} aufz\"{a}umen, indem man einfach zur FE-Einflussfunktion $G_h$ den fehlenden Anteil $\Np_l$ addiert, und so genau die Einflussfunktion $G$ erh\"{a}lt
\begin{align}
G = G_h + \Np_l\,,
\end{align}
wie sie der Ingenieur sehen will. Wenn man diese mit der Belastung \"{u}berlagert, ist das Ergebnis die volle Lagerkraft
\begin{align}
R = R_{FE} + R_d = \int_0^{\,l} G(y,x)\,p(y)\,dy\,.
\end{align}
In der Stabstatik geschieht die Addition $R = R_{FE} + R_{d}$ \glq automatisch\grq{}, n\"{a}mlich in dem Moment, in dem die lokalen L\"{o}sungen elementweise zur FE-L\"{o}sung addiert werden.

Das Seil in Abb. \ref{U194} a wird mit einer konstanten Streckenlast $p = 10$\,kN/m belastet. Eine Unterteilung in zwei lineare Elemente\footnote{Das ist die nicht-reduzierte globale Steifigkeitsmatrix $\vek K_G$.}
\begin{align}
        \left[ \barr {r @{\hspace{4mm}}r @{\hspace{4mm}}r}
      2 & -1 & 0  \\
      -1 & 2 & -1 \\
      0 & -1 & 2  \\
      \earr \right]\,\left[ \barr {r} 0 \\ 5 \\ 0 \earr \right] = \left[ \barr {r} f_1 \\ 10 \\ f_3 \earr \right]
      \end{align}
hat das Ergebnis $u_2 = 5$ und damit erh\"{a}lt man in der Nachlaufrechnung als Lagerkr\"{a}fte $f_1 = f_3 = - 5$. Dazu m\"{u}sste man noch die Lagerkr\"{a}fte aus der direkten Reduktion addieren. Das geschieht aber automatisch wie folgt: Elementweise werden die lokalen L\"{o}sungen bestimmt, s. Abb. \ref{U194} c und d, und zur FE-L\"{o}sung addiert. Die Folge ist, dass der Seildurchhang $w = w_{FE} + w_{loc}$ jetzt sch\"{o}n rund ist, und die Lagerkr\"{a}fte genau \glq passen\grq{}.

%%%%%%%%%%%%%%%%%%%%%%%%%%%%%%%%%%%%%%%%%%%%%%%%%%%%%%%%%%%%%%%%%%%%%%%%%%%%%%%%%%%%%%%%%%%%%%%%%%%
{\textcolor{sectionTitleBlue}{\section{Einflussfunktion f\"{u}r ein nachgiebiges Lager}}}
Der Fusspunkt eines elastischen Lagers ist ein fester Punkt, ein festes Lager und so kann man die Einflussfunktion f\"{u}r die Fusspunktskraft mit der obigen Technik berechnen, s. Abb. \ref{U179}.

Einfacher ist es nat\"{u}rlich, wenn man in den Kopf des nachgiebigen Lagers eine Kraft $\bar{P} = 1$ stellt, die Reaktion $\bar{w}$ des Tragwerks darauf berechnet
\begin{align}
\bar{w} = \frac{1}{3\,EI/l^3 + k}
\end{align}
und diese Figur (= Einflussfunktion f\"{u}r die Zusammendr\"{u}ckung der Feder) mit der Steifigkeit $k$ der Feder multipliziert, s. Abb. \ref{U475} a,
\begin{align}
F = k \int_{0}^{l} G_0(y,l)\,p(y)\,dy\,.
\end{align}

%-----------------------------------------------------------------
\begin{figure}[tbp]
\centering
\includegraphics[width=0.9\textwidth]{\Fpath/U179}
\caption{Gelenkig gelagerte Platte mit Innenst\"{u}tzen, \textbf{ a)} System \textbf{ b)} Einflussfunktion f\"{u}r die vordere, linke St\"{u}tzenkraft. Von der angesetzten Spreizung von 1000 mm werden $\sim 200$\,mm von der St\"{u}tze \glq verschluckt\grq{}, also rund $80\, \% \doteq 800 \, \text{mm}/1000\,\text{mm}$ einer Punktlast gehen direkt in die St\"{u}tze darunter. Die restlichen $20 \%$ tr\"{a}gt die Platte} \label{U179}
\end{figure}%
%-----------------------------------------------------------------

%-----------------------------------------------------------------
\begin{figure}[tbp]
\centering
\includegraphics[width=0.8\textwidth]{\Fpath/U475}
\caption{Elastische Einspannung, $k_\Np$ ist die Steifigkeit der Drehfeder, die Steifigkeit der Einspannung ist $3\,EI/l + k_\Np$, \textbf{ a)} Einflussfunktion $G_1$ f\"{u}r die Verdrehung der Einspannung \textbf{ b)} Momentenverteilung aus der Belastung} \label{U475}
\end{figure}%
%-----------------------------------------------------------------
Sinngem\"{a}{\ss} dasselbe gilt f\"{u}r Drehfedern, s. Abb. \ref{U475} b. Man verdreht die Einspannung mit einem Moment $\bar{M} = 1$, arbeitet also gegen die Drehsteifigkeit des Balkens plus der Drehsteifigkeit $k_\Np$ der Drehfeder
\begin{align}
\tan\,\bar{\Np} = \frac{1}{3\,EI/l + k_\Np}
\end{align}
und diese Biegelinie $G_1(y,l)$, gewichtet mit $k_\Np$, ist die Einflussfunktion f\"{u}r das Moment in der Feder aus einer Streckenlast $p$
\begin{align}
M = k_\Np \int_0^{\,l} G_1(y,l)\,p(y)\,dy\,.
\end{align}

%-----------------------------------------------------------------
\begin{figure}[tbp]
\centering
\includegraphics[width=0.7\textwidth]{\Fpath/U422}
\caption{Innenwand, \textbf{ a)} Lagerkr\"{a}fte, \textbf{ b)} Einflussfunktion f\"{u}r die Knotenkraft (= \glq halbe Lagerkraft\grq{}) am Wandende; sie wird ausgel\"{o}st durch eine Absenkung des Elements} \label{U422}
\end{figure}%
%-----------------------------------------------------------------

%%%%%%%%%%%%%%%%%%%%%%%%%%%%%%%%%%%%%%%%%%%%%%%%%%%%%%%%%%%%%%%%%%%%%%%%%%%%%%%%%%%%%%%%%%%%%%%%%%%
{\textcolor{sectionTitleBlue}{\section{Wandknoten}}\label{Korrektur31}
Mit den Knotenkr\"{a}ften an den Enden der W\"{a}nde\index{Wandknoten} werden die Durchstanznachweise\index{Durchstanznachweise} gef\"{u}hrt. Wenn man den Praktiker fragt, wie denn die Einflussfunktion f\"{u}r eine solche Knotenkraft aussieht, dann wird er sagen: \glq Man muss den Knoten um 1 m absenken\grq{}.

Das ist richtig, aber wir wollen doch die Situation etwas genauer betrachten. Statt nach der Knotenkraft fragen wir nach dem Wert der \glq halben\grq{} Lagerkraft im letzten Element der Wand, also dem Integral
\begin{align}
J(w)= \int_0^{\,l_e} \frac{x}{l_e}\,l(x)\,dx  \qquad l(x) = \text{Lagerkraft}\,.
\end{align}
Die Funktion $x/l_e$, die von $0$ auf den Endwert $1$ ansteigt, ist die Wichtungsfunktion. Wir stellen uns das so vor, dass wir das letzte Element entfernen und die Platte im Bereich des (entfernten) Elementes so belasten, dass die Biegefl\"{a}che dort diesen linearen Verlauf aufweist. Diese Biegefl\"{a}che $G(\vek x)$ ist die Einflussfunktion.  Aus dem Satz von Betti
\begin{align}
A_{1,2} = \int_{\Omega} G(\vek x)\,p(\vek x)\,d\Omega - \int_0^{\,l_e} \frac{x}{l_e}\,l(x)\,dx = A_{2,1} = 0
\end{align}
ergibt sich f\"{u}r die \glq halbe\grq{} Lagerkraft der Wert
\begin{align}
J(w) = \int_{\Omega} G(\vek x)\,p(\vek x)\,d\Omega\,.
\end{align}
Die Arbeit $A_{2,1} = 0 $ ist null, weil die Linienkr\"{a}fte $\delta^{@l}$, die die Platte l\"{a}ngs des Elements nach unten dr\"{u}cken, um die Biegefl\"{a}che $G(\vek x)$ zu erzeugen, auf den Wegen $w = 0$ des Wandelementes keine Arbeit leisten.

Gem\"{a}{\ss} {\em Betti extended\/} lautet das FE-Ergebnis
\begin{align}\label{Eq138}
J(w_h)= \int_{\Omega} G_h(\vek x)\,p(\vek x)\,d\Omega\,,
\end{align}
wenn $G_h$ die FE-Biegefl\"{a}che ist, die l\"{a}ngs des Elementes den Verlauf $x/l_e $ hat.  Diese Biegefl\"{a}che wird durch die Absenkung des Wandendes um 1 m erzeugt. Das ist die Einflussfl\"{a}che des Praktikers.

Das FE-Ergebnis (\ref{Eq138}) kann man im \"{u}brigen, wie wir kurz zeigen wollen, direkt aus dem Satz von Betti
\begin{align}
A_{1,2} = \int_{\Omega} G_h(\vek x)\,p(\vek x)\,d\Omega - \int_0^{\,l_e} \frac{x}{l_e}\,l(x)\,dx = A_{2,1}  = 0
\end{align}
herleiten.

Zun\"{a}chst bemerken wir, dass wir f\"{u}r $p$ auch $p_h$ in dem obigen Integral setzen k\"{o}nnen. Mit dem {\em switch\/} $p \to p_h$ ist die Arbeit $A_{2,1}$ das Integral
\begin{align}
A_{2,1} = \int_{\Omega} \delta_h^{@l}(\vek x)\,w_h(\vek x)\,d\Omega\,,
\end{align}
wenn $\delta_h^l(\vek x)$ die Last ist, die die N\"{a}herung $G_h(\vek x)$ mit dem $x/l_e$ l\"{a}ngs des Elements erzeugt. Nun ist $w_h$ l\"{a}ngs des Elements null und somit ist es auch das Integral.

Wenn man dasselbe mit einem Innenknoten der Wand macht, dann erzeugt die Absenkung des Knotens nat\"{u}rlich die Einflussfunktion f\"{u}r die Summe der beiden \glq halben\grq{} Wandkr\"{a}fte links und rechts vom Knoten.

%-----------------------------------------------------------------
\begin{figure}[tbp]
\centering
\includegraphics[width=1.0\textwidth]{\Fpath/U311}
\caption{Punktgest\"{u}tzte Platte, LF g, \textbf{ a)} Biegefl\"{a}che, \textbf{ b)} Lagerkr\"{a}fte} \label{U311}
\end{figure}%
%-----------------------------------------------------------------
%-----------------------------------------------------------------
\begin{figure}[tbp]
\centering
\includegraphics[width=1.0\textwidth]{\Fpath/U312}
\caption{W\"{a}nde unter einer Hochbaudecke, auch diese Lagerkr\"{a}fte kann ein FE-Programm relativ genau ermitteln} \label{U312} % Position 503
\end{figure}%
%-----------------------------------------------------------------

%%%%%%%%%%%%%%%%%%%%%%%%%%%%%%%%%%%%%%%%%%%%%%%%%%%%%%%%%%%%%%%%%%%%%%%%%%%%%%%%%%%%%%%%%%%%%%%%%%%
{\textcolor{sectionTitleBlue}{\section{Genauigkeit der Lagerkr\"{a}fte}}
Die Einflussfunktionen f\"{u}r St\"{u}tzen sind einfache Senken in der Platte, die durch eine Spreizung der Gr\"{o}{\ss}e Eins zwischen St\"{u}tzenkopf und Platte ausgel\"{o}st werden. Sie \"{a}hneln vom Typ her den Einflussfunktionen f\"{u}r Durchbie\-gungen und sie sind daher auch schon auf relativ groben Netzen gut anzun\"{a}hern. Das ist der Grund, warum man bei einer FE-Berechnung keine gro{\ss}en Zweifel an der H\"{o}he der ausgewiesenen St\"{u}tzenkr\"{a}fte haben muss, s. Abb. \ref{U311}. Das gilt sinngem\"{a}{\ss} auch f\"{u}r W\"{a}nde, s. Abb. \ref{U312} und auch f\"{u}r Unterz\"{u}ge, obwohl da in der Praxis zum Teil heftig diskutiert wird, was die Modellierung von Unterz\"{u}gen angeht, s. Abb. \ref{U308}.

%%%%%%%%%%%%%%%%%%%%%%%%%%%%%%%%%%%%%%%%%%%%%%%%%%%%%%%%%%%%%%%%%%%%%%%%%%%%%%%%%%%%%%%%%%%%%%%%%%%
{\textcolor{sectionTitleBlue}{\section{Punktlasten und Punktlager bei Scheiben}}
Punktkr\"{a}fte bei Scheiben f\"{u}hren zu unendlich gro{\ss}en Spannungen und so k\"{o}nnen Scheiben auch nicht auf Punktlager gesetzt werden, weil das Material einfach zu flie{\ss}en anfangen w\"{u}rde. Andererseits erh\"{a}lt man aber mit finiten Elementen doch sinnvolle Ergebnisse in Punktlagern, s. Abb. \ref{U84}. Wie geht das?

Es geht, weil auch die finiten Elemente einen geb\"{u}hrenden Abstand von echten Punktkr\"{a}ften einhalten. Im Ausdruck stehen zwar \"{a}quivalente Knotenkr\"{a}fte $f_i$, aber sie sind ja nur Stellvertreter f\"{u}r die wahren St\"{u}tzkr\"{a}fte, die als Fl\"{a}chen- und Linienkr\"{a}fte in der Umgebung des Knotens die Scheibe halten.
%-----------------------------------------------------------------
\begin{figure}[tbp]
\centering
\includegraphics[width=0.8\textwidth]{\Fpath/U84}
\caption{Einflussfunktion f\"{u}r die FE-Lagerkraft in dem Punktlager einer Scheibe} \label{U84}
\end{figure}%
%-----------------------------------------------------------------

%---------------------------------------------------------------------------------
\begin{figure}
\centering
\if \bild 2 \sidecaption[t] \fi
{\includegraphics[width=0.9\textwidth]{\Fpath/U109}}
\caption{Original und FE-Netz mit \"{a}quivalenten Knotenkr\"{a}ften}
\label{U109}%
\end{figure}%
%---------------------------------------------------------------------------------

Auf der Lastseite gilt dasselbe. Wenn der Anwender eine Knotenkraft eingibt, dann behilft sich ein FE-Programm damit eine Schar von \"{a}quivalenten Fl\"{a}chen- und Linienkr\"{a}ften in die N\"{a}he des Knotens zu plazieren.

Dies ist die Stelle, wo man sieht, wie die finiten Elemente ihr Eigenleben entwickeln. Formal sind sie nur ein numerisches Werkzeug, aber der Ingenieur findet gar nichts dabei, die Knotenkr\"{a}fte f\"{u}r real zu nehmen und so kommt er mit Hilfe der finiten Elemente sehr elegant \"{u}ber die Stolpersteine hinweg, die ihm die Elastizit\"{a}tstheorie in den Weg legt.

%---------------------------------------------------------------------------------
\begin{figure}
\centering
\if \bild 2 \sidecaption[t] \fi
{\includegraphics[width=1.0\textwidth]{\Fpath/U281}}
\caption{Wandscheibe auf Punktlagern (Ausschnitt)}  % Pos. NE2
\label{U281}%
\end{figure}%
%---------------------------------------------------------------------------------

So ist das statische Problem der Wandscheibe in Abb. \ref{U28}, die sich auf zwei Punktlager st\"{u}tzt, vom  Standpunkt der Mathematik aus ein schlecht gestelltes Problem, weil die Spannungen und Verformungen der Scheibe in den Lagerpunkten gem\"{a}{\ss} der Elastizit\"{a}tstheorie unendlich gro{\ss} werden und echte Punktlager die Scheibe auch nicht festhalten k\"{o}nnten.

Mit finiten Elementen ist man jedoch noch ein gutes St\"{u}ck von dieser Grenze entfernt. Zudem ist
die Scheibe statisch bestimmt gelagert und so ist es nur nat\"{u}rlich, die $f_i$ wie echte Kr\"{a}fte zu behandeln.

Viele Ingenieure interpretieren das FE-Modell einer Scheibe ja als eine Art \glq Fachwerk\grq{}\index{Fachwerkmodell}, wo kleine Scheibenelemente in den Knoten miteinander verbunden sind und wo das Gleichgewicht in den Knoten, $\vek K\,\vek u = \vek f$, die Gleichgewichtslage $\vek u$ bestimmt, s. Abb. \ref{U109}. Das ist aber eine Interpretation \glq als ob\grq{}, schon deswegen, weil auf beiden Seiten von $\vek K\,\vek u = \vek f$ Arbeiten stehen und keine Kr\"{a}fte; es werden Arbeiten bilanziert.

Das (scheinbar) merkw\"{u}rdige ist, dass man f\"{u}r diese \glq fiktiven\grq{} \"{a}quivalenten Knotenkr\"{a}fte $f_i$ Einflussfunktionen aufstellen kann. Einfach so, wie man das auch bei einem Fachwerk machen w\"{u}rde: man verschiebt den Lagerknoten um einen Meter und bilanziert so die Arbeiten, die von der Belastung und der Lagerkraft bei einer Verr\"{u}ckung des Lagers geleistet werden.

%-----------------------------------------------------------------
\begin{figure}[tbp]
\centering
\includegraphics[width=1.0\textwidth]{\Fpath/U308}
\caption{Hochbauplatte, \textbf{ a)} Unterkonstruktion, \textbf{ b)} Hauptmomente im LF $g$} \label{U308}
\end{figure}%
%-----------------------------------------------------------------

%%%%%%%%%%%%%%%%%%%%%%%%%%%%%%%%%%%%%%%%%%%%%%%%%%%%%%%%%%%%%%%%%%%%%%%%%%%%%%%%%%%%%%%%%%%%%%%%%%%
{\textcolor{sectionTitleBlue}{\section{Punktlager sind hot spots}}}\index{hot spots}\label{Punktlager}
Wenn man einen Knoten festh\"{a}lt, dann wird die Scheibe dort praktisch \glq geerdet\grq{}. Das ist so, als ob man mit der einen Hand eine Hochspannungsleitung ber\"{u}hrt und mit der anderen die Erde. Der steile Anstieg der Verschiebungen vom festen Lager zu den freien Knoten produziert gro{\ss}e Spannungen in den Elementen, die mit dem festen Knoten verbunden sind, s. Abb. \ref{U281}.

Je kleiner die Elemente in der N\"{a}he des Festpunktes werden, um so steiler ist der Verschiebungsgradient in den Elementen und um so gr\"{o}{\ss}er sind somit auch die Spannungen in den Elementen.
%---------------------------------------------------------------------------------
\begin{figure}
\centering
\if \bild 2 \sidecaption[t] \fi
{\includegraphics[width=0.4\textwidth]{\Fpath/U20}}
\caption{Durch das letzte Element vor dem Punktlager muss die ganze Lagerkraft flie{\ss}en...}
\label{U20}%
\end{figure}%
%---------------------------------------------------------------------------------

Warum die Spannungen unendlich gro{\ss} werden, ja unendlich gro{\ss} werden m\"{u}ssen, versteht man, wenn man sich die finiten Elemente anschaut.
%---------------------------------------------------------------------------------
\begin{figure}
\centering
\if \bild 2 \sidecaption[t] \fi
{\includegraphics[width=0.9\textwidth]{\Fpath/U337}}
\caption{Einflussfunktion f\"{u}r die Querkraft $q_x$, \textbf{ a)} die Scherbewegung kann sich frei ausbilden, \textbf{ b)} die starre St\"{u}tze behindert die Scherbewegung, die weiter abliegenden Knotenkr\"{a}fte sind im \"{U}bergewicht und treiben so die Platte nach oben---auch in Bereichen die vom Aufpunkt weiter weg entfernt liegen}
\label{U337}%
\end{figure}%
%---------------------------------------------------------------------------------

%---------------------------------------------------------------------------------
\begin{figure}
\centering
\if \bild 2 \sidecaption[t] \fi
{\includegraphics[width=0.9\textwidth]{\Fpath/U350}}
\caption{Dieselbe Platte wie in Abb. \ref{U337}; die Einflussfunktion f\"{u}r das Biegemoment $m_{xx}$ (im selben Punkt) zeigt dasselbe Verhalten, aber links und rechts in \textbf{ a)} und \textbf{ b)} sind ausgewogener, weil die Einflussfunktion durch eine symmetrische Quelle (ein Quattropol) erzeugt wird}
\label{U350}%
\end{figure}%
%---------------------------------------------------------------------------------

Angenommen in dem Punktlager wirkt eine vertikale Kraft $f_i = 10 $ kN. Wenn man also den Lagerknoten um einen Meter nach oben dr\"{u}ckt (das ist rein rechnerisch), dann leistet die Knotenkraft dabei die Arbeit $\delta A_a =10$ kNm, s. Abb. \ref{U20}.
%---------------------------------------------------------------------------------
\begin{figure}
\centering
\if \bild 2 \sidecaption[t] \fi
{\includegraphics[width=0.65\textwidth]{\Fpath/U294}}
\caption{Generierung der Einflussfunktion f\"{u}r $\sigma_{yy}$, \textbf{ a)} die Knotenkr\"{a}fte, die die Spreizung (n\"{a}herungsweise) erzeugen sind jeweils in allen vier Knoten gleich und h\"{a}ngen nur von der Maschenweite $h$ ab, \textbf{ b)} je kleiner die Elemente werden, um so gr\"{o}{\ss}er werden die Knotenkr\"{a}fte und damit die Verformung der Scheibe. Das Verh\"{a}ltnis zwischen den Kr\"{a}ften steht $2:1$, weil das feste Lager eine Kraft neutralisiert (\glq amputierter Dipol\grq{})}
\label{U294}%
\end{figure}%
%---------------------------------------------------------------------------------

%---------------------------------------------------------------------------------
\begin{figure}
\centering
%\if \bild 2 \sidecaption[t] \fi
{\includegraphics[width=0.60\textwidth]{\Fpath/U295}}
\caption{Generierung der Einflussfunktionen f\"{u}r $\sigma_{yy}$, \textbf{ a)} und \textbf{ b)} bei der Spreizung der Nachbarelemente des Lagerelementes dr\"{u}cken zwei Kr\"{a}fte nach oben und zwei Kr\"{a}fte nach unten und so bleiben die  Verschiebungen (= Spannungen als Einflussfunktion) auch in der Grenze, $h \to 0$, endlich (\glq echter Dipol\grq{})}
\label{U295}%
\end{figure}%
%---------------------------------------------------------------------------------
Die Bewegung des Lagerknotens teilt sich dem Element $\Omega_e$ mit, auf dem das Lager liegt, und so muss die virtuelle innere Energie $\delta A_i$ in dem Element gleich $\delta A_a$ sein
\begin{align}
\delta A_a = 1\cdot f_i = \int_{\Omega_e} \sigma_{ij}\,\delta \varepsilon_{ij}\,d\Omega = \delta A_i\,.
\end{align}
Die Verzerrungen $\delta \varepsilon_{ij}$ resultieren dabei aus der Lagerbewegung $u_i = 1$.

Alle anderen Elemente sp\"{u}ren nichts davon, weil alle anderen Knoten bei dem Man\"{o}ver festgehalten werden. \\

\hspace*{-12pt}\colorbox{highlightBlue}{\parbox{0.98\textwidth}{Dieses letzte vor dem Lager liegende Element muss also ganz allein die n\"{o}tige Energie aufbringen, um die Lagerarbeit ins gleiche zu setzen!}}\\

Wenn nun das Element immer kleiner wird, weil man ja genaue Ergebnisse haben will..., dann m\"{u}ssen die Spannungen in dem Element immer mehr anwachsen, weil immer weniger Fl\"{a}che vorhanden ist, \"{u}ber die man integrieren kann, und so hat man keine Chance irgend etwas vern\"{u}nftiges zu berechnen. Man muss dann umschalten und in \"{a}quivalenten Knotenkr\"{a}ften denken.

%%%%%%%%%%%%%%%%%%%%%%%%%%%%%%%%%%%%%%%%%%%%%%%%%%%%%%%%%%%%%%%%%%%%%%%%%%%%%%%%%%%%%%%%%%%%%%%%%%%
{\textcolor{sectionTitleBlue}{\section{Der amputierte Dipol}}}\index{amputierter Dipol}\label{Korrektur22}
Um die Singularit\"{a}t in Punktlagern besser zu verstehen, betrachten wir die Einflussfunktion f\"{u}r die Querkraft $q_x$ in einer Platte, s. Abb. \ref{U337}, die ja durch eine Scherbewegung entsteht. Wenn der Aufpunkt frei im Innern liegt, dann halten sich die beiden Scherbewegungen die Balance und die Auslenkung der Platte bleibt, bis auf den Aufpunkt selbst, beschr\"{a}nkt. R\"{u}ckt jedoch der Aufpunkt in die N\"{a}he eines Punktlagers, dann wird diese Balance durch die St\"{u}tze gest\"{o}rt, und die Folge davon ist, dass bei Ann\"{a}herung an die St\"{u}tze die Auslenkung der Platte \"{u}ber alle Grenzen w\"{a}chst, s. Abb. \ref{U337} b.

Tendenziell sieht man den Effekt auch bei der \"{u}ber den Tr\"{a}ger wandernden Einflussfunktion f\"{u}r die Querkraft in Abb. \ref{1GreenF74}, S. \pageref{1GreenF74}. Je mehr der Aufpunkt sich dem Zwischenlager n\"{a}hert, desto mehr wird die Einflussfunktion nach oben gedr\"{u}ckt, ger\"{a}t die Balance zwischen links und rechts aus dem Lot und am Ende m\"{u}ndet die Scherbewegung in einem einseitigen Versatz, einem einseitigen {\em uplift\/}. Nur ist es so, dass bei der Platte die Spitze dieses {\em uplifts\/} unendlich weit \"{u}ber der St\"{u}tze liegt.

Diese Argumentation k\"{o}nnen wir nun direkt auf die Scheibe \"{u}bertragen, denn die Einflussfunktion f\"{u}r die Spannungen $\sigma_{ij}$ werden auch durch solche Scherbewegungen erzeugt, s. Abb. \ref{U294}.  Numerisch sind es vier Kr\"{a}fte, die dort die Einflussfunktion f\"{u}r $\sigma_{yy}$ erzeugen.

Liegt der Aufpunkt in dem Element mit dem Lagerknoten, dann steht es 2:1 f\"{u}r die nach oben treibenden Kr\"{a}fte, d.h. {\em zwei\/} Knotenkr\"{a}fte dr\"{u}cken nach oben, aber nur {\em eine\/} Knotenkraft dr\"{u}ckt nach unten, weil die Knotenkraft im Lager ausf\"{a}llt. So gelingt es also den $f_i$ die Oberkante der Scheibe in der Grenze in \glq den Himmel\grq{} zu verschieben. In den Nachbarelementen wirken hingegen,  s. Abb. \ref{U295}, alle {\em vier = zwei + zwei\/} Kr\"{a}fte gleichzeitig und halten so die Balance mit der Konsequenz, dass die Auslenkung (= Spannung) endlich bleibt.

Um die Tendenz $\sigma_{ij} \to \infty$ auch statisch zu verstehen, denken wir uns der Einfachheit halber das Element als eine kleine Kreisscheibe mit einem Radius $R$. Die Elementverzerrungen aus der Verschiebung des Lagerknotens verhalten sich wie
\begin{align}
\delta \varepsilon_{ij} = O( \frac{1}{R})\,.
\end{align}
(Macht man den Durchmesser $2\,R$ eines Zeltes kleiner, beh\"{a}lt aber die H\"{o}he 1 bei, dann wird die Neigung der Zeltbahn gr\"{o}{\ss}er).
%---------------------------------------------------------------------------------
\begin{figure}
\centering
\if \bild 2 \sidecaption[t] \fi
{\includegraphics[width=1.0\textwidth]{\Fpath/U86}}
\caption{Je kleiner die vier Elemente um die Kraft herum werden, um so gr\"{o}{\ss}er m\"{u}ssen die Spannungen werden, um dieselbe Knotenkraft $f_i$ auf schrumpfender Fl\"{a}che zu erzeugen}
\label{U86}%
\end{figure}%
%---------------------------------------------------------------------------------
Sinngem\"{a}{\ss} gilt daher
\begin{align}
 \int_{\Omega_e} \sigma_{ij}\,\delta \varepsilon_{ij}\,d\Omega \sim\int_0^{\,2\,\pi} \int_0^{\,R}   \sigma_{ij}\,\frac{1}{R}\,r\,dr\,d\Np = \int_0^{\,2\,\pi} \sigma_{ij} \,\frac{1}{2}\,R\,d\Np\,,
\end{align}
und daher muss sich $\sigma_{ij}$ wie $1/R$ verhalten, damit in der Grenze, $R \to 0$, die Knotenkraft $f_i$ \"{u}brig bleibt
\begin{align}
\lim_{R \to 0} \int_{\Omega_e} \sigma_{ij}\,\delta \varepsilon_{ij}\,d\Omega = f_i\,.
\end{align}

\begin{remark}
Der Vollst\"{a}ndigkeit halber sei noch erw\"{a}hnt, dass Linien\-lager\index{Linienlager $3-D$} im $3-D$ dasselbe Schicksal erleiden, wie Punktlager bei Scheiben, weil die Spannungen in einem Linienlager (= unendlich feiner Draht) das Material zum Flie{\ss}en bringen. Aus diesem Grund kann man theoretisch auch kein Linienlager um ein oder zwei Zentimeter senken. In der Praxis geht es nat\"{u}rlich schon, weil die finiten Elementen weit weg sind von solchen Fallstricken der Elastizit\"{a}tstheorie.
\end{remark}

%---------------------------------------------------------------------------------
\begin{figure}
\centering
\if \bild 2 \sidecaption[t] \fi
{\includegraphics[width=1.0\textwidth]{\Fpath/U417}}
\caption{LF Eigengewicht: Die adaptive Verfeinerung in der N\"{a}he der Punktlager m\"{u}ndet im Grenzfall $h \to 0$ theoretisch in Einzelkr\"{a}ften. Damit w\"{u}rde die Scheibe ihren Halt verlieren, weil die unendlich gro{\ss}en Spannungen das Material zum Flie{\ss}en bringen w\"{u}rden}
\label{U417}%
\end{figure}%
%---------------------------------------------------------------------------------
\vspace{-1cm}
%%%%%%%%%%%%%%%%%%%%%%%%%%%%%%%%%%%%%%%%%%%%%%%%%%%%%%%%%%%%%%%%%%%%%%%%%%%%%%%%%%%%%%%%%%%%%%%%%%%
{\textcolor{sectionTitleBlue}{\section{Einzelkr\"{a}fte als Knotenkr\"{a}fte}}}\index{Einzelkr\"{a}fte als Knotenkr\"{a}fte}
Eine Einzelkraft $ P$ \"{u}bersetzt das FE-Programm in eine \"{a}quivalente Knotenkraft $f_i = P \cdot 1$\,[F$\cdot$ L]. Eine solche \"{a}quivalente Knotenkraft $f_i$ ist f\"{u}r ein FE-Programm zun\"{a}chst nur ein Signal, dass in der N\"{a}he des Knotens Lasten vorhanden sind, die bei einer virtuellen Verr\"{u}ckung des Knotens die Arbeit $f_i $ leisten, s. Abb. \ref{U86}. Sollten die $f_i$ in den Nachbarknoten null sein, dann kann es nur so sein, dass es wirklich eine Einzelkraft ist, die genau in dem Knoten angreift.

Wenn nun der Benutzer die Elemente immer kleiner macht und weiterhin darauf beharrt, dass in dem Knoten -- und nur in diesem einen Knoten -- eine \"{a}quivalente Knotenkraft $f_i $ steht, dann ist das f\"{u}r das FE-Programm ein Signal, dass es die Spannungen in der N\"{a}he dieses Knotens tendenziell unendlich gro{\ss} machen muss, weil sonst die Bilanz
\begin{align}
 \int_{\Omega_\square} \sigma_{ij}\,\delta \varepsilon_{ij} \,d\Omega = f_i
\end{align}
nicht einzuhalten ist, s. Abb. \ref{U417}.

Die Verzerrungen $\delta \varepsilon_{ij}$ kommen aus der Einheitsverschiebung des Knotens in Richtung der Kraft und weil sie nur auf den Elementen nicht null sind, auf denen der Knoten liegt, muss immer weniger Gebiet immer gr\"{o}{\ss}ere Spannungen $\sigma_{ij}$ und Verzerrungen $\delta \varepsilon_{ij}$ produzieren. Dass {\em beide\/} gegen $\infty$ gehen m\"{u}ssen, haben wir oben gezeigt. Auch so kommen singul\"{a}re Spannungen in die finiten Elemente hinein. Sie sind in dieser Situation eine \glq Schutzma{\ss}nahme\grq{}, um bei immer kleiner werdenden Elementen am Ziel $f_i$ festhalten zu k\"{o}nnen.

Verfeinert man ein Netz in Gegenwart von Fl\"{a}chenkr\"{a}ften, dann werden die $f_i$ kleiner und  eine Verfeinerung ist daher unsch\"{a}dlich. Hier ist es aber so, dass $f_i$ seine H\"{o}he {\em beibeh\"{a}lt\/} und so erzwingt $h \to 0$ die Reaktion $\sigma \to \infty$.

%%%%%%%%%%%%%%%%%%%%%%%%%%%%%%%%%%%%%%%%%%%%%%%%%%%%%%%%%%%%%%%%%%%%%%%%%%%%%%%%%%%%%%%%%%%%%%%%%%%
\textcolor{sectionTitleBlue}{\section{Vorverformungen}}\index{Vorverformungen}\label{Korrektur42}
Wir wollen auch noch kurz erl\"{a}utern, wie Vorverformungen bei St\"{a}ben, die nach Theorie II. Ordnung gerechnet werden, im Rahmen der FEM behandelt werden.

 \glq {\em Bei Ansatz von Vorverformungen bedeutet Theorie II. Ordnung die Formulierung des Gleichgewichts am {\em gesamtverformten\/} System, wobei Gesamtverformung = Vorverformung + Lastverformung ist'\/}, \cite{Rubin} S. 77. Wir schreiben das als
\begin{align}
w(x) = s(x) + w_{L}(x) \qquad s\,\text{wie \glq Schlangenlinie\grq{}}
\end{align}
 Die so zweigeteilte Biegelinie muss der Differentialgleichung
\begin{align}
EI\,w^{IV}(x) + P\,w''(x) = p(x) \qquad  P = |P| \,\,\text{als Druckkraft}
\end{align}
oder
\begin{align}
EI\,w_{L}^{IV}(x) + P\,w_{L}''(x) = p(x) - (EI\,s^{IV}(x) + P\,s''(x))
\end{align}
gen\"{u}gen. Die Vorverformung wird durch ihre Interpolierende $s_I$ ersetzt
\begin{align}\label{Eq86}
s_I(x) = \sum_j s_j\,\Np_j(x)\,,
\end{align}
wobei die $s_j$ die Knotenwerte von $s(x)$ sind (Durchbiegungen und Verdre\-hungen -- Hermite Interpolation!). Weil die {\em shape functions\/} elementweise homogene L\"{o}sungen sind, $EI\,\Np_i^{IV} = 0$, reduziert sich das in jedem Element auf
\begin{align}
EI\,w_{L}^{IV}(x) + P\,w_{L}''(x) = p(x) - P\,s_{I}''(x)\,.
\end{align}
Zu den \"{a}quivalenten Knotenkr\"{a}ften aus der Last sind also noch die \"{a}quivalenten Knotenkr\"{a}fte der Interpolierenden zu addieren. Auf jedem Element gilt (partielle Integration)
\begin{align}
\text{\normalfont\calligra G\,\,}_e(s_I,\Np_i) = \int_0^{\,l_e} -P\,s_I''\,\Np_i\,dx + [P\,s_I'\,\Np_i]_0^{l_e} - P \int_0^{\,l_e} s_I'\,\Np_i'\,dx = 0\,.
\end{align}
Nun hat die Hermite-Interpolierende $s_I$ stetige erste Ableitungen in den Knoten und die $\Np_i(x)$ sind stetig in den Knoten, so dass sich bei der Summation \"{u}ber die Elemente
die Randarbeiten in den Innenknoten wegheben\footnote{An freien Enden verbleibt ein zus\"{a}tzlicher Anteil $P\,s_I' \cdot \Np_i$ der zur Knotenkraft $f_{si}$ zu addieren ist}
\begin{align}
\sum_e \text{\normalfont\calligra G\,\,}_e(s_I,\Np_i) = \int_0^{\,l} -P\,s_I''\,\Np_i\,dx  - P \int_0^{\,l} s_I'\,\Np_i'\,dx = 0\,,
\end{align}
was mit (\ref{Eq86}) gerade das Resultat
\begin{align}
f_{si} - \sum_j P \int_0^{\,l} \Np_i'\,\Np_j'\,dx\,s_j =  f_{si} - \sum_j\,G_{ij}\,s_j = 0\qquad G_{ij} = P\,\int_0^{\,l} \Np_i'\,\Np_j'\,dx
\end{align}
ergibt. Man erh\"{a}lt also den Vektor der \"{a}quivalenten Knotenkr\"{a}fte aus der Vorverformung, wenn man die {\em geometrische Steifigkeitsmatrix\/}, (\ref{IIN}) S. \pageref{IIN}, mit den Knotenwerten $\vek s$ der Vorverformung multipliziert
\begin{align}
\vek f_s = \vek G\,\vek s\,,
\end{align}
und mit Vorverformungen l\"{o}st man das System $\vek K_{II}\,\vek u = \vek f + \vek f_s$. Hierbei ist $\vek K_{II}$ die Steifigkeitsmatrix nach Theorie II. Ordnung, die aber meist durch die Steifigkeitsmatrix $\vek K_I$ nach Theorie I. Ordnung und die geometrische Steifigkeitsmatrix $\vek G$ ersetzt wird
\begin{align}
\vek K_{II} \simeq \vek K_I + \vek G\,.
\end{align}


%%%%%%%%%%%%%%%%%%%%%%%%%%%%%%%%%%%%%%%%%%%%%%%%%%%%%%%%%%%%%%%%%%%%%%%%%%%%%%%%%%%%%%%%%%%%%%%%%%%
{\textcolor{sectionTitleBlue}{\section{Die Grenzen von FE-Einflussfunktionen}}}\index{Grenzen von FE-Einflussfunktionen}
Wenn wir die Einflussfunktion f\"{u}r die zweite Ableitung $u''(x)$ der L\"{a}ngsverschiebung eines Stabes aufstellen wollen, und mit der Berechnung der \"{a}quivalenten Knotenkr\"{a}ften $j_i$ beginnen, dann merken wir schnell, dass alle
\begin{align}
j_i = \Np_i''(x) = 0
\end{align}
null sind und wegen $\vek K\vek g = \vek 0$ daher auch die Knotenverschiebungen $g_i$ der Einflussfunktion, d.h. die FE-Einflussfunktion f\"{u}r die zweite Ableitung ist identisch null
\begin{align}
G_2^h(y,x ) = 0\,.
\end{align}
Das ist immer so. Man kann mit finiten Elementen Einflussfunktionen nur f\"{u}r Ableitungen bis zu der Ordnung der Ansatzfunktionen berechnen
\begin{align}
\text{Stab} \qquad \Np_i(x) &= \text{linear} &&\qquad \rightarrow \qquad \text{max $u'(x)$}\\
\text{Balken} \qquad \Np_i(x) &= \text{kubisch} &&\qquad \rightarrow \qquad \text{max $u'''(x)$}\,.
\end{align}
Die h\"{o}heren Ableitungen kennen die finiten Elemente nicht oder, anders gesagt, sie haben keine Vorstellung davon, dass es so etwas geben k\"{o}nnte.\\

Das stimmt im \"{u}brigen mit der $h$-{\em Vertauschungsregel\/}\index{$h$-Vertauschungsregel} \"{u}berein, s. S. \pageref{Eq56},
\begin{align}
J_h(u) = J(u_h)\,.
\end{align}
Denn sei $J(u) = u''(x)$, und $u_h$ eine lineare FE-L\"{o}sung, dann ist die zweite Ableitung null, $J(u_h) = 0$, und daher muss auch $J_h(u) = 0$ null sein -- f\"{u}r alle $u$, also etwa alle Polynome beliebig hoher Ordnung. Das geht nur so, dass $G_2^h(y,x)$ identisch null ist, was eben bedeutet, dass man eine Einflussfunktion f\"{u}r zweite Ableitungen mit linearen Elementen nicht darstellen kann.

%Man kann es auch anders sagen: Ein FE-Programm reduziert alle Lasten in die Knoten, es kennt also nur Knotenkr\"{a}fte und Knotenmomente. Im Feld gibt es keine Belastung, keine Streckenlast $p$ und deswegen ist die vierte Ableitung zwischen den Knoten null, $EI\,w_h^{IV} = 0$. Warum eine Einflussfunktion f\"{u}r $w_h^{IV}$ konstruieren, wenn man wei{\ss}, dass die Funktion null ist?

%Bei Fl\"{a}chentragwerken gilt sinngem\"{a}{\ss} dasselbe, wenn auch die Situation etwas komplizierter wird, wenn man h\"{o}here Ans\"{a}tze benutzt.

%Mit quadratischen Scheibenelementen kann man schon Einflussfunktionen f\"{u}r zweite Ableitungen berechnen, weil die zugeh\"{o}rigen \"{a}quivalenten Knotenkr\"{a}fte nicht null sind. Nur ist die Frage, was man mit diesen Einflussfunktionen will? Man k\"{o}nnte mit ihnen den FE-Lastfall $p_h$ punktweise berechnen, obwohl es viel einfacher w\"{a}re, direkt \"{u}ber die FE-L\"{o}sung $\vek u_h$ zu gehen.


%%%%%%%%%%%%%%%%%%%%%%%%%%%%%%%%%%%%%%%%%%%%%%%%%%%%%%%%%%%%%%%%%%%%%%%%%%%%%%%%%%%%%%%%%%%%%%%%%%%
{\textcolor{sectionTitleBlue}{\section{Bemerkung zu den finiten Elementen}}}
Die finiten Elemente haben eine beispiellose Erfolgsgeschichte hinter sich, und der Grund ist, wie wir meinen, dass das Konzept des finiten Elements in nat\"{u}rlicher Weise die Kr\"{a}fte freigelegt hat, die in dem Begriff der Funktion schlummern. Wir wollen hier einen Erkl\"{a}rungsversuch wagen.

Jede $C^1$-Funktion \"{u}ber einem Intervall $(0,l) $ besitzt die Integraldarstellung
\begin{align}
u(x) = u(0) + \int_{0}^{x}\,u'(y)\cdot 1\,dy = \int_{\Gamma} \ldots + \int_{\Omega} \ldots
\end{align}
-- das ist einfach nur partielle Integration -- und jede $C^2$-Funktion \"{u}ber einem Gebiet $\Omega$ besitzt die Integraldarstellung
\begin{align} \label{Eq192}
u(\vek x) = &\int_{\Gamma} [g(\vek y, \vek x) \,\frac{\partial u(\vek y)}{\partial n} - \frac{\partial g(\vek y, \vek x)}{\partial n}\,u(\vek y)]\,ds_{\vek y} + \int_{\Omega} g(\vek y, \vek x)\,(- \Delta u(\vek y))\,\,d\Omega_{\vek y}\,.
\end{align}
Der Kern $g(\vek y, \vek x) = -1/(2\,\pi) \ln |\vek x - \vek y|$ ist die Fundamentall\"{o}sung der Laplacegleichung, $- \Delta g = \delta(\vek y - \vek x)$. (Der Kern oben ist die {\em Heaviside-function\/} $g = 1$ bis zum Punkt $x$)\index{Heaviside-function}.

Eine $C^2$-Funktion in einem Gebiet $\Omega $ ist also eindeutig durch ihren Randwert $u$, ihre \glq Spur\grq, und die Normalableitung $\partial u/\partial n$ auf dem Rand $\Gamma$ und die \glq Last\grq\ im Feld $\Omega$, die Summe der zweiten Ableitungen $\Delta u = u,_{y_1 y_1} + u,_{y_2 y_2}$, bestimmt.

Der Kern, der diese Daten zum Leben erweckt, aus ihnen $u(\vek x)$ generiert, und somit auch bestimmt, wie sich $u(\vek x)$ \"{a}ndert, wenn man sich von einem Punkt $\vek x_1$ zu einem Punkt $\vek x_2$ bewegt, ist in der Ebene der $\ln r$ und im Raum der Abstand $r = |\vek y - \vek x|$ selbst vom Aufpunkt $\vek x$ zu den Daten in $\vek y$. (Woher der Unterschied?)

%Der Abstand $r = |\vek y - \vek x|$, genauer der {\em Logarithmus naturalis\/}, bestimmt, wie sich eine Funktion \"{a}ndert, wenn man sich von einem Punkt $\vek x_1$ zu einem Punkt $\vek x_2$ bewegt.

Wir halten die Formel (\ref{Eq192}) f\"{u}r den Schl\"{u}ssel zur Diffe\-ren\-tial- und Integralrechnung: Gebiet, Rand und Funktion bilden mathematisch eine Einheit und die finiten Elementen sind die logische Umsetzung dieser Idee -- deswegen sind sie so erfolgreich
\begin{align}\label{Eq191}
\text{Finites Element} = \text{Gebiet} + \text{Rand} + \text{Funktion}\,.
\end{align}
Das hat der Bauingenieur {\em Clough\/} instinktiv richtig erfasst und so ist er zu den finiten Elementen gekommen. Mathematiker oder Elektroingenieure denken nicht in {\em shapes\/}, sie h\"{a}tten die finiten Elemente nicht erfinden k\"{o}nnen. Ihnen fehlt -- man verzeihe uns das Vorurteil -- das intuitive Verst\"{a}ndnis f\"{u}r diesen Zusammenhang, das einen Bauingenieur oder Maschinenbauer auszeichnet.

Es ging nicht darum, $\sin (x)$ und $\cos (x)$ durch kurze H\"{u}tchen-Funktion zu ersetzen -- das haben auch andere Autoren vorgeschlagen --  sondern das Bauteilkonzept, die Idee \glq alles in einem\grq, \glq {\em tutto insieme\/}\grq, das wie nat\"{u}rlich aus (\ref{Eq191}) erw\"{a}chst, ist die eigentlich Idee hinter den finiten Elementen und von dieser Idee geht die Faszination der finiten Elemente aus. 
%%%%%%%%%%%%%%%%%%%%%%%%%%%%%%%%%%%%%%%%%%%%%%%%%%%%%%%%%%%%%%%%%%%%%%%%%%%%%%%%%%%%%%%%%%%%%%%%%%%
\textcolor{chapterTitleBlue}{\chapter{Betti Extended}}

In dem vorhergehenden Kapitel haben wir gesehen, dass die FE-L\"{o}sung  $u_h(x)$ die \"{U}berlagerung der gen\"{a}herten Einflussfunktion $G_h(y,x)$ mit der Belastung $p(y)$ ist
\begin{align}\label{Eq96}
u_h(x) = \int_0^{\,l} G_h(y,x)\,p(y)\,dy\,.
\end{align}
Diese f\"{u}r die finiten Elemente zentrale Gleichung beruht auf einem Satz, den wir {\em Betti extended\/} nennen. \\

\begin{theorem}[Betti extended]\index{Betti extended}
Man darf in dem {\em Satz von Betti\/} -- bei unver\"{a}nderter Belastung -- die exakten L\"{o}sungen $u_1$ und $u_2$ durch ihre FE-L\"{o}sungen $u_{1@h}$ und $u_{2@h}$ ersetzen.
\end{theorem}

Wenn also die Gleichung
\begin{align}
A_{1,2} = \int_0^{\,l} p_1\,\underset{\uparrow}{u_2}\,dx = \int_0^{\,l} p_2\,\underset{\uparrow}{u_1}\,dx = A_{2,1}
\end{align}
richtig ist, dann ist auch die Gleichung
\begin{align}
A_{1,2}^h  = \int_0^{\,l} p_1\,\underset{\uparrow}{u_{2@h}}\,dx = \int_0^{\,l} p_2\,\underset{\uparrow}{u_{1@h}}\,dx= A_{2,1}^h
\end{align}
richtig.

%----------------------------------------------------------
\begin{figure}[tbp]
\centering
\if \bild 2 \sidecaption \fi
\includegraphics[width=0.6\textwidth]{\Fpath/U128}
\caption{Der Satz von Maxwell} \label{U128}
\end{figure}%%
%----------------------------------------------------------

Wir behaupten nicht, dass $A_{1,2} = A_{1,2}^h$ ist, sondern nur, dass wenn $A_{1,2} = A_{2,1}$ gilt, dass dann auch $A_{1,2}^h = A_{2,1}^h$ richtig ist; knapp gesagt gilt also
\begin{align}
(p_1,u_2) = (p_2,u_1) \qquad \Rightarrow \qquad (p_1,u_{2@h}) = (p_2,u_{1@h})\,.
\end{align}
Was {\em Betti extended\/} bedeutet, macht man sich am einfachsten an Hand des {\em Satzes von Maxwell\/} klar, s. Abb. \ref{U128}, der ja nur eine spezielle Variante des {\em Satzes von Betti\/} ist
\begin{align}
P_1 \cdot w_2(x_1) = P_2 \cdot w_1(x_2)\,.
\end{align}
Wenn man die beiden Biegelinien $w_1(x)$ und $w_2(x)$ mit finiten Elementen berechnet, dann erh\"{a}lt man nicht die exakten Durchbiegungen in den beiden Punkten $x_1$ und $x_2$
\begin{align}
w_{1@h}(x_2) \neq w_1(x_2) \qquad w_{2@h}(x_1) \neq w_2(x_1)\,,
\end{align}
aber der {\em Satz von Maxwell\/}, das \glq \"{u}ber Kreuz gleich\grq{}, gilt gem\"{a}{\ss} {\em Betti extended\/} auch f\"{u}r diese N\"{a}herungen
\begin{align}
 P_1\cdot w_{2@ h}(x_1 ) = P_2\cdot w_{@1h}(x_2) \,.
\end{align}
Damit ist im \"{u}brigen auch gezeigt, dass der {\em Satz von Maxwell\/} auch f\"{u}r FE-L\"{o}sungen gilt, man setze $P_1 = P_2 = 1$, was ja nicht unbedingt selbstverst\"{a}ndlich ist. Dass er f\"{u}r die Knotenwerte gelten muss, war, wegen der Symmetrie der Steifigkeitsmatrizen, klar. {\em Betti extended\/} garantiert dies aber auch f\"{u}r alle Punkte dazwischen.

%%%%%%%%%%%%%%%%%%%%%%%%%%%%%%%%%%%%%%%%%%%%%%%%%%%%%%%%%%%%%%%%%%%%%%%%%%%%%%%%%%%%%%%%%%%%%%%%%%%
{\textcolor{sectionTitleBlue}{\section{Beweis}}}}

Der Beweis f\"{u}r {\em Betti extended\/} beruht auf den beiden Gleichungen
\begin{subequations}
\begin{align}
\int_{\Omega} p_{1@h}\,u_{2@h} \,d\Omega &= \int_{\Omega} p_1\,u_{2@h} \,d\Omega  \label{Eq91}\\
\int_{\Omega} p_{2@h}\,u_{1@h} \,d\Omega &= \int_{\Omega} p_2\,u_{1@h} \,d\Omega\,,
\end{align}
\end{subequations}
und dem {\em Satz von Betti\/}
\beq
\text{\normalfont\calligra B\,\,}(u_{1@h},u_{2@h}) = \int_{\Omega} p_{1@h}\,u_{2@h} \,d\Omega - \int_{\Omega} p_{2@h}\,u_{1@h} \,d\Omega = 0\,,
\eeq
so dass
\beq
  \int_{\Omega} p_1 \,u_{2@h} \,d\Omega = \int_{\Omega} p_{1@h}\, u_{2@h} \,d\Omega = \int_{\Omega} p_{2@h}\, u_{1@h} \,d\Omega = \int_{\Omega} p_2 \, u_{1@h}\,d\Omega\,,
\eeq
oder
\begin{align}
\int_{\Omega} p_1 \,u_{2@h} \,d\Omega = \int_{\Omega} p_2 \, u_{1@h}\,d\Omega\,,
\end{align}
was die Erweiterung von
\begin{align}
\int_{\Omega} p_1 \,u_2 \,d\Omega = \int_{\Omega} p_2 \, u_1\,d\Omega
\end{align}
auf die FE-L\"{o}sungen ist.

Zu (\ref{Eq91}) kommt man wie folgt: Die {\em Galerkin-Orthogonalit\"{a}t\/} besagt, dass
\begin{align}
\delta A_i = a(u_1 - u_{1@h},\Np_i) = 0
\end{align}
oder, wenn man das mit \"{a}u{\ss}erer statt mit innerer Arbeit schreibt, $\delta A_i = \delta A_a$,
\begin{align}
\int_{\Omega} (p_1 - p_{1@h})\,\Np_i \,d\Omega = 0 \quad i = 1,2,\ldots n \quad \Rightarrow \quad \int_{\Omega}(p_1 - p_{1@h})\, u_{2@h}\,d\Omega = 0\,,
\end{align}
weil ja $u_{2@h} $ eine Linearkombination der $\Np_i$ ist. Sinngem\"{a}{\ss} gilt das auch f\"{u}r die zweite Gleichung.

Mit {\em Betti extended\/} ist der Beweis
der zentralen Gleichung (\ref{Eq96}) sehr einfach, denn in der Einflussfunktion f\"{u}r $u(x)$
\begin{align}
A_{1,2} = 1 \cdot u(x) = \int_0^{\,l} \delta(y-x)\,u(y)\,dy = \int_0^{\,l} G(y,x)\,p(y)\,dy = A_{2,1}
\end{align}
darf man $u$ und $G$ durch die beiden FE-L\"{o}sungen $u_h$ und $G_h$ ersetzen
\begin{align}
A_{1,2}^h = \int_0^{\,l} \delta(y-x)\,\underset{\uparrow}{u_h}(y)\,dy = \int_0^{\,l} \underset{\uparrow}{G_h}(y,x)\,p(y)\,dy = A_{2,1}^h
\end{align}
und somit gilt
\begin{align}
u_h(x) = \int_0^{\,l} G_h(y,x)\,p(y)\,dy \,.
\end{align}
Diese Substitutionen, $u \to u_h$ und $G \to G_h$, kann man bei allen linearen Funktionalen, also allen Integralen wie
\begin{align}\label{Eq47}
J(u) = \int_0^{\,l} \delta(y-y)\,u(y)\,dy = \int_0^{\,l} G(y,x)\,p(y)\,dy
\end{align}
vornehmen, d.h. man darf jederzeit $u$ und $G(y,x)$ durch ihre FE-N\"{a}herungen ersetzen und erh\"{a}lt so
\begin{align}\label{Eq76}
J(u_h) = \int_0^{\,l} \delta(y-y)\,u_h(y)\,dy = \int_0^{\,l} G_h(y,x)\,p(y)\,dy\,.
\end{align}

%%%%%%%%%%%%%%%%%%%%%%%%%%%%%%%%%%%%%%%%%%%%%%%%%%%%%%%%%%%%%%%%%%%%%%%%%%%%%%%%%%%%%%%%%%%%%%%%%%%
{\textcolor{sectionTitleBlue}{\section{In welchen Punkten ist die FE-L\"{o}sung exakt?}}
Mit Hilfe von{\em Betti extended\/} haben wir nun auch Klarheit dar\"{u}ber, wann und wo FE-Ergebnisse exakt sind.

Wir studieren diese Frage an einem vorgespannten Seil und, um mit der Standard-Notation der finiten Elemente konform zu gehen, bezeichnen wir die Durchbiegung des Seils im folgenden mit dem Buchstaben $u$.

Die Einflussfunktion $G(y,x)$ f\"{u}r die Durchbiegung des Seils, s. Abb. \ref{U236}, in dem Punkt $x = 1.5$ ist die Antwort des Seils auf eine Einzelkraft $P = 1$, ein Dirac Delta $\delta(y-x)$.

Die Einzelkraft zwischen den zwei Knoten kann das FE-Programm nicht darstellen und so setzt es statt dessen zwei halb so gro{\ss}e Einzelkr\"{a}fte in die beiden Nachbarknoten. Dies ist -- in unserer Notation -- der Lastfall $\delta_h(y,x)$ und die zugeh\"{o}rige Durchbiegung $G_h(y,x)$ ist die gen\"{a}herte Einflussfunktion.

%----------------------------------------------------------
\begin{figure}[tbp]
\centering
\if \bild 2 \sidecaption \fi
\includegraphics[width=.8\textwidth]{\Fpath/U236}
\caption{Einflussfunktion f\"{u}r die Durchbiegung im Punkt $x = 1.5$, \textbf{ a)} exakte Einflussfunktion, \textbf{ b)} gen\"{a}herte Einflussfunktion \textbf{ c)}
FE-L\"{o}sung unter Gleichlast $p = 1$} \label{U236}
\end{figure}%%
%----------------------------------------------------------

Es gibt also zwei Dirac Deltas, das exakte und das gen\"{a}herte
\begin{align}
\delta(y-x)  \quad \downarrow  \qquad \delta_h(y-x) \quad \frac{1}{2}\,\downarrow  + \frac{1}{2}\,\downarrow
\end{align}
und ebenso zwei Einflussfunktionen
\begin{align} \label{Eq49}
G(y,x) \quad \text{(ein Knick)} \qquad G_h(y,x) \quad \text{(zwei Knicke)} \,.
\end{align}
Mit finiten Elementen suchen wir eine N\"{a}herungsl\"{o}sung in dem LF $p$ (Streckenlast) f\"{u}r den Seildurchhang auf dem Raum $\mathcal{V}_h$, also all den Polygonz\"{u}gen (Seilecken), die mit den drei $\Np_i(x)$ dargestellt werden k\"{o}nnen. Spiegelbildlich zu diesem Raum gibt es einen Raum $\mathcal{V}_h^*$
\index{$\mathcal{V}_h^*$}, der all die Knotenkr\"{a}fte $f_1, f_2, f_3$ enth\"{a}lt, die die Seilecke in $\mathcal{V}_h$ erzeugen.

Es gilt nun: wenn eine Funktion $u_h $ in $\mathcal{V}_h $ liegt (also ein Seileck ist), dann ist das gen\"{a}herte Dirac Delta (2 halbe Einzelkr\"{a}fte) so gut, wie  das exakte Dirac Delta (eine Einzelkraft)
\begin{align}\label{Eq94}
u_h(x) = \int_0^{\,l} \delta(y-x)\,u_h(y)\,dy = \int_0^{\,l} \delta_h(y-x)\,u_h(y)\,dy\,.
\end{align}
Konkret hei{\ss}t das also hier
\begin{align}
1 \cdot u_h(x) = \frac{1}{2}\, \cdot u_h(x_1) + \frac{1}{2}\, \cdot u_h(x_2)\,,
\end{align}
was einleuchtet, weil der Wert einer Geraden zwischen zwei Knoten gerade der Mittelwert der Knotenwerte ist.

Und weil der FE-Lastfall $p_h$ in $\mathcal{V}_h^*$ liegt, also aus drei Knotenkr\"{a}ften besteht, ist die gen\"{a}herte Einflussfunktion $G_h(y,x)$ so gut wie die exakte
\begin{align}
u_h(x) = \int_0^{\,l} G(y,x)\,p_h(y)\,dy = \int_0^{\,l} G_h(y,x)\,p_h(y)\,dy\,.
\end{align}
Auch das ist einfach zu verstehen. Weil die Lastf\"{a}lle $p_h$ nur aus Knotenlasten $f_i$ bestehen, wird bei der Auswertung der Einflussfunktion nicht integriert, sondern nur \"{u}ber die Knoten summiert
\begin{align}\label{Eq195}
u_h(x) = \int_0^{\,l} G_h(y,x)\,p_h(y)\,dy = \sum_{i = 1}^3\,G_h(y_i,x)\,f_i\,.
\end{align}
Die FE-Einflussfunktionen f\"{u}r die Durchbiegung der Knoten $y_i$ sind aber exakt, $G_h(y_i,x) = G(y_i,x)$ in jedem Punkt $x$, und das erkl\"{a}rt, warum die Formel f\"{u}r jeden Punkt $x$ genau den richtigen Wert $u_h(x)$ liefert.\\


\hspace*{-12pt}\colorbox{highlightBlue}{\parbox{0.98\textwidth}{Auf $\mathcal{V}_h$ bzw. $\mathcal{V}_hh^*$ sind die Ergebnisse, die man mit den N\"{a}herungen $\delta_h(y,x)$ bzw. $G_h(y,x)$ erzielt, exakt.}}\\

Um dieses Ergebnis richtig zu w\"{u}rdigen, muss man verstehen, dass mit $u_h$ hier nicht notwendig die FE-L\"{o}sung gemeint ist, sondern dass $u_h$ eine beliebige Funktion aus $\mathcal{V}_h$ sein kann.

Das gen\"{a}herte Dirac Delta $\delta_h$ ist auf $\mathcal{V}_h$ also so gut, wie das exakte, denn (\ref{Eq94}) gilt f\"{u}r alle $u_h \in \mathcal{V}_h$. Und ist $p_h$ der Lastfall, der dem Seil die Gestalt $u_h$ gibt, dann kann man mit der gen\"{a}herten Einflussfunktion $G_h(y,x)$ den Wert $u_h(x)$ aus $p_h$ berechnen. Dies ist der Inhalt von (\ref{Eq195}).


Es ist aber noch eine Steigerung m\"{o}glich. Das gen\"{a}hrte Dirac Delta, also die beiden \glq halben\grq{} Punktlasten in den Nachbarknoten, stellt ja ein eigenes Funktional
\begin{align}
J_h(u) = \int_0^{\,l} \delta_h(y-x)\,u(y)\,dy = \frac{1}{2}\,(u(x_1) + u(x_2))
\end{align}
dar, das man auf beliebige Funktionen anwenden kann -- nicht nur auf die Seilecke in $\mathcal{V}_h$.
Angewandt auf $u(x)= \sin\,\pi\,x/4$ erh\"{a}lt man z.Bsp. den Wert
\begin{align}
J_h(u) = \frac{1}{2}\, (\sin \frac{1.0\,\pi}{4} + \sin \frac{2.0\,\pi}{4}) = 0.85
\end{align}
w\"{a}hrend $J(u) = \sin (1.5\,\pi/4) = 0.92$ ist. Es besteht also ein Unterschied im Ergebnis und damit zwischen $J$ und $J_h$.
%----------------------------------------------------------
\begin{figure}[tbp]
\centering
\if \bild 2 \sidecaption \fi
\includegraphics[width=1.0\textwidth]{\Fpath/U390A}
\caption{Gelenkig gelagerte Platte mit zentrischer St\"{u}tze. Wenn man die vier R\"{a}der des SLW, LF $p$, auf die Einflussfl\"{a}che $G_h$ stellt, erh\"{a}lt man die St\"{u}tzenkraft $S_h$ der FE-L\"{o}sung. Dasselbe Ergebnis erh\"{a}lt man aber auch, wenn man die FE-Belastung $p_h$ (hier in einer symbolischen Darstellung als Blocklast) auf die exakte Einflussfl\"{a}che stellt. Ebenso gilt $S_h = (G_h,p) =  (G_h,p_h)$ Bild b und d} \label{U390}
\end{figure}%%
%----------------------------------------------------------

Mit Blick auf die exakte L\"{o}sung $u(x)$ und die FE-N\"{a}herung $u_h(x)$ gilt jedoch die {\em $h$-Vertauschungsregel\/}\index{$h$-Vertauschungsregel}\index{Vertauschungsregel}
\begin{align}\label{Eq56}
\boxed{J_h(u) = J(u_h)}\,,
\end{align}
die wir im Falle des Seils auch leicht verifizieren k\"{o}nnen
\begin{align}
J_h(u) &= \frac{1}{2}\,(u(1.0) + u(2.0)  ) =  \frac{1}{2}\,(1.5 + 2.0) = 1.75 \\
 J(u_h) &= u_h(1.5) = 1.75\,.
\end{align}
Das Funktional $J_h(u)$ misst $u$ in den beiden Punkten $x = 1.0$ und $x = 2.0$, w\"{a}hrend das Funktional $J(u_h)$ die Biegelinie $u_h$ nur im Aufpunkt $x = 1.5$ misst. Aber beide Messergebnisse sind gleich!

Die Vertauschungsregel basiert auf der Tatsache, dass eine FE-L\"{o}sung auf sechs verschiedene Arten darstellbar ist
\begin{align}
\label{sixways}
u_h(x) &= \int_0^{\,l}  G(y, x)\, p_h(y) \,dy = \int_0^{\,l}  G_h(y, x) \,p(y) \,dy\nn \\
 &= \int_0^{\,l}  G_h(y, x)\, p_h(y) \,dy \nn \\
  &= \int_0^{\,l}  \delta(y, x)\, u_h(y) \,dy =\int_0^{\,l}  \delta_h(y,x)\, u_h(y)\,dy \nn \\
&=\int_0^{\,l}  \delta_h(y,x)\, u(y) \,dy\,,
\end{align}
und wenn wir noch die Formeln
\beq
u_h(x) = a(G,u_h) = a(G_h,u_h) = a(G_h,u)
\eeq
mitz\"{a}hlen, die im Grunde Varianten der Mohrschen Arbeitsgleichung sind, sind es sogar neun.

Auf den ersten beiden Gleichungen
\begin{align}
J(u_h) = \int_0^{\,l} G(y, x) p_h(y) \,dy = \int_0^{\,l}  G_h(y, x) p(y) \,dy = J_h(u)
\end{align}
basiert die $h$-Vertauschungsregel, wobei wir gleich $u_h(x)$ zu $J(u_h)$ erweitert haben, denn (\ref{sixways}) gilt ja nicht nur f\"{u}r das Punktfunktional $J(u) = u(x)$, sondern f\"{u}r jedes lineare Funktional $J$.

Ob man die exakte Einflussfunktion $G$ mit den FE-Lasten $p_h$ \"{u}berlagert, oder die gen\"{a}herte Einflussfunktion $G_h$ mit der Originalbelastung $p$, macht keinen Unterschied -- das Resultat ist dasselbe.
%----------------------------------------------------------
\begin{figure}[tbp]
\centering
\if \bild 2 \sidecaption \fi
\includegraphics[width=.65\textwidth]{\Fpath/U199}
\caption{Wenn $w$ auf $l_e$ linear ist, ist das gen\"{a}herte Dirac Delta in Abb. \textbf{ b)}
genauso gut, wie das exakte, in Abb. \textbf{ a)}, und wenn $w$ auf $l_e$ ein kubisches Polynom ist, gilt dasselbe f\"{u}r \textbf{ c)} und \textbf{ d)}. Die gen\"{a}herten Dirac Deltas sind die \"{a}quivalenten Knotenkr\"{a}fte (= Festhaltekr\"{a}fte $\times (-1)$) aus dem exakten Dirac Delta, wenn man sich also das Element $l_e$ links und rechts eingespannt denkt} \label{U199}
\end{figure}%%
%----------------------------------------------------------

Abb. \ref{U390} demonstriert dies am Beispiel einer St\"{u}tze unter einer Platte, auf der ein Schwerlastwagen (SLW) steht. Dargestellt ist die exakte Einflussfl\"{a}che f\"{u}r die St\"{u}tzenkraft, Abb. \ref{U390} a, und die gen\"{a}herte, Abb. \ref{U390} b. Die Radlasten des SLW stellen den LF $p$ dar, und die Blocklast soll symbolisch f\"{u}r den FE-Lastfall $p_h$ stehen. Es gibt nur eine Formel f\"{u}r die exakte St\"{u}tzenkraft
\begin{align}
S = \int_{\Omega} G(\vek y,\vek x)\,p(\vek y)\,d\Omega_{\vek y}\,,
\end{align}
aber drei M\"{o}glichkeiten die N\"{a}herung $S_h$ zu berechnen
\begin{align}
S_h &= \int_{\Omega} G(\vek y,\vek x)\,p_h(\vek y)\,d\Omega_{\vek y} = \int_{\Omega} G_h(\vek y,\vek x)\,p_h(\vek y)\,d\Omega_{\vek y}\nn \\
 &= \int_{\Omega} G_h(\vek y,\vek x)\,p(\vek y)\,d\Omega_{\vek y}\,.
\end{align}
\\

\begin{remark}
Wie Abb. \ref{U199} am Beispiel eines Seils bzw. eines Balkens illustriert, sind die \"{a}quivalenten Knotenkr\"{a}fte aus dem Dirac Delta, das sind die Kr\"{a}fte und Momente in Abb. \ref{U199} b und d, so gut, wie das exakte Dirac Delta, wenn die Biegelinie \"{u}ber die Elementl\"{a}nge $l_e$ linear verl\"{a}uft
\begin{align}
w(x) = \int_0^{\,l} \delta(y-x)\,w(y)\,dy = \frac{1}{2}\, (w(x_i) + w(x_{i + 1}))
\end{align}
bzw. dort kubisch ist
\begin{align}
w(x) = \int_0^{\,l} \delta(y-x)\,w(y)\,dy &= \frac{1}{2}\,w(x_i) + w'(x_i) \cdot \frac{l_e}{8}\nn \\
&+ \frac{1}{2}\,w(x_{i+1}) -  w'(x_{i+1}) \cdot \frac{l_e}{8}\,.
\end{align}
Das ist die anschauliche Interpretation der $h$-Vertauschungsregel.

Aus der Sicht des FE-Programms reichen die gen\"{a}herten Dirac Deltas $\delta_h$ vollkommen aus, weil sie ja auf $\mathcal{V}_h$ die exakten Werte liefern, $(\delta,\Np_i) = (\delta_h,\Np_i)$.

\end{remark}

%%%%%%%%%%%%%%%%%%%%%%%%%%%%%%%%%%%%%%%%%%%%%%%%%%%%%%%%%%%%%%%%%%%%%%%%%%%%%%%%%%%%%%%%%%%%%%%%%%%
{\textcolor{sectionTitleBlue}{\section{Exakte Werte}}}
Wir k\"{o}nnen nun auch sagen, wann die FE-L\"{o}sung
in einem Punkt exakt ist.\\

\begin{theorem}[Exakte Werte]\index{exakte Werte} \vspace{0.3cm}
\\Hinreichende Bedingungen
\begin{enumerate}
  \item Wenn die Einflussfunktion $G$ eines Funktionals $J$ in $\mathcal{V}_h$ liegt, dann ist sie identisch mit der FE-N\"{a}herung, $G_h = G$, d.h. dann gilt
\beq
J_h(u) = J(u) \qquad \text{f\"{u}r alle}\,\,\,u \in V
\eeq
und daher auch
\beq
J(u_h) = J_h(u) = J(u)\,.
\eeq
  \item Wenn die exakte L\"{o}sung in $\mathcal{V}_h$ liegt, $u = u_h$ (die Projektion $u_h$ ist identisch mit $u$), dann ist der Fehler in jeder Einflussfunktion orthogonal zur rechten Seite $p$
      \beq\label{Eq52}
      J(u) - J(u_h) = \int_{\Omega}(G(\vek y, \vek x) - G_h(\vek y, \vek x))\,p(\vek y)\,d\Omega_{\vek y} = 0\,.
      \eeq
\end{enumerate}
Notwendige Bedingung
\begin{enumerate}
  \item Wenn ein Wert exakt ist, $J(u_h) = J(u)$, dann muss der Fehler in der Einflussfunktion orthogonal sein zur rechten Seite $p$
      \beq
      J(u) - J(u_h) = \int_{\Omega}(G(\vek y, \vek x) - G_h(\vek y, \vek x))\,p(\vek y)\,d\Omega_{\vek y} = 0\,.
      \eeq
\end{enumerate}
\end{theorem}

%%%%%%%%%%%%%%%%%%%%%%%%%%%%%%%%%%%%%%%%%%%%%%%%%%%%%%%%%%%%%%%%%%%%%%%%%%%%%%%%%%%%%%%%%%%%%%%%%%%
{\textcolor{sectionTitleBlue}{\section{Eindimensionale Probleme}}}
Der obige Satz fasst im Grunde die ganze Theorie zusammen, aber vielleicht ist es sinnvoll, auf einzelne Aspekte doch noch n\"{a}her einzugehen.

Es ist bekannt, dass bei eindimensionalen Problemen wie St\"{a}ben und Balken, die FE-L\"{o}sung mit der exakten L\"{o}sung in den Knoten \"{u}bereinstimmt. Dies liegt daran, wie wir jetzt wissen, dass die Einflussfunktionen f\"{u}r die Knotenwerte in dem Ansatzraum $\mathcal{V}_h$ liegen.

Bei Differentialgleichungen zweiter Ordnung wie St\"{a}ben, Seilen, Schubtr\"{a}gern, etc. arbeitet man mit st\"{u}ckweise linearen Ansatzfunktionen. F\"{u}r die Darstellung der Einflussfunktionen der Knoten reichen diese Ans\"{a}tze jedoch aus, wie man zum Beispiel an dem Seil in Abb. \ref{U236} sieht, weil die Einflussfunktionen ja auch nur st\"{u}ckweise linear sind.

Bei Differentialgleichungen vierter Ordnung, wie dem Biegebalken, basieren die finiten Elemente auf Hermite-Polynomen (kubische Ans\"{a}tze) mit denen man die Einflussfunktionen f\"{u}r die Durchbiegungen und Verdrehungen der Knoten exakt darstellen kann. Die FE-L\"{o}sung ist daher in den Knoten  exakt.
%------------------------------------FIGURE-----------------------
\begin{figure}
\centering
{\includegraphics[width=0.65\textwidth]{\Fpath/U130}}
  \caption{Quadratische Elemente interpolieren die exakte L\"{o}sung nicht im Mittelknoten, sondern nur an den Endknoten des Elements, weil die {\em bubble function\/} des Mittelknotens zu glatt ist, \textbf{ a)} Ansatzfunktionen, \textbf{ b)} Punktlast im Mittelknoten und FE--L\"{o}sung, \textbf{ c)} Punktlast  am Endknoten und die FE-L\"{o}sung, die in diesem Falle exakt ist}
  \label{U130}
\end{figure}%
%------------------------------------FIGURE-----------------------

Das Ganze gilt nicht mehr, wenn die homogenen L\"{o}sungen der Differentialgleichungen aus diesem Raster herausfallen, wie im Fall eines in L\"{a}ngsrichtung gebetteten ($c$) Stabes
\begin{align}
- EA\,u''(x) + c\,u(x) = p_x\,,
\end{align}
wo die homogene L\"{o}sung die Gestalt
\beq\label{Eq50}
u(x) = c_1 \,e^{\alpha x } + c_2\, e^{-\alpha x} \qquad \alpha = \sqrt{\frac{c}{EA}}
\eeq
hat oder im Fall eines elastisch gelagerten Balkens,
\begin{align}
 EI\,w^{IV} + c\,w(x) = p_z\,,
\end{align}
wo die homogene L\"{o}sung die Gestalt
\begin{align}\label{Eq51}
\!\!\!\!\!\!\!\!w(x) &= e^{\beta\,x}(c_1\,\cos \beta\,x + c_2\, \sin \beta\,x) +
e^{-\beta\,x}(c_3\,\cos \beta\,x + c_4\, \sin
\beta\,x)\\
\beta &= \sqrt[4]{\frac{c}{EI}}
\end{align}
hat.

Denn alle Einflussfunktionen setzen sich st\"{u}ckweise aus den homogenen L\"{o}sungen der zu Grunde liegenden Differentialgleichung zusammen, aber normalerweise enth\"{a}lt der Ansatzraum $\mathcal{V}_h$ nicht solche \glq exotischen\grq{} Funktionen wie (\ref{Eq50}) und (\ref{Eq51}).

Zu den Merkw\"{u}rdigkeiten geh\"{o}rt auch, dass man mit eigentlich besseren Ansatzfunktionen unter Umst\"{a}nden die F\"{a}higkeit verliert, die exakte L\"{o}sung in den Knoten zu interpolieren. Dies passiert, wenn man zum Beispiel eine Seillinie mit quadratischen Elementen ann\"{a}hert, s. Abb. \ref{U130}.

In der Mitte des Elementes ist eine quadratische FE-L\"{o}sung glatt, dort regiert die {\em bubble function\/}\index{bubble function} des Mittenknotens. Was man aber br\"{a}uchte, w\"{a}re die M\"{o}glichkeit, dort einen Sprung in der ersten Ableitung darstellen zu k\"{o}nnen, damit man die Wirkung einer Einzelkraft in dem Mittenknoten wiedergeben kann. Die Ableitung der {\em bubble function\/} ist aber glatt, sie springt nicht, und deswegen stimmt z.B. die FE-L\"{o}sung des Problems
\begin{align}
- u''(x) = x \qquad u(0) = u(l) = 0\,,
\end{align}
nicht mit der exakten L\"{o}sung in den Mittenknoten \"{u}berein. Die Einflussfunktion f\"{u}r den Mittenknoten liegt nicht in $\mathcal{V}_h$\footnote{Wenn $u$ in $\mathcal{V}_h$ liegt, z.B. wenn $u$ quadratisch ist, dann ist der Fehler $u(x) - u_h(x) = 0$, weil in dem Fall der Fehler in der Einflussfunktion orthogonal zu $p$ ist.}.
%-----------------------------------------------------------------
\begin{figure}[tbp]
\centering
\includegraphics[width=0.72\textwidth]{\Fpath/U158}
\caption{Platte im LF $g$, \textbf{ a)} Einflussfunktion f\"{u}r $m_{xx}$ in Plattenmitte, \textbf{ b)} Momente $m_{xx}$ im L\"{a}ngsschnitt, \textbf{ c)} Momente $m_{xx}$ mit St\"{u}tze} \label{U158}
\end{figure}%
%-----------------------------------------------------------------

Ein FE-Programm muss also eine Balance finden zwischen der Regularit\"{a}t, die von der Wechselwirkungsenergie verlangt wird, damit man die Beitr\"{a}ge $k_{ij} = a(\Np_i,\Np_j) $ der Steifigkeitsmatrix berechnen kann und der \glq Nicht-Regularit\"{a}t\grq{}, die man braucht, um einen Sprung in der ersten Ableitung erzeugen zu k\"{o}nnen.

%%%%%%%%%%%%%%%%%%%%%%%%%%%%%%%%%%%%%%%%%%%%%%%%%%%%%%%%%%%%%%%%%%%%%%%%%%%%%%%%%%%%%%%%%%%%%%%%%%%
{\textcolor{sectionTitleBlue}{\section{Fl\"{a}chentragwerke}}}
Wenn bei Fl\"{a}chentragwerken die Ergebnisse in einem Punkt $\vek x$ exakt sind, dann ist das in der Regel Zufall. Dann kann es nur so sein, dass der Fehler $G(\vek y,\vek x) - G_h(\vek y,\vek x)$ in der Einflussfunktion orthogonal zur Belastung $p$ ist,
\begin{align}
u(\vek x) - u_h(\vek x) = \int_{\Omega} (G(\vek y,\vek x) - G_h(\vek y,\vek x))\,p(y)\,\,d\Omega_{\vek y} = 0\,,
\end{align}
denn die exakten Einflussfunktionen liegen bei Fl\"{a}chentragwerken nicht in dem Ansatzraum $\mathcal{V}_h$ der finiten Elemente, weil die verwendeten {\em shape functions\/} keine homogenen L\"{o}sungen der Scheiben- bzw. Plattengleichung sind.

%---------------------------------------------------------------------------------
\begin{figure}[tbp]
\centering
\if \bild 2 \sidecaption \fi
\includegraphics[width=0.8\textwidth]{\Fpath/U139}
  \caption{3-D Darstellung der Momente $m_{xx}$ der Quadratplatte mit einer zentrischen St\"{u}tze im LF $g$}
  \label{U139}
\end{figure}
%---------------------------------------------------------------------------------
%---------------------------------------------------------------------------------
\begin{figure}[tbp]
\centering
\if \bild 2 \sidecaption \fi
\includegraphics[width=0.8\textwidth]{\Fpath/U138}
  \caption{Quadratplatte, 8 m $\times$ 8 m, mit Einzelst\"{u}tze in der Mitte, (dargestellt ist die untere H\"{a}lfte der Platte), Serie von Einflussfunktionen f\"{u}r $m_{xx}$ auf der $x$-Achse}
  \label{U138}
\end{figure}
%---------------------------------------------------------------------------------
Es ist aber auch klar, dass die Art der Belastung
\begin{align}
\text{{\em Gleichlast\/}} \qquad\text{{\em Linienlast\/}} \qquad \text{{\em Punktlast\/}}
\end{align}
einen Einfluss auf die Gr\"{o}{\ss}e des Fehlers hat. Je gleichm\"{a}{\ss}iger die Belastung verteilt ist, um so eher gleichen sich die Fehler in den Einflussfunktionen im Mittel aus, wie man am Beispiel einer gelenkig gelagerten Quadratplatte sehen kann.

In Abb.  \ref{U158} a ist die Einflussfunktion f\"{u}r das Moment $m_{xx}$ in Plattenmitte dargestellt. Im Lastfall $g$ ist das Moment
\begin{align}
m_{xx}(\vek x) = \int_{\Omega} G_2(\vek y,\vek x)\,g\,\,d\Omega_{\vek y} = g \cdot \int_{\Omega} G_2(\vek y,\vek x)\,d\Omega_{\vek y} = g \cdot V
\end{align}
gleich dem Volumen $V$ von $G_2$ mal $g$. Und anscheinend kann das FE-Pro\-gramm das Volumen der Einflussfunktionen relativ gut bestimmen, die
Momentenverteilung in Abb. \ref{U158} b wirkt \"{u}berzeugend.

Eine Einzelkraft $ P$ aus einer St\"{u}tze, $\Omega_S = a \times b$, ist jedoch von einem anderen Kaliber, s. Abb. \ref{U139},
\begin{align}\label{Eq134}
m_{xx}(\vek x) = \int_{\Omega} G_2(\vek y,\vek x)\,g\,\,d\Omega_{\vek y} + \frac{P}{\Omega_S} \int_{\Omega_S} G_2(\vek y,\vek x)\,d\Omega_{\vek y}\,,
\end{align}
denn $G_2$ ist ja singul\"{a}r in der St\"{u}tzenmitte.

Genau genommen rechnet das Programm zwar mit einer Ersatzlast $g_h(\vek y)$, die das Eigengewicht wie die St\"{u}tzenkraft \glq \"{a}quivalent\grq{} beinhaltet
\begin{align}\label{Eq135}
m_{xx}^h(\vek x) = \int_{\Omega} G_2^h(\vek y,\vek x)\,g_h(\vek y)\,\,d\Omega_{\vek y}\,,
\end{align}
aber auch diese Formel kommt an dem Grundproblem, $G_2(\vek x, \vek x) = \infty$, und dem scharfen Anstieg von $G_2$ zur St\"{u}tze hin nicht vorbei.

Die Abb. \ref{U138} illustriert, wie sich die Einflussfunktion f\"{u}r $m_{xx}$ \"{a}ndert, wenn sich der Aufpunkt der St\"{u}tze n\"{a}hert. Im Grunde bleibt sie sich immer gleich, nur wird sie bei der Ann\"{a}herung an die St\"{u}tze nach oben geschoben. Und das Ma{\ss}, um wieviel sie nach oben geschoben werden muss, das ist umso schwerer zu bestimmen, je n\"{a}her der Aufpunkt der St\"{u}tze kommt.\\

%%%%%%%%%%%%%%%%%%%%%%%%%%%%%%%%%%%%%%%%%%%%%%%%%%%%%%%%%%%%%%%%%%%%%%%%%%%%%%%%%%%%%%%%%%%%%%%%%%%
{\textcolor{sectionTitleBlue}{\section{Punktlager bei Scheiben und Platten und der Unterschied}}}
Bei Scheiben liegen die Dinge \"{a}hnlich. Es gibt jedoch einen bemerkenswerten Unterschied zwischen Scheibe und Platte.

Wenn eine Scheibe sich auf Punktlager abst\"{u}tzt, dann kennt man die Lagerkr\"{a}fte relativ genau (bei statisch bestimmter Lagerung sogar exakt), aber das hilft einem nicht bei der Eingrenzung der (mit dem Ingenieurverstand vertr\"{a}glichen) maximalen Spannungen in der N\"{a}he der Punktlager. Hier kann man sich aber so behelfen, dass man die Lagerkraft \"{u}ber eine gewisse Breite gleichm\"{a}{\ss}ig verteilt und die Scheibe f\"{u}r diese Spannungen bemisst, s. Abb. \ref{U133} a.

%---------------------------------------------------------------------------------
\begin{figure}[tbp]
\centering
\if \bild 2 \sidecaption \fi
\includegraphics[width=0.5\textwidth]{\Fpath/U133}
  \caption{\textbf{ a)} Die Lagerkraft $R$ und die Lagerpressung bei einer Scheibe, \textbf{ b)} bei einem Balken kann man jedoch keine Beziehung zwischen dem St\"{u}tzmoment $M$ und der Lagerkraft $R$ herstellen}
  \label{U133}
\end{figure}
%---------------------------------------------------------------------------------
Bei einer Platte ist die Situation im Grunde dieselbe. Die Lagerkr\"{a}fte in den St\"{u}tzen sind bei einer FE-Berechnung relativ genau, aber das hilft einem nicht -- und das ist der Unterschied -- bei der Eingrenzung der maximalen Biegemomente \"{u}ber der St\"{u}tze, weil es keinen direkten Zusammenhang zwischen der St\"{u}tzenkraft und den Biegemomenten gibt; die Ausmitte $e$ bleibt unbekannt.

Es reicht ein Zwischenlager eines Balkens zu betrachten, s. Abb. \ref{U133} b. Die Summe der Querkr\"{a}fte $V_l$ und $V_r$ muss gleich der Lagerkraft $R$ sein, aber es gelingt nicht, das St\"{u}tzmoment $M$ in irgendeiner Weise mit $R$ zu verkn\"{u}pfen.

Es gibt ja eine Vielzahl von Tr\"{a}gern, die \"{u}ber einem Zwischenlager bei gleicher Lagerkraft ganz unterschiedliche Momente aufweisen. Dazu passt, dass Momente keine Arbeiten leisten, wenn man sie hebt oder senkt -- Kr\"{a}fte schon. Man m\"{u}sste die Platte neigen, um die Ausmitte $e$ zu finden. Die Wirkung eines Moments geht mit der Neigung der Einflussfunktion, Wirkung = $G' \cdot M$.

%%%%%%%%%%%%%%%%%%%%%%%%%%%%%%%%%%%%%%%%%%%%%%%%%%%%%%%%%%%%%%%%%%%%%%%%%%%%%%%%%%%%%%%%%%%%%%%%%%%
{\textcolor{sectionTitleBlue}{\section{Wenn die L\"{o}sung in $\mathcal{V}_h$ liegt}}}
Wenn die L\"{o}sung in $\mathcal{V}_h$ liegt, weil zum Beispiel bilineare Ans\"{a}tze ausreichen, um die Verformungen einer Scheibe darzustellen, dann ist $\vek p_h = \vek p$, d.h. dann ist der FE-Lastfall identisch mit dem Originallastfall. Den FE-Lastfall $\vek p_h $ erh\"{a}lt man ja, indem man die FE-L\"{o}sung in die Originalgleichung einsetzt und dabei muss in diesem Falle gerade die Belastung $\vek p$ herauskommen, die man aufgebracht hat.
%---------------------------------------------------------------------------------
\begin{figure}[tbp]
\centering
\if \bild 2 \sidecaption \fi
\includegraphics[width=1.0\textwidth]{\Fpath/U237}
  \caption{FE-Einflussfunktionen f\"{u}r $\sigma_{xx}$. Diese \glq eckigen\grq{} Einflussfunktionen sind bestimmt nicht richtig, aber das Integral \"{u}ber den Lastrand ergibt in beiden F\"{a}llen, und in jedem anderen Fall auch, den korrekten Wert f\"{u}r $\sigma_{xx}$.}
  \label{U237}
\end{figure}
%---------------------------------------------------------------------------------

Es bleibt aber ein Problem. Ein FE-Programm berechnet alle Werte und also auch die Spannungen mit gen\"{a}herten Einflussfunktionen
\begin{align}
\sigma_{xx}^h(\vek x) = \int_{\Omega} \vek G_h(\vek y,\vek x)\dotprod \vek p(\vek y)\,d\Omega_{\vek y}\,.
\end{align}
Also m\"{u}ssten die FE-Spannungen doch nur N\"{a}herungswerte sein, warum sind sie aber exakt?

Der Grund ist, dass der Fehler in den Einflussfunktionen orthogonal zu der Belastung ist, wenn die L\"{o}sung in $\mathcal{V}_h$ liegt,
\begin{align}
\sigma_{xx}(\vek x) - \sigma_{xx}^h(\vek x)= \int_{\Omega} (\underbrace{\vek G(\vek y,\vek x) - \vek G_h(\vek y,\vek x)}_{Fehler})\dotprod \vek p(\vek y)\,d\Omega_{\vek y} = 0\,.
\end{align}
Jedes solches $\vek p$ \glq neutralisiert\grq{} also den Fehler in den Einflussfunktionen.\\

\hspace*{-12pt}\colorbox{highlightBlue}{\parbox{0.98\textwidth}{Der Fehler in einer gen\"{a}herten Einflussfunktion ist orthogonal zu allen Lastf\"{a}llen $p$, die sich auf $\mathcal{V}_h$ exakt l\"{o}sen lassen}}\\

Die Scheibe in Abb. \ref{U237} ist so gelagert, dass sich unter Zug ein gleichf\"{o}rmiger Spannungszustand aufbaut, den man mit bilinearen Elementen  exakt wiedergeben kann. Die  exakte L\"{o}sung liegt also in $\mathcal{V}_h $.
%----------------------------------------------------------
\begin{figure}[tbp]
\centering
\if \bild 2 \sidecaption \fi
\includegraphics[width=1.0\textwidth]{\Fpath/U129}
\caption{Dirac Delta $\delta(\vek y- \vek x)$ und das gen\"{a}herte Dirac Delta $\delta_h(\vek y, \vek x)$ bei einer Scheibe, auf $\mathcal{V}_h$ liefern beide dasselbe Ergebnis } \label{U129}
\end{figure}%%
%----------------------------------------------------------

Aber die Einflussfunktion f\"{u}r die horizontale Spannung $ \sigma_{xx} $, gleich welchen Punkt $\vek x$ man betrachtet, liegt nicht in $\mathcal{V}_h $, weil dazu Punktversetzungen n\"{o}tig sind, die sich mit bilinearen Elementen nicht erzeugen lassen, aber trotzdem sind die Ergebnisse exakt.

Das geht nur so, dass die horizontalen Randverschiebungen der FE-Einflussfunktion in der Summe (dem Integral) genau den exakten Wert treffen, denn sonst w\"{a}re $\sigma_{xx}^h$ nicht gleich $\sigma_{xx}$
\begin{align}
\sigma_{xx}^h(\vek x) = \int_{\Gamma} \vek G_h(\vek y,\vek x) \dotprod \vek t(\vek y)\,ds_{\vek y} = \sigma_{xx}(\vek x)\,.
\end{align}
Hier ist $\Gamma$ der rechte Rand und $\vek t = \{t,0\}^T$ ist die Randlast.\\

\begin{remark}
Die Gleichung
\begin{align}\label{Eq111}
\int_{\Omega} (\vek G(\vek y,\vek x) - \vek G_h(\vek y,\vek x))\dotprod \vek p_h(\vek y)\,\,d\Omega_{\vek y} = 0
\end{align}
ist das Gegenst\"{u}ck zu der {\em Galerkin-Orthogonalit\"{a}t\/}, die ja besagt, dass die Abweichung des FE-Lastfalls $\vek p_h$ vom exakten Lastfall $\vek p$ orthogonal zu allen Ansatzfunktionen $\vek \Np_i$ ist
\begin{align}
\int_{\Omega} (\vek p(\vek x) - \vek p_h(\vek x))\dotprod \vek \Np_i(\vek x)\,d\Omega = 0\,.
\end{align}
Galerkin testet mit den Ansatzfunktionen, w\"{a}hrend der Test bei den Einflussfunktionen, (\ref{Eq111}), die Lastf\"{a}lle $\vek p_h$ sind, die man auf $\mathcal{V}_h$ exakt l\"{o}sen kann. F\"{u}r jedes solche $\vek p_h$ muss  die Differenz $\vek G - \vek G_h$ orthogonal zu $\vek p_h$ sein.

Die {\em Galerkin-Orthogonalit\"{a}t\/} gilt auch f\"{u}r Punktlasten, s. Abb. \ref{U129}, und daher ist auf $\mathcal{V}_h$ kein Unterschied zwischen dem exakten und dem gen\"{a}herten Dirac Delta ($\Np_{i}^H$ = horizontale Komponente von $\vek \Np_i(\vek x)$ im Punkt $\vek x$)
\begin{align}
\Np_{i}^H(\vek x) = \int_{\Omega} \vek \delta(\vek y - \vek x) \dotprod \vek \Np_i(\vek y) \,d\Omega_{\vek y} = \int_{\Omega} \vek \delta_h(\vek y, \vek x) \dotprod \vek \Np_i(\vek y) \,d\Omega_{\vek y}\,.
\end{align}
\end{remark}
%----------------------------------------------------------------------
\begin{figure}[h]
\if \bild 2 \sidecaption \fi
\includegraphics[width=0.8\textwidth]{\Fpath/U533}
\caption{Um das Netz an den richtigen Stellen zu verfeinern, muss man die exakte L\"{o}sung nicht kennen. Dort wo die Knicke in der FE-L\"{o}sung gro{\ss} sind, dort verfeinert man. In der Statik sind die Knicke die Spannungsspr\"{u}nge zwischen den Elementen}\label{U533}
\end{figure}
%----------------------------------------------------------------------
\vspace{-0.5cm}
%%%%%%%%%%%%%%%%%%%%%%%%%%%%%%%%%%%%%%%%%%%%%%%%%%%%%%%%%%%%%%%%%%%%%%%%%%%%%%%%%%%%%%%%%%%%%%%%%%%
{\textcolor{sectionTitleBlue}{\section{Patch-Test}}}\label{Patch-Test}\index{Patch-Test}
Bei einem Patch-Test konstruiert man eine L\"{o}sung, die in $\mathcal{V}_h$ liegt, vorzugsweise sind das L\"{o}sungen mit einfachen Spannungsfeldern, und kontrolliert, ob sich die exakten Ergebnisse ergeben. Der Patch-Test lebt davon, dass der Fehler in den Einflussfunktionen orthogonal zur Belastung ist, wenn die L\"{o}sung $u$ in  $\mathcal{V}_h$ liegt, wenn also $p_h = p$ ist, denn mit (\ref{Eq92}) gilt
\begin{align}
\int_{\Omega} (G(\vek y,\vek x) - G_h(\vek y,\vek x))\,p(\vek y)\,d\Omega_{\vek y} = \int_{\Omega} G(\vek y,\vek x)\,(p(\vek y) - p_h(\vek y))d\Omega_{\vek y}\,,
\end{align}
und weil die rechte Seite null ist muss auch die linke Seite null sein. Tests, die garantiert schief gehen, sind Einzelkr\"{a}fte bei Fl\"{a}chentragwerken. Sie \"{u}berfordern jedes Netz. Kein $\mathcal{V}_h$ enth\"{a}lt die exakte L\"{o}sung.

%----------------------------------------------------------
\begin{figure}[tbp]
\centering
\if \bild 2 \sidecaption[t] \fi
\includegraphics[width=.7\textwidth]{\Fpath/U271}
\caption{Adaptive Verfeinerung einer Scheibe in der Umgebung der kritischen Punkte} \label{U271}
\end{figure}%%
%----------------------------------------------------------

%%%%%%%%%%%%%%%%%%%%%%%%%%%%%%%%%%%%%%%%%%%%%%%%%%%%%%%%%%%%%%%%%%%%%%%%%%%%%%%%%%%%%%%%%%%%%%%%%%%
{\textcolor{sectionTitleBlue}{\section{Adaptive Verfeinerung}}}\label{SecAdaptiveVerfeinerung}
Den Fehler einer FE-L\"{o}sung kann man nicht direkt berechnen, denn man kennt weder die exakten Verschiebungen noch die exakten Spannungen. Man kann nur die Abweichung zwischen dem Originallastfall $\vek p$ und dem FE-Lastfall $\vek p_h$ messen und dann das Netz dort verfeinern, s. Abb. \ref{U271}, wo diese Differenzen am gr\"{o}{\ss}ten sind.

Theoretisch geht es sogar noch einfacher, wie Abb. \ref{U533} demonstriert: Man achtet nur auf die 'Knicke' (= Spannungsspr\"{u}nge in der FE-L\"{o}sung) und verfeinert das Netz an diesen Stellen.

Da es, wie so oft, nur auf die Algebra ankommt, wollen wir die Gleichungen skalar schreiben und die Biegefl\"{a}che $u(\vek x)$ einer Membran, die unter einem Druck $p$ steht, betrachten.

Aus den Darstellungen
\begin{align}
u(\vek x)= \int_{\Omega} G(\vek y,\vek x)p(\vek y)\,d\Omega_{\vek y} \qquad u_h(\vek x)= \int_{\Omega} G_h(\vek y,\vek x)p(\vek y)\,d\Omega_{\vek y}
\end{align}
ergibt sich der Fehler der FE-L\"{o}sung zu
\begin{align}
u(\vek x) - u_h(\vek x) = \int_{\Omega} (G(\vek y,\vek x) - G_h(\vek y,\vek x))\,p(\vek y)\,d\Omega_{\vek y}\,.
\end{align}
Nun wissen wir aber auch, dass der Fehler einer FE-L\"{o}sung darauf beruht, dass ein FE-Programm den FE-Lastfall $p_h$ statt des exakten Lastfalls $p$ l\"{o}st
\begin{align}
u(\vek x) - u_h(\vek x) = \int_{\Omega} G(\vek y,\vek x)(p(\vek y) - p_h(\vek y))\,d\Omega_{\vek y}\,,
\end{align}
und somit gilt
\begin{subequations}\label{Eq92}
\begin{align}
u(\vek x) - u_h(\vek x) &= \int_{\Omega} G(\vek y,\vek x)(p(\vek y) - p_h(\vek y))\,d\Omega_{\vek y} \label{Eq92:SubEq1} \\
&= \int_{\Omega} (G(\vek y,\vek x) - G_h(\vek y,\vek x))\,p(\vek y)\,d\Omega_{\vek y} \label{Eq92:SubEq2} \\
&= \int_{\Omega} (G(\vek y,\vek x) - G_h(\vek y,\vek x))\,(p(\vek y) - p_h(\vek y))\,d\Omega_{\vek y} \label{Eq92:SubEq3}\,.
\end{align}
\end{subequations}
Den Schritt von der zweiten Gleichung zur dritten Gleichung macht die {\em Galerkin-Orthogonalit\"{a}t\/}
\begin{align}
\int_{\Omega} (G(\vek y,\vek x) - G_h(\vek y,\vek x))\,p_h(\vek y)\,d\Omega_{\vek y} = 0
\end{align}
m\"{o}glich; wir haben nur eine null addiert.

Auf der ersten Gleichung (\ref{Eq92:SubEq1}) basiert die klassische adaptive Verfeinerung, bei der das Netz dort verfeinert wird, wo die Abweichungen zwischen $p$ und dem FE-Lastfall $p_h$ gro{\ss} sind.
%----------------------------------------------------------------------------
\begin{figure}[tbp] \centering
\if \bild 2 \sidecaption \fi
\includegraphics[width=0.8\textwidth]{\Fpath/TOTTENHAM24}
\caption{Bilineare Elemente, LF $g$, Berechnung von $\sigma_{xx}$ {\bf a)} Ausgangsnetz
{\bf b)} halbe Elementl\"{a}nge {\bf c)} adaptive Verfeinerung {\bf d)} Verfeinerung mittels
Dualit\"{a}tstechnik}\label{Tottenham}
\end{figure}
%----------------------------------------------------------------------------

Die andere Strategie w\"{a}re es, den Fehler $G-G_h$ in der Einflussfunktion kleiner zu machen, wie dies (\ref{Eq92:SubEq2}) nahelegt.

Dabei stellt sich die Frage, wie wir den Abstand $G - G_h$ messen wollen, denn wir kennen das exakte $G$ nicht. Dies Problem l\"{o}sen wir wie folgt: Zun\"{a}chst ersetzen wir die Einflussfunktion (\ref{Eq92:SubEq2}) durch ihre schwache Form
\begin{align}\label{Eq180}
u(\vek x) - u_h(\vek x) &= \int_{\Omega} (G(\vek y,\vek x) - G_h(\vek y,\vek x))\,p(\vek y)\,d\Omega_{\vek y} = a(G - G_h,u)
\end{align}
und schreiben diese Wechselwirkungsenergie \"{a}quivalent als \"{a}u{\ss}ere Arbeit\footnote{Das ist der Satz von Betti, (\ref{Eq180}) = $W_{12} = W_{21}$ = (\ref{Eq181})}
\begin{align}\label{Eq181}
u(\vek x) - u_h(\vek x) &=a(G - G_h,u) = \int_{\Omega} (\delta(\vek y -\vek x) - \delta_h(\vek y -\vek x))\,u(\vek y)\,d\Omega_{\vek y}\,,
\end{align}
und die Abweichung $\delta - \delta_h$ k\"{o}nnen wir messen, s. Abb. \ref{U129}. Nicht direkt, weil $\delta$ ja nur ein fl\"{u}chtiges Symbol ist, aber die Kr\"{a}fte $\delta_h$ k\"{o}nnen wir plotten und das Netz so verfeinern, dass sie weitgehend vom Bildschirm verschwinden, sich in einem Punkt zusammenschn\"{u}ren.

Die dritte Variante ist die {\em zielorientierte adaptive Verfeinerung\/} ({\em goal oriented adaptive refinement)\/}. Bei dieser Technik wird der Fehler $G- G_h$ in der Einflussfunktion und gleichzeitig der Fehler $p - p_h$ in der Belastung kleiner gemacht. Sie basiert auf (\ref{Eq92:SubEq3}). Schreiben wir diese Gleichung in schwacher Form, dann lautet sie
\begin{align}
u(\vek x) - u_h(\vek x) &= a(\underbrace{G(\vek y,\vek x) - G_h(\vek y,\vek x)}_f, \underbrace{\vphantom{G(\vek y,\vek x)} u - u_h}_g) =: a(f,g)
\end{align}
und jetzt kann man die {\em Schwarzsche Ungleichung\/}\index{Schwarzsche Ungleichung} zu Hilfe nehmen,
\begin{align}\label{Eq169}
|u(\vek x) - u_h(\vek x)| = | a(f,g)|  \leq a(f,f)^{1/2} \cdot a(g,g)^{1/2} = \|f\| \cdot \|g\|\,,
\end{align}
um die linke Seite durch die Energienorm von $f$  und von $g$ abzusch\"{a}tzen.

Theoretisch steht (\ref{Eq169}) auf \glq wackligen F\"{u}{\ss}en\grq{}, weil die Energienorm der Einflussfunktion ja in der Regel unendlich gro{\ss} ist, aber wenn wir das schlicht ignorieren -- der Erfolg gibt uns recht -- dann ist das die schnellste Art, den Fehler der FE-L\"{o}sung in einem Punkt klein zu machen.

Technisch geht man bei der  zielorientierten adaptiven Verfeinerung so vor, dass man neben dem {\em primalen Problem\/} $- \Delta u = f$ noch das {\em duale Problem\/} $- \Delta G = \delta_0$ l\"{o}st und das Netz so optimiert, dass die Fehler in beiden L\"{o}sungen klein werden, wie bei der Scheibe in Abb. \ref{Tottenham}, wo es darum ging die Spannung $\sigma_{xx}$ in dem Aufpunkt m\"{o}glichst genau zu bestimmen.

Man k\"{o}nnte nach diesen Bemerkungen nun vermuten, dass fehlende Spr\"{u}nge eine Garantie daf\"{u}r sind, dass die FE-L\"{o}sung \glq genau\grq{} ist, aber das ist nicht garantiert, \cite{Ha5}, denn Fernfeldfehler, Stichwort {\em pollution\/}, k\"{o}nnen zu einem {\em drift\/} der FE-L\"{o}sung f\"{u}hren, den man nicht registriert, wenn man nur auf die Spr\"{u}nge achtet.

%%%%%%%%%%%%%%%%%%%%%%%%%%%%%%%%%%%%%%%%%%%%%%%%%%%%%%%%%%%%%%%%%%%%%%%%%%%%%%%%%%%%%%%%%%%%%%%%%%%
{\textcolor{sectionTitleBlue}{\section{Pollution}}}
Der englische Begriff {\em pollution\/}\index{pollution} meint das Ph\"{a}nomen, dass die L\"{o}sung in einem Teil $A$ des Tragwerks von Fehlerquellen, die in einem abliegenden Teil $B$ auftreten, negativ beeinflusst wird. Bei Fl\"{a}chentragwerken haben die Einflussfunktionen mit diesem Problem zu k\"{a}mpfen, s. Kapitel 6.10. Die Fehler in den Ecken strahlen ins Innere aus.
%---------------------------------------------------------------------------------
\begin{figure}
\centering
{\includegraphics[width=1.0\textwidth]{\Fpath/1GreenF156}}
\caption{Einflussfunktion f\"{u}r die Scherkraft $N_{yx}$ im Schnitt $A-A$, \textbf{ a)} Wandscheibe und FE-Netz, \textbf{ b)} FE-Einflussfunktion $G_h$, \textbf{ c)} verbessertes Modell, \textbf{ d)} verbesserte FE-L\"{o}sung $G_h$, \cite{Ha6}}
\label{1GreenF156}%
%
\end{figure}%
%---------------------------------------------------------------------------------

Die Wandscheibe in Abb. \ref{1GreenF156} illustriert dieses Ph\"{a}nomen sehr gut. Die Einflussfunktion f\"{u}r die Scherkraft $N_{yx}$ im horizontalen Schnitt $A-A$ ist eine Seitw\"{a}rtsbewegung des Teils oberhalb des Schnittes um einen Meter nach rechts.

Auf einem FE-Netz kann man diese Bewegung aber nicht nachfahren, weil man dann die Elemente auseinander schneiden m\"{u}sste. So produziert ein FE-Programm, das bilineare Elemente benutzt, die Verformungsfigur in Abb. \ref{1GreenF156} b, bei der sich die Oberkante der Scheibe um 2.3 m nach rechts bewegt! Die Schwierigkeiten, die das FE-Programm hat, die Gleitbewegung im Schnitt $A-A$ darzustellen, f\"{u}hren also zu einem gro{\ss}en Fehler an einer anderen Stelle, der Oberkante der Scheibe. Die Konsequenz ist, dass von einer Belastung von 1 kN an der Oberkante der Scheibe das 2.3-fache im Schnitt $A-A$ ankommt, was wahrlich ein gro{\ss}er Fehler ist.

Wie man in Abb. \ref{1GreenF156} d sieht, kann man durch eine adaptive Verfeinerung des Netz den Fehler in der oberen Auslenkung deutlich von 2.3 - 1.0 auf 1.2 - 1.0 verringern.\\
%----------------------------------------------------------------------------
\begin{figure}[tbp]
\centering
\if \bild 2 \sidecaption \fi \label{Korrektur4}
\includegraphics[width=1.0\textwidth]{\Fpath/U369}
\caption{Zugstab mit sich ver\"{a}nderndem Querschnitt und Einflussfunktion f\"{u}r das Stab\-ende \textbf{ a)} System \textbf{ b)} Einflussfunktion f\"{u}r Verschiebung des Stabendes \textbf{ c)} dieselbe Figur kann man auch an einem Stab $EA = 1$ mit Hilfe von Kr\"{a}ften $j^+$ erzeugen, s. S. \pageref{jplus} } \label{U369}
\end{figure}%
%----------------------------------------------------------------------------
Es gilt auch, dass, wenn man den Schnitt $A-A $ genau durch die {\em Mitte\/} der Elementreihe f\"{u}hrt, sich die Oberkante wirklich um 1 m nach rechts bewegt, also die Belastung an der Oberkante richtig im Schnitt $A-A$ ankommt. Die resultierende Scherkraft $N_{yx}$ in der Mitte des Elements ist dann also exakt.

Wenn man das sogenannte {\em Wilson-Element\/}\index{Wilson-Element} benutzt, \cite{Ha5}, dann werden die Ergebnisse auch f\"{u}r einen Schnitt $A-A $ richtig, der nicht durch die Mitte der Elementreihe geht. Nur den Schnitt innerhalb des Elements selbst, die abrupte Scherbewegung, kann man auch mit dem Wilson-Element nur \glq gleitend\grq{} nachvollziehen.

{\textcolor{sectionTitleBlue}{\subsubsection*{Ursachen}}}
{\em Pollution\/} hat im wesentlichen drei Ursachen, \cite{Babuska5}:\\

\begin{enumerate}
  \item Unstetigkeiten in der Belastung, also zum Beispiel Spr\"{u}nge in der Verkehrslast einer Platte (ein mildes Problem)
  \item Gro{\ss}e und kleine Elemente nebeneinander, also ein unausgewogenes Netz; andererseits gilt aber, dass man mit gradierten Netzen {\em pollution\/} d\"{a}mpfen kann
  \item Nicht-glatte Einflussfunktionen, s. Abb. \ref{U369}
\end{enumerate}
Wenn die Koeffizienten in der Differentialgleichung springen, weil z.B. die Dicke $d$ der Platte springt, dann schl\"{a}gt das auf die Einflussfunktionen durch, d.h. die Regularit\"{a}t (\glq die Gl\"{a}tte\grq{}) der Einflussfunktionen nimmt ab, es treten Knicke auf wie in Abb. \ref{U369} b und damit vergr\"{o}{\ss}ert sich die {\em pollution\/}, weil das FE-Programm mehr M\"{u}he hat, solche Einflussfunktionen zu approximieren.

Alle FE-L\"{o}sungen -- bis auf Stabtragwerke ($EA$ und $EI$ konstant) -- weisen einen globalen Fehler auf, k\"{a}mpfen mit {\em pollution\/}.

Anschaulich \"{a}u{\ss}ert sich der Fehler als {\em drift\/}\index{drift} in den Knoten, \cite{Ha5}. Die exakte L\"{o}sung und die FE-L\"{o}sung sind in den Knoten nicht deckungsgleich. Das muss im Grunde auch so sein, es ist ein \glq Qualit\"{a}tsmerkmal\grq{}, weil die Interpolierende eine schlechtere L\"{o}sung als die FE-L\"{o}sung ist -- die Fehler in den Spannungen sind im Mittel gr\"{o}{\ss}er als bei der FE-L\"{o}sung.

Nur bei Stabtragwerken fallen Interpolierende $u_I$ und FE-L\"{o}sung $u_h$ zusammen, ist der Fehler in den Spannungen der beiden L\"{o}sungen daher gleich. Dieser Fehler wird behoben, indem das FE-Programm stabweise die lokalen L\"{o}sungen zur FE-L\"{o}sung addiert und so auf die exakten Ergebnisse kommt. Vergisst man diesen Schritt, dann ist die FE-L\"{o}sung $u_h = u_I$ nicht besser als die Interpolierende und die \glq L\"{u}cken\grq{} $u - u_h$ (\glq B\"{o}gen\grq{}, die von Knoten zu Knoten spannen) sind die  {\em lokalen Fehler\/}\index{lokaler Fehler} der FE-L\"{o}sung.\\

{\em Interpolierende:\/} \index{Interpolierende} Wenn man die exakte L\"{o}sung $u$ kennt und man interpoliert sie in den Knoten mit den {\em shape functions\/} aus $\mathcal{V}_h$, dann ist das die Interpolierende $u_I$ auf dem FE-Netz.

Viele Fehlersch\"{a}tzer basieren auf der Tatsache, dass der Fehler der FE-L\"{o}sung $\|u - u_h\|$ (gemessen in der Energienorm $\sim$ Fehlerquadrat der Spannungen) kleiner als der Fehler $\|u - u_I\|$ der Interpolierenden ist. Netze, auf denen man Funktionen gut interpolieren kann, sind auch gute FE-Netze, weil die FE-L\"{o}sung immer noch eine Idee besser ist als die Interpolierende.





%%%%%%%%%%%%%%%%%%%%%%%%%%%%%%%%%%%%%%%%%%%%%%%%%%%%%%%%%%%%%%%%%%%%%%%%%%%%%%%%%%%%%%%%%%%%%%%%%%%
\textcolor{chapterTitleBlue}{\chapter{Steifigkeits\"{a}nderungen und Reanalysis}}}\label{Steifigkeits\"{a}nderungen}
%%%%%%%%%%%%%%%%%%%%%%%%%%%%%%%%%%%%%%%%%%%%%%%%%%%%%%%%%%%%%%%%%%%%%%%%%%%%%%%%%%%%%%%%%%%%%%%%%%%

Das Thema in diesem Kapitel sind lokale Steifigkeits\"{a}nderungen in einzelnen Elementen, die sinngem\"{a}{\ss} zu einem neuen Satz von Gleichungen f\"{u}hren
 \begin{align}
(\vek K�+ \vek  \Delta \vek K)\,(\vek u + \vek \Delta \vek u) = \vek  f
\end{align}
und wir wollen die Effekte studieren, die solche Steifigkeits\"{a}nderungen in einem Tragwerk bewirken. Weil wir den Verschiebungsvektor $\vek u_c = \vek u + \vek \Delta \vek u$ des modifizierten Tragwerkes mittels der Matrix $\vek K$ des urspr\"{u}nglichen Tragwerkes
\begin{align}
\vek K\,\vek u = \vek f
\end{align}
berechnen, nennt man das auch {\em Reanalysis\/}\index{Reanalysis}.

In der Literatur werden viele Techniken diskutiert, mit denen sich solche \glq St\"{o}rungsrechnungen\grq{} numerisch behandeln lassen, s. z.B. \cite{Haug}. Wir konzentrieren uns hier aber auf die M\"{o}glichkeiten, die die Einflussfunktionen bieten, d.h. wir konzentrieren uns auf die Inverse $\vek K^{-1}$, denn Einflussfunktionen und Inverse sind Synonyme.

Die wesentliche Einsicht ist es, dass der neue Verschiebungsvektor $\vek u_c$ mit der urspr\"{u}nglichen Steifigkeitsmatrix berechnet werden kann, wenn man zur rechten Seite einen Vektor $\vek f^+$ addiert\index{$\vek f^+$}
\begin{align}
\vek K\,\vek u_c = \vek f + \vek f^+
\end{align}
und dass dieser Vektor $\vek f^+$ zu allen Starrk\"{o}rperbewegungen $\vek u_0 = \vek a + \vek x \times \vek b$ orthogonal ist, was impliziert, dass die Effekte von lokalen Steifigkeits\"{a}nderungen in der Regel rasch abklingen.

%----------------------------------------------------------------------------------------------------------
\begin{figure}[tbp]
\centering
\if \bild 2 \sidecaption \fi
\includegraphics[width=0.7\textwidth]{\Fpath/U230}
\caption{Steifigkeits\"{a}nderung} \label{U230}
\end{figure}%
%----------------------------------------------------------------------------------------------------------

%%%%%%%%%%%%%%%%%%%%%%%%%%%%%%%%%%%%%%%%%%%%%%%%%%%%%%%%%%%%%%%%%%%%%%%%%%%%%%%%%%%%%%%%%%%%%%%%%%%
\textcolor{sectionTitleBlue}{\section{Ein erster Versuch}}

Wir beginnen mit einem einfachen Beispiel, einem Stab, dessen L\"{a}ngsverschiebung $u(x)$ ja der Differentialgleichung $- EA\,u''(x) = p(x)$ gen\"{u}gt. Wenn man an dem Stab in Abb. \ref{U230} mit einer Kraft $f = 1 $ kN zieht, dann verl\"{a}ngert er sich, bei einer angenommenen L\"{a}ngssteifigkeit von $EA = 1$ kN, um den Betrag
\begin{align}
u = \frac{f \cdot l}{EA} = \frac{1 \cdot 2}{1} = 2\,\text{m}.
\end{align}
Wenn wir die L\"{a}ngssteifigkeit $EA$ verdoppeln, $EA = 2$, dann halbiert sich die L\"{a}ngsverschiebung
\begin{align}
u_c = \frac{f \cdot l}{EA} = \frac{1 \cdot 2}{2} = 1\,\text{m}.
\end{align}
Wir stellen uns jetzt die folgende Frage: Wie muss man die Zugkraft $f$ \"{a}ndern, $f \to f_c$, damit an dem urspr\"{u}nglichen Stab unter der Wirkung der modifizierten Zugkraft $f_c = f + f^+$ sich dieselbe  L\"{a}ngsverschiebung einstellt, wie an dem Stab mit der doppelten Steifigkeit?

Das f\"{u}hrt auf die Gleichung
\begin{align}
\frac{(f + f^+) \cdot l}{EA} = \frac{f \cdot l}{2 \cdot EA}= u_c
\end{align}
oder
\begin{align}
f^+ = -\frac{f}{2}\,.
\end{align}
Man darf also an dem urspr\"{u}nglichen System nur mit der halben Kraft ziehen,
\begin{align}
f_c = f + f^+ = f - \frac{f}{2} = \frac{f}{2}\,,
\end{align}
wenn man denselben Effekt haben will, wie an dem versteiften Stab.

Dieses Beispiel ist nat\"{u}rlich sehr einfach, aber wir erkennen hier schon die Grundidee.
%----------------------------------------------------------------------------------------------------------
\begin{figure}[tbp]
\centering
\if \bild 2 \sidecaption \fi
\includegraphics[width=0.65\textwidth]{\Fpath/U231}
\caption{Steifigkeits\"{a}nderung} \label{U231}
\end{figure}%
%----------------------------------------------------------------------------------------------------------
Steifigkeits\"{a}nderungen f\"{u}hren dazu, dass sich die Steifigkeitsmatrix \"{a}ndert, $\vek K \to \vek K + \vek \Delta \vek K$, und damit auch der Vektor der Knotenverschiebungen, $\vek u \to \vek u_c$,
\begin{align}
(\vek K + \vek \Delta \vek K)\,\vek u_c = \vek f\,.
\end{align}
Durch einfaches Umstellen sieht man, dass diese Gleichung mit dem System
\beq
\vek K\,\vek u_c = \vek f - \vek \Delta\,\vek K\,\vek u_c = \vek f + \vek f^+
\eeq
identisch ist. Der neue Vektor $\vek u_c$ kann also als L\"{o}sung des urspr\"{u}nglichen Systems gelten, wenn man zu der rechten Seite $\vek f$ den Vektor
\beq
\vek f^+ := - \vek \Delta\,\vek K\,\vek u_c
\eeq
addiert.

%%%%%%%%%%%%%%%%%%%%%%%%%%%%%%%%%%%%%%%%%%%%%%%%%%%%%%%%%%%%%%%%%%%%%%%%%%%%%%%%%%%%%%%%%%%%%%%%%%%
\textcolor{sectionTitleBlue}{\section{Zweites Beispiel}}
Bevor wir das weiter diskutieren, wollen wir noch ein zweites, einfaches Beispiel studieren. Der Stab in Abb. \ref{U231} ist in zwei Elemente mit der glei\-chen L\"{a}ngssteifigkeit $EA = 1$ kN unterteilt und eine Kraft $f_2 = 10$ kN zieht an seinem rechten Ende
\begin{align} \label{Eq42}
\left[ \barr {r @{\hspace{4mm}}r @{\hspace{4mm}}r
@{\hspace{4mm}}r @{\hspace{4mm}}r}
      2 & -1  \\
      -1 & 1 \\
     \earr \right]\left [\barr{c}  u_1 \\  u_2\earr \right ]
=  \left [\barr{c}  0 \\  10\earr \right ]\,.
\end{align}
Dieses System hat die L\"{o}sung $u_1 = 10\,\text{m}, u_2 = 20$ m.

Nun verdoppeln wir die Steifigkeit des rechten Elementes, $EA \to 2 EA$,
\begin{align}
\left[ \barr {r @{\hspace{4mm}}r @{\hspace{4mm}}r
@{\hspace{4mm}}r @{\hspace{4mm}}r}
      3 & -2  \\
      -2 & 2 \\
     \earr \right]\left [\barr{c}  u_1^c \\  u_2^c\earr \right ]
=  \left [\barr{c}  0 \\  10\earr \right ]
\end{align}
und wir erhalten so die L\"{o}sung $u_1^c = 10, u_2^c = 15$ m.

Wir fragen jetzt, wie muss man die rechte Seite $\vek f$ des Systems (\ref{Eq42}) modifizieren, damit das urspr\"{u}ngliche System dieselbe L\"{o}sung hat,
\begin{align}
\left[ \barr {r @{\hspace{4mm}}r @{\hspace{4mm}}r
@{\hspace{4mm}}r @{\hspace{4mm}}r}
      2 & -1  \\
      -1 & 1 \\
     \earr \right]\left [\barr{c}  u_1^c \\  u_2^c\earr \right ]
=  \left [\barr{c}  0 \\  10\earr \right ] + X\, \left[\barr{r}  1 \\  -1\earr \right ]\,.
\end{align}
Der letzte Vektor auf der rechten Seite ist der Vektor $\vek f^+$. Wir haben ihn von Anfang an so konstruiert, dass er ein Gleichgewichtsvektor ist, sich seine Komponenten gegenseitig aufheben.
Die Rechnung ergibt in der Tat, dass diese Annahme richtig war, denn mit $X = 5 $ ergibt sich eine L\"{o}sung.

Die gesuchten Knotenkr\"{a}fte lauten also
\begin{align}
\vek f + \vek f^+ = \left [\barr{c}  0 \\  10\earr \right ] + \left [\barr{r}  5 \\  -5\earr \right ] = \left [\barr{c}  5 \\  5\earr \right ]\,,
\end{align}
und wir \"{u}berzeugen uns leicht, dass die neuen Knotenverschiebungen $u_i^c $ genau die L\"{o}sungen des so modifizierten urspr\"{u}nglichen Systems sind
\begin{align}
\left[ \barr {r @{\hspace{4mm}}r @{\hspace{4mm}}r
@{\hspace{4mm}}r @{\hspace{4mm}}r}
      2 & -1  \\
      -1 & 1 \\
     \earr \right]\left [\barr{c}  10 \\  15\earr \right ]
=  \left [\barr{c}  5 \\  5\earr \right ]\,.
\end{align}

%%%%%%%%%%%%%%%%%%%%%%%%%%%%%%%%%%%%%%%%%%%%%%%%%%%%%%%%%%%%%%%%%%%%%%%%%%%%%%%%%%%%%%%%%%%%%%%%%%%
\textcolor{sectionTitleBlue}{\section{Strategie}}
Diese Ergebnisse nehmen wir nun zum Anlass unsere Strategie wie folgt zu formulieren: Wir \"{a}ndern nicht die Steifigkeitsmatrix, sondern die rechte Seite.
Wir suchen also eine Erg\"{a}nzung $\vek f^+$ zu dem Knotenkraftvektor $\vek f $ so, dass $\vek  f+ \vek f^+$ an dem System $\vek K $ dieselben Knotenverschiebungen verursacht, wie $\vek f $ an dem System $\vek K_c$.

Die Schwierigkeit dabei ist nat\"{u}rlich, dass der Vektor $\vek f^+ = -\vek \Delta\,\vek K\,\vek u_c$ von dem neuen Vektor $\vek u_c$ abh\"{a}ngt, den wir ja nicht kennen. (N\"{a}herungsweise kann man f\"{u}r $\vek u_c$ den Vektor $\vek u$ setzen).
Aber es geht uns nicht darum, eine neue Rechenmethode zur Nachverfolgung von Steifigkeits\"{a}nderungen in die Praxis einzuf\"{u}hren, sondern es geht uns prim\"{a}r um das statische Verst\"{a}ndnis.

Wir werden sehen, dass alle Steifigkeits\"{a}nderungen verstanden werden k\"{o}nnen, als die Addition von solchen {\em Gleichgewichts\-kr\"{a}ften\/} $\vek f^+$.

Das bedeutet: wenn $\vek u_0 = \vek a + \vek b \times \vek x$ eine Starrk\"{o}rperbewegung des Tragwerks ist, also eine Translation $\vek a$ plus einer m\"{o}glichen Drehung um eine Achse $\vek b$, dann ist die Arbeit der Kr\"{a}fte $\vek f^{+}$ null,
\begin{align}
\vek f^{+@T}\,\vek u_0 = 0\,.
\end{align}
Gleichgewichtskr\"{a}fte wie der Vektor $\vek f^+$ leisten also keine Arbeit auf Starrk\"{o}rperbewegungen. Daraus k\"{o}nnen wir, wie wir sehen werden, den folgenden Schluss  ziehen: \\

\hspace*{-12pt}\colorbox{highlightBlue}{\parbox{0.98\textwidth}{
Steifigkeits\"{a}nderungen sind in ihren Auswirkungen lokal begrenzt, weil sich die Wirkungen der Gleichgewichtskr\"{a}fte $\vek f^+$ in der Ferne aufheben.}}\\

%-----------------------------------------------------------------
\begin{figure}
\centering
\if \bild 2 \sidecaption \fi
{\includegraphics[width=0.6\textwidth]{\Fpath/U110}} \caption{Eine Steifigkeits\"{a}nderung $\vek K + \vek \Delta \vek K $ bedeutet, dass man ein Zusatzelement $\Omega_e^+$ mit der
 Steifigkeit $\vek \Delta \vek K $ an das Tragwerk anheftet, \cite{Ha6}}
\label{U110}%
\end{figure}%\\
%-----------------------------------------------------------------



%%%%%%%%%%%%%%%%%%%%%%%%%%%%%%%%%%%%%%%%%%%%%%%%%%%%%%%%%%%%%%%%%%%%%%%%%%%%%%%%%%%%%%%%%%%%%%%%%%%
\textcolor{sectionTitleBlue}{\section{Addition oder Subtraktion von Steifigkeiten}}

Die \"{A}nderung der Steifigkeit in einem Element kann man so deuten, dass man vor das urspr\"{u}ngliche Element ein zweites Element legt, dessen Steifigkeit gerade so gro{\ss} ist, dass die beiden Elemente zusammen die angezielte Steifigkeit haben, s. Abb. \ref{U110}.
%-----------------------------------------------------------------
\begin{figure}
\centering
\if \bild 2 \sidecaption \fi
{\includegraphics[width=0.8\textwidth]{\Fpath/U97}}
\caption{Zunahme und Abnahme der L\"{a}ngssteifigkeit $EA$ in dem ersten Element. Bei den Lastf\"{a}llen in der zweiten Reihe sind die L\"{a}ngsverformungen gleich den Lastf\"{a}llen in der ersten Reihe}
\label{U97}%
\end{figure}%
%-----------------------------------------------------------------

Das zus\"{a}tzliche Element muss nun durch Koppelkr\"{a}fte mit dem urspr\"{u}ng\-lichen Tragwerk synchron geschaltet werden, und die dazu n\"{o}tigen Koppelkr\"{a}fte sind gerade die Kr\"{a}fte $\vek f^+$. Das macht auch verst\"{a}ndlich, warum die Kr\"{a}fte $\vek f^+$ Gleichgewichtskr\"{a}fte sind, denn w\"{a}ren sie das nicht, dann w\"{u}rde das vorgeschaltete Element wegfliegen.

Dies gilt f\"{u}r eine Zunahme der Steifigkeit ebenso, wie f\"{u}r eine Abnahme der Steifigkeit. Wenn die Steifigkeit des Elementes gr\"{o}{\ss}er wird, dann haben die Koppelkr\"{a}fte $\vek f^+ $ die Tendenz, die Verformung des Elementes zu behindern, sie steifen sozusagen das Element aus.
%-----------------------------------------------------------------
\begin{figure}
\centering
\if \bild 2 \sidecaption \fi
{\includegraphics[width=0.8\textwidth]{\Fpath/U98}}
\caption{Zunahme und Abnahme der Biegesteifigkeit $EI$ in dem ersten Element. }
\label{U98}%
\end{figure}%
%-----------------------------------------------------------------
Umgekehrt, wenn die Steifigkeit in dem Element abnimmt, dann erh\"{o}hen die Koppelkr\"{a}fte $\vek f^+$ noch die Verformungen des Elementes, sie wirken wie zus\"{a}tzliche Lasten in den Knoten des Elementes, s. Abb. \ref{U97} und Abb. \ref{U98}.


%%%%%%%%%%%%%%%%%%%%%%%%%%%%%%%%%%%%%%%%%%%%%%%%%%%%%%%%%%%%%%%%%%%%%%%%%%%%%%%%%%%%%%%%%%%%%%%%%%%
\textcolor{sectionTitleBlue}{\section{Dipole und Monopole}}

Zwei gegengleiche Kr\"{a}fte $f_i^+ = \pm 1/h$, die \"{u}ber alle Grenzen wachsen, wenn ihr Abstand $h$ gegen null geht, bilden ein Dipol.

Bleiben die beiden gegengleichen Kr\"{a}fte hingegen auch im Grenzfall $h = 0$ endlich, dann nennen wir dies einen {\em Pseudo-Dipol\/}\index{Pseudo-Dipol}. Das Proton (+) und das Elektron (-) in einem Wasserstoffatom bilden einen solchen Pseudo-Dipol, und der Abstand der beiden entgegengesetzten Elementarladungen ist so klein, dass sich ihre Wirkungen auf eine Punktladung au{\ss}erhalb des Atomes praktisch aufheben.

In \"{a}hnlicher Weise stellen die Kr\"{a}fte $f_i^+$ Pseudo-Dipole dar, d.h. zu jeder aufw\"{a}rts gerichteten Kraft $f_i^+$ gibt es eine entgegengesetzt wirkende Kraft $f_j^+$, so dass die beiden Kr\"{a}fte $f^+$ aus der Ferne betrachtet einem Pseudo-Dipol gleichen, s. Abb. \ref{U113}.

Die Wirkung der Kr\"{a}fte $f_i^+$ auf irgendeinen Punkt $x$ des Tragwerks h\"{a}ngt davon ab, wie gro{\ss} die Laufzeitunterschiede von dem Punkt $x$ zu der Kraft $+f_i^+$ und der Gegenkraft $-f_i^+$ sind.
Wenn zwei Kr\"{a}fte $\pm f_i^+$ fast deckungsgleich sind, weil das Element $\Omega_e$ sehr klein ist, dann heben sich ihre Wirkungen nahezu auf, weil die Einflussfunktion sich auf dem winzigen Element kaum \"{a}ndert, $g' \simeq 0$.

Betrachten wir ein einfaches Beispiel. Ein Stabelement \"{a}ndere seine Steifigkeit, $EA_c = EA + \Delta EA$. In einem entfernten Element wird durch die Spreizung eines Punktes die Einflussfunktion f\"{u}r die Normalkraft in dem entfernten Element erzeugt.

Diese Einflussfunktion pflanzt sich nun bis zu dem Element $EA_c = EA + \Delta EA$ fort und wir beobachten jetzt, was dort passiert. Vereinbarungsgem\"{a}{\ss} wirken dort zwei Zusatzkr\"{a}fte $\pm f^+$, die den Effekt der Steifigkeits\"{a}nderung in dem Element nachbilden.
Am Anfang des Stabelementes mit der L\"{a}nge $l_e$ ziehe die Kraft $f_i^+$ nach links und am Ende ziehe eine gleichgro{\ss}e Kraft $f_{i + 1}^+$ nach rechts, und die Einflussfunktion f\"{u}r die Normalkraft
\begin{align}
G(y,x) = \sum_j g_j(x)\,\Np_j(y)
\end{align}
habe im linken Knoten den Wert $g_i$ und im rechten Knoten den Wert $g_{i + 1}$. Dann betr\"{a}gt der  Unterschied in der Normalkraft $N$, die wir als Funktional $N = J(u)$ lesen, also der Unterschied $N_{neu} - N_{alt} = N_c - N$,
\begin{align}
J(u_c) - J(u) = f_i^+ g_i - f_{i+1}^+ @g_{i+1} = f_i^+ \cdot (g_i - g_{i+1}) \simeq f_i^+ \cdot G'\cdot l_e\,.
\end{align}
Die Wirkung der Steifigkeits\"{a}nderung wird also nur dann merkbar sein, wenn die Einflussfunktion
in dem ge\"{a}nderten Element halbwegs merkbar ansteigt oder f\"{a}llt, $G' \gg 0$. Die Wirkung der $\pm f^+$ lebt also von der Differenz zwischen Elementanfang und Elementende, also kurz gesagt von $G'$.\\

\hspace*{-12pt}\colorbox{highlightBlue}{\parbox{0.98\textwidth}{Die Kr\"{a}ftepaare $\pm f_i^+$ registrieren die Unterschiede in den Einflussfunktionen (z.B.) am Elementanfang und Elementende, sie \glq differenzieren\grq{} die Einflussfunktionen. }}\\

Bei einem Balken registrieren die $\pm f^+$, es sind jetzt Momente, die Unterschiede in der ersten Ableitung der Einflussfunktion an den Balkenenden,
\begin{align}
J(u_c) - J(u) = f_i^+ g_i' - f_{i+1}^+ @g_{i+1}' = f_i^+ \cdot (g_i' - g_{i+1}') \simeq f_i^+ \cdot G''\cdot l_e
\end{align}
sie reagieren also auf die Gr\"{o}{\ss}e der zweiten Ableitung, der  Momente der Einflussfunktion in dem Element. (Der Balken sei fest gelagert).
%-----------------------------------------------------------------
\begin{figure}[tbp]
\centering
\includegraphics[width=0.9\textwidth]{\Fpath/U113}
\caption{Steifigkeits\"{a}nderung in einem Element und die zugeh\"{o}rigen Koppelkr\"{a}fte $\vek f_i^+$. Diese Kr\"{a}fte folgen den Hauptspannungsrichtungen (- - -)  und es sind Gleichgewichtskr\"{a}fte, die Pseudo-Dipolen gleichen.}
\label{U113}
\end{figure}%
%-----------------------------------------------------------------

%%%%%%%%%%%%%%%%%%%%%%%%%%%%%%%%%%%%%%%%%%%%%%%%%%%%%%%%%%%%%%%%%%%%%%%%%%%%%%%%%%%%%%%%%%%%%%%%%%%
\textcolor{sectionTitleBlue}{\section{Weggr\"{o}{\ss}en und Kraftgr\"{o}{\ss}en}}

Betrachten wir nun die \"{A}nderungen und die Rolle, die der Vektor $\vek f^+$ dabei spielt, etwas systematischer. Es sei zun\"{a}chst $J(u)$ eine Weggr\"{o}{\ss}e, also etwa $u(x)$ oder $w(x)$. Die Ergebnisse vor und nach der Steifigkeits\"{a}nderung lauten
\begin{align}
J(u) = \vek g^T \vek f = \vek u^T \vek j
\end{align}
bzw.
\begin{align}
J(u_c) = \vek g_c^T \vek f = \vek u_c^T \vek j_c\,.
\end{align}
Nun ist kein Unterschied zwischen den beiden Vektoren $\vek j$ und $\vek j_c$, weil in die Definition der Weggr\"{o}{\ss}en die Steifigkeiten nicht eingehen.
%-----------------------------------------------------------------
\begin{figure}[tbp]
\centering
\includegraphics[width=0.9\textwidth]{\Fpath/U383}
\caption{Steifigkeits\"{a}nderung in einem Stab unter einer Gleichstreckenlast, die $f_i^+$ sind antimetrisch, die EF-$u$ ist symmetrisch und die EF-$N$ ist antimetrisch. Man beachte aber, dass die genaue \"{A}nderung $N_c - N$ nicht einfach die $f_i^+$ mal den Knotenverschiebungen der Einflussfunktion EF-$N$ ist, s. (\ref{Eq137})}\label{Korrektur28}
\label{U383}
\end{figure}%
%-----------------------------------------------------------------

Ist z.B. $J(u) = u(x)$ die Verschiebung in einem Punkt $x$, dann ist
\begin{align}
\vek j = \{\Np_1(x), \Np_2(x), \ldots, \Np_n(x) \}^T = \vek j_c
\end{align}
und so folgt
\begin{align}
J(u_c) - J(u) &= \vek j^T (\vek u_c - \vek u) = \vek j^T\vek K^{-1}(\vek f + \vek f^+) - \vek j^T \vek K^{-1} \vek f\nn\\
 &= \vek j^T\,\vek K^{-1} \vek f^+ = \vek g^T \vek f^+\,.
\end{align}
Bei Kraftgr\"{o}{\ss}en ist das unter Umst\"{a}nden anders. Wenn der Aufpunkt $x$ auf dem Element liegt, dessen Steifigkeit sich \"{a}ndert, $EA \to EA_c$, dann sind, etwa im Fall $J(u) = EA\,u'(x)$, die Vektoren $\vek j$
\begin{align}
\vek j = \{EA\,\Np_1'(x), EA\, \Np_2'(x), \ldots, EA\,\Np_n'(x) \}^T
\end{align}
und
\begin{align}
\vek j_c = \{EA_c\,\Np_1'(x), EA_c\, \Np_2'(x), \ldots, EA_c\,\Np_n'(x) \}^T
\end{align}
nicht gleich\footnote{Nur die $\Np_i(x)$, die einen \glq Fu{\ss}\grq{} auf dem Element mit dem Aufpunkt $x$ haben, sind in dem Vektor $\vek j$ bzw. $\vek j_c$ nicht null, so dass diese Vektoren ziemlich leer sind. }, so dass nun das Ergebnis lautet
\begin{align}\label{Eq102}
J(u_c) - J(u) &= \vek j_c^T \vek K^{-1}(\vek f + \vek f^+) - \vek j^T \vek K^{-1}\vek f\nn \\
&= (\vek j_c - \vek j)^T \vek u + \vek j_c^T \vek K^{-1}\,\vek f^+  \,.
\end{align}
Auf der rechten Seite stehen die Korrekturterme, um von $J(u)$ auf $J(u_c)$ zu kommen, wobei man die erste Korrektur, um den Fehler in der Steifigkeit, $EA$ statt $EA_c$, zu beheben, an $\vek u$ vornehmen kann, und der zweite Term ist der Einfluss der Kr\"{a}fte $\vek f^+$ auf die Schnittgr\"{o}{\ss}en.

Setzen wir $\vek u^+ = \vek K^{-1}\,\vek f^+$\index{$\vek u^+$} und sei $\vek j_c = \alpha\,\vek j$, dann ist dasselbe wie
\begin{align}\label{Eq137}
J(u_c) - J(u) = \vek j_c^T (\vek u + \vek u^+) - \vek j^T\,\vek u = (\alpha -1)\,\vek g^T\,\vek f + \alpha\,\vek g^T\,\vek f^+\,.
\end{align}
Wenn der Punkt $x$ auf einem Element liegt, dessen Steifigkeit sich nicht \"{a}ndert, dann ist wegen $\vek j_c = \vek j$ die Formel dieselbe, wie bei den Weggr\"{o}{\ss}en
\begin{align}
J(u_c) - J(u) = \vek j^T\,\vek u^+ = \vek j^T\,\vek K^{-1} \vek f^+ = \vek g^T \vek f^+\,.
\end{align}
Bei dem Stab in Abb. \ref{U383} halbiert sich im zweiten Element die L\"{a}ngssteifigkeit von $EA = 1$ auf $EA_c = 0.5$. Die Belastung ist eine Gleichlast von 1 kN/m. Die Elementl\"{a}nge betr\"{a}gt $l_e = 1.0$ m. Die Matrizen lauten
\begin{align}
\vek K = \left[ \barr {r @{\hspace{4mm}}r @{\hspace{4mm}}r @{\hspace{4mm}}r} 2 & -1 & 0 & 0 \\ -1 & 2 & -1 & 0\\ 0 &-1 &2 &-1\\ 0 & 0 &-1 &2\earr \right]
\qquad \vek  \Delta \vek K = \left[ \barr {r @{\hspace{4mm}}r @{\hspace{4mm}}r @{\hspace{4mm}}r} -0.5 & 0.5 & 0 & 0 \\ 0.5 & -0.5 & 0 & 0\\ 0 &0 &0 &0\\ 0 & 0 &0 &0\earr \right]
\end{align}
und die Vektoren haben die Gestalt
\begin{align}
\vek u = \left[ \barr {r } 2.0 \\ 3.0 \\ 3.0 \\ 2.0 \earr \right]\quad \vek u_c = \left[ \barr {r } 1.833 \\3.500 \\3.333 \\2.167 \earr \right] \quad \vek f^+ = \left[ \barr {r } -0.833 \\ 0.833 \\ 0.0 \\ 0.0 \earr \right] \quad \vek j =\left[ \barr {r } -1.0 \\ 1.0 \\ 0.0 \\ 0.0 \earr \right]\,.
\end{align}
Es ist $\vek j_c = 0.5 \cdot \vek j$ wegen $EA_c = 0.5\,EA$.

Die Differenz der Normalkr\"{a}fte in der Mitte des zweiten Elements ergibt sich somit zu
\begin{align}
N_c - N = J(\vek u_c) - J(\vek u) = - 0.5\,\vek j^T\,\vek u + 0.5\,\vek j^T\,\vek K^{-1}\,\vek f^+ = -0.1665\,.
\end{align}
In der Praxis berechnet man $N_c - N$ nat\"{u}rlich direkt aus $\vek u_c$ und $\vek u$. Weil die Elemente linear sind ist $N_c = 0.5 \cdot (3.500 - 1.833) = 0.8335$ und $N = 1.0 \cdot (3.0 - 2.0) = 1.0 $. Es ging hier prim\"{a}r um den Nachweis, dass die obige Formel (\ref{Eq102}) das richtige Ergebnis liefert.

%%%%%%%%%%%%%%%%%%%%%%%%%%%%%%%%%%%%%%%%%%%%%%%%%%%%%%%%%%%%%%%%%%%%%%%%%%%%%%%%%%%%%%%%%%%%%%%%%%%
\textcolor{sectionTitleBlue}{\section{Symmetrie und Antimetrie}}\label{Korrektur26}
Die Einflussfunktionen f\"{u}r Verschiebungen bei einem Stab (Scheibe) werden durch Monopole erzeugt
und die Einflussfunktionen f\"{u}r Normalkr\"{a}fte (Spannungen) durch Dipole. Nun sind ja die $f_i^+$ antimetrisch und somit sollten die Effekte von Steifigkeits\"{a}nderungen bei den Spannungen ({\em antimetrisch $\times$ antimetrisch\/}) gr\"{o}{\ss}er sein als bei den Verschiebungen ({\em antimetrisch $\times$ symmetrisch\/}). Mit
\begin{align}
 \frac{N_c - N}{N} = \frac{0.8335 - 1.0}{1.0}= - 0.1665 \qquad \frac{u_c - u}{u} = \frac{2.667 - 2.5}{2.5} = 0.0666
\end{align}
best\"{a}tigt sich das bei diesem Beispiel.

Bei statisch bestimmten Systemen ist es aber gerade umgekehrt: Die Verformungen \"{a}ndern sich, aber die Kr\"{a}fte nicht. Der Grund ist, dass die $f_i^+$ Gleichgewichtskr\"{a}fte sind, die auf den st\"{u}ckweise linearen Einflussfunktionen (f\"{u}r $N, M, V$) in der Summe null Arbeit leisten. Die Einflussfunktionen f\"{u}r die Verformungen sind jedoch \glq krumme\grq{} Linien zu denen die $f_i^+$ im energetischen Sinne nicht orthogonal sind. Deswegen \"{a}ndern sich die Verformungen. \label{Korrektur8}

%-----------------------------------------------------------------
\begin{figure}[tbp]
\centering
\includegraphics[width=0.99\textwidth]{\Fpath/U428}
\caption{Einflussfunktion f\"{u}r eine Querkraft. In den Stielen des obersten Stockwerks ist der Verlauf praktisch konstant, was bedeutet, dass \"{A}nderungen $EI \pm \Delta EI$ in den Stielen sehr geringen Einfluss auf die Querkraft haben werden, weil die zugeh\"{o}rigen $f_i^+$ zu diesen Bewegungen orthogonal sind}\label{Korrektur34}
\label{U428}
\end{figure}%
%-----------------------------------------------------------------

%-----------------------------------------------------------------
\begin{figure}[tbp]
\centering
\includegraphics[width=0.99\textwidth]{\Fpath/U429}
\caption{Einflussfl\"{a}che f\"{u}r das Moment $m_{yy}$ einer zweiseitig gelagerten Decke. Die lange Seite ist eingespannt, die kurze gelenkig gelagert. Im rechten Teil verl\"{a}uft die Fl\"{a}che nahezu linear, so dass Steifigkeits\"{a}nderungen in diesem Teil einen geringen Einfluss auf das Moment $m_{yy}$ im Aufpunkt haben sollten}\label{Korrektur35}
\label{U429}
\end{figure}%
%-----------------------------------------------------------------
%%%%%%%%%%%%%%%%%%%%%%%%%%%%%%%%%%%%%%%%%%%%%%%%%%%%%%%%%%%%%%%%%%%%%%%%%%%%%%%%%%%%%%%%%%%%%%%%%%%
\textcolor{sectionTitleBlue}{\section{Das Abklingen der Effekte}}

Je weiter man sich vom Aufpunkt entfernt, um so \glq linearer\grq{} wird die Einflussfunktion, um so mehr dominieren die linearen Anteile in den Einflussfunktionen, d.h. der Vektor $\vek g$ der Knotenverschiebungen gleicht mit wachsendem Abstand mehr und mehr einem Vektor $\vek u_0= \vek a + \vek x \times \vek b$ und das bedeutet, weil der Vektor $\vek f^+$ orthogonal zu den Vektoren $\vek u_0$ ist, dass der Einfluss der Kr\"{a}fte $\vek f^+$ mit wachsendem Abstand vom Aufpunkt gegen null tendiert\\

\hspace*{-12pt}\colorbox{highlightBlue}{\parbox{0.98\textwidth}{
\beq
J(e) =  J(u_c) - J(u)  = \vek g^T\,\vek f^+ \simeq (\vek a + \vek x \times \vek b)^T\,\vek f^+ = 0\,.
\eeq
Dies ist der Grund, warum es m\"{o}glich ist, mit gemittelten Materialparametern genaue Ergebnisse zu erhalten.}}\\

\vspace{0.7cm}

Wenn $J(u)$ eine Kraftgr\"{o}{\ss}e ist und der Aufpunkt auf dem betroffenen Element liegt, dann ist, wie oben gezeigt, die Formel um die Korrektur $EA_c - EA$ auf dem Element zu erweitern
\begin{align}
J(e) = J(u_c) - J(u) &= \underbrace{(\vek j_c - \vek j)^T \vek u}_{Korrektur} + \vek j_c^T \vek K^{-1}\,\vek f^+\,.
\end{align}
Die erste Korrektur ist rein lokal, geschieht nur im Aufpunkt, w\"{a}hrend der zweite Term den Einfluss der Kr\"{a}fte $f_i^+$ erfasst, deren Effekte aber mit der Entfernung relativ rasch abklingen.


%%%%%%%%%%%%%%%%%%%%%%%%%%%%%%%%%%%%%%%%%%%%%%%%%%%%%%%%%%%%%%%%%%%%%%%%%%%%%%%%%%%%%%%%%%%%%%%%%%%
\textcolor{sectionTitleBlue}{\section{Die Bedeutung f\"{u}r die Praxis}}
Die Bedeutung dieser Ergebnisse f\"{u}r die Praxis liegt darin, dass sie erkl\"{a}ren, warum {\em Homogenisierungsmethoden\/}\index{Homogenisierungsmethoden} erfolgreich sind.

Beton setzt sich aus den unterschiedlichsten Kiessorten und Zementstein zusammen. Jedes Zuschlagskorn hat ja einen anderen Elastizit\"{a}tsmodul und daher m\"{u}ssten wir eigentlich jedes Zuschlagskorn durch ein eigenes finites Element modellieren. Stattdessen rechnen wir aber mit einem gemittelten Elastizit\"{a}tsmodul und erhalten durchaus glaubhafte Ergebnisse.
%-----------------------------------------------------------------
\begin{figure}[tbp]
\centering
\includegraphics[width=0.9\textwidth]{\Fpath/U114}
\caption{Eigengewicht und Kr\"{a}fte $f_i^+$. Die Scheibe wurde erst mit einem einheitlichen E-Modul $E = 1$ berechnet und dann wurde der E-Modul in den Elementen
zuf\"{a}llig, $0.5 < E_i < 1.5$, variiert, und es wurden die Kr\"{a}fte $f_i^+$ berechnet. Diese Kr\"{a}fte $f_i^+$ plus den Kr\"{a}ften $f_i$ aus dem Lastfall Eigengewicht
erzeugen in dem Modell $\vek K$ den Verschiebungsvektor $\vek u_c$ der Scheibe, $\vek K\,\vek u_c = \vek f + \vek f^+$. Es ist derselbe Vektor $\vek u_c$ wie in dem
Modell $\vek K_c\,\vek u_c = \vek f$ wobei die Matrix $\vek K_c$ auf den zuf\"{a}llig gestreuten Werten $E_i$ beruht. }
\label{U114}
\end{figure}%
%-----------------------------------------------------------------

%-----------------------------------------------------------------
\begin{figure}[tbp]
\centering
\includegraphics[width=0.9\textwidth]{\Fpath/U421}\label{Korrektur27}
\caption{Wandscheibe unter Eigengewicht, in den Elementen mit den Knotenkr\"{a}ften $f_i^+$ wurde der E-Modul um 90 \% verringert. Im Grunde verhalten sich die geschw\"{a}chten Bereiche wie \"{O}ffnungen. Bemerkenswert ist, dass die Kr\"{a}fte $f_i^+$ auf den Rand konzentriert sind, sie ziehen die \glq \"{O}ffnung\grq{} zusammen. Die Analogie ist naheliegend: In einem St\"{u}ck Blech, das man zur Verst\"{a}rkung mit Heftn\"{a}hten auf eine Stahlwand schwei{\ss}t, d\"{u}rften dieselben Kr\"{a}fte auftreten, nur dass sie kontinuierlich \"{u}ber den geschwei{\ss}ten Rand verteilt sind und die umgekehrte Richtung haben   }
\label{U421}
\end{figure}%
%-----------------------------------------------------------------
Dies d\"{u}rfte wesentlich daran liegen, dass die Knotenkr\"{a}fte $f_i^+$ auf der H\"{u}lle des Zuschlagskorns, mit denen wir ja die Abweichungen des Elastizit\"{a}tsmoduls vom Mittelwert korrigieren, Gleichgewichtskr\"{a}fte sind, die nahe beieinanderliegen, und ihre Fernwirkungen daher gegen null tendieren.
%-----------------------------------------------------------------
\begin{figure}[tbp]
\centering
\includegraphics[width=0.95\textwidth]{\Fpath/U362}  % war UE234
\caption{Ausfall einer St\"{u}tze zwischen zwei Decken und zwischen der untersten Decke und dem Fundament}
\label{U362}
\end{figure}%
%-----------------------------------------------------------------

Eine Scheibe $\Omega$ bestehe z.Bsp. aus einer Reihe von unterschiedlichen Elementen, $\Omega = \Omega_1 \cup \Omega_2 \cup \ldots \Omega_n$, die alle einen eigenen $E$-Modul $E_i$ aufweisen, der um einen Betrag $\Delta E_i = E - E_i$ von dem Mittelwert $E$ abweicht, s. Abb. \ref{U114}. Der exakte Knotenverschiebungsvektor $\vek u_c$ w\"{a}re daher die L\"{o}sung des Systems
\beq
\vek K_c\,\vek u_c = \vek f\,,
\eeq
wobei die Matrix  $\vek K_c$ sich aus den unterschiedlichen Elementmatrizen $\vek K_e(E_i)$ zusammensetzt. Wenn man hingegen mit einem einheitlichen $E$-Modul rechnet, also einer vereinfachten Matrix $\vek K$,
\beq
\vek K\,\vek u = \vek f\,,
\eeq
dann ist der Fehler in einer Verschiebung
\beq
J(u_c) - J(u) = \vek g^T\,(\vek f + \vek f^+) -  \vek g^T\,\vek f = \vek g^T\,\vek f^+
\eeq
relativ klein, weil die Kr\"{a}fte $f_i^+$, die von den Fehlertermen $\Delta E_i = E_i - E$ herr\"{u}hren
\beq
\vek K\,\vek u_c = \vek f + \vek f^+\,,
\eeq
zum einen $(1)$ Gleichgewichtsgruppen bilden und $(2)$ zum andern sich positive Abweichungen $\Delta E_i > 0$ und negative Abweichungen $\Delta E_j < 0$ ungef\"{a}hr die Waage halten werden, so dass diese beiden Effekte zusammen daf\"{u}r sorgen, dass die Fernfeldfehler klein sein werden, s. Abb. \ref{U114}. Man muss nicht jedes Zuschlagskorn durch ein eigenes Element darstellen, die Mathematik sorgt daf\"{u}r, dass sich die Fehler aufheben.

Bei der ersten Wandscheibe in Abb. \ref{U114} war die Streuung in den Steifigkeiten zuf\"{a}llig verteilt, jetzt in den Bildern \ref{U421} a und \ref{U421} b gibt es zwei Bereiche, in denen die Steifigkeit aller Elemente um denselben Betrag, um $90 \%$ reduziert wurde. Bemerkenswert ist hierbei die Dominanz der Kr\"{a}fte $f_i^+$ am Rand der beiden Bereiche, s. Abb. \ref{U421}.

%-----------------------------------------------------------------
\begin{figure}[tbp]
\centering
\includegraphics[width=0.75\textwidth]{\Fpath/U243}
\caption{Ausfall einer Eckst\"{u}tze, \textbf{ a)} Momente im LF $g$, \textbf{ b)} Kraft $f^+$ und zugeh\"{o}rige Momente, \textbf{ c)} a+ b = Momente nach Ausfall der Eckst\"{u}tze}
\label{U243}
\end{figure}%
%-----------------------------------------------------------------

Man ist versucht, weil die Kr\"{a}fte $f_i^+$ Gleichgewichtskr\"{a}fte sind, die immer paarweise auftreten, die durch die $f_i^+$ ausgel\"{o}sten Effekte zu ignorieren. Wenn, wie in  Abb. \ref{U362},  eine St\"{u}tze zwischen zwei Geschossen ausf\"{a}llt, dann kann man das am Originaltragwerk durch den Angriff von zwei gegengleichen Knotenkr\"{a}ften $f_i^+$ korrigieren und weil beide gleich gro{\ss} sind, heben sich ihre Wirkungen in der Ferne auf. Das ist richtig.


Wenn aber eine Fundamentst\"{u}tze ausf\"{a}llt, dann wirken zwar auch wieder zwei Kr\"{a}fte $\pm f_i^+$, aber die untere Kraft ist am Boden verankert und so bleibt von dem Paar $\pm f_i^+$ nur die Kraft $f_i^+$ am St\"{u}tzenkopf \"{u}brig, die, weil sie keinen Antagonisten hat, weiter ausstrahlen wird, als ein gegengleiches Kr\"{a}ftepaar $\pm f_i^+$. \\

\hspace*{-12pt}\colorbox{highlightBlue}{\parbox{0.98\textwidth}{
Der Ausfall einer Fundamentst\"{u}tze ist kritischer, als der Ausfall einer Zwischenst\"{u}tze.}}\\

%%%%%%%%%%%%%%%%%%%%%%%%%%%%%%%%%%%%%%%%%%%%%%%%%%%%%%%%%%%%%%%%%%%%%%%%%%%%%%%%%%%%%%%%%%%%%%%%%%%
\textcolor{sectionTitleBlue}{\section{Rahmen}}
In Abb. \ref{U243} a sind die Momente eines Rahmens im LF $g$ dargestellt und darunter, in Abb. \ref{U243} b, die Momente im selben Lastfall, wenn eine Gescho{\ss}st\"{u}tze ausf\"{a}llt. Statisch entspricht der \"{U}bergang von Bild a zu Bild c der Wirkung von Zusatzkr\"{a}ften $f_i^+$, wie in Abb. \ref{U243} b gezeigt,
\begin{align}
(\vek K + \vek \Delta \,\vek K)\,\vek u_c = \vek f \qquad \vek K\,\vek u_c = \vek f + \vek f^+\,.
\end{align}
Die Momente, die diese Kr\"{a}fte $f_i^+$ erzeugen, zu den Momenten des Ausgangszustandes addiert, ergeben die Momente des geschw\"{a}chten Systems.

%Das Beispiel des aufgest\"{a}nderten Balkens in Abb. \ref{U246} soll demonstrieren, dass lokale \"{A}nderungen in den Steifigkeiten zu wellenf\"{o}rmigen Effekten f\"{u}hren, die aber rasch abklingen.

%-----------------------------------------------------------------
\begin{figure}[tbp]
\centering
\includegraphics[width=0.8\textwidth]{\Fpath/U296}
\caption{Biegebalken, \textbf{ a)} Gleichlast, \textbf{ b)} Einflussfunktion f\"{u}r $M(x)$ erzeugt durch zwei gegengleiche Momente $M_l = M_r$, \textbf{ c)} Lagersenkung $\Delta = 1$}
\label{U296}
\end{figure}%
%-----------------------------------------------------------------
\pagebreak
%%%%%%%%%%%%%%%%%%%%%%%%%%%%%%%%%%%%%%%%%%%%%%%%%%%%%%%%%%%%%%%%%%%%%%%%%%%%%%%%%%%%%%%%%%%%%%%%%%%
\textcolor{sectionTitleBlue}{\section{Starre Lager}}
Ein elastisches Lager ($k$) gibt unter Last um einen Weg $u$ nach. Ein eventueller Ausfall des Lagers entspricht einem Sprung $\Delta k = -k$ in der Lagersteifigkeit und so muss eine Kraft $f^+ = - (-k)\,u_c = k\,u_c$ zur rechten Seite addiert werden, um rechnerisch den Ausfall des Lagers zu kompensieren.

Bei starren Lager ist die $f^+$-Technik nicht anwendbar, weil ein Lagerweg $u_c $ n\"{o}tig ist, um eine Kraft $f^+ = \Delta k\,u_c$ zu generieren. In einer solchen Situation w\"{u}rde man die Lagerkraft in umgekehrter Richtung auf das modifizierte Tragwerk aufbringen (nach dem Ausfall des Lagers) und die Ergebnisse zu den urspr\"{u}nglichen Ergebnissen addieren.

Wie wir die Effekte erfassen, die der Ausfall eines starren Lagers auf die inneren Kr\"{a}fte hat, wollen wir beispielhaft an dem Tr\"{a}ger in Abb. \ref{U296} zeigen. Es soll untersucht werden, wie sich das Biegemoment in der Mitte des Balkens \"{a}ndert, wenn das starre Lager am Ende des Balkens ausf\"{a}llt. (Der Praktiker w\"{u}rde dieses Problem nat\"{u}rlich ganz anders und schneller l\"{o}sen, aber es geht hier um die Systematik, die auch noch greift, wenn 10 Stockwerke \"{u}bereinander liegen).

Zuerst zeigen wir, dass die Lagerreaktion $R_G$, die zur Einflussfunktion f\"{u}r das Moment $M(x)$ in Abb. \ref{U296} b geh\"{o}rt, gleich dem Moment $M(x)$ ist, wenn sich das Lager um einen Meter senkt, $\Delta = 1$,\footnote{Das entspricht der Formel $J_1(G_2) = J_2(G_1)$ Mit $J_1(w) = R$ (Lagerreaktion die zu $w$ geh\"{o}rt) und $J_2(w) = M(x)$ (Moment von $w$ in Balkenmitte) mit $G_2$ wie in Abb. \textbf{ b)} und $G_1 = w_\Delta$ wie in Abb. \textbf{ c)}.}
\begin{align}
R_G = M(x) \qquad \text{aus Setzung $\Delta = 1$}\,.
\end{align}
Das ergibt sich aus dem {\em Satz von Betti\/} $\text{\normalfont\calligra B\,\,}(G_2,w_\Delta) = 0$
\begin{align}
A_{1,2} = \underbrace{M_l\,w_\Delta'(x_{-}) - M_r\,w_\Delta'(x_+)}_{= \,0} + R_G \cdot 1 = M(x) \cdot 1 + R_\Delta \cdot 0 = A_{2,1} \,.
\end{align}
Wenn das Lager unter einer Last $p$ ausf\"{a}llt, dann muss die vorherige Lagerkraft $R_p$ als \"{a}u{\ss}ere Kraft und in umgekehrter Richtung aufgebracht werden. Das f\"{u}hrt dazu, dass sich das Balkenende um den Betrag
\begin{align}\label{Eq118}
w(l) = R_p \cdot \frac{1}{k_S} \qquad k_S =  \frac{3\,EI}{l^3}
\end{align}
senkt. Wenn eine Setzung um einen Meter, $\Delta = 1$, ein Moment $M(x) = R_G$ verursacht, dann verursacht die Durchbiegung in (\ref{Eq118}) das Moment
\begin{align}\label{Eq119}
M(x) = R_G \cdot R_p \cdot \frac{1}{k_S} = R_G \cdot \text{Durchbiegung aus $R_p$} \,.
\end{align}
Das Moment $M(x) $ ist das $\Delta M(x) $, das zu dem urspr\"{u}nglichen Moment im Punkt $x$ addiert werden muss, um den Verlust des starren Lagers auszugleichen.
%-----------------------------------------------------------------
\begin{figure}[tbp]
\centering
\includegraphics[width=0.99\textwidth]{\Fpath/U540}
\caption{Das rasche Abklingen der Momente aus den $X_i$ bei einem Durchlauftr\"{a}ger (ein \glq Urph\"{a}nomen\grq{} der Statik) weist auf die enge Verwandtschaft zwischen den $f_i^+$ und den $X_i $ hin; die $f_i^+$ sind sozusagen die $X_i$ beim \"{U}bergang vom System $\vek K $ zum System $ \vek K + \vek \Delta \vek K$}
\label{U540}
\end{figure}%
%-----------------------------------------------------------------

Die Zahl $k_S$ ist die Steifigkeit der Struktur -- nach dem Ausfall des Lagers -- in Richtung von $w$. Wenn in dem Lager eine gewisse Reststeifigkeit $k_R$ verbleibt, dann bilden $k_S$ und $k_R$ eine Federkette und dann muss $1/k_S$ sinngem\"{a}{\ss} durch $1/k$ ersetzt werden
\begin{align}
\frac{1}{k} = \frac{1}{k_S} + \frac{1}{k_R}\,.
\end{align}

%-----------------------------------------------------------------
\begin{figure}[tbp]
\centering
\includegraphics[width=0.8\textwidth]{\Fpath/U394a}
\caption{Berechnung der Einflussfunktion f\"{u}r eine Eckverschiebung mit einem Dirac Delta, das in diskreter Form einer Knotenkraft $j$ gleich ist. \"{A}ndert sich in einem Element die Steifigkeit, dann braucht man auf der rechten Seite, $\vek K\,\vek g_c = \vek j + \vek j^+$, zus\"{a}tzliche Knotenkr\"{a}fte $\vek j^+$, wenn man mit der Matrix $\vek K$ rechnet; s. auch Abb. \ref{U369} auf S. \pageref{U369}}
\label{U394}
\end{figure}%
%-----------------------------------------------------------------

%%%%%%%%%%%%%%%%%%%%%%%%%%%%%%%%%%%%%%%%%%%%%%%%%%%%%%%%%%%%%%%%%%%%%%%%%%%%%%%%%%%%%%%%%%%%%%%%%%%
\textcolor{sectionTitleBlue}{\section{Das Kraftgr\"{o}{\ss}enverfahren}}

Zwischen dem Kraftgr\"{o}{\ss}enverfahren und den  $f_i^+$ besteht ein enger Zusammenhang, denn beim Rechnen mit den $f_i^+$ \"{a}ndern wir nicht die Steifigkeitsmatrix, sondern wir \"{a}ndern die rechte Seite, aus dem Vektor $\vek f$ wird der Vektor $\vek f + \vek f^+$. Genauso geht auch das Kraftgr\"{o}{\ss}enverfahren vor.

Das Kraftgr\"{o}{\ss}enverfahren w\"{a}hlt ein statisch bestimmtes {\em Hauptsystem\/} und in der Folge findet alles Rechnen an diesem System statt. Die statisch \"{U}berz\"{a}hligen $X_i $ spielen dabei dieselbe Rolle wie die $f_i^+$. W\"{a}hrend die $f_i^+ $ das zus\"{a}tzliche Element an die Struktur koppeln, beseitigen die  $X_i $ die Klaffungen. Beide, die $X_i$ wie die $f_i^+$, sind {\em Zusatzlasten\/}, die auf der rechten Seite erscheinen, w\"{a}hrend die eigentliche Analyse am unver\"{a}nderten Hauptsystem (Matrix $\vek K$) vor sich geht.

Das hat den Vorteil, dass wir nicht zwei S\"{a}tze von Einflussfunktionen brauchen: Einen Satz f\"{u}r die Schnittgr\"{o}{\ss}en am Hauptsystem (Matrix $\vek K$) und einen zweiten Satz f\"{u}r das statisch unbestimmte System (Matrix $\vek K_c$).

Es gibt aber noch, wie Abb. \ref{U540} zeigt, eine weitere Verwandtschaft zwischen den $f_i$ und den $X_i$. Die $f_i^+$ sind sozusagen die $X_i$, die das System $\vek K $ auf das Niveau $\vek K + \vek \Delta \vek K $ heben.

%%%%%%%%%%%%%%%%%%%%%%%%%%%%%%%%%%%%%%%%%%%%%%%%%%%%%%%%%%%%%%%%%%%%%%%%%%%%%%%%%%%%%%%%%%%%%%%%%%%
\textcolor{sectionTitleBlue}{\section{Kr\"{a}fte $\vek j^+$}} \label{Korrektur1}\label{jplus}

Die Formel $J(e) = \vek g^T\,\vek f^+$ legt nahe zu probieren, ob es nicht auch eine Formel
\begin{align}
J(e) = \vek u^T\,\vek j^+
\end{align}
gibt, s. Abb. \ref{U394}.\index{$\vek j^+$}

Und die gibt es in der Tat, denn die Verschiebung $u(x)$ in einem Punkt kann man auf zwei Arten berechnen
\begin{align}
u(x) = \int_0^{\,l} G(y,x)\,p(y)\,dy = \int_0^{\,l} u(y)\,\delta(y-x)\,dy\,.
\end{align}
In der linearen Algebra der finiten Elemente lautet die zweite Gleichung
\begin{align}
J(u) = u(x) = \vek u^T\,\vek j(x)\,,
\end{align}
wenn $\vek j(x)$ die \"{a}quivalenten Knotenkr\"{a}fte zu einem Dirac Delta sind, das im Punkt $x$ angreift. Aus $(\vek K + \vek \Delta \vek K)\,\vek g_c = \vek j$ folgt
\begin{align}
\vek K\,\vek g_c = \vek j - \vek \Delta \vek K\,\vek g_c = \vek j + \vek j^+
\end{align}
und nach Multiplikation von links mit $\vek u$ (es ist $\vek u^T\,\vek K = \vek f^T$)
\begin{align}
\underbrace{\vek u^T\,\vek K\,\vek g_c}_{J(\vek u_c)} = \underbrace{\vek u^T\vek j}_{J(\vek u)} - \vek u^T\vek \Delta \vek K\,\vek g_c = \vek u^T\vek j + \vek u^T\vek j^+\,,
\end{align}
ergibt sich
\begin{align}\label{Eq104}
J(\vek e) = u_c(x) - u(x) = \vek u^T\,\vek j^+
\end{align}
mit
\begin{align}
\vek j^+ = -\vek \Delta \,\vek K \,\vek g_c\,.
\end{align}
Wenn man z.B. mit Knotenkr\"{a}ften $\vek j$ die Einflussfunktion f\"{u}r die Querkraft in einem Riegel im 5. Stock generiert und im 1. Stock \"{a}ndert sich die Steifigkeit eines Stiels, dann muss man die Knoten dieses Stiels mit Knotenkr\"{a}ften $\vek j^+$ belasten, um am  urspr\"{u}nglichen Tragwerk durch L\"{o}sen des Systems $\vek K\,\vek g_c = \vek j + \vek j^+$ die ge\"{a}nderte Einflussfunktion, $\vek g \to \vek g_c$, zu berechnen. Von der Spreizung $\vek g_c$ im 5. Stock wird aber nicht viel im 1. Stock ankommen, so dass die notwendigen Zusatzkr\"{a}fte $\vek j^+$ klein sein werden.

Das ganze ist insofern theoretisch, weil man ja $\vek g_c$ nicht kennt (man k\"{o}nnte $\vek g_c\sim \vek g $ setzen) und somit den Vektor $\vek j^+$ nicht berechnen kann, aber es veranschaulicht, was die $j_i^+$ bedeuten.

Nehmen wir eine Verschiebung als Beispiel. Um auf dem \glq alten\grq{} Netz $G_c$ zu erzeugen, muss man zus\"{a}tzlich die Kr\"{a}fte $j_i^+$ anbringen. Formuliert man den Satz von Betti mit der {\em alten\/} L\"{o}sung $u$ und dem {\em neuen\/} $G_c$ so entsteht
\begin{align}
1 \cdot u + \sum_i\,j_i^+\,u_i = \int_{\Omega} G_c\,p\,d\Omega_{\vek y} = 1 \cdot u_c\,,
\end{align}
was bedeutet, dass die Arbeit der Kr\"{a}fte $\vek j$ und $\vek j^+$ auf den alten Wegen $\vek u$  gleich dem neuen $u_c$ im Aufpunkt ist.

%%%%%%%%%%%%%%%%%%%%%%%%%%%%%%%%%%%%%%%%%%%%%%%%%%%%%%%%%%%%%%%%%%%%%%%%%%%%%%%%%%%%%%%%%%%%%%%%%%%
\textcolor{sectionTitleBlue}{\section{Austausch als Alternative}}
Statt mit den Kr\"{a}ften $f_i^+$ zu operieren, gibt es auch andere Techniken.
Man stelle sich vor, dass eine St\"{u}tze zwischen zwei Decken, s. Abb. \ref{U126}, ausgetauscht werden muss. Wenn die neue St\"{u}tze dieselbe Steifigkeit $k = EA$ hat, dann
\begin{itemize}
  \item muss sie auf dem Bauhof dieselbe L\"{a}nge haben, wie die alte unbelastete St\"{u}tze
  \item muss die neue St\"{u}tze so vorgespannt werden, dass sie bei dem Einbau dieselbe L\"{a}nge hat wie die alte St\"{u}tze.
\end{itemize}
Nach dem Einbau k\"{o}nnen die Pressen an der neuen St\"{u}tze weggenommen werden und  nichts hat sich ge\"{a}ndert\footnote{Eigentlich ist $k = EA/l$, aber zu Vereinfachungen lassen wir den Faktor $1/l$ weg}.\label{Korrektur3}
%-----------------------------------------------------------------
\begin{figure}
\centering
\if \bild 2 \sidecaption \fi
{\includegraphics[width=0.9\textwidth]{\Fpath/U126}}
\caption{Zunahme und Abnahme der L\"{a}ngssteifigkeit $EA$ in einer St\"{u}tze}
\label{U126}%
\end{figure}%
%-----------------------------------------------------------------

Wenn stattdessen die neue St\"{u}tze eine Steifigkeit $EA_c > EA$ hat, dann dr\"{u}cken Zusatzkr\"{a}fte (= $f_i^+$) gegen die obere und untere Decke, sobald die Pressen entfernt werden, s. Abb. \ref{U126}. Denn um eine St\"{u}tze mit $EA_c > EA$ zusammenzudr\"{u}cken, sind gr\"{o}{\ss}ere Kr\"{a}fte notwendig als im Fall $EA_c = EA$. Und wenn die Pressen entfernt werden, dann dr\"{u}cken diese Kr\"{a}fte gegen die Decken.

Wenn die neue St\"{u}tze eine geringere Steifigkeit hat, $EA_c < EA$, dann ist die Situation im Grunde dieselbe, wir m\"{u}ssen nur das Vorzeichen der $f_i^+$ umdrehen.

Statt also ein zweites Element $\Omega_e^+$ vor das erste Element $\Omega_e$ zu legen, k\"{o}nnen wir uns auch vorstellen, dass wir das Element als Ganzes ersetzen. Weil aber die Steifigkeit {\em neu -- alt\/} unterschiedlich ist, $EA_c \neq EA$, sind Zusatzkr\"{a}fte $f_i^+$ an den Enden des Elements notwendig. Es gibt vier m\"{o}gliche Szenarien:

\begin{itemize}
  \item $EA_c > EA$ und $N > 0$ (Zug), dann ziehen die $f_i^+$ die St\"{u}tze zusammen, ($\rightarrow\,\,\leftarrow$).
  \item $EA_c > EA$ und $N < 0$ (Druck), dann dr\"{u}cken die $f_i^+$ die St\"{u}tze auseinander, ($\leftarrow \,\,\rightarrow$).
  \item $EA_c < EA$ und $N > 0$, dann ziehen die $f_i^+$ die St\"{u}tze auseinander, ($\leftarrow \,\,\rightarrow$).
\item $EA_c < EA$ und  $N < 0$, dann ziehen die $f_i^+$ die St\"{u}tze zusammen, ($\rightarrow\,\,\leftarrow$).
\end{itemize}
Wenn $EA_c > EA$ ist, dann nehmen die Verformungen der St\"{u}tze ab und wenn  $EA_c < EA$ ist, dann nehmen sie zu.


%-----------------------------------------------------------------
\begin{figure}[tbp]
\centering
\includegraphics[width=.99\textwidth]{\Fpath/UE352}
\caption{Eine lokale Steifigkeits\"{a}nderung, \"{a}ndert die Einflussfunktion f\"{u}r  $V(x)$ im ganzen Rahmen}
\label{UE352}
\end{figure}%
%-----------------------------------------------------------------

%-----------------------------------------------------------------
\begin{figure}[tbp]
\centering
\includegraphics[width=.99\textwidth]{\Fpath/UE366}
\caption{Risse in einem Balken}
\label{UE366}
\end{figure}%
%-----------------------------------------------------------------

%%%%%%%%%%%%%%%%%%%%%%%%%%%%%%%%%%%%%%%%%%%%%%%%%%%%%%%%%%%%%%%%%%%%%%%%%%%%%%%%%%%%%%%%%%%%%%%%%%%%%%%%
\textcolor{sectionTitleBlue}{\section{Integration \"{u}ber das defekte Element}}
Auch wenn sich nur eine Zahl $k_{ij}$ in der Steifigkeitsmatrix $\vek K$ \"{a}ndert, \"{a}ndert sich die ganze Inverse $\vek K^{-1}$. Praktisch bedeutet dies z.B., dass die Schw\"{a}chung eines Elementes, $ EI \to EI + \Delta EI$, die Querkraft $V(x)$ in allen Punkten des Rahmens \"{a}ndert
\begin{align}
V_c(x) - V(x) = \sum_e \int_0^{\,l_e} (G_3^c(y,x) - G_3(y,x))\,p(y)\,dy\,,
\end{align}
weil sich eben die Einflussfunktion, $G_3(y,x) \rightarrow G_3^c(y,x)$, \"{u}berall \"{a}ndert, und somit die \"{U}berlagerung der Belastung mit $G_3^c(y,x)$ andere Werte ergibt, s. Abb. \ref{UE352}.

Es gibt aber die M\"{o}glichkeit, das Nachrechnen auf das Element, das sich \"{a}ndert, zu beschr\"{a}nken. Wir betrachten hierzu den Zweifeldtr\"{a}ger in Abb. \ref{UE366}. Zu Anfang war die Steifigkeit $EI$ in beiden Feldern gleich gro{\ss} und so lautet die schwache Form der Balkengleichung $EI\,w^{IV} = p$
\begin{align}
\int_0^{\,l} EI\,w''\,\delta w''\,dx = \int_0^{\,l} p\,\delta w\,dx\,,
\end{align}
oder kurz
\begin{align}
a(w, \delta w) = (p, \delta w)\,.
\end{align}
Dann \"{a}ndert sich die Steifigkeit im zweiten Feld auf einen Wert $EI + \Delta EI$, die Biegelinie \"{a}ndert sich mit, $w \to w_c$, und die schwache Form geht \"{u}ber in
\begin{align}\label{Eq177}
\int_0^{\,l} EI\,w_c''\, \delta w''\,dx + \underbrace{\int_{l/2}^{\,l} \Delta EI\,w_c''\, \delta w''\,dx}_{d(w_c,\delta w)} = \int_0^{\,l} p\,\delta w\,dx
\end{align}
oder\footnote{Diese additive Zerlegung, $a(.,.) + d(.,.)$ ist der Schl\"{u}ssel zu $J(e) = - d(G,u_c)$.}
%-----------------------------------------------------------------
\begin{figure}[tbp]
\centering
\includegraphics[width=.90\textwidth]{\Fpath/U465}
\caption{Die Integration \"{u}ber das gerissene Element oben links reicht aus, um die \"{A}nderung in der horizontalen Verschiebung zu bestimmen \textbf{ a)} Spannungsverteilung nach dem Auftreten der Risse im Element oben links \textbf{ b)} Spannungen aus der Einzelkraft $P = 1$ (Einflussfunktion f\"{u}r $u_x$) am oberen Kragarmende}
\label{U465}
\end{figure}%
%-----------------------------------------------------------------
\begin{align}
a(w_c, \delta w) + d(w_c, \delta w) = (p, \delta w)\,.
\end{align}
Wenn wir die beiden Gleichungen voneinander abziehen, dann folgt
\begin{align}
a(w_c - w, \delta w) + d(w_c, \delta w) = 0
\end{align}
oder mit $e = w_c - w$
\begin{align}\label{Eq10}
a(e, \delta w) = -  d(w_c, \delta w)\,.
\end{align}
W\"{a}hlen wir als virtuelle Verr\"{u}ckung $\delta w$ die Einflussfunktion $G$ eines Funktionals $J(w)$, dann ist das Ergebnis die \"{A}nderung in dem Funktional
$J(e) = a(e, G) = - d(w_c, G)$, also
\begin{align}\label{Eq165}
J(e) = - d(w_c, G)\,.
\end{align}
Man beachte, dass
\begin{align}
d(w_c,\delta w) = \int_{l/2}^{\,l} \Delta EI\,w_c''\, \delta w''\,dx = \frac{\Delta EI }{EI} \int_{l/2}^{\,l} \frac{M_c\,M}{EI_c}\,dx\,.
\end{align}
In der Notation der linearen Algebra ist die Herleitung noch einfacher. Bei einer \"{A}nderung der Steifigkeitsmatrix, $\vek K \to \vek K + \vek \Delta \vek K$, wird aus der urspr\"{u}nglichen schwachen Formulierung
\begin{align}
\vek \delta \vek u^T\,\vek K\,\vek u = \vek \delta \vek u^T\,\vek f
\end{align}
die Gleichung
\begin{align}
\vek \delta \vek u^T\,(\vek K + \vek \Delta\,\vek K)\,\vek u_c = \vek \delta \vek u^T\,\vek f\,,
\end{align}
so dass  der Vektor $\vek e = \vek u_c - \vek u$ der Gleichung
\begin{align}\label{Eq143}
\vek  \delta \vek u^T\,\vek K\,\vek e = - \vek \delta \vek u^T\,\vek \Delta \vek K\,\vek u_c \qquad \text{f\"{u}r alle} \,\vek \delta \vek u
\end{align}
gen\"{u}gt. An dem Zustandsvektor $\vek u$ eines Systems k\"{o}nnen wir mittels einer Einflussfunktion, dem Vektor $\vek g$, Messungen vornehmen
\begin{align}
J(\vek u) = \vek g^T\,\vek K\,\vek u
\end{align}
und so folgt, wenn wir in (\ref{Eq143}) f\"{u}r $\vek  \delta \vek u$ den Vektor $\vek g$ setzen, das Ergebnis
\begin{align}\label{Eq188}
J(\vek e) =  - \vek g^T\,\vek \Delta \vek K\,\vek u_c \qquad (= - d(w_c, G))\,,
\end{align}
zu dessen Berechnung wir nur auf dem oder den betroffenen Element(en) messen m\"{u}ssen, s. Abb. \ref{U465}. Diese Gleichung entspricht (\ref{Eq165}).

Im Grunde steht in (\ref{Eq188}) die Wechselwirkungsenergie, denn die \"{A}nderung $J(\vek e)$ kann man, und das gilt f\"{u}r alle Bauteile, in der Form
\begin{align}
J(\vek e) = - d(w_c, G) = -\alpha\cdot a(G,w_c)_{\Omega_e}
\end{align}
schreiben, also als die Wechselwirkungsenergie zwischen der Einflussfunktion $G$ f\"{u}r $J(w)$ und der neuen L\"{o}sung $w_c$ in dem Element $\Omega_e$, dessen Steifigkeit sich \"{a}ndert, wobei
\begin{align}
\alpha = \frac{\Delta E}{E} =  \frac{\Delta EI}{EI} =  \frac{\Delta EA}{EA} =\frac{\Delta k}{k} = \frac{\Delta K}{K} \qquad \text{etc.}
\end{align}
das Verh\"{a}ltnis der Steifigkeits\"{a}nderung zur urspr\"{u}nglichen Steifigkeit ist.

%%%%%%%%%%%%%%%%%%%%%%%%%%%%%%%%%%%%%%%%%%%%%%%%%%%%%%%%%%%%%%%%%%%%%%%%%%%%%%%%%%%%%%%%%%%%%%%%%%%%%%%%
\textcolor{sectionTitleBlue}{\section{Observable}}
Nachdem nun klar ist, dass man die Effekte von \"{A}nderungen durch eine lokale Analyse ermitteln kann, wollen wir uns im folgenden etwas systematischer und im Kontext der klassischen Statik mit dieser Strategie besch\"{a}ftigen.

Jede Gr\"{o}{\ss}e $O$ (\glq Observable\grq{})\index{Observable} in einem Tragwerk ist im Rahmen der Theorie I. Ordnung mittels einer Einflussfunktion berechenbar. Diese Einflussfunktionen sind aber selbst Verschiebungen, d.h. zu ihnen geh\"{o}ren Biegemomente $M_G$, Querkr\"{a}fte $V_G$ und Normalkr\"{a}fte $N_G$. Das ist wichtig, weil in der Methode, die wir jetzt erl\"{a}utern wollen, das Augenmerk auf die innere Energie gerichtet ist. Wir benutzen eine Variante des Mohrschen Arbeitsintegral, bei der die \"{A}nderungen in der Gr\"{o}{\ss}e $O$, der \glq Observablen\grq{},  anhand der Wechselwirkungsenergie zwischen der Einflussfunktion und den Tragwerksverschiebungen gemessen wird.

Nehmen wir an, dass die Gr\"{o}{\ss}e $O = w(x)$ die Durchbiegung des Balkens in einem Punkt  $x$ ist.  Das Mohrsche Arbeitsintegral liefert f\"{u}r $O = w(x)$ den Wert
\bfo\label{Eq142}
O = \int_0^{\,l} \frac{M(y)\,M_G(y,x)}{EI} \,dy\,.
\efo
Dann entsteht eine neue Situation: In einem Element des Balkens \"{a}ndert sich die Steifigkeit, $EI \to EI + \Delta EI$. Das bedeutet, dass sich Momente $M \to M_c$ und $M_G \to M_G^c$ \"{a}ndern und wir m\"{u}ssen wieder von vorne anfangen
\bfo
O_c = \int_0^{\,l} \frac{M_c(y)\,M_G^c(y,x)}{EI_c} \,dy\,.
\efo
Hierbei ist $EI_c$ die {\em step function\/}
\begin{align}
EI_c = \left \{ \barr{l} EI + \Delta EI \qquad \,\,\,\text{im Intervall $[x_a,x_b]$} \\EI \qquad \qquad\qquad \text{sonst}  \earr \right. \,.
\end{align}
Nun kommt der interessante Punkt. \\

\hspace*{-12pt}\colorbox{highlightBlue}{\parbox{0.98\textwidth}{Wie wir oben gezeigt haben, (\ref{Eq188}), kann man die \"{A}nderung von $O$ alleine durch Integration \"{u}ber das Element, in dem sich die Steifigkeit \"{a}ndert, berechnen}}\\

\bfo\label{Eq100}
\boxed{O_c - O = -\frac{\Delta\,EI}{EI}\int_{x_a}^{\,x_b} \frac{M_c\,M_G}{EI_c}\,dy}
\efo
Hier ist $M_c$ das Moment in dem Element {\em nach\/} der \"{A}nderung der Steifigkeit und $M_G$ ist das Moment der Einflussfunktion {\em vor\/} der \"{A}nderung.

Wenn wir das Biegemoment durch seine Ableitungen ersetzen, $M = - EI w''$, wird die Gleichung vielleicht transparenter
\bfo
O_c - O = -\frac{\Delta\,EI}{EI}\int_{x_a}^{\,x_b} \frac{M_c\,M_G}{EI_c}\,dy = - \Delta EI\,\int_{x_a}^{\,x_b} w_c''\,G''\,dy\,.
\efo
Es ist auch nicht wichtig, ob wir $M_c$ mit $M_G$ kombinieren oder $M$ mit $M_G^c$
\begin{align}
M \times M_G^c \equiv M_c \times M_G\,,
\end{align}
das Ergebnis ist dasselbe
\bfo
\boxed{O_c - O = -\frac{\Delta\,EI}{EI}\int_{x_a}^{\,x_b} \frac{M\,M_G^c}{EI_c}\,dy}
\efo
Es sei betont, dass der Punkt $x$ {\bf an beliebiger Stelle} im Tragwerk liegen kann. Die \"{A}nderung von $O(x)$ im Punkt $ x$ kann man {\em allein\/} durch Integration \"{u}ber das gesch\"{a}digte Element $[x_a,x_b]$ berechnen. Der Punkt $x$ kann im zehnten Stock liegen und das Element $[x_a,x_b]$ im dritten Stock eingebaut sein. Entscheidend ist, was von der Einflussfunktion f\"{u}r $O(x)$, die im zehnten Stock startet, im dritten Stock ankommt, wie gro{\ss} das Moment $M_G$ der Einflussfunktion im Element $[x_a,x_b]$ noch ist -- und nat\"{u}rlich wie gro{\ss} das Lastmoment $M_c$ dort ist.

Die Formel (\ref{Eq100}) hat allerdings den Nachteil, dass man den Momentenverlauf $M_c$ am ge\"{a}nderten System kennen muss. Aber wenn man den kennt, dann braucht man die Formel nicht mehr...

Also ersetzen wir $M_c$ durch das Moment $M$ am urspr\"{u}nglichen Tragwerk und kommen so zur N\"{a}herung
\bfo
O_c - O \simeq -\frac{\Delta\,EI}{EI}\int_{x_a}^{\,x_b} \frac{M\,M_G}{EI_c}\,dy\,.
\efo
Je nach der Art der Steifigkeits\"{a}nderung muss man diese Formel nat\"{u}rlich modifizieren. Wenn sich die L\"{a}ngssteifigkeit $EA$ einer St\"{u}tze \"{a}ndert, dann lautet die Formel
\bfo
O_c - O \simeq  -\frac{\Delta EA}{EA}\int_{x_a}^{\,x_b} \frac{N  N_G}{EA_c} \,dy\,,
\efo
und \"{a}ndert sich bei einer Scheibe in einem Element $\Omega_e$ der E-Modul, dann lautet die Formel
\begin{align}
O_c - O \simeq -\frac{\Delta E}{E} \int_{\Omega_e} \sigma_{ij}\,\varepsilon_{ij}^G \,d\Omega_{\vek y}\,,
\end{align}
und so kann diese N\"{a}herung auf alle Tragwerke und alle interessierenden Gr\"{o}{\ss}en angewandt werden.

%-----------------------------------------------------------------
\begin{figure}[tbp]
\centering
\includegraphics[width=0.9\textwidth]{\Fpath/U186}
\caption{Momente $M$, \textbf{ a)} aus Wind und \textbf{ b)} Momente $\bar{M}$ durch die Verdrehung des Fu{\ss}punktes, Einflussfunktion f\"{u}r das Fu{\ss}punktsmoment. Stabweise wird das Integral von $M$ und $M_G$ mit dem Quotienten $\Delta EI/EI$ gewichtet und man bestimmt so den Einfluss, den eine \"{A}nderung $\Delta EI$ in dem Stab auf das Fu{\ss}punktsmoment haben w\"{u}rde. Genau genommen m\"{u}sste man $M$ durch $M_c$ ersetzen, aber dies kann durch einen Korrekturfaktor, $M_c \simeq \alpha M$ n\"{a}herungsweise ausgeglichen werden.}
\label{U186}
\end{figure}%
%-----------------------------------------------------------------

Das Erstaunliche an der Formel (\ref{Eq100}), die ja auf der ersten Greenschen Identit\"{a}t in der Formulierung als {\em Prinzip der virtuellen Kr\"{a}fte\/} beruht,
\begin{align}\label{Eq121}
\text{\normalfont\calligra G\,\,}(G,w) = \delta A_a^* - \delta A_i^* = 0\,,
\end{align}
ist, dass man mit ihr auch die Auswirkungen von Steifigkeits\"{a}nderungen auf Schnittgr\"{o}{\ss}en, etwa $V(x) \to V(x) + \Delta V$, voraussagen kann, w\"{a}hrend man ja sonst mit dem {\em Prinzip der virtuellen Kr\"{a}fte\/} Kraftgr\"{o}{\ss}en eigentlich nicht berechnen kann.


%%%%%%%%%%%%%%%%%%%%%%%%%%%%%%%%%%%%%%%%%%%%%%%%%%%%%%%%%%%%%%%%%%%%%%%%%%%%%%%%%%%%%%%%%%%%%%%%%%%
\textcolor{sectionTitleBlue}{\section{Nah und fern}}

Die \"{A}nderung $O_c - O$ wird nur dann auffallend sein, wenn die Wechselwirkungsenergie $\delta A_i = a(G,w_c)_{\Omega_e}$ in dem Element \glq gro{\ss}\grq{} ist\footnote{Wir lassen das Sternchen an $\delta A_i^*$ weg, weil es mathematisch \glq Verzierung\grq{} ist}. Das ist dann der Fall, wenn die Schnittgr\"{o}{\ss}en aus der  Einflussfunktion und gleichzeitig die Schnittgr\"{o}{\ss}en aus der Belastung in dem Bauteil gro{\ss} sind.

Weil die Einflussfunktionen (in der Regel) rasch abklingen, kann man davon ausgehen, dass Steifigkeits\"{a}nderungen in weit abliegenden Bauteilen nur sehr geringen Einfluss auf die Schnittgr\"{o}{\ss}en im \glq Vordergrund\grq{} haben werden.

Und weil $\delta A_i$ ein Skalarprodukt ist, kann es, wie bei Vektoren, Funktionen $M$ und $M_G$ geben, die senkrecht aufeinander stehen, deren \"{U}berlagerung also null ergibt; wenn etwa $M$ antimetrisch ist und $M_G$ symmetrisch. In solchen F\"{a}llen ist, unabh\"{a}ngig von der Gr\"{o}{\ss}e von $M$ und $M_G$, die \"{A}nderung $O_c - O$ null.

Vor jeder Berechnung kann man also, allein durch das Studium der Einflussfunktionen,
absch\"{a}tzen, welche Steifigkeits\"{a}nderungen Effekt machen und welche nicht.


Der Rahmen in Abb. \ref{U186} wird seitlich vom Wind angeblasen. Es soll abgesch\"{a}tzt werden, welche Steifigkeits\"{a}nderungen $\Delta EI$ in den Riegeln und Stielen das Anschnittsmoment im linken Fu{\ss}punkt am meisten beeinflussen ({\em Focus auf einen Punkt\/}).

Hierzu wird die Einflussfunktion $G$ f\"{u}r das Moment aufgestellt. Allerdings interessiert nicht, wie sonst \"{u}blich, der Verlauf von $G$, sondern es interessieren die {\em Momente\/} $M_G$ aus der Einflussfunktion.

\"{A}ndert sich in einem Stiel oder Riegel die Biegesteifigkeit, $EI \to EI + \Delta EI$, so ist die \"{A}nderung im Anschnittsmoment n\"{a}herungsweise gleich
\bfo
M_c - M \simeq -\frac{\Delta EI}{EI} \int_0^{\,l_e} \frac{M_G\,M}{EI}\,dy\,.
\efo
Die Bauteile, in denen die Momente $M_G$ und $M$ {\em beide\/} gro{\ss} sind, (und nicht orthogonal zueinander), sind die Bauteile mit dem gr\"{o}{\ss}ten Einfluss auf das Anschnittsmoment. Nicht \"{u}berraschend sind das die beiden Stiele im Erdgescho{\ss}.

Eine verwandte Fragestellung ist die Suche nach dem maximalen Effekt den die Steifigkeits\"{a}nderung in {\em einem\/} Element in dem Rahmen hervorruft ({\em Focus auf ein Element\/}). Das kann man schreiben als
\begin{align}
max\,\, J(e) \qquad \text{{\em alle interessierenden\/} $G$}\,,
\end{align}
also als Suche nach der Kurve $M_G$, die mit $M_c$ auf dem Element \"{u}berlagert den gr\"{o}{\ss}ten Wert $J(e)$ liefert, wobei $M_G $ das Moment der Einflussfunktion ist, die zu $J(w)$ geh\"{o}rt.

%-----------------------------------------------------------------
\begin{figure}[tbp]
\centering
\includegraphics[width=0.9\textwidth]{\Fpath/U466}
\caption{Elastische Einspannung eines Tr\"{a}gers}
\label{U466}
\end{figure}%
%-----------------------------------------------------------------


%%%%%%%%%%%%%%%%%%%%%%%%%%%%%%%%%%%%%%%%%%%%%%%%%%%%%%%%%%%%%%%%%%%%%%%%%%%%%%%%%%%%%%%%%%%%%%%%%%%
\textcolor{sectionTitleBlue}{\section{Federn}}
Beispiele f\"{u}r Federn sind drehelastische Einspannungen in Rahmenecken oder Fundamenten. Wenn sich die Drehsteifigkeit \"{a}ndert, muss man nicht das ganze Tragwerk neu berechnen, sondern man kann sich den Effekt zu nutze machen, dass die \"{A}nderung $J(e) = - d(G,w_c)$ eines Funktionals -- also {\em eines Moments, einer Querkraft, einer Durchbiegung\/} -- direkt an der Feder verfolgt werden kann. Gerade bei komplexen 3-D Modellen mit einer gro{\ss}en Anzahl von Freiheitsgraden ist das ein nicht zu untersch\"{a}tzender Vorteil.

Die schwache Form der Balkengleichung des Tr\"{a}gers in Abb. \ref{U466} lautet\footnote{Wir schreiben die Drehfedersteifigkeit hier $k$ und nicht wie sonst $k_{\Np}$}
\begin{align}
\int_0^{\,l} \frac{M\,\delta M}{EI}\,dx + \delta w'\,w'\,k = \int_0^{\,l} p\,\delta w\,dx\,.
\end{align}
Vertrauter ist dem Leser wahrscheinlich der Ausdruck
\begin{align}
 \frac{\delta M\,M}{k} = \delta w'\,w'\,k \qquad (\delta M = k\,\delta w'\,, M = k\, w')\,.
\end{align}
\"{A}ndert sich die Drehfedersteifigkeit, $k \to k + \Delta k$, dann wird daraus
\begin{align}
\underbrace{\int_0^{\,l} \frac{M_c\,\delta M}{EI}\,dx + \delta w'\,w_c'\,k}_{a(w_c,\delta w)} +  \underbrace{\vphantom{\int_0^{\,l}} \Delta k\,w_c' \,\delta w'}_{d(w_c,\delta w)} = \int_0^{\,l} p\,\delta w\,dx\,.
\end{align}
Die \"{A}nderung in einem Funktional $J(w)$, $M_G$ ist das Moment der Einflussfunktion in der Feder vor der \"{A}nderung, ergibt sich somit zu
\begin{align}\label{Eq152}
J(e) = J(w_c) - J(w)  = - d(w_c,w_G) = - \Delta k\,w_c'\,w_G' = - \Delta k\,\frac{M_c}{k_c}\,\frac{M_G}{k} \,.
\end{align}
Zur Anwendung dieser Formel fehlt uns das Moment $M_c = k_c\,w_c'$. Dieses kann man jedoch, wenn man nicht die N\"{a}herung $M_c \sim M$ verwenden will, auf zwei Arten bestimmen: Durch {\em Iteration\/}, oder, wenn man auf die Inverse $K^{-1}$ Zugriff hat, durch {\em direkte L\"{o}sung\/}, \cite{Carl2}.

Den Zugang zur Iteration finden wir, wenn wir als Funktional das Moment in der Feder w\"{a}hlen, $J(w) = M$. Dann ist (\ref{Eq152}) \"{a}quivalent mit
\begin{align}
J(e) = M_c - M = - \Delta k\,\frac{M_c}{k_c}\,\frac{M_G}{k} \,.
\end{align}
Daraus kann man eine Iterationsvorschrift f\"{u}r eine Folge $M_c^{(i)}$ ableiten, die, wie sich zeigt, schnell gegen $M_c$ konvergiert
\begin{align}
M_c^{(i+1)} = - \Delta k\,\frac{M_c^{(i)}}{k_c}\,\frac{M_G}{k}  + M \qquad M_c^{(0)} = M\,.
\end{align}
Hat man $M_c = k_c\,w_c'$ bestimmt, kennt man auch $w_c'$ und dann kann man mit (\ref{Eq152}) die \"{A}nderung in jedem Funktional $J(w)$ verfolgen. Man braucht nur das Moment $M_G$ der Einflussfunktion des Funktionals $J(w)$ in der Feder. Das kann man aber am \glq alten\grq{} System bestimmen.

\"{A}ndern sich $m$ Federn in $m$ Punkten $x_i$, dann geht (\ref{Eq152}) in ein lineares Gleichungssystem \"{u}ber, das man wieder per Iteration l\"{o}sen kann, eventuell mit einer Konvergenzbeschleunigung wie auf S.  \pageref{Eq59} beschrieben,
\begin{align} \label{Eq156}
M_{c}^{(i+1)}(x_j) &= - \sum_{l = 1}^m \Delta k(x_l)\,\frac{M_{c}^{(i)}(x_l)}{k_{c}(x_l)}\,\frac{M_{G}(x_l,x_j)}{k(x_l)}  + M(x_j)  \quad j &= 1,2, \ldots m\,.
\end{align}
Hier ist $M_{G}(x_l,x_j)$ das Moment in der Feder $x_l$ (dem Ort), das von der Einflussfunktion f\"{u}r das Moment im Punkt $x_j$ (der Ursache) in $x_l$ erzeugt wird (Berechnung am \glq alten\grq{} System).

Die Stahlhallen in Abb. \ref{CarlStahlbau1} und \ref{CarlStahlbau2} wurden so untersucht. Bei dem ersten System ging es um den \"{U}bergang von einem starren Anschluss in den Rahmenecken zu einer drehelastischen Einspannung. Bei dem zweiten System ging es um den Einfluss der drehelastischen Einspannung der St\"{u}tzen auf die horizontale Verschiebung der Kranbahn in einem ausgew\"{a}hlten Punkt $x_P$. In einer ersten Statik waren die St\"{u}tzen voll eingespannt gerechnet worden.

Zuerst wurde der Punkt mit einer horizontalen Einzelkraft $P = 100$ kN belastet (100 wegen der \glq Sichtbarkeit\grq{}) und die Einspannmomente $M_{G}^P(x_i), i = 1,2,\ldots 8$ dieser Einflussfunktion (am \glq alten\grq{}, dem voll eingespannten System) in den acht Fundamenten berechnet. Dann wurde die Iterationsvorschrift (ein $8 \times 8$-System) zur Bestimmung der acht Einspannmomente $M_c(x_i)$ aus der Verkehrslast ({\em Kran (25t-Hublast) - vert. Randlasten + Schr\"{a}glaufkr\"{a}fte in Reihe 4\/}) an dem drehelastischen System aufgestellt und die $M_c(x_i)$ wie in (\ref{Eq156}) iterativ bestimmt. Wegen der urspr\"{u}nglichen Festeinspannung $k(x_i) = \infty$ aber mit der Modifikation wie in (\ref{Eq153}) beschrieben
\begin{align}
M_{c}^{(i+1)}(x_j) &= - \sum_{l = 1}^m \frac{1}{k_{c}(x_l)}\,M_{c}^{(i)}(x_l)\,M_{G}(x_l,x_j)  + M(x_j)  \quad j &= 1,2, \ldots m\,.
\end{align}
Mit den auf Eins normierten Momenten $M_{G}^P(x_i)/100$ und den $M_c(x_i)$ kann man dann die horizontale Verschiebung $u_c(x_P)$ berechnen
\begin{align}\label{Eq157}
u_c(x_P) = - \sum_{l = 1}^8 \frac{1}{k_c(x_l)} M_c(x_l)\, M_G^P(x_l) \frac{1}{100}\,.
\end{align}
Hat man einmal die Einspannmomente $M_c(x_l)$ der St\"{u}tzen aus dem ma{\ss}gebenden Lastfall ermittelt, dann kann man die \"{A}nderung in jeder anderen Weg- oder Kraftgr\"{o}{\ss}e $J(w)$ der Halle berechnen. Was man braucht, sind nur die acht Einspannmomente $M_G(x_l)$ (am \glq alten\grq{} System), die zu der Einflussfunktion f\"{u}r die jeweilige Weg- oder Kraftgr\"{o}{\ss}e $J(w)$ geh\"{o}ren, s. Abb. \ref{U474}. Diese mit den ge\"{a}nderten Momenten $M_c(x_l)$ aus der Verkehrslast \"{u}berlagert, sinngem\"{a}{\ss} wie in \ref{Eq157}, ergibt die \"{A}nderung $J(e)$.
%-----------------------------------------------------------------
\begin{figure}[tbp]
\centering
\includegraphics[width=0.9\textwidth]{\Fpath/CarlStahlbau1}
\caption{3D-Modell eines Rahmens mit 47 St\"{a}ben und 480 Freiheitsgraden. Die dreh\-elastische Einspannung reduzierte das Einspannmoment von $M = -86,6$ kNm auf $M_c = - 75,0$ kNm, \cite{Carl4} }
\label{CarlStahlbau1}
\end{figure}%
%-----------------------------------------------------------------

Die Alternative zur Iteration ist die direkte L\"{o}sung. Ist das Funktional $J(w) = w'$ der Tangens des Drehwinkels in der Feder,
\begin{align}\label{Eq151}
J(e) = w_c' - w' = -\Delta k\,w_c'\,w_G'\,,
\end{align}
dann kann man das nach $w_c'$ aufl\"{o}sen
\begin{align}
w_c' =   w'\,\frac{1}{1 + \Delta k\,w_G'}\,.
\end{align}
Wenn die Feder den Drehfreiheitsgrad $u_7$ hat, dann ist $w_G'$ der Eintrag $f_{7,7}$ der Flexibilit\"{a}tsmatrix $\vek F = \vek K^{-1}$. Mit $w_c'$ kann man dann das ge\"{a}nderte Moment $M_c = (k + \Delta k)\,w_c'$ in der Feder berechnen oder die \"{A}nderung $J(e)$ in jedem anderen Funktional, z.B. der Verschiebung $u_3$ in einem Knoten. Man muss  nur f\"{u}r $w_G'$ in (\ref{Eq152}) den entsprechenden Wert einsetzen. Hat die Drehfeder, wie angenommen, den Freiheitsgrad $u_7$, dann ist $w_G'$ gleich dem Eintrag $f_{3,7}$ in der Flexibilit\"{a}tsmatrix $\vek F = \vek K^{-1}$
\begin{align}
J(e) = u_{c 3} - u_3 = - \Delta k\,w_c'\,f_{3,7}\,.
\end{align}
\"{A}ndern sich die Steifigkeiten in zwei Federn, die an den Stellen $x_1$ und $x_2$ liegen, dann lautet der Zusatzterm zur schwachen Form
\begin{align}
d(w_c, \delta w) = \Delta k_1\,w_c'(x_1)\,\delta w'(x_1) + \Delta k_2\,w_c'(x_2)\,\delta w'(x_2)\,.
\end{align}
Das Ziel ist die Bestimmung von $w_c'(x_1)$ und $w_c'(x_2)$, um daraus z.B. die Momente
\begin{align}
M_c(x_1) = (k_1 + \Delta k_1)\,w_c'(x_1) \qquad M_c(x_2) = (k_2 + \Delta k_2)\,w_c'(x_2)
\end{align}
in den Federn zu berechnen.

Die Verdrehungen in den Federn sind zwei Funktionale
\begin{align}
J_1(w) &= w'(x_1) = \int_0^{\,l} G_1(y,x_1)\,p(y)\,dy \\
J_2(w) &= w'(x_2) = \int_0^{\,l} G_2(y,x_2)\,p(y)\,dy
\end{align}
zu denen die Einflussfunktionen $G_i(y,x)$ geh\"{o}ren. Die \"{A}nderungen in den Verdrehungen ergeben sich zu
%-----------------------------------------------------------------
\begin{figure}[tbp]
\centering
\includegraphics[width=0.99\textwidth]{\Fpath/CarlStahlbau2}
\caption{Stahlhalle, 1\,643 St\"{a}be, 5\,340 FG, 121 LF. Zu bestimmen war der Einfluss der drehelastischen Einspannung der acht St\"{u}tzen auf die horizontale Verschiebung in H\"{o}he der Kranbahn. Die hier vorgestellte Technik reduzierte das Problem auf ein $8 \times 8$ Gleichungssystem, das per Iteration gel\"{o}st wurde, \cite{Carl4}}
\label{CarlStahlbau2}
\end{figure}%
%-----------------------------------------------------------------
\begin{align}
J_1(e) &= w_c'(x_1) - w'(x_1) = - \Delta k_1\,w_c'(x_1)\,w_{G_1}'(x_1) - \Delta k_2\,w_c'(x_2)\,w_{G_2}'(x_1) \nn \\
J_2(e) &= w_c'(x_2) - w'(x_2) = - \Delta k_1\,w_c'(x_1)\,w_{G_1}'(x_2) - \Delta k_2\,w_c'(x_2)\,w_{G_2}'(x_2)\,,
\end{align}
was dem linearen (symmetrischen) Gleichungssystem
\begin{align}
\left[ \barr {r @{\hspace{4mm}}r @{\hspace{4mm}}r
@{\hspace{4mm}}r @{\hspace{4mm}}r}
      1 + a_{11} &  a_{12} \\
      a_{21} & 1 + a_{22}
     \earr \right]\left [\barr{c}  w_c'(x_1) \\  w_c'(x_2)\earr \right ]
=  \left [\barr{c}  w'(x_1) \\  w'(x_2)\earr \right ]
\end{align}
mit
\begin{align}
a_{11} = \Delta k_1\,w_{G_1}'(x_1) \qquad a_{12} = \Delta k_2\,w_{G_2}'(x_1)\qquad a_{22} = \Delta k_2\,w_{G_2}'(x_2)
\end{align}
entspricht. W\"{a}ren die Drehfreiheitsgrade in den beiden Federn $u_7$ und $u_9$, dann findet man die Koeffizienten $w_{G_i}'(x_j)$ wieder an den entsprechenden Stellen in der Inversen $\vek F = \vek K^{-1}$
\begin{align}
w_{G_1}'(x_1) = f_{7,7} \qquad w_{G_1}'(x_2) = f_{7,9} \qquad w_{G_2}'(x_1) = f_{9,7} \qquad w_{G_2}'(x_2) = f_{9,9}\,.
\end{align}
%-----------------------------------------------------------------
\begin{figure}[tbp]
\centering
\includegraphics[width=0.99\textwidth]{\Fpath/U474}
\caption{Die Einflussfunktion ($G$) f\"{u}r ein Moment, eine Querkraft, eine Durchbiegung erzeugt Momente ($M_y^G$) in den acht Fusspunkten der St\"{u}tzen, und dieser Vektor $\vek M_y^G$ skalar multipliziert mit dem Vektor $\vek M_c$ der acht Fusspunkts-Momente $k^{-1}_c(x_i)M_c(x_i)$ ergibt die \"{A}nderung $J(e) = \vek M_c^T \vek M_y^G$ des Moments, der Querkraft, etc. im Aufpunkt}
\label{U474}
\end{figure}%
%-----------------------------------------------------------------

Die \"{A}nderung in der horizontalen Verschiebung $u_3$ erg\"{a}be sich dann sinngem\"{a}{\ss} wie folgt
\begin{align}
J(e) = u_{c 3} - u_3 =- \Delta k_1\,w_c'(x_1)\,f_{3,7} - \Delta k_2\,w_c'(x_2)\,f_{3,9} \,.
\end{align}

\begin{remark}
Das hier dargestellte Verfahren ist theoretisch nicht anwendbar, wenn ein urspr\"{u}nglich starres Lager nachgibt, weil der neue Freiheitsgrad in dem System $\vek K$ nicht vorkommt. Dies sollte jedoch eine Ausnahme sein, weil man ja Lager mit ihrer nat\"{u}rlichen Steifigkeit $k < \infty$ rechnen sollte.
%-----------------------------------------------------------------
\begin{figure}[tbp]
\centering
\if \bild 2 \sidecaption \fi
\includegraphics[width=0.7\textwidth]{\Fpath/U469}
\caption{Bestimmung des Einspannmoments $X_1$} \label{U469}
\end{figure}%%
%-----------------------------------------------------------------

Man kann aber eine Umformung vornehmen, denn in der Grenze $k \to \infty$ gilt
\begin{align}\label{Eq153}
J(e) = -\Delta k\,w_c' \, w_G' = -\frac{k_c - k}{k_c\,k} M_c\, M_G = -(\frac{1}{k} - \frac{1}{k_c})\,M_c\, M_G = \frac{1}{k_c}\,M_c\, M_G\,,
\end{align}
wenn $M_G$ das Einspannmoment der Einflussfunktion ist.

Angewandt auf das Lager selbst mit dem unbekannten $M_c$ ergibt sich
\begin{align}\label{Eq154}
M_c - M = \frac{1}{k_c}\,M_c\, M_G\,,
\end{align}
wobei $M_G$ das Moment ist, das das gelenkig gemachte Balkenende um $\tan\,\Np = 1$ spreizt, und nun kann man per Iteration eine Folge $M_i$  bestimmen, die gegen $M_c$ konvergiert
\begin{align}
M_{i+1} = \frac{1}{k_c}\,M_i\, M_G + M \qquad i = 0, 1, 2, \ldots \qquad M_0 = M\,.
\end{align}
Ist $M_c$ bestimmt, dann kann man mit (\ref{Eq153}) jede \"{A}nderung $J(e)$ verfolgen, wenn man f\"{u}r $M_G$ das Einspannmoment setzt, das zur Einflussfunktion f\"{u}r $J(w)$ am urspr\"{u}nglichen Tragwerk geh\"{o}rt.

Das Moment $M_G$ in (\ref{Eq154}) ist das Moment $X_1$, das zwischen dem Balken und der Feder \"{u}bertragen wird, s. Abb. \ref{U469}. Mit
\begin{align}
\delta_{10 } = 1 \qquad \delta_{11} = \frac{\ell}{3\,EI} + \frac{1}{k} \qquad \delta_{11}\,X_1 + \delta_{10} = 0
\end{align}
folgt
\begin{align}
X_1 = \frac{-1}{\frac{\ell}{3\,EI} + \frac{1}{k}}
\end{align}
und damit in der Grenze, $k \to \infty$, $X_1 = 3\,EI/\ell = M_G$.
\end{remark}


%-----------------------------------------------------------------
\begin{figure}[tbp]
\centering
\includegraphics[width=1.0\textwidth]{\Fpath/U461}
\caption{Semi-integrale Br\"{u}cke auf Pf\"{a}hlen\textbf{ a)} FE-Modell \textbf{ b)} Einflussfunktion f\"{u}r das Moment $M(x)$}
\label{U461}
\end{figure}%
%-----------------------------------------------------------------

%%%%%%%%%%%%%%%%%%%%%%%%%%%%%%%%%%%%%%%%%%%%%%%%%%%%%%%%%%%%%%%%%%%%%%%%%%%%%%%%%%%%%%%%%%%%%%%%%%%
\textcolor{sectionTitleBlue}{\section{Integrale Br\"{u}cken}}
Eine beispielhafte Anwendung finden diese Ideen auch bei integralen Br\"{u}cken, bei denen die Widerlager, die Pfeiler und der \"{U}berbau monolithisch miteinander verbunden sind, um die Wartungsarbeiten zu vereinfachen.

Die Idee ist relativ neu und als daher im Zuge der Autobahnerneuerung der A3 die {\em Fahrbachtalbr\"{u}cke\/} bei Aschaffenburg, eine semi-integrale Br\"{u}cke (keine monolithische Verbindung mit den Widerlagern) errichtet werden sollte, hat das Br\"{u}ckenbauamt f\"{u}r Nordbayern darauf bestanden, dass Untersuchungen an Pf\"{a}hlen vorgenommen wurden, um den Einfluss der Grenzwerte $c \pm \Delta c$ der elastischen Bettung $c$ auf den \"{U}berbau absch\"{a}tzen zu k\"{o}nnen, \cite{Schiefer}.

In einer Diplomarbeit wurden die hier entwickelten Ideen benutzt, um dies rechnerisch zu verfolgen, \cite{Sopoth}. Das Beispiel eignet sich gut, weil ja zwischen den Aufpunkten im \"{U}berbau der Br\"{u}cke und den Pf\"{a}hlen eine relativ lange \glq Strecke\grq{} liegt, s. Abb. \ref{U461}, und der Einfluss den Weg praktisch zweimal gehen muss, vom \"{U}berbau zu den Pf\"{a}hlen um die Kr\"{a}fte $\vek f^+$ zu erzeugen und die Kr\"{a}fte $\vek f^+$ gehen denselben Weg zur\"{u}ck, um die Schnittgr\"{o}{\ss}en zu \"{a}ndern, $M(x) \to M_c(x)$, $V(x) \to V_c(x)$ etc. Eine Situation, die typisch f\"{u}r {\em Substrukturen\/}\index{Substrukturen} ist.

In Substrukturen spielen die Einfl\"{u}sse nach einer Steifigkeits\"{a}nderung \glq Ping-Pong\grq{}, wechseln die Signale zwischen dem Lastgurt und der Substruktur hin und her bis die Signale ausgeglichen sind. Die Kr\"{a}fte $\vek f + \vek f^+$ erzeugen die neue Gleichgewichtslage $\vek u_c$ und die muss genau so gro{\ss} sein, dass $\vek u_c$ die Kr\"{a}fte $\vek f^+$ erzeugt.

Zur\"{u}ck zu der Br\"{u}cke: Betrachten wir zum Beispiel das Moment $M(x)$ im \"{U}berbau an einer Stelle $x$, das sich durch die \"{U}berlagerung der Einflussfunktion $G_2(y,x)$ mit der Belastung $p$ berechnen l\"{a}sst
\begin{align}
M(x) = \int_0^{\,l} G_2(y,x)\,p(y)\,dy = \vek g^T\,\vek f\,.
\end{align}
Die Frage, wie sich das Moment \"{a}ndert, wenn sich der Bettungsmodul der Pf\"{a}hle \"{a}ndert, $c \to c + \Delta c$, zielt auf den Unterschied zwischen der Einflussfunktion $G_2$ (Modul $c$, Matrix $\vek K$) und der Einflussfunktion $G_{2 c}$, die am System $\vek K + \vek \Delta \vek K$ mit dem Modul $c + \Delta c$ berechnet wird
\begin{align}
M_c(x) = \int_0^{\,l} G_{2 c}(y,x)\,p(y)\,dy= \vek g_c^T\,\vek f \,.
\end{align}
Wegen
\begin{align}
\vek g_c^T\,\vek f = \vek g^T\,(\vek f + \vek f^+)
\end{align}
kann man das auf die Bedeutung des Vektors $\vek f^+$ f\"{u}r das System $\vek K$ zur\"{u}ckspielen.

Die Vektoren $\vek g$ und $\vek g_c$ sind die Knotenwerte der beiden Einflussfunktionen am Modell $\vek K$ bzw. $\vek K_c = \vek K + \vek \Delta \,\vek K$. Der Vektor $\vek f^+ = -\vek \Delta \vek K\,\vek u_c$  sind die Zusatzkr\"{a}fte in den Knoten der Pf\"{a}hle aus der \"{A}nderung des Bettungsmoduls, $c \to c + \Delta$.

Schreiben wir die Biegeverformung im Bereich der Pf\"{a}hle elementweise als Taylorreihe
\begin{align}
w_c(x) = w_c(0) + w_c'(0) \cdot x + \frac{1}{2}\,w_c''(0) \cdot x^2 + \ldots
\end{align}
dann liefern nur die quadratischen Terme Beitr\"{a}ge zu $\vek f^+ = -\vek \Delta \vek K\,\vek u_c$, weil in jedem Element die Zeilen von $\vek \Delta\,\vek K_e$ orthogonal zu Starrk\"{o}rperbewegungen sind -- den ersten beiden Termen. Die quadratischen Terme werden jedoch die kleinsten der drei Terme sein, so dass auch $\vek f^+$ relativ klein sein wird. Dazu kommt noch, dass die Pf\"{a}hle relativ weit vom \"{U}berbau entfernt liegen, so dass der Einfluss der Pfahlkr\"{a}fte $\vek f^+$ auf den \"{U}berbau, unabh\"{a}ngig von ihrer Gr\"{o}{\ss}e, relativ klein sein wird. Die Rechenergebnisse best\"{a}tigten das, die Momente $M(x)$ und $M_c(x)$ weichen kaum voneinander ab.

Man kann das Ganze auch andersherum aufz\"{a}umen, indem man direkt die \"{A}nderungen $G_2(y,x) \to G_{2c}(y,x)$ verfolgt. Die Knotenwerte $\vek g$ der Einflussfunktion $G_2$ sind die L\"{o}sung des Systems $\vek K\vek g = \vek j$ mit $j_i = M(\Np_i)(x)$ und die Knotenwerte $\vek g_c$ der Einflussfunktion $G_{2 c}$ sind die L\"{o}sung des Systems
\begin{align}
\vek K\,\vek g_c = \vek j + \vek j^+
\end{align}
mit dem Vektor $\vek j^+ = -\Delta \vek K\,\vek g_c$. Die durch die Spreizung des Aufpunktes $x$  erzeugte Bewegung $g_c(x) = g_c(0) + g_c'(0)\,x + 0.5 * g_c''(0)^2/x\ldots$ (in den Pf\"{a}hlen) d\"{u}rfte aber demselben Argument unterliegen wie oben. Die Eintr\"{a}ge in dem Vektor $\vek j^+ =-
\vek \Delta\,\vek K\,\vek g_c$ sollten relativ klein sein und weil die Knoten der Pf\"{a}hle vom \"{U}berbau relativ weit weg liegen, sollte der Einfluss der $\vek j^+$ klein sein und somit auch der Unterschied zwischen den Einflussfunktionen $G_2(y,x)$ und $G_{2c}(y,x)$.

%%%%%%%%%%%%%%%%%%%%%%%%%%%%%%%%%%%%%%%%%%%%%%%%%%%%%%%%%%%%%%%%%%%%%%%%%%%%%%%%%%%%%%%%%%%%%%%%%%%
\textcolor{sectionTitleBlue}{\section{Klassische Formulierung}}
In der Optimierung\index{Optimierung} nennt man das Operieren mit Einflussfunktionen die {\em adjoint method of analysis\/}\index{adjoint method of analysis}. Wir wollen daher zeigen, dass die Formel
\begin{align}\label{Eq12}
J(e) = J(u_c) - J(u) \simeq - d(G,u)
\end{align}
mit den klassischen Ergebnissen der Sensitivit\"{a}tsanalyse identisch ist. Die exakte Formel w\"{a}re $J(e) = -d(G_c,u)$ oder $J(e) = - d(G,u_c)$.


Wir nehmen an, dass die Steifigkeitsmatrix
\begin{align}
\vek K = \vek K(p_1, p_2, \ldots, p_m)
\end{align}
eine Funktion von $m$ Modellparametern $p_i$ ist. Wir wollen die Sensitivit\"{a}t des Funktionals $J(\vek u) = \vek j^T\,\vek u $ bez\"{u}glich der Parameter $p_i$ bestimmen
\begin{align} \label{Eq9}
\frac{d\,J}{d p_i} = \frac{d\,}{d p_i} (\vek j^T\,\vek u) = \frac{d}{d p_i} \vek j^T\,\vek u + \vek j^T\,\frac{\partial }{\partial  p_i}\,\vek u\,.
\end{align}
Weil $\vek K\,\vek u - \vek f = \vek 0$ ein konstanter Vektor ist (als Funktion der $p_i$), sind die Ableitungen null
\begin{align}
\frac{d}{d p_i} (\vek K\,\vek u - \vek f) = \frac{\partial \vek K}{\partial p_i}\,\vek u + \vek K\,\frac{\partial \vek u}{\partial p_i} - \frac{\partial \vek f}{\partial p_i} =0\,,
\end{align}
was
\begin{align} \label{F2}
\frac{\partial \vek u}{\partial p_i} = - \vek K^{-1}\,\frac{\partial \vek K}{\partial p_i}\,\vek u
\end{align}
ergibt, wenn wir annehmen dass $\partial \vek f/\partial p_i = 0$ ist. Wenn wir dieses Ergebnis in (\ref{Eq9}) einsetzen, dann folgt
\begin{align}\label{Eq11}
\frac{d\,J}{d p_i} &= \frac{d}{d p_i} \vek j^T\,\vek u - \vek j^T\,\vek K^{-1}\,\vek u
= \frac{d}{d p_i}\,\vek j^T\,\vek u - \vek j^T\,\vek K^{-1} \frac{\partial \vek K}{\partial p_i}\,\vek u \nn \\
&= \frac{d}{d p_i} \vek j^T\,\vek u - \vek g^T \,\frac{\partial \vek K}{\partial p_i}\,\vek u\,.
\end{align}
Der Vektor
\begin{align}
\frac{\partial \vek K}{\partial p_i}\,\vek u
\end{align}
der genau dem Ausdruck $\vek \Delta \vek K\,\vek u$ entspricht, wird {\em Pseudo-Lastvektor\/}\index{Pseudo-Lastvektor} genannt.

Man beachte, dass wir hier nicht die \"{A}nderung $J(u_c) - J(u)$ berechnen, sondern nur den Gradienten
\begin{align}
\vek \nabla J(u) = \{\frac{d\,J}{d p_1}, \ldots, \frac{d\,J}{d p_m}\}^T
\end{align}
 von $J(u)$, der dann erst mittels Taylor die N\"{a}herung
\begin{align}
J(u_c) - J(u) \simeq \vek \nabla J(u) \dotprod \vek \Delta \vek p = d(G,u)
\end{align}
ergibt, wenn $\vek p$ sich nach $\vek p + \vek \Delta \vek p$ entwickelt. Der Ausdruck $d(G,u) \equiv J' \cdot \Delta p$ in (\ref{Eq12}) ist sozusagen das  Differential, das Inkrement des Funktionals, w\"{a}hrend $dJ/dp_i = J'$ nur die Ableitung ist.

Der erste Term in (\ref{Eq11}) spielt eine Rolle, wenn das Funktional von den $p_i$ abh\"{a}ngt. Das ist z.B. der Fall, wenn $J(\vek u)$ eine Kraftgr\"{o}{\ss}e in dem Element ist, dessen Steifigkeit sich \"{a}ndert, denn weil in die Definition der Schnittkr\"{a}fte die Steifigkeiten eingehen, etwa $J(u) = N(x)  = EA\,u'(x)$ muss auch das Funktional korrigiert werden, wenn sich $EA$ \"{a}ndert.

Der Vektor $\vek g$ in (\ref{Eq11}) ist nat\"{u}rlich der Vektor $\vek g$, der zu dem Funktional
geh\"{o}rt,
\begin{align}
J(\vek u) = \vek j^T\,\vek u = (\vek K\,\vek g)^T\,\vek u = \vek g^T\,\vek K\,\vek u = \vek g^T\,\vek f\,.
\end{align}
In der {\em adjoint method of analysis\/} hei{\ss}t $\vek g$ die {\em adjoint variable\/}\index{adjoint variable} und wird oft mit $\vek \lambda$ bezeichnet, also $\vek K\,\vek \lambda = \vek j$ statt $\vek K\,\vek g = \vek j$.

%%%%%%%%%%%%%%%%%%%%%%%%%%%%%%%%%%%%%%%%%%%%%%%%%%%%%%%%%%%%%%%%%%%%%%%%%%%%%%%%%%%%%%%%%%%%%%%%%%%
\textcolor{sectionTitleBlue}{\section{Direkte Differentiation}}
Noch k\"{u}rzer kann man die Ergebnisse mit der {\em direkten Differentiation\/}\index{direkte Differentiation} herleiten. Das Bauteil setze sich aus zwei Elementen zusammen und jedes Element habe einen eigenen E-Modul $E_i$, so dass die Steifigkeitsmatrix $\vek K$ eine Funktion dieser zwei Werte $E_i$ ist
\begin{align}
\vek K(E_1,E_2) \,\vek u = \vek f\,.
\end{align}
Mit $\vek K$ ist nat\"{u}rlich auch der Vektor $\vek u$ eine Funktion der $E_i$. Nun kann man fragen: Wie \"{a}ndert sich der Vektor $\vek u \rightarrow\,\vek u_c$, wenn sich der Wert $E_1$ im ersten Element \"{a}ndert?

Bezeichnen wir die Ableitung nach $E_1$ mit $(')$, dann gilt, wenn wir annehmen, dass der Vektor $\vek f$ nicht von $E_1$ abh\"{a}ngt
\begin{align}
\vek K'\,\vek u + \vek K\,\vek u' = \vek 0
\end{align}
oder
\begin{align}
\vek u' = -\vek K^{(-1)}\,\vek K'\,\vek u\,.
\end{align}
Brechen wir die Taylor-Entwicklung nach dem ersten Glied ab, dann gilt n\"{a}herungsweise
\begin{align}
\vek u_c - \vek u \sim \vek u' \cdot \Delta E_1 = -\vek K^{(-1)}\,\vek K'\,\vek u \cdot \Delta E_1 = -\vek K^{(-1)} \vek \Delta \vek K\,\vek u = -\vek K^{(-1)}\,\tilde{\vek f}^+\,.
\end{align}
Wenn hier $\vek \Delta \vek K\,\vek u_c$ st\"{u}nde, dann w\"{a}re $\tilde{\vek f}^+$ der exakte Vektor $\vek f^+$.
Der naheliegende Vorschlag, dass man $\vek f^+ = \vek \Delta\,\vek K\,\vek u_c$ durch $\vek f^+ \sim \vek \Delta\,\vek K\,\vek u$ ann\"{a}hert, entspricht also einer linearen Interpolation.

%%%%%%%%%%%%%%%%%%%%%%%%%%%%%%%%%%%%%%%%%%%%%%%%%%%%%%%%%%%%%%%%%%%%%%%%%%%%%%%%%%%%%%%%%%%%%%%%%%%
\textcolor{sectionTitleBlue}{\section{Berechnung von $\vek u_c$}}\label{Eq59}
Das Grundproblem bei der Reanalysis,
\begin{align}
J(\vek e) = - \vek g^T\,\vek \Delta \vek K\,\vek u_c
\end{align}
ist, dass die neue Gleichgewichtslage $\vek u_c$ des Tragwerks unbekannt ist bzw. ersatzweise durch den alten Vektor $\vek u$ angen\"{a}hert werden muss. Es gibt aber zwei Techniken {\em Iteration\/} und {\em direkte Bestimmung\/}, mit denen sich das Problem l\"{o}sen l\"{a}sst.

\textcolor{sectionTitleBlue}{\subsection{Iteration}}
Wir multiplizieren die Gleichung
\begin{align}\label{F2}
(\vek K + \vek \Delta \vek K)\,\vek u_c = \vek f
\end{align}
von links mit $\vek K^{-1}$
\begin{align} \label{F7}
(\vek I + \vek K^{-1}\,\vek \Delta\,\vek K)\,\vek u_c = \vek u
\end{align}
und schreiben Sie in der Form
\begin{align}\label{F6}
\vek u_c = - \vek K^{-1}\,\vek \Delta\,\vek K\,\vek u_c + \vek u
\end{align}
was die folgende {\em fixed-point iteration\/} f\"{u}r $\vek u_c$
\begin{align}
\vek u_c^{i+1} = - \vek K^{-1}\,\vek \Delta\,\vek K\,\vek u_c^i + \vek u \qquad i = 1,2,\ldots
\end{align}
mit $\vek u^0 = \vek u$ als Startvektor nahelegt.

Die Konvergenz kann man beschleunigen, wenn man den Vektor $\vek u_c^{i+1}$ mit dem vorhergehenden Vektor $\vek u_c^i$ wichtet, \cite{Carl2},
\begin{align}
\vek u_c^{i+1} = \vek u_c^i \cdot c_{max} \cdot q + \vek u_c^{i+1} \cdot q = \frac{1}{1 + c_{max}} \cdot (\vek u_c^i \cdot c_{max} + \vek u_c^{i+1})
\end{align}
wobei $c_{max}$ der gr\"{o}{\ss}te Skalenfaktor $c_e$ unter den \"{A}nderungen, $\vek \Delta \vek K_e = c_e\,\vek K_e$ ist und
\begin{align}
q = \frac{1}{1 + c_{max}}\,.
\end{align}
Theoretisch versagt die Iteration, wenn das Tragwerk durch die Wegnahme eines Elementes kinematisch wird, wenn also die Steifigkeitsmatrix
\begin{align}
\vek K_c = \vek K + \vek \Delta \vek  K
\end{align}
singul\"{a}r ist. Denn dann gibt es einen Vektor $\vek u_0 \neq \vek 0$ so, dass
\begin{align}
(\vek K +  \vek \Delta \vek K) \vek u_0 = \vek 0\,,
\end{align}
und wenn wir diese Gleichung von links mit der Inversen $\vek K^{-1}$ multiplizieren, so zeigt sich,
\begin{align}
\vek K^{-1} (\vek K +  \vek \Delta \vek K) \vek u_0 = (\vek I - \vek K^{-1}  \vek K) \vek u_0 = \vek 0\,,
\end{align}
dass der Vektor $\vek u_0$ ein Eigenvektor von $\vek K^{-1} \vek \Delta \vek K$  mit dem Eigenwert 1 ist und der Fehler $\vek e_{i+1} = \vek u_{i+1} - \vek u_i$ daher nicht schrumpft
\begin{align}
\vek e_{i+1} = - \vek K^{-1}  \vek \Delta \vek K \vek e_i   \qquad    i = 1,2,\ldots\,,
\end{align}
weil eine Fixpunktiteration nur Erfolg hat, wenn die Eigenwerte der Iterationsmatrix kleiner als 1 sind.

Man beachte, dass
\begin{align}
\vek e_{i+1} &= \vek u^{i+1} - \vek u^i = - \vek K^{-1} \vek \Delta  \vek K \vek u_i  + \vek u  + \vek K^{-1} \vek \Delta  \vek K \vek u_{i-1} - \vek u\\
 &= - \vek K^{-1} \vek \Delta  \vek K (\vek u_i - \vek u_{i-1}) = - \vek K^{-1} \vek \Delta  \vek K \vek e_i\,.
\end{align}
Aber es ist bemerkenswert, dass bei nicht zu gro{\ss}en St\"{o}rungen die Iteration konvergiert, selbst dann wenn das Entfernen eines Elementes das Tragwerk instabil macht, weil das Programm nicht versucht, die nicht existierende Inverse der singul\"{a}ren Steifigkeitsmatrix $\vek K + \vek \Delta \vek K$ zu berechnen.

\textcolor{sectionTitleBlue}{\subsection{Direkte Berechnung -- \glq von 2 auf 100\grq{}}}
Wiederholen wir noch einmal die Gleichung, die zu l\"{o}sen ist
\begin{align} \label{FX7}
(\vek I + \vek K^{-1}\,\vek \Delta\,\vek K)\,\vek u_c = \vek u\,.
\end{align}
Das Produkt $ \label{F1} \vek K^{-1}\,\vek \Delta \vek K $
ist eine schwach besetzte Matrix, weil die zur vollen Gr\"{o}{\ss}e $n \times n$ erweiterte Elementmatrix $\vek \Delta\,\vek K$ viel \glq Luft\grq{}, viele Nullen enth\"{a}lt. Nehmen wir an, dass die Matrix $\vek \Delta \vek K$ nur vier Eintr\"{a}ge enth\"{a}lt, die nicht null sind
\begin{align}
\vek \Delta \vek K = \left[\barr{r r r r r r r r}
\phantom{0} & \phantom{0}  & \phantom{0}& ... & \phantom{0} &\phantom{0} &\phantom{0}\\
.&0 & 0  & 0 & 0 &0 &0 &.\\
.&0 & k^\Delta_{33}  & 0 & 0 &k^\Delta_{35} & 0&.\\
.&0 & 0  & 0 & 0 &0 &0 &.\\
.&0 & k^\Delta_{53}  & 0 & 0 &k^\Delta_{55} & 0&.\\
.&0 & 0  & 0 & 0 &0 &0&.\\
\phantom{0} &\phantom{0} & \phantom{0}  & ... & \phantom{0} &\phantom{0} &\phantom{0}
\earr\right]
\end{align}
zwei in Zeile $3$ und zwei in Zeile $5$ (wenn wir zum Beispiel ein Federelement \"{a}ndern).

Das Produkt von zwei $n \times n$ Matrizen $\vek A$ und $\vek B$ kann als die Summe von $n$ Matrizen geschrieben werden,
\begin{align}
\vek A \vek B = \vek c_1\,\vek r_1 + \vek c_2\,\vek r_2 + \ldots + \vek c_n \vek r_n\,,
\end{align}
Spalte $\vek c_1$ von $\vek A$ mal Zeile $\vek r_1$ von $\vek B$ plus Spalte $\vek c_2$ von $\vek A$ mal Zeile $\vek r_2$ von $\vek B$ etc. Jedes Produkt $\vek c_i\,\vek r_i$ ist eine $(n \times n)$ Matrix.

Weil $\vek \Delta\vek K$ schwach besetzt ist, es enth\"{a}lt ja nur zwei Zeilen, die nicht null sind, folgt
\begin{align}
\vek K^{-1} \vek \Delta\vek K = \vek c_3\, \vek r_3 + \vek c_5\, \vek r_5 \qquad (\text{Summe von zwei $n \times n$ Matrizen})
\end{align}
was, wenn wir das in Spaltenform schreiben, das  Ergebnis
\begin{align}
\vek K^{-1} \vek \Delta\vek K = \left[\vek 0\quad \vek  0\quad \vek s_3 \quad \vek 0 \quad\vek s_5\quad\vek 0\,\ldots \,\vek 0\right] \qquad \text{Spaltenvektoren}
\end{align}
ergibt, mit den beiden Spalten
\begin{align}
\vek s_3 = k^\Delta_{33} \cdot \vek c_3 + k^\Delta_{5 3}\cdot \vek c_5  \qquad \vek s_5 = k^\Delta_{3 5}\cdot\vek c_3 + k^\Delta_{5 5}\cdot\vek c_5\,,
\end{align}
die Linearkombinationen der beiden Spalten $\vek c_3$ und $\vek c_5$ von $\vek K^{-1}$ sind.
%-----------------------------------------------------------------
\begin{figure}[tbp]
\centering
\includegraphics[width=1.0\textwidth]{\Fpath/U353}
\caption{One-Click Reanalysis (ohne Neuberechnung der Steifigkeitsmatrix) \textbf{ a)} die angeklickten Elemente \textbf{ b)} die Momentenverteilung ohne die beiden St\"{u}tzen}
\label{U353}
\end{figure}%
%-----------------------------------------------------------------
Somit geht das System (\ref{FX7}) \"{u}ber in
\begin{align}\label{Eq8}
\vek u_c + \vek s_3\,u_{c 3} + \vek s_5 \,u_{c 5} = \vek u\,.
\end{align}
Wir w\"{a}hlen die Zeilen drei und f\"{u}nf dieses Systems und bestimmen an Hand dieser beiden Gleichungen die beiden Werte $u_{c 3}$ und $u_{c 5}$
\begin{align} \label{F3}
 \left[\barr{c c  } 1 + s_{33} & s_{35} \\
 s_{53} & 1 + s_{55} \earr\right] \left[\barr{c } u_{c3} \\u_{c5}\earr\right]= \left[\barr{c } u_{3} \\u_{5}\earr\right]\,.
\end{align}
Die Kenntnis der beiden Zahlen $u_{c 3}$ und $u_{c 5}$ reicht aus, das ist das Erstaunliche, um die ganze L\"{o}sung zu bestimmen
\begin{align}\label{Eq148}
\vek u_c &= - \vek K^{-1}\,\vek \Delta \vek K\,\vek u_c + \vek u = - u_{c 3}\,\vek s_3 - u_{c 5}\,\vek s_5 + \vek u \nn \\
&= - (u_{c3}\,k^\Delta_{33} + u_{c5}\,k^\Delta_{35})\,\vek c_3 - (u_{c5}\,k^\Delta_{53} + u_{c5}\,k^\Delta_{55})\,\vek c_5 + \vek u \nn \\
&= \alpha_3 \cdot \vek c_3 + \alpha_5 \cdot \vek c_5 + \vek u\,.
\end{align}
Bei Betrachtung von (\ref{Eq148}) und (\ref{Eq149}) (s.u.) erkennt man das Muster: \\

\hspace*{-12pt}\colorbox{highlightBlue}{\parbox{0.98\textwidth}{
Der neue Vektor $\vek u_c$ ist eine Summe aus den Spalten $\vek c_i$ der Steifigkeiten $k_i$, mit Vorfaktoren $\alpha_i$, die sich aus einem linearen System wie (\ref{F3}) ergeben, plus dem alten Vektor $\vek u$.}}\\

Um die Technik allgemeiner zu fassen, schreiben wir die beiden Matrizen als eine Folge von Spaltenvektoren $\vek c_i$ bzw. Zeilenvektoren $\vek r_i$
\begin{align}
\vek K^{-1} = [\vek c_1, \vek c_2, \ldots, \vek c_n] \qquad \vek \Delta \vek K = [\vek r_1, \vek r_2, \ldots, \vek r_n]^T\,.
\end{align}
Das System $\vek K^{-1}\,\vek \Delta K = [\vek s_1, \vek s_2, \ldots, \vek s_n] $ hat dann die Spalten
\begin{align}
\vek s_i = \sum_{j = 1}^n\,\vek c_j\,r_{ji}
\end{align}
und so lautet das Gleichungssystem (\ref{FX7}) zeilenweise
\begin{align}
u_{ci} + \sum_{j = 1}^n s_{ij}\,u_{cj} = u_i \qquad i = 1,2,\ldots, n\,.
\end{align}
Weil die meisten Zeilen $\vek r_i$ null sein werden, wird sich das wieder auf ein wesentlich kleineres System reduzieren lassen, das man wie eben erl\"{a}utert behandeln kann.

Man kann so auch die ganze Inverse $\vek K_c^{-1} = [\vek c_1^c, \vek c_2^c, \ldots, \vek c_n^c]$ spaltenweise berechnen, indem man f\"{u}r $\vek u$ nacheinander die alten Spalten $\vek c_i$ setzt.

Die \glq offizielle\grq{} Methode um $(\vek K + \vek \Delta \vek K)^{-1}$ aus $\vek K^{-1}$ zu berechnen, ist die {\em Sherman-Morrison-Woodbury\/}-Formel, \cite{Golub},\index{Sherman-Morrison-Woodbury}\label{Korrektur14}
\begin{align}
(\vek K + \vek \Delta\,\vek K)^{-1} = \vek K^{-1} - \vek K^{-1} \vek U\,(\vek I + \vek V\,\vek K^{-1}\,\vek U)^{-1} \,\vek V\,\vek K^{-1}
\end{align}
wobei $\vek \Delta \vek K = \vek U\,\vek V^T$. Sie gleicht einer {\em black box\/}, bei der es schwerf\"{a}llt, den statischen Gehalt hinter der Formel zu entdecken.

\"{A}ndert sich nur die L\"{a}ngssteifigkeit $k \to k + \Delta k$ in einer St\"{u}tze, dann muss man sogar nur eine skalare Gleichung l\"{o}sen, um die neue Gleichgewichtslage $\vek u_c$ zu finden.

Sei $u_3$ die Zusammendr\"{u}ckung der St\"{u}tze, dann reduziert sich das obige System  auf die Gleichung
\begin{align}
(1 + s_{33})\, u_{c3} = u_3 \qquad s_{33} = \Delta k \cdot c_{33}\,,
\end{align}
wenn $\vek c_{3}$ die Spalte 3 der Inversen $\vek K^{-1}$ ist, und die ge\"{a}nderte L\"{o}sung ergibt sich somit zu
\begin{align}\label{Eq149}
\vek u_c = - u_{c 3}\,\vek s_3 + \vek u = \boxed{-\frac{u_3}{(1 + \Delta k \cdot c_{33})}\,\Delta k } \cdot \vek c_3\ + \vek u\,.
\end{align}
Der Vektor $\vek c_3$ sind die Koeffizienten der Einflussfunktion f\"{u}r die Zusammendr\"{u}ckung $u_3$ der St\"{u}tze und eine \"{A}nderung $\Delta k$ ist daher dann merklich, wenn der Vorfaktor vor $\vek c_3$ gro{\ss} ist und die Einflussfunktion weit ausstrahlt.

%%%%%%%%%%%%%%%%%%%%%%%%%%%%%%%%%%%%%%%%%%%%%%%%%%%%%%%%%%%%%%%%%%%%%%%%%%%%%%%%%%%%%%%%%%%%%%%%%%%%%%%%
\textcolor{sectionTitleBlue}{\section{One-Click Reanalysis}}
In dem Programm BE-FRAMES, s. S. \pageref{SoftwareDownload}, sind beide Techniken, Iteration und direkte L\"{o}sung, zu Lehrzwecken als {\em One-Click Reanalysis\/} implementiert, s. Abb. \ref{U353}. Der Student kann durch einfache Klicks Ver\"{a}nderungen an einem Rahmen vornehmen und die Effekte, die dadurch entstehen, studieren\index{BE-FRAMES}.

Implementiert sind \"{A}nderungen der Art $\vek \Delta \vek K_e = c \cdot \vek K_e$,
also eine Skalierung einzelner Elementmatrizen mit unterschiedlichen Faktoren $c$, wobei $c = 0$ einem kompletten Verlust des Elements entspricht.

Eine Serie von Modifikationen, etwa \"{A}nderungen der Elemente  5, 7 und 9, bedeutet einfach, dass $\vek \Delta \vek K$ eine Ansammlung von skalierten Elementmatrizen $\vek \Delta \vek K_e$ ist
\begin{align}
\vek u_c = - \vek K^{-1}\,(\vek \Delta\,\vek K_5 + \vek \Delta\,\vek K_7 + \vek \Delta\,\vek K_9 )\,\vek u_c + \vek u \qquad \text{(direkte Lsg.)}
\end{align}
Nat\"{u}rlich k\"{o}nnen auch einzelne Lager entfernt werden, nachtr\"{a}glich Gelenke eingebaut werden (das geht mit Dirac Deltas) und Einflussfunktionen f\"{u}r alle interessierenden Gr\"{o}{\ss}en berechnet werden.
%-----------------------------------------------------------------
\begin{figure}[tbp]
\centering
\includegraphics[width=1.0\textwidth]{\Fpath/U363}
\caption{Verschieblichkeiten in einem Rahmen lassen sich mit dem \glq zweifachen\grq{} Gauss entdecken}
\label{U363}
\end{figure}%
%-----------------------------------------------------------------

\textcolor{sectionTitleBlue}{\subsection{Wenn die Last \glq getroffen\grq{} wird}}

Wenn der Student ein Element $\Omega_e$ entfernt, das eine Belastung tr\"{a}gt, dann ist der neue Vektor $\vek u_c$ die L\"{o}sung des Systems
\begin{align}
(\vek K + \vek \Delta \vek K)\,\vek u_c = \vek f - \vek f_e\,,
\end{align}
wobei die Eintr\"{a}ge in dem Vektor $\vek f_e$ die vorherigen \"{a}quivalenten Knotenkr\"{a}fte aus dem Element $\Omega_e$ sind. In diesem Fall muss das Programm also auch das Original der rechten Seite \"{a}ndern, $\vek f \to  \vek f - \vek f_e$.

%%%%%%%%%%%%%%%%%%%%%%%%%%%%%%%%%%%%%%%%%%%%%%%%%%%%%%%%%%%%%%%%%%%%%%%%%%%%%%%%%%%%%%%%%%%%%%%%%%%%%%%%
\textcolor{sectionTitleBlue}{\subsection{Singul\"{a}re Steifigkeitsmatrizen}}
Weil es in dem Programm BE-FRAMES implementiert ist, sei noch erw\"{a}hnt, dass man mit dem Gau{\ss} Algorithmus auch Beweglichkeiten in Rahmen aufdecken kann. Der Gau{\ss} Algorithmus formt ja die Steifigkeitsmatrix in eine obere Dreiecksmatrix um. Geht man noch einen Schritt weiter, und wendet den Gau{\ss} Algorithmus auch auf die obere Dreiecksmatrix an, dann erh\"{a}lt man am Schluss -- bei regul\"{a}ren Matrizen -- die Einheitsmatrix. Wenn allerdings die Matrix singul\"{a}r ist, dann f\"{u}hren die letzten Spalten in dem Ergebnis auf die Eigenvektoren zu dem Eigenwert $\lambda = 0$, also einfach die  m\"{o}glichen Starrk\"{o}rperbewegungen, die in dem Rahmen stecken, $\vek K\,\vek u = \vek 0$. Die kann man dann auf dem Bildschirm anzeigen und so den Benutzer auf fehlende Festhaltungen aufmerksam machen, s. Abb. \ref{U363}.

Die Matrix, die bei diesem \glq zweimaligen\grq{} Gauss entsteht, nennt man die {\em row reduced echelon form\/} \index{row reduced echelon form} einer Matrix und unter diesem Namen findet man den Algorithmus auch in der Literatur.

%-----------------------------------------------------------------
\begin{figure}[tbp]
\centering
\includegraphics[width=0.95\textwidth]{\Fpath/U326D}
\caption{Der Einbau von plastischen Gelenken in einem Durchlauftr\"{a}ger \textbf{ a)} urspr\"{u}ngliche Momentenverteilung, \textbf{ b)} nach dem Einbau der Gelenke, \textbf{ c)} Biegelinie}
\label{U326}
\end{figure}%
%-----------------------------------------------------------------

%%%%%%%%%%%%%%%%%%%%%%%%%%%%%%%%%%%%%%%%%%%%%%%%%%%%%%%%%%%%%%%%%%%%%%%%%%%%%%%%%%%%%%%%%%%%%%%%%%%%%%%%
\textcolor{sectionTitleBlue}{\section{Nachtr\"{a}glicher Einbau von Gelenken}}
Andere m\"{o}gliche Defekte, die sich unter Last ausbilden, sind plastische Gelenke. Mathematisch ist  ein Gelenk eine Nullstelle im Momentenverlauf, $M(x) = 0$. Was wir daher machen ist, dass wir ein Dirac Delta $\delta_2$ im Punkt $x$ wirken lassen, d.h. wir berechnen die Einflussfunktion f\"{u}r  $M(x)$ in diesem Punkt und wir berechnen, wie gro{\ss} das Moment $M_2(x)$ der Einflussfunktion selbst in diesem Punkt $x$ ist. Dann skalieren wir das Dirac Delta mit einem Faktor $a$ derart, dass $a \cdot M_2(x) + M(x) = 0$.

So wurden die Ergebnisse in Abb. \ref{U326} erzielt. Auch dies steht als \glq {\em one-click\/}\grq{} operation in dem Programm zur Verf\"{u}gung.

Im Detail geht es wie folgt: Es sei $x_0$  der Punkt, an dem ein Gelenk eingebaut werden soll -- oder besser -- ein Moment $M_p$ aus einem LF $p$ zu Null gemacht werden soll. Der Punkt muss nicht am Ende eines Feldes liegen, wie die Punkte in Abb. \ref{U326}.

(1) Man berechnet zun\"{a}chst die Einflussfunktion $G_2$ f\"{u}r das Moment $M$ im Punkt $x_0$, indem man in den  Nachbarknoten die \"{a}quivalenten Knotenkr\"{a}fte, die zu der Einflussfunktion geh\"{o}ren, s. Abb. \ref{U451}, aufbringt. (2) Man addiert zu dieser L\"{o}sung die lokale L\"{o}sung, also die Einflussfunktion am beidseitig eingespannten Tr\"{a}ger. (3) Man berechnet das Moment $M_G(x)$ der so zusammengesetzten Einflussfunktion im Punkt $x_0$. (4) Man skaliert dann den LF Einflussfunktion so, dass das Moment $M_{G}(x_0)$ gerade $- M_p(x_0)$ ist. (5) Die Ergebnisse dieses Lastfalls zu dem LF $p$ addiert, sind die Ergebnisse mit einem Gelenk im Punkt $x_0$ im LF $p$.

Werden mehrere Gelenke eingebaut, dann muss man die Spreizung der Gelenke untereinander abstimmen, also ein lineares Gleichungssystem l\"{o}sen, so dass in allen Punkten gleichzeitig die Momente gegengleich zu den Momenten im LF $p$ sind.

%%%%%%%%%%%%%%%%%%%%%%%%%%%%%%%%%%%%%%%%%%%%%%%%%%%%%%%%%%%%%%%%%%%%%%%%%%%%%%%%%%%%%%%%%%%%%%%%%%%
\textcolor{sectionTitleBlue}{\section{Knicklasten}}
Auch die Knicklasten\index{Knicklasten} -- also  die Eigenwerte der Steifigkeitsmatrix $\vek K$ (nach Theorie II. Ordnung) -- \"{a}ndern sich, wenn sich Steifigkeiten \"{a}ndern. Das ist ein komplexes Thema, weil ja die Rechentechnik des Ingenieurs, Stichwort: {\em Nachweis am Einzelstab\/}, mindestens eine genauso gro{\ss}e Rolle spielt, wie die Mathematik f\"{u}r die wir z.B. auf \cite{Bangerth} verweisen.

Wir wollen hier nur das {\em Hellmann-Feynman-Theorem\/} zitieren, das aus der Quantenmechanik stammt, und das festh\"{a}lt, wie sich die Eigenwerte eines Systems \"{a}ndern, wenn sich die Modellparameter \"{a}ndern.

Gegeben sei das Eigenwertproblem
\bfo
(\vek K(\vek p)  - \lambda\,\vek  I)\,\vek x = \vek 0\,.
\efo
Die $(n \times n)$-Matrix $\vek K(\vek p)$ h\"{a}nge von $m$ Modellparametern $p_i$ ab. Es
sei $\lambda$ ein Eigenwert und $\vek x$ ein zugeh\"{o}riger, normierter Eigenvektor, $|\vek
x| = 1$.

Dann gilt
\bfo
\frac{\partial \lambda}{\partial p_i} = \vek x^T\,\vek K,_{p_i}\,\vek x\,.
\efo
{\em Beweis:\/} Weil $\lambda$ ein Eigenwert ist, gilt
\bfo
\lambda = \vek x^T\,\vek K\,\vek x\,.
\efo
Differenziert man beide Seiten nach $p_i$ so folgt
\bfo
\frac{\partial \lambda}{\partial p_i} = \vek x^T\,\vek K,_{p_i}\,\vek x + \vek
x,_{p_i}^T\,\vek K\,\vek x + \vek x^T\,\vek K\,\vek x,_{p_i}\,.
\efo
Nachdem $\vek x = \vek x(\vek p)$ ein Eigenvektor ist, kann dies geschrieben werden als
\begin{align}
\frac{\partial \lambda}{\partial p_i} &= \vek x^T\,\vek K,_{p_i}\,\vek x + \lambda(\vek
p)\,\vek x,_{p_i}^T\,\vek x + \lambda(\vek p)\,\vek x^T\,\vek x,_{p_i} \nn \\
&=  \vek x^T\,\vek
K,_{p_i}\,\vek x + \lambda(\vek p) \left[\vek x,_{p_i}^T\,\vek x + \vek x^T\,\vek
x,_{p_i}\right].
\end{align}
Weil die (normierte) L\"{a}nge des Eigenvektors invariant ist gegen\"{u}ber den Parametern $p_i$
\bfo
|\vek x| = \vek x^T\,\vek x = 1 \qquad \Rightarrow \qquad \vek x,_{p_i}^T\,\vek x + \vek
x^T\,\vek x,_{p_i} =0
\efo
folgt schlie{\ss}lich
\bfo
\frac{\partial \lambda}{\partial p_i} &=& \vek x^T\,\vek K,_{p_i}\,\vek x\,.
\efo
Der praktische Nutzen dieses Theorems besteht darin, dass uns versichert wird, dass die \"{A}nderungen in den Eigenwerten proportional zu den Ableitungen der Steifigkeitsmatrix nach den Parametern $p_i$ sind.

%%%%%%%%%%%%%%%%%%%%%%%%%%%%%%%%%%%%%%%%%%%%%%%%%%%%%%%%%%%%%%%%%%%%%%%%%%%%%%%%%%%%%%%%%%%%%%%%%%%
\textcolor{sectionTitleBlue}{\section{Die Vektoren $\vek f^+, \vek u^+, \vek g^+, \vek j^+$}}\label{Korrektur2}
Wir fassen zusammen: Die Vektoren $\vek u$ und $\vek g$ sind Wege, die Vektoren $\vek f$ und $\vek j$ sind \"{a}quivalente Knotenkr\"{a}fte.
\index{$\vek f^+$}\index{$\vek j^+$}\index{$\vek u^+$}\index{$\vek g^+$}
Das System $(\vek K + \vek \Delta \vek K)\,\vek u_c = \vek f$ ist identisch mit $\vek K\,\vek u_c = \vek f - \vek \Delta \vek K\,\vek u_c$ und hat die L\"{o}sung
\begin{align}
\vek u_c &= \vek u + \vek u^+ = \vek K^{-1}\,(\vek f + \vek f^+)
\end{align}
wobei
\begin{align}
\vek f^+ = - \vek \Delta\, \vek K\,\vek u_c \qquad \vek u^+ = \vek K^{-1}\,\vek f^+\,.
\end{align}
Das System $(\vek K + \vek \Delta \vek K)\,\vek g_c = \vek j_c$ zur Berechnung der Knotenwerte $g_i$ der Einflussfunktion f\"{u}r ein Funktional $J_c(\vek u_c) = \vek j_c^T\,\vek u_c = \vek g_c^T\,\vek f$ hat die L\"{o}sung
\begin{align}
\vek g_c &= \bar{\vek g} + \vek g^+ = \vek K^{-1}\,(\vek j_c + \vek j_c^+)\,,
\end{align}
wobei
\begin{align}
\bar{\vek g} = \vek K^{-1}\,\vek j_c \qquad \vek j_c^+ = - \Delta\, \vek K\,\vek g_c \qquad \vek g^+ = \vek K^{-1}\,\vek j_c^+\,.
\end{align}
Wenn sich die Steifigkeit des Elements, in dem $J_c(\vek u)$ ausgewertet wird, nicht \"{a}ndert, wenn also $J(\vek u) = J_c(\vek u)$ ist, dann ist $\vek j = \vek j_c$, und entsprechend nennen wir $\vek j_c^+ = \vek j^+$, und $\bar{\vek g} = \vek g$, so dass dann also mit (\ref{Eq104}) gilt
\begin{align}
J(\vek u_c) = (\vek g + \vek g^+)^T\,\vek f = (\vek j + \vek j^+)^T\,\vek u\,.
\end{align}
Vorsicht!  Es ist nicht $\vek j_c = \vek j + \vek j^+$. Der Vektor $\vek j^+$ ist nur eine Hilfsgr\"{o}{\ss}e, der die Formel
\begin{align}\label{Eq80}
J(\vek u_c) = (\vek j + \vek j^+)^T\,\vek u
\end{align}
m\"{o}glich macht, mit der man aus dem alten $\vek u$ die Werte des neuen $\vek u_c$ berechnen kann.
%-----------------------------------------------------------------
\begin{figure}[tbp]
\centering
\includegraphics[width=0.95\textwidth]{\Fpath/U473}
\caption{Das Federgesetz $u = 1/k \cdot f$ impliziert, dass Steifigkeits\"{a}nderungen, $\pm \Delta k$, zu unterschiedlich gro{\ss}en Korrekturen f\"{u}hren}
\label{U473}
\end{figure}%
%-----------------------------------------------------------------

Wenn $\vek j_c = \vek j + \vek j^+$ der richtige Vektor $\vek j_c$ w\"{a}re, dann m\"{u}sste man mit ihm $J(\vek u_c)$ aus $\vek u_c$ berechnen k\"{o}nnen,
\begin{align}
J(\vek u_c) = (\vek j + \vek j^+)^T\,\vek u_c \qquad \text{(?)}
\end{align}
was aber (\ref{Eq80}) widerspricht. Es kann nicht beides richtig sein.

Wie man an Abb. \ref{U394} sieht, sind die Kr\"{a}fte $j_i^+$ in den Ecken des ge\"{a}nderten Elements Zusatzkr\"{a}fte, um auf dem \glq alten\grq{} Netz (Matrix $\vek K$) die Wirkung des Dirac Deltas (Einzelkraft in der oberen rechten Ecke) darstellen zu k\"{o}nnen, also $\vek g_c$ zu generieren. Ohne die $j_i^+$ w\"{u}rde das Dirac Delta den Vektor $\vek g$ erzeugen, mit den $j_i^+$ nimmt das \glq alte\grq{} Netz die Form $\vek g_c$ an.

In Abb. \ref{U369} auf S. \pageref{U369} wurde diese Technik bei dem 1-D Problem eines Stabes benutzt, um mit Hilfe von Kr\"{a}ften $j^+$ die Einflussfunktion des {\em stepped bars\/} an einem Stab mit konstantem Querschnitt zu berechnen.


%%%%%%%%%%%%%%%%%%%%%%%%%%%%%%%%%%%%%%%%%%%%%%%%%%%%%%%%%%%%%%%%%%%%%%%%%%%%%%%%%%%%%%%%%%%%%%%%%%%
\textcolor{sectionTitleBlue}{\subsection{Unsymmetrie in den Ausgleichsbewegungen}}
Zum Schluss wollen wir noch anmerken, dass, wegen $u = 1/k \cdot f$, \"{A}nderungen $k \pm \Delta k$ unsymmetrisch ablaufen, s. Abb. \ref{U473}. Die Ausschl\"{a}ge $\Delta u$ bei einer Abnahme $-\Delta k$ sind gr\"{o}{\ss}er als die Verk\"{u}rzungen $\Delta u$ bei einer Zunahme $+\Delta k$. Anders gesagt: Wenn man die Steifigkeit $k$  einer Feder um $\Delta k$ verringert und dann wieder $\Delta k$ dazu addiert, ist man nicht da, wo man vor dem Man\"{o}ver war, sondern $f$ h\"{a}ngt tiefer.

Der Grund ist, dass man sich bei der Abnahme auf der Kurve $u = 1/k \cdot f$ befindet und beim R\"{u}ckweg auf der Kurve $u = 1/(k - \Delta k) \cdot f$.

Wenn man aus einem Reifen (5 l) ein Liter Luft (= 20 \%) herausl\"{a}sst und dann wieder hinzuf\"{u}gt, dann hat der Wagen seine alte H\"{o}he. Aber bezogen auf das reduzierte Volumen (4 l) ist 1 Liter Luft 25 \%. Rechnerisch h\"{a}tte man nur 0.8 l hinzuf\"{u}gen d\"{u}rfen, was nicht gereicht h\"{a}tte, um auf die alte H\"{o}he zu kommen. 
%%%%%%%%%%%%%%%%%%%%%%%%%%%%%%%%%%%%%%%%%%%%%%%%%%%%%%%%%%%%%%%%%%%%%%%%%%%%%%%%%%%%%%%%%%%%%%%%%%%
\textcolor{chapterTitleBlue}{\chapter{Singularit\"{a}ten}}}
%%%%%%%%%%%%%%%%%%%%%%%%%%%%%%%%%%%%%%%%%%%%%%%%%%%%%%%%%%%%%%%%%%%%%%%%%%%%%%%%%%%%%%%%%%%%%%%%%%%
In diesem Kapitel geht es um die Frage, wann Spannungsspitzen auftreten, und wie man ihnen mit Einflussfunktionen auf die Spur kommen kann.

%%%%%%%%%%%%%%%%%%%%%%%%%%%%%%%%%%%%%%%%%%%%%%%%%%%%%%%%%%%%%%%%%%%%%%%%%%%%%%%%%%%%%%%%%%%%%%%%%%%%
\textcolor{sectionTitleBlue}{\section{Singul\"{a}re Spannungen}}
%%%%%%%%%%%%%%%%%%%%%%%%%%%%%%%%%%%%%%%%%%%%%%%%%%%%%%%%%%%%%%%%%%%%%%%%%%%%%%%%%%%%%%%%%%%%%%%%%%%%
Spannungen sind proportional zu den Dehnungen, $\sigma = E\cdot \varepsilon$, also proportional zu den Ableitungen der Verschiebungen, s. Abb. \ref{U76}, und so entstehen singul\"{a}re Spannungen immer dann, wenn die Verschiebungen sich schlagartig \"{a}ndern, sie praktisch aus dem Stand heraus von null
nach oben schie{\ss}en, s. Abb. \ref{U76} a.

%----------------------------------------------------------------------------------------------------------
\begin{figure}[tbp]
\centering
\if \bild 2 \sidecaption \fi
\includegraphics[width=.9\textwidth]{\Fpath/U76}
\caption{Je nachdem, wie die Verschiebungen ausklingen, sind die Spannungen endlich oder
unendlich. Die schnellste Verbindung von $A$ nach $B$ im Schwerefeld der Erde ist nicht die k\"{u}rzeste Verbindung ($---$), sondern eine
Zykloide. Weil die Anfangsbeschleunigung in den tieferen Startpunkten $A_1$ bzw. $A_2$
kleiner ist als in $A$ (flachere Tangenten), dauert die Reise von dort aus nach $B$
genauso lange wie von $A$ aus} \label{U76}
\end{figure}%%
%----------------------------------------------------------------------------------------------------------


Die {\em Brachystochrone\/}\index{Brachystochrone}, ($\beta\rho\alpha\chi\upsilon\sigma$ = kurz), illustriert die Situation am besten. Die Brachystochrone ist die Kurve, die zwei vorgegebene Punkte $A$ und $B$ so verbindet, dass man mit Hilfe des Schwerefelds der Erde m\"{o}glichst schnell von $A$ nach $B$ kommt. Die L\"{o}sung dieses ber\"{u}hmten Problems ist eine {\em Zykloide\/}, s. Abb. \ref{U76}\index{Zykloide}.

Es ist also nicht der k\"{u}rzeste Weg, der am schnellsten zum Ziel f\"{u}hrt, sondern der Weg, bei dem wir am Anfang  m\"{o}glichst viel Geschwindigkeit holen, indem wir uns senkrecht nach unten fallen lassen.

Genauso verhalten sich die Bauteile, denn das Material will m\"{o}glichst schnell weg aus der Gefahrenzone, wie etwa einem Riss, s. Abb. \ref{U244}, und so l\"{a}uft die vertikale Verschiebung $u_y$ mit unendlich gro{\ss}em \glq Tempo\grq{}, unendlich gro{\ss}er Steigung aus dem Rissgrund heraus und dies f\"{u}hrt damit nat\"{u}rlich zu unendlich gro{\ss}en Spannungen $\sigma_{yy}$.

%----------------------------------------------------------------------------------------------------------
\begin{figure}[tbp]
\centering
\if \bild 2 \sidecaption \fi
\includegraphics[width=.6\textwidth]{\Fpath/U244}
\caption{Die Spannungen $\sigma_{yy}$ im Rissgrund sind unendlich gro{\ss}, weil $u_y$ mit unendlich gro{\ss}er Steigung aus dem Rissgrund herausl\"{a}uft ($\nu = 0$)} \label{U244}
\end{figure}%%
%----------------------------------------------------------------------------------------------------------

Beim Auto sagt man: {\em  Wo der Weg (= Bremsweg) null ist, ist die Kraft
unendlich\/} und was beim Auto die Beschleunigung $a = dv/dt$ ist, ist bei Tragwerken die Verzerrung $\varepsilon = du/dx$ (Scheiben) bzw. die Kr\"{u}mmung $\kappa = d^{\,2}w/dx^2$ (Platten).

Rei{\ss}t eine Scheibe auf, dann ist, weil die Bruchfl\"{a}chen vorher den Abstand $dx = 0$ hatten, bei noch so kleiner Riss\"{o}ffnung $du$ die Verzerrung unendlich gro{\ss}, $du/dx = du/0 = \infty$.

Sinngem\"{a}{\ss} dasselbe gilt f\"{u}r einen Knick in einer Platte, weil in einem solchen Punkt der Kr\"{u}mmungskreisradius $R$ null ist und der Kehrwert $\kappa = 1/R$ somit unendlich gro{\ss} wird.
%---------------------------------------------------------------------------------
\begin{figure}
\centering
\includegraphics[width=0.90\textwidth]{\Fpath/U108}
\caption{Durchlauftr\"{a}ger \textbf{ a)}
Je k\"{u}rzer das mittlere Feld wird, um so steiler werden die Momente und um so gr\"{o}{\ss}er damit die Querkraft, $V = M'$, \textbf{ b)}
Einflussfunktion f\"{u}r die Querkraft in Feldmitte}
\label{U108}%
\end{figure}%
%---------------------------------------------------------------------------------

%---------------------------------------------------------------------------------
\begin{figure}
\centering
\includegraphics[width=0.90\textwidth]{\Fpath/U526}
\caption{Vierendeeltr\"{a}ger als Kragtr\"{a}ger. Je k\"{u}rzer die vertikalen Sprossen werden, desto gr\"{o}{\ss}er werden die Querkr\"{a}fte in den Sprossen}
\label{U526}%
\end{figure}%

%---------------------------------------------------------------------------------
%%%%%%%%%%%%%%%%%%%%%%%%%%%%%%%%%%%%%%%%%%%%%%%%%%%%%%%%%%%%%%%%%%%%%%%%%%%%%%%%%%%%%%%%%%%%%%%%%%%%
{\textcolor{sectionTitleBlue}{\section{Singul\"{a}re Lagerkr\"{a}fte}}}\label{Korrektur9}

Mit der ungeschickten Plazierung von Festpunkten kann man sich beliebig gro{\ss}e Kr\"{a}fte, sprich Probleme, einhandeln, s. Abb. \ref{U108}.

Die Momente in dem Durchlauftr\"{a}ger, die die Einzelkraft erzeugt, sind zickzackf\"{o}rmig und je enger die beiden Innenlager beieinander stehen, um so gr\"{o}{\ss}er wird die Querkraft, weil die Querkraft ja die Ableitung des Momentes ist
\begin{align}
V = \frac{dM}{dx}\,,
\end{align}
sie also dem Steigungsdreieck des Momentes entspricht. Sinngem\"{a}{\ss} dasselbe gilt f\"{u}r den Vierendeeltr\"{a}ger in Abb. \ref{U526}.
%---------------------------------------------------------------------------------
\begin{figure}
\centering
\includegraphics[width=0.85\textwidth]{\Fpath/U472}
\caption{Blick auf eine Platte -- ein kleiner Versatz in den tragenden Innenw\"{a}nden und die gro{\ss}en Folgen. Die W\"{a}nde wurden mit $EA = \infty$ gerechnet. Elastische Lagerung d\"{u}rfte die Effekte d\"{a}mpfen}
\label{U472}%
\end{figure}%
%---------------------------------------------------------------------------------

Die Einflussfunktion f\"{u}r die Querkraft schwingt immer weiter aus, je k\"{u}rzer der Abstand der beiden Lager wird. Die Aktion, die die Einflussfunktion ausl\"{o}st, die Spreizung $\pm 0.5$, ist immer gleich gro{\ss}, aber die Flanken der Einflussfunktion werden mit $h \to 0$ immer steiler und so w\"{o}lbt sich die Einflussfunktion im Feld immer weiter auf.

Bei der Deckenplatte in Abb. \ref{U472} ist es der Versatz der Innenw\"{a}nde, der diesen Effekt produziert. Den W\"{a}nden gelingt es, wegen ihres kurzen Abstandes, nur sehr schwer das Versatzmoment auszubalancieren und auch die Platte leidet unter der Situation, wie die Oszillationen in den  Hauptmomenten belegen.

%%%%%%%%%%%%%%%%%%%%%%%%%%%%%%%%%%%%%%%%%%%%%%%%%%%%%%%%%%%%%%%%%%%%%%%%%%%%%%%%%%%%%%%%%%%%%%%%%%%%
{\textcolor{sectionTitleBlue}{\section{Einzelkr\"{a}fte}}}\label{EinzelF}
Oft sind singul\"{a}re Spannungen ein R\"{a}tsel. \glq Liegt es an den Elementen oder liegt es an der Statik\grq? Dagegen ist die Situation klar, wenn eine Einzelkraft $ P = 1$ in der Mitte einer Scheibe angreift, s. Abb. \ref{U224} a.
%---------------------------------------------------------------------------------
\begin{figure}
\centering
\includegraphics[width=0.75\textwidth]{\Fpath/U224}
\caption{Einzelkraft bei einer Scheibe und bei einer Platte. Bei einer Platte wachsen die Querkr\"{a}fte ($v_n$) auch wie $1/r$, aber weil $w$ das dreifache Integral der Querkr\"{a}fte ist, ist $w = r^2\,\ln\,r$ auch im Punkt $r = 0$ endlich; $v_n$ ist der Kirchhoffschub }
\label{U224}%
\end{figure}%
%---------------------------------------------------------------------------------

Wenn wir um den Aufpunkt Kreise mit dem Radius $r $ schlagen, dann m\"{u}ssen die \"{u}ber den Kreis aufintegrierten horizontalen Spannungen die Punktlast ergeben -- auch wenn $r \to 0$.

Um dies nun genauer zu fassen, m\"{u}ssen wir etwas ausholen. Was wir \"{u}ber den Kreisumfang integrieren, sind nicht die horizontalen Spannungen, sondern die horizontalen {\em tractions\/}, um hier das englische Wort zu benutzen. Ist $\vek S$ der Spannungstensor in der Scheibe,
\begin{align}
\vek S =  \left[\barr{r r r} \sigma_{xx} &  \sigma_{xy}\\
  \sigma_{yx} & \sigma_{yy}  \earr\right]\,,
\end{align}
dann geh\"{o}rt zu einem Schnitt mit der Schnittnormalen $\vek n = \{n_x,n_y\}^T$ der Spannungsvektor
\begin{align}\label{PPX}
\vek t = \vek S\,\vek n = \left[\barr{r r r} \sigma_{xx} &  \sigma_{xy}\\
  \sigma_{yx} & \sigma_{yy}  \earr\right]\, \left[\barr{c} \cos\,\Np  \\ \sin\,\Np
  \earr \right] \qquad \text{auf dem Kreisumfang}
\end{align}
und das Gleichgewicht verlangt, dass das Integral des Spannungsvektors \"{u}ber den Umfang des Kreises $\Gamma$ gleich der Einzelkraft ist
\begin{align}\label{Ergebnis}
\int_{\Gamma} \vek t\,ds + P\cdot \vek e_1  =  \int_{\Gamma}  \left [\barr{c}  t_x \\  t_y\earr \right ] \,ds + \left [\barr{c}  1 \\  0\earr \right ]=  \left [\barr{c}  0 \\  0\earr \right ]\,.
\end{align}
Nun geht mit immer kleiner werdendem Radius, $r \to 0$, der Umfang des Kreises, $U = 2\,\pi\,r$, gegen null und damit am Ende das Integral der horizontalen Spannungen weiterhin den Wert -1 ergibt, muss sich $t_x$ wie $-1/(2\,\pi\,r)$ verhalten
\begin{align}
\lim_{r \to 0} \int_{\Gamma}t_x\,ds = \int_0^{\,2\,\pi} t_x\,r\,d\Np = -\int_0^{\,2\,\pi}\frac{1}{2\,\pi\,r}\,r\,d\Np = -1\,,
\end{align}
und damit in der Grenze, $r \to 0$, unendlich gro{\ss} werden\footnote{Wegen Details s. S. \pageref{BeweisP}}.

Frage: Um wieviel verschiebt sich der Aufpunkt? Dies finden wir heraus, indem wir die Verzerrungen integrieren. Setzen wir der Einfachheit halber die Querdehnungszahl $\nu = 0$, dann h\"{a}ngt die Dehnung $\varepsilon_{xx} = 1/E\cdot\sigma_{xx}$ nur von der horizontalen Spannung ab und wegen
\begin{align}
\sigma_{xx} =  -\frac{1}{2\,\pi\,r} =  E \cdot \varepsilon_{xx} =  E \cdot \frac{\partial u}{\partial x} \simeq -\frac{1}{r}
\end{align}
folgt, dass sich die horizontale Verschiebung $u $ wie $- \ln\,r$ verh\"{a}lt, weil dies die Stammfunktion von $-1/r$ ist. Dies bedeutet, dass die Verschiebung im Aufpunkt unendlich gro{\ss} wird, denn $-\ln 0 = \infty$.
 %---------------------------------------------------------------------------------
\begin{figure}
\centering
\if \bild 2 \sidecaption[t] \fi
{\includegraphics[width=0.8\textwidth]{\Fpath/U66}}
\caption{Haltekr\"{a}fte = Fl\"{a}chenkr\"{a}fte + Kantenkr\"{a}fte nahe einem Lagerknoten. Die Fl\"{a}chenkr\"{a}fte $\vek p_h$ sind nur \"{u}ber ihre Integrale, Glg. (\ref{Eq63}), das sind die Zahlen in den Elementen, dargestellt. Die Kantenkr\"{a}fte sieht man als Pfeile}
\label{U66}%
\end{figure}%
%---------------------------------------------------------------------------------

Es gilt also:
\begin{itemize}
  \item Unter Einzelkr\"{a}ften werden die Spannungen unendlich gro{\ss}
  \item Die unendlich gro{\ss}en Spannungen f\"{u}hren dazu, dass das Material flie{\ss}t und der Aufpunkt unendlich weit wegwandert.
  \item Punktlager (= Punktkr\"{a}fte) k\"{o}nnen eine Scheibe daher nicht festhalten und man kann auch keine Lagerverschiebung in einem solchen Lager vorschreiben.
 \end{itemize}

Nun kann man aber, all diesem zu Trotz, bei einer FE-Berechnung Knoten festhalten und auch Knotenverschiebungen vorgeben. Wie das?

Des R\"{a}tsels L\"{o}sung ist nat\"{u}rlich, dass die FE-L\"{o}sung keine exakte L\"{o}sung ist. In einem Lagerknoten sind die Verschiebungen $u_i = 0$ in der Tat auf null abgebremst, aber das sind verteilte Kr\"{a}fte, die diesen Halt zuwege bringen, s. Abb. \ref{U66}, und keine echten Einzelkr\"{a}fte.

Im Ausdruck steht zwar eine Knotenkraft $f_i$, aber das ist eine rein rechnerische Gr\"{o}{\ss}e, eine {\em \"{a}quivalente Knotenkraft\/}, die stellvertretend f\"{u}r die wahren Haltekr\"{a}fte wie in Abb. \ref{U66} steht. Es sind Linienkr\"{a}fte l\"{a}ngs den Elementkanten und Fl\"{a}chenkr\"{a}fte, die die Scheibe st\"{u}tzen. Die Zahlen in Abb. \ref{U66} sind die aufintegrierten Fl\"{a}chenkr\"{a}fte der FE-L\"{o}sung pro Element
\begin{align}\label{Eq63}
\int_{\Omega_e} (p_x^2 + p_y^2)\,d\Omega\,.
\end{align}

\vspace{-0.5cm}
%%%%%%%%%%%%%%%%%%%%%%%%%%%%%%%%%%%%%%%%%%%%%%%%%%%%%%%%%%%%%%%%%%%%%%%%%%%%%%%%%%%%%%%%%%%%%%%%%%%
{\textcolor{sectionTitleBlue}{\section{Das Abklingen der Spannungen}}}
Aus einem \"{a}hnlichen Grund wie oben, m\"{u}ssen die Spannungen und Verzerrungen mit wachsendem Abstand von der Last immer kleiner werden. Nur ist es  nicht das Gleichgewicht, sondern der {\em Energieerhaltungssatz\/}, der das zwingend vorschreibt.

Schlagen wir um den Mittelpunkt der Last einen Kreis mit Radius $R$, dann muss die innere Energie in der Kreisscheibe (wir lassen den Faktor $1/2$ weg)
\begin{align}
A_i &= \int_{\Omega} \sigma_{ij}\,\varepsilon_{ij}\,d\Omega = \int_0^{\,R} \int_0^{\,2\,\pi}  \sigma_{ij}\,\varepsilon_{ij}\,r\,dr\,d\Np \nn \\
&=  \int_0^{\,R} \int_0^{\,2\,\pi} ( \sigma_{11}\,\varepsilon_{11} + 2\, \sigma_{12}\,\varepsilon_{12} +  \sigma_{22}\,\varepsilon_{22})\,r\,dr\,d\Np
\end{align}
 %---------------------------------------------------------------------------------
\begin{figure}
\centering
\if \bild 2 \sidecaption[t] \fi
{\includegraphics[width=0.9\textwidth]{\Fpath/U21}}
\caption{Der Energieerhaltungssatz impliziert, dass die Momente abklingen}
\label{U21}%
\end{figure}%
%---------------------------------------------------------------------------------
gleich der \"{a}u{\ss}eren Arbeit $A_a(P)$ der Belastung sein, also \"{u}berschl\"{a}gig gleich der {\em Verformung aus der Belastung $\times$ der Belastung\/} plus der Arbeit $A_a(\Gamma)$ der Schnittkr\"{a}fte entlang dem Kreisumfang $\Gamma$. Ab einem gewissen Radius $R$ ist der Kreis gro{\ss} genug, um die ganze Belastung zu umfassen, und ab diesem Zeitpunkt \"{a}ndert sich die \"{a}u{\ss}ere Arbeit $A_a$ nur noch, weil mit $R$ auch der Umfang $\Gamma $ w\"{a}chst. Die Energiebilanz verlangt
\begin{align}
A_a = A_a(P) + A_a(\Gamma) = a(\vek u,\vek u)_{\Omega} = A_i
\end{align}
und daher ist der Term
\begin{align}
a(\vek u,\vek u)_{\Omega} - A_a(\Gamma) = A_a(P)
\end{align}
konstant. Mit wachsendem Radius $R$ muss sich die Verzerrungsenergie gegenl\"{a}ufig verhalten, um den Zuwachs an Fl\"{a}che $\Omega$ und Rand $\Gamma$ auszubalancieren
\begin{align}
\sigma_{ij}\, \varepsilon_{ij} \simeq \frac{1}{R^2}\,.
\end{align}
\"{A}hnlich muss bei dem Durchlauftr\"{a}ger in Abb. \ref{U21} die innere Energie $A_i$ in jedem rechten Teil $[x',\ell]$ -- von einem Punkt $x' $ im Innern bis zum Ende -- gleich der \"{a}u{\ss}eren Arbeit der Kragarmlast plus der Arbeit $A_a(x')$ der Schnittkraft an der Stelle $x'$ sein
\begin{align}
A_i = \int_{x'}^{\,\ell} \frac{M^2}{EI}\,dx = P \cdot w(\ell) + A_a(x') = A_a\,,
\end{align}
und dies impliziert, dass $M$ abf\"{a}llt, je weiter $x'$ nach links r\"{u}ckt.\\

\hspace*{-12pt}\colorbox{highlightBlue}{\parbox{0.98\textwidth}{Der Energieerhaltungssatz ist also der Grund, warum Einflussfunktionen in der Regel rasch abklingen.

Die Ausnahme sind Einflussfunktionen in statisch bestimmten Tragwerken, weil kinematische Ketten null Energie haben und sie somit nicht gegen den Energieerhaltungssatz versto{\ss}en, wenn sie unter Umst\"{a}nden immer weiter anwachsen.}}\\


%---------------------------------------------------------------------------------
\begin{figure}
\centering
{\includegraphics[width=.9\textwidth]{\Fpath/U285}}
\caption{Einzelkraft an Geb\"{a}udeecke (Stockwerkrahmen) }
\label{U261}%
\end{figure}%
%---------------------------------------------------------------------------------\\

\begin{remark}
Je gr\"{o}{\ss}er der Abstand $R$ eines Betrachters von der Sonne ist, um so schw\"{a}cher scheint ihm das Licht, weil sich die abgestrahlte Energie $E$ \"{u}ber eine immer gr\"{o}{\ss}ere Sph\"{a}re $S$ verteilt
\begin{align}
E = \int_{S} q \,dS = q \cdot 4\,\pi\,R^2\,,
\end{align}
und die Energiedichte $q = E/S$ pro $m^2$ daher wie $1/R^2$ abnimmt.

Dieses Argument benutzt implizit auch der Ingenieur, der Last\"{a}nderungen in abliegenden Punkten einer Platte ignoriert, weil er wei{\ss}, dass das, was an Biegeenergie hinzukommt, mit zunehmenden Abstand vom Quellpunkt, wie das Licht der Sonne, abklingen muss.

Allerdings kann man die Regel nicht blindlings anwenden. Die Abmessungen und die  Lagerbedingungen spielen eine gro{\ss}e Rolle, wie etwa bei dem Stockwerkrahmen in Abb. \ref{U261}, bei dem die Fu{\ss}punkte zwar den gr\"{o}{\ss}ten Abstand vom Kraftangriffspunkt haben, aber die Fu{\ss}punktsmomente mit zu den gr\"{o}{\ss}ten Momenten geh\"{o}ren.

Der Stockwerkrahmen tr\"{a}gt zwar wie ein Schubtr\"{a}ger, aber er ist \"{a}hnlich empfindlich wie ein sehr langer Kragtr\"{a}ger, bei dem eine Zusatzlast $\Delta P$ am Kragarmende zu einem gro{\ss}en zus\"{a}tzlichen Ausschlag $\Delta w$ am Kragarmende f\"{u}hrt und so die Energiebilanz
\begin{align}
\Delta P \cdot \Delta w = \int_0^{\,l} \frac{\Delta M^2}{EI}\,dx
\end{align}
geradezu verlangt, dass sich das Einspannmoment merklich \"{a}ndert.

Anders gesagt, wenn die Zusatzbelastung gro{\ss}e Wege geht, ihre Eigenarbeit gro{\ss} ist, dann muss man genau hinschauen, w\"{a}hrend man in
allen anderen F\"{a}llen davon ausgehen kann, dass die Effekte \glq versickern\grq{}.
\end{remark}

%%%%%%%%%%%%%%%%%%%%%%%%%%%%%%%%%%%%%%%%%%%%%%%%%%%%%%%%%%%%%%%%%%%%%%%%%%%%%%%%%%%%%%%%%%%%%%%%%%%
{\textcolor{sectionTitleBlue}{\section{Kragtr\"{a}ger}}}\index{Kragtr\"{a}ger}
Wir wollen diese Beobachtungen zum Anlass nehmen, auf die besondere Rolle der Kragtr\"{a}ger hinzuweisen. Bei einem Durchlauftr\"{a}ger klingen Momente um so schneller ab, je mehr Felder er hat. Der Kragtr\"{a}ger ist das genaue Gegenteil. Wenn man einen Kragtr\"{a}ger nur lang genug macht, dann kann man das Einspannmoment beliebig gro{\ss} machen, ohne dass sich die Last am Kragarmende \"{a}ndert, weil die Einflussfunktion f\"{u}r das Einspannmoment eine Verdrehung des Tr\"{a}gers um $45^\circ$ ist.

Richtet man einen sehr starken Laserstrahl von der Erde auf den Mond, dann bewegen sich die Lichtpunkte auf dem Mond bei einer winzigen Drehung des Lasers mit einer Geschwindigkeit, die gr\"{o}{\ss}er ist als die Lichtgeschwindigkeit! (Was kein Widerspruch zur Relativit\"{a}tstheorie von Einstein ist).

Starrk\"{o}rperdrehungen sind also mit Vorsicht zu betrachten. Wenn diese m\"{o}glich sind, dann muss man mit allem rechnen... \\
%---------------------------------------------------------------------------------
\begin{figure}
\centering
{\includegraphics[width=1.0\textwidth]{\Fpath/U225}}
\caption{Hauptspannungen in einer geschlitzten Scheibe, rechts die Spannungen $\sigma_{yy}$}
\label{U225}%
\end{figure}%
%---------------------------------------------------------------------------------
%---------------------------------------------------------------------------------
\begin{figure}
\centering
\if \bild 2 \sidecaption[t] \fi
{\includegraphics[width=0.6\textwidth]{\Fpath/U226}}
\caption{Eine Versetzung im Rissgrund, muss unendlich gro{\ss}e Verschiebungen verursachen}
\label{U226}%
\end{figure}%
%---------------------------------------------------------------------------------

\pagebreak
%%%%%%%%%%%%%%%%%%%%%%%%%%%%%%%%%%%%%%%%%%%%%%%%%%%%%%%%%%%%%%%%%%%%%%%%%%%%%%%%%%%%%%%%%%%%%%%%%%%
{\textcolor{sectionTitleBlue}{\section{Unendlich gro{\ss}e Spannungen}}}
Singul\"{a}re Punkte \index{singul\"{a}re Punkte}, also Punkte, in denen die Spannungen unendlich gro{\ss} werden, liegen typischerweise auf dem Rand und dort in Eckpunkten oder Punkten, in denen sich die Lagerbedingungen \"{a}ndern, siehe Abb. \ref{U225}.
%---------------------------------------------------------------------------------
\begin{figure}
\centering
\includegraphics[width=0.9\textwidth]{\Fpath/U227}
\caption{\textbf{ a)} Dreiecksf\"{o}rmige Elemente, Erzeugung der Einflussfunktion f\"{u}r $2 \cdot \sigma_{yy}$ im Rissgrund, \textbf{ b)} die Kinematik, \textbf{ c)} am Au{\ss}enrand}
\label{U227}%
\end{figure}%
%---------------------------------------------------------------------------------

Wenn wir der Meinung sind, dass man mit Einflussfunktionen auch diese Spannungen -- vielleicht nicht direkt in der Ecke, aber in der N\"{a}he -- voraussagen kann, dann stehen wir vor einem Problem: Wie gelingt es einer Punktversetzung (= Einflussfunktion f\"{u}r die Spannung $\sigma_{yy}$ im Rissgrund)
den oberen und unteren Rand der Scheibe in Abb. \ref{U226} in die Richtungen $\pm \infty$ zu dr\"{u}cken?  Anders kann es ja nicht sein, wenn wir der \"{U}berzeugung sind, dass die Einflussfunktionen auch in der N\"{a}he solcher singul\"{a}rer Punkte noch anwendbar sind
\beq
\sigma(\vek x) = \int_{\Gamma} \textcolor{chapterTitleBlue}{\vek G(\vek y, \vek x)}\dotprod \vek  p(\vek y) ds_{\vek y} =  \infty\,.
\eeq
Wie funktioniert das? Wie kann eine Punktversetzung ein unendlich weit ausschwingendes Verschiebungsfeld erzeugen?
%---------------------------------------------------------------------------------
\begin{figure}
\centering
\includegraphics[width=0.9\textwidth]{\Fpath/U49}
\caption{In der einspringenden Ecke werden die Spannungen unendlich gro{\ss}}
\label{U49}%
\end{figure}%
%---------------------------------------------------------------------------------

Im Grunde haben wir das Ph\"{a}nomen schon bei der Einflussfunktion f\"{u}r die Querkraft kennengelernt. Stellen wir uns vor, wir benutzen dreiecksf\"{o}rmige Elemente, die Querdehnzahl $\nu$ sei null (der Einfachheit halber), und wir wollen die Einflussfunktion f\"{u}r die Summe $\sigma_{yy} + \sigma_{yy}$ der Spannungen in den beiden Elementen, die dem Rissgrund unmittelbar benachbart sind, berechnen. Weil wir die Summe berechnen, bleibt die Symmetrie des Problems erhalten.

Wir m\"{u}ssen also die Spannungen $\sigma_{yy} $ der Ansatzfunktionen als Knotenkr\"{a}fte aufbringen. Das ergibt die Abb. \ref{U227} a, wenn wir die Knotenkr\"{a}fte, die sich gegenseitig aufheben, weglassen. In dem r\"{u}ckw\"{a}rtigen Knoten, der ja auf der Symmetrielinie liegt, muss die vertikale Verschiebung null sein. Damit ist die Situation im Grunde dieselbe wie bei einem Kragtr\"{a}ger. Wenn die beiden Kr\"{a}fte die Risskanten auseinander treiben, dann drehen sie praktisch die freien Schenkel um diesen r\"{u}ckw\"{a}rtigen Knoten und wenn $h $ gegen null geht, m\"{u}ssen sich die Schenkel um 90$^\circ$ aufstellen d.h. die vertikalen Verschiebungen werden unendlich gro{\ss}.

Wie ist das nun, wenn wir dasselbe Man\"{o}ver an dem glatten Rand einer Scheibe fahren? Wir berechnen wieder die Einflussfunktion f\"{u}r $\sigma_{yy} + \sigma_{yy} $, aber nun ragt kein Teil der Scheibe \"{u}ber den Au{\ss}enrand hinaus, s. Abb. \ref{U227} c. Jetzt kann sich nichts verdrehen und daher bleiben die Verformungen (in den abliegenden Punkten) endlich.

Auch die singul\"{a}ren Spannungen bei der L-Scheibe in Abb. \ref{U49} r\"{u}hren daher, dass die beiden Knotenkr\"{a}fte, die die Einflussfunktion f\"{u}r die Summe der beiden schr\"{a}gen Spannungen $2 \cdot \sigma_{\xi\xi}$ erzeugen, im Grenzfall, $h \to 0 $, die Schenkel um $90^\circ $ verdrehen.

%%%%%%%%%%%%%%%%%%%%%%%%%%%%%%%%%%%%%%%%%%%%%%%%%%%%%%%%%%%%%%%%%%%%%%%%%%%%%%%%%%%%%%%%%%%%%%%%%%%
\textcolor{sectionTitleBlue}{\section{Symmetrie der Wirkungen}}\label{Symmetrie der Wirkungen}
Es gibt noch ein theoretisches Argument, das diese \"{U}berlegungen unterst\"{u}tzt. Gehen wir noch einmal zur\"{u}ck zu der gerissenen Scheibe. Im Grunde sind hier zwei Einflussfunktionen am Werk: zum einen die Einflussfunktion f\"{u}r die Spannung $\sigma_{yy} $ im Rissgrund und zum anderen die Einflussfunktion f\"{u}r die vertikalen Verschiebungen am oberen bzw. unteren Rand der Scheibe. Der Einfachheit halber nehmen wir an, dass wir je einen Punkt auf dem oberen und unteren Rand als Aufpunkte w\"{a}hlen, in denen wir die vertikalen Verschiebungen berechnen. Durch die Wahl von zwei Punkten, unten und oben, bleibt die Symmetrie erhalten.

Das sind also zwei Funktionale, die wir
\begin{align}
J_1(\vek u) = \sigma_{yy}(\vek u) \qquad J_2(\vek u) = u_y(oben \,\,Mitte) + u_y(unten\,\, Mitte)
\end{align}
nennen. Zu dem ersten Funktional geh\"{o}rt die Einflussfunktion $\vek G_1$ und zu dem zweiten Funktional die Einflussfunktion $\vek G_2$.

Nun kann man zeigen, dass die beiden Funktionale \glq \"{u}ber Kreuz\grq{} gleich sind, d.h.
\begin{align}
J_1(\vek G_2) = J_2(\vek G_1)\,,
\end{align}
was \"{u}brigens f\"{u}r alle Paare von Funktionalen und deren Einflussfunktionen gilt -- nicht nur hier.

Gleich bedeutet hier das folgende: $\vek G_2$ wird von zwei Einzelkr\"{a}ften $P = \pm 1$ generiert, die am oberen und unteren Rand der Scheibe ziehen. Die Wirkung dieser beiden Kr\"{a}fte f\"{u}hrt zu unendlich gro{\ss}en vertikalen Spannungen $\sigma_{yy} $ in der Rissfuge, also
\begin{align}
J_1(\vek G_2) = \infty \qquad J_1\,\,\text{misst $\sigma_{yy}$} \,\text{von $\vek G_2$}\,.
\end{align}
Umgekehrt f\"{u}hrt die Spreizung der Rissfuge, wie wir uns \"{u}berzeugt haben, zu unendlich gro{\ss}en Verschiebungen an der oberen und unteren Kante, also
\begin{align}
J_2(\vek G_1) = \infty \qquad J_2\,\,\text{misst $u_y(\text{oben/unten})$} \,\text{von $\vek G_1$}\,,
\end{align}
und die Theorie sagt, dass diese beiden Werte gleich gro{\ss} sind. Wenn also der eine Wert unendlich ist, dann muss es auch der andere sein.

%---------------------------------------------------------------------------------
\begin{figure}
\centering
\includegraphics[width=1.0\textwidth]{\Fpath/U228}
\caption{Das Eigengewicht der Kragscheibe erzeugt unendlich gro{\ss}e Spannungen in den Randfasern}
\label{U228}%
\end{figure}%
%---------------------------------------------------------------------------------

%---------------------------------------------------------------------------------
\begin{figure}
\centering
\includegraphics[width=1.0\textwidth]{\Fpath/U229}
\caption{Berechnung der Einflussfunktion f\"{u}r die Spannung $\sigma_{xx}$ im Eckpunkt,
  \textbf{ a)} Netz,  \textbf{ b)}\"{a}quivalente Knotenkr\"{a}fte, \textbf{ c)} vertikale Verschiebung der oberen rechten Ecke in Abh\"{a}ngigkeit von der Elementl\"{a}nge $h$}
\label{U229}%
\end{figure}%
%---------------------------------------------------------------------------------
%%%%%%%%%%%%%%%%%%%%%%%%%%%%%%%%%%%%%%%%%%%%%%%%%%%%%%%%%%%%%%%%%%%%%%%%%%%%%%%%%%%%%%%%%%%%%%%%%%%
{\textcolor{sectionTitleBlue}{\section{Kragscheibe}}}
Aber selbst in einer scheinbar harmlosen Standardsituation, wie der Scheibe in Abb. \ref{U228}, treten im LF $g$ unendlich gro{\ss}e Spannungen in den \"{a}u{\ss}ersten Fasern auf. Wir d\"{u}rfen annehmen, dass das auch passieren w\"{u}rde, wenn das Eigengewicht durch eine Einzelkraft $P$ ersetzt w\"{u}rde, die in irgendeinem inneren Punkt $\vek y_P$ der Scheibe angreift.
%---------------------------------------------------------------------------------
\begin{figure}
\centering
{\includegraphics[width=0.80\textwidth]{\Fpath/U46}}
\caption{Kragscheibe, \textbf{ a)} Hauptspannungen (\glq Stromlinien\grq{}) (BE-Scheibe), \textbf{ b)} Krafteck in verschiedenen Schnitten, \textbf{ c)} nahe dem linken Rand wird das Krafteck nahezu unendlich flach und unendlich lang, \textbf{ d)} Stra{\ss}enlaterne---dasselbe Prinzip }
\label{U46}%
\end{figure}%

Wenn dies richtig ist, dann muss die Einflussfunktion f\"{u}r die obere Randspannung $\sigma_{xx}$ den Wert $\infty $ in fast allen Punkten der Scheibe haben
\beq
\sigma_{xx} (\vek x) = \textcolor{chapterTitleBlue}{\vek G(\vek y_P, \vek x) }\dotprod  \vek P = \textcolor{chapterTitleBlue}{\vek \infty} \dotprod  \vek P\,.
\eeq
Wir rechnen das nach: In dem FE-Modell muss man zur Berechnung der Einflussfunktion die Spannungen $\sigma_{xx}$ der Ansatzfunktionen des Netz in dem Eckpunkt als Belastung aufbringen.
%---------------------------------------------------------------------------------
\begin{figure}
\centering
{\includegraphics[width=0.75\textwidth]{\Fpath/U47}}
\caption{Wenn man die Ecken ausrundet, dann k\"{o}nnen sich die \glq Stromlinien\grq{} (= Hauptspannungen) verdrehen und dann haben sie es leichter der vertikalen Belastung das Gleichgewicht zu halten}
\label{U47}%
\end{figure}%
%---------------------------------------------------------------------------------
%---------------------------------------------------------------------------------
\begin{figure}
\centering
\includegraphics[width=0.9\textwidth]{\Fpath/U48}
\caption{Spannungsverteilung ($\sigma_{xx}$) in der Einspannfuge, wenn die Ecken ausgerundet werden}
\label{U48}%
\end{figure}%
%---------------------------------------------------------------------------------

Weil nur die Ansatzfunktionen des Eckelementes selbst in der Ecke Spannungen $\sigma_{xx}\neq 0$ generieren, werden nur die Knoten des Eckelementes mit diesen Spannungen als Knotenkr\"{a}fte, $f_i = \sigma_{xx}(\vek \Np_i)$ belastet, und diese Kr\"{a}fte/Spannungen sind proportional zu $E/h$, wobei $E$ der Elastizit\"{a}tsmodul des Materials ist, $E = 2.1 \cdot 10^5$ N/mm$^2$, und $h$ ist die Elementl\"{a}nge.

Beim numerischen Test, siehe Abb. \ref{U229}, mit adaptiver Verfeinerung wuchs die Eckverschiebung in der Tat exponentiell mit $h \to 0$ an.

Um hinter das Geheimnis der singul\"{a}ren Spannungen zu kommen, ersetzen wir in Gedanken die Hauptspannungen durch paarweise orthogonale Pfeile (\glq Stromlinien\grq{}), siehe Abb. \ref{U46} a und  \ref{U46} b. Die Vektorsumme der Pfeile muss dann gleich der Resultierenden der aufgebrachten Belastung sein. Anders gesagt, die Versetzung, die die beiden Pfeile verursachen, muss den Angriffspunkt der Resultierenden um eine L\"{a}ngeneinheit anheben.

Damit ist auch klar, warum die Spannungen in den \"{a}u{\ss}ersten Fasern singul\"{a}r werden. Je n\"{a}her die Stromlinien dem linken Rand kommen, um so flacher m\"{u}ssen sie verlaufen, weil der Rand in vertikaler Richtung festgehalten wird, und das bedeutet, dass sich die Stromlinien weiter strecken m\"{u}ssen, damit ihre immer kleiner werdenden vertikalen Komponenten der Belastung das Gleichgewicht halten k\"{o}nnen.

Das ist wie bei einer Stra{\ss}enlaterne, die an einem Seil zwischen zwei H\"{a}usern h\"{a}ngt. Bevor man das Seil richtig straff ziehen kann, rei{\ss}t das Seil.

Wenn die Ecken ausgerundet werden, dann k\"{o}nnen die Stromlinien sich drehen, und dann haben sie es leichter, das Gleichgewicht mit der vertikalen Belastung zu halten, siehe Abb. \ref{U47} und \ref{U48}. Dann besteht kein Grund mehr, unendlich gro{\ss}e Spannungen zu generieren.\\

%---------------------------------------------------------------------------------
\begin{figure}
\centering
\includegraphics[width=0.8\textwidth]{\Fpath/U147}
\caption{Die Biegespannungen $\sigma_{xx}$ in der Einspannfuge bleiben in diesen Lastf\"{a}llen endlich}
\label{U147}%
\end{figure}%
%---------------------------------------------------------------------------------

\begin{remark}
 Numerische Tests belegen, dass horizontale Lasten, die mit einem Lastmoment einhergehen,  nicht zu singul\"{a}ren Spannungen in der Einspannfuge f\"{u}hren, s. Abb. \ref{U147}, und ebenso gilt das f\"{u}r vertikale Kr\"{a}ftepaare.
 \end{remark}


%%%%%%%%%%%%%%%%%%%%%%%%%%%%%%%%%%%%%%%%%%%%%%%%%%%%%%%%%%%%%%%%%%%%%%%%%%%%%%%%%%%%%%%%%%%%%%%%%%%
\textcolor{sectionTitleBlue}{\section{Standardsituationen}}
Es braucht aber nicht eine Kragscheibe, um Singularit\"{a}ten zu produzieren. Singularit\"{a}ten treten auch an so harmlos scheinenden Stellen wie den Ecken von \"{O}ffnungen auf, s. Abb. \ref{U269}. In der Praxis bemerkt man diese Singularit\"{a}ten in der Regel nicht, weil man nicht so stark verfeinert, gleichwohl wird man aber auch auf gr\"{o}beren Netzen schon erste Anzeichen daf\"{u}r entdecken. Die konstruktive Bewehrung ist aber in der Regel in der Lage, solche Effekte aufzufangen.
%---------------------------------------------------------------------------------
\begin{figure}
\centering
\includegraphics[width=0.8\textwidth]{\Fpath/U269}
\caption{Wandscheibe mit adaptiv verfeinertem Netz. Die Spannungen in den Ecken der \"{O}ffnungen werden konstruktiv, wie angedeutet, durch L\"{a}ngs- und Schr\"{a}gbewehrung aufgenommen. Nur die Punktlager sollte man besser durch kurze Linienlager ersetzen}
\label{U269}% % Pos. U28A
\end{figure}%
%---------------------------------------------------------------------------------

Wenn man aber wirklich in die Ecken hineingeht wie in Abb. \ref{U134}, dann sieht man, dass die Spannungen in der Tat unendlich gro{\ss} werden. Bei hochbelasteten Bauteilen im Maschinenbau, etwa Turbinenschaufeln, sind solche Spannungsspitzen durchaus bemessungsrelevant.

%---------------------------------------------------------------------------------
\begin{figure}
\centering
\includegraphics[width=0.8\textwidth]{\Fpath/U134}
\caption{Wandscheibe unter Eigengewicht \textbf{ a)} Hauptspannungen \textbf{ b)} Spannungen $\sigma_{yy} $ in einigen horizontalen Schnitten (Randelementl\"{o}sung)}
\label{U134}%
\end{figure}%
%---------------------------------------------------------------------------------

%%%%%%%%%%%%%%%%%%%%%%%%%%%%%%%%%%%%%%%%%%%%%%%%%%%%%%%%%%%%%%%%%%%%%%%%%%%%%%%%%%%%%%%%%%%%%%%%%%%
{\textcolor{sectionTitleBlue}{\section{Singularit\"{a}ten in Einflussfunktionen}}\label{SingInf}
Zu dem Thema  {\em pollution\/} geh\"{o}rt noch ein anderes Ph\"{a}nomen, n\"{a}mlich die Auswirkung von Singularit\"{a}ten auf die Ergebnisse. Bei Fl\"{a}chentragwerken treten in der Regel immer Singularit\"{a}ten auf, d.h. Lagerkr\"{a}fte oder Spannungen neigen dann zu den typischen  Oszillationen. Der Ingenieur tut das in der Regel mit der Bemerkung ab, {\em \glq das Material ist kl\"{u}ger\grq{}\/}, und achtet nicht weiter darauf, weil er aus Erfahrung wei{\ss}, dass au{\ss}erhalb der gest\"{o}rten Zone die Ergebnisse doch ganz vern\"{u}nftig aussehen und er sich nicht vorstellen kann, dass Singularit\"{a}ten in abliegenden Ecken die Genauigkeit negativ beeinflussen sollen.
%---------------------------------------------------------------------------------
\begin{figure}
\centering
\includegraphics[width=0.8\textwidth]{\Fpath/UE340}
\caption{Membran \"{u}ber L-f\"{o}rmigem Grundriss unter Winddruck}
\label{U171}%
\end{figure}%
%---------------------------------------------------------------------------------

Die Singularit\"{a}ten propagieren aber in die Einflussfunktionen hinein, sie verschlechtern die Qualit\"{a}t der Einflussfunktionen -- auch \glq im Feld\grq{}, denn auch die finiten Elemente sind in einem versteckten Sinn Randelemente.

Um dies zu verstehen, betrachten wir ein Segeltuch, also eine vorgespannte Membran, die \"{u}ber einen L-f\"{o}rmigen Grundriss gespannt ist und dem Winddruck standhalten muss, s. Abb. \ref{U171}.

Es ist klar, dass die Querkr\"{a}fte $v_x$ und $v_y$ in der Membran an der einspringenden Ecke unendlich gro{\ss} werden, weil die Querkr\"{a}fte proportional zu den Neigungen der Biegefl\"{a}che $w$ sind
\begin{align}
v_x = H\,w,_x \qquad v_y = H\,w,_y \qquad \text{$H$ = Vorspannung}\,,
\end{align}
und in der einspringenden Ecke sich die Membran an die W\"{a}nde anlegen wird, $w,_x = \infty$ und $w,_y = \infty$, dort also ein singul\"{a}rer Punkt liegt.

Die Querkr\"{a}fte sind am Rand die Aufh\"{a}ngekr\"{a}fte.
%----------------------------------------------------------
\begin{figure}[tbp]
\centering
\if \bild 2 \sidecaption \fi
\includegraphics[width=1.0\textwidth]{\Fpath/U263}
\caption{Gelenkig gelagerte Platte; Plot der Momente aus dem LF $P = 1$. Das sind also die Momente der Einflussfunktion $G_0(\vek y,\vek x)$ f\"{u}r die Durchbiegung $w(\vek x)$. In den einspringenden Ecken werden die Momente singul\"{a}r und das hat einen negativen Einfluss auf die G\"{u}te der FE-Einflussfunktion }
\label{U263}
\end{figure}%% % Position ZK
%----------------------------------------------------------

Wenn wir nun, wie gewohnt, die Durchbiegung (FE-L\"{o}sung) der Membran durch ihre Einflussfunktion darstellen
\begin{align}
w_h(\vek x) = \int_{\Omega} G_h(\vek y,\vek x)\,p(\vek y)\,d\Omega_{\vek y}\,,
\end{align}
dann deutet nichts darauf hin, dass der Kern $G_h(\vek y,\vek x)$ der Einflussfunktion von minderer Qualit\"{a}t sein soll. Wo versteckt sich die Singularit\"{a}t?

Wir wollen die Antwort nur skizzieren\footnote{F\"{u}r mehr Details siehe \cite{Ha2} und \cite{Ha6}}. Wer sich mit der {\em Potentialtheorie\/} oder der Methode der Randelemente auskennt, wei{\ss}, dass man jede L\"{o}sung  der Gleichung $- \Delta w = p$ wie folgt darstellen kann\footnote{{\em Jede\/} $C^2$-Funktion $w$ kann man also aus ihren Randwerten $w$ und $\partial w/\partial n$ und dem $- \Delta w$ im Feld generieren.}
\begin{align}
w(\vek x) = \int_{\Gamma} (g(\vek y,\vek x)\,\frac{\partial w}{\partial n}(\vek y) - \frac{\partial g(\vek y,\vek x)}{\partial n}\,w(\vek y))\,ds_{\vek y} + \int_{\Omega} g(\vek y,\vek x)\,p(\vek y) \,d\Omega_{\vek y}\,.
\end{align}
Die Funktion
\begin{align}
g(\vek y,\vek x) = - \frac{1}{2\,\pi}\,\ln |\vek y - \vek x|
\end{align}
hei{\ss}t {\em Fundamentall\"{o}sung\/}\index{Fundamentall\"{o}sung}, weil sie der Gleichung $-\Delta g(\vek y,\vek x) = \delta(\vek y-\vek x)$ gen\"{u}gt.

Man rekonstruiert also die Biegefl\"{a}che $w $ mit Hilfe von $g(\vek y,\vek x)$ aus ihren Randwerten $w$ und $\partial w/ \partial n$ und dem Winddruck $p$, der auf ihr lastet.


Diese Integraldarstellung kann man auch auf die Einflussfunktion $G(\vek y,\vek x)$ anwenden. Nun ist aber das $p$, das zu $G(\vek y,\vek x)$ geh\"{o}rt, ein Dirac Delta $\delta(\vek y-\vek x)$ und $G(\vek y,\vek x)$ ist null auf dem Rand $\Gamma$, so dass sich die Einflussfunktion auf
\begin{align}
G(\vek y,\vek x) &=  \int_{\Gamma}[ g(\vek \xi,\vek y)\,\frac{\partial G_h}{\partial n}(\vek \xi, \vek x) - \frac{\partial g(\vek \xi,\vek y)}{\partial n}\,G(\vek \xi, \vek x)]\,ds_{\vek \xi} \nn \\ &+ \underbrace{\int_{\Omega} g(\vek \xi,\vek y)\, \delta(\vek \xi - \vek x)\,d\Omega_{\vek \xi}}_{= \,{\displaystyle g}(\vek y,\vek x)}
=  \int_{\Gamma} g(\vek \xi,\vek y)\,\frac{\partial G}{\partial n}(\vek \xi, \vek x)\,ds_{\vek \xi} + g(\vek y,\vek x)
\end{align}
verk\"{u}rzt.

Diese Formel gilt sinngem\"{a}{\ss} auch f\"{u}r die FE-N\"{a}herung $G_h(\vek y,\vek x)$, die ja die L\"{o}sung des Randwertproblems
\begin{align}
- \Delta G_h(\vek y,\vek x) = \delta_h(\vek y,\vek x) \qquad G_h = 0 \qquad \text{auf $\Gamma$}
\end{align}
ist, wobei $\delta_h(\vek y,\vek x)$ allerdings ein Flickenteppich von Lasten ist, die versuchen einer Punktlast nahe zu kommen, sie zu simulieren. Also gilt f\"{u}r $G_h(\vek y,\vek x)$ die Darstellung
\begin{align}\label{Eq97}
G_h(\vek y,\vek x) &=  \int_{\Gamma} g(\vek \xi,\vek y)\,\underset{\uparrow}{\frac{\partial G_h}{\partial n}}(\vek \xi, \vek x)\,ds_{\vek \xi} + \int_{\Omega}g(\vek \xi,\vek y) \,\delta_h(\vek \xi,\vek x)\,d\Omega_{\vek \xi}\,.
\end{align}
Und jetzt sieht man die kritische Stelle. Spannungsspitzen an der einspringenden Ecke bedeuten, dass dort die Normalableitung des Segeltuchs $\partial G_h/ \partial n$,
also die Neigung des Segeltuchs zum Rand hin, unendlich gro{\ss} ist, weil sich das Segeltuch dort wahrscheinlich an die W\"{a}nde anlegt. (Wir reden jetzt \"{u}ber den Lastfall $\delta_h(\vek y,\vek x)$).

Solche singul\"{a}ren Verl\"{a}ufe kann man aber mit finiten Elementen sehr schlecht ann\"{a}hern, d.h. die Normalableitung $\partial G_h/\partial n$ der FE-L\"{o}sung wird in der  Ecke sehr ungenau sein. Und weil diese Ungenauigkeit nun gem\"{a}{\ss} (\ref{Eq97}) auf $G_h(\vek y,\vek x)$ durchschl\"{a}gt, ist auch $G_h(\vek y,\vek x)$ von minderer Qualit\"{a}t -- nicht nur in der Ecke, sondern \"{u}berall im Feld, wo immer der Aufpunkt $\vek x$ liegt.

Das ist der Grund, warum Singularit\"{a}ten das Ergebnis negativ beeinflussen. Sie machen es dem FE-Programm schwer, die Einflussfunktionen, von denen ja alles abh\"{a}ngt, gut anzun\"{a}hern.\\

\begin{remark}
Genau genommen m\"{u}sste man das Gebietsintegral in (\ref{Eq97}) noch um die Anteile aus den Linienkr\"{a}ften $l_h$ (= Spr\"{u}nge in der Normal\-ableitung der Biegefl\"{a}che, also den Knicken) auf den Elementkanten $\Gamma_i$ erweitern
\begin{align}
\int_{\Omega}g(\vek \xi,\vek y) \,\delta_h(\vek \xi,\vek x)\,d\Omega_{\vek \xi} + \sum_i \int_{\Gamma_i} g(\vek \xi,\vek y) \,l_h(\vek y)\,ds_{\vek y}\,.
\end{align}
Wir k\"{o}nnen das aber in Gedanken dem Gebietsintegral zuschlagen. Hier, an dieser Stelle, geht es nur um die Normalableitung auf dem Rand und deren Beitrag. {\em Der ist kritisch\/}.
\end{remark}

Bei einer Platte sind es die Momente und der Kirchhoffschub (Querkr\"{a}fte) auf dem Rand, von deren Qualit\"{a}t die FE-Einflussfunktionen im wesentlichen abh\"{a}ngen, s. Abb. \ref{U263}. Die Formel lautet hier, in der Notation stark vereinfacht,
\begin{align}\label{Eq144}
w(\vek x) = \int_{\Gamma} (g\,w''' + g' w'' + g'' w' + g''' w)\,ds_{\vek y} + \int_{\Omega} g\,p\,d\Omega_{\vek y}
\end{align}
wobei
\begin{align}
g(\vek y,\vek x) = \frac{1}{2\,\pi\,K}\,r^2\,\ln\,r \qquad K = \frac{E\,h^3}{12\,(1 - \nu^2)}
\end{align}
die Fundamentall\"{o}sung ist und $K$ die Plattensteifigkeit.

Symbolisch steht hier $w'$ f\"{u}r die Verdrehung (Normalableitung) am Rand, $w''$ f\"{u}r das Moment senkrecht zum Rand und $w'''$ f\"{u}r den Kirchhoffschub. Die exakte Einflussfunktion $G$ hat die Randwerte $G = 0$ und $G'' = 0$ (gelenkig gelagerter Rand), so dass sich die Formel f\"{u}r die Einflussfunktion einer solchen Platte auf
\begin{align}
G(\vek y,\vek x) = \int_{\Gamma} (g\,G''' + g'' \,G')\,ds_{\vek y} + g(\vek y,\vek x)
\end{align}
verk\"{u}rzt.

Die FE-Einflussfunktion erf\"{u}llt die Momentenbedingung $G'' = 0$ auf dem gelenkig gelagerten Rand aber nur n\"{a}herungsweise, so dass man $G_h''$ mit ber\"{u}cksichtigen muss
\begin{align}
G_h(\vek y,\vek x) = \int_{\Gamma} (g\,G_h''' + g' G_h'' + g'' G_h')\,ds_{\vek y} + \int_{\Omega} g \,\delta_h\,d\Omega_{\vek y}\,,
\end{align}
woran man abliest, dass die Qualit\"{a}t der FE-Einflussfunktion von $G_h'$, (der Neigung am Rand), und den Randmomenten $G_h''$ und den Randkr\"{a}ften $G_h'''$ der Einflussfunktion abh\"{a}ngt. Wenn nun in den Ecken die Momente
\begin{align}
G_h'' \equiv m_{xx}\, n_x^2 + 2\,m_{xy}\,n_x\,n_y + m_{yy}\,n_y^2
\end{align}
oder die Randkr\"{a}fte $G_h'''$ singul\"{a}r werden, dann hat das offensichtlich einen negativen Einfluss auf die Qualit\"{a}t der Einflussfunktion.

Sinngem\"{a}{\ss} gilt all dies auch f\"{u}r die Einflussfunktionen der Schnittkr\"{a}fte einer Platte, also
\begin{align}
m_{xx}(\vek x) &= \int_{\Gamma} (G_2\,w''' + \ldots  \qquad v_x(\vek x) = \int_{\Gamma} (G_3\,w''' + \ldots
\end{align}
und diese reagieren eher noch empfindlicher auf Singularit\"{a}ten, weil sie ja zweite bzw. dritte Ableitungen berechnen. Man sieht das sehr sch\"{o}n, wenn man (\ref{Eq144}) f\"{u}r alle vier Gr\"{o}{\ss}en $w, w', w'', w'''$ anschreibt und nur die charakteristischen Singularit\"{a}ten der Kerne zitiert
\begin{align}\label{Eq150}
\left[\barr{l} w \\ w'\\ w''\\ w'''\earr\right] = \int_{\Gamma}\left[\barr{r @{\hspace{2mm}}r @{\hspace{2mm}}r @{\hspace{2mm}}r} \boxed{r^{-1}} & \ln r & r \ln r & r^2\ln r \\ r^{-2} &\boxed{r^{-1}} & \ln r & r\ln r\\ r^{-3} & r^{-2} & \boxed{r^{-1}} & \ln\, r\\ r^{-4} & r^{-3} & r^{-2} &\boxed{r^{-1}}\earr\right]
\,\left[\barr{c} w \\w' \\ w'' \\ w''' \earr \right] \,ds_{\vek y} + \int_{\Omega} \left[\barr{l} G_0 \\ G_1\\ G_2\\ G_3\earr\right]\,p\,d\Omega_{\vek y}\,.
\end{align}
In Spalte 1 steht der Kirchhoffschub der $G_i$, in Spalte 2 stehen die Momente, dann die Normalableitungen und schlie{\ss}lich die $G_i$ selbst. Solange der Aufpunkt $\vek x$ nicht auf dem Rand liegt, $r > 0$, sind die Integrale berechenbar.

Das $r^{-1} = \varepsilon^{-1}$ macht, dass bei der Herleitung der obigen Einflussfunktionen
  \begin{align}
  \text{\normalfont\calligra B\,\,}(G_i,w) = \lim_{\varepsilon \to 0} \text{\normalfont\calligra B\,\,}(G_i,w)_{\Omega_\varepsilon} = w^{(i)}(\vek x) - \int_{\Gamma} \ldots - \int_{\Omega} \ldots = 0
    \end{align}
 in der Grenze, $\varepsilon \to 0$, aus dem Integral \"{u}ber den Umkreis $\Gamma_{N_{\varepsilon}}$ des Aufpunktes $\vek x$ das $w, w', w'', w'''$ herausspringt
\begin{align}
w^{(i)}(\vek x) = \lim_{\varepsilon \to 0} \int_0^{2\pi} \frac{1}{\varepsilon}\,w^{(i)}(\vek y)\,\varepsilon \,d\Np \qquad \vek y = \vek x + \varepsilon \,\left[\barr{l} \cos\,\Np \\ \sin\,\Np\earr\right]\,.
\end{align}
So sind die Punktwerte auf der linken Seite entstanden. Wie man sieht, muss in dem Kern noch ein Faktor $(2\,\pi)^{-1}$ vorkommen.

So weit die Theorie. In der Praxis d\"{u}rften jedoch die Auswirkungen von Singularit\"{a}ten in der Regel nicht so dramatisch sein, wie man das nach diesen Ausf\"{u}hrungen vielleicht vermuten k\"{o}nnte, denn im Bauwesen sind die Toleranzen doch relativ gro{\ss} und der erfahrene Ingenieur hat zudem ein gut entwickeltes Gesp\"{u}r daf\"{u}r, was glaubhaft ist und was nicht. \\

\hspace*{-12pt}\colorbox{highlightBlue}{\parbox{0.98\textwidth}{Finite Elemente im Bauwesen sind ja immer beides: Modellierung und \glq Rechenschieber\grq{} und der Ingenieur ist daher sehr flexibel -- um nicht zu sagen: nachsichtig -- bei der Interpretation von FE-Ergebnissen.}}\\

Vielleicht passt an diese Stelle auch ein Wort \"{u}ber die unterschiedliche Rolle der finiten Elemente in der Mathematik und in der Praxis. Der Mathematiker versteht unter finiten Elementen die {\em shape functions\/} -- Ansatzfunktionen mit endlicher, finiter Ausdehnung, w\"{a}hrend f\"{u}r den Ingenieur finite Elemente kleine Balken, Scheiben und Platten sind, mit denen er ein Tragwerk nachbildet und daher interessiert den Ingenieur  nicht nur der Approx\-imationsfehler, sondern auch der Modellfehler.

Beide Fehler sind miteinander verschr\"{a}nkt. Anders als beim Hausbau, muss man bei finiten Elementen die Fundamente nachbessern, wenn man schon am Dachstuhl ist, das Modell bleibt st\"{a}ndig in der Schwebe. Die Analyse des Modellfehlers muss daher {\em gleichgewichtig\/} neben der Analyse des Approx\-imationsfehlers stehen und hier ist vor allem der Sachverstand des Ingenieurs gefragt.

Lange Zeit konzentrierte man sich eigentlich nur auf den numerischen Fehler, ging das Problem rein mathematisch an, entwickelte ausgekl\"{u}gelte asymptotische Fehlersch\"{a}tzer, das sind Ausdr\"{u}cke vom Typ $O(h^n)$, aber nun versucht man auch Absch\"{a}tzungen f\"{u}r den Modellfehler zu entwickeln, wie wir das im vorhergehenden Kapitel (Steifigkeits\"{a}nderungen) getan haben. Daf\"{u}r eignen sich Einflussfunktionen sehr gut, weil sie ja direkt die Sensitivit\"{a}ten eines Tragwerks repr\"{a}sentieren und damit n\"{a}her an der Kernproblematik der numerischen Modellbildung sind.

Die Stichworte an dieser Stelle lauten {\em Verification and Validation\/}\index{Verification and Validation}. Wurde die Gleichung richtig gel\"{o}st -- {\em Verification\/} -- und ist das Modell \"{u}berhaupt in der Lage die gew\"{u}nschte Antwort zu liefern -- {\em Validation\/}? Diese Frage zu beantworten, ist Aufgabe des Ingenieurs.




%%%%%%%%%%%%%%%%%%%%%%%%%%%%%%%%%%%%%%%%%%%%%%%%%%%%%%%%%%%%%%%%%%%%%%%%%%%%%%%%%%%%%%%%%%%%%%%%%%%
\textcolor{chapterTitleBlue}{\chapter{Erg\"{a}nzungen}}
%%%%%%%%%%%%%%%%%%%%%%%%%%%%%%%%%%%%%%%%%%%%%%%%%%%%%%%%%%%%%%%%%%%%%%%%%%%%%%%%%%%%%%%%%%%%%%%%%%%

%%%%%%%%%%%%%%%%%%%%%%%%%%%%%%%%%%%%%%%%%%%%%%%%%%%%%%%%%%%%%%%%%%%%%%%%%%%%%%%%%%%%%%%%%%%%%%%%%%%
{\textcolor{sectionTitleBlue}{\section{Grundlagen}}}
Bevor wir den Katalog der Differentialgleichungen erweitern und Erg\"{a}nzungen anf\"{u}gen, wollen wir kurz den Zugang zu den Gleichungen der Statik in diesem Buch schildern.

{\textcolor{sectionTitleBlue}{\subsubsection*{Euler Gleichung}}}\label{Korrektur41}
Der Ausgangspunkt in diesem Buch sind die {\em Euler Gleichungen\/}, wie die Gleichung $- EA u'' = p$, die das Gleichgewicht an einem infinitesimalen Stab\-element  $dx$, s. Abb. \ref{UE357}, formuliert
\begin{align}
- N + N + dN + p\,dx = 0 \qquad (N = EA u')\,.
\end{align}
Wir \"{u}berlagern die linke Seite von $- EA\,u'' = p$ mit einer Testfunktion $\delta u$
\begin{align}\label{Eq173}
\int_0^{\,l} - EA\,u''\,\delta u\,dx \,,
\end{align}
formen das Integral mir partieller Integration um und kommen so zur ersten Greenschen Identit\"{a}t
\begin{align}
\text{\normalfont\calligra G\,\,}(u,{\delta u}) = \underbrace{\int_0^{\,l} - EA\,u''(x)\,{\delta u(x)}\,dx + [N\,{\delta u}]_{@0}^{@l}}_{\text{virt. \"{a}u{\ss}ere Arbeit}} - \underbrace{\int_0^{\,l} \frac{N\,{\delta N}}{EA}\,dx}_{\text{virt. innere Energie}} = 0\,,
\end{align}
und damit zu den Variationsprinzipien der Statik.
%-----------------------------------------------------------------
\begin{figure}[tbp]
\centering
\includegraphics[width=0.4\textwidth]{\Fpath/UE357}
\caption{Stabelement $dx$}
\label{UE357}
\end{figure}%
%-----------------------------------------------------------------

{\textcolor{sectionTitleBlue}{\subsubsection*{Prinzip der virtuellen Verr\"{u}ckungen}}}
Bei der umgekehrten Formulierung startet man mit dem {\em Prinzip der virtuellen Verr\"{u}ckungen\/}. Das bedeutet: Man definiert, was $\delta A_a$ sein soll, wie $\delta A_i$ aussehen soll und verlangt, dass diese beiden Ausdr\"{u}cke f\"{u}r alle $\delta u$ gleich sein sollen\footnote{Hier bei der Modellbildung ist ein Argument wie $\delta u =$ {\em klein\/} legitim, weil der Ingenieur damit ja die Kinematik festlegt, die er dem Modell zu Grunde legt}
\begin{align}\label{Eq141}
\delta A_a = \int_0^{\,l} p\,\delta u\,dx = \int_0^{\,l} \frac{N\, \delta N}{EA}\,dx = \delta A_i\,.
\end{align}
Was diesen Zugang so beliebt macht, ist, dass man so die Statik scheinbar aus einem \glq Naturgesetz\grq{} entwickelt: {\em Wenn ein Tragwerk im Gleichgewicht ist, dann ist bei jeder virtuellen Verr\"{u}ckung $\delta A_a = \delta A_i$\/}\footnote{Die Euler Gleichungen beruhen, wenn man so will, auf $\delta A_a = 0$, weil am infinitesimalen Element mit Starrk\"{o}rperbewegungen getestet wird, was auf $\sum H = 0$ oder $\sum V = 0$ f\"{u}hrt}.

Aber der Zugang ist nicht ungef\"{a}hrlich, weil man ja $\delta A_i$ partiell integrieren kann und da muss dann  $\delta A_a$ herauskommen. In den Lehrb\"{u}chern werden, wenn, so vor allem die Standardgleichungen der Stabtheorie bzw. der Elastizit\"{a}tstheorie eingef\"{u}hrt und da hat man mit der ersten Greenschen Identit\"{a}t eine Vorlage, wei{\ss} wie man $\delta A_a$ und $\delta A_i$ zu definieren hat, damit am Schluss alles zusammenpasst. Die eigentliche Herausforderung entsteht, wenn sich mehrere Bewegungen \glq \"{u}berlagern\grq{}, wie etwa bei der Verformung (3-D) von offenen Profilen im Stahlbau, wo vor allem die Kinematik gemeistert werden muss, um zu einem stimmigen $\delta A_a = \delta A_i$ zu kommen.

Was den Zugang $\delta A_a = \delta A_i $ in den Augen des Ingenieurs weiter favorisiert ist die Tatsache, dass das $\delta A_a = \delta A_i $ auch die Grundlage der finiten Elemente ist und man so direkt mit der Diskretisierung anfangen kann, ohne jemals eine Differentialgleichung angeschrieben zu haben. Aber die Differentialgleichung ist trotzdem \glq da\grq{}, sie zieht die F\"{a}den. {\em Mit dem Begriff der schwachen L\"{o}sung ist man die Differentialgleichung nicht los!\/}

Sie kommt durch die Hintert\"{u}r herein, denn wenn man (\ref{Eq141}) partiell integriert (Annahme: $\delta u(0) = \delta u(l) = 0$), dann zeigt sich, dass (\ref{Eq141})  \"{a}quivalent ist mit
\begin{align}
\int_0^{\,l} \underbrace{(- EA\,u'' - p)}_{Euler\,\, Glg.}\,\delta u\,dx = 0 \qquad \text{f\"{u}r alle $\delta u$}\,,
\end{align}
was besagt:\\

\hspace*{-12pt}\colorbox{highlightBlue}{\parbox{0.98\textwidth}{Die Forderung $\delta A_a = \delta A_i $ ist {\em gleichbedeutend\/} damit, dass $u$ die L\"{o}sung einer Differentialgleichung ist, der Euler Gleichung.}}\\

Das ganze mathematische Ger\"{u}st, das zu dem statischen Problem geh\"{o}rt, steckt in der Differentialgleichung, in der Euler-Gleichung, sie bildet, zusammen mit der zugeh\"{o}rigen Greenschen Identit\"{a}t, den eigentlichen Kern und alles, was die finiten Elemente machen, geht konform mit der ersten Greenschen Identit\"{a}t. Ja man kann mit Fug und Recht behaupten:\\

\hspace*{-12pt}\colorbox{highlightBlue}{\parbox{0.98\textwidth}{Ein FE-Programm ist die erste Greensche Identit\"{a}t in {\em bits\/} und {\em bytes\/}}}\\

{\textcolor{sectionTitleBlue}{\subsubsection*{Zus\"{a}tzliche Gleichungen}}}
In den ersten Kapiteln dieses Buches haben wir uns zun\"{a}chst auf die wichtigsten Differentialgleichungen der Statik konzentriert und in diesem Kapitel wollen wir diese Liste erweitern\footnote{Eine nahezu vollst\"{a}ndige, und sehr pr\"{a}zise Darstellung der Differentialgleichungen der Stabstatik und der zugeh\"{o}rigen Energieprinzipe findet der Leser in \cite{Ramm}}.

Solange eine Differentialgleichung linear und selbstadjungiert ist, ist die Algebra dieselbe wie zuvor unabh\"{a}ngig von der Form der einzelnen Gleichungen. Insbesondere ist die Wechselwirkungsenergie symmetrisch, $a(u, v) = a(v, u)$.

Bei nichtlinearen Problemen geht die Symmetrie verloren, $a(u,v) \neq a(v,u)$. Wenn wir die Elastizit\"{a}tstheorie als Beispiel nehmen, dann ist die Wechselwirkungsenergie nicht mehr das Skalarprodukt zwischen dem Spannungs- und Verzerrungstensor der beiden Verschiebungsfelder $\vek u$ und $\vek \delta \vek u$
\begin{align}
a(\vek u, \vek \delta\, \vek u) = \int_{\Omega} \vek S(\vek u) \dotprod  \vek E(\vek \delta \vek u) \,d\Omega \qquad \text{Lineare Theorie}\,,
\end{align}
sondern
\begin{align}
a(\vek u, \vek \delta \vek u) = \int_{\Omega} \vek E_{\vek u}(\vek \delta \vek u) \dotprod \vek S\,d\Omega\qquad \text{Nichtlineare Theorie}
\end{align}
ist das Skalarprodukt der Gateaux Ableitung des Verzerrungstensors mit dem Spannungstensor und dieser Ausdruck ist nicht symmetrisch.

Die Konsequenz ist, dass die einfache Algebra
\begin{align}
\text{\normalfont\calligra B\,\,}(u,\hat{u}) &= \text{\normalfont\calligra G\,\,}(u, \hat{u}) - \text{\normalfont\calligra G\,\,}(\hat{u}, u)= 0\,,
\end{align}
auf der der Satz von Betti beruht, nicht zur Verf\"{u}gung steht. Bei nichtlinearen Problemen gibt es keinen \glq Betti\grq{}.

\pagebreak
%%%%%%%%%%%%%%%%%%%%%%%%%%%%%%%%%%%%%%%%%%%%%%%%%%%%%%%%%%%%%%%%%%%%%%%%%%%%%%%%%%%%%%%%%%%%%%%%%%%
\textcolor{sectionTitleBlue}{\section{Notation}}
Bei den folgenden Integrals\"{a}tzen benutzen wir die partielle Integration\index{partielle Integration}, die in zwei und drei Dimensionen die Gestalt
\begin{align}\label{Eq19}
\int_{\Omega} u,_{x_i}\,\delta u \,d\Omega = \int_{\Gamma} u \, n_i\, \delta u\,ds - \int_{\Omega} u\,\delta u,_{x_i} \,d\Omega
\end{align}
hat, wobei $n_i$ die $i$-te Komponente des Normalenvektors $\vek n$ mit $|\vek n| = 1$ auf dem Rand $\Gamma$ ist.

Der Gradient\index{Gradient} einer skalarwertigen Funktion $u$ ist ein Vektor und der Gradient einer vektorwertigen Funktion
$\vek u = \{u_1, u_2\}^T$ ist eine Matrix\index{$\nabla$},
\bfo
\nabla u = \left [ \barr {c} u,_1 \\ u,_2 \earr \right] \qquad \nabla \vek u = \left [
\barr {c @{\hspace{2mm}} c} u_1,_1 & u_1,_2 \\ u_2,_1 & u_2,_2 \earr \right] \qquad u_i,_j :=
\frac{\partial u_i}{\partial x_j}\,.
\efo
Das formale Gegenst\"{u}ck hierzu ist der Operator div\index{$\mbox{div}$}, denn die Divergenz einer
matrixwertigen Funktion ist eine vektorwertige Funktion und die Divergenz einer vektorwertigen Funktion
$\vek q = \{q_1,q_2\}^T$ ist eine skalarwertige Funktion
\bfo
\mbox{div \vek S} = \left[\barr{c} \sigma_{11},_1 + \sigma_{12},_2 \\ \sigma_{21},_1 +
\sigma_{22},_2\earr \right] \qquad \mbox{div} \,\vek q = q_1,_1 + q_2,_2\,.
\efo
Die folgende Identit\"{a}t verkn\"{u}pft mittels (\ref{Eq19}) diese beiden Operatoren
\bfo
\int_{\Omega} \mbox{div} \,\vek S \dotprod \vek \delta \vek u \,d\Omega = \int_{\Gamma} \vek
S\,\vek n \dotprod \vek \delta \vek u \,ds - \int_{\Omega} \vek S \dotprod  \nabla \,\vek \delta \vek u \,d\Omega\,.
\efo
Wenn $\vek S $ symmetrisch ist, $\vek S = \vek S^T$, dann gilt
\bfo
\int_{\Omega} - \mbox{div} \,\vek S \dotprod \vek \delta \vek u \,d\Omega +\!\! \int_{\Gamma} \vek
S\,\vek
n \dotprod \vek \delta \vek u \,ds &=& \!\!\int_{\Omega} \vek S \dotprod  \nabla \,\vek \delta \vek u \,d\Omega\nn \\
&=&\!\!\int_{\Omega} \vek S \dotprod \frac{1}{2}(\nabla \vek \delta \vek u + \nabla  \vek \delta \vek u^T) \,d\Omega
\efo
was das {\em Prinzip der virtuellen Verr\"{u}ckungen\/} f\"{u}r eine Scheibe ist, $\delta A_a = \delta A_i$, wenn man $\vek \delta \vek u = \{\delta u_x, \delta u_y\}^T$ als virtuelle Verr\"{u}ckung interpretiert.

Vektorwertige Funktionen $\vek u= \{u_x, u_y\}^T$ gen\"{u}gen der gleichen Regel
\bfo
\int_{\Omega} \mbox{div}\,\vek  u \, \delta u \,d\Omega = \int_{\Gamma} (\vek u \dotprod
\vek n)\, \delta u\,ds - \int_{\Omega} \vek u\,\dotprod \nabla \delta u \,d\Omega\,,
\efo
und bei eindimensionalen Problemen sind div = $( )'$ und $\nabla = ( )'$ dasselbe
\bfo
\int_0^{\,l} u'\,\delta u\,dx = [u\,\delta u]_{@0}^{@l} - \int_0^{\,l} u\,\delta u'\,dx \,.
\efo
Vektoren sind Spaltenvektoren und ein Punkt kennzeichnet das Skalarprodukt zwischen zwei Vektoren
\bfo
 \vek f \dotprod \vek  u = f_x \,u_x + f_y\, u_y \,.
\efo
Gelegentlich benutzen wir auch die Notation $ \vek f \dotprod \vek  u  = \vek f^T \vek  u$. Der Punkt
bezeichnet ebenso das Skalarprodukt zwischen zwei Matrizen, wie etwa dem Verzerrungs- und Spannungstensor
\index{Skalarprodukt von Matrizen}
\bfo
A_i &=& \frac{1}{2} \int_\Omega \vek E \dotprod \vek S \, d\Omega\nn \\
 &=&\frac{1}{2} \int_{\Omega}\underbrace{[ \varepsilon_{xx} \, \sigma_{xx} + \varepsilon_{xy} \, \sigma_{xy} +
\varepsilon_{yx} \, \sigma_{yx} + \varepsilon_{yy} \, \sigma_{yy}]}_{\mbox{{\em Skalarprodukt\/}}} \,d\Omega \,.
\efo
In der Literatur werden auch die Bezeichnungen
\bfo
 \vek E \dotprod \vek S = \mbox{tr}\,(\vek E \otimes \vek S) = \vek E : \vek S \qquad \mbox{(tr = trace)}
\efo
benutzt, wobei $\vek E \otimes \vek S$ das {\em direkte Produkt\/} der beiden Tensoren $\vek E$
und $\vek S$ ist. Das direkte Produkt \index{direktes Produkt}\index{$\otimes$} zweier Vektoren ist eine Matrix
\bfo
\vek f \otimes \vek u = \left[ \barr{c} f_x \\ f_y \earr \right] \otimes \left[ \barr{c}
u_x \\ u_y \earr \right] = \left[ \barr{c @{\hspace{2mm}} c} f_x \cdot u_x & f_x \cdot u_y \\ f_y \cdot
u_x & f_y \cdot u_y \earr \right] = \vek f\,\vek u^T = \vek A
\efo
mit $a_{ij} = f_i \cdot u_j$. Solche Matrizen haben immer den Rang 1.

Die Multiplikationstabelle $\vek T_{10 \times 10} = \vek z\,\vek z^T$ der Zahlen von 1 bis 10 ist das direkte Produkt des Vektors $\vek z = \{1, 2, \ldots, 10\}^T$ mit sich selbst.

%%%%%%%%%%%%%%%%%%%%%%%%%%%%%%%%%%%%%%%%%%%%%%%%%%%%%%%%%%%%%%%%%%%%%%%%%%%%%%%%%%%%%%%%%%%%%%%%%%%
\textcolor{sectionTitleBlue}{\section{FE-Notation}}
Ein Spaltenvektor $\vek a$ mal einem Zeilenvektor $\vek b^T$ ergibt eine Matrix
\begin{align}
\vek M = \vek a\,\vek b^T = \left [ \barr{c @{\hspace{4mm}} c} a_1 \cdot b_1 & a_1 \cdot b_2 \\
a_2 \cdot b_1 & a_2 \cdot b_2\earr \right]\,.
\end{align}
In der FE-Literatur wird daher die Steifigkeitsmatrix eines Stabes mit zwei linearen Ansatzfunktionen $\Np_1(x)$ und $\Np_2(x)$ meist wie folgt geschrieben
\begin{align}
\vek K^e_{(2 \times 2)} = \int_0^{\,l_e} \left [ \barr{l} \Np_1' \\ \Np_2' \earr \right] \,\underbrace{\left [ \barr{c} EA  \earr \right]}_{\vek E}\,\left [ \barr{c c} \Np_1' & \Np_2' \earr \right]\vphantom{]}\,dx= \int_0^{\,l_e} \vek B^T_{(2 \times 1)}\,\vek E_{(1 \times 1)}\, \vek B_{(1 \times 2)}\,dx
\end{align}
was elementweise
\begin{align}
k_{ij}^e = a(\Np_i,\Np_j) = \int_0^{\,l_e} \Np_i'\,EA\,\Np_j'\,dx
\end{align}
entspricht.

Bei einem Balkenelement mit seinen vier Einheitsverformungen $\Np_i(x)$ ist
\begin{align}
\vek B = \{-\frac{6}{l_e^2} + \frac{12\,x}{l_e^3},\,\, \,\,- \frac{4}{l_e} + \frac{6\,x}{l_e^2}, \,\,\,\,\frac{6}{l_e}^2 - \frac{12\,x}{l_e^3}, \,\,\,\,- \frac{2}{l_e} + \frac{6\,x}{l_e^2}\}\,,
\end{align}
der Vektor der zweiten Ableitungen der $\Np_i(x)$ und die Steifigkeitsmatrix kann man schreiben als
\begin{align}
\vek K^e_{4 \times 4} = \int_0^{\,l_e} \vek B^T\,EI\,\vek B\,dx\,.
\end{align}
Bei Scheibenelementen mit $n$ Verschiebungsfeldern, z.B. $n = 3 \cdot 2 = 6$ bei einem {\em CST-Element\/} mit drei Knoten, die sich in $x$- und $y$-Richtung verschieben k\"{o}nnen, stehen in
\begin{align}
\vek B = \{ \vek \varepsilon(\vek \Np_1), \vek \varepsilon(\vek \Np_2), \ldots, \vek \varepsilon(\vek \Np_n) \}
\end{align}
die Verzerrungen der $n$ Felder als Vektoren
\begin{align}
\vek \varepsilon(\vek \Np_i) = \{\varepsilon_{xx}^{(i)}, \varepsilon_{yy}^{(i)}, \varepsilon_{xy}^{(i)}\}^T
\end{align}
und in\footnote{$\vek E$ = Ebener Spannungszustand mit $2 (1-\nu)$ wegen $2\,\sigma_{xy}$}
\begin{align}
\vek E \vek B =  \{ \vek \sigma(\vek \Np_1), \vek \sigma(\vek \Np_2), \ldots, \vek \sigma(\vek \Np_n) \} \qquad\vek E = \frac{E}{1 - \nu^2}\left [ \barr{c  @{\hspace{2mm}} c  @{\hspace{2mm}} c} 1 & \nu & 0 \\ \nu & 1 & 0 \\ 0 & 0 & 2(1-\nu) \earr \right]
\end{align}
die Spannungen
\begin{align}
\vek \sigma(\vek \Np_i) = \{\sigma_{xx}^{(i)}, \sigma_{yy}^{(i)}, 2 \sigma_{xy}^{(i)}\}^T\,,
\end{align}
so dass wir unseren gewohnten Ausdruck wiederfinden
\begin{align}
k_{ij }^e &= a(\vek \Np_i,\vek \Np_j) = \int_{\Omega_e} \vek \sigma^{(i)} \dotprod \vek \varepsilon^{(j)} \,d\Omega \nn \\
&= \int_{\Omega_e} (\sigma_{xx}^{(i)}\,\varepsilon_{xx}^{(j)} + \sigma_{yy}^{(i)}\,\varepsilon_{yy}^{(j)}+ 2 \cdot \sigma_{xy}^{(i)}\,\varepsilon_{xy}^{(j)} ) \,d\Omega\,.
\end{align}
\pagebreak
%%%%%%%%%%%%%%%%%%%%%%%%%%%%%%%%%%%%%%%%%%%%%%%%%%%%%%%%%%%%%%%%%%%%%%%%%%%%%%%%%%%%%%%%%%%%%%%%%%%
\textcolor{sectionTitleBlue}{\section{Die Algebra der Identit\"{a}ten}}
Mit Matrizen l\"{a}sst sich das Thema am einfachsten behandeln, wir entschuldigen uns aber zugleich, wenn das Thema trotzdem noch zu mathematisch geraten ist.

Die zweite Greensche Identit\"{a}t  ist nicht auf quadratische Matrizen $\vek A$ beschr\"{a}nkt
\begin{align}
\text{\normalfont\calligra B\,\,}(\vek u,\vek v)  = \vek v_m^T\,\vek A_{m \times n}\,\vek u_{n} - \vek u^T_n\,\vek A^T_{n \times m}\vek v_m = 0\,.
\end{align}
Bezeichne $C(\vek A) \subset \mathbb{R}^m$ den Spaltenraum \index{Spaltenraum}, $R(\vek A) = C(\vek A^T) \subset \mathbb{R}^n$ den Zeilenraum \index{Zeilenraum} und $N(\vek A) \subset \mathbb{R}^n$ bzw. $N(\vek A^T) \subset \mathbb{R}^m$ den Nullraum \index{Nullraum} von $\vek A $ bzw. $\vek A^T$, dann folgt aus der obigen Identit\"{a}t, dass die Vektoren $\vek u_0 \in N(\vek A)$ orthogonal zu den Vektoren in $C(\vek A^T)$ sind\footnote{Jede der $m$ Spalten $\vek c_i$ von $\vek A^T$ ist orthogonal zu $\vek u_0$, also $\vek u_0^T \vek c_i = 0$.}
\begin{align}
\text{\normalfont\calligra B\,\,}(\vek u_0,\vek v)  = \vek v_m^T\,\vek 0 - \underbrace{\vek u_0^T \,\vek A^T}_{[0,0,\ldots 0]}\,\vek v = 0\,,
\end{align}
und analog die Vektoren $\vek v_0 \in N(\vek A^T)$ orthogonal zu den Vektoren in $C(\vek A)$.

%$R(\vek A)$ und $C(\vek A)$ haben die gleiche Dimension $r$ und $N(\vek A)$ hat die Dimension $n - r$ und $N(\vek A^T)$ hat die Dimension $m - r$. Die Dimension eines Raums\index{Dimension eines Raums} ist die Zahl der linear unabh\"{a}ngigen Vektoren, die den Raum aufspannen. Es gilt
%\begin{subequations}
%\begin{alignat}{3}
%\text{dim} \,C(\vek A) + \text{dim} \,N(\vek A^T) &= m &&\quad \text{dim}\,C(\vek A^T) + \text{dim} %\,N(\vek A) &&= n\\
%C(\vek A) + N(\vek A^T) &= \mathbb{R}^m &&\quad C(\vek A^T) + N(\vek A) = \mathbb{R}^n \\
%N(\vek A) &\perp C(\vek A^T) &&\quad N(\vek A^T) \perp C(\vek A)
%\end{alignat}
%\end{subequations}
Jeder Vektor $\vek x = \vek x_C + \vek x_{N'} \in \mathbb{R}^m$ ist ein Teil $\vek x_C$ aus $C(\vek  A)$ und ein Teil $\vek x_{N'}$ aus $N(\vek A^T)$ und analog ist eine solche Aufspaltung des $\mathbb{R}^n$ m\"{o}glich, $\vek x = \vek x_R + \vek x_N \in \mathbb{R}^n$, {\em \glq The four fundamental subspaces\grq{}\/} \cite{Strang4}.

In der Statik ist $\vek K = \vek A$ die symmetrische Steifigkeitsmatrix eines Tragwerks und $C(\vek K) = R(\vek K)$ und $N(\vek K) = N(\vek K^T)$ enth\"{a}lt, wenn das Tragwerk nicht kinematisch ist, nur den Nullvektor $\vek 0$. Ist es beweglich, dann ist eine Gleichgewichtslage $\vek u$ nur m\"{o}glich, wenn die Kr\"{a}fte $\vek f$ orthogonal zu den $\vek u_0$ sind, $\vek u_0^T\,\vek f = 0$, wenn es also Gleichgewichtskr\"{a}fte sind -- denn dann ist garantiert, dass $\vek f$ in $C(\vek K)$ liegt, dass es eine L\"{o}sung $\vek K\vek u = \vek f$ gibt. Die Regel lautet
\begin{align}
\vek f \in C(\vek K) \qquad \Leftrightarrow \qquad \vek f \perp N(\vek K^T)
\end{align}
Man nehme die $2 \times 2$ Steifigkeitsmatrix eines Stabelements
\begin{align}
\vek K = \frac{EA}{l}\left [ \barr{r r} 1 & -1 \\ -1 & 1 \earr \right ]\,.
\end{align}
Jeden Vektor $\vek u = \vek u_R + \vek u_N \in \mathbb{R}^2$ kann man in eine Translation $\vek u_N = c\,\{1, 1\}^T$, $c$ beliebig, des Elements und eine Streckung $\vek u_R$ (ungleiche Verschiebungen der Enden) des Elements aufspalten. F\"{u}r jedes $\vek u_R$ sind die Kr\"{a}fte $\vek f = \vek K\,\vek u_R$ im Gleichgewicht, sind sie orthogonal zu den $\vek u_{N'} = c\,\{1 , 1\}^T \in N(\vek K^T) = N(\vek K)$.

Der Vektor $\vek f = \{1,0\}^T$ steht nicht senkrecht auf $N(\vek K^T)$, liegt also nicht in $C(\vek K)$ und daher hat $\vek K\,\vek u = \vek f$ keine L\"{o}sung $\vek u$.

Wir haben diese Beispiele aus der linearen Algebra hier so ausf\"{u}hrlich diskutiert, weil die Greenschen Identit\"{a}ten ja im Grunde auch solche Null-Summen sind und viele wichtige Ergebnisse, man denke nur an die Galerkin Orthogonalit\"{a}t, $a(u-u_h\,,\Np_i) = 0$, algebraischer Natur sind.

Bei Funktionen, wie etwa beim Stab,
\begin{align}
\text{\normalfont\calligra B\,\,}(u,\hat{u}) &= \int_0^{\,l} - EA\,u''(x)\,\hat{u}(x)\,dx + [N\,\hat{u}]_{@0}^{@l}\nn \\
&- [u \hat{N}]_{@0}^{@l} - \int_0^{\,l} u(x)\,(- EA\,\hat{u}''(x))\,dx = 0\,,
\end{align}
f\"{u}hrt die Orthogonalit\"{a}t, das Operieren mit den Null-L\"{o}sungen, $\hat{u}(x) = 1$, zum Beispiel gerade auf die Gleichgewichtsbedingungen
\begin{align}
\text{\normalfont\calligra B\,\,}(u,1) = \int_0^{\,l} - EA\,u''(x) \cdot 1\,dx + N(l) \cdot 1 - N(0) \cdot 1 = 0\,.
\end{align}
Am freigeschnittenen Stab m\"{u}ssen also die Normalkr\"{a}fte an den Stabenden die Streckenlast ausbalancieren.

Ein anderes Beispiel: Die Einheitsverformungen $\Np_i(x)$ eines Stabes oder Balken sind orthogonal zu den L\"{o}sungen $u_p$ oder $w_p$ des fest eingespannten Stabes, $a(u_p,\Np_i) = 0$. Genauso sind die Momente $M_i$ der $X_i$ beim Kraftgr\"{o}{\ss}enverfahren orthogonal zum endg\"{u}ltigen Momentenverlauf, $(M_i,M) = 0$ und jede Biegelinie $w$ aus einer Lagersenkung ist orthogonal zu den virtuellen Verr\"{u}ckungen des Systems, $a(w,\,\delta w) = 0$.

Oder man nehme die singul\"{a}re Steifigkeitsmatrix eines nicht festgehaltenen Stabes aus zwei Elementen. Das Produkt $\vek K\,\vek u$ ergibt den Vektor
\begin{align}
\frac{EA}{l_e}\left[ \barr {r @{\hspace{4mm}}r @{\hspace{4mm}}r}
      1  & -1 & 0\\
      -1 & 2 & -1\\
      0 &-1 & 1    \earr \right] \left[ \barr {c }
      u_1\\
      u_2\\
      u_3   \earr \right] = \frac{EA}{l_e}\left[ \barr {c }
      u_1 - u_2\\
      - u_1 + 2 \cdot u_2 - u_3\\
      -u_2 + u_3   \earr \right]\,.
\end{align}
Die Zeilensumme dieses Vektors ist null
\begin{align}
(u_1 - u_2) + (- u_1 + 2 \cdot u_2 - u_3) + (- u_2 + u_3) = 0
\end{align}
wie immer auch die $u_i$ aussehen und das bedeutet, dass das singul\"{a}re System $\vek K\,\vek u = \vek f$ nur eine L\"{o}sung hat, wenn $f_1 + f_2 + f_3 = 0$ ist, was nat\"{u}rlich gerade die Gleichgewichtsbedingung ist.
%----------------------------------------------------------------------------------------------------------
\begin{figure}[tbp]
\centering
\if \bild 2 \sidecaption \fi
\includegraphics[width=0.3\textwidth]{\Fpath/U544}
\caption{Der Dreier-Schritt, hier am Stab, $-d/dx\,(EA\,d/dx)\, u = p$} \label{U544}
\end{figure}%

%----------------------------------------------------------------------------------------------------------
Im Sinne der obigen Formulierungen ist es die Forderung $\vek f \perp N(\vek K^T) = N(\vek K)$ also $\vek f^T\,\vek u_0 = f_1 + f_2 + f_3 = 0$, denn $\vek u_0 = \{1,1,1\}^T$ ist der Vektor, der $\vek N(\vek K^T)$ aufspannt.

Jede Biegelinie $w$ eines Balkens gen\"{u}gt den Gleichgewichtsbedingungen
\begin{align}
\text{\normalfont\calligra G\,\,}(w,a\,x + b) = 0 \qquad a\,x + b = \text{Null-Eigenl\"{o}sungen}\,.
\end{align}
Analog gilt, dass das Produkt der Balkenmatrix $\vek K$ mit einem beliebigen Vektor $\vek w$ Knotenkr\"{a}fte $\vek f = \vek K\,\vek w$ ergibt, die orthogonal zu den Null-Eigenvektoren $\vek \delta \vek w_0$, Translationen und Rotationen, von $\vek K$ sind, $\vek \delta \vek  w_0^T\,\vek f = 0$, der Vektor $\vek f$ also den Gleichgewichtsbedingungen gen\"{u}gt.

Zur Algebra geh\"{o}rt in gewissem Sinn auch die Tatsache, dass man die Steifigkeitsmatrix $\vek K = \vek A^T\,\vek C \vek A$ eines Stabes (und anderer Bauteile analog) als Produkt dreier Matrizen schreiben kann, entsprechend der Zerlegung der Gleichung $-EA\,u'' = p$ in die drei Teile
\begin{align}
u' = \varepsilon \qquad EA\,\varepsilon = N \qquad - N' = p\,.
\end{align}
Die Matrix $\vek A = [\ldots 0\,\, \,1/l_i\,\,-1/l_i\,\,\,0 \ldots]$ ($1/l_i$ auf der Diagonalen und $-1/l_i$ auf der rechten Nebendiagonalen) und ihre Transponierte $\vek A^T = - \vek A$ \glq differenzieren\grq{}, sie bilden $d/dx$ und $-d/dx$ nach und die Diagonalmatrix $\vek C$ enth\"{a}lt auf ihrer Diagonalen die Steifigkeiten $EA_i$ der einzelnen Elemente $i$ mit der L\"{a}nge $l_i$. Dieses Schema, s. Abb. \ref{U544}, durchzieht die ganze Mechanik, \cite{Strang4}.

Ein Fachwerk ist genau dann {\em statisch bestimmt\/}\index{statisch bestimmt}, wenn $\vek A$ quadratisch ist, weil man dann aus $\vek A^T\,\vek N = \vek f$ die Normalkr\"{a}fte $N_i$ berechnen kann  und aus diesen dann im Nachlauf, $ \vek C\,\vek A \,\vek u = \vek N$,  die Knotenverschiebungen $u_i$.

Wenn das Fachwerk {\em statisch unbestimmt\/}\index{statisch unbestimmt} ist, es gibt mehr St\"{a}be, als Knotenverschiebungen, $ \vek A$ ist ein rechteckige Matrix, f\"{u}hren die Handmethoden wie der {\em Ritterschnitt\/} nicht zum Ziel, dann braucht man einen Computer, dann muss man erst aus $\vek K\,\vek u = \vek f $ die Knotenverschiebungen $u_i$ berechnen und aus den Dehnungen der St\"{a}be dann die Normalkr\"{a}fte $N_i $.

%%%%%%%%%%%%%%%%%%%%%%%%%%%%%%%%%%%%%%%%%%%%%%%%%%%%%%%%%%%%%%%%%%%%%%%%%%%%%%%%%%%%%%%%%%%%%%%%%%%
\textcolor{sectionTitleBlue}{\section{Die Algebra der finiten Elemente}}
FE-L\"{o}sungen $u_h = \sum_i u_i\Np_i(x)$ sind Variationsl\"{o}sungen
\begin{align}
a(u_h,\Np_i) - (p,\Np_i) = 0 \qquad i = 1,2,\ldots, n
\end{align}
und daher lassen sich eine ganz Reihe von Gleichungen und Ungleichungen mit der FE-L\"{o}sung linearer Probleme formulieren, \cite{Ha4} Chapter 7.12, {\em Important equations and inequalities\/}. Wir pr\"{a}sentieren einen Ausschnitt.

Der L\"{o}sungsraum $\mathcal{V}$ enth\"{a}lt alle $u$, die die geometrischen Lagerbedingungen erf\"{u}llen und $\mathcal{V}_h \subset \mathcal{V}$ ist der Ansatzraum der finiten Elemente. Es ist $u$ die exakte L\"{o}sung, $u_h$ die FE-L\"{o}sung, $e = u - u_h$ ist der Fehler und $v_h$ ist eine Testfunktion aus $\mathcal{V}_h$. In einem LF $p$ gilt:\\

\begin{itemize}
\item Auf $\mathcal{V}$ gilt (Achtung, dieser Ausdruck ist nur auf $\mathcal{V}_h$ null, s. (\ref{E7Portho}))
\bfo\label{Theo199}
a(e,v) = (p,v)- a(u_h,v) = (p,v) - (p_h,v) \qquad v \in \mathcal{V}\,.
\efo
\item Insbesondere also
\bfo
(p,u) - (p,u_h) = (p,e) = a(e,u) = (p,u) - (p_h,u)
\efo
und daher auch
\bfo
(p,u_h) = (p_h, u) \qquad \mbox{{\em Symmetrie\/}}
\efo
\item Galerkin Orthogonalit\"{a}t
\begin{equation}
a(e, v_h)=0\quad  v_h\in \mathcal{V}_h \,.
\end{equation}
\item Die Fehlerkr\"{a}fte $p - p_h$ sind orthogonal zu den Testfunktionen
\begin{equation}\label{E7Portho}
a(e, v_h)= (p,v_h) - (p_h,v_h)=0\quad  v_h\in \mathcal{V}_h.
\end{equation}
\item Der FE-Lastfall $p_h$ ist orthogonal zu $e$
\begin{equation}
a( u_h,  e)= (p_h, e)=0.
\end{equation}
\item Die Einheitslastf\"{a}lle $p_i$, die {\em shape forces\/}, sind orthogonal zu $e$
\begin{equation}
a( \Np_i,  e) = (p_i, e)=0.
\end{equation}
\item Die FE-L\"{o}sung minimiert den Fehler in der inneren Energie
\begin{equation}\label{E7Ungla}
a( e, e)\leq a( u-  v_h, u - v_h)\quad  v_h\in \mathcal{V}_h\,,
\end{equation}
denn f\"{u}r jedes $v_h \in \mathcal{V}_h$ gilt
\bfo\label{E7Proof1}
a(e + v_h,e + v_h) = \underbrace{a(e,e)}_{>\, 0} + 2\underbrace{a(e,v_h)}_{=\, 0} +
\underbrace{a(v_h,v_h)}_{>\, 0}\,.
\efo
Die Energie $a(e,e)$, die n\"{o}tig ist, um die FE-L\"{o}sung \glq zurechtzur\"{u}cken\grq{}, $u_h \to u$, ist bei der FE-L\"{o}sung im Vergleich mit allen anderen N\"{a}herungen $v_h$ in $\mathcal{V}_h$ am kleinsten.
\item Die innere Energie der FE-L\"{o}sung ist kleiner als die exakte Energie (hier ohne den Faktor $1/2$)
\begin{equation}
a( u_h, u_h)\leq a( u, u) \qquad \mbox{in einem LF $p$}\,,
\end{equation}
weil
\bfo\label{IneqE7Proof}
0 < a(u,u) &=& a(u_h + e,u_h + e)\nn\\ &=& a(u_h,u_h) + 2\,\underbrace{a(e,u_h)}_{=\, 0}
+ \underbrace{a(e,e)}_{>\, 0}\,.
\efo
\item Es gilt
\begin{equation}
\Pi( u)\leq \Pi( u_h)\,,
\end{equation}
weil
\bfo
\Pi(u_h) &=& \Pi(u - e) = \frac{1}{2}\,a(u,u) - a(u,e) + \frac{1}{2}\,a(e,e) - (p,u) +
(p,e)\nn \\
&=& \Pi(u) \underbrace{- a(u,e) + (p,e)}_{G(u,e) = 0} +
\frac{1}{2}\,\underbrace{a(e,e)}_{>\, 0}\,,
\efo
\item und ebenso
\bfo
(p,u_h) = a(u_h,u_h) < a(u,u) = (p,u)\,.
\efo
\end{itemize}
In einem LF $\Delta$ (Lagersenkung) enth\"{a}lt der L\"{o}sungsraum $\mathcal{S} = w_\delta \oplus \mathcal{V}$ nicht die Funktion $w = 0$, denn $\mathcal{V}$ wird um eine feste Funktion $w_\delta $, die die Lagersenkung beschreibt, \glq geshiftet\grq{}, ist $\mathcal{S}$ also kein Vektorraum, sondern nur noch eine Mannigfaltigkeit (die Summe $u + v$ zweier Funktionen aus $\mathcal{S}$ liegt nicht in $\mathcal{S}$) und daher muss man bei der Anwendung der obigen Formeln aufpassen, \cite{Ha5}.


%%%%%%%%%%%%%%%%%%%%%%%%%%%%%%%%%%%%%%%%%%%%%%%%%%%%%%%%%%%%%%%%%%%%%%%%%%%%%%%%%%%%%%%%%%%%%%%%%%%
\textcolor{sectionTitleBlue}{\section{Galerkin}}
Bei der Methode von Galerkin ist die FE-L\"{o}sung
\begin{align}
u_h = \sum_i\,u_i\,\Np_i(\vek x)
\end{align}
die Projektion der exakten L\"{o}sung auf den Ansatzraum $\mathcal{V}_h$
\begin{align}
a(u - u_h,\Np_i) = 0 \qquad \text{f\"{u}r alle $\Np_i \in \mathcal{V}_h$}\,.
\end{align}
Da man das, was man projizieren will, die exakte L\"{o}sung $u$, nicht kennt, ersetzt man mittels der ersten Greenschen Identit\"{a}t
\begin{align}
\text{\normalfont\calligra G\,\,}(w,\Np_i) &= \delta A_a - \delta A_i = 0
\end{align}
die virtuelle innere Arbeit $\delta A_i = a(u,\Np_i)$ durch die virtuelle \"{a}u{\ss}ere Arbeit der Lasten $\delta A_a = (p,\Np_i) = f_i$ und kommt so zu dem System
\begin{align}
\vek K\,\vek u = \vek f\,.
\end{align}
Es geht aber nat\"{u}rlich auch direkt.

Die Biegelinie $w(x)$ eines zwischen zwei W\"{a}nden aufgeh\"{a}ngten Seils $w(x)$
\begin{align}\label{Eq3}
- H\,w''(x) = p(x)  \qquad w(0) = w(l) = 0
\end{align}
ist wegen der ersten Greenschen Identit\"{a}t auch die L\"{o}sung des Variationsproblems: {\em Finde ein $w$ so, dass\/}
\begin{align}\label{Eq25}
\text{\normalfont\calligra G\,\,}(w,\delta w) = \int_0^{\,l} p\,\delta w\,dx - \int_0^{\,l} H\,w'\,\delta w'\,dx = 0 \qquad \text{f\"{u}r alle $\delta w \in \mathcal{V} $}\,.
\end{align}
Mit finiten Elementen formuliert man dieses Problem auf einem Teilraum $\mathcal{V}_h \subset \mathcal{V} \,(= \text{alle $w$ mit null Randwerten})$ und kommt so direkt auf das System
\begin{align}
\vek K\,\vek u = \vek f\,,
\end{align}
w\"{a}hrend Galerkin ja den Zwischenschritt $a(w,\Np_i) = (p,\Np_i)$ einschalten muss.

Die klassischen L\"{o}sungen der Statik sind {\em Minimumsl\"{o}sungen\/}\index{Minimumsl\"{o}sung}, sie machen die potentielle Energie des Tragwerks zum Minimum. L\"{o}st man die Probleme nur n\"{a}herungsweise, weil man nur auf einem Teilraum $\mathcal{V}_h \subset \mathcal{V}$ sucht, dann f\"{u}hrt das auf dasselbe Gleichungssystem $\vek K\,\vek u = \vek f$ wie das Galerkin Verfahren. Das $\vek K\,\vek u - \vek f = \vek 0$ ist sozusagen die Bedingung $\Pi'(x) = 0$ (horizontale Tangente) in der Schulmathematik. Damit die Funktion $\Pi(x) = 1/2\,a\,x^2 - f\,x$ im Punkt $x$ ein Extremum hat, muss $\Pi'(x) = a\,x - f = 0$ sein.

%%%%%%%%%%%%%%%%%%%%%%%%%%%%%%%%%%%%%%%%%%%%%%%%%%%%%%%%%%%%%%%%%%%%%%%%%%%%%%%%%%%%%%%%%%%%%%%%%%%
\textcolor{sectionTitleBlue}{\section{Schwache L\"{o}sung}}
Die L\"{o}sung des Randwertproblems (\ref{Eq3}) nennt man eine {\em starke L\"{o}sung\/}\index{starke L\"{o}sung} und die L\"{o}sung des Variationsproblems (\ref{Eq25}) eine {\em schwache L\"{o}sung\/}\index{schwache L\"{o}sung}.

Diese Bezeichnung wird gew\"{o}hnlich damit erkl\"{a}rt, dass eine schwache L\"{o}sung nicht zweimal differenzierbar sein muss, wie das $w$ in der Differentialgleichung $-H\,w'' = p$, sondern nur einmal, wie das $w'$ im Integral der Wechselwirkungsenergie.

Treffender scheint uns die folgende Interpretation. In der Mathematik kennt man den Begriff der {\em schwachen Konvergenz\/}\index{schwache Konvergenz}. Man \"{u}berzeugt sich nicht direkt davon, dass eine Schar von Funktionen $f_n(x)$ gegen eine Zielfunktion $f(x)$ konvergiert, sondern indirekt. Die Ann\"{a}herung der $f_n(x)$ an $f(x)$ wird mit einer Schar von Kontrollfunktionen $\Np_i(x)$ getestet. Man sagt, dass die Folge $f_n(x)$ {\em schwach\/} gegen $f(x)$ konvergiert, wenn
\begin{align}
\lim_{n \to \infty} \int_0^{\,l} f_n(x)\,\Np_i(x)\,dx = \int_0^{\,l} f(x)\,\Np_i(x)\,dx \qquad \text{f\"{u}r alle $\Np_i$}\,.
\end{align}
Schwache Konvergenz ist so etwas, wie die Konvergenz von Funktionalen. Jede Funktion $f_n(x)$ kann man ja einem Funktional $J_n(.)$  gleichsetzen
\begin{align}
J_n(\Np_i) = \int_0^{\,l} f_n(x)\,\Np_i(x)\,dx
\end{align}
und schwache Konvergenz bedeutet, dass die Funktionale $J_n(.)$ gegen das Zielfunktional
\begin{align}
\int_0^{\,l} f(x)\,\Np_i(x)\,dx
\end{align}
konvergieren -- das ist wieder die \glq Wackel\"{a}quivalenz\grq{}.
%----------------------------------------------------------------------------------------------------------
\begin{figure}[tbp]
\centering
\if \bild 2 \sidecaption \fi
\includegraphics[width=.9\textwidth]{\Fpath/WAAGE4D}
\caption{Die Marktfrau kontrolliert das Gleichgewicht einer Waage mittels des Prinzips
der virtuellen Verr\"{u}ckungen} \label{Waage}
\end{figure}%
%----------------------------------------------------------------------------------------------------------

Und diese Terminologie passt genau auf die finiten Elemente. Die FE-L\"{o}sung ist eine schwache L\"{o}sung, weil ihre \"{U}bereinstimmung mit der exakten L\"{o}sung nicht an der Differentialgleichung festgemacht wird, sondern sie indirekt durch $i = 1,2,\ldots$ \glq Wackeltests\grq{} kontrolliert wird
\begin{align}
\lim_{h \to 0} a(u_h,\Np_i) = a(u, \Np_i) \qquad \text{f\"{u}r alle $\Np_i$}\,.
\end{align}
Wegen $\delta A_i = \delta A_a$ ist dies mit
\begin{align}
\lim_{h \to 0} \int_0^{\,l} p_h\,\Np_i\,dx = \int_0^{\,l} p\,\Np_i\,dx  \qquad \text{f\"{u}r alle $\Np_i$}
\end{align}
identisch, also der \"{A}quivalenz in den \"{a}u{\ss}eren virtuellen Arbeiten. Praktisch ist es nat\"{u}rlich so, dass die finiten Elemente die Grenze $h \to 0$ nie erreichen und auch nur endlich viele Tests gefahren werden, weil es auf einem Netz nur endlich viele {\em shape functions\/} $\Np_i$ gibt.\\

\hspace*{-12pt}\colorbox{highlightBlue}{\parbox{0.98\textwidth}{
Formal betrachtet kann man die Methode der finiten Elemente als ein Verfahren ansehen, ein Funktional $J(\delta u) = (p, \delta u)$ durch ein Funktional $J_h(\delta u) = (p_h, \delta u)$ zu ersetzen, bzw., wenn man unendlich viel Geduld hat, $h \to 0$, durch eine Folge von Funktionalen $J_h(\delta u) = (p_h, \delta u)$ anzusteuern.}}\\

%----------------------------------------------------------------------------------------------------------
\begin{figure}[tbp]
\centering
\if \bild 2 \sidecaption \fi
\includegraphics[width=.8\textwidth]{\Fpath/ROLLEN}
\caption{Der Werkzeugmacher pr\"{u}ft die Exzentrizit\"{a}t, indem er den Zylinder \"{u}ber den Tisch rollt} \label{Rollen}
\end{figure}%
%----------------------------------------------------------------------------------------------------------

Die gro{\ss}e praktische Bedeutung des Begriffs der schwachen L\"{o}sung erkennt man, wenn man der Marktfrau zuschaut, s. Abb. \ref{Waage}. Auch sie schlie{\ss}t indirekt. Sie muss die Gleichung
\begin{align}
P_l \cdot h_l = P_r \cdot h_r
\end{align}
l\"{o}sen. Diese Gleichung bedeutet, wie sie wei{\ss}, dass bei jeder Drehung $\delta \Np$ des Waagebalkens die Arbeiten auf der linken und rechten Seite gleich sind
\begin{align}
P_l \cdot h_l = P_r \cdot h_r \qquad \Rightarrow \qquad P_l \cdot h_l \cdot \tan \delta \Np = P_r \cdot h_r \cdot \tan\delta \Np
\end{align}
und so schlie{\ss}t sie, indem sie an der Waage wackelt, indirekt, schlie{\ss}t sie \glq r\"{u}ckw\"{a}rts\grq{}
\begin{align}
P_l \cdot h_l = P_r \cdot h_r \qquad \Leftarrow \qquad P_l \cdot h_l \cdot \tan\delta \Np = P_r \cdot h_r \cdot \tan\delta \Np\,.
\end{align}
Dasselbe macht der Werkzeugmacher, der einen Zylinder mit den Fingern hin und her rollt, s. Abb. \ref{Rollen}. Er wei{\ss}: Wenn der Zylinder eine perfekte Kreisform hat, dann ver\"{a}ndert der Schwerpunkt bei einer  Drehung des Zylinders seine H\"{o}he \"{u}ber der Tischkante nicht. Wenn der Test fehlschl\"{a}gt, wenn die Finger eine leichte vertikale Bewegung sp\"{u}ren, dann ist es kein perfekter Zylinder.
%----------------------------------------------------------------------------------------------------------
\begin{figure}[tbp]
\centering
\if \bild 2 \sidecaption \fi
\includegraphics[width=.4\textwidth]{\Fpath/CIRCLE}
\caption{Das Achteck ist einem Zylinder f\"{u}r alle Drehungen, die ein Vielfaches von $45^\circ$ sind, \"{a}quivalent } \label{Circle}
\end{figure}%
%----------------------------------------------------------------------------------------------------------

Der Lehrling, der aus einem quadratischen Eisen einen Zylinder schleifen soll, macht es wie die finiten Elemente. Zu Beginn ist das quadratische Eisen einem Zylinder hinsichtlich aller Drehungen $\delta \Np$, die ein Vielfaches von $90^\circ$ sind, \"{a}quivalent, \"{a}ndert der Schwerpunkt bei einer $90^\circ$-Drehung seine H\"{o}he nicht. Indem der Lehrling nun in das Profil mehr und mehr Kanten ($n$) schleift, vergr\"{o}{\ss}ert er den Testraum, $\mathcal{V}_4 \to \mathcal{V}_8 \to \mathcal{V}_{16} \ldots$
\begin{align}
\mathcal{V}_n = \{\text{alle Vielfachen von} \,\delta \Np = \frac{360}{n}\} \qquad n = 4, 8, 16 \ldots \text{Kanten}
\end{align}
und er n\"{a}hert sich so indirekt (auf dem Weg der \glq Dreh\"{a}quivalenz\grq{}) dem Zylinder, s. Abb. \ref{Circle}, \cite{Ha4}, \cite{Ha5}.

{\em \"{A}quivalenz\/} ist, wenn wir hier weiter ausholen d\"{u}rfen, der Schl\"{u}sselbegriff der finiten Elemente. Die FEM l\"{o}st nicht den urspr\"{u}nglichen Lastfall, sondern einen dazu \"{a}quivalenten Lastfall. Eine \"{A}quivalenzrelation liegt vor, wenn aus $a \sim b $ und $b \sim c $ folgt, dass auch $a \sim c$, also
\begin{align}
p \sim \Np_i \qquad \text{und} \qquad p_h \sim \Np_i \qquad \Rightarrow \qquad p \sim p_h
\end{align}
Das Merkmal der finiten Elemente ist, dass diese \"{A}quivalenz \glq endlich\grq\ ist, d.h. wir stellen die \"{A}quivalenz nur bez\"{u}glich endlich vieler Testfunktionen $\Np_i, i = 1,2,\ldots n$ her.

Auch wenn wir mit einem Zollstock die L\"{a}nge zweier Bretter $A$ und $B$ vergleichen, nutzen wir eine \"{A}quivalenzrelation. Die Bretter sind gleich lang, sind zueinander \"{a}quivalent, wenn sie mit dem Zollstock in identischen Relationen stehen. \"{A}quivalenz ist indirekte Gleichheit, ist wie die schwache Konvergenz, und sie m\"{u}ndet in eine echte Identit\"{a}t, $A \equiv B$, (alle Stellen nach dem Komma sind gleich), wenn die Relation alle Tests besteht, also auch den Test mit dem Urmeter in Paris.\\

\begin{remark}
Dem Begriff der schwachen Konvergenz ist auch der Unterschied zwischen {\em schwachen} und {\em starken Randbedingungen\/}\index{schwache Randbedingung} \index{starke Randbedingung}geschuldet. Geometrische Lagerbedingungen, wie $w = 0$ werden von allen Ansatzfunktionen $\Np_i \in \mathcal{V}_h$ erf\"{u}llt, sind starke Randbedingungen, w\"{a}hrend eine statische Randbedingung wie $m_n = 0$ am gelenkig gelagerten Plattenrand nur im integralen Mittel -- im schwachen Sinne -- erf\"{u}llt ist, $(m_n,\Np_i) = 0$, aber nicht punktweise. Deswegen nennt man statische Randbedingungen schwache Randbedingungen. Ihre Einhaltung ist erst in der Grenze $h \to 0 $ garantiert.
\end{remark}
\pagebreak
%%%%%%%%%%%%%%%%%%%%%%%%%%%%%%%%%%%%%%%%%%%%%%%%%%%%%%%%%%%%%%%%%%%%%%%%%%%%%%%%%%%%%%%%%%%%%%%%%%%
\textcolor{sectionTitleBlue}{\section{Variation und Greensche Identit\"{a}t}}

Die potentielle Energie eines links festgehaltenen Stabes, $u(0) = 0$, mit einem freien Ende, $ N(l) = 0 $, lautet
\begin{align}
\Pi(u) = \frac{1}{2} \int_{0}^{l} \frac{N^2}{EA}\,dx - \int_{0}^{l}p\,u\,dx = \frac{1}{2} a(u,u) - (p,u)\,.
\end{align}
Wenn $u $ die Gleichgewichtslage des Stabs unter der Streckenlast $p$ ist, dann sollte die erste Variation der potentiellen Energie in diesem Punkt Null sein, $\Pi(u)$ dort eine \glq horizontale Tangente\grq\  haben.

Der Wert von $\Pi$ in einem benachbarten Punkt $u + \varepsilon \,\delta u$ betr\"{a}gt
\begin{align}
\Pi(u + \varepsilon \delta u) &= \frac{1}{2} a(u + \varepsilon \delta u,u + \varepsilon \delta u) - (p,u + \varepsilon \delta u) \nn \\
&= \frac{1}{2}a(u,u) + \varepsilon \cdot a(u,\delta u) + \varepsilon^2 \cdot \frac{1}{2} a(\delta u, \delta u) - (p,u) - \varepsilon \cdot (p,\delta u)
\end{align}
und damit lautet die erste Variation
\begin{align}
\delta \Pi(u, \delta u) &= \frac{d}{d\varepsilon} \Pi(u + \varepsilon \delta u)|_{\varepsilon = 0}
= a(u, \delta u) - (p, \delta u)
\end{align}
was mit
\begin{align}
\text{\normalfont\calligra G\,\,}(u,\textcolor{red}{\delta u}) = \int_0^{\,l} p \,\textcolor{red}{\delta u(x)}\,dx - \int_0^{\,l} \frac{N\,\textcolor{red}{\delta N}}{EA}\,dx = (p, \delta u) - a(u, \delta u) = 0
\end{align}
identisch ist, denn die Randarbeiten $[\ldots]$ fallen wegen $N(l) = 0 $ und $\delta u(0) = 0$ weg.

%%%%%%%%%%%%%%%%%%%%%%%%%%%%%%%%%%%%%%%%%%%%%%%%%%%%%%%%%%%%%%%%%%%%%%%%%%%%%%%%%%%%%%%%%%%%%%%%%%%
\textcolor{sectionTitleBlue}{\section{Kraftgr\"{o}{\ss}enverfahren versus Weggr\"{o}{\ss}enverfahren}}
Die Statik ist nicht das einzige Gebiet, in dem es die Wahl zwischen einer Weggr\"{o}{\ss}en- und einer Kraftgr\"{o}{\ss}enformulierung gibt. In der Elektrotechnik kann man zwischen der {\em Kirchhoffschen Maschenregel\/}: Summe der Spannungsabf\"{a}lle in einer Masche ist null, (dem entspricht, dass die Einheitsspannungszust\"{a}nde der $X_i$ Gleichgewichtszust\"{a}nde sind), und der {\em Kirchhoffschen Knotenregel\/}: Summe der Str\"{o}me in jedem Knoten ist null, (\glq Knotengleichgewicht\grq{}) w\"{a}hlen.

In der {\em computational mechanics\/} werden, wie {\em Gilbert Strang\/} bemerkt hat, die Knotenpunktsmethoden den Maschengleichungen (= Einheitsspannungszust\"{a}nde $X_i$) vorgezogen, {\em \glq Loop equations versus nodal equations\grq{}\/}, \cite{Strang4} p. 158. Die Maschengleichungen haben heute nur noch als Handmethode bei kleinen Problemen eine Chance, sind den Knotenpunktsmethoden dann sogar \"{u}berlegen. Man darf sie nur nicht  in Matrizenschreibweise formulieren, das wirkt umst\"{a}ndlich und fremd.

Unbesehen davon ist das Kraftgr\"{o}{\ss}enverfahren nat\"{u}rlich ein hervorragendes Mittel, um Statik zu lernen!

%----------------------------------------------------------------------------------------------------------
\begin{figure}[tbp]
\centering
\if \bild 2 \sidecaption \fi
\includegraphics[width=0.7\textwidth]{\Fpath/U426}
\caption{Drehung eines Balkens und daraus abgelesene Einflussfunktionen f\"{u}r die horizontale Lagerkraft $H$ bei vertikal bzw. horizontal gerichteter Wanderlast. Genau genommenen m\"{u}sste man, bei der angenommenen Wirkrichtung der Wanderlasten, die 1 und das $1/\tan\, \alpha$ im Antrag der beiden EF mit  (-1) multiplizieren } \label{U426}
\end{figure}%
%----------------------------------------------------------------------------------------------------------
%%%%%%%%%%%%%%%%%%%%%%%%%%%%%%%%%%%%%%%%%%%%%%%%%%%%%%%%%%%%%%%%%%%%%%%%%%%%%%%%%%%%%%%%%%%%%%%%%%%
\textcolor{sectionTitleBlue}{\section{Pseudodrehungen}}\label{Korrektur33}
Wir hatten in Kapitel 1 erw\"{a}hnt, dass in der linearen Statik alle Drehungen Pseudodrehungen sind\index{Pseudodrehungen}. Die Frage eines Studenten hat uns jedoch klar gemacht, dass wir dieses Thema noch etwas deutlicher herausarbeiten sollten.

Auch in der Statik sind Drehungen echte Drehungen, sind die Ausl\"{o}ser von Einflussfunktionen echte Drehungen, nur nimmt sich die Statik die Freiheit beim {\em Auswerten\/} der Drehungen die Dinge zu vereinfachen.

Um die Einflussfunktion f\"{u}r die horizontale Lagerkraft des Balkens in Abb. \ref{U426} zu berechnen, wird das Lager gel\"{o}st und der Balken um sein linkes Lager gedreht und zwar so weit, bis die horizontale Lagerkraft den Weg -1 gegangen ist. Es handelt sich also um eine physikalisch echte Drehung, wie die Drehung einer Kompassnadel. Was die Statik aber dann macht ist, dass sie annimmt, dass bei dieser Drehung alle Punkte sich so bewegen, als ob sie der Tangente an den Drehkreis (der f\"{u}r jeden Punkt einen anderen Radius hat) folgen w\"{u}rden. So entstehen die Einflussfunktion in Abb. \ref{U426}. {\em Es wird also richtig gedreht, aber \glq falsch\grq{} gemessen\/}.

%----------------------------------------------------------------------------------------------------------
\begin{figure}[tbp]
\centering
\if \bild 2 \sidecaption \fi
\includegraphics[width=0.5\textwidth]{\Fpath/Einfluss6}
\caption{Die Lagerkraft $A_z$ wird nur dann null, wenn sich $P_x$ auf der Tangente an den Drehkreis um $A_z$ bewegt} \label{Einfluss6}
\end{figure}%
%----------------------------------------------------------------------------------------------------------
Die urspr\"{u}ngliche Begr\"{u}ndung f\"{u}r diese Vereinfachung, ist die Beobachtung, dass bei kleinen Drehungen der Fehler vernachl\"{a}ssigbar klein ist -- so wird die Linearisierung ja motiviert. Ist man mit dieser Argumentation durchgekommen, hat keine Autorit\"{a}t dagegen gemurrt, dann hat man freie Fahrt, und wendet das Ergebnis nun bedenkenlos auch auf gro{\ss}e Drehungen an, wie das ja auch Archimedes tat, als er sein Hebelgesetz bewies. Der Hebelarm $h$ hat f\"{u}r ihn immer dieselbe L\"{a}nge, egal wie gro{\ss} $\delta \Np$ ist, weil sich (in \"{U}bereinstimmung mit dem Kreuzprodukt: {\em Moment = Kraft $\times$ Ortsvektor\/}) seine Hand auf der Tangente an den Drehkreis bewegt. Das ist der Punkt, an dem sich die Physik \"{a}ndert. Aus dem zugestandenen {\em infinitesimal kleinen $\delta \Np$\/} wird ein gro{\ss}es $\delta \Np$ und die virtuellen Verr\"{u}ckungen des Balkens k\"{o}nnen beliebig gro{\ss} werden.

Das Gl\"{u}ck f\"{u}r den Statiker ist, dass die Mathematik derselben Meinung ist. Sie braucht das Vehikel infinitesimal klein gar nicht, sie hat sich die Welt von Anfang an schon immer so gedacht, sind die Starrk\"{o}rperbewegungen $\delta w = a + b\,x $ des mathematischen Balkens $EI\,w^{IV} = p$ genau von diesem Typ, sind es Bewegungen auf der Tangente an den Drehkreis.

Man kann es aber auch anders sehen: Die Mathematik betritt die Szene erst, nachdem die Schlachten geschlagen sind, die Modellbildung, die eben die Linearisierung beinhaltet, abgeschlossen ist und das Ergebnis in der Gleichung $EI\,w^{IV} = p$ in kanonisierter Form vorliegt und die Mathematik dann in ruhigem Fahrwasser ihre ganze formale Kraft ausspielen kann.

Die Abb. \ref{Einfluss6} soll zeigen, dass man nur mit einer \glq falschen Messung\grq{} auf den korrekten Wert $A_z = 0$ kommt.

%%%%%%%%%%%%%%%%%%%%%%%%%%%%%%%%%%%%%%%%%%%%%%%%%%%%%%%%%%%%%%%%%%%%%%%%%%%%%%%%%%%%%%%%%%%%%%%%%%%
\textcolor{sectionTitleBlue}{\section{Der adjungierte Operator und die Greensche Funktion}}
Wenn es ein Skalarprodukt gibt, dann gibt es zu einem Operator $L$ den adjungierten Operator $L^*$ und zu jedem linearen Funktional existiert eine Greensche Funktion, die in der linearen Algebra auch ein Vektor sein kann. \\

\hspace*{-12pt}\colorbox{highlightBlue}{\parbox{0.98\textwidth}{
Die Greensche Funktion $g$ des linearen Funktionals $J(u) = (j, u)$ ist die L\"{o}sung der adjungierten Gleichung $L^* g = j$.}}\\

Zu der symmetrischen, selbstadjungierten Matrix $\vek K$ geh\"{o}rt die Identit\"{a}t
\begin{align}
\text{\normalfont\calligra B\,\,}(\vek u,\vek v)  = \vek v^T\,\vek K\,\vek u - \vek u^T\,\vek K\vek v = 0\,.
\end{align}
Ist $\vek J(\vek u) = \vek j^T\,\vek u$ ein lineares Funktional angewandt auf die L\"{o}sung von $\vek K\,\vek u = \vek f$, dann folgt
\begin{align}
J(\vek u) = \vek j^T\,\vek u = \vek j^T\,\vek K^{-1}\,\vek f = \vek g^T\,\vek f\,,
\end{align}
wenn $\vek g$ die L\"{o}sung des adjungierten Systems $\vek K\,\vek g = \vek j$ ist.

Der Operator $- EA u''$ in dem Randwertproblem
\begin{align}
- EA\,u''(x) = p(x) \qquad u(0) = u(1) = 0
\end{align}
ist ebenfalls selbstadjungiert, denn
\begin{align}\label{Eq14}
\text{\normalfont\calligra B\,\,}(u,v) &= \int_0^{\,l} - EA\,u''(x)\,v(x)\,dx + \ldots  - \int_0^{\,l} u\,(-EA\,v'') \,dx = 0\,.
\end{align}
Ist nun
\begin{align}
J(u) = \int_0^{\,l} j(x)\,u(x)\,dx
\end{align}
ein lineares Funktional und ist $g(x)$ die L\"{o}sung des adjungierte Randwertproblems
\begin{align} \label{Eq1}
- EA\,g''(x) = j(x) \qquad g(0) = g(1) = 0\,,
\end{align}
dann folgt nach zweimaliger partiellen Integration, s. (\ref{Eq14}),
\begin{align}
J(u) = \int_0^{\,l} j(x)\,u(x)\,dx = \int_0^{\,l}-EA\, g''(x)\,u(x)\,dx = \int_0^{\,l} g(x)\,p(x)\,dx\,,
\end{align}
was sinngem\"{a}{\ss} $\vek J(\vek u) = \vek j^T\,\vek u = \vek g^T\,\vek f$ entspricht. Die L\"{o}sung $g(x)$ von (\ref{Eq1}) ist also die Greensche Funktion des Funktionals $J(u)$.

Nun betrachten wir das {\em Anfangswertproblem\/}
\begin{align}
u'(t) = f(t) \qquad u(0) = 0
\end{align}
zu dem die Identit\"{a}t, ($T > 0$ beliebig),
\begin{align}
\text{\normalfont\calligra B\,\,}(u,v) &= \int_0^{\,T} u'(t)\,v(t)\,dt  - [u\,v]_{@0}^{@T} + \int_0^{\,T} u(t)\,v'(t) \,dt = 0
\end{align}
geh\"{o}rt. Der Operator $d/dt$ ist, wie man sieht, nicht selbstadjungiert, denn $(u',v) = (u,-v')$ (ohne Randwerte).

Ist
\begin{align}
J(u) = \int_0^{\,T} j(t)\,u(t)\,dt
\end{align}
ein lineares Funktional und ist $g(t)$ die L\"{o}sung des adjungierten Problems
\begin{align}
-g'(t) = j(t) \qquad g(T) = 0\,,
\end{align}
dann liefert partielle Integration die Darstellung
\begin{align}
J(u) &= \int_0^{\,T} j(t)\,u(t)\,dt = [g\,u]_0^T + \int_0^{\,T} g(t)\,u'(t)\,dt = \int_0^{\,T} g(t)\,f(t)\,dt\,.
\end{align}
Partielle Integration hat zwei Grenzen, $(0,l)$ oder $(0,T)$. Bei Randwertproblemen werden beide Grenzen bedient, bei Anfangswertproblemen dagegen nur die untere und deswegen muss man die Greensche Funktion bei solchen Problemen am {\em oberen Ende\/} $T$ festmachen, $g(T) = 0$. Die Greensche Funktion l\"{a}uft, wenn man so will, \glq r\"{u}ckw\"{a}rts\grq{}.

Wenn man einen Berg hinaufschaut, dann ist die Steigung positiv. Wenn man, oben angekommen ($t = T$), zur\"{u}ckschaut, dann ist die Steigung negativ. Das mag als Erkl\"{a}rung f\"{u}r den Wechsel im Vorzeichen
\begin{align}
 u'(t) = f(t) \qquad \qquad -g'(t) = j(t)
\end{align}
dienen.\\
\begin{flushleft}{\em Beispiel\/} \end{flushleft} Gegeben sei das Anfangs\-wert\-problem
\begin{align}
u' = 1- t \qquad u(0) = 0
\end{align}
und $J(u)$ sei das Integral der L\"{o}sung $u(t) = t - 0.5\,t^2$
\begin{align} \label{Eq2}
J(u) = \int_0^{\,T} 1 \cdot u(t)\,dt =  \frac{T^2}{2} - \frac{T^3}{6} \qquad j(t) = 1\,.
\end{align}
Die adjungierte Gleichung
\begin{align}
- g' = 1 \qquad g(T) = 0
\end{align}
hat die L\"{o}sung $g(t) = T - t$ und das Skalarprodukt von $g(t)$ und $f(t) $ ist genau der Wert in (\ref{Eq2})
\begin{align}
J(u) = \int_0^{\,T} g(t) f(t) dt = \int_0^{\,T}(T - t) \cdot  (1 - t) \,dt = \frac{T^2}{2} - \frac{T^3}{6}\,.
\end{align}


%%%%%%%%%%%%%%%%%%%%%%%%%%%%%%%%%%%%%%%%%%%%%%%%%%%%%%%%%%%%%%%%%%%%%%%%%%%%%%%%%%%%%%%%%%%%%%%%%%%
\textcolor{sectionTitleBlue}{\section{Gateaux Ableitung}}
Bei nichtlinearen Problemen tritt eine neue Ableitung auf, die sogenannte {\em Gateaux Ableitung\/}. Sei
\begin{align}
J(u) = \int_0^{\,l} F(u)\,dx
\end{align}
ein (m\"{o}glicherweise nichtlineares) Funktional, dann bezeichnet man den Ausdruck
\begin{align}
J_{u}(\delta u) = \frac{d}{d\varepsilon} J(u + \varepsilon \delta u) _{|_{\varepsilon = 0}}
\end{align}
als die Gateaux Ableitung von $J(u)$ in Richtung des Inkrements $\delta u$.

Man bildet mit einer Testfunktion $\delta u$ (virtuellen Verr\"{u}ckung) den Ausdruck $J(u + \varepsilon \delta u)$, differenziert nach $\varepsilon$, und setzt am Schluss den Faktor $\varepsilon = 0$.

Diese Ableitung sieht ganz wie ein Notbehelf aus, wenn man etwas nicht richtig differenzieren kann, dann ersetzt man es durch einen Differenzenquotient.

\"{U}berraschenderweise erscheint diese Ableitung jedoch {\em automatisch\/} in vielen nichtlinearen Formulierungen, wie zum Beispiel der Greenschen Identit\"{a}t der nichtlinearen Elastizit\"{a}tstheorie
\beq\label{E4Gfb}
\text{\normalfont\calligra G\,\,}(\vek u, \vek \delta \vek u) = \int_{\Omega} \vek p \dotprod \vek \delta \vek u\,d\Omega +
\int_{\Gamma_N}  \bar{\vek t} \dotprod \vek \delta \vek u\,ds - \int_{\Omega} \vek E_{\vek
u}(\vek \delta \vek u) \dotprod \vek S\,d\Omega = 0\,,
\eeq
wo
\beq
\vek E_{\vek u}(\vek \delta \vek u) := \frac{1}{2}\,(\nabla\vek \delta \vek u + \nabla\vek \delta \vek u^T
+ \nabla \vek u^T\,\nabla\,\vek \delta \vek u + \nabla\,\vek \delta \vek u^T\,\nabla\,\vek u)
\eeq
die Gateaux Ableitung \index{Gateaux Ableitung} des Verzerrungstensors $\vek E(\vek u)$ ist,
\beq
\frac{d}{d\varepsilon} [\vek E(\vek u + \varepsilon\,\vek \delta \vek u)]_{|_{\varepsilon = 0}}
=\vek E_{\vek u}(\vek \delta \vek u)\,.
\eeq
Was -- auf den ersten Blick -- wie ein Trick aussieht, ist also ein wesentlicher Bestandteil der Integralformulierungen nichtlinearer Probleme.

Bei nichtlinearen Problemen tritt in der ersten Greenschen Identit\"{a}t an die Stelle des symmetrischen Integrals
\begin{align}
a(\vek u, \vek \delta \vek u) = \int_{\Omega} \vek E(\vek \delta \vek u) \dotprod \vek S(\vek u)\,d\Omega  = \int_{\Omega} \vek E(\vek u) \dotprod \vek S(\vek  \delta\vek u)\,d\Omega  = a(\vek \delta \vek  u, \vek u) \,,
\end{align}
-- symmetrisch wegen\footnote{$\vek S(\vek u) = \vek C[\vek E(\vek u)]$ ist der Spannungstensor des Feldes $\vek u$, beim Stab ist $C[\varepsilon(u)] = EA u'$}
\begin{align}
\vek E(\vek \delta \vek u) \dotprod \vek S(\vek u) = \vek E(\vek \delta \vek u) \dotprod \vek C[\vek E(\vek u)] = \vek E( \vek u) \dotprod \vek C[\vek E(\vek \delta \vek u)] = \vek E(\vek u) \dotprod \vek S(\vek \delta \vek u)
\end{align}
das unsymmetrische Integral
\begin{align}
a(\vek u, \vek \delta \vek u) =\int_{\Omega} \vek E_{\vek
u}(\vek \delta \vek u) \dotprod \vek S(\vek u)\,d\Omega \,,
\end{align}
das sozusagen den Zuwachs an innerer Energie beschreibt, wenn $\vek u$ sich in Richtung $\vek u + \vek \delta \vek u$ entwickelt.


%----------------------------------------------------------------------------------------------------------
\begin{figure}[tbp]
\centering
\if \bild 2 \sidecaption \fi
\includegraphics[width=.8\textwidth]{\Fpath/U453a}
\caption{Die Statik des Seils} \label{U453}
\end{figure}%
%----------------------------------------------------------------------------------------------------------

Bei nichtlinearen Problemen ist die erste Greensche Identit\"{a}t so etwas, wie eine \glq Inkrement-Betrachtung\grq{} der Null-Summe $ \delta A_a - \delta A_i = 0$.

%%%%%%%%%%%%%%%%%%%%%%%%%%%%%%%%%%%%%%%%%%%%%%%%%%%%%%%%%%%%%%%%%%%%%%%%%%%%%%%%%%%%%%%%%%%%%%%%%%%
\textcolor{sectionTitleBlue}{\section{Das Seil}}\index{Seil}\label{Seilstatik}
Wir beschr\"{a}nken uns hier auf den Fall des urspr\"{u}nglich geraden Seils, das mit einer Kraft $H$ vorgespannt wird. Die Seilkraft $S$ und die vertikale Kraft $V$ und der Horizontalzug $H$ bilden ein rechtwinkliges Dreieck, $V/H = \tan\,\Np = w'$, s. Abb. \ref{U453}. Aus dem Gleichgewicht am Element, $V + \Delta V - V + p\,\Delta x = 0$ folgt $-V' = p$ und das ergibt zusammen mit $V = H\,w'$ die Seilgleichung $-H\,w'' = p$.

Soll unter Gleichlast $p$ der Durchhang des Seils auf einen Wert $f$ beschr\"{a}nkt sein, dann muss die Zugkraft den Wert
\begin{align}
H = \frac{p\,l^2}{8\,f}
\end{align}
haben und die L\"{a}nge $L$ des Seils darf nicht gr\"{o}{\ss}er sein als
\begin{align}
L = \int_0^{\,l} ds \simeq l \cdot (1 + \frac{8\,f^2}{3\,l^2})\,.
\end{align}

%%%%%%%%%%%%%%%%%%%%%%%%%%%%%%%%%%%%%%%%%%%%%%%%%%%%%%%%%%%%%%%%%%%%%%%%%%%%%%%%%%%%%%%%%%%%%%%%%%%
\textcolor{sectionTitleBlue}{\section{Der vollst\"{a}ndige Balken (Bernoulli-Balken)}}\index{Balken, vollst\"{a}ndig}\index{Bernoulli-Balken}
Die Differentialgleichung $EI\,w^{IV}(x) = p(x)$ basiert auf dem System
\begin{subequations}\label{Eq139}
\begin{alignat}{3}
\hspace{-2cm} \mbox{Kr\"{u}mmungen}\qquad && \kappa - w'' &= \kappa_0  \\
\hspace{-2cm} \mbox{Materialgesetz}\qquad &&EI\,\kappa + M  &= M_0 \\
\hspace{-2cm} \mbox{Gleichgewicht}\qquad&&-M'' &= m' + p&
\end{alignat}
\end{subequations}
mit Vorkr\"{u}mmungen $\kappa_0$, z.B. aus Temperatur $\kappa_0 = \alpha_T \Delta T/h$, eingepr\"{a}gten Momenten $M_0$, Linienmomenten $m$ [kNm/m] und der Streckenlast $p$. Wenn $EI$ konstant ist, die Vorkr\"{u}mmungen null sind und keine Linienmomente wirken, $m = 0$, dann reduziert sich das System auf $EI\,w^{IV} = p$.
Dieser \glq Dreier-Schritt\grq{} wiederholt sich im folgenden und kann sinngem\"{a}{\ss} auch f\"{u}r die anderen Differentialgleichungen wie $-EA\,u'' = p_x$, $- GA\,w_s'' = p_z$ etc. formuliert werden. Man braucht ja nur die Scheiben- bzw. Plattengleichung als Vorbild nehmen. Die {\em gemischten Verfahren\/} \index{gemischte Verfahren} basieren auf diesen Formulierungen.

Vordehnungen und Vorkr\"{u}mmungen, bei Fl\"{a}chentragwerken sind das Tensoren $\vek E_0$ und $\vek K_0$, haben ihre Ursache meist in Temperatur\"{a}nderungen oder \"{a}hnlichen Effekten und lassen sich mit dem \glq Dreier-Schritt\grq{} verfolgen.

%%%%%%%%%%%%%%%%%%%%%%%%%%%%%%%%%%%%%%%%%%%%%%%%%%%%%%%%%%%%%%%%%%%%%%%%%%%%%%%%%%%%%%%%%%%%%%%%%%%
\textcolor{sectionTitleBlue}{\subsection{Die zugeh\"{o}rigen Identit\"{a}ten}}\label{Korrektur40}\label{TempIdentit}
Die Effekte von Temperatur\"{a}nderungen werden in der Mohrschen Arbeitsgleichung von zwei einfachen Zusatztermen erfasst
\begin{align}
\textcolor{red}{1} \cdot \delta = \ldots \int \textcolor{red}{\bar{M}}\,\alpha_T\,\frac{\Delta T}{h}\,dx + \int \textcolor{red}{\bar{N}}\,\alpha_T\,T\,dx
\end{align}
und so scheint es, dass sich Einflussfunktionen f\"{u}r den Lastfall Temperatur nahtlos aus den Gleichungen $EI\,w^{IV} = p_z$ und $-EA\,u'' = p_x$ entwickeln lassen. Das ist aber nicht richtig, sondern man muss dazu \"{u}ber das obige System  (\ref{Eq139}) gehen. Wie man dabei vorgeht, soll hier f\"{u}r den Balkenanteil gezeigt werden. Die Herleitung f\"{u}r den Stabanteil findet man in \cite{Ha5} S. 399.
%----------------------------------------------------------------------------------------------------------
\begin{figure}[tbp]
\centering
\if \bild 2 \sidecaption \fi
\includegraphics[width=0.9\textwidth]{\Fpath/U433}
\caption{Lastfall Temperatur bei einem Balken } \label{U433}
\end{figure}%
%----------------------------------------------------------------------------------------------------------

Zun\"{a}chst ben\"{o}tigen wir die zweite Greensche Identit\"{a}t des obigen Systems, das wir als die Anwendung eines Operators $\text{\normalfont\calligra A\,\,}(\vek S)$ auf das Tripel $\vek S = \{w, \kappa, M\}$ aus {\em Biegelinie, Kr\"{u}mmung\/} und {\em Moment\/} lesen k\"{o}nnen. Ist $ \vek \delta \vek S = \{\delta w, \delta \kappa, \delta M\} $ eine beliebige Testfunktion (virtuelle Verr\"{u}ckung), dann ist die virtuelle \"{a}u{\ss}ere Arbeit der Ausdruck
\begin{align}
<\text{\normalfont\calligra \!A\,\,}(\vek S), \vek  \delta \vek S\!> := \int_0^{\,l} [(\kappa - w'')\,\delta M + (EI\,\kappa + M)\,\delta \kappa - M''\,\delta w]\,dx\,.
\end{align}
Partielle Integration f\"{u}hrt auf die erste Greensche Identit\"{a}t
\begin{align}
\text{\normalfont\calligra G\,\,}(\vek S, \vek \delta \vek S)
&= <\text{\normalfont\calligra \!A\,\,}(\vek S), \vek  \delta\vek S\!>
 + [w'\,\delta M + M'\,\delta w]_0^l + a(\vek S, \vek  \delta \vek S) = 0
\end{align}
mit der symmetrischen Wechselwirkungsenergie
\begin{align}
a(\vek S, \vek  \delta \vek S) = \int_0^{\,l} [EI\,\kappa\, \delta \kappa + \kappa \,\delta M + M\,\delta \kappa + w'\,\delta M' + M'\,\delta w']\,dx
\end{align}
und dann direkt weiter zur zweiten Greenschen Identit\"{a}t
\begin{align}
\text{\normalfont\calligra B\,\,}(\vek S, \vek \delta \vek S)
&= <\text{\normalfont\calligra \!A\,\,}(\vek S), \vek  \delta\vek S\!>
 + [w'\,\delta M + M'\,\delta w]_0^l\nn \\
  &- [\delta w'\, M + \delta M'\, w]_0^l - <\vek S,\text{\normalfont\calligra \!A\,\,}(\vek  \delta \vek S)\!> = 0\,.
\end{align}
In einem Lastfall $\Delta T$ stehen auf der rechten Seite des Systems (\ref{Eq139}) die drei Funktionen $\{\alpha_T\,\Delta T/h, 0, 0\}$ und die \"{a}u{\ss}ere virtuelle Arbeit im Feld reduziert sich daher auf das Integral
\begin{align}
<\text{\normalfont\calligra \!A\,\,}(\vek S), \vek  \delta \vek S\!> = \int_0^{\,l} \alpha_T\,\frac{\Delta T}{h} \,\delta M\,dx\,.
\end{align}
Die Einflussfunktion f\"{u}r das Moment $M(x)$ in einem Punkt $x$ entsteht durch die Spreizung eines dort eingebauten Gelenks. Sei $\vek S_2 = \{w_2, M_2, \kappa_2\}$ die Einflussfunktion (in drei Teilen), also $M_2$ das zugeh\"{o}rige Biegemoment, dann folgt nach den \"{u}blichen Schritten (Zweiteilung des Felds im Aufpunkt)
\begin{align}\label{Eq140}
\text{\normalfont\calligra B\,\,}(\vek S_2, \vek S)
&= - M(x) + \int_0^{\,l} \alpha_T\,\frac{\Delta T}{h}\,M_2\,dx = 0
\end{align}
was genau der Mohrschen Arbeitsgleichung entspricht, wenn man $\bar{M} = M_2$ setzt. In Temperaturlastf\"{a}llen kann man also mit der Mohrschen Arbeitsgleichung Schnittkr\"{a}fte berechnen, was ja eigentlich nicht gehen sollte. Es geht, weil der Ausdruck (\ref{Eq140}) eine starke Einflussfunktion ist, die bei der Mohrschen Arbeitsgleichung als (angebliche) schwache Einflussfunktion \glq mitsegelt\grq{}.

Im Temperaturlastfall in Abb. \ref{U433} ist die rechte Seite des Systems $\{\kappa_0,0,0\}$. Die Lagerbedingungen verlangen einen Ansatz $w(x) = c\,(x^3 - x^2)$, was auf $\kappa = \kappa_0 + c\,(6\,x - 2)$ und $M(x) = - EI\,(\kappa_0 + c\,(6\,x - 2))$ f\"{u}hrt. Aus $M(1) = 0$ ergibt sich $c = - \kappa_0/4$ und damit die Verl\"{a}ufe in Abb. \ref{U433}.

%%%%%%%%%%%%%%%%%%%%%%%%%%%%%%%%%%%%%%%%%%%%%%%%%%%%%%%%%%%%%%%%%%%%%%%%%%%%%%%%%%%%%%%%%%%%%%%%%%%
\textcolor{sectionTitleBlue}{\section{Der schubweiche Balken (Timoshenko Balken)}}\index{schubweicher Balken}\index{Timoshenko Balken}
%------------------------------------------------------------------
\begin{figure}[tbp]
\centering
\if \bild 2 \sidecaption \fi
\includegraphics[width=0.4\textwidth]{\Fpath/U238}
\caption{Schubweicher Balken} \label{WinkelAngle}
\end{figure}%%
%------------------------------------------------------------------

Ein schubweicher Balken bildet unter einer Einzelkraft einen Knick aus, und der Balken kann auch mit einem Knick aus der Wand herauslaufen, was f\"{u}r einen schubstarren Balken unm\"{o}glich w\"{a}re. Solche Balken \"{a}hneln also von der Form her vorgespannten Seilen, nur dass sie, anders als Seile, eine Biegesteifigkeit haben und sich damit Biegemomente ausbilden k\"{o}nnen.

Wie \"{u}blich setzen wir voraus, dass die Biegesteifigkeit $EI$, der Schubquerschnitt  $A_s$
und der Schubmodul $G$ l\"{a}ngs des Balkens konstant sind.

Die Kinematen sind die Durchbiegung $w$ und die Verdrehung $\theta$ (s. Abb. \ref{WinkelAngle}).

Das grundlegende System besteht aus den Gleichungen
\begin{subequations}
\begin{alignat}{3}
\hspace{-2cm} \mbox{Verzerrungen}\qquad && \theta' - \kappa &= 0 & \qquad w' + \theta - \gamma &= 0 \\
\hspace{-2cm} \mbox{Materialgesetz}\qquad &&GA_s \gamma - V &= 0 & \qquad EI\,\kappa - M &= 0 \\
\hspace{-2cm} \mbox{Gleichgewicht}\qquad&&M' - V &= 0& \qquad - V' &= p
\end{alignat}
\end{subequations}
oder
\begin{subequations}
\bfo\label{TimoshenkoDGL}
- EI\,\theta'' + GA_s\,(w' + \theta) &=& 0\\
\label{TimoshenkoDGL2} - GA_s\,(w'' + \theta') &=& p \,.
\efo
\end{subequations}
Das System kann als die Anwendung eines Operators $- \vek L$ auf die vektorwertige Funktion $\vek u = \{w, \theta\}^T$ gelesen werden. Die zugeh\"{o}rige erste Greensche Identit\"{a}t lautet
\bfo
\text{\normalfont\calligra G\,\,}(\vek u,\vek \delta \vek u) = \!\!\int_0^{\,l}\!\!\! - \vek L\,\vek u \dotprod \vek \delta \vek u\,dx +
\left[V\,\delta w+ M\,\delta \theta\right]_{@0}^{@l} - a(\vek u,\vek \delta \vek u) = 0\,,
\efo
wobei
\bfo
a(\vek u,\vek \delta \vek u) &=& \int_0^{\,l} [V\,\delta \gamma + M\,\delta \kappa ]\,dx \nn \\
&=& \int_0^{\,l} [GA_s(w' + \theta)\,(\delta w' + \delta \theta) +
EI\,\theta'\,\delta \theta' \, ]\,dx
\efo
die Wechselwirkungsenergie ist.

%-----------------------------------------------------------------
\begin{figure}[tbp]
\centering
\if \bild 2 \sidecaption \fi
\includegraphics[width=0.9\textwidth]{\Fpath/U272}
\caption{Der \"{U}bergang vom Seil zum Segeltuch, \cite{Int1}} \label{U272}
\end{figure}%%
%-----------------------------------------------------------------
%%%%%%%%%%%%%%%%%%%%%%%%%%%%%%%%%%%%%%%%%%%%%%%%%%%%%%%%%%%%%%%%%%%%%%%%%%%%%%%%%%%%%%%%%%%%%%%%%%%
\textcolor{sectionTitleBlue}{\section{Poisson Gleichung}}
Die {\em Poisson Gleichung\/} beschreibt u.a. die Durchbiegung $u(\vek x)$ einer vorgespannten Membran unter Winddruck $p$, s. Abb. \ref{U272},
\begin{align}\label{Eq66}
- \Delta u(\vek x) = p(\vek x) \qquad u = 0\quad \,\,\text{auf dem Rand $\Gamma$}\,.
\end{align}
Zu ihr geh\"{o}rt die Identit\"{a}t
\begin{align}
\text{\normalfont\calligra G\,\,}(u,\delta u) = \int_{\Omega} - \Delta u(\vek x)\,\delta u(\vek x)\,d\Omega + \int_{\Gamma} \frac{\partial u(\vek x)}{\partial n}\,\delta u(\vek x)\,ds - a(u,\delta u) = 0
\end{align}
mit der Wechselwirkungsenergie
\begin{align}
a(u,\delta u) = \int_{\Omega} \nabla u(\vek x) \dotprod \nabla\,\delta u(\vek x)\,d\Omega = \int_{\Omega} (u,_{x_1}\,\delta u,_{x_1} + u,_{x_2}\,\delta u,_{x_2})\,d\Omega\,.
\end{align}
Die Bedeutung der Poisson Gleichung beruht darauf, dass man mit ihren Fundamentall\"{o}sungen
\begin{align}
 g(\vek y, \vek x) = - \frac{1}{2\,\pi}\ln r \qquad \text{(2-$D$)} \qquad g(\vek y, \vek x) = \frac{1}{4\,\pi}\frac{1}{r} \qquad \text{ (3-$D$)} \qquad
\end{align}
jede $C^2$-Funktionen \"{u}ber einem Gebiet $\Omega$ aus den Randwerten $u $ und $\partial u/\partial n $ und den zweiten Ableitungen $\Delta u = u,_{x_1 x_1} + u,_{x_2 x_2}$ berechnen kann
\begin{align}\label{Eq175}
u(\vek x) = &\int_{\Gamma} [g(\vek y, \vek x) \,\frac{\partial u(\vek y)}{\partial n} - \frac{\partial g(\vek y, \vek x)}{\partial n}\,u(\vek y)]\,ds_{\vek y} + \int_{\Omega} g(\vek y, \vek x)\,(- \Delta u(\vek y))\,\,d\Omega_{\vek y}\,.
\end{align}
Diese Gleichung ist praktisch die Erweiterung der Gleichung (partielle Integration)
\begin{align}
w(x) &= w(0) + \int_{0}^{x} w'(y)\,dy = \int_{\Gamma} \ldots + \int_{\Omega} \ldots \,,
\end{align}
auf h\"{o}here Dimensionen und sie ist der eigentliche Hauptsatz der Differential-und Integralrechnung\index{Hauptsatz der Differential- und Integralrechnung}. {\em Eine Fl\"{a}che ist durch ihre \glq Spur\grq{} auf dem Rand, $u$ und $\partial u/\partial n$, und ihre \glq Kr\"{u}mmung\grq{} $\Delta u$ eindeutig bestimmt\/}.

Die L\"{o}sungen $- \Delta u(\vek x) = 0$ nennt man {\em harmonische Funktionen\/}\index{harmonische Funktion}. Der Wert in einem Punkt $\vek x $ ist der Mittelwert  aus den Nachbarn im Norden, Osten, S\"{u}den und Westen. Eine Gerade, $- u'' = 0$, ist z.B. eine harmonische Funktion.

Harmonische Funktionen kann man also allein aus ihren Randwerten berechnen. Umgekehrt folgt daraus aber auch,  dass das Integral der Normal\-ab\-leitung einer harmonischen Funktion \"{u}ber den Rand null sein muss, denn
\begin{align}
\text{\normalfont\calligra G\,\,}(u,1) = \int_{\Omega} - \Delta u(\vek x)\cdot 1\,d\Omega + \int_{\Gamma} \frac{\partial u(\vek x)}{\partial n} \cdot 1\,ds =  \int_{\Gamma} \frac{\partial u(\vek x)}{\partial n} \cdot 1\,ds = 0\,.
\end{align}
Ist es nicht null, dann ist $u$ keine harmonische Funktion und dann muss man, damit die Integraldarstellung richtig bleibt, eine Belastung $p$ als \glq Gegengewicht\grq{} so w\"{a}hlen, dass das Integral von $p = - \Delta u$ \"{u}ber $\Omega$ gerade das Integral der Normalableitung ist -- das ist einfach die Gleichgewichtsbedingung. Man kann auch nicht einfach zwei Randfunktionen $a = u, b = \partial u/\partial n$ und eine dritte Funktion $c = - \Delta u$ frei w\"{a}hlen und dann glauben, dass die damit konstruierte Funktion (\ref{Eq175}) diese Werte annimmt. Das geht nur gut, wenn die drei Funktionen die Integralgleichung (\ref{Eq175}), setze f\"{u}r $\vek x$ die Punkte auf dem Rand, erf\"{u}llen\footnote{Im Unterschied dazu stammen in (\ref{Eq175}) ja alle Funktionen von demselben $u$}.

Das ist dieselbe Logik, wie beim Zugstab, wo man ja auch nicht die L\"{a}ngsverschiebung $u $ und die Kraft $P$ unabh\"{a}ngig voneinander w\"{a}hlen kann, sondern die beiden Zahlen m\"{u}ssen zueinander passen, m\"{u}ssen eben der \glq Integralgleichung\grq{} $u = P\,l/EA$ gen\"{u}gen.

Die Poisson Gleichung kann in ein System erster Ordnung
\begin{subequations}
\begin{align}
\nabla w - \vek \sigma &= \vek 0_{\,(2)}  \\
- \text{div} \,\vek \sigma &= p_{\,(1)}
\end{align}
\end{subequations}
f\"{u}r zwei Funktionen, $u$ und $\vek \sigma$ oder $\vek v = \{u, \vek \sigma\}^T$ aufgespalten werden.

Zu diesem System geh\"{o}rt die Identit\"{a}t
\begin{align}
 \text{\normalfont\calligra G\,\,}(\vek v,\vek  \delta \vek v) &= \int_{\Omega} \left[(\nabla w - \vek \sigma) \dotprod \vek \delta \vek \sigma - \text{div} \,\vek \sigma\, \delta w\right]\,d\Omega
+ \int_{\Gamma} \vek \sigma \dotprod \vek n\,\delta w\,ds \nn \\
&- \underbrace{\int_{\Omega} (\nabla w \dotprod \vek  \delta \vek \sigma + \nabla \delta w \dotprod \vek \sigma\,- \vek \sigma \dotprod \vek  \delta \vek \sigma) \,d\Omega}_{a(\vek v, \vek  \delta \vek v)} = 0\,.
\end{align}

%----------------------------------------------------------------------------------------------------------
\begin{figure}[tbp] %415
\centering
\if \bild 2 \sidecaption \fi
\includegraphics[width=0.99\textwidth]{\Fpath/U439}
\caption{Gelochte Scheibe unter Zug} \label{U439}
%
\end{figure}%
%----------------------------------------------------------------------------------------------------------

%%%%%%%%%%%%%%%%%%%%%%%%%%%%%%%%%%%%%%%%%%%%%%%%%%%%%%%%%%%%%%%%%%%%%%%%%%%%%%%%%%%%%%%%%%%%%%%%%%%
\textcolor{sectionTitleBlue}{\section{Die Scheibengleichung}}
Die konstitutiven Gleichungen lauten in der Reihenfolge {\em Verzerrungen, Spannungen, Gleichgewicht\/}
\begin{subequations}\label{Eq54}
\begin{align}
\vek E(\vek u) - \vek E &= \vek E_0 \\
\vek C[\vek E] - \vek S &= \vek 0 \\
- \text{div}\,\vek S &= \vek p\,,
\end{align}
\end{subequations}
wobei $\vek E = [\varepsilon_{ij}]$ und $\vek S = [\sigma_{ij}]$ der Verzerrungs- bzw. Spannungstensor sind, $\vek  E_0$ sind Anfangsdehnungen (z.B. aus Temperatur) und $\vek E()$ ist der Operator
\begin{align}\label{Eq48}
\vek E(\vek u) = \frac{1}{2}\,(\nabla \vek u + \nabla \vek u^T) = \frac{1}{2}\,\left[ \barr {c @{\hspace{4mm}}c }
      2 \,u_1,_1 & u_1,_2 + u_2,_1\\
      u_2,_1 + u_1,_2 & 2\,u_2,_2
    \earr \right]\,.
\end{align}
Mit dem Operator
\begin{align}\label{Eq60}
\vek C[\vek E] = 2\mu\cdot\vek E + \lambda\,(\text{tr}\,\vek E)\cdot\vek I
\end{align}
wird der Spannungstensor $\vek S$ aus dem Verzerrungstensor berechnet. Es ist
\begin{align}
\lambda = \frac{2\,\mu\,\nu}{1 - 2\,\nu} \qquad \text{tr}\,\vek E = \varepsilon_{11} + \varepsilon_{22} \qquad \text{(trace)}
\end{align}
und $\vek I$ ist der Einheitstensor  (Einheitsmatrix).

Die linke Seite des Systems (\ref{Eq54}) kann man als die Anwendung eines Operators $\vek A(\, )$ auf das Triple $\vek \Sigma = \{\vek u, \vek E, \vek S\}$ lesen und
\begin{align}
<\vek A(\vek \Sigma), \vek  \delta \vek \Sigma> &= \int_{\Omega} (\vek E(\vek u) - \vek E) \dotprod \vek  \delta \vek S \,d\Omega + \int_{\Omega} (\vek C[\vek E] - \vek S)\dotprod  \vek  \delta E \,d\Omega \nn \\
&+ \int_{\Omega} - \text{div}\,\vek S \dotprod  \vek  \delta \vek  u\,d\Omega
\end{align}
ist dann die zugeh\"{o}rige virtuelle Arbeit mit  $\vek  \delta \vek \Sigma = \{\vek \delta \vek u, \vek  \delta \vek E, \vek  \delta \vek S\}$ als \glq virtueller Verr\"{u}ckung\grq{} oder  \glq Testfeld\grq{}.

Der Punkt $\dotprod $ bezeichnet hier das Skalarprodukt von zwei Matrizen
\begin{align}
\vek S \dotprod  \vek E = \sigma_{11}\,\varepsilon_{11} + \sigma_{12}\,\varepsilon_{12} + \sigma_{21}\,\varepsilon_{21} + \sigma_{22}\,\varepsilon_{22}\,.
\end{align}
Ist $\vek S \in C^1(\Omega)$ eine symmetrische Matrix und $\vek \delta \vek u \in C^1(\Omega) $ ein beliebiges Verschiebungsfeld, dann gilt wegen den Regeln der partiellen Integration
\begin{align}
\int_{\Omega} - \text{div}\,\vek S \dotprod \vek \delta \vek u \,d\Omega = - \int_{\Gamma} \vek S\,\vek n \dotprod \vek \delta \vek u \,ds + \int_{\Omega} \vek S \dotprod \vek E(\vek \delta \vek u)\,d\Omega
\end{align}
und mit diesem Hilfssatz folgt
\begin{align}
<\vek A(\vek \Sigma), \vek  \delta \vek \Sigma> &=\int_{\Omega} (\vek E(\vek u) - \vek E) \dotprod \vek  \delta \vek S \,d\Omega + \int_{\Omega} (\vek C[\vek E] - \vek S)\dotprod  \vek \delta \vek E \,d\Omega \nn \\
&+ \int_{\Omega} \vek S \dotprod \vek E(\vek \delta \vek u) \,d\Omega - \int_{\Gamma} \vek S\,\vek n \dotprod  \vek \delta \vek u \,ds\,.
\end{align}
Die drei Gebietsintegrale bilden wegen der Symmetrie $\vek C[\vek E] \dotprod  \vek  \delta \vek E = \vek E \dotprod \vek  C[\vek  \delta \vek E]$ eine symmetrische Bilinearform
\begin{align}
a(\vek \Sigma, \vek  \delta \vek \Sigma) :&= \int_{\Omega} (\vek E(\vek u) - \vek E) \dotprod \vek  \delta\vek S \,d\Omega + \int_{\Omega}\vek C[\vek E] \dotprod  \vek  \delta\vek E\,d\Omega \nn \\
&+ \int_{\Omega} \vek S \dotprod (\vek E(\vek \delta \vek u) - \vek  \delta\vek E) \,d\Omega\,,
\end{align}
die wir die Wechselwirkungsenergie zwischen $\vek \Sigma$ und $\vek  \delta \vek \Sigma $ nennen.

Damit lautet die erste Greensche Identit\"{a}t des Operators $\vek A(\vek \Sigma)$
\begin{align}
\text{\normalfont\calligra G\,\,}(\vek \Sigma,\vek  \delta \vek \Sigma) = <\vek A(\vek \Sigma), \vek  \delta \vek \Sigma> + \int_{\Gamma}\vek S\,\vek n \dotprod  \vek \delta \vek u\,ds - a(\vek \Sigma, \vek  \delta\vek \Sigma) = 0\,,
\end{align}
aus der alles weitere, insbesondere auch der {\em Satz von Betti\/} und das {\em Hu-Washizu-Prinzip\/}\index{Hu-Washizu-Prinzip} folgt, \cite{Ha1}.

Das System $\vek A(\vek \Sigma)$ f\"{u}r das Tripel $\vek \Sigma = \{\vek u, \vek E, \vek S\}$ kann man nun auf ein System f\"{u}r das Verschiebungsfeld $\vek u$ allein reduzieren, indem man die Gleichungen (\ref{Eq4}) ineinander einsetzt ($\vek S_0 = \vek C[\vek E_0]$)
\begin{align}
- \vek L\,\vek u := - [\mu\,\Delta \vek u + \frac{\mu}{1 - 2\,\nu} \,\nabla\,\text{div}\,\vek u ]= \vek p - \text{div} \,\vek S_0\,.
\end{align}
Zu dem Operator geh\"{o}rt die Identit\"{a}t
\begin{align}
\text{\normalfont\calligra G\,\,}(\vek u,\vek  \delta u) &= \underbrace{\int_{\Omega}- \vek L\,\vek u \dotprod  \vek \delta \vek u\,d\Omega + \int_{\Gamma} \vek \tau(\vek u) \dotprod \vek \delta \vek u\,ds}_{\delta A_a}\nn \\
 &- \underbrace{\int_{\Omega} \vek E(\vek u) \dotprod \vek C[\vek E(\vek  \delta\vek u)] \,d\Omega}_{\delta A_i} = 0\,,
\end{align}
wobei $\vek \tau(\vek u)$ der Spannungsvektor $\vek S\,\vek n$ des Feldes $\vek u$ auf dem Rand $\Gamma$ ist.

Bei einer wie folgt belasteten Scheibe mit Rand $\Gamma = \Gamma_D \cup \Gamma_N$
\begin{align}
- \vek L\,\vek  u = \vek p \qquad \vek \tau(\vek u) = \bar{\vek t} \,\,\text{auf $\Gamma_N$} \qquad \vek u = \vek 0 \,\,\text{auf $\Gamma_D$}
\end{align}
lautet also das {\em Prinzip der virtuellen Verr\"{u}ckungen\/}, wenn $\vek \delta \vek u = \vek 0$ auf $\Gamma_D$ ist,
\begin{align}
\text{\normalfont\calligra G\,\,}(\vek u,\vek  \delta \vek u) = \underbrace{\int_{\Omega} \vek p \dotprod  \vek  \delta \vek u\,d\Omega + \int_{\Gamma_N} \bar{\vek t} \dotprod \vek  \delta \vek u\,ds}_{\delta A_a} - \underbrace{\int_{\Omega} \vek E(\vek u) \dotprod \vek C[\vek E(\vek \delta \vek  u] \,d\Omega}_{\delta A_i} = 0\,.
\end{align}
Mit Anfangsdehnungen $\vek E_0$ ist $\delta A_a$ um das Gebietsintegral $(- \text{div}\,\vek S_0 \dotprod$ \vek  \delta \vek u, 1) zu erweitern.

Die Elemente der Steifigkeitsmatrix $\vek K$ einer Scheibe sind die Wechselwirkungsenergien zwischen den Knotenverschiebungen $\vek \Np_i$ und $\vek \Np_j$, die ja selbst Verschiebungsfelder sind, also aus horizontalen und vertikalen Komponenten bestehen
\begin{align}
k_{ij} = \int_{\Omega} \vek E(\vek \Np_i) \dotprod \vek C[\vek E(\vek \Np_j] \,d\Omega  =\int_{\Omega} (\sigma_{xx}^{(i)}\,\varepsilon_{xx}^{(j)} + 2\,\sigma_{xy}^{(i)}\,\varepsilon_{xy}^{(j)} + \sigma_{yy}^{(i)}\,\varepsilon_{yy}^{(j)})\,d\Omega\,.
\end{align}

%%%%%%%%%%%%%%%%%%%%%%%%%%%%%%%%%%%%%%%%%%%%%%%%%%%%%%%%%%%%%%%%%%%%%%%%%%%%%%%%%%%%%%%%%%%%%%%%%%%
\textcolor{sectionTitleBlue}{\section{Die schubstarre Platte (Kirchhoff)}}\index{Kirchhoffplatte}
Bei einer schubstarren Platte (Kirchhoffplatte) lauten die entsprechenden Gleichungen
\begin{subequations}
\begin{align}\label{Eq65}
\vek K - \vek K(w) &= \vek K_0 \\
\vek C[\vek M] + \vek M &= \vek 0 \\
- \text{div}^2\,\vek M &= p\,,
\end{align}
\end{subequations}
was als die Anwendung eines Operators $\vek A()$ auf das Tripel $\vek \Sigma = \{w, \vek K, \vek M\}$ gelesen werden kann. $\vek K_0$ sind m\"{o}gliche Anfangskr\"{u}mmungen.

Der Operator $\vek K()$ angewandt auf $w$ sind nat\"{u}rlich die zweiten Ableitungen (\glq Kr\"{u}mmungen\grq{})
\begin{align}
\vek K(w) = \left[ \barr {r @{\hspace{4mm}}r  }
      w,_{11} & w,_{12} \\
      w,_{21} & w,_{22} \\
     \earr \right]\,.
\end{align}
Mit partieller Integration erh\"{a}lt man mit symmetrischen Matrizen $\vek M \in C^2(\Omega)$ und Funktionen $\delta w \in C^2(\Omega)$ das Resultat
\begin{align}
\int_{\Omega} - \text{div}^2\,\vek M\,\delta w \,d\Omega &= - \int_{\Gamma} (V_n\,\delta w - M_n\,\frac{\partial \delta w}{\partial n}) \,ds - [[M_{nt}\,\delta w]] \nn \\
&- \int_{\Omega} \vek M \dotprod \vek K(\delta w)\,d\Omega\,,
\end{align}
wobei (in Tensorschreibweise)
\begin{align}
V_n = \frac{d}{ds}\,M_{nt} + Q_n \quad M_{nt} = M_{ij}\,n_i\,t_j \quad M_n = M_{ij}\,n_i\,n_j \quad Q_n = M_{ij,i}\,n_j\,,
\end{align}
und $\vek n = \{n_1, n_2\}^T$  und $\vek t = \{t_1, t_2\}^T$ sind der Normalen- und Tangentenvektor auf dem Rand (jeweils mit der L\"{a}nge 1).

Das System (\ref{Eq65}) kann man als die Anwendung eines Operators $\vek A()$ auf das Triple $\vek \Sigma = \{w, \vek K, \vek M\}$ lesen und
\begin{align}
<\vek A(\vek \Sigma), \vek  \delta\vek \Sigma> &= \int_{\Omega} (\vek K(w) - \vek K) \dotprod \vek  \delta\vek M \,d\Omega + \int_{\Omega} (\vek C[\vek K] - \vek M)\dotprod  \vek  \delta K \,d\Omega \nn \\
&+ \int_{\Omega} - \text{div}^2\,\vek M \, \delta w\,d\Omega
\end{align}
ist dann die zugeh\"{o}rige \glq Paarung\grq{} mit  $\vek  \delta \vek \Sigma = \{\delta w, \vek  \delta \vek K, \vek  \delta \vek M\}$ als \glq virtueller Verr\"{u}ckung\grq{}.

Die drei Gebietsintegrale in diesem Ausdruck,
\begin{align}
a(\vek \Sigma, \vek  \delta\vek \Sigma) &= \int_{\Omega} (\vek K(w) - \vek K) \dotprod \vek  \delta\vek M \,d\Omega + \int_{\Omega}\vek C[\vek K] \dotprod  \hat{\vek K}\,d\Omega \nn \\
&+ \int_{\Omega} \vek M \dotprod (\vek K(\delta w) - \vek  \delta\vek K) \,d\Omega\,,
\end{align}
bilden eine symmetrische Bilinearform und so werden wir auf die Identit\"{a}t
\begin{align}
\text{\normalfont\calligra G\,\,}(\vek \Sigma,\vek  \delta\Sigma) &= <\vek A(\vek \Sigma),\vek  \delta\vek \Sigma> + \int_{\Gamma} (V_n\,\delta w - M_n\,\frac{\partial \delta w}{\partial n}) \,ds + [[M_{nt}\,\delta w]] \nn \\
&- a(\vek \Sigma, \vek  \delta\vek \Sigma) = 0
\end{align}
gef\"{u}hrt.

Das Symbol
\begin{align}
[[M_{nt}\,\delta w]] = \sum_i\,F_i\,\delta w(\vek x_i)
\end{align}
steht f\"{u}r die virtuellen Arbeit der Eckkr\"{a}fte $F_i$, die sich ja aus den Spr\"{u}ngen des Torsionsmoment $M_{nt}$ in den Ecken $\vek  x_i$ herleiten.


Das System $\vek A(\vek \Sigma)$ f\"{u}r das Tripel $\vek \Sigma = \{w, \vek K, \vek M\}$ kann man nun auf ein System f\"{u}r die Durchbiegung $w$ allein reduzieren, indem man die Gleichungen (\ref{Eq65}) ineinander einsetzt
\begin{align}
 K\,\Delta\Delta w = p\,.
\end{align}
Zur linken Seite geh\"{o}rt die Identit\"{a}t
\begin{align}\label{GPlatte}
\text{\normalfont\calligra G\,\,}(w,\delta w) &= \underbrace{\int_{\Omega}K\,\Delta \Delta w \,\delta w\,d\Omega + + \int_{\Gamma} (V_n\,\delta w - M_n\,\frac{\partial \delta w}{\partial n}) \,ds + [[M_{nt}\,\delta w]]}_{\delta A_a}\nn \\
 &- \underbrace{\int_{\Omega} \vek K(w) \dotprod \vek C[\vek K(\delta w)] \,d\Omega}_{\delta A_i} = 0\,,
\end{align}
die das {\em Prinzip der virtuellen Verr\"{u}ckungen\/} bzw. das {\em Prinzip der virtuellen Kr\"{a}fte\/} formuliert. Und selbstverst\"{a}ndlich ist
\begin{align}
\text{\normalfont\calligra B\,\,}(w,\delta w) = \text{\normalfont\calligra G\,\,}(w,\delta w) - \text{\normalfont\calligra G\,\,}(\delta w,w) = 0
\end{align}
der {\em Satz von Betti\/}.



%%%%%%%%%%%%%%%%%%%%%%%%%%%%%%%%%%%%%%%%%%%%%%%%%%%%%%%%%%%%%%%%%%%%%%%%%%%%%%%%%%%%%%%%%%%%%%%%%%%
\textcolor{sectionTitleBlue}{\section{Die schubweiche Platte (Reissner-Mindlin)}}\index{Reissner-Mindlin Platte}
Die Kinematen sind die Durchbiegung und die Verdrehungen um die beiden Achsen
\begin{align}
w(\vek x) \qquad \vek \Np = \{\Np_1,\Np_2\}\,.
\end{align}
Die zugeh\"{o}rigen Verzerrungen bestimmen sich gem\"{a}{\ss}
\begin{subequations}
\begin{align}
\vek E(\vek \Np) - \vek E &= \vek 0_{\,\,(2 \times 2)} \\
\vek \varepsilon(\vek \varphi ,w) - \vek \varepsilon &= \vek 0_{\,\,(2)}\,.
\end{align}
\end{subequations}
Die konstitutiven Gleichungen sind
\begin{subequations}
\begin{align}
\vek C[\vek E] - \vek M &= \vek 0_{\,\,(2 \times 2)}  \\
a\,\vek \varepsilon - \vek q = \vek 0_{\,\,(2)}
\end{align}
\end{subequations}
und die Gleichgewichtsbedingungen lauten
\begin{subequations}
\begin{align}
- \text{div}\,\vek M + \vek q = b\,\nabla\,p_{\,\,(2)} \\
- \text{div}\,\vek q = p_{\,\,(1)}\,.
\end{align}
\end{subequations}
Es ist
\begin{align}
\vek E(\vek \varphi ) &= \left[ \barr {r @{\hspace{4mm}}r  }
      \varphi_1,_1 & \frac{1}{2}\,(\varphi_1,_2 + \varphi_2,_1) \\
      \text{sym.} & \varphi_2,_2 \\
     \earr \right] \qquad \vek \varepsilon(\vek \varphi ,w) = \left[ \barr {r }
      \varphi_1 + w,_1 \\
      \varphi_2 + w,_2
     \earr \right]
\end{align}
und
\begin{align}
 \vek C[\vek E] = K\,(1 - \nu)\,\vek E + \nu\,K\,(\text{tr}\,\vek E)\,\vek I\,.
\end{align}
Die Parameter lauten
\begin{align}
K = \frac{E\,h^3}{12\,(1 - \nu^2)} \qquad a = K\,\frac{1 - \nu}{2}\,\bar{\lambda}^2 \qquad b = \frac{\nu}{1 - \nu}\,\frac{1}{\bar{\lambda}^2} \qquad \bar{\lambda}^2 = \frac{10}{h^2}\,.
\end{align}
Setzt man die Gleichungen ineinander ein, dann erh\"{a}lt man das folgende Differentialgleichungssystem f\"{u}r die drei Kinematen $w, \Np_1, \Np_2$
\begin{subequations}
\begin{align}
- \text{div}\,\vek C[\vek E(\vek \varphi )] + a\,\vek \varepsilon(\vek \varphi, w) &= b\,\nabla\,p_{\,\,(2)} \\
- \text{div}\,(a\,\vek \varepsilon(\vek \varphi ,w)) = p_{\,\,(1)}\,.
\end{align}
\end{subequations}
Zu diesem System geh\"{o}rt die Identit\"{a}t
\begin{align}
\text{\normalfont\calligra G\,\,}(\vek \varphi ,w; \hat{\vek \varphi} ,\hat{w}) &= \!\!\int_{\Omega} [- \text{div}\,\vek C[\vek E(\vek \varphi )] \dotprod \hat{\vek \varphi } - a\,\text{div}(\vek \varepsilon(\vek \varphi ,w)\,\hat{w}\,d\Omega \nn \\
& \!\!+ \int_{\Gamma} (\vek C[\vek E(\vek \varphi )]\,\vek n \dotprod  \vek \varphi + a\,\vek \varepsilon(\vek \varphi ,w) \dotprod \vek n\,\hat{w}\,)ds
- a(\vek \varphi, w; \hat{\vek \varphi }, \hat{w}) = 0
\end{align}
mit der symmetrischen Bilinearform
\begin{align}
a(\vek \varphi, w; \vek  \delta \vek \varphi , \delta w) = \int_{\Omega} (\vek C[\vek E(\vek \varphi )] \dotprod  \vek E(\vek  \delta\vek \varphi ) + a\,\vek \varepsilon (\vek \varphi , w) \dotprod  \vek \varepsilon(\vek  \delta \vek \varphi , \delta w) \,d\Omega\,.
\end{align}



%%%%%%%%%%%%%%%%%%%%%%%%%%%%%%%%%%%%%%%%%%%%%%%%%%%%%%%%%%%%%%%%%%%%%%%%%%%%%%%%%%%%%%%%%%%%%%%%%%%
\textcolor{sectionTitleBlue}{\section{Nichtlineare Formulierungen}}
%%%%%%%%%%%%%%%%%%%%%%%%%%%%%%%%%%%%%%%%%%%%%%%%%%%%%%%%%%%%%%%%%%%%%%%%%%%%%%%%%%%%%%%%%%%%%%%%%%%

Das wichtigste vorweg: Auch bei nichtlinearen Problemen gibt es eine erste Greensche Identit\"{a}t
\begin{align}
\text{\normalfont\calligra G\,\,}(u,\delta u) = a_u(u,\delta u) - (p, \delta u) = 0 \,,
\end{align}
die als Vorlage f\"{u}r die FE-L\"{o}sung $u_h = \sum_j u_j\,\Np_j(x)$ dient
\begin{align}
a_u(u_h,\Np_i) - (p,\Np_i) = 0 \qquad i = 1,2, \ldots, n\,.
\end{align}
Entscheidend bei einer nichtlinearen Formulierung ist, dass man die Gleichungen, also die drei Schritte  $u \to \varepsilon \to \sigma \to p$, im Griff hat. Ist dieser Pfad verstanden und richtig gesetzt, dann ergibt sich die erste Greensche Identit\"{a}t von selbst und dann ist der Rest nur noch Algebra und ein schneller Computer.

Auf der rechten Seite der Gleichung
\begin{align}
\vek k(\vek u) = \vek f
\end{align}
steht weiterhin die Arbeit der Belastung $p$ auf den Wegen $\Np_i $, aber die linke Seite ist nicht mehr die dazu \"{a}quivalente Arbeit $\vek f_h$ der {\em shape forces\/}, sondern es ist das Inkrement der inneren Arbeit auf den Wegen $\Np_i $. Man bewegt sich sozusagen aus dem Gleichgewichtspunkt $\vek u $ probeweise in eine Richtung $\Np_i$ und kontrolliert, ob dabei der Zuwachs an innerer Energie gleich dem Zuwachs an \"{a}u{\ss}erer Arbeit ist.

Das ist wie in der Schule. Wenn die Funktion $F(x) = f(x) - p \cdot x $ im Punkt $x $ ein Minimum hat, dann muss in erster N\"{a}herung das Inkrement $df = f'(x)\,dx$ bei einer St\"{o}rung $dx $  gleich dem Inkrement von $p$ sein, $f'(x)\,dx = p\,dx$.

Dass die Eintr\"{a}ge $k_i$ des Vektors $\vek k(\vek u)$ die Inkremente der inneren Energie sind, liest man an der ersten Greenschen Identit\"{a}t ab
\begin{align}
\text{\normalfont\calligra G\,\,}(u,\Np_i) = a_u(u,\Np_i) - (p, \Np_i) \equiv f'(x)\,dx - p\,dx = 0\,,
\end{align}
denn bei nichtlinearen Problemen steht dort nicht $ a(u,\Np_i)$, sondern die {\em Gateaux-Ableitung\/} $a_u(u,\Np_i)$ der inneren Energie im Punkt $u$ in  Richtung von $\Np_i$, also sinngem\"{a}{\ss} das $f'(x)\,dx$ mit $dx = \Np_i$.

\hspace*{-12pt}\colorbox{highlightBlue}{\parbox{0.98\textwidth}{Die partielle Integration des Arbeitsintegrals $(L\,u,\Np_i)$ f\"{u}hrt bei nichtlinearen Problemen automatisch(!) auf diese inkrementelle Betrachtungsweise.}}\\
Anlass, sich einmal mehr zu wundern, wieviel Intelligenz in die partielle Integration eingebaut ist\footnote{$L$ = der Differentialoperator des Problems}.

%%%%%%%%%%%%%%%%%%%%%%%%%%%%%%%%%%%%%%%%%%%%%%%%%%%%%%%%%%%%%%%%%%%%%%%%%%%%%%%%%%%%%%%%%%%%%%%%%%%
{\textcolor{sectionTitleBlue}{\section{Nichtlinearer Stab}}}\index{nichtlinearer Stab}\label{Korrektur5}
Zum Einstieg betrachten wir den nichtlinearen Stab, \cite{Ha5} S. 404.
\begin{subequations}
\begin{alignat}{3}
\hspace{-2cm} \mbox{Verzerrungen}\qquad && \varepsilon - (u' + \frac{1}{2}\, \,(u')^2) &= 0  \\
\hspace{-2cm} \mbox{Materialgesetz}\qquad &&\sigma - E\,\varepsilon  &= 0 \\
\hspace{-2cm} \mbox{Gleichgewicht}\qquad&&-N' &= p&
\end{alignat}
\end{subequations}
mit der Normalkraft
\begin{align}
N = A\,(\sigma + u'\,\sigma) \,.
\end{align}
Diese Definition stimmt sinngem\"{a}{\ss} mit der Definition $\vek S + \vek \nabla \vek u\, \vek S$ bei der Scheibe \"{u}berein, s. S. \pageref{Eq54}.

Partielle Integration des Arbeitsintegrals
\begin{align}
\int_0^{\,l} - N'\,\delta u\,dx = -[N\,\delta u]_0^l+ \int_0^{\,l} N \,\delta u'\,dx = 0
\end{align}
ergibt die zugeh\"{o}rige erste Greensche Identit\"{a}t
\begin{align} \label{Eq105}
\text{\normalfont\calligra G\,\,}(u,\delta u) = \int_0^{\,l} - N'\,\delta u\,dx + [N\,\delta u]_0^l- \underbrace{\int_0^{\,l} \varepsilon_u(\delta u)\,\sigma\,A\,dx}_{a_u(u, \delta u)} = 0
\end{align}
wobei
\begin{align}
\varepsilon_u(\delta u) = (1 + u')\,\delta u'
\end{align}
die Gateaux Ableitung
\begin{align}
\frac{d}{d\eta} \varepsilon(u + \eta \,\delta u)|_{\eta = 0}
\end{align}
von $\varepsilon(u)$ in Richtung von $\delta u$ ist.

%%%%%%%%%%%%%%%%%%%%%%%%%%%%%%%%%%%%%%%%%%%%%%%%%%%%%%%%%%%%%%%%%%%%%%%%%%%%%%%%%%%%%%%%%%%%%%%%%%%
{\textcolor{sectionTitleBlue}{\subsection{Finite Elemente}}}
Zu bestimmen sei zum Beispiel die L\"{a}ngsverschiebung $u(x) $ eines links festgehaltenen Stabes, $u(0) = 0$, mit einem freien Ende, $N(l) = 0$.

F\"{u}r die FE-L\"{o}sung machen wir den Ansatz
\begin{align}
u_h = \sum_j u_j\,\Np_j(x)
\end{align}
und bestimmen die Knotenverschiebungen $u_i $ so, dass
\begin{align}
a_u(u_h,\Np_i) - \int_{0}^{l} p\,\Np_i\,dx = k_i(\vek u) - f_i = 0 \qquad i = 1,2, \ldots, n\,.
\end{align}
Diese $n$ Gleichungen bilden ein System\footnote{Der Vektor $\vek k(\vek u)$ h\"{a}ngt von dem Vektor $\vek u$ ab. In der linearen Statik ist $\vek k(\vek u) = \vek K\,\vek u$} von $n$ nichtlinearen Gleichungen $\vek k(\vek u) = \vek f$, das iterativ mit dem {\em Newton-Verfahren\/}\index{Newton-Verfahren} gel\"{o}st wird
\begin{align}\label{Eq184}
\vek K_T(\vek u_i)\,(\vek u_{i+1} - \vek u_i) = \vek f - \vek k(\vek u_i)\,,
\end{align}
oder
\begin{align}
\vek u_{i+1}= \vek u_i + \vek K_T^{-1}(\vek u_i)\,(\vek f - \vek k(\vek u_i))\,.
\end{align}
Schreibt man das Newton-Verfahren f\"{u}r eine Gleichung $g(x) = k(x) - f = 0$ daneben
\begin{align}
x_{i+1} = x_i - \frac{g(x_i)}{g'(x_i)} \qquad \text{oder} \qquad g'(x_i)\,(x_{i + 1} -x_i) = f - k(x_i)\,,
\end{align}
dann erkennt man, wie (\ref{Eq184}) entsteht. Die tangentiale Steifigkeitsmatrix $\vek K_T $
entspricht dem $g'$, ist also die Ableitung von $\vek k(\vek u)$ nach den $u_i$ (genauer, ist der Gradient von $\vek k(\vek u)$).\index{tangentiale Steifigkeitsmatrix}

%%%%%%%%%%%%%%%%%%%%%%%%%%%%%%%%%%%%%%%%%%%%%%%%%%%%%%%%%%%%%%%%%%%%%%%%%%%%%%%%%%%%%%%%%%%%%%%%%%%
\textcolor{sectionTitleBlue}{\section{Geometrisch nichtlinearer Balken}}
Die Biegesteifigkeit $EI$ und L\"{a}ngssteifigkeit $EA$ l\"{a}ngs des Balkens sind konstant und die Streckenlasten lauten $p_x$ und $p_z$. Die Kinematen sind die L\"{a}ngsverschiebung $u$ und die Durchbiegung $w$, die man zu $\vek v = \{u, w\}^T$ zusammenfassen kann,
\begin{subequations}
\begin{alignat}{3}
&& \varepsilon &= u' + \frac{1}{2}\,(w')^2 &\quad \kappa &= w''\\
&&N &= EA\,\varepsilon & \quad M &= - EI\,\kappa \\
&&- N' &= p_x& \quad - M'' - (N\,w')' &= p_z\,.
\end{alignat}
\end{subequations}
Daraus ergibt sich das folgende System von Differentialgleichungen f\"{u}r $u$ und $w$
\begin{subequations}
\begin{align} \label{Eq98}
- EA\,(u' + \frac{1}{2}\, (w')^2)' &= p_x \\
EI\,w^{IV} - (EA\,(u' + \frac{1}{2}\, (w')^2)\,w')' &= p_z\,,
\end{align}
\end{subequations}
oder in einer etwas \glq transparenteren\grq{} Fassung
\begin{subequations}
\begin{align}
- N'\, &= p_x \\
EI\,w^{IV} - (N\,w')' &= p_z\,.
\end{align}
\end{subequations}
Es sei
\begin{align}
N = N(\vek v) = EA\,(u' + \frac{1}{2}\, (w')^2)\,, \qquad M = M(w) = - EI\,w''\,,
\end{align}
und $\vek L\,\vek v$ bezeichne die linke Seite des obigen Systems, dann l\"{a}sst sich das Arbeitsintegral
mittels partieller Integral wie folgt umschreiben
\begin{align}
\int_0^{\,l} \vek L\,\vek v \dotprod \vek \delta\,\vek v\,dx &= \int_0^{\,l} ((Eq_1) \cdot \delta u + (Eq_2) \cdot \delta w)\, dx \nn \\
&= \int_0^{\,l} [(- N'\,\delta u - (M'' + (N\,w')')\,\delta w]\,dx \nn \\
&= - [ N\,\delta u + (M' + N\,w')\,\delta w - M\,\delta w']_{@0}^{@l} + a_{\vek v}(\vek v, \vek \delta\,\vek v)\,,
\end{align}
wobei
\begin{align}
a_{\vek v}(\vek v, \vek \delta\,\vek v) &= \int_0^{\,l} (- M \,\delta w'' + N\,(\delta u' + w'\, \delta w'))\,dx \nn \\
&= \int_0^{\,l} (\frac{M(w)\,M_w(\delta w)}{EI} + \frac{N(\vek v)\,N_{\vek v}(\vek \delta \vek v)}{EA})\,dx
\end{align}
das Inkrement der Wechselwirkungsenergie ist, das wir im zweiten Teil unter Benutzung der {\em Gateaux-Ableitungen\/} von $M$ bzw. $N$,
\begin{align}
M_w(\delta w) = [\frac{d}{d\varepsilon} M(w + \varepsilon\,\delta w)]_{|_{\varepsilon} = 0} \qquad N_{\vek v}(\vek \delta \vek v) = [\frac{d}{d\varepsilon} N(\vek v + \varepsilon\,\vek \delta \vek v)]_{|_{\varepsilon} = 0}
\end{align}
angeschrieben haben. Die erste Greensche Identit\"{a}t lautet somit
\begin{align}
G(\vek v, \vek  \delta \vek  v) &= \int_0^{\,l} \vek L\,\vek u \dotprod \vek \delta u\,dx +  [ N\,\delta u + (M' + N\,w')\,\delta w - M\,\delta w']_{@0}^{@l} \nn \\
&- a_{\vek v}(\vek v, \vek \delta\,\vek v) = 0\,.
\end{align}
Auf diesen Gleichungen beruht die Theorie II. Ordnung bei Balken, nur ist es so, dass man von einer konstanten Normalkraft $N$ ausgeht, die zudem als bekannt angenommen wird, so dass sich das System (\ref{Eq98}) auf
\begin{subequations}
\begin{align}
- EA\,N'\, &= 0 \\
EI\,w^{IV} - N\,w'' &= p_z
\end{align}
\end{subequations}
reduziert, also in dem letzten Ausdruck die bekannte Gleichung der Theorie II. Ordnung \"{u}brig bleibt.

%%%%%%%%%%%%%%%%%%%%%%%%%%%%%%%%%%%%%%%%%%%%%%%%%%%%%%%%%%%%%%%%%%%%%%%%%%%%%%%%%%%%%%%%%%%%%%%%%%%
\textcolor{sectionTitleBlue}{\section{Geometrisch nichtlineare Kirchhoffplatte}}
Die Formulierung verl\"{a}uft im Grunde wie bei dem geometrisch nichtlinearen Balken, nur sind noch mehr Gleichungen anzuschreiben. Wir verweisen daher interessierte Leser auf  S. 325-328 in \cite{Ha1}.

%%%%%%%%%%%%%%%%%%%%%%%%%%%%%%%%%%%%%%%%%%%%%%%%%%%%%%%%%%%%%%%%%%%%%%%%%%%%%%%%%%%%%%%%%%%%%%%%%%%
\textcolor{sectionTitleBlue}{\section{Nichtlineare Elastizit\"{a}tstheorie}}
In dem Triple $\{\vek u,\vek E, \vek S\}$ ist der Tensor $\vek E$ der Green-Lagrange Verzerrungstensor\index{Green-Lagrange Verzerrungstensor} und $\vek S$ ist der zweite Piola-Kirchhoff Spannungstensor\index{zweiter Piola-Kirchhoff Spannungstensor}. Wir nehmen an, dass das Material hyperelastisch\index{hyperelastisches Material} ist, d.h. es gibt eine Verzerrungsenergiefunktion $\vek W$ derart, dass $\vek S = \partial \vek W/\partial \vek E$.

In der Gegenwart von Volumenlasten $\vek
p$ gen\"{u}gt der elastische Zustand $ \vek \Sigma = \{\vek u,\vek E, \vek S\}$ in jedem Punkt
$\vek x$ des unverformten K\"{o}rpers dem System
\begin{subequations}\label{E4NlS}
\begin{alignat}{3}
\vek E(\vek u) - \vek E &= \vek 0 &\qquad \frac{1}{2}\,(u_i,_j + u_j,_i + u_k,_i\,u_k,_j)
-
\varepsilon_{ij} &= 0  \\
\vek W'(\vek E) - \vek S &= \vek 0 &\qquad \frac{\partial W}{\partial \varepsilon_{ij}} -
\sigma_{ij}&= 0 \\
- \mbox{div}(\vek S + \nabla\,\vek u\,\vek S) &= \vek p &\qquad - (\sigma_{ij } +
u_i,_k\,\sigma_{kj}),_j &= p_i
\end{alignat}
\end{subequations}
mit passenden Verschiebungsrandbedingungen $\vek u = \bar{\vek u}$ auf dem Teil
$\Gamma_D$ des Randes und Spannungsrandbedingungen $\vek t(\vek S,\vek u) =
\bar{\vek t}$ auf dem anderen Teil $\Gamma_N$ des Randes wobei
\beq
\vek t(\vek S,\vek u) := (\vek S + \nabla\vek u\,\vek S)\,\vek n
\eeq
der Spannungsvektor in einem Randpunkt mit der nach Au{\ss}en gerichteten Randnormalen $\vek n$ ist.

Symmetrische Spannungstensoren $\vek S$ gen\"{u}gen der Identit\"{a}t
\begin{align}
\int_{\Omega} - \mbox{div} (\vek S &+ \nabla\,\vek u\,\vek S)\dotprod \vek \delta \vek u
\,d\Omega \nn \\
&= - \int_{\Gamma} \vek t(\vek S,\vek u) \dotprod \vek \delta \vek u\,ds + \int_{\Omega}\vek
E_{\vek u}(\vek \delta \vek u) \dotprod \vek S\,d\Omega\,,
\end{align}
wobei
\beq
\vek E_{\vek u}(\vek \delta \vek u) := \frac{1}{2}\,(\nabla\vek \delta \vek u + \nabla\vek \delta \vek u^T
+ \nabla \vek u^T\,\nabla\,\vek \delta \vek u + \nabla\,\vek \delta \vek u^T\,\nabla\,\vek u)
\eeq
die Gateaux Ableitung \index{Gateaux Ableitung} des Tensors $\vek E(\vek u)$ ist,
\beq
\frac{d}{d\varepsilon} [\vek E(\vek u + \varepsilon\,\vek \delta \vek u)]_{|_{\varepsilon = 0}}
=\vek E_{\vek u}(\vek \delta \vek u)\,.
\eeq
Wir k\"{o}nnen so die erste Greensche Identit\"{a}t des Operators $\vek A(\vek \Sigma )$, also des Systems (\ref{E4NlS}), anschreiben
\beq\label{E4Gfa}
\text{\normalfont\calligra G\,\,}(\vek \Sigma ,\vek \delta \vek \Sigma) = \underbrace{\langle\vek
A(\vek \Sigma ),\vek  \delta \vek \Sigma \rangle +\int_{\Gamma} \vek t(\vek S,\vek u) \dotprod
\vek \delta \vek u\,ds }_{\delta A_a} - \underbrace{\!\!\!\!
\phantom{\int_{\Gamma}}a_{\vek \Sigma}(\vek \Sigma,\vek  \delta \vek \Sigma)}_{\delta A_i} = 0\,,
\eeq
wobei
\begin{align}\label{E4Similar}
\langle A(\vek \Sigma ),\vek  \delta \vek \Sigma \rangle :&= \int_0^{\,l} (\vek E(\vek u) - \vek
E) \dotprod \vek  \delta\vek S \,d\Omega +
\int_{\Omega} (\vek C[\vek E] - \vek S)\dotprod \vek  \delta\vek E\,d\Omega \nn \\
&+ \int_{\Omega} - \mbox{div}\,\vek S \dotprod \vek \delta \vek u\,d\Omega
\end{align}
und
\begin{align}
a_{\vek \Sigma}(\vek \Sigma ,\vek  \delta \vek \Sigma) &= \int_{\Omega} (\vek E(\vek u) - \vek E) \dotprod
\vek  \delta\vek S\,d\Omega \nn \\&+ \int_{\Omega} (\vek W'(\vek E) - \vek S)\dotprod
\vek  \delta\vek E\,d\Omega + \int_{\Omega} \vek E_{\vek u}(\vek \delta \vek u)\dotprod \vek S
\,d\Omega\,.
\end{align}
Bei einer reinen Verschiebungsformulierung, bei der alles aus $\vek u$ abgeleitet wird, $\vek \Sigma = \{\vek u, \vek E(\vek u),
\vek W'(\vek E(\vek u))\}$, und mit der Randbedingung $\vek \delta \vek u = \vek 0$ auf $\Gamma_D$, reduziert sich das System
(\ref{E4Gfa}) auf
\beq\label{E4Gfb}
\text{\normalfont\calligra G\,\,}(\vek u, \vek \delta \vek u) = \int_{\Omega} \vek p \dotprod \vek \delta \vek u\,d\Omega +
\int_{\Gamma_N}  \bar{\vek t} \dotprod \vek \delta \vek u\,ds - \int_{\Omega} \vek E_{\vek
u}(\vek \delta \vek u) \dotprod \vek S\,d\Omega = 0\,,
\eeq
wobei $\vek S = \vek W'(\vek E(\vek u))$.

Die FE-L\"{o}sung $\vek u_h = \sum_j u_j\,\vek \Np_j$ bestimmt man aus, $\vek S_h = \vek W'(\vek E(\vek u_h))$,
\begin{align}
\int_{\Omega} \vek p \dotprod \vek \delta \vek \Np_i\,d\Omega +
\int_{\Gamma_N}  \bar{\vek t} \dotprod \vek \delta \vek \Np_i\,ds - \int_{\Omega} \vek E_{\vek
u}(\vek \delta \vek \Np_i) \dotprod \vek S_h\,d\Omega = 0 \qquad i = 1,2,\ldots n\,.
\end{align}
Alternativ kann man nat\"{u}rlich andere Variationsformulierungen aus dem Grundsystem (\ref{E4Gfa}) herleiten, etwa gemischte Verfahren bei denen die Verschiebungen $\vek u$ und die Spannungen, also der Tensor $\vek S$, als unabh\"{a}ngige Gr\"{o}{\ss}en behandelt werden, \cite{Ha1}.

%%%%%%%%%%%%%%%%%%%%%%%%%%%%%%%%%%%%%%%%%%%%%%%%%%%%%%%%%%%%%%%%%%%%%%%%%%%%%%%%%%%%%%%%%%%%%%%%%%%
{\textcolor{sectionTitleBlue}{\section{Der Linearisierungspunkt}}}
Arbeit als das Produkt von {\em Kraft und Weg\/}, ist eine distributive Verkn\"{u}pfung, wie die \"{U}berlagerung von Funktionen, das $L_2$-Skalar\-pro\-dukt\index{$L_2$-Skalarprodukt},
\begin{align}
\int_0^{\,l} p\,(w_1 + w_2)\,dx = \int_0^{\,l} p\,w_1\,dx + \int_0^{\,l} p\,w_2\,dx \,.
\end{align}
Dasselbe gilt f\"{u}r die Differentialgleichungen der linearen Statik
\begin{align}
-EA\,(u_1 + u_2)'' = -EA\,u_1'' - EA\,u_2''\,.
\end{align}
aber nicht f\"{u}r eine nichtlineare Differentialgleichung wie
\begin{align}
-EA\,u''\,(1 + u') = p\,.
\end{align}
Sie ist auch nicht selbstadjungiert und daher gibt es keinen {\em Satz von Betti\/} und das enthebt uns des Problems dar\"{u}ber nachdenken zu m\"{u}ssen, wie denn eine Einflussfunktion f\"{u}r eine nichtlineare Gleichung aussehen k\"{o}nnte.

Aber im Linearisierungspunkt des Newton-Algorithmus kann man Einflussfunktionen aufstellen und das Verfahren des {\em goal-oriented refinement\/} mit Erfolg anwenden. Uns fehlt hier leider der Raum auf diese Dinge n\"{a}her einzugehen, sie sind auch sehr technisch und im Bauwesen nicht unbedingt das prim\"{a}re Problem, deswegen sei an dieser Stelle auf die Literatur verwiesen, \cite{Ha1}, \cite{Ha6}.


F\"{u}r weitere Beispiele zu dem Thema nichtlineare Probleme, erste Greensche Identit\"{a}t und finite Elemente sei auf \cite{Ha5}, Chapter 4.21, verwiesen.

%%%%%%%%%%%%%%%%%%%%%%%%%%%%%%%%%%%%%%%%%%%%%%%%%%%%%%%%%%%%%%%%%%%%%%%%%%%%%%%%%%%%%%%%%%%%%%%%%%%
\textcolor{sectionTitleBlue}{\section{Potentialtheorie}}\index{Potentialtheorie}
Die Potentialtheorie besch\"{a}ftigt sich mit den Eigenschaften der Felder, die von Punktladungen erzeugt werden. Das Charakteristikum dieser Felder
\begin{align}
 g(\vek y, \vek x) = - \frac{1}{2\,\pi}\ln r \qquad \text{2-$D$} \qquad g(\vek y, \vek x) = \frac{1}{4\,\pi}\frac{1}{r} \qquad \text{ 3-$D$} \qquad
\end{align}
ist, dass sie in allen Punkten $\vek y \neq \vek x$  L\"{o}sungen der {\em Laplace Gleichung\/} sind
\begin{align}
\Delta g = \frac{\partial^2 g}{\partial y_{1}^2 } +\frac{\partial^2 g}{\partial y_{2}^2 } = 0\,.
\end{align}
Wir differenzieren hier nach der Laufvariablen $\vek y = (y_1,y_2)$.

Mit diesen {\em Fundamentall\"{o}sungen\/}\index{Fundamentall\"{o}sung} kann man Integraldarstellungen f\"{u}r beliebige Funktionen $u$ herleiten
\begin{align} \label{U207}
u(\vek x) = &\int_{\Gamma} [g(\vek y, \vek x) \,\frac{\partial u(\vek y)}{\partial n} - \frac{\partial g(\vek y, \vek x)}{\partial n}\,u(\vek y)]\,ds_{\vek y} + \int_{\Omega} g(\vek y, \vek x)\,(- \Delta u(\vek y))\,\,d\Omega_{\vek y}\,,
\end{align}
wenn man von den Funktionen den Randwert $u$ und die Normalableitung $\partial u/\partial n = \nabla u \dotprod \vek n$ auf dem Rand $\Gamma$ kennt und im Feld $\Omega$ die Summe der zweiten Ableitungen $\Delta u = u,_{y_1 y_1} + u,_{y_2 y_2}$.

Worauf wir an dieser Stelle hinaus wollen ist, dass sich das Gleichungssystem $\vek K\,\vek u_c = \vek f + \vek f^+$ auch aus dieser Gleichung herleiten l\"{a}sst.

Wir stellen uns eine Membran vor, die mit einer Kraft $H_a$  vorgespannt wird und unter Winddruck $p$ steht. Die Fundamentall\"{o}sung hat in diesem Fall die Gestalt
\begin{align}
g(\vek y, \vek x) = - \frac{1}{2\,\pi} \,\frac{1}{H_a}\,\ln\,r\,.
\end{align}
In einem Teil $\Omega_b$ im Innern der Membran habe die Vorspannung jedoch davon abweichend den Wert $H_b$ (wie so etwas zu realisieren sei, interessiert hier nicht). Technisch bedeutet dies, dass die Biegefl\"{a}che $u$ den beiden Differentialgleichungen
\begin{align}
- H_a\,\Delta u = p \quad \text{in $\Omega_a$} \qquad - H_b\,\Delta u = p \quad \text{in $\Omega_b$}
\end{align}
gen\"{u}gt.

Man kann nun zeigen\footnote{\cite{Ha3}, S. 139, dort findet man die genauen Details}, dass man in diesem Fall die Einflussfunktion (\ref{U207}) um ein Integral erweitern muss
\begin{align}
u(\vek x) = \ldots + \int_{\Gamma_i} g_i(\vek y, \vek x)\,t(\vek y)\,ds_{\vek y} \qquad g_i(\vek y, \vek x) = - \frac{1}{2\,\pi}\frac{H_b - H_a}{H_a\,H_b}\,\ln\,r\,.
\end{align}
Hier ist $\Gamma_i$ das {\em interface\/}, die Grenzlinie zwischen $\Omega_a$ und $\Omega_b$ und
\begin{align}
t(\vek y) = H_a\,\frac{\partial u}{\partial n}(\vek y) = -  H_b\,\frac{\partial u}{\partial n}(\vek y)
\end{align}
sind die gegengleichen Zugkr\"{a}fte auf $\Gamma_i$, einmal aus der Sicht von $\Omega_a$ bzw. aus der Sicht von $\Omega_b$; die Normalenvektoren $\vek n$ weisen jeweils in das andere Gebiet. Weil aber die Differenz $g_a\,t - g_b\,t =: g_i\,t$ nicht null ist, wirkt auf $\Gamma_i$ die mit dem Kern $g_i$ gewichtete Zugkraft $t$ als \"{a}u{\ss}ere Kraft.

Die Zugkr\"{a}fte $t$ auf $\Gamma_i$ f\"{u}hren, wenn man die Durchbiegung der Membran mit finiten Elementen ann\"{a}hert, genau auf die $f_i^+$
\begin{alignat}{2}
\vek K\,\vek u &= \vek f \qquad &&\text{\"{u}berall die gleiche Vorspannung $H_a$} \nn \\
\vek K\,\vek u_c &= \vek f + \vek f^+ \qquad &&\text{Vorspannung $H_a$ in $\Omega_a$ und $H_b$ in $\Omega_b$} \nn
\end{alignat}
Man sieht die $f_i^+$ sehr sch\"{o}n in Abb. \ref{U421} S. \pageref{U421}.


%%%%%%%%%%%%%%%%%%%%%%%%%%%%%%%%%%%%%%%%%%%%%%%%%%%%%%%%%%%%%%%%%%%%%%%%%%%%%%%%%%%%%%%%%%%%%%%%%%%
\textcolor{sectionTitleBlue}{\section{Erg\"{a}nzungen}}
Die weiteren Texte sollen die Ausf\"{u}hrungen im vorderen Teil des Buchs abrunden und vertiefen.

%%%%%%%%%%%%%%%%%%%%%%%%%%%%%%%%%%%%%%%%%%%%%%%%%%%%%%%%%%%%%%%%%%%%%%%%%%%%%%%%%%%%%%%%%%%%%%%%%%%
\textcolor{sectionTitleBlue}{\subsection{Einzelkraft in einer Scheibe}}\label{BeweisP}
Das folgende ist eine Erg\"{a}nzung zu dem Text auf S. \pageref{PPX}. Greift eine Einzelkraft in einer Scheibe an, dann kann man sich das wie folgt zurechtlegen. Man l\"{a}sst die Einzelkraft in einer unendlichen Scheibe wirken (LF 1) und addiert zu diesem Lastfall einen zweiten Lastfall (LF 2) derart, dass die Randbedingungen an der endlichen Scheibe von den beiden L\"{o}sungen zusammen eingehalten werden.

Die Spannungen aus dem LF 1 werden im Aufpunkt singul\"{a}r, aber die aus dem LF 2 sind endlich, sie sind beschr\"{a}nkt, und daher tendieren auch die Integrale der Spannungen aus dem LF 2 \"{u}ber sich immer enger zusammenschn\"{u}rende Kreise um den Aufpunkt gegen null, weil der Umfang der Kreise ja schrumpft. Wir m\"{u}ssen also nur das Integral der singul\"{a}ren Spannungen betrachten.

Das Spannungsfeld in der unendlich ausgedehnten Scheibe kennt man genau. Wenn in einem Punkt $\vek x$ eine Kraft $\vek e_i$ angreift, dann hat der Spannungsvektor in einem Punkt $\vek y$ mit der Schnittnormalen $\vek \nu = \{\nu_1, \nu_2\}^T$ die Komponenten, \cite{Ha3}, Glg. (4.7),
\begin{align}\label{Eq122}
T_{ij}(\vek y,\vek x) &= - \frac{1}{4\,\pi\,(1-\nu)\,r}\,[\frac{\partial r}{\partial \nu}(( 1- 2\,\nu)\,\delta_{ij} + 2\,r,_i\,r,_j) \nn\\
&- (1- 2\,\nu) (r,_i \,\nu_j(\vek y) - r,_j\,\nu_i(\vek y))]
\end{align}
mit
\begin{align}
r,_i := \frac{\partial r}{\partial y_i} =  \frac{y_i - x_i}{r}\,.
\end{align}
Liegt der Punkt $\vek y$ auf einem Kreis mit Radius $r$ um $\vek x$, dann sind die $\nu_i$ und die $r,_i$ gleich
\begin{align}
\nu_1 = r,_1 = \cos\,\Np \qquad \nu_2 = r,_2 = \sin\,\Np\,,
\end{align}
und somit gilt auf dem Kreis
\begin{align}
\frac{\partial r}{\partial \nu} = \nabla r \dotprod \vek \nu = \left [\barr{c}  \cos\,\Np \\  \sin\,\Np\earr \right ] \dotprod  \left [\barr{c}  \cos\,\Np \\  \sin\,\Np\earr \right ] = \cos^2\,\Np + \sin^2\,\Np = 1\,.
\end{align}
Da die Kraft in $x$-Richtung wirkt, setzen wir in (\ref{Eq122}) $i = 1$, und  so hat der Spannungsvektor auf dem Kreis die beiden Komponenten
\begin{align}
t_x &= T_{11} = - \frac{1}{4\,(1-\nu)\, \pi\,r} \cdot [(1- 2\,\nu) + 2\,\cos^2\,\Np] \\
t_y &= T_{12} = - \frac{1}{4\,(1-\nu)\,\pi\,r} \cdot [2\,\cos \Np \,\sin\,\Np]
\end{align}
und die Integration ergibt
\begin{align}
\int_0^{\,2\,\pi} t_x\,d\Np = -\frac{1}{r} \qquad \int_0^{\,2\,\pi} t_y\,d\Np = 0\,.
\end{align}
\begin{remark}
Gelegentlich gibt es auch einen Richtungsvektor im Aufpunkt $\vek x$, der dann $\vek n$ hei{\ss}t, und deshalb nennen wir, um die Vektoren auseinanderzuhalten, den Normalenvektor im Integrationspunkt $\vek y$ hier $\vek \nu$ und nicht $\vek n$ wie auf S. \pageref{Ergebnis}.
\end{remark}
%----------------------------------------------------------
\begin{figure}[tbp]
\centering
\if \bild 2 \sidecaption[t] \fi
\includegraphics[width=1.0\textwidth]{\Fpath/U303}
\caption{Erg\"{a}nzung zu S. \pageref{Footnote1}, Monopol---Dipol---Quadropol---Octopol und finite Differenzen} \label{U303}
\end{figure}%%
%----------------------------------------------------------

%%%%%%%%%%%%%%%%%%%%%%%%%%%%%%%%%%%%%%%%%%%%%%%%%%%%%%%%%%%%%%%%%%%%%%%%%%%%%%%%%%%%%%%%%%%%%%%%%%%
\textcolor{sectionTitleBlue}{\subsection{Multipole}}\label{Multipole}
Die Abb. \ref{U303} und der folgende Text ist eine Erg\"{a}nzung zu S. \pageref{Footnote1}. Die Taylorreihe f\"{u}r $w(x)$
\begin{align}
w(x) = w(0) + w'(0)\, x + \frac{1}{2}\,w''(0)\,x^2 + \ldots
\end{align}
f\"{u}hrt auf die Darstellung
\begin{align}
w(x) &= \int_0^{\,l} G_0(y,0)\,p(y)\,dy + \int_0^{\,l} G_1(y,0)\,p(y)\,dy \cdot x \nn \\
&+ \frac{1}{2}\, \int_0^{\,l} G_2(y,0)\,p(y)\,dy \cdot x^2 + \ldots
\end{align}
wenn wir mit $G_1$ und $G_2$ die Einflussfunktionen f\"{u}r $w'$ bzw. $w''$ bezeichnen.

Es ist aber ebenso gut m\"{o}glich, den Kern $G_0(y,x)$ in eine Taylorreihe um den Schwerpunkt $y_s$ der Linienlast zu entwickeln, was dann auf
\begin{align}
w(x) &= \int_0^{\,l} G_0(y,x)\,p(y)\,dy \nn\\
&= G_0(y_s,x) \int_0^{\,l} \,p(y)\,dy + \frac{d}{dy}\, G_0(y_s,x)\int_0^{\,l} \,p(y)\,(y - y_s)\,dy \nn \\
&+ \frac{1}{2}\,\frac{d^2}{dy^2}\, G_0(y_s,x) \int_0^{\,l} \,p(y)\,(y-y_s)^2\,dy + \ldots
\end{align}
f\"{u}hrt, also
\begin{align}\label{Eq136}
w(x) &=  G_0(y_s,x)\,R + \frac{d}{dy}\, G_0(y_s,x)\,M + \frac{1}{2}\, \frac{d^2}{dy^2}\, G_0(y_s,x)\,M_2 \,+ \ldots
\end{align}
wenn $R$ die Resultierende der Belastung ist, $M (= 0)$ das Moment der Belastung um $y_s$ ist und $M_2$ das \glq quadratische\grq{} Moment der Belastung um $y_s$ ist. Diese Entwicklung nennt man auch die {\em multipole expansion\/} von $w(x)$.

Mit der Anziehungskraft der Erde ist es \"{a}hnlich. Wenn die Erde eine perfekte, homogene Kugel w\"{a}re, dann k\"{o}nnte man die ganze Masse der Erde im Mittelpunkt der Erde konzentrieren und die Satelliten w\"{u}rden auf perfekten Kreisbahnen die Erde umkreisen. So braucht man Computer, um die Bahnkurven der Satelliten zu bestimmen.

An Hand von (\ref{Eq136}) kann man den Fehler absch\"{a}tzen, den man begeht, wenn man z.B. eine Trapezlast durch ihre Resultierende ersetzt.

%%%%%%%%%%%%%%%%%%%%%%%%%%%%%%%%%%%%%%%%%%%%%%%%%%%%%%%%%%%%%%%%%%%%%%%%%%%%%%%%%%%%%%%%%%%%%%%%%%%
\textcolor{sectionTitleBlue}{\subsection{Die Dimension der $f_i$}}\label{Dimensionsbetrachtung}
Eine oft diskutierte Frage ist, welche Dimension die $f_i$ beim Balken haben
\begin{align}
 \frac{EI}{l^3} \left[
\begin{array}{r r r r}
 12 & -6l & -12 &-6l \\
 -6l & 4l^2 & 6l &2l^2 \\
 -12 & 6l & 12 & 6l \\
 -6l &2l^2 &6l &4l^2
 \end{array}
  \right]\,\left [\barr{c} u_1 \\ u_2 \\ u_3 \\ u_4 \earr \right ] = \left [\barr{c}  f_1 \\ f_2 \\ f_3 \\ f_4 \earr \right ]\,.
\end{align}
Die Antwort lautet -- je nachdem. Wenn man die Eintr\"{a}ge $k_{ij}$ der Steifigkeitsmatrix mit der Formel
\begin{align}\label{Eq1280}
k_{ij } = a(\Np_i,\Np_j) = EI\,\int_0^{\,l} \Np_i''\,\Np_j''\,dx = \text{kNm$^2$ } \frac{1}{\text{m}}\frac{1}{\text{m}}\,\text{m} = \text{kNm}
\end{align}
berechnet, wie man das bei finiten Elementen tut, dann sind die $k_{ij}$ Arbeiten, die $u_i$ sind dimensionslos und so sind die $f_i$ Arbeiten.

Zur Erl\"{a}uterung von (\ref{Eq1280}): wenn $\Np_1(x)$ die Dimension Meter hat, dann haben die Ableitungen die Dimension
\begin{align}
\Np_1(x) \,\,\text{[m]} \qquad \Np_1'\,\,[\,] \qquad\Np_1'' = [\frac{1}{\text{m}}] \qquad\Np_1''' = [\frac{1}{\text{m$^2$}}]\,,
\end{align}
weil bei jeder Ableitung $d/dx$ durch Meter dividiert wird.

Man kann die Matrix $\vek K$ aber auch auf statischem Wege herleiten, indem man die  Balkenendkr\"{a}fte und -momente der Einheitsverformungen $\Np_i(x)$ berechnet und diese Werte in die jeweilige Spalte $i$ eintr\"{a}gt. Wenn man so vorgeht, dann sind die $k_{ij}$ der Dimension nach Kr\"{a}fte  bzw. Momente pro Auslenkung/Verdrehung $w_i = 1$ und das Ergebnis, $\vek K\,\vek w = \vek f$, sind dann die Kr\"{a}fte und Momente, die zur Auslenkung $\vek w$ geh\"{o}ren, $\vek K\,\vek w = \vek f$.

%%%%%%%%%%%%%%%%%%%%%%%%%%%%%%%%%%%%%%%%%%%%%%%%%%%%%%%%%%%%%%%%%%%%%%%%%%%%%%%%%%%%%%%%%%%%%%%%%%%
\textcolor{sectionTitleBlue}{\subsection{Transformationen}}
Mit den Bezeichnungen in Abb. \ref{U371} und der orthogonalen Matrix
\begin{align}
\vek T = \left[\barr{r r } \cos \alpha & - \sin \alpha \\ \sin \alpha  & \cos \alpha\earr\right] \qquad \vek T^{-1} = \vek T^T
\end{align}
transformiert sich ein Vektor $\vek f = \{f_x, f_z\}^T$ wie folgt
\begin{align}
\vek f = \vek T\,\bar{\vek f} \qquad  \bar{\vek f} = \vek T^T\,\vek f\,.
\end{align}
%------------------------------------------------------------------
\begin{figure}[tbp]
\centering
\if \bild 2 \sidecaption \fi
\includegraphics[width=.95\textwidth]{\Fpath/U371}
\caption{ Globales und lokales Koordinatensystem} \label{U371}
\end{figure}%%
%------------------------------------------------------------------
Die auf die Gr\"{o}{\ss}e $4 \times 4$ aufgeweitete Stabmatrix\footnote{Ein oberer Index $e$ bedeutet, dass sich die Matrix $\vek K^e$ auf das lokale Koordinatensystem bezieht.}
\begin{align}
\vek K_{4 \times 4}^e = \frac{EA}{\ell} \left[\barr{r @{\hspace{4mm}}r @{\hspace{4mm}}r @{\hspace{4mm}}r} 1 & 0 & -1 & 0 \\  0 & 0 & 0 & 0\\ -1 & 0 & 1 &0 \\ 0 & 0 & 0 &0\earr\right]\,,
\end{align}
wird die Matrix
\begin{align}
\vek K_e = \vek T^T\,\vek K^e \vek T = \frac{EA}{\ell }
 \left[\barr{r @{\hspace{4mm}}r @{\hspace{4mm}}r @{\hspace{4mm}}r} c^2 & - cs & -c^2 & cs \\ - cs & s^2 & cs & -s^2\\ -c^2 & cs & c^2 &-cs \\ cs & -s^2 & -cs &c^2\earr\right]
\end{align}
wobei die Transformationsmatrix $\vek T$ die Gestalt
\begin{align}
\vek T =
\left[\barr{r @{\hspace{4mm}}r @{\hspace{4mm}}r @{\hspace{4mm}}r} c & - s & 0 & 0 \\ s & c &0 & 0\\0 & 0 & c & -s \\ 0 & 0 & s & c\earr \right]
 \qquad c = \cos \alpha\,, s = \sin \alpha
\end{align}
hat, und aus der $4 \times 4$-Seilmatrix ($S$ = Zugkraft im Seil)
\begin{align}
\vek K^e = \frac{S}{\ell} \left[\barr{r @{\hspace{4mm}}r @{\hspace{4mm}}r @{\hspace{4mm}}r} 0 & 0 & 0 & 0\\ 0 & 1 & 0 & -1 \\  0 & 0 & 0 & 0\\ 0 & -1 & 0 &1 \earr\right]\,.
\end{align}
wird die Matrix
\begin{align}
\vek K_e = \vek T^T\,\vek K^e \vek T = \frac{S}{\ell }
 \left[\barr{r r r r} s^2 &  cs & -s^2 & -cs \\ cs & c^2 & -cs & -c^2\\ -c^2 & -cs & c^2 &cs \\ -cs & -c^2 & cs &c^2\earr\right]\,.
\end{align}
Die beiden Matrizen zusammen bilden die Steifigkeitsmatrix eines Kabelelements, s. Abb. \ref{CableElement},
\bfo
\vek K_e = \frac{EA}{\ell}\left[
\begin{array}{r r r r}
 c^2 & -c\cdot s & -c^2 & c \cdot s \\
 -c\cdot s & s^2 & c\cdot s & -s^2 \\
 -c^2 & c\cdot s & c^2 & -c\cdot s \\
 c\cdot s & -s^2 & -c\cdot s & s^2
\end{array}
\right] + \frac{S}{l}\left[
\begin{array}{r r r r}
 s^2 & c\cdot s & -s^2 & -c\cdot s \\
 c\cdot s & c^2 & -c\cdot s & -c^2 \\
 -s^2 & -c\cdot s & s^2 & c\cdot s \\
 -c\cdot s & -c^2 & c\cdot s & c^2
\end{array}
\right]\,.
\nn \\
\efo
In der Praxis bestimmt man $S$ iterativ. Man rechnet mit einem gesch\"{a}tzten $S$ und kontrolliert, ob die L\"{a}nge der Seilkurve gleich der ungedehnten L\"{a}nge $L$ plus der Dehnung des Seils ist
\begin{align}
\int_0^{\,l} ds  = \int_0^{\,l} \sqrt{1 + (w')^2}\,dx = L + \int_0^{\,l} \frac{S}{EA}\,ds\,.
\end{align}
%----------------------------------------------------------------------------------------------------------
\begin{figure}[tbp]
\centering
\if \bild 2 \sidecaption \fi
\includegraphics[width=.69\textwidth]{\Fpath/CABLEELEMENT}
\caption{Kabelelement}
\label{CableElement}%
\end{figure}%
%----------------------------------------------------------------------------------------------------------
Die Transformation der Steifigkeitsmatrix des Balkenelements beruht auf der Matrix
\begin{align}\label{Eq72}
\vek T =
\left[\barr{r @{\hspace{4mm}}r @{\hspace{4mm}}r @{\hspace{4mm}}r @{\hspace{4mm}}r @{\hspace{4mm}}r} c &-s & 0 & 0 & 0 &0\\
s & c & 0 & 0 & 0 &0\\
0 & 0 & 1 & 0 & 0 &0\\
0 & 0 & 0 & c &-s &0\\
0 & 0 & 0 & s & c &0\\
0 & 0 & 0 & 0 & 0 &1
\earr \right]\,.
\end{align}
Aus der Matrix
\begin{align}\label{Eq167}
\vek K^e = \left[ \barr{c c c c c c} \displaystyle{\frac{EA}{\ell}} & \displaystyle{0} & \displaystyle{0} &-\displaystyle{\frac{EA}{\ell}} & \displaystyle{0} & \displaystyle{0} \vspace{0.2cm}\\
\displaystyle{0} & \displaystyle{\frac{12\,EI}{\ell^3}} & - \displaystyle{\frac{6\,EI}{\ell^2}} &\displaystyle{0} &- \displaystyle{\frac{12\,EI}{\ell^3}} &  - \displaystyle{\frac{6\,EI}{\ell^2}} \vspace{0.2cm}\\
\displaystyle{0} & - \displaystyle{\frac{6\,EI}{\ell^2}} & \displaystyle{\frac{4\,EI}{\ell}} & \displaystyle{0}  & \displaystyle{\frac{6\,EI}{\ell^2}} & \displaystyle{\frac{2\,EI}{\ell}}\vspace{0.2cm}\\
-\displaystyle{\frac{EA}{\ell}} & \displaystyle{0} & \displaystyle{0} &\displaystyle{\frac{EA}{\ell}} & \displaystyle{0} & \displaystyle{0} \vspace{0.2cm}\\
\displaystyle{0} & -\displaystyle{\frac{12\,EI}{\ell^3}} & \displaystyle{\frac{6\,EI}{\ell^2}} &\displaystyle{0}  &\displaystyle{\frac{12\,EI}{\ell^3}} &   \displaystyle{\frac{6\,EI}{\ell^2}}\vspace{0.2cm}\\
\displaystyle{0} & - \displaystyle{\frac{6\,EI}{\ell^2}} & \displaystyle{\frac{2\,EI}{\ell}} & \displaystyle{0}  & \displaystyle{\frac{6\,EI}{\ell^2}} & \displaystyle{\frac{4\,EI}{\ell}} \earr \right]
\end{align}
wird die Matrix
\renewcommand{\arraystretch}{2.2}

\begin{multline}\label{Eq166}
\hspace*{-3mm}  \vek K_e ={\vek  T}^T{\vek  K}^e{\vek  T}= \\[.5cm]
  \footnotesize{\left[\begin{array}{c c c c c c} c^2\,\frac{\displaystyle
          EA}{\displaystyle \ell}+s^2\, 12\frac{\displaystyle
          EI}{\displaystyle \ell^3}&
        -cs\, \frac{\displaystyle
          EA}{\displaystyle \ell}+cs\, 12\frac{\displaystyle EI}{\displaystyle
          \ell^3}&
        -s\, 6\frac{\displaystyle EI}{\displaystyle
          \ell^2}& -c^2\,\frac{\displaystyle EA} {\displaystyle
          \ell}-s^2\, 12\frac{\displaystyle EI}{\displaystyle
          \ell^3}& cs\,\frac{\displaystyle EA}{\displaystyle
          \ell}-cs\,12\frac{\displaystyle EI}{\displaystyle \ell^3}
        &
        -s\,6\frac{\displaystyle EI}{\displaystyle \ell^2}\\
        -cs\, \frac{\displaystyle
          EA}{\displaystyle \ell}+cs\, 12\frac{\displaystyle EI}{\displaystyle
          \ell^3}&
        s^2\, \frac{\displaystyle EA}{\displaystyle \ell}+c^2\, 12\frac{\displaystyle EI}{\displaystyle \ell^3}&
        -c\, 6\frac{\displaystyle EI}{\displaystyle \ell^2}&
        cs\,\frac{\displaystyle EA} {\displaystyle \ell}-cs\, 12\frac{\displaystyle EI}{\displaystyle \ell^3}&
        -s^2\,\frac{\displaystyle EA}{\displaystyle \ell}-c^2\,12\frac{\displaystyle EI}{\displaystyle \ell^3}&
        -c\,6\frac{\displaystyle EI}{\displaystyle \ell^2}\\
        -s\, 6\frac{\displaystyle EI}{\displaystyle
          \ell^2}&-c\, 6\frac{\displaystyle EI}{\displaystyle \ell^2}&
        4\frac{\displaystyle EI}{\displaystyle \ell}&
        s\, 6\frac{\displaystyle EI}{\displaystyle \ell^2} &
        c\,6\frac{\displaystyle EI}{\displaystyle \ell^2} &
        2\frac{\displaystyle EI}{\displaystyle \ell}\\
        -c^2\,\frac{\displaystyle EA} {\displaystyle
          \ell}-s^2\, 12\frac{\displaystyle EI}{\displaystyle
          \ell^3}&cs\,\frac{\displaystyle EA} {\displaystyle \ell}-cs\, 12\frac{\displaystyle EI}{\displaystyle \ell^3}&s\,6\frac{\displaystyle EI}{\displaystyle \ell^2}&
        c^2\,\frac{\displaystyle EA} {\displaystyle \ell}+s^2\, 12\frac{\displaystyle EI}{\displaystyle \ell^3}&
        -cs\,\frac{\displaystyle EA}{\displaystyle \ell}+cs\,12\frac{\displaystyle EI}{\displaystyle \ell^3}&
        s\,6\frac{\displaystyle EI}{\displaystyle \ell^2}\\
        cs\,\frac{\displaystyle EA}{\displaystyle
          \ell}-cs\,12\frac{\displaystyle EI}{\displaystyle \ell^3}&-s^2\,\frac{\displaystyle EA}{\displaystyle \ell}-c^2\,12\frac{\displaystyle EI}{\displaystyle \ell^3}&c\,6\frac{\displaystyle EI}{\displaystyle \ell^2}&-cs\,\frac{\displaystyle EA}{\displaystyle \ell}+cs\,12\frac{\displaystyle EI}{\displaystyle \ell^3}&
        s^2\,\frac{\displaystyle EA}{\displaystyle \ell}+c^2\,12\frac{\displaystyle EI}{\displaystyle \ell^3}&
        c\,6\frac{\displaystyle EI}{\displaystyle \ell^2}\\
        -s\,6\frac{\displaystyle EI}{\displaystyle \ell^2}&-c\, 6\frac{\displaystyle EI}{\displaystyle \ell^2}&2\frac{\displaystyle EI}{\displaystyle \ell}&s\,6\frac{\displaystyle EI}{\displaystyle \ell^2}&c\,6\frac{\displaystyle EI}{\displaystyle \ell^2}&
        4\frac{\displaystyle EI}{\displaystyle \ell}
      \end{array}\right]}
\end{multline}
\\[0.5cm]

%%%%%%%%%%%%%%%%%%%%%%%%%%%%%%%%%%%%%%%%%%%%%%%%%%%%%%%%%%%%%%%%%%%%%%%%%%%%%%%%%%%%%%%%%%%%%%%%%%%
\textcolor{sectionTitleBlue}{\subsection{N\"{a}herungen}}
Bei den Steifigkeitsmatrix f\"{u}r elastisch gebettete Balken und f\"{u}r die Theorie zweiter Ordnung benutzt man in der Praxis oft N\"{a}herungen, indem man die Wechselwirkungsenergien
\begin{alignat}{2}
k_{ij} &= \int_0^{\,\ell} (EI\,\Np_i''\,\Np_j'' + c\,\Np_i\,\Np_j)\,dx \qquad &&\text{elast. gebetteter Balken} \\
k_{ij} &= \int_0^{\,\ell} (EI\,\Np_i''\,\Np_j'' + P\,\Np_i'\,\Np_j')\,dx \qquad &&\text{Th. II. Ordnung}
\end{alignat}
mit den vier Balken-Einheitsverformungen $\Np_i(x)$ (\ref{Phi1Bis4}) des normalen Balken formuliert und nicht mit den exakten Einheitsverformungen. F\"{u}r einen elastisch gelagerten Balken ergibt das die Matrix
\bfo\label{N1}
\tilde{\vek K}^e = \frac{EI}{\ell^3} \left[
\begin{array}{r r r r}
 12 & -6\,\ell & -12 &-6\,\ell \\
 -6\,\ell & 4\,\ell^2 & 6\,\ell &2\,\ell^2 \\
 -12 & 6\,\ell & 12 & 6\,\ell \\
 -6\,\ell &2\,\ell^2 &6\,\ell &4\,\ell^2
 \end{array}
  \right] + \frac{c}{420}
\left[ \begin{array}{r r r r}
 156\,\ell  & -22\,\ell^2  & 54\,\ell  &13 \,\ell^2  \\
 -22\,\ell^2 & 4\,\ell^3  & -13 \,\ell^2 &-3\,\ell^3  \\
 54\,\ell & 13 \,\ell^2 & 156\,\ell  & 22\,\ell^2  \\
 13 \,\ell^2 &-3\,\ell^3 &22\,\ell^2 &4\,\ell^3
 \end{array}  \right]\nn \\
\efo
und f\"{u}r einen Druckstab die Matrix
\bfo\label{IIN} \tilde{ \vek K}^e =
\frac{EI}{\ell^3}\left[ \begin{array}{r r r r}
 12 & -6\ell & -12 &-6\ell \\
 -6\ell & 4\ell^2 & 6\ell &2\ell^2 \\
 -12 & 6\ell & 12 & 6\ell \\
 -6\ell &2\ell^2 &6\ell &4\ell^2
 \end{array}
  \right] - \frac{P}{30\,\ell}
 \left[ \begin{array}{r r r r}
 36 & -3\,\ell & -36 & -3\,\ell \\
 -3\,\ell & 4\ell^2 & 3\,\ell &-\ell^2 \\
 -36 & 3\ell & 36 & 3\,\ell \\
 -  3\,\ell &-\ell^2 &3\,\ell &4\,\ell^2
 \end{array}
  \right]\,.
\efo
Die zweite Matrix ist die sogenannte {\em geometrische Elementsteifigkeitsmatrix\/}\index{geometrische Elementsteifigkeitsmatrix}.

Auch f\"{u}r die Steifigkeitsmatrix eines {\em Timoshenko Balkens\/}\index{Timoshenko Balken} gibt es solche N\"{a}herungen, \cite{Ha5} Chapter 3.5 und ebenso f\"{u}r die W\"{o}lbkrafttorsion, \cite{Kindmann}.

%%%%%%%%%%%%%%%%%%%%%%%%%%%%%%%%%%%%%%%%%%%%%%%%%%%%%%%%%%%%%%%%%%%%%%%%%%%%%%%%%%%%%%%%%%%%%%%%%%%
\textcolor{sectionTitleBlue}{\subsection{Schwache und starke Einflussfunktionen}}
Noch ein Wort zu diesem Thema: Eine Einflussfunktion ist ein Ausdruck, in den man etwas einsetzt, und man bekommt eine Durchbiegung, ein Moment, etc. heraus. Ist es eine schwache Einflussfunktion, dann setzt man das Moment aus der Belastung ein und ist es eine starke Einflussfunktion, dann setzt man die Belastung selbst ein. Akzeptiert man diese Definition, dann  gilt, dass es keine schwachen Einflussfunktionen f\"{u}r Schnittgr\"{o}{\ss}en wie Momente und Querkr\"{a}fte gibt.

Was man allerdings machen kann, ist, dass man z.B. eine Folge $G_2^{\varepsilon}$ von glatten Funktionen definiert, die gegen einen Knick konvergieren, so dass am Ende aus der Wechselwirkungsenergie das Moment herausspringt, s. \cite{Ha6} S. 67 {\em \glq A sequence that converges to $G_1$'\/}, (dort $N(x)$, hier $M(x)$)
\begin{align}
\lim_{\varepsilon \to 0} a(w,G_2^\varepsilon) = M(x)\,.
\end{align}
Nehmen wir ein Einfeldtr\"{a}ger und machen wir's unkompliziert, und ersetzen den Grenzprozess durch Dirac Deltas. Die Einflussfunktion $G_2$ f\"{u}r das Moment ist ein Dreieck. Die erste Ableitung des Dreiecks ist eine Sprungfunktion und die zweite Ableitung ist ein Dirac Delta $\delta_0$ und so liefert die  Wechselwirkungsenergie das gew\"{u}nschte Ergebnis
\begin{align}
a(w, G_2) = \int_0^{\,l} M(y)\,\delta_0(y-x) \,dy = M(x)\,.
\end{align}
Bei der Querkraft ist es dasselbe. Die Einflussfunktion ist eine Scherbewegung, s. Abb. \ref{U117} d. Die erste Ableitung ist ein $\delta_0$ und die zweite Ableitung ist ein $\delta_1$, was das Resultat
\begin{align}
a(w, \delta_1) = \int_0^{\,l} M(y)\,\delta_1(y-x)\,dy = M'(x) = V(x)
\end{align}
ergibt. Auf diesem Wege geht es also schon, aber es sind keine Ergebnisse in dem Sinne, dass man sie als Formeln im Betonkalender abdrucken k\"{o}nnte. Der Grenzprozess liefert ein Ergebnis, aber operiert man mit der \glq geknickten\grq{} Zielfunktion $G_2$ der Folge $G_2^{\varepsilon}$, dann ist das Ergebnis, wenn man das Loch $N_{\varepsilon}(x)$ um den Knick schlie{\ss}t, $\Omega_{\varepsilon} = \Omega - N_{\varepsilon}(x) \to \Omega$, null
\begin{align}
\lim_{\varepsilon \to 0} a(w,G_2^\varepsilon) = M(x) \qquad \lim_{\varepsilon \to 0}\,a(w,G_2)_{\Omega_\varepsilon} = 0\,.
\end{align}
Und noch eine Bemerkung zur Beobachtung, dass man mit finiten Elementen Spannungen (scheinbar) mit schwachen Einflussfunktionen berechnen kann, s. S. \pageref{EE7Equationforz}. Das sieht nur so aus,  denn in Wirklichkeit berechnen auch finite Elemente Spannungen, wie etwa die Normalkraft in einem Stab, mit einer starken Einflussfunktion
\begin{align}\label{Eq155}
N_h(x) = \int_0^{\,l} G_1^h(y,x)\,p(y)\,dy = \int_0^{\,l} \sum_i\,j_i\,\Np_i(y)\,p(y)\,dy = \vek j^T\,\vek f\,,
\end{align}
nur ist es so, dass man das wegen $\vek f = \vek K\,\vek u$ in die schwache Form
\begin{align}
N_h(x) = \vek j^T\,\vek f = \vek j^T\,\vek K\,\vek u
\end{align}
umschreiben kann, was dann so aussieht, als k\"{o}nnten finite Elemente $N_h(x)$ mit einer schwachen Einflussfunktion berechnen.

%%%%%%%%%%%%%%%%%%%%%%%%%%%%%%%%%%%%%%%%%%%%%%%%%%%%%%%%%%%%%%%%%%%%%%%%%%%%%%%%%%%%%%%%%%%%%%%%%%%
\textcolor{sectionTitleBlue}{\subsection{Wie kommt der Einbettungssatz zu seinem Namen?}}\index{Sobolevscher Einbettungssatz}
So, wie man die Einwohner einer Stadt nach verschiedenen Kriterien ordnen kann, {\em Alter, Gr\"{o}{\ss}e, $\ldots$\/}, so kann man auch die Biegefl\"{a}chen $w(\vek x)$ einer Platte $\Omega$ in verschiedener Weise klassifizieren. Eine dieser m\"{o}glichen Skalen ist die sogenannte {\em Sobolevnorm\/}\index{Sobolevnorm}. Die Sobolevnorm der Ordnung $m = 2$ einer Biegefl\"{a}che $w$ ist der Ausdruck
\begin{align}\label{Eq71}
\|w\|_2 = \sqrt{\int_{\Omega} (w^2 + w,_x^2 + w,_y^2 + w,_{xx}^2 + w,_{xy}^2 + w,_{yy}^2) \,d\Omega}\,.
\end{align}
Man kann nach diesem Muster Sobolevnormen $\|w\|_m$ beliebig hoher Ordnung $m$ definieren: Die Ableitungen bis zur Ordnung $m$ werden quadrat-integriert und aus der Summe die Wurzel gezogen.
Die Funktionen, die eine endliche Sobolevnorm der Ordnung $m$ haben, bilden den Sobolevraum\footnote{Zu 99\% sind das die Standardfunktionen $\sin(x), x^3\cdot  y^2, e^x  \ldots$} $H^m(\Omega)$.

Eng damit verwandt ist der sogenannte {\em Energieraum\/}, das sind alle Funktionen, die eine endliche Biegeenergie/Verzerrungsenergie $|a(u,u)| < \infty$ haben.

Es zeigt sich nun, dass der Energieraum einer Platte mit dem Sobolevraum $H^2$ identifiziert werden kann, und der Energieraum einer Scheibe mit $\vek H^1 = H^1 \times H^1$ (horizontale $u_x$ und vertikale Verschiebung $u_y$). Weil endliche Energie endliche Sobolevnormen $\|w\|_2$ bzw. $\|\vek u\|_1$ bedeutet, behandelt man die Begriffe Sobolevnorm und Energienorm\index{Energienorm} wie Synonyme\footnote{Technisch ist es so, dass die Energienorm $\|w\|_E = \sqrt{a(w,w)}$ und die Sobolevnorm $\|w\|_m$ \"{a}quivalente Normen sind, wenn die Starrk\"{o}rperbewegungen ausgeschlossen sind.}.

Der russische Mathematiker Sobolev hat gezeigt, dass der Raum $H^m(\Omega)$ in den Raum $C(\Omega)$ der stetigen Funktionen \"{u}ber $\Omega$ eingebettet ist,
\begin{align}
H^m(\Omega) \subset C(\Omega)
\end{align}
und die Einbettung sogar stetig ist
\begin{align} \label{Eq24}
\text{max}\, |w| < c \cdot \|w\|_m\,,
\end{align}
wenn die folgende Ungleichung gilt
\begin{align} \label{Eq22}
m > \frac{n}{2}\,,
\end{align}
wenn also der Index $m$ des Sobolevraums gr\"{o}{\ss}er als die \glq halbe Dimension\grq{} des Raums ist. Bei einer Platte ist $n = 2$. Der Wert $\text{max}\, |w|$ ist die Norm von $w$ auf $C(\Omega)$ und (\ref{Eq24}) bedeutet, dass die Norm des Bildes kleiner als die Ausgangsnorm mal einem Skalenfaktor $c$ ist. Wenn das gilt, dann sagt man, dass die Abbildung, also hier $w \in H^2(\Omega) \to w \in C(\Omega)$ stetig ist. Die Zahl $c$ in der Ungleichung (\ref{Eq24}) ist eine feste Konstante, die nur von der Gestalt von $\Omega$ abh\"{a}ngt.

Das Interessante an diesem Ergebnis ist der \"{U}bergang vom Integral zum Punkt. Die Sobolevnorm ist ja ein integrales Ma{\ss}, aber wenn die Ungleichung (\ref{Eq22}) gilt, und f\"{u}r eine Platte, $m = 2, n = 2$, gilt sie, dann ist die maximale Auslenkung der Platte durch die Biegeenergie begrenzt. Geht die Biegeenergie gegen null, dann geht auch die Durchbiegung der Platte gegen null -- \"{u}berall, in jedem Punkt! {\em Ein Integral majorisiert Punktwerte!\/}

Bei einer Scheibe ist das anders. Ihr Energieraum ist $\vek H^1$, aber die Zahlen $m = 1, n = 2$ erf\"{u}llen die Ungleichung (\ref{Eq22}) nicht, und daher garantiert eine endliche Energie $\|\vek u\|_1 < \infty$ nicht, dass die Einbettung stetig ist in dem Sinne, dass wenn zwei Verschiebungsfelder $\vek u_1$ und $\vek u_2$ eine kleinen Abstand in der Energie haben, dass dann auch ihre maximalen Verschiebungen nahezu gleich sind, was bei einer Platte gilt
\begin{align}
\text{max}\,  |w_1 - w_2|  < c \cdot \|w_1 - w_2\|_2\,.
\end{align}

Wenn der Leser noch etwas Geduld hat, k\"{o}nnen wir diese Ideen noch etwas weiter verfolgen.

Menschen kann man nach ihrem Alter ($A$) klassifizieren oder nach ihrem Gewicht ($G$). Auf der Menge $A$ ist das Funktional \glq Schuhgr\"{o}{\ss}e\grq{} nicht stetig, weil ein kleiner Abstand im Alter nicht garantiert, dass auch die Schuhgr\"{o}{\ss}en \"{a}hnlich ausfallen. Auf der Menge $G$ erwarten wir hingegen (n\"{a}herungsweise) einen solchen Zusammenhang.

Ein Funktional ist also dann stetig, wenn aus einem kleinen Abstand im Input eine kleine Differenz im Ergebnis folgt
\begin{align}
|J(u_1) - J(u_2) | < c \cdot \|u_1 - u_2\|\,.
\end{align}
Genauer gesagt, wenn es eine globale Konstante $c$ gibt, die f\"{u}r alle Funktionen $u$ in der Ausgangsmenge gilt. Stetigkeit ist also immer davon abh\"{a}ngig, welche Abstandsma{\ss}e man auf der Ausgangsmenge und der Zielmenge hat. Setzt man $u_2 = 0$, dann steht $|J(u)| < c \cdot \|u\|$ da. \\

\hspace*{-12pt}\colorbox{highlightBlue}{\parbox{0.98\textwidth}{Wenn ein lineares Funktional stetig ist, dann ist es auch beschr\"{a}nkt.}}\\

Bei einer Kirchhoffplatte ist das Punktfunktional
\begin{align}
J(w) = w(\vek x) \qquad \text{Durchbiegung in einem Punkt $\vek x$}
\end{align}
auf $H^2(\Omega)$ ein stetiges und beschr\"{a}nktes Funktional, weil die stetige Einbettung (\ref{Eq24}) garantiert, dass die Differenz in zwei Werten durch die Differenz in der Energie nach oben begrenzt ist
\begin{align}
|J(w_1) - J(w_2)| = |w_1(\vek x) - w_2(\vek x)| < c \cdot \|w_1 - w_2\|_2\,,
\end{align}
aber bei einer Scheibe gilt dies nicht mehr. Die Differenz in der horizontalen Verschiebung zweier Verschiebungsfelder in einem Punkt $\vek x$
\begin{align}
|J(\vek u) - J(\vek v)| = |u_{x}(\vek x) - v_{x}(\vek x)|  \nless  c \cdot \|\vek u - \vek v\|_1
\end{align}
l\"{a}sst sich nicht {\em f\"{u}r alle (!)\/} $\vek u \in \vek H^1(\Omega)$ durch die Sobolevnorm $\|\vek u - \vek v\|_1$ nach oben begrenzen.

Die Betonung liegt auf {\em f\"{u}r alle\/}. Die Membranbiegefl\"{a}che $u = - \ln (- \ln^{-1} r)$ hat im Punkt $r = 0$ den Wert unendlich, $J(u) = \infty$, aber in einem Kreis $\Omega$ mit Radius $R = 0.5$ um den Nullpunkt ist die $H^1$-Norm beschr\"{a}nkt, s. \cite{Ha6} S. 98. Es gibt demnach keine Zahl $c $ so, dass
\begin{align}
J(u) = u(\vek x) = \infty < c \cdot \|u\|_1 \qquad\text{?}
\end{align}
Auf $H^1(\Omega)$ ist $J(u)$ also kein stetiges und beschr\"{a}nktes Funktional. Ein Gegenbeispiel reicht aus, um dieses Urteil f\"{a}llen zu k\"{o}nnen.

Bei einem Stab, $n = 1$, $m = 1$, ist der Energieraum $H^1(0,l)$, und daher gilt dort
\begin{alignat}{2}
J(u) &= u(x) \qquad &m - i = 1 - 0 = 1 > \frac{1}{2} \qquad &\text{stetig} \\
J(u) &= u'(x)\qquad &m - i = 1 - 1 = 0 \ngtr \frac{1}{2} \qquad &\text{unstetig}
\end{alignat}
Der Energieraum eines Balkens, $n = 1$, $m = 2$, ist der $H^2(0,l)$, und daher gilt, in der Reihenfolge $i = 0, 1, 2, 3$,
\begin{align}\label{Eq21}
\underbrace{J(w) = w(x) \qquad J(w) = w'(x)}_{stetig} \qquad \underbrace{J(w) = w''(x) \qquad J(w) = w'''(x)}_{nicht\,\, stetig}
\end{align}
{\em Wenn ein Funktional stetig ist, dann ist die Energie der Greenschen Funktion endlich, sonst unendlich und dann ist es auch die \"{a}u{\ss}ere Arbeit $A_a = Weg \times Kraft$. Einer der beiden Gr\"{o}{\ss}en muss im Aufpunkt unendlich sein\/}.\\

Die Greensche Funktion f\"{u}r das (unstetige) Funktional $J(u) = EA\,u'(x)$ der Normalkraft $N(x)$ in einem Stab ist eine Einheitsversetzung, die sich nur unter dem Einsatz von unendlich gro{\ss}en Kr\"{a}ften erzeugen l\"{a}sst
\begin{align}
\int_0^{\,l} \frac{N^2}{EA}\,dx = \infty \qquad N = \text{Normalkraft aus Versatz}\,.
\end{align}
Wir folgen hier dem Mathematiker, der die Sprungfunktion in eine Fourierreihe entwickelt, s. S. \pageref{Fourierreihe}, und nicht dem Ingenieur, der erst ein Normalkraftgelenk einbaut und dann das Gelenk spreizt.

Nun noch ein Kommentar zur Ungleichung aus Kapitel 1, (\ref{Eq23}), die wir hier wiederholen
\begin{align}
m - i > \frac{n}{2}\,.
\end{align}
Wenn man eine Funktion, die in $H^m(\Omega)$ liegt, differenziert, dann liegt ihre Ableitung (m\"{o}glicherweise) nur noch in $H^{m-1}(\Omega)$. Dies erkl\"{a}rt, warum wir von $m$ den Index $i = 0, 1, 2, 3$ des Dirac Delta $\delta_i$ abziehen, also das Signal wie oft das Dirac Delta die Funktion $u$ differenziert
\begin{align}
J(u) = \int_{\Omega} \delta_i(\vek y - \vek x)\,u(\vek y)\,d\Omega_{\vek y} \cong \text{Ableitung $i$-ter Ordnung}\,.
\end{align}

\begin{remark}
Nicht stetig hei{\ss}t in (\ref{Eq21}): $J(w) = - EI\,w''(x) = M(x)$ ist auf $H^2(0,l)$ kein stetiges {\em Funktional\/}. Es gibt kein $c$ so, dass {\em f\"{u}r alle\/} Biegelinien $w \in H^2(0,l)$ die Ungleichung
\begin{align}\label{Eq130}
|J(w)| = |M(x)| < c \cdot \|w\|_2
\end{align}
gilt. Eine Biegelinie $w$ mit einer logarithmischen Singularit\"{a}t im Momentenverlauf, $M(x) = \ln (x-x_0)^2$ in einem Punkt $x_0 \in (0,l)$, hat eine endliche Norm, $\|w\|_2 < \infty$, aber $M(x_0) = \infty$.

%-----------------------------------------------------------------
\begin{figure}[tbp]
\if \bild 2 \sidecaption \fi
\makebox[\textwidth]{%
\includegraphics[width=.8\textwidth]{\Fpath/FELDLINIENA}}
\caption{In der N\"{a}he der Einzelkraft sind die Kraftlinien so dicht gepackt, dass das Material einer Scheibe zu flie{\ss}en anf\"{a}ngt, w\"{a}hrend eine Linienlast die hohe Konzentration der Kraftlinien in einem Punkt vermeidet---die Energie bleibt endlich, \cite{Ha5}} \label{Feldlinien}
\end{figure}%
%-----------------------------------------------------------------

Man kann es auch so lesen: Das singul\"{a}re $M(x)$ liegt in $H^0(0,l)$, weil man $M(x)^2$ integrieren kann, aber die Norm auf $H^0$, also das Integral $\|M\|_0 = (M,M)^{1/2}$, kann  Punktwerte nicht majorisieren, $\text{max}\, |M| < c \cdot \|M\|_0$ gilt nicht, weil die Einbettung von $H^0(0,l)$ in $C^0(0,l)$ nicht stetig ist, es gibt Ausrei{\ss}er und einer davon ist das obige Moment. Noch einfacher: \\

\hspace*{-12pt}\colorbox{highlightBlue}{\parbox{0.98\textwidth} {Daraus, dass man eine Funktion $M(x)$ quadrat-integrieren kann, folgt nicht, dass die Funktion $M(x)$ auf dem Intervall $(0,l)$ beschr\"{a}nkt ist.}}\\

Wenn aber auch die Ableitung $M'(x)$ quadrat-integrierbar ist, wenn $M(x)$ also in $H^1(0,l)$ liegt, dann ist der Schluss zul\"{a}ssig, weil die Ungleichung $m > n/2$, setze $m = 1, n = 1$, dann erf\"{u}llt ist.

Die Ableitung $M'(x) = 2/(x-x_0)$ von $M(x) = \ln \,(x-x_0)^2$ ist aber nicht quadrat-integrierbar, das \glq Schlupfloch\grq{} $M \in H^1(0,l)$ steht also nicht zur Verf\"{u}gung.

\end{remark}

\begin{remark}
Die mathematische Theorie der finiten Elemente hat sich parallel zu der Anwendung der finiten Elemente in den Ingenieurwissenschaften entwickelt und liegt heute in abgeschlossener Form vor.
Uns interessiert hier ein Satz aus der Theorie der schwachen L\"{o}sungen:\\

 {\em Eine Einflussfunktion
\begin{align}
J(u) = \int_{0}^{l}G(y,x)\,p(y)\,dy
\end{align}
existiert genau dann, wenn das Funktional $J(u)$ linear und beschr\"{a}nkt ist, $|J(u)| < c\,\|u\|$\/}.\\


Folgt man dieser Vorgabe, dann d\"{u}rfte es keine Einflussfunktion $G_2(y,x)$ f\"{u}r das Moment in einem Balken geben, weil ja das Funktional $J(w) = M(x)$ auf dem Energieraum $H^2(0,l)$ unbeschr\"{a}nkt ist. Dies steht aber im Widerspruch zum Vorgehen des Ingenieurs, der in der Mitte des Balkens einfach einen Knick erzeugt und mit dieser Einflussfunktion das exakte Biegemoment erh\"{a}lt.

Der obige Satz meint aber nur, dass es keine Einflussfunktion mit {\em endlicher Energie\/} gibt, keine Einflussfunktion $G_2(y,x) \in H^2(0,l)$, was sich, wenn wir nachrechnen s. S. \pageref{Fourierreihe}, best\"{a}tigt: Die abgeknickte Biegelinie hat eine unendliche Energie, liegt nicht in $H^2(0,l)$. Jetzt kann man nat\"{u}rlich sagen, was liegt daran: Wenn Mathematiker so spitzfindig sind und so enge Grenzen ziehen, dann muss das den Ingenieur nicht k\"{u}mmern.

Es zeigt sich hier aber eine Schw\"{a}che der mathematischen Theorie der finiten Elemente, nach der ja die FEM eine {\em Energieraummethode\/} ist und L\"{o}sungen, die unendliche Energie haben, liegen au{\ss}erhalb dieser Theorie. Wie will man den Abstand in der Energie minimieren,
\begin{align}
a(u-u_h,u-u_h) = \| u - u_h\|^2 \qquad \rightarrow \,\,\text{Minimum}\,,
\end{align}
wenn die Energie der exakten L\"{o}sung, $\|u\| = \infty$, unendlich ist?

Es ergibt sich so eine kuriose Situation: Ein FE-Programm berechnet alle Schnittgr\"{o}{\ss}en mit Einflussfunktionen, die N\"{a}herungsl\"{o}sungen von {\em schlecht gestellten Problemen\/} sind, schlecht gestellt, weil die exakten L\"{o}sungen nicht im L\"{o}sungsraum, im Energieraum liegen -- alle Einflussfunktionen f\"{u}r Schnittgr\"{o}{\ss}en haben eine unendliche Energie.

Eine Situation, vor der jeder Mathematiker warnt: Man soll sich doch bitte vorher davon \"{u}berzeugen, dass \"{u}berhaupt eine L\"{o}sung existiert, bevor man den Computer startet...
\end{remark}
%--------------------------------------------------------------------------------------
\begin{table}\caption{{Endliche ($\checkmark$) und unendliche ($\infty$) Energie}}\label{janein}
%\vspace{0.2cm}
\begin{tabular}{r  c  c  c  c}
\toprule \noalign{\smallskip}
\multicolumn{2}{c }{}  &    &    &   \\[-1.0cm]
\multicolumn{2}{c }{} Dimension & $n=1$       & $n=2$    &   $n=3$
\\ \multicolumn{2}{c }{}  &    &    &
\\[-0.6cm]\noalign{\hrule\smallskip} \multicolumn{2}{c}{}  &    &    &
\\[-0.9cm] \multicolumn{2}{c }{$m=1$}   & Seil, Stab, &  &   \\[-0.2cm]
\multicolumn{2}{c }{} & Timoshenko Balken & Reissner--Mindlin Platte & E-Th. \\
\noalign{\hrule\smallskip} & & & & \\[-0.8cm]
&Punktlasten & & &  \\[-0.1cm]
$i=0:$ & {\LARGE $\downarrow$} & $\large{\checkmark}$ & $\large{\infty}$ & $\large{\infty}$ \\[0.2cm]
$i=1:$ & \versetzung &  $\large{\infty}$ & $\infty$ & $\infty$ \\[0.2cm]
\noalign{\hrule\smallskip} \multicolumn{2}{c}{}  &    &    & \\[-0.5cm]
\multicolumn{2}{c}{$m=2$} & Euler--Bernoulli Balken  & Kirchhoff Platte & \\
\noalign{\hrule\smallskip} &  & & & \\[-0.6cm]
$i=0:$ & {\LARGE $\downarrow$} & $\large{\checkmark}$ & $\large{\checkmark}$ & \\[0.2cm]
$i=1:$ & {\LARGE $\curvearrowright$} & $\large{\checkmark}$ & $\large{\infty}$
&  \\[0.2cm] $i=2:$ & \knick & $\large{\infty}$ & $\large{\infty}$ & \\[0.2cm]
$i=3:$ & \versetzung & $\large{\infty}$ & $\large{\infty}$ & \\[0.2cm]
\bottomrule
\end{tabular}
\end{table}\label{TabelleSobolev}\index{Sobolev, Tabelle}
%%%%%%%%%%%%%%%%%%%%%%%%%%%%%%%%%%%%%%%%%%%%%%%%%%%%%%%%%%%%%%%%%%%%%%%%%%%%%%%%%%%%%%%%%%%%%%%%%%%
\textcolor{sectionTitleBlue}{\subsection{Punktlasten und ihre Energie}}\index{Punktlasten, Energie}
Zum Schluss sei noch eine Tabelle nachgetragen, in der verzeichnet ist, welche Punktlasten, $i = 0, 1, 2, 3$, also {\em Einzelkr\"{a}fte, Momente, Knicke\/} oder {\em Versetzungen\/} Verschiebungen mit endlicher Energie erzeugen.\index{Sobolevscher Einbettungssatz}

Die Tabelle \ref{janein} wertet die Bedingung $m -i > n/2$ aus, die erf\"{u}llt sein muss, damit die Energie endlich bleibt. Es ist $m$ die Ordnung der Energie

\bfo
m&=&1 \qquad\mbox{Timoshenko Balken, Reissner--Mindlin Platten, Scheiben, {\em 3-D solids\/}}\nn\\
m&=&2 \qquad\mbox{Euler--Bernoulli Balken, Kirchhoff Platten}\nn
\efo
Die Ordnung der Energie entspricht der h\"{o}chsten Ableitung in der Wechselwirkungsenergie $a(u,u)$. Sie ist immer halb so gro{\ss} wie die Ordnung $2\,m$ der Differentialgleichung.

Der {\em Timoshenko-Balken\/} ist der schubweiche Balken im Gegensatz zum schubstarren Euler-Bernoulli Balken $EI\,w^{IV} = p$. Schubtr\"{a}ger, wie kurze Konsolen, sind {\em Timoshenko-Balken\/}.

Abb. \ref{Feldlinien} illustriert, warum man Punktlasten besser in kurze Linienlasten umwandeln sollte.


%%%%%%%%%%%%%%%%%%%%%%%%%%%%%%%%%%%%%%%%%%%%%%%%%%%%%%%%%%%%%%%%%%%%%%%%%%%%%%%%%%%%%%%%%%%%%%%%%%%
\textcolor{sectionTitleBlue}{\subsection{Early Birds}}\index{early birds}
Wir kennen inzwischen neben der Arbeit von {\em Tottenham\/} (Southampton), \cite{Tottenham},  eine zweite zeitgleich erschienene Arbeit von {\em Kol\'{a}\v{r}\/} (Br\"{u}nn), \cite{Kolar}, beide aus 1970, die das Thema finite Elemente und Einflussfunktionen behandeln.

Wahrscheinlich gibt es noch andere, fr\"{u}he Arbeiten. F\"{u}r Hinweise w\"{a}ren wir dankbar.



%%%%%%%%%%%%%%%%%%%%%%%%%%%%%%%%%%%%%%%%%%%%%%%%%%%%%%%%%%%%%%%%%%%%%%%%%%%%%%%%%%%%%%%%%%%%%%%%%%%
\textcolor{chapterTitleBlue}{\chapter{Nachwort }}
%%%%%%%%%%%%%%%%%%%%%%%%%%%%%%%%%%%%%%%%%%%%%%%%%%%%%%%%%%%%%%%%%%%%%%%%%%%%%%%%%%%%%%%%%%%%%%%%%%%
Der Gedanke, ein solches Buch zu schreiben, besch\"{a}ftigte uns schon l\"{a}ngere Zeit, aber den endg\"{u}ltigen Ausschlag gab dann ein eher zuf\"{a}lliger Blick in ein Statikskriptum, in dem der Autor die Einflussfunktion f\"{u}r ein Biegemoment herleitete und dies auf eine (aus mathematischer Sicht) eher wunderliche Weise.

Um Balkenstatik und Anschauung unter einen Hut zu bringen, musste er, f\"{u}r unser Empfinden, die Anschauung schon arg strapazieren.

Unsere Kritik und unser Standpunkt wird sicherlich nicht von allen Kollegen geteilt, deswegen haben wir sie auch hier in den Anhang verbannt, weil wir hier mehr Raum haben, unsere Sicht der Dinge darzulegen und wir hoffen zumindest Verst\"{a}ndnis f\"{u}r unseren Standpunkt zu wecken. Diese Diskussion mag im Nachhinein auch unseren etwas axiomatischen Zugang rechtfertigen. Wir wollten Klarheit!

{\em Virtuelle Arbeit ist ein Begriff der Analytischen Mechanik bzw. der Technischen Mechanik und bezeichnet die Arbeit, die eine Kraft an einem System bei einer virtuellen Verschiebung verrichtet. Unter einer virtuellen Verschiebung versteht man eine Gestalt- oder Lage\"{a}nderung des Systems, die mit den Bindungen (z. B. Lager) vertr\"{a}glich und \glq instantan\grq{}, sonst aber willk\"{u}rlich und au{\ss}erdem infinitesimal klein ist. Das Prinzip der virtuellen Arbeit wird zur Berechnung des Gleichgewichts in der Statik und zum Aufstellen von Bewegungsgleichungen (d'Alembertsches Prinzip) verwendet.\/}

So wird in {\em Wikipedia\/} der Begriff der virtuellen Arbeit eingef\"{u}hrt, \cite{VA}. In \"{a}hnlichen S\"{a}tzen wird das {\em Prinzip der virtuellen Verr\"{u}ckungen\/}, der {\em Energieerhaltungssatz\/} und das {\em Prinzip der virtuellen Kr\"{a}fte\/} beschrieben.

Aber was ist bitte {\em \glq infinitesimal klein\grq{}\/}? Und wie passt zu dieser Forderung, dass die folgende Gleichung, die ja doch angeblich auf dem {\em Prinzip der virtuellen Verr\"{u}ckungen\/} beruht\footnote{Ein gelenkig gelagerter Tr\"{a}ger unter Gleichlast $p$}
\begin{align}\label{Eq82}
\delta A_a = \int_0^{\,l} p\,\delta w\,dx = \int_0^{\,l} \frac{M\,\delta M}{EI} \,dx = \delta A_i,
\end{align}
auch f\"{u}r eine virtuelle Verr\"{u}ckung wie $\delta w = \sin (\pi x/l)$ richtig ist, die, mit einer Amplitude von 1 m nun sicherlich nicht mehr klein ist. Ja die Amplitude k\"{o}nnte beliebig gro{\ss} sein, weil sie sich einfach herausk\"{u}rzt.

Und wieso kann man -- das ist eigentlich das Problem -- mathematische Resultate, wie die Gleichheit der beiden obigen Integrale, mit Prinzipien der Mechanik beweisen? Nur wenige Ingenieure verstehen \"{u}berhaupt, dass diese Frage doch berechtigt ist.

Man kann S\"{a}tze in einem Gebiet $A$ (der Mathematik) nicht mit S\"{a}tzen aus einem Gebiet $B$ (der Mechanik) beweisen. Das geht logisch nicht, wenn auch Ingenieure immer wieder dazu tendieren, weil ihnen die Arbeits- und Energieprinzipe der Statik lieb und teuer sind, wie das folgende Zitat aus einer Korrespondenz mit einem Kollegen belegen mag:

{\em \glq Die Energie- und  Arbeitss\"{a}tze sind Naturgesetze und daher fundamental. Denn wenn der Energieerhaltungssatz nicht gelten w\"{u}rde, k\"{o}nnte man bei jeder Form\"{a}nderung eines Tragwerks Arbeit bzw. Energie gewinnen: ein perpetuum mobile! Diese S\"{a}tze haben also zuerst einmal nichts mit der Mathematik zu tun, sondern bestehen an sich.'\/}

Und weil sie nichts mit der Mathematik zu tun haben, kann man mit ihnen kein mathematisches Ergebnis beweisen.

Die Eleganz der Mechanik, ihre Geschlossenheit, ihr innerer Reichtum, der ja nirgends so sichtbar wird, wie bei der mathematischen Formulierung, verf\"{u}hrt Ingenieure dazu, mathematische Resultate aus mechanischen Prinzipien \glq herzuleiten\grq{}, was aber nicht geht.

Unsere Bem\"{u}hungen um ein besseres Verst\"{a}ndnis der Grundlagen quittierte ein Kollege einmal mit dem Satz: \glq {\em Den Satz von Land kennt jeder Ingenieur, aber die zweite Greensche Identit\"{a}t?'\/} Was doch eigentlich nur beweist, dass der Kollege noch nie versucht hat, den Satz von Land herzuleiten, denn der Satz von Land beruht auf der zweiten Greenschen Identit\"{a}t.

Die Gr\"{u}ndungsv\"{a}ter der Statik m\"{u}ssen sehr gut Mathematik gekonnt haben, denn es gab ja noch keinen {\em Mohr\/}, keinen {\em Engesser\/}, dem man h\"{a}tte \"{u}ber die Schulter schauen k\"{o}nnen. Man musste alles selbst herleiten und das ging nur auf mathematischem Wege, \cite{Ku}.

Nachdem aber das Grundger\"{u}st stand, entdeckte man, wie sich fast spielerisch aus der Integralform des Gleichgewichts
\begin{align}
\int_0^{\,l} EI\,w^{IV}\,\delta w\,dx = \int_0^{\,l} p\,\delta w\,dx
\end{align}
Integrals\"{a}tze ergaben, die wir heute das {\em Prinzip der virtuellen Verr\"{u}ckungen\/}, das {\em Prinzip der virtuellen Kr\"{a}fte\/} und den {\em Energieerhaltungssatz\/} nennen. Und das passte alles so wunderbar zueinander, dass diese Prinzipe heute nach Meinung der Ingenieure das Fundament der Statik darstellen. Wenn der Ingenieur eine Gleichung auf das {\em Prinzip der virtuellen Verr\"{u}ckungen\/} zur\"{u}ckgef\"{u}hrt hat, dann hat er seiner Meinung nach die Gleichung bewiesen.

Die Schieflage,  in die die Statik und die Mechanik auf diese Art und Weise gekommen sind, erkennt man am besten an dem Thema virtuelle Verr\"{u}ckungen. {\em \glq Virtuelle Verr\"{u}ckungen m\"{u}ssen klein sein, oder, besser noch, infinitesimal klein sein\grq{}\/}. Wir haben auch schon gelesen, dass virtuelle Kr\"{a}fte angeblich klein sein m\"{u}ssen.

Zu welchen \glq Umwegen\grq{} das teilweise f\"{u}hrt, m\"{o}ge das folgende Beispiel aus einem Statikskript belegen, in dem die Einflussfunktion f\"{u}r ein Moment hergeleitet wird.
%----------------------------------------------------------------------------------------------------------
\begin{figure}[tbp]
\centering
\if \bild 2 \sidecaption \fi
\includegraphics[width=0.8\textwidth]{\Fpath/U120}
\caption{Einflussfunktion f\"{u}r ein Moment} \label{U120}
\end{figure}%
%----------------------------------------------------------------------------------------------------------

Der Autor baut hierzu ein Momentengelenk ein, f\"{u}gt zum besseren Verst\"{a}ndnis eine Zeichnung hinzu, s. Abb. \ref{U120}, in der er die durch eine  Spreizung \glq $\Delta \Np = 1$\grq{} ausgel\"{o}ste virtuelle Verr\"{u}ckung antr\"{a}gt, aber gleich darauf aufmerksam macht, dass die Zeichnung die Situation so darstelle, \glq wie man sie durch eine Lupe\grq{} sehe, denn in Wirklichkeit seien die Verr\"{u}ckungen infinitesimal klein. Seine Analyse f\"{u}hrt ihn dann auf das Ergebnis
\begin{align}
A_{1,2} = M \cdot \Delta \Np + P\cdot \delta w(x) = 0
\end{align}
oder aufgel\"{o}st nach $M = - P \cdot \delta w(x)/\Delta \Np$
womit sich
\begin{align}
\beta(x) = \frac{\delta w(x)}{\Delta \Np} = \frac{\delta w(x)}{1}
\end{align}
als die gesuchte Einflussfunktion ergibt. Hierzu bemerkt der Autor: \glq Zwar ist $\delta w(x)$ infinitesimal klein, aber der Quotient $\beta$ ist endlich gro{\ss}, da sich die virtuellen Verr\"{u}ckungen bei der Division herausk\"{u}rzen.\grq{}
%----------------------------------------------------------------------------------------------------------
\begin{figure}[tbp]
\centering
\if \bild 2 \sidecaption \fi
\includegraphics[width=0.8\textwidth]{\Fpath/U247}
\caption{{\em Satz von Betti\/}---Einflussfunktion f\"{u}r ein Moment \textbf{ a)} Tr\"{a}ger mit Belastung \textbf{ b)} dasselbe System unbelastet aber mit einer Spreizung $\tan \Np_l + \tan \Np_r = 1$ des Gelenks} \label{U247A}
\end{figure}%
%----------------------------------------------------------------------------------------------------------

Nun lautet die mathematische Definition der Spreizung \glq $\Delta \Np = 1$\grq{}
\begin{align} \label{Eq77}
\Delta \Np = \tan\,\Np_l + \tan\,\Np_r = 1\,,
\end{align}
und wenn man die Endtangenten derart verdreht, dann ist $\delta w(x)$ nicht mehr \glq klein\grq{}, und auch nicht beliebig skalierbar,  weil es ja doch nur  eine Kurve $\delta w(x)$ gibt, die die Bedingung (\ref{Eq77}) erf\"{u}llt und gleichzeitig die Lagerbedingungen des Tr\"{a}gers einh\"{a}lt!

Was der Autor wahrscheinlich meint ist, dass man das Gelenk beliebig spreizen kann, solange man nicht vergisst,  die dadurch ausgel\"{o}ste Verr\"{u}ckung des Balkens  $\delta w(x)$ durch die Gr\"{o}{\ss}e der Spreizung $\Delta \Np =  \tan\,\Np_l + \tan\,\Np_r$ zu dividieren. Dann bleibt das Ergebnis $\beta(x) = \delta w(x)/ \Delta \Np$ immer gleich, weil die Balkengleichung $EI\,w^{IV}(x)$ linear ist und dann kann $\delta w$ beliebig gro{\ss} oder klein sein.

Es geht aber auch ohne Lupe. In den Tr\"{a}ger wird ein Gelenk eingebaut, s. Abb. \ref{U247A}, um das innere Moment $M(x)$ \glq sichtbar\grq{} zu machen und der Tr\"{a}ger wird -- ohne Belastung -- darunter noch einmal angezeichnet aber so verschoben, dass die Spreizung im Gelenk genau $\tan \Np_l + \tan \Np_r = 1$ betr\"{a}gt.

Nach dem {\em Satz von Betti\/} gilt $\text{\normalfont\calligra B\,\,}(w_1,w_2) = A_{1,2} - A_{2,1} = 0 $ und weil die nicht vorhandenen Kr\"{a}fte am Tr\"{a}ger 2 null Ar\-beit auf den Wegen $w_1(x)$ leisten, $A_{2,1} = 0$, ist die Ar\-beit der Kr\"{a}fte am Tr\"{a}ger 1 auf den Wegen $w_2(x)$ somit ebenso null
\begin{align}
A_{1,2} = -M_l\,\tan\,\Np_l - M_r\,\tan\,\Np_r + P \cdot w_2(x) = - M \cdot 1 + P\cdot w_2(x) = 0
\end{align}
oder $ M = P\cdot w_2(x) $, was bedeutet, dass $w_2(x)$ die Einflussfunktion f\"{u}r $M(x)$ ist.\\

\hspace*{-12pt}\colorbox{highlightBlue}{\parbox{0.98\textwidth}{Das Grundproblem ist die Interpretation der Gleichungen. Geht es um die statische Bedeutung oder um den mathematischen Gehalt?}}\\

Bei einer Diskussion mit einem Kollegen hat der zweite Autor einmal vorgeschlagen, man k\"{o}nnte ja die Gleichung, \"{u}ber die diskutiert wurde, mit einer beliebigen Zahl multiplizieren. Worauf der Kollege den Kopf zur Seite neigte, die Augen zur Zimmerdecke richtete und einwand: {\em \glq Das k\"{o}nnen Sie nicht, die Zahl muss sehr klein sein, eben eine virtuelle Verr\"{u}ckung, ... ansons\-ten kann man das nicht!'\/} Der Kollege hat die Mechanik hinter der Gleichung gesehen, aber nicht die Mathematik.\label{Korrektur30}

Statische Probleme werden mit mathematischen Hilfsmitteln gel\"{o}st. Aber anders als der Mathematiker, der nie den Kreis seiner abstrakten Symbole verl\"{a}sst, muss der Ingenieur diese Grenze \"{u}berschreiten und die Mathematik auf reale Probleme anwenden. Das ist ein schwieriger Prozess, aber die Belohnung ist immens. Dass es m\"{o}glich ist, mit Mathematik reale Probleme zu l\"{o}sen, ist ein Wunder, wie Eugene Wigner erstaunt bemerkt hat, \cite{Wigner}.
%----------------------------------------------------------------------------------------------------------
\begin{figure}[tbp]
\centering
\if \bild 2 \sidecaption \fi
\includegraphics[width=.6\textwidth]{\Fpath/UE359A}
\caption{Der {\em cosinus\/} des Winkels $\gamma$ ist f\"{u}r die Effekte der speziellen Relativit\"{a}tstheorie verantwortlich} \label{UE359}
\end{figure}%%
%----------------------------------------------------------------------------------------------------------

Euler fand die Knicklast einer St\"{u}tze
\begin{align}
P_{krit} = \frac{\pi^2\,EI}{l^2}\,,
\end{align}
indem er den kleinsten Eigenwert $\lambda > 0$ einer Differentialgleichung bestimmte, aber es ist dann der Ingenieur, der den Mut haben muss, dieses mathematische Ergebnis auf die Wirklichkeit zu \"{u}bertragen, St\"{u}tzen danach zu dimensionieren.

Aus mathematischer Sicht ist {\em Heisenbergs Unsch\"{a}rferelation\/} eine Eigenschaft der {\em Fouriertransformation\/}, aber das wahre Wunder ist, dass sie in der Welt um uns herum, in der Quantenwelt, gilt und die vielen Diskussionen, die die neue Quantenmechanik ausgel\"{o}st hat, sind ein Hinweis, dass die Frage nach dem {\em Sein\/}, dem \glq {\em Was ist?\/}\grq{} -- die in der Mathematik nie vorkommt(!) -- von fundamentaler Bedeutung f\"{u}r den Physiker und auch f\"{u}r den Ingenieur ist.

Angenommen man erfindet spielerisch einen Satz von Regeln (Axiomen) und man wei{\ss} nicht, ob diese Regeln vollst\"{a}ndig und widerspruchsfrei sind und pl\"{o}tzlich entdeckt man physikalische Objekte, die sich genau an diese Regeln halten. W\"{u}rde das ein Mathematiker als ein Beweis f\"{u}r Konsistenz und Vollst\"{a}ndigkeit gelten lassen?

Die Abtriebskraft an einem Hang ist eine Funktion des {\em sinus\/} des Hangwinkels $\Np$
\begin{align}
F = m \cdot g \cdot \sin \Np
\end{align}
und so kommt es, dass eine mathematische (nichtlineare) Funktion, $\sin \Np$, die Erosion der H\"{a}nge in den Gebirgen bestimmt\footnote{Der Laie nimmt in der Regel an, dass die Erosion linear vom Winkel abh\"{a}ngt}.

Oder man nehme die {\em Lorentz Kontraktion\/} $l \to l'$ eines Objektes, das sich mit der Geschwindigkeit ($v$) bewegt
\begin{align}
l' = l \cdot \sqrt{1 - \frac{v^2}{c^2}} = l \cdot \cos\,\gamma
\end{align}
und die {\em Zeitdehnung\/} $t \to t'$, die dazu geh\"{o}rt
\begin{align}
t' = t \cdot \frac{1}{\cos \gamma}\,.
\end{align}
Beide Ausdr\"{u}cke h\"{a}ngen von dem {\em cosinus\/} des Winkels $\gamma$ ab, s. Abb. \ref{UE359},
\begin{align}
\cos \gamma = 1 - \frac{\gamma^2}{2!} + \frac{\gamma^4}{4!} - \frac{\gamma^6}{6!} + \ldots
\end{align}
Was hat Trigonometrie mit spezieller Relativit\"{a}tstheorie zu tun? Die Physiker erkl\"{a}ren uns, dass die Genauigkeit des GPS-Systems von diesen Korrekturen abh\"{a}ngt.

Es ist dieses enge Wechselspiel zwischen Physik und Mathematik, das es dem Ingenieur schwer macht, beides sauber zu trennen.

%Der Ingenieur verl\"{a}sst sich auf seinen 'Instinkt'. Die Energieprinzipe und das Prinz der virtuellen Verr\"{u}ckungen,  $\delta A_i = \delta A_a$, stehen auf seiner Werteskala ganz oben, denn nirgendwo ist man der Statik so nahe, wie wenn man diese Prinzipe verstanden hat.

Aber gute Mechanik hat gute Mathematik n\"{o}tig. Wie oft sind wir nicht gezwungen anzuerkennen, dass mathematische Fehler entsprechende Fehler in den Berechnungen zur Folge haben. {\em Garbage in, garbage out\/}, wie man im englischen sagt. Anscheinend existiert eine unterirdische Verbindung zwischen den Computern und der Mathematik. Computer honorieren gute Mathematik.

Um diese enge Verbindung zwischen Mathematik und Statik sichtbar zu machen, haben wir die ersten Kapitel dieses Buches mit einem relativ spitzen Bleistift geschrieben. Unsere Absicht war dabei nicht, die Statik zu einem Zweig der Mathematik zu machen, sondern das Ziel war, die Grundlagen der Statik besser zu verstehen.

{\textcolor{chapterTitleBlue}{\subsubsection*{Der Satz von Castigliano}}}
Wie sich Mathematik und Anschauung in die Quere kommen k\"{o}nnen, macht in einer exemplarischen Weise der Satz von Castigliano deutlich. {\em Wikipedia\/} schreibt \"{u}ber den Satz von Castigliano\index{Satz von Castigliano}, \cite{Ca}:

{\em Die partielle Ableitung der in einem linear elastischen K\"{o}rper gespeicherten Form\"{a}nderungsenergie nach der \"{a}u{\ss}eren Kraft ergibt die Verschiebung $v_k$ des Kraftangriffspunktes in Richtung dieser Kraft\/}.

Aber die Form\"{a}nderungsenergie eines elastischen K\"{o}rpers, der eine Einzelkraft tr\"{a}gt, ist unendlich gro{\ss}, und daher kann man keine Ableitung berechnen und auch die Verschiebung $v_k$ des Kraftangriffspunktes ist unendlich gro{\ss}, wie man an {\em Sobolevs Einbettungssatz\/} ablesen kann. Der Satz von Castigliano gilt zwar f\"{u}r Stabtragwerke, aber f\"{u}r 2-D und 3-D Probleme, wozu auch elastische K\"{o}rper geh\"{o}ren, gilt er, mit Ausnahme der Kirchhoff-Platte, nicht. Der Nachsatz {\em\glq wenn die Energie endlich ist\grq{}\/} w\"{u}rde als Korrektur ausreichen. (Sinngem\"{a}{\ss} dasselbe gilt f\"{u}r den {\em Satz von Menabrea\/} und den {\em Satz von Engesser\/}).\index{Satz von Engesser}\index{Satz von Menabrea}

Castigliano hat den nach ihm benannten Satz zun\"{a}chst f\"{u}r Fachwerke aufgestellt und dann auf elastische K\"{o}rper verallgemeinert, weil er sich einen solchen K\"{o}rper als ein Fachwerk mit unendlich vielen St\"{a}ben dachte.

Nat\"{u}rlich klingt der Satz von Castigliano so sch\"{o}n, dass man ihn allein schon deswegen f\"{u}r richtig h\"{a}lt, aber Castigliano's Schluss vom Fachwerk auf elastische K\"{o}rper ist eben voreilig. Man kann nicht mathematische Ergebnisse aus S\"{a}tzen der Mechanik herleiten!

Auch dann nicht, wenn die S\"{a}tze {\em Energieerhaltungssatz\/} oder {\em Prinzip der virtuellen Verr\"{u}ckungen\/} hei{\ss}en. Diese S\"{a}tze haben keine Autorit\"{a}t im Gebiet der Mathematik. Und das {\em Rechnen\/} in der Statik ist doch angewandte Mathematik. Hinter jeder Zahl im Ausdruck oder auf dem Bildschirm steht ein mathematisches Gesetz. Welche F\"{u}lle von statischen Details erschlie{\ss}t die Mathematik nicht in den H\"{a}nden eines geschickten Ingenieurs, man lese nur die B\"{u}cher von {\em Christian Petersen\/} oder {\em Karl Girkmann\/}.

Wir wollen mit einem Zitat von Robert Taylor, dem Co-Autor von O.C. Zienkiewicz schlie{\ss}en, \cite{Taylor}. Robert Taylor kann man sicherlich nicht vorwerfen die Mathematik ihrer selbst wegen zu pflegen, dazu ist er viel zu sehr Ingenieur, aber es war auf einer Tagung in den USA, wo er vor dem Plenum verk\"{u}ndete\\

\hspace*{-12pt}\colorbox{highlightBlue}{\parbox{0.98\textwidth}{The principle of virtual displacements is nothing else than integration by parts.}}\\

Dass Robert Taylor den Mut hatte, das \glq vor versammelter Mannschaft\grq{} zu sagen und dass er meinte, es sagen zu m\"{u}ssen, hat uns wiederum Mut gemacht, dieses Buch zu schreiben.


\textcolor{chapterTitleBlue}{\chapter{Software}}\index{downloads}\index{software}

F\"{u}nf Programme stehen zum {\em download\/} bereit:\\

\begin{itemize}\label{SoftwareDownload}
  \item BE-PLATTE \qquad \,\,\href{http://www.be-statik.de/BE-Platte\_demo.html}{http://www.be-statik.de/BE-Platte\_demo.html} \index{BE-PLATTE}
  \item BE-SCHEIBE \qquad \href{http://www.be-statik.de/BE-Scheibe\_demo.html}{http://www.be-statik.de/BE-Scheibe\_demo.html} \index{BE-SCHEIBE}
  \item BE-FRAMES \qquad \,\href{http://www.be-statik.de}{http://www.be-statik.de}
\end{itemize}
und
\begin{itemize}
  \item WINFEM   \qquad \qquad \!\!\href{http://www.winfem.de/software.htm}{http://www.winfem.de/software.htm}
  \item WINFEM-P
\end{itemize}
Die ersten beiden Programme basieren auf der Methode der Randelemente. Viele der Zeichnungen in diesem Buch wurden mit diesen beiden Programmen erstellt, weil man mit Randelementen glattere Verl\"{a}ufe erh\"{a}lt, was ja gerade bei der grafischen Darstellung erw\"{u}nscht ist.

Auf \href{http://www.winfem.de/software.htm}{http://www.winfem.de/software.htm} liegen englischsprachige Versionen
\begin{itemize}
  \item BE-SLABS
  \item BE-PLATES
\end{itemize}
 der ersten beiden Programme.

Das dritte Programm BE-FRAMES ist ein englischsprachiges Lehr-Pro\-gramm zur Berechnung von ebenen Rahmen mit Focus auf der Darstellung von Einflussfunktionen. Dieses Programm demonstriert ferner die Anwendung der {\em one-click-reanalysis\/}, s. Kapitel 5, bei der man mit einfachen Mausklicks St\"{a}be entfernen kann oder in ihrer Steifigkeit modifizieren kann, ohne dass die Steifigkeitsmatrix neu aufgestellt werden muss.\\

Die WINFEM-Programme sind Programme zur Berechnung von Scheiben und Platten mit Focus auf der Berechnung von Einflussfunktionen, s. z.B. \ref{U38G}, \ref{U75}, \ref{U370}, \ref{U35}, \ref{U84}, \ref{U294}, \ref{U295}, \ref{U237}, \ref{U129}, \ref{U271}, \ref{Tottenham}, \ref{1GreenF156}. Das Scheibenprogramm rechnet auf Wunsch adaptiv, \ref{U417}. Beide Programme k\"{o}nnen den Lastfall $p_h$ darstellen, s. z.B. \ref{U28}, \ref{U168}.

Autor von WINFEM ist Prof. Dr.-Ing. T. Gr\"{a}tsch, \href{mailto:thomas.graetsch@haw-hamburg.de}{thomas.graetsch@haw-hamburg.de}.

Erg\"{a}nzungen: Prof. Dr.-Ing. D. Materna, \href{mailto:daniel.materna@hs-owl.de}{daniel.materna@hs-owl.de} \\

Handb\"{u}cher

\begin{tabular}{l l}
  BE-PLATTE & \hspace*{1.2cm}\href{http://www.be-statik.de/data/pdf/Platte.pdf}{http://www.be-statik.de/data/pdf/Platte.pdf} \\
  BE-SCHEIBE & \hspace*{1.2cm}\href{http://www.be-statik.de/data/pdf/scheibe.pdf}{http://www.be-statik.de/data/pdf/scheibe.pdf}\\
  BE-FRAMES & \hspace*{1.2cm}\href{http://www.be-statik.de/data/pdf/BE-Frames.pdf}{http://www.be-statik.de/data/pdf/BE-Frames.pdf}\\
  WINFEM &
  \hspace*{1.2cm}\href{http://www.be-statik.de/data/pdf/WinFem.pdf}{http://www.be-statik.de/data/pdf/WinFem.pdf}
 \end{tabular}


\begin{thebibliography}{}
\bibitem{Altenbach} Altenbach H, Altenbach J, Kissing W (2004) Mechanics of Composite Structural Elements. Springer Verlag
\bibitem{Altenbach2} Altenbach H, Altenbach J, Naumenko K (2016) Ebene Fl\"{a}chentragwerke. Springer Verlag, 2. Auflage
\bibitem{Alastuey} Alastuey A, Clusel M, Magro M, Pujol P (2016) Physics and Mathematical Tools. World Scientific Publishing
\bibitem{Baar} Baar S (2015) Lohmeyer Baustatik 1, Lohmeyer Baustatik 2. Springer Verlag
\bibitem{Babuska5} Babu\v{s}ka I, Strouboulis T (2001) The Finite Element Method and its Reliability. Oxford University Press
\bibitem{Babuska6} Babu\v{s}ka I (1997) Der Doktorand war Roland Maucher, Ort: Cafeteria der Bauhaus-Universit\"{a}t in Weimar, IKM-Tagung, s. S. \pageref{Eq145}
\bibitem{Bangerth} Bangerth W, Rannacher R. (2003) Adaptive Finite Element Methods for Differential
Equations. Birkh\"{a}user Verlag Basel Boston Berlin
\bibitem{Barth} Barth C, Walter R (2013) Finite Elemente in der Baustatik-Praxis. Beuth
\bibitem{Basar} Ba\c{s}ar Y, Kr\"{a}tzig WB (1985) Mechanik der Fl\"{a}chentragwerke. Vieweg \& Sohn
\bibitem{Bebr} Bebr A (1971) \lqq Einflusslinien torsionssteifer Tr\"{a}gerroste\rqq, Bautechnik 48, Heft 7, 233-237
\bibitem{Beyer} Beyer K (1956) Die Statik im Stahlbetonbau. Reprint Springer Verlag 1987
\bibitem{Bletzinger} Bletzinger K U, Dieringer F (2014) Aufgabensammlung zur Baustatik: \"{U}bungsaufgaben zur Berechnung ebener Stabtragwerke. Hanser Verlag
\bibitem{Blaauwendraad} Blaauwendraad J (2010) Plates and FEM. Springer Verlag
\bibitem{Block} Block P, Gengnagel C (2015) Faustformel Tragwerksentwurf. DVA
\bibitem{Bochmann} Bochmann F, Kirsch W (2011) Statik im Bauwesen. Huss Medien
\bibitem{Borst} de Borst R, Crisfield MA, Remmers J, Verhoosel C (2014) Nichtlineare Finite-Elemente-Analyse von Festk\"{o}rpern und Strukturen. Wiley VCH
\bibitem{Bouma} Bouma A L (1993) Mechanik schlanker Tragwerke. Springer Verlag
\bibitem{Carl3} Carl O, Zhang C (2010) \lqq Static and dynamic analysis of cracked or weakened structures\rqq, Proc. Appl. Math. Mech. 10:145--146
\bibitem{Carl2} Carl O (2011) Statische und dynamische Sensitivit\"{a}tsanalysen von gesch\"{a}digten Tragwerken mit Greenschen Funktionen, Dissertation, Universit\"{a}t Siegen
\bibitem{Carl1} Carl O, Villamil P, Zhang C (2011) \lqq Stress and free vibration analysis of functionally graded beams using static Green's functions\rqq, Proc. Appl. Math. Mech. 11, 199--200
\bibitem{Carl4} Carl O, Hartman F, Zhang C (2017) \lqq Schnelle Berechnung von \"{A}nderungen und Varianten bei komplexen Tragsystemen (3D-Modellen) - Neue Ans\"{a}tze in der Baustatik unter Verwendung von Einflussfunktionen\rqq, Stahlbau (M\"{a}rz 2017) 217-224
\bibitem{Ca} https://de.wikipedia.org/wiki/Satz\_von\_Castigliano (7. 8. 2017)
\bibitem{Cirak2} \c{C}irak F, Ramm E (2000) \lqq A posteriori error estimation and adaptivity for elastoplasticity using the reciprocal theorem\rqq, Int. J. Num. Methods in Eng.  47:379--393
\bibitem{Dallmann} Dallmann R (2015) Baustatik 1, 2, 3. Hanser Verlag
\bibitem{Dhatt} Dhatt G, Touzot G, Lefran\c{c}ois (2012) Finite Element Method. ISTE, Wiley
\bibitem{Dinkler} Dinkler D (2014) Grundlagen der Baustatik. Springer Verlag
\bibitem{Eddy} Eddy W (2013) Baustatik -- einfach und anschaulich. Beuth
\bibitem{Ebel} Ebel H (1979) \lqq Zur Ber\"{u}cksichtigung von Verformungslastf\"{a}llen in den Reziprozit\"{a}tss\"{a}tzen von Betti, Maxwell und Krohn-Land\rqq, Stahlbau 49, Heft 5, 137-140
\bibitem{FK} Franke W, Kunow T (2007) Kleines Einmaleins der Baustatik. Kassel University Press
\bibitem{Gaul} Gaul L, Fiedler C (2013) Methode der Randelemente in Statik und Dynamik. Springer, 2. Auflage
\bibitem{Fuchs} Fuchs, M B (2016) Structures and Their Analysis. Springer Verlag
    \bibitem{Girkmann} Girkmann K (2013) Fl\"{a}chentragwerke. Springer Wien, Nachdruck der 6. Auflage von 1986
\bibitem{Golub} Golub GH, van Load CF (2013) Matrix Computations, 4th ed., The John Hopkins University Press Baltimore
\bibitem{Graf} Graf W, Vassilev T (2006) Einf\"{u}hrung in computerorientierte Methoden der Baustatik. Ernst \& Sohn
\bibitem{G1} Gr\"{a}tsch T, Hartmann F (2000) \lqq Zum Gleichgewicht bei finiten Elementen\rqq, Bautechnik 77, 30-36
\bibitem{G2} Gr\"{a}tsch T, Hartmann F, Katz, C (2004) \lqq Einflussfunktionen und finite Elemente\rqq, Bauingenieur 11, 489-497
\bibitem{G3} Gr\"{a}tsch T, Hartmann F (2006) \lqq Pointwise error estimation and adaptivity for the finite element method using fundamental solutions\rqq, Computational Mechanics, 37, 5, 394-407
\bibitem{G4} Gr\"{a}tsch T, Hartmann F (2001) \lqq \"{U}ber ein Fehlerbild bei der Schnittgr\"{o}{\ss}enermittlung mit finiten Elementen, Teil 1: Scheiben\rqq, Bautechnik 78, 327-332
\bibitem{G5} Gr\"{a}tsch T, Hartmann, F (2003) \lqq \"{U}ber ein Fehlerbild bei der Schnittgr\"{o}{\ss}enermittlung mit finiten Elementen, Teil 2: Platten\rqq, Bautechnik 80, 162-168
\bibitem{G8} Gr\"{a}tsch T (2002) $L_2$-Statik. Dissertation Universit\"{a}t Kassel
\bibitem{G6} Gr\"{a}tsch T, Hartmann F (2003) \lqq Finite element recovery techniques for local quantities of linear problems using fundamental solutions\rqq, Computational Mechanics, 33:15--21
\bibitem{G7} Gr\"{a}tsch T, Hartmann F (2004) \lqq Duality and Finite Elements\rqq, Finite Elements in Analysis and Design, 40, 1005--1020
\bibitem{Gr7} Gr\"{a}tsch T, Bathe KJ (2005) \lqq Influence functions and goal-oriented error estimation for finite element analysis of shell structures\rqq, International Journal for Numerical Methods in Engineering, 63(5), 631--788
\bibitem{Hake} Hake E, Meskouris K (2001) Statik der Fl\"{a}chentragwerke. Springer Verlag
\bibitem{Ha1} Hartmann F (1985) The Mathematical Foundation of Structural Mechanics. Springer Verlag
\bibitem{Ha2} Hartmann F, (1986) Methode der Randelemente. Springer Verlag
\bibitem{Ha3} Hartmann F, (1989) Introduction to Boundary Elements. Springer Verlag
\bibitem{HaM2} Hartmann F, Maucher R (1997) \lqq Zum Momentengleichgewicht bei Tragwerksberechnungen nach Theorie zweiter Ordnung\rqq,  Tagung IKM Weimar
\bibitem{HaJa2} Hartmann F, Jahn, P (1999) \lqq Integral Representations for the Deflection and the Slope of a Plate on an  Elastic Foundation\rqq,  Journal of Elasticity 56, 145-158
\bibitem{HaJa3} Hartmann F, Jahn P (2001) \lqq Boundary Element Analysis of Raft Foundations on Piles\rqq, Meccanica 36, 351-366
\bibitem{Ha4} Hartmann F, Katz C (2002) Statik mit finiten Elementen. Springer Verlag
\bibitem{HaK} Hartmann F, Kunow T (2005) \lqq The shift of Green's functions and the domain of influence\rqq, 2nd MIT Conference on Computational Fluids and Solid Mechanics
\bibitem{Ha5} Hartmann F, Katz C (2010) Structural Analysis with Finite Elements, 2nd ed. Springer Verlag
\bibitem{Ha6} Hartmann F (2013) Green's Functions and Finite Elements. Springer Verlag
\bibitem{HaJa} Hartmann F, Jahn P (2014) \lqq Steifigkeits\"{a}nderungen bei finiten Elementen\rqq, Bau\-ingenieur 89, 209-215
\bibitem{Ha7} Hartman F, Jahn P (2017) Statics and Influence Functions---From a Modern Perspective. Springer Verlag
\bibitem{HaJa2} Hartman F, Jahn P (2018) Statik und Einflussfunktionen---vom modernen Standpunkt, 2. Aufl. http://nbn-resolving.de/urn:nbn:de:hebis:34-2018030554714
\bibitem{Hake} Hake E, Meskouris K (2007) Statik der Fl\"{a}chentragwerke. Springer Verlag
\bibitem{Hartsuijker} Harsuijker C, Welleman J W (1999) Engineering Mechanics. Springer Verlag
\bibitem{Haug} Haug E J, Choi K K, Komkov V (1986) Design Sensitivity Analysis of Structural Systems. Academic Press
\bibitem{Hirschfeld} Hirschfeld K (2006) Baustatik - Theorie und Beispiele. Springer
\bibitem{Hoehland} Hoehland G (1957) St\"{u}tzmomenten-Einflussfelder durchlaufender Platten. Springer Verlag
\bibitem{Holzer} Holzer S (2015) Statische Beurteilung historischer Tragwerke, 1 und 2. Ernst \& Sohn
\bibitem{Hsiao} Hsiao GC, Wendland WL (2008) Boundary Integral Equations. Springer Verlag
\bibitem{Int1} http://phys.org/news/2015-10-multigrid-method-simulation.html
\bibitem{Irslinger} Irslinger J (2013) Mechanische Grundlagen und Numerik dreidimensionaler Schalenelemente. Diss. Uni Stuttgart, http://nbn-resolving.de/urn:nbn:de:bsz:93-opus-89596
\bibitem{Karnovsky} Karnovsky IA, Lebed O (2010) Advanced Methods of Structural Analysis. Springer Verlag
\bibitem{Katz1} Katz C, Stieda J (1993) \lqq Praktische FE-Berechnung mit Plattenbalken\rqq. Bauinformatik 1/92 30-34
\bibitem{Katz2} Katz C, Werner H (1982) \lqq Implementation of nonlinear boundary conditions in
finite element analysis\rqq. Computers \& Structures Vol. 15, No. 3, 299-304
\bibitem{Katz3} Katz C (1995) \lqq Kann die FE-Methode wirklich alles?\rqq
    FEM 95 - Finite Elemente in der Baupraxis, Ed. E. Ramm, E. Stein, W. Wunderlich
    Ernst \& Sohn, Berlin
\bibitem{Katz4} Katz C (1986) \lqq Berechnung von allgemeinen Pfahlwerken\rqq. Bauingenieur 61 Heft 12
\bibitem{Katz5} Katz C (1997) \lqq Flie{\ss}zonentheorie mit Interaktion aller Stabschnittgr\"{o}{\ss}en bei Stahltragwerken\rqq.
 Stahlbau 66, Heft 4, pp. 205-213, (1997)
\bibitem{Katz6} Katz C (1996) \lqq  Vertrauen ist gut, Kontrolle ist besser\rqq, in:
Software f\"{u}r Statik und Konstruktion, Eds. C. Katz, B. Protopsaltis, A.A. Balkema
Rotterdam, (1996)
\bibitem{Kemmler} Kemmler R, Ramm E (2001) \lqq Modellierung mit der Methode der Finiten Elemente\rqq, Beton-Kalender 2001. Ernst \& Sohn Berlin
\bibitem{Kersten} Kersten R (1962) Das Reduktionsverfahren der Baustatik. Springer Verlag, 2. Auflage
\bibitem{Kindmann} Kindmann R, Kraus M (2007) Finite-Elemente-Methoden im Stahlbau. Ernst \& Sohn
\bibitem{Koeppl} K\"{o}ppl C (2009) Einflusslinien und ihre Anwendung. Bachelor Projekt TU Graz
\bibitem{Kolar} Kol\'{a}\v{r} V (1970) \lqq The Influence Functions in the Finite Element Method\rqq, ZAMM 50 T 129--T 131\label{Korrektur13}
\bibitem{Kiener} Kiener G, Wunderlich W (2004) Statik der Stabtragwerke. Teubner Verlag
\bibitem{Kraetzig1} Kr\"{a}tzig W B,  Wittek U (1995) Tragwerke 1. Springer Verlag
\bibitem{Kraetzig2} Kr\"{a}tzig W B, Harte R (2016)  Tragwerke 2. Springer Verlag
\bibitem{Kraetzig3} Kr\"{a}tzig W B, Ba\c{s}ar Y (1997) Tragwerke 3. Springer Verlag
\bibitem{Krings} Krings W (2015) Kleine Baustatik. Springer Verlag
\bibitem{Ku} Kurrer K-E (2016) Geschichte der Baustatik. Ernst \& Sohn, 2. Auflage
\bibitem{Kurrer} Kurrer K-E (2018) The History of the Theory of Structures. Wiley Ernst \& Sohn
\bibitem{Lawo} Lawo M, Klingm\"{u}ller O, Thierauf G (1980) Stabtragwerke, Matrizenmethoden der Statik und Dynamik, 2 B\"{a}nde. Vieweg
\bibitem{Lehmann} Lehmann C, Maurer B (2006) Karl Culmann und die graphische Statik. Ernst \& Sohn
\bibitem{Link} Link M (2014) Finite Elemente in der Statik und Dynamik. Teubner Verlag, 4. Auflage
\bibitem{Lumpe} Lumpe G, Gensichen V (2014) Evaluierung der linearen und nichtlinearen Stabstatik in Theorie und Software. Ernst \& Sohn
\bibitem{Mann} Mann W (1997) Vorlesungen \"{u}ber Statik und Festigkeitslehre. Teubner-Verlag
\bibitem{Marti} Marti P (2014) Baustatik: Grundlagen--Stabtragwerke--Fl\"{a}chentragwerke. Ernst \& Sohn, 2. Auflage
\bibitem{Mehlhorn} Mehlhorn G, (1995) Der Ingenieurbau, 9 Bde, Baustatik/Baudynamik. Ernst \& Sohn
\bibitem{Merkel} Merkel M, \"{O}chsner A (2015) Eindimensionale Finite Elemente: Ein Einstieg in die Methode. Springer Verlag
\bibitem{Meskouris} Meskouris K, Butenweg C, Haker E, Holler S (2005) Baustatik in Beispielen. Springer Verlag
\bibitem{Meskouris2} Meskouris K (2009) Statik der Stabtragwerke: Einf\"{u}hrung in die Tragwerkslehre. Springer Verlag
\bibitem{Moersch} M\"{o}rsch E (1947) Statik der Gew\"{o}lbe und Rahmen, Teil A und Teil B, Wittwer Stuttgart
\bibitem{Nasdala} Nasdala L (2015) FEM-Formelsammlung Statik und Dynamik. Springer Verlag
\bibitem{Palkowski} Palkowski S (1989) Statik der Seilkonstruktionen. Springer Verlag
\bibitem{Peters} Peters K-H (2004) Der Zusammenhang von Mathematik und Physik am Beispiel der Geschichte der Distributionen. Dissertation. Universit\"{a}t Hamburg
\bibitem{Petersen0} Petersen C (1966) Beitrag zur praktischen Berechnung zylindrischer Tonnenschalen mit ver\"{a}nderlichem Kr\"{u}mmungshalbmesser. Dissertation TH M\"{u}nchen
\bibitem{Petersen1} Petersen C (2011) Statik und Stabilit\"{a}t der Baukonstruktionen. Vieweg+Teubner Verlag
\bibitem{Petersen3} Petersen C (1990) Stahlbau. Vieweg
\bibitem{Petersen2} Petersen C, Werkle H (2018) Dynamik der Baukonstruktionen. 2. Aufl. Springer Vieweg
\bibitem{Ramm} Ramm E, Hofmann TJ (1996) Stabtragwerke, in Der Ingenieurbau, Bd. 5, 1-350, Ed. G. Mehlhorn. Ernst \& Sohn
\bibitem{Pfl\"{u}ger} Pfl\"{u}ger A (1978) Statik der Stabtragwerke. Springer Verlag
\bibitem{Pucher} Pucher A (1977) Einflussfelder elastischer Platten. Springer Wien, 7. Auflage
\bibitem{Rene} Ren\'{e} H (2012) Statik im Bauwesen. Beuth
\bibitem{Rothert} Rothert H, Gensichen V (1987) Nichtlineare Stabstatik. Springer Verlag
\bibitem{Rubin} Rubin H (1993) Baustatik ebener Stabwerke, in Stahlbau-Handbuch Teil A. Stahlbau-Verlagsgesellschaft
\bibitem{Rust} Rust W (2011) Nichtlineare Finite-Elemente-Berechnungen. Vieweg 2. Auflage
\bibitem{Schade} Schade D (2003) \lqq Einflusslinien f\"{u}r Ausnutzungsgrade in Stabwerken\rqq. Stahlbau 72, Heft 2, 79-82
\bibitem{Schiefer} Schiefer S, Fuchs M, Brandt B, Maggauer G, Egerer A (2006) \lqq Besonderheiten beim Entwurf semi-integraler Spannbetonbr\"{u}cken\rqq, Beton und Stahlbetonbau, 790-802
\bibitem{Schwartpaul} Schwartpaul K, Zhang C, Carl O (2012) \lqq Sensitivity Analysis of Weakened Beams on Elastic Foundation with Green's Functions\rqq. Proc. Appl. Math. Mech. Vol. 12, 233-234
\bibitem{Silber} Silber G, Steinwender F (2005) Bauteilberechnung und Optimierung mit der FEM. Teubner
\bibitem{Sof}  SOFiSTiK AG, www.sofistik.de
\bibitem{Sopoth} Sopoth M, Sopoth G (2008) Sensitivit\"{a}tsanalyse an einem Br\"{u}ckenbauwerk in semi-integraler Bauweise. Diplomarbeit Universit\"{a}t Kassel
\bibitem{Spitzer} Spitzer P, Horschig R (2012) Statik im Bauwesen, 3 B\"{a}nde + Aufgabensammlung. Beuth
\bibitem{Strang0} Strang G, Fix GJ (2008) An Analysis of the Finite Element Method.
Wellesley-Cambridge Press, 2nd ed.
\bibitem{Strang4} Strang G (2010)  Computational Science and Engineering. Wellesley-Cambridge Press. Dt. Ausgabe: Wissenschaftliches Rechnen. Springer Verlag
\bibitem{Szabo2} Szabo I (1997) Geschichte der mechanischen Prinzipien. Birkh\"{a}user Verlag
\bibitem{Szabo3} Szabo G, Babuska I (1991) Finite Element Analysis. John Wiley \& Sons, Inc.
\bibitem{Taylor} Robert Taylor machte diese Bemerkung in der Vollversammlung des
 Ninth US Congress in Computational Mechanics in San Francisco in 2007
\bibitem{Tottenham} Tottenham H (1970) \lqq Basic Principles\rqq,
in: Finite Element Techniques in Structural Mechanics. (Eds. Tottenham H, Brebbia C), Southampton University Press
\bibitem{Turner} Turner MJ, Clough RW, Martin HC, Topp LJ (1956) \lqq Stiffness and deflection analysis of complex structures\rqq. Journal of the Aeronautical Sciences, Vol. 23, No. 9, 805-823
\bibitem{VA} https://de.wikipedia.org/wiki/Virtuelle\_Arbeit
\bibitem{Wagner} Wagner R (2016) Bauen mit Seilen und Membranen. Beuth
\bibitem{Werner} Werner K (2011) Statik im Bauwesen. Beuth
\bibitem{Werkle1} Werkle H (2000) \lqq Konsistente Modellierung von St\"{u}tzen bei der Finite-Element-Berechnung von Flachdecken\rqq, Bautechnik 77, 416--425
\bibitem{Werkle2} Werkle H (2008) Finite Elemente in der Baustatik. Springer Vieweg. 3. Auflage
\bibitem{Werkle3} Werkle H (2006) Vorlesung Baustatik III, Skriptum

\bibitem{Wigner} Wigner E P (1960) 'The unreasonable effectiveness of mathematics in the natural sciences. Richard Courant lecture in mathematical sciences delivered at New York University, May 11, 1959'. Communications on Pure and Applied Mathematics. 13: 1--14.
\bibitem{Wiki1} https://de.wikipedia.org/wiki/Prinzip\_von\_St.\_Venant
\bibitem{Wriggers} Wriggers P (2001) Nichtlineare Finite-Element-Methoden. Springer
\bibitem{Z1} Zienkiewicz OC, Taylor RL, Zhu JZ (2006) Finite Element Method: Volume 1 -- Its Basis \& Fundamentals. Butterworth Heinemann
\end{thebibliography}

{\em The History of the Theory of Structures\/} von Kurrer \cite{Kurrer} enth\"{a}lt eine ausf\"{u}hrliche Bibliographie zu dem Thema Statik und Einflussfunktionen in seiner geschichtlichen Entwicklung.

Die Diplomarbeiten aus Kassel sind \"{u}ber www.winfem.de/products.htm erreichbar. 

\printindex
%{\textcolor{blau2}{\section*{\"{A}nderungen}}

{\flushleft 05.10.2017} \,$\leftrightarrows$ Abb. \ref{U258}, S. 153\\
06.10.2017 \,+ 'Die Kr\"{a}fte $j^+$', S. \pageref{Korrektur1}\\
06.10.2017 \,+  'Die Vektoren $f^+, u^+ ...$', S. \pageref{Korrektur2} \\
06.10.2017 \, $\leftrightarrows$ Abb. \ref{U369}, S. \pageref{Korrektur4} \\
06.10.2017 \,+  'Nichtlinearer Stab', S. \pageref{Korrektur5} \\
09.10.2017 \,+ Abb. \ref{U77}, S. \pageref{Korrektur6} \\
09.10.2017 \,+ Abb. \ref{U417}, S. \pageref{Korrektur7} \\
13.10.2017 \,$\leftrightarrows$  'Singul\"{a}re Lagerkr\"{a}fte', S. \pageref{Korrektur9} \\
15.10.2017 \,+  'Weich aufgeh\"{a}ngte Tragwerke', S. \pageref{Korrektur10} \\
18.10.2017 \,+ $-(EA(x)\,u')' = p$ und $(EI(x)\,w'')'' = p$, S. \pageref{Korrektur11} \\
22.10.2017 \,+  'Maximale Verform./Momente', S. \pageref{Korrektur12} \\
23.10.2017 \,+ Lit. Kol\'{a}\v{r} V (1970) S. \pageref{Korrektur13} \\
26.10.2017 \,+ {\em Sherman-Morrison-Woodbury Formel\/}, S. \pageref{Korrektur14} \\
29.10.2017 \,+  'Einstein und die Statik', S. \pageref{Korrektur15} \\
29.10.2017 \,+ Tabelle zur endlichen und unendlichen Energie, S. \pageref{Korrektur16} \\
31.10.2017 \,+  'Lagersenkung \'{a} la FEM', S. \pageref{Korrektur17} \\
05.11.2017 \,+  'Lagersenkung', S. \pageref{Korrektur18} \\
05.11.2017 \,+  '\"{A}quivalente Spannungs Transformation', S. \pageref{Korrektur41}\\
05.11.2017 \,$\leftrightarrows$ Abb. \ref{U370}, S. \pageref{Korrektur19}\\
09.11.2017 \,$\leftrightarrows$  'Die Lagerkr\"{a}fte der FE-L\"{o}sung', S. \pageref{Korrektur20}\\
09.11.2017 \,+ Abb. \ref{U438}, S. \pageref{Korrektur21}\\
11.11.2017 \,+ Text erg\"{a}nzt im  'Der amputierte Dipol', S. \pageref{Korrektur22} \\
11.11.2017 \,$\leftrightarrows$ Abb. \ref{U417} ausgetauscht, S. \pageref{Korrektur23}\\
14.11.2017 \,+  'Symmetrie und Antimetrie', S. \pageref{Korrektur26}\\
16.11.2017 \,+ Abb. \ref{U421}, S. \pageref{Korrektur27}\\
16.11.2017 \,$\leftrightarrows$  Abb. \ref{U383}, S. \pageref{Korrektur28}\\
18.11.2017 \,$\leftrightarrows$  'Goal oriented adaptive refinement', S. \pageref{Korrektur25}\\
21.11.2017 \,$\leftrightarrows$ Abb. \ref{U290} korrigiert, S. \pageref{Korrektur29}\\
21.11.2017 \,+ Text im Nachwort erg\"{a}nzt, 'Bei einer Diskussion...', S. \pageref{Korrektur30}\\
27.11.2017 \,+  'Wandknoten', S. \pageref{Korrektur31}\\
02.12.2017 \,+ Abb. \ref{U425}, S. \pageref{Korrektur32}\\
05.12.2017 \,+  'Pseudodrehungen', S. \pageref{Korrektur33}\\
06.12.2017 \,+ Abb. \ref{U428}, S. \pageref{Korrektur34}\\
06.12.2017 \,+ Abb. \ref{U429}, S. \pageref{Korrektur35}\\
09.12.2017 \,+ Abb. \ref{U431}, S. \pageref{Korrektur36}\\
09.12.2017 \,+  'Zusammenfassung', S. \pageref{Korrektur37}\\
10.12.2017 \,+ Abb. \ref{U432}, S. \pageref{Korrektur38}\\
10.12.2017 \,+  'Temperatur\"{a}nderungen', S. \pageref{Korrektur39}\\
10.12.2017 \,+  'Die zugeh\"{o}rigen Identit\"{a}ten', S. \pageref{Korrektur40}\\
14.12.2017 \,+  'Euler Gleichung', S. \pageref{Korrektur41}\\
25.12.2017 \,+  'Vorverformungen', S. \pageref{Korrektur42}
%\vspace{0.3cm}
%\flushleft{$\leftrightarrows$ = ge\"{a}ndert, $+$ = hinzugef\"{u}gt}   
\end{document}



