 Der Mathematiker sagt, wenn eine Gleichung null ist, $g = 0$, dann kann ich sie doch mit einer beliebigen Zahl $\delta u$ multiplizieren, und das Ergebnis \"{a}ndert sich nicht, $\delta u \cdot g = 0$,

 Mit dem Arbeitsbegriff kommt die \"{u}berragende Rolle des Skalarproduktes in die Statik hinein. Wir kennen das Skalarprodukt, als das Skalarprodukt zweier Vektoren

 Wir denken uns also den Stab wie ein Laib Brot in lauter kleine Scheibenelemente der Dicke $ dx $ zerlegt. Vorder- und R\"{u}ckseite eines Scheibeelementes wird von einer Normalkraft $N(x) + dN$ bzw. $ N (x)$ gedr\"{u}ckt oder gezogen (der Einfachheit halber nehmen wir an, dass der Zuwachs $ dN $ Null ist) und sie bewegt sich um die Strecke $d\,\delta\,u(x)$, so dass die Arbeit der beiden Normalkr\"{a}fte gleich
\begin{align}
- N(x)\,\delta u(x) + N(x) \,(\delta u(x) + d\, \delta u(x))  = N(x)\,d\,\delta u(x)
\end{align}

\section{??}
Wir beginnen mit dem Gleichgewicht. Die L\"{a}ngsverschiebung eines beidseitig eingespannten Stabes, der eine konstante Streckenlast $p$ [kN/m] tr\"{a}gt, ist die L\"{o}sung des Randwertproblems
\begin{align}
- EA\,u''(x) = p(x) \qquad u(0) = u(l) = 0\,.
\end{align}
Nun ist es nicht schwer, die L\"{o}sung dieser Differentialgleichung zu finden
\begin{align}
u(x) =
\end{align}
 und damit auch den Verlauf der Normalkraft
 \begin{align}
 N(x) = p\,\frac{l}{2} -
 \end{align}
Die L\"{o}sung wird normalerweise so verifiziert, dass man die Funktion $u$ in die Differentialgleichung einsetzt. Was ist aber mit dem Gleichgewicht? Wer kontrolliert denn eigentlich die Normalkr\"{a}fte an den beiden Stabenden? Bei der L\"{o}sung der Aufgabe hat man doch an keiner Stelle einen Gedanken an die Gleichgewichtsbedingung
\begin{align}
N(l) - N(0) + \int_0^{\,l} p\,dx = 0
\end{align}
verschwendet. Nun, die Gleichgewichtsbedingung ist garantiert, von der ersten Greenschen Identit\"{a}t, denn setzen wir f\"{u}r die Funktion $\hat{u}$ die konstante Funktion $\hat{u} = 1$, dann folgt
\begin{align}
G(u,1) = \int_0^{\,l} p\,dx + N(l) - N(0) = 0\,.
\end{align}
Weil jedes zul\"{a}ssige Paar $\{u, \hat{u}\}$ eine Nullstelle der ersten Greenschen Identit\"{a}t ist, ist es auch das Paar $\{u, 1\}$, d.h. die Gleichgewichtsbedingungen sind {\em automatisch\/} erf\"{u}llt.\\

Wenn man genauer hinschaut, wir setzen einfachheitshalber $EA = 1$, dann dr\"{u}ckt die Gleichgewichtsbedingung nur die Tatsache aus, dass die Normalkraft die Stammfunktion der zweiten Ableitung ist
\begin{align}
\int_0^{\,l} - u''(x)\,dx = - u'(l) + u(0)
\end{align}
Anders gesagt, weil Gau{\ss} die Regeln der partiellen Integration entdeckt hat, ist garantiert, dass die Normalkr\"{a}fte an den Stabenden mit der Streckenlast im Gleichgewicht sind. \\

Das kann man mit jeder Funktion $u$, wie $u(x) = e^x,$ ausprobieren
\begin{align}
\int_0^{\,l} e^x\,dx = e^l - e^0 = e^l  - 1\,.
\end{align}
\subsection{Das Prinzip der virtuellen Verr\"{u}ckungen}
Das {\em Prinzip der virtuellen Verr\"{u}ckungen\/} besagt: Wenn ein Stab im Gleichgewicht ist, dann ist bei jeder virtuellen Verr\"{u}ckung $ \delta u $ die virtuelle \"{a}u{\ss}ere Arbeit $ \delta W_a $ gleicht der virtuellen inneren Energie $\delta W_i$ in dem Stab.
\begin{align}
\delta A_a = \delta A_i
\end{align}
mit
\begin{align}
\delta A_a = \int_0^{\,l} p\,\delta u\,dx = N(l)\, \delta u(l) - N(0)\,\delta u(0)
\end{align}
und
\begin{align}
\delta A_i = \int_0^{\,l} \frac{N\,\delta N}{EA}\,dx
\end{align}
Nun diese Gleichheit der virtuellen inneren und \"{a}u{\ss}eren Arbeiten ist nichts anderes als die Aussage, dass die beiden Funktionen $ u $ und $\hat{u}$ eine Nullstelle der ersten Greenschen Identit\"{a}t sind, denn $G(u,\delta u) = 0$ garantiert eben
\begin{align} \label{Eq1}
G(u,\delta u) = \int_0^{\,l} p\,\delta u\,dx + N(l)\,\delta u(l) - N(0)\,\delta u(0)  - \int_0^{\,l} \frac{N\,\delta N}{EA}= 0\,.
\end{align}
was das Prinzip der virtuellen Verr\"{u}ckung ist.\\

Man beachte, dass wir hier keine Einschr\"{a}nkungen an die virtuelle Verr\"{u}ckung $ \delta u $ machen. In der Literatur wird das Prinzip der virtuellen Verr\"{u}ckung oft auf sogenannte {\em zul\"{a}ssige virtuelle Verr\"{u}ckungen\/} eingeschr\"{a}nkt, das sind Verr\"{u}ckungen, die mit den Lagerbedingungen des Stabes vertr\"{a}glich sind also hier mit den Festhaltungen links und rechts an den Stabenden, was bedeutet, dass die virtuellen Verr\"{u}ckungen an den Stabenden Null sind. \\

F\"{u}r solche virtuellen Verr\"{u}ckungen $ \delta u $ reduziert sich Glg. (\ref{Eq1}) auf
\begin{align}
G(u,\delta u) = \int_0^{\,l} p\,\delta u\,dx  - \int_0^{\,l} \frac{N\,\delta N}{EA}= 0\,.
\end{align}
Das ist aber, wie gesagt, eine Einschr\"{a}nkung, die die Mechanik macht und nicht die Mathematik. Die Mathematik l\"{a}sst auch virtuelle Verr\"{u}ckungen $\delta u \in C^1$ zu, die die Lager verschieben. Auch f\"{u}r diese gilt noch $\delta A_a = \delta A_i$. \\

\subsection{Das Prinzip der virtuellen Kr\"{a}fte}
Dieses Prinzip besagt, dass die \"{a}u{\ss}ere Arbeit $\delta A_a^*$ von virtuellen Kr\"{a}ften auf den Wegen $ u $ des Stabes gleich der virtuellen inneren Energie $\delta A_i^*$ ist.\\

Der Hintergrund dabei ist der folgende: man gibt sich ein System von virtuellen Kr\"{a}ften vor, also eine Streckenlast $ p^* $ und die zugeh\"{o}rigen Normalkr\"{a}fte $ N(l)^*, N(0)^*$ an den Stabenden
und l\"{a}sst nun diese auf den Wegen $ u $ des Stabes \"{a}u{\ss}ere und innere Arbeiten verrichten. Dann ist die Bilanz
\begin{align}
\delta A_a^+ = \delta A_i^*\,.
\end{align}

 Unsere Vorgehensweise ist hier etwas atypisch, oder besser gesagt typisch Universit\"{a}t, weil den Statiker normalerweise nicht die L\"{a}ngsverschiebung interessiert sondern eigentlich nur der Verlauf der Normalkraft
 Den Statiker interessiert normalerweise nicht die L\"{a}ngsverschiebung, sondern eigentlich nur der Verlauf der Normalkraft. Der Statiker betrachtet also gar nicht die Differentialgleichung,

Uns interessiert eigentlich nur der Fall, wenn alle die Koeffizienten $c_i$ und $d_i$ eins sind
\begin{align}
a(u_1 + u_2,\hat{u}_1 + \hat{u}_2) = a(u_1,\hat{u}_1) + a(u_2,\hat{u}_1) + a(u_2,\hat{u}_1) + a(u_2,\hat{u}_2)
\end{align}

. So wollen wir also die erste Greensche Identit\"{a}t lesen
\begin{align}
G(w,\hat{w}) = G(w,\hat{w})_{(0,l/2) }+ G(w,\hat{w})_{(l/2,l)} = 0 + 0 = 0\,.
\end{align}


, wenn es Punkte gibt, in denen die Voraussetzungen verletzt werden, dann nehmen wir das zum Anlass, die erste Greensche Identit\"{a}t abschnittsweise zu formulieren.

Wenn sie null sind, dann sind auch die Momente und Querkr\"{a}fte an den Balkenenden null. Bei einem Kragtr\"{a}ger mit einer Einzelkraft am rechten Ende kann die Biegelinie h\"{o}chstens ein Polynom dritten Grades sein (wegen $EI\,w^{IV} = 0$).

Ist der Kragtr\"{a}ger aber elastisch gebettet, dann muss die Biegelinie mindestens ein Polynom vierten Grades sein, weil sonst wegen $EI\,w^{IV} = 0 $ auch die Querkr\"{a}fte an den Balkenenden null sein m\"{u}ssten, s.o., was mit der Belastung nicht kompatibel ist.

Wenn sie Null sind, dann muss auch das Integral von $EI\,w^{IV} $ null sein und somit folgt
\begin{align}
G(w,1) &= \int_0^{\,l} (EI\,w^{IV} + c\,w)\,dx - \int_0^{\,l} c\,w(x)\,dx =\nn \\
& \int_0^{\,l} c\,w(x)\,dx - \int_0^{\,l} c\,w(x)\,dx = 0\,.
\end{align}
d.h. wir lernen weiter nichts dabei.

Dies ist ein Test und dieses noch ein Test und noch ein TestUnd noch ein Und noch ein Test und noch ein Test noch ein Test
Wenn wir den elastisch gebetteten Balken auf seiner ganzen L\"{a}nge um eine L\"{a}ngeneinheit nach dr\"{u}cken, $w(x) = 1$, durch Aufbringen einer konstanten Streckenlast $p = c\,w = c\,1$
\begin{align}
EI
\end{align}

Wenn die Funktionen $w(x)$ und $\hat{w}(x)$ diese Voraussetzungen nicht erf\"{u}llen, weil zum Beispiel die Querkr\"{a}fte oder die Momente in irgendeinem Punkt springen, dann unterbrechen wir die Integration an diesem Punkt und setzte sie hinter dem Punkt fort, d.h. wir formulieren die erste Greensche Identit\"{a}t dann in zwei Teilen.

Das ist im Grunde \"{a}hnlich 'trivial' wie zum Beispiel die durch Umstellen der Gleichung $\hat{x} \cdot 3 \cdot x = x\cdot 3 \cdot \hat{x}$ erhaltene Identit\"{a}t
\begin{align}
G(x,\hat{x}) = \hat{x} \cdot 3\,x - x\cdot 3 \cdot \hat{x} = 0\,.
\end{align}
Die erste Greensche Identit\"{a}t beruht auf der partiellen Integration der Arbeitsgleichung des Balkens, dem ersten Integral auf der rechten Seite. Damit die partielle Integration angewandt werden darf, m\"{u}ssen die beiden Funktionen $w \in C^4(0,l)$ und $\hat{w} \in C^2(0,l)$ liegen. Wenn dies erf\"{u}llt ist, dann ist das Ergebnis garantiert null.

Um Einzelkr\"{a}fte zu beschreiben, benutzen Mathematiker das Dirac-Delta und so w\"{a}re die Biegelinie des Balkens die L\"{o}sung des Randwertproblems
\begin{align}
EI\,w^{IV}(x) = \delta( 0.5\,l - x)\,P \qquad w(0) = w(l) = 0 \qquad M(0) = M(l) = 0
\end{align}

\begin{align}
G(w,\hat{w}) &= \int_0^{\,l} (EI\,w^{IV}(x) + P\,w'(x))\,\hat{w}(x)\,dx \nn \\
&+ [(- EI\,w'''(x) - P\,w'(x))\,\hat{w} + EI\,w''(x)\,\hat{w}'(x)]_{@0}^{@l}\nn \\
&- \int_0^{\,l} \frac{M\,\hat{M}}{EI} - P\,w'(x)\,\hat{w}(x)\,dx = 0\,.
\end{align}

Vom didaktischen Standpunkt aus ist das voll gelungen. Man w\"{u}sste es nicht besser zu machen, denn ohne diese Prinzipe w\"{u}rde man sich bei der statischen Untersuchung von Stockwerkrahmen mittels Arbeitsprinzipen (Mohr, Betti, etc.) hoffnungslos in der Formulierung der Greenschen Identit\"{a}ten verstricken.

Das Rechnen in der Statik ist zu 100 \% Mathematik, weil Zahlen in die Mathematik geh\"{o}ren. Der Tr\"{a}ger auf dem Papier ist eine mathematische Idealisierung und seine Eigenschaften kann man nur aus mathematischen Gesetzen herleiten nicht aber aus mechanischen Prinzipien.

Kraft und Weg nennt man duale Gr\"{o}{\ss}en, weil eine Kraft auf einem Weg eine Arbeit leistet. Der Arbeitsbegriff hat eine fundamentale Bedeutung f\"{u}r die Statik und Mechanik. Er kommt eigentlich spielerisch in die Statik hinein. Wenn zwei gegengleiche Kr\"{a}fte $\pm P$ an einem Lineals ziehen, sie also im Gleichgewicht sind
\begin{align}
P - P = 0\,,
\end{align}
dann kann man diese Gleichung mit einer beliebigen Zahl $ \delta u $ multiplizieren und sie bleibt richtig
\begin{align}
\delta u \,(P - P) = 0\,.
\end{align}
Anschaulich gesprochen hei{\ss}t dies, dass wenn man das Lineal \"{u}ber den Tisch schiebt, ihm also eine 'virtuelle' Verr\"{u}ckung $ \delta u $ erteilt, dass dann die virtuelle Arbeit der beiden Kr\"{a}fte null ist.

Virtuell soll hei{\ss}en, dass es sich dabei um eine gedachte Verschiebung handelt, um ein, wie Einstein sagen w\"{u}rde, blo{\ss}es Gedankenexperiment. Aber in unserem Beispiel ist diese virtuelle Verr\"{u}ckung nat\"{u}rlich ganz real.  \"{U}ber virtuelle Verr\"{u}ckung und virtuelle Verschiebungen und ihre Bedeutung f\"{u}r die Statik werden wir gleich noch N\"{a}heres zu sagen haben. Vorab sei aber schon einmal gesagt, dass ihnen nichts geheimnisvolles anhaftet, es sind einfach Funktionen, mit denen man eine Gleichung testet.

Wichtig ist an dieser Stelle nur, dass wir sehen, wie der Arbeitsbegriff und damit die Dualit\"{a}t zwischen Kraft und Weg, in die Statik und in die Mechanik hinein kommt---spielerisch.//

Das gilt \"{u}brigens auch f\"{u}r das {\em Prinzip der virtuellen Verr\"{u}ckungen\/}. Die Balkenendverformungen $\delta w$ und $\delta w'$ in $G(w,\delta w)$, s. (\ref{EqPvV}), m\"{u}ssen zu der Funktion $\delta w$ geh\"{o}ren, deren Moment $\delta M = - EI \,\delta w''(x)$ in der inneren Energie steht und sie d\"{u}rfen nicht frei erfunden sein. Normalerweise ist es aber so, dass man von einer bekannten Funktion $\delta w$ ausgeht und alles weitere aus dieser Funktion durch Differentiation berechnet. Dann passt es nat\"{u}rlich automatisch.//

Das gilt \"{u}brigens auch f\"{u}r das Prinzip der virtuellen Verr\"{u}ckungen. Die Balkenendverformungen $\delta w$ und $\delta w'$ in $G(w,\delta w)$, s. (\ref{EqPvV}), m\"{u}ssen zu der Funktion $\delta w$ geh\"{o}ren, deren Moment $\delta M = - EI \,\delta w''(x)$ in der inneren Energie steht und sie d\"{u}rfen nicht frei erfunden sein. Normalerweise ist es aber so, dass man von einer bekannten Funktion $\delta w$ ausgeht und alles weitere aus dieser Funktion durch Differentiation berechnet. Dann passt es nat\"{u}rlich automatisch. //

Und es ist nicht so, dass erst das Prinzip der virtuellen Verr\"{u}ckungen da war und auf der n\"{a}chsten Stufe die Greensche Identit\"{a}t, sondern es ist umgekehrt: Am Anfang war die Differentialgleichung aus der man durch partielle Integration des Arbeitsintegrals die Greensche Identit\"{a}t abgeleitet hat und man hat dann im Nachhinein die Ausdr\"{u}cke in der ersten Greenschen Identit\"{a}t als virtuelle \"{a}u{\ss}ere Arbeit bzw. als virtuelle innere Arbeit interpretiert und so kommt man hat
genau genommen sind es immer mathematische Beziehungen, die den Ausgangspunkt bilden und die dann nach eventuellen identischen Umformung
man immer finden wird, dass die virtuell \"{a}u{\ss}ere Arbeit gleich der virtuell inneren Energie ist//

Man beachte, dass $\delta w = x^2 $ sicherlich keine kleine virtuelle Verr\"{u}ckung ist, noch weniger, dass sie infinitesimal klein ist, denn die Auslenkung am Kragarmende betr\"{a}gt stolze 5 m. Wir k\"{o}nnen daher getrost den Einwand, dass das Prinzip der virtuellen Verr\"{u}ckungen nur f\"{u}r kleine oder infinitesimal kleine Verr\"{u}ckungen gilt, fallen lassen.
Es gibt mathematisch keinen Grund, warum die virtuellen Verr\"{u}ckungen 'klein' sein m\"{u}ssen.
//

\subsubsection*{Prinzip der virtuellen Verr\"{u}ckung}
\vspace{-0.7cm}
\begin{align}
G(w, \delta w) = \delta A_a - \delta A_i = 0
\end{align}
\\

Zu jedem der beiden St\"{a}be 1, 2, aus denen der Rahmen besteht geh\"{o}ren L\"{a}ngsverformungen $u_1, u_2$ und Biegeverformungen $w_1, w_2$ und zu jeder dieser Verformungen geh\"{o}rt eine Greensche Identit\"{a}t. Jede dieser  Identit\"{a}ten ist f\"{u}r sich null und daher kann man sie einfach addieren
\begin{align}
0 + 0 + \ldots + 0 = 0\,.
\end{align}
Ordnet man diese Identit\"{a}ten nach \"{a}u{\ss}erer und innerer Arbeit, dann sind sie, je nach Kontext, von der Gestalt
\begin{align} \label{Eq13}
A_a = A_i \qquad \delta A_a = \delta A_i \qquad \delta A_a^* = \delta A_i^*
\end{align}\\



Diese Identit\"{a}ten sehen zum Teil sehr kompliziert aus und der Leser wird sich jetzt vielleicht fragen muss ich denn all diese Terme mitnehmen, wenn ich einen Rahmen analysiere? Kann ich denn nicht bei meinem gewohnten Zugang bleiben?

Ja, nat\"{u}rlich, denn wenn man alle diese Terme zusammentr\"{a}gt und aufaddiert, dann zeigt sich, dass  viele Terme sich einfach an den Balkenenden gegenseitig aufheben und man am Schluss wieder bei den einfachen Ausdr\"{u}cken ist, die man aus der Statik kennt.

Jetzt kann man sich nat\"{u}rlich fragen, warum dieser ganze Aufwand, warum nicht gleich im alten Gleis bleiben, wenn doch dasselbe herauskommt? Aber nur auf diesem Wege verstehen wir, wo die einzelnen Terme herkommen und wie sich das alles zu einem Ganzen f\"{u}gt.\\


was bedeutet, dass sie in beiden Argumenten linear ist
und dass  die Reihenfolge der Argumente egal ist, $a(w,\hat{w}) = a(\hat{w},w)$\\

Formal ist $a(w,\hat{w})$ eine {\em symmetrische Bilinearform\/}, was bedeutet, dass sie in beiden Argumenten linear ist
und dass  die Reihenfolge der Argumente egal ist, $a(w,\hat{w}) = a(\hat{w},w)$.

Und man hat auch noch die Wahl der Greenschen Identit\"{a}t frei, d.h. man kann auch noch die Differentialgleichung frei w\"{a}hlen: So ist auch die Funktion $\sin\,(a\,x) $, wenn sie als L\"{a}ngsverschiebung eines Stabes genommen wird, im 'horizontalen' Gleichgewicht
\begin{align}
G(\sin\,a\,x,1) = \int_0^{\,l} - EA\,(- a^2\sin\,a\,x)\,dx + [EA\,\cos\,a\,x]_{@0}^{@l} = 0\,.
\end{align}
Das kann man mit allen Differentialgleichungen, also allen Identit\"{a}ten ausprobieren. Man wird immer finden, dass $G(\sin (ax), 1) = 0 $ ist.


Dass glatte alle glatten Funktionen $u$ oder $w$ im Gleichgewicht sind ist mehr oder minder trivial, denn dahinter steckt der Hauptsatz der Differential- und Integralrechnung
\begin{align}
\int_0^{\,l} F'(x)\,dx = F(l) - F(0)\,.
\end{align}
Am einfachsten sieht man das bei Differentialgleichungen zweiter Ordnung, wie beim Stab
\begin{align}
\int_0^{\,l} - EA\,u''(x)\,dx = - EA\,u'(l) + EA\,u'(0)
\end{align}
\\

In der Statik ist A die Belastung und B ist die virtuelle Verr\"{u}ckung. Wir lernen etwas \"{u}ber die Gr\"{o}{\ss}e einer Belastung, und wie sie verteilt ist, indem wir mit verschiedenen virtuellen Verr\"{u}ckungen an ihr wackeln. Darin liegt die gro{\ss}e Bedeutung der virtuellen Verr\"{u}ckungen. Um etwas \"{u}ber eine Kraft $\vek F$ zu klassifizieren, berechnen wir die Arbeiten, die die Kraft $\vek F $ auf den Wegen $\vek e_i $ leistet, wenn man sie also nacheinander in Richtung der Einheitsvektoren verschiebt. Die Koordinaten $ $ einer Kraft sind also eigentlich Arbeiten
\begin{align}
F_i = \vek e_i \dotprod \vek F
\end{align}
. Wir haben ein $A$ und ein $B$, also zwei Terme, zwei Gr\"{o}{\ss}en. Und das Skalarprodukt zwischen diesen Termen (Arbeit!) benutzen wir, um etwas \"{u}ber A zu erfahren.
, die Breite eines Schranks, indem wir einen Zollstock dagegen halten, das Gewicht eines Koffers, indem wir ihn anheben, die Gestalt eines Hauses, in dem wir es auf m\"{o}glichst viele Ebenen projizieren (Vorderansicht, Seitenansichten, R\"{u}ckansicht, etc.)
\\

Vereinfachen wir die erste Greensche Identit\"{a}t einmal auf den Ausdruck
\begin{align}
G(w, \delta w) = \int_0^{\,l} EI w^{IV}\,\delta w\,dx + [V\,\delta w]_{@0}^{@l} - a(w, \delta w) = 0
\end{align}
dann sieht man, dass man durch  geeignete Wahl der virtuellen Verr\"{u}ckung $\delta w$ eine der beiden Querkr\"{a}fte an den Balkenenden berechnen kann.

Vertauschen wir die Pl\"{a}tze
\begin{align}
G(w, \delta w) = \int_0^{\,l} EI \delta w^{IV}\, w\,dx + [\delta V\, w]_{@0}^{@l} - a(w, \delta w) = 0
\end{align}
dann sieht man, dass man durch eine geeignete Einzelkraft $P = 1 $ etwa in der Balkenmitte die Durchbiegung $w(0.5\,l)$ in der Balkenmitte berechnen kann.

\begin{align} \label{Eq13}
A_a = A_i \qquad \delta A_a = \delta A_i \qquad \delta A_a^* = \delta A_i^*
\end{align}
W\"{a}hlt man f\"{u}r $\hat{w} $ eine virtuelle Verr\"{u}ckung $\delta w $, dann entsteht das Prinzip der virtuellen Verr\"{u}ckungen. Bleibt man auf der Diagonalen, $\hat{w} = w $, dann formuliert man den Energieerhaltungssatz (bis auf den Faktor $1/2$), und setzt man $\delta w $ an die erste Stelle und $w $ an die zweite Stelle, dann formuliert man das {\em Prinzip der virtuellen Kr\"{a}fte\/}. Durch das Vertauschen sind es jetzt die Kr\"{a}fte, die zu $\delta w $ geh\"{o}ren, (also $EI\,\delta w^{IV}, \delta V, \delta M$), die virtuelle Arbeit auf den Wegen der Originalfunktion $w $ leisten.

In der Literatur wird bei der Formulierung von $G(\delta w^*,w) = 0$ die Funktion $\delta w $ meist mit einem Stern geschrieben.
Die gro{\ss}e Bedeutung der Identit\"{a}ten f\"{u}r die Statik beruht darauf, dass sie das Fundament der Arbeits- und Energieprinzipe der Statik bilden. \\



Das {\em Prinzip der virtuellen Kr\"{a}fte\/} ist im Grunde identisch mit dem Prinzip der virtuellen Verr\"{u}ckungen. Nur das wir uns diesem jetzt aus einer anderen Richtung n\"{a}hern. Jetzt spielt die Biegelinie $w$ des Tr\"{a}gers die virtuelle Verr\"{u}ckung an einem System von Kr\"{a}ften $\delta K^*$.

An virtuellen Kr\"{a}ften ist nichts geheimnisvolles oder ungew\"{o}hnliches. Es sind normale Lasten + zugeh\"{o}rigen Lagerkr\"{a}ften, an denen mit $w$ gewackelt wird und so wie in jedem Lastfall das Prinzip der virtuellen Verr\"{u}ckungen gilt, so auch hier.

Wertvoll ist das Prinzip deswegen, weil man durch geschickte Wahl des Systems $\delta K^*$ aus der Identit\"{a}t $G(\delta w^*, w) = 0$ Informationen \"{u}ber Einzelverformungen ziehen kann. Wir werden das an einem Beispiel erl\"{a}utern.


\subsubsection{Addition der inneren Energien}
Wie sieht es mit der inneren Energie aus, wenn federnde Lager vorhanden sind? Eine Schraubenfeder, die einen Durchlauftr\"{a}ger st\"{u}tzt kann man in Gedanken einem Pendelstab gleichsetzen, also wie ein vollwertiges weiteres Bauteil ansehen. Nur dass die L\"{a}ngssteifigkeit der Feder nicht $EA $ ist, sondern gleich der Federsteifigkeit $k$. Bei einem Stab wird integriert
\begin{align}
\delta A_i = \int_0^{\,l} \frac{N\,\delta N}{EA}\,dx\,,
\end{align}
bei einer Feder, die ja nur eine Kopf- und Fu{\ss}verschiebung hat, $u_1$ und $u_2$, wird dagegen multipliziert, das Skalarprodukt
\begin{align}
\delta A_i = \vek \delta \,\vek u^T \,\vek K\,\vek u =   \left [ \delta u_1 \,\, \delta u_2 \right ]   \left[ \barr {r @{\hspace{4mm}}r @{\hspace{4mm}}r
@{\hspace{4mm}}r @{\hspace{4mm}}r}
      k & -k  \\
      -k & k \\
     \earr \right]\left [\barr{c}  u_1 \\  u_2\earr \right ]
  \, ,
\end{align}
ausgewertet, wobei $\vek K $ die Steifigkeitsmatrix der Feder ist.

Nun sind wir neugierig geworden, wie lautet denn die erste Greensche Identit\"{a}t einer Feder? Die gibt es nicht, wenn wir alles durch die Bewegungen $u_1, u_2 $ an den Federenden beschreiben. In
matrizieller Schreibweise lautet das Federgesetz
\begin{align}
\vek K\,\vek u = \vek f
\end{align}
und zu dieser symmetrischen Matrix geh\"{o}rt die Identit\"{a}t
\begin{align}
G(\vek u, \vek \delta\,\vek u) = \vek \delta\,\vek u^T\,\vek K\,\vek u -  \vek u^T\,\vek K\,\vek \delta\,\vek u = 0\,,
\end{align}
die alles beinhaltet, was man f\"{u}r die Energie- und Arbeitsprinzipe der Feder ben\"{o}tigt.

Wir werden sp\"{a}ter, im Zusammenhang mit dem Thema lineare Algebra und Statik noch einmal darauf zur\"{u}ckkommen.

Bei einer Feder, die sich auf dem Boden abst\"{u}tzt, ist ein Freiheitsgrad null, etwa $u_2 = 0$, und so reduziert sich die virtuelle innere Energie eines Balkens mit einem federnden Endlager auf den Ausdruck
\begin{align}
\delta A_i = \int_0^{\,l} \frac{M\,\delta M}{EI}\,dx + \delta u_1\,k\,u_1
\end{align}
und die innere Energie kann man daraus sofort ablesen, wenn man auf die Diagonale geht, $\delta u = u$ und $\delta u_1 = u_1$, und pflichtgem\"{a}{\ss} alles mit dem Faktor $1/2$ multipliziert
\begin{align}
A_i = \frac{1}{2}\, \int_0^{\,l} \frac{M\, M}{EI}\,dx + \frac{1}{2}\, u_1\,k\,u_1
\end{align}
%%%%%%%%%%%%%%%%%%%%%%%%%%%%%%%%%%%%%%%%%%%%%%%%%%%%%%%%%%%%%%%%%%%%%%%%%%%%%%%%%%%%%%%%%%%%%%%%%%%
\section{Details}

Im folgenden wollen wir nun auch im Detail zeigen, wie die Arbeits- und Energieprinzipe der Statik auf der ersten Greenschen Identit\"{a}t beruhen.\\

Ein noch einfacheres Beispiel f\"{u}r diesen \"{U}bergang von leeren Formalismen zur Statik ist das folgende: Eine Feder habe die Federsteifigkeit $k$. Bei der Multiplikation von $k$ mit zwei Zahlen $u$ und $\hat{u}$ spielt die Reihenfolge keine Rolle und deswegen ist
\begin{align} \label{Eq23}
G(u,\hat{u}) = u\,k\,\hat{u} - \hat{u}\,k\,u = 0
\end{align}
eine Identit\"{a}t. Nun sei $u$ die Auslenkung der Feder bei einer Belastung mit einer Kraft $f$, gen\"{u}ge also der Gleichung $k\,u = f$, dann folgt aus (\ref{Eq23}) die Gleichung
\begin{align}
\delta A_i = u\,k\,\hat{u} = \hat{u}\,f = \delta A_a \,,
\end{align}
also die G\"{u}ltigkeit des Prinzips der virtuellen Verr\"{u}ckungen f\"{u}r die Feder. Die Ausgangs\-gleich\-ung (\ref{Eq23}) ist trivial, mit der Ersetzung $k\,u = f$ wird daraus Statik.\\

%----------------------------------------------------------------------------------------------------------
\subsubsection{Die Arbeitsgleichung}
Die Mohrsche Arbeitsgleichung ist das Universalwerkzeug, um Verformungen an rahmenartigen Tragwerken zu berechnen
\begin{align}
\bar{1}\cdot\delta = \int_0^{\,l} \frac{M\,\bar{M}}{EI}\,dx\,.
\end{align}
Wir wollen an dieser Stelle, wenigstens einmal, einen Beweis f\"{u}r diese Formel geben, also zeigen, wie genau dieses Ergebnis zustande kommt.

Zun\"{a}chst erstellt man eine 1:1 Kopie des Tr\"{a}gers und belastet diese Kopie in Richtung der gesuchten Verformung mit einer Kraft $\bar{P} = \bar{1} $. Die Biegelinie $ $ unter der Einzelkraft kann man nicht in geschlossener Form angeben, sie liegt nicht in $C^4$, weil die Querkraft in der Balkenmitte springt
\begin{align}
\bar{V}_1 - \bar{V}_2 = \bar{P}\,.
\end{align}
Die beiden Teile $ $ und $ $ der Biegelinie sind homogene L\"{o}sungen der Balkengleichung (keine Belastung auf der freien Strecke) und in der Mitte des Balkens gehen sie, bis auf den Sprung in der Querkraft stetig ineinander \"{u}ber, $\bar{w}_1 = \bar{w}_2$, $\bar{w}_1' = \bar{w}_2'$ und $\bar{M_1} = \bar{M_2}$, so dass sich bei der abschnittsweisen Formulierung der ersten Greenschen Identit\"{a}t in der Reihenfolge $\bar{w},w $. F\"{u}r den ersten Abschnitt erhalten wir, $\dotprod = l/2$
\begin{align}
G(\bar{w}_1,w)_{(0,\dotprod)} &= \int_0^{\dotprod} EI\,\bar{w}_1^{IV}\,w\,dx + \bar{V}_1(\dotprod)\,w(\dotprod) - M_1(\dotprod)\,w'(\dotprod)\nn \\
 &- \bar{V}_1(0)\,w(0) + \bar{M}_1(0)\,w'(0) - \int_0^{\,\dotprod} \frac{M\,\bar{M}_1}{EI}\,dx = 0
\end{align}
Wegen $EI\,\bar{w}_1^{IV} = 0$ und $w(0) = 0$ und $\bar{M}_1(0) = 0$ reduziert sich das auf
\begin{align}
G(\bar{w}_1,w)_{(0,\dotprod)} &= \bar{V}_1(\dotprod)\,w(\dotprod) - M_1(\dotprod)\,w'(\dotprod)  - \int_0^{\dotprod} \frac{M\,\bar{M}_1}{EI}\,dx = 0
\end{align}
und analog f\"{u}r das zweite Intervall
\begin{align}
G(\bar{w}_2,w)_{(\dotprod,l)} &= -\bar{V}_2(\dotprod)\,w(\dotprod) + M_2(\dotprod)\,w'(\dotprod)  - \int_{\dotprod}^{\,l} \frac{M\,\bar{M}_1}{EI}\,dx = 0
\end{align}
In der Mitte ist $\bar{M}_1 = \bar{M}_2$ und $w'$ ist stetig, so dass
\begin{align}
G(\bar{w}_1,w)_{(0,\dotprod)} + G(\bar{w}_2),w)_{(\dotprod,l)} = \bar{P}\,w(\dotprod) - \int_0^{\,l} \frac{\bar{M}\,M}{EI}\,dx = 0
\end{align}
oder
\begin{align}
w(\dotprod) =  \int_0^{\,l} \frac{\bar{M}\,M}{EI}\,dx\,.
\end{align}
Das ist die Mohrsche Arbeitsgleichung. In der Literatur wird zum Beweis dieser Formel an das {\em Prinzip der virtuellen Kr\"{a}fte\/} appelliert, das Ergebnis durch Anwendung eines mechanischen Prinzips hergeleitet.

Das {\em Prinzip der virtuellen Kr\"{a}fte\/} besagt das folgende: wenn $w$ die Biegelinie eines Balkens ist, (zu vorgegebener \"{a}u{\ss}erer Belastung), das unter der Einwirkung von \"{a}u{\ss}eren Kr\"{a}ften im Gleichgewicht ist


 kann man Durchbiegungen berechnen. Auch diese Gleichung basiert auf der ersten Greenschen Identit\"{a}t und zwar auf dem {\em Prinzip der virtuellen Kr\"{a}fte\/}
\begin{align}
G(\delta w^*,w) = \int_0^{\,l} EI\,\delta w^{*IV}\,w\,dx + [\delta V^*\,w - M*\,\delta w'] - \int_0^{\,l} \frac{M\,M*}{EI} dx = 0
\end{align}
Das erste Argument ist also $\delta u*$ und erst das zweite Argument ist die Biegelinie des Tr\"{a}gers.

Anschaulich geschieht das Folgende: Man belastet eine Kopie des Tr\"{a}gers mit einer Einzelkraft $\bar{P} = \bar{1}$ (in Richtung der gesuchten Verschiebung). Die Arbeit, die diese 'virtuelle' Kraft auf dem Weg $w(x)$ am Ort von $\bar{P}$ leistet, ist gleich der virtuellen inneren Energie zwischen $w*$, der Biegelinie zu $\bar{P} = \bar{1}$, und der Biegelinie $w$ des Tr\"{a}gers.\\

So weit die verbale Beschreibung, aber mit Begriffen l\"{a}sst sich nichts beweisen, dazu muss man Mathematik machen.
\\

\subsubsection{Prinzip der virtuellen Verr\"{u}ckungen}
\begin{align}
G(w,\delta w) = \int_0^{\,l} EI\,w^{IV}(x)\,\delta w(x)\,dx + [V\,\delta w - M\,\delta w']_{@0}^{@l} - \int_0^{\,l} \frac{M\, \delta M}{EI}\,dx = 0
\end{align}
\subsubsection{Prinzip der virtuellen Kr\"{a}fte}
\begin{align}
G(\delta w, w) = \int_0^{\,l} EI\,\delta w^{IV}(x)\, w(x)\,dx + [\delta V\, w - \delta M\, w']_{@0}^{@l} - \int_0^{\,l} \frac{\delta M\,  M}{EI}\,dx = 0
\end{align}
\subsubsection{Energieerhaltung}
\begin{align}
\frac{1}{2}\,G(w, w) &= \frac{1}{2}\, \int_0^{\,l} EI\,w^{IV}(x)\, w(x)\,dx + \frac{1}{2}\, [V\, w - M\, w']_{@0}^{@l} \nn\\
&- \frac{1}{2}\, \int_0^{\,l} \frac{M\,  M}{EI}\,dx = 0
\end{align}
\subsubsection{{\em Satz von Betti\/}}
\begin{align}
\text{\normalfont\calligra B\,\,}(w_1, w_2) &= \text{\normalfont\calligra G,\,}(w_1,w_2) - \text{\normalfont\calligra G\,\,}(w_2,w_1) = \int_0^{\,l} EI\,w_1^{IV}\,w_2\,dx\nn \\
& + [V_1\,w_2 - M_1\,w_2']_{@0}^{@l} - [V_2\,w_1 - M_2\,w_1']_{@0}^{@l} - \int_0^{\,l} w_1\,EI\,w_2^{IV}dx = 0
\end{align}

\subsection{Die Balkengleichung}
Wenn die Biegesteifigkeit $ EI $ l\"{a}ngs des Balkens konstant ist, dann wird der Zusammenhang zwischen der Durchbiegung $ w(x) $ und der Belastung $ p(x) $ beschrieben durch die Differentialgleichung vierter Ordnung
\begin{align}
EI\,w^{IV}(x) = p(x)\,.
\end{align}
Die erste Greensche Identit\"{a}t dieser Differentialgleichung lautet
\begin{align}
G(w,\hat{w}) = \int_0^{\,l} EI\,w^{IV}(x)\,\hat{w}(x)\,dx + [V\,\hat{w} - M\,\hat{w}']_{@0}^{@l} - \int_0^{\,l} \frac{M\,\hat{M}}{EI}\,dx = 0
\end{align}
mit
\begin{align}
M(x) = - EI\,w''(x) \qquad V(x) = - EI\,w'''(x)
\end{align}
als dem Biegemoment bzw. der Querkraft der Biegelinie.

Diese erste Greensche Identit\"{a}t erh\"{a}lt man, wenn man das Integral
\begin{align}
\int_0^{\,l} EI\,w^{IV}(x)\,\hat{w}(x)
\end{align}
mittels partieller Integration umformt. Damit man partielle Integration anwenden kann, setzen wir voraus, dass die beiden Funktionen $w$ und $\hat{w}$ hin\-reich\-end glatt sind, $w(x) \in C^4(0,l)$ und $\hat{w}(x) \in C^2(0,l)$. Die Forderung $w \in C^4 $ ist ziemlich restriktiv. Schon wenn die Belastung $p$ in der Mitte des Balkens springt, also dort eine Unstetigkeitsstelle hat, liegt $ w $ nicht mehr in $C^4$, weil die vierte Ableitung von $ w $ diesen Sprung ja enthalten muss.

Aber das kann man leicht umgehen, indem man die erste Greensche Identit\"{a}t abschnittsweise formuliert, also f\"{u}r das Intervall $(0, l/2) $, vom linken Lager bis zur Balkenmitte, und dann noch einmal von der Balkenmitte bis zum rechten Lager, also das Intervall $(l/2,l)$
\begin{align}
G(w,\hat{w}) = G(w,\hat{w})_{(0,l/2) }+ G(w,\hat{w})_{(l/2,l)} = 0 + 0 = 0\,.
\end{align}
Insbesondere Einzel\-kr\"{a}fte $P$ oder Einzelmomente $M$ machen solche Zwangs\-pausen n\"{o}tig. Die Bilanz $G(w,\hat{w}) = 0$ gilt wirklich nur f\"{u}r Intervalle, in denen $w$ in $C^4$ liegt und $\hat{w}$ in $C^2$.

Ein Beispiel soll f\"{u}r viele gelten. Der Balken in Bild X wird in seiner Mitte durch eine Einzelkraft $ P = 10 $ belastet. Um dieses Problem zu l\"{o}sen, muss man den Balken in zwei Teile unterteilen, $w_L$ und $w_R$, vom linken Lager bis zur Mitte und von der Mitte bis zum rechten Lager. An der \"{U}bergangsstelle $\dotprod = l/2$ zwischen diesen beiden Intervallen gilt
\begin{align}
w_L(\dotprod) = w_R(\dotprod) \qquad w_L'(\dotprod) = w_R'(\dotprod) \qquad M_L(\dotprod) = M_R(\dotprod)
\end{align}
aber die Querkr\"{a}fte springen um den Betrag der Einzelkraft $P$
\begin{align}
V_R(\dotprod ) - V_L(\dotprod ) = P
\end{align}
Den Energieerhaltungssatz formuliert man jetzt f\"{u}r das linke Intervall und f\"{u}r das rechte Intervall separat und addiert die beiden Gleichungen und erh\"{a}lt so
\begin{align}
G(w,w)_{(0,l/2)} + G(w_R,w_R)_{(l/2, 0) } = P \,w(\dotprod) - \int_0^{\,l} \frac{M^2}{EI}\,dx = 0
\end{align}
Mit dem weiteren k\"{o}nnen wir uns kurz fassen, weil es nur eine Wiederholung des oben schon Gesagten ist.


%----------------------------------------------------------------------------------------------------------
\begin{figure}[tbp]
\centering
\if \bild 2 \sidecaption \fi
\includegraphics[width=0.9\textwidth]{\Fpath/S2}
\caption{Einfeldtr\"{a}ger} \label{S2}
%
\end{figure}%

%\end{document}

%%%%%%%%%%%%%%%%%%%%%%%%%%%%%%%%%%%%%%%%%%%%%%%%%%%%%%%%%%%%%%%%%%%%%%%%%%%%%%%%%%%%%%%%%%%%%%%%%%%
\subsection{Das Prinzip vom Minimum der potentiellen Energie}

Um dieses Prinzip herzuleiten, beginnen wir am besten mit dem einfachsten m\"{o}glichen statischen Element, einer Feder, s. Bild \ref{FederEnergie}.
Das Federgesetz
\begin{align}
k\,u = f
\end{align}
besagt, dass die Auslenkung $ u $ der Feder proportional zur aufgebrachten Kraft $ f $ ist. Der Faktor $ k $ hat die Dimension [F/L] und wird die Steifigkeit der Feder genannt.

Die Kraft $ f $, die die Feder nach unten zieht, leistet eine Arbeit und weil es Eigenarbeit ist tr\"{a}gt sie den Faktor $1/2$
\begin{align}
A_a = \frac{1}{2}\,f\,u\,.
\end{align}
Diese \"{a}u{\ss}ere Arbeit wird als innere Energie in der Feder gespeichert
\begin{align}\label{Eq5}
A_i = \frac{1}{2}\, k\,u^2\,.
\end{align}
Und wir erwarten nat\"{u}rlich, dass in der Gleichgewichtslage die \"{a}u{\ss}ere Arbeit und die innere Energie \"{u}bereinstimmen, $A_a = A_i$
\begin{align}
A_a = \frac{1}{2}\, f\,u = \frac{1}{2}\, k\,u\,u = \frac{1}{2}\, k\,u^2 = A_i\,.
\end{align}
Um diesen \"{U}bergang von $A_a$ zu $A_i$ m\"{o}glich zu machen, muss die innere Energie genau die Form (\ref{Eq5}) haben.

Tr\"{a}gt man den Verlauf der Funktion $1/2\,f\,u $ und der Funktion $1/2\,k\,u^2 $ in einem Koordinatenkreuz auf,
dann ist die Auslenkung $ u$ der Feder  unter der Wirkung der Kraft $ f $ genau der Punkt $u$, in dem sich die beiden Kurven schneiden, siehe Bild \ref{FederEnergie}.

Nun gibt es noch eine weitere Kurve in Bild \ref{FederEnergie} und das ist die potentielle Energie $\Pi$ der Feder
\begin{align}
\Pi(u) = \frac{1}{2}\, k\,u^2 - f\,u\,.
\end{align}
Der Faktor $1/2$ macht, dass sich bei der Bildung der Ableitung die 2 wegk\"{u}rzt
\begin{align}
\Pi'(u) = k\,u - f
\end{align}
und so, weil die Auslenkung $ u $ der Feder dem Federgesetz $k\,u = f $ gen\"{u}gt, die potentielle Energie im Gleichgewichtspunkt $u$ eine horizontale Tangente, $\Pi'(u) = 0$, hat.

Die interessante Beobachtung ist nun, siehe Bild \ref{FederEnergie}, dass der Punkt, in dem sich die \"{a}u{\ss}ere und innere Arbeit schneiden,  auch gleichzeitig der Punkt ist, in dem die potentielle Energie ihr Minimum hat (Es ist wirklich ein Minimum denn $\Pi'' = k > 0$).

Wie man im Bild \ref{FederEnergie} sieht, steigt am Anfang die \"{a}u{\ss}ere Arbeit schneller als die innere Energie, aber dann passieren die beiden Kurven einen Punkt, von dem ab die innere Energie schneller w\"{a}chst als die \"{a}u{\ss}ere Arbeit. Dieser Schnittpunkt ist genau der Gleichgewichtspunkt. Nur in diesem Punkt gilt $A_a = A_i$.

W\"{u}rde von Anfang an die innere Energie schneller steigen, als die \"{a}u{\ss}ere Arbeit, dann w\"{u}rde sich die Feder \"{u}berhaupt nicht bewegen, dann w\"{a}re schon im Nullpunkt der Wettlauf zu Ende.

Setzen wir alles auf eins, also $ k = 1$ und $ f = 1 $, dann liegt der Gleich\-gewichtspunkt genau bei $ u = 1$. Woraus folgt, dass die Mechanik im Grunde auf der Tatsache beruht, dass im Intervall $(0,1)$
die Ungleichung $u > u^2$ gilt und danach $u^2 > u$ ist. Die Zahl $u = 0.5$ ist gr\"{o}{\ss}er als ihr Quadrat $u^2 = 0.25$, aber $u = 1.5$ ist kleiner als sein Quadrat $u^2 = 2.25$. Einzig im Punkt $u = 1$ ist $u = u^2$.

Das Prinzip vom Minimum der potentiellen Energie fasst nun diese Beobachtungen wie folgt zusammen: Die Auslenkung $ u $ der Feder unter der Wirkung der Kraft $ f $ macht die potentielle Energie der Feder zum Minimum. Wenn man also nur lange genug Zufallszahlen $ u $ in die Funktion $\Pi(u)$ einsetzen w\"{u}rde, sich eine Liste der Wert $\Pi(u)$ machen w\"{u}rde, dann w\"{u}rde man automatisch zu der gesuchten Gleichgewichtslage $u$ der Feder gef\"{u}hrt.

\subsubsection{Minimum oder Maximum?}
Nun kommt eine wichtige Beobachtung. In der tiefsten Lage ist die potentielle Energie negativ, wie man durch Einsetzen ($ k\,u = f$) verifiziert
\begin{align}
\Pi(u) = \frac{1}{2}\,k\,u^2 - f\,u = \frac{1}{2}\, f\,u - f\,u = - \frac{1}{2}\, f\,u\,.
\end{align}
Nun ist aber die Auslenkung $ u $ der Sieger in einem Wettbewerb, es gibt keine andere Zahl, die die potentielle Energie kleiner macht. Und das hei{\ss}t doch anschaulich, dass $ u $ den Abstand $|\Pi(u)|$ vom Nullpunkt m\"{o}glichst gro{\ss} macht. Also ist doch das Prinzip vom Minimum der potentiellen Energie eigentlich ein Maximumsprinzip: M\"{o}glichst weit weg von null mit $|\Pi(u)|$, das ist das Bestreben von $u$. Nur weil die potentielle Energie in der Gleichgewichtslage negativ ist, ist das dasselbe, wie das Minimum der potentiellen Energie. Aber viele Ingenieure interpretieren das Prinzip eben so, wie es die Wortwahl (anscheinend)  suggeriert, mit m\"{o}glichst wenig Anstrengung zum Ziel kommen, die potentielle Energie m\"{o}glichst klein machen, m\"{o}glichst nahe an Null r\"{u}cken, w\"{a}hrend die wahre Bedeutung genau das Gegenteil ist. Die Kraft $ f $ strebt danach m\"{o}glichst viel Energie aus dem Federsystem herauszuziehen, $|\Pi(u)|$ m\"{o}glichst gro{\ss} zu machen, und sie entscheidet sich f\"{u}r die Auslenkung, die Zahl $u$, die diesem Zweck am besten dient.\\

Diese Umkehr dessen, was das Minimum der potentiellen Energie eigentlich bedeutet, zieht sich durch die ganze Statik und Mechanik. Wir werden sp\"{a}ter sehen, dass man die statischen Probleme in zwei Klassen einteilen kann: Ent\-weder werden Kr\"{a}fte aufgebracht oder Verformungen. Wenn Kr\"{a}fte aufgebracht werden, dann ist die potentielle Energie in der Gleichgewichtslage negativ und das Ziel der Belastung ist es, $|\Pi(u)|$ m\"{o}glichst gro{\ss} zu machen, m\"{o}glichst weit von null zu kommen. Wenn Verformungen eingepr\"{a}gt werden, dann ist die potentielle Energie positiv, liegt also rechts vom Nullpunkt und wenn man jetzt die potentielle Energie minimiert, dann sucht man den Verformungszustand $ u $, der die potentielle Energie m\"{o}glichst nahe an Null r\"{u}ckt. Dann hat das Prinzip vom Minimum der potentiellen Energie genau die Bedeutung, die der Ingenieur ihm normalerweise unterlegt. Das Tragwerk versucht mit m\"{o}glichst wenig Aufwand an innerer Energie die Verformungen zu erdulden, die ihm aufgezwungen werden.


%%%%%%%%%%%%%%%%%%%%%%%%%%%%%%%%%%%%%%%%%%%%%%%%%%%%%%%%%%%%%%%%%%%%%%%%%%%%%%%%%%%%%%%%%%%%%%%%%%%
\subsection{Der Stab}
Wir wollen nun diese Ergebnisse auf einen Stab \"{u}bertragen und zwar den Stab in Bild eins. Der Einfachheit halber, nehmen wir an das die Zugkraft am rechten Ende des Stabes null ist, dass die L\"{a}ngsverschiebung des Stabes also die folgenden Eigenschaften hat
\begin{align} \label{Eq10}
-EA\,u''(x) = p(x) \qquad u(0) = 0\,, \qquad N(l) = 0\,.
\end{align}
(Mit einer Zugkraft am rechten Ende geht der Beweis aber genauso).

Den Ausdruck
\begin{align}
\Pi(u) = \frac{1}{2}\, \int_0^{\,l} \frac{N^2}{EA}\,dx - \int_0^{\,l} p(x)\,u(x)\,dx
\end{align}
nennt man die potentielle Energie des Stabes. Warum hier ein Faktor $1/2$ vor dem ersten Integral steht, aber bei dem zweiten Integral fehlt hat damit zu tun, dass unter dem ersten Integral
ein Quadrat steht, $N^2 = (EA\,u')^2$, und wenn man dieses nach $u$ ableitet, dann k\"{u}rzt sich der Faktor $1/2$ weg.
Dies soll an dieser Stelle als Hinweis gen\"{u}gen. Zur genauen Kl\"{a}rung der Details ben\"{o}tigt man den Begriff der Gateaux-Ableitung, was hier zu weit f\"{u}hren w\"{u}rde.

Mathematisch ist die potentielle Energie ein sogenanntes {\em Funktional\/}, ein Ausdruck, in den man eine Funktion einsetzt und eine Zahl zur\"{u}ckbekommt, also, eine Funktion von Funktionen.

\"{A}hnlich wie bei der Feder gilt auch f\"{u}r den Stab ein Minimumsprinzip: Die L\"{a}ngsverschiebung $ u(x) $ des Stabes macht die potentielle Energie auf $V$ zum Minimum, ist also der Sieger in einem Wettbewerb, der auf $ V $ stattfindet. Was ist $V$? Die Menge $V$, oder was besser klingt, der Ansatzraum $V$, wird gebildet von allen Funktionen $ u(x)$, die den Lagerbedingungen des Stabes gen\"{u}gen, die also an der Stelle $ x = 0$ den Wert null haben.

Eine Schar von solchen Funktionen sind zum Beispiel die Funktionen
\begin{align}\label{Eq6}
\sin(x), \sin(x^2), \sin(x^3), \ldots, x, x^2, x^3, \ldots x^n, e^x - 1, e^{x^2} - 1,
\end{align}
Alle diese Funktionen, und viele weitere mehr, nehmen also an dem Wettbewerb teil, bei dem es darum geht die Funktion $u(x)$ zu finden, die den kleinsten Wert f\"{u}r die potentielle Energie liefert. Wir behaupten, dass der Sieger die L\"{a}ngsverschiebung $u(x)$ des Stabes ist.

Um dies zu zeigen, argumentieren wir wie folgt: Wenn $u(x)$ den tiefsten Punkt markiert, dann muss in jedem Nachbarpunkt  $u(x) + \hat{u}(x)$ die potentielle Energie gr\"{o}{\ss}er sein, es muss also gelten
\begin{align} \label{Eq7}
\Pi(u + \hat{u}) - \Pi(u) > 0\,.
\end{align}
Die Funktion, die wir zu $ u(x) $ hinzu addieren ist irgendeine Funktion $\hat{u}$ aus $V$, so dass die Bedingung $u(0) + \hat{u}(0) = 0 $ erhalten bleibt.

Um (\ref{Eq7}) zu zeigen, ist es sinnvoll, die potentielle Energie in eine abk\"{u}rzenden Form
\begin{align}
\Pi(u) = \frac{1}{2}\, \int_0^{\,l} \frac{N^2}{EA}\,dx - \int_0^{\,l} p(x)\,u(x)\,dx = \frac{1}{2}\, a(u,u) - (p,u)
\end{align}
zu schreiben. Dabei ist (man beachte, dass das zweite Argument $\hat{u}$ zun\"{a}chst nicht dasselbe ist wie das erste)
\begin{align}
a(u,\hat{u}) = \int_0^{\,l} EA\,u'\,\hat{u}'\,dx = \int_0^{\,l} \frac{N\,\hat{N}}{EA}\,dx
\end{align}
und
\begin{align}
(p,u) = \int_0^{\,l} p(x)\,u(x)\,dx\,.
\end{align}
Der Ausdruck $a(u,\hat{u})$ ist in beiden Argumenten linear. Das bedeutet, wenn $ u(x) $ und $ \hat{u}(x)$ zusammengesetzte Funktionen sind,
\begin{align}
u(x)= c_1\,u_1(x) + c_2\,u_2(x) \qquad \hat{u}(x) = d_1\,\hat{u}_1(x) + d_2\,\hat{u}(x)\,,
\end{align}
hier sind $c_1, c_2$ und $d_1, d_2$  beliebige Zahlen, dass dann die Form $a(u,\hat{u}$) in ihre einzelnen Bestandteile zerlegt werden kann
\begin{align}
a(u,\hat{u}) = c_1\,d_1\,a(u_1,\hat{u}_1) + c_1\,d_2\,a(u,\hat{u}_2) + c_2\,d_1\,a(u_1,\hat{u}_1) + c_2\,d_2\,a(u_2,\hat{u}_2)
\end{align}
Wegen dieser 'zweifachen' Linearit\"{a}t, im ersten und im zweiten Argument, nennt man $ a(u,\hat{u}) $ eine {\em Bilinearform\/} und weil $a(u,\hat{u})$ dasselbe ist wie $a(\hat{u},u)$, eine
{\em symmetrische Bilinearform\/}.

Das Arbeitsintegral $(p,u)$ hei{\ss}t eine Linearform, weil es in dem zweiten Argument linear ist,
\begin{align}
(p,u_1 + u_2) = (p,u_1) + (p,u_2)\,.
\end{align}
So vorbereitet k\"{o}nnen wir nun an den Beweis von (\ref{Eq7}) gehen. Die potentielle Energie an der Nachbarstelle hat den Wert
\begin{align} \label{Eq8}
\Pi(u + \hat{u}) &= \frac{1}{2}\,a(u + \hat{u},u + \hat{u}) - (p,u + \hat{u}) \nn \\
&= \frac{1}{2}\,a(u,u) + a(u,\hat{u}) + \frac{1}{2}\, a(\hat{u},\hat{u}) - (p,u) - (p,\hat{u})\,.
\end{align}
Hierbei haben wir die Symmetrie der Bilinearform ausgenutzt
\begin{align}
\frac{1}{2}\, a(u,\hat{u}) + \frac{1}{2}\, a(\hat{u},u)  = a(u,\hat{u})\,.
\end{align}
Man sieht leicht, dass (\ref{Eq8}) identisch ist mit
\begin{align}
\Pi(u + \hat{u}) &=\Pi(u) + a(u,\hat{u}) - (p,\hat{u}) + \frac{1}{2}\, a(\hat{u},\hat{u})
\end{align}
und somit folgt
\begin{align} \label{Eq9}
\Pi(u + \hat{u}) -\Pi(u) = a(u,\hat{u}) - (p,\hat{u})+ \frac{1}{2}\, a(\hat{u},\hat{u})
\end{align}
Nun ist aber, ausgeschrieben,
\begin{align}
a(u, \hat{u}) - (p,\hat{u}) = \int_0^{\,l} \frac{N(x)\,\hat{N}(x)}{EA})\,dx - \int_0^{\,l} p(x)\,\hat{u}(x)\,dx
\end{align}
was ja gerade die erste Greenschen Identit\"{a}t ist, von der wir wissen, dass sie null ist,
\begin{align}
G(u,\hat{u}) = \int_0^{\,l} p(x)\,\hat{u}(x)\,dx + \underbrace{N(l)\,\hat{u}(l) - N(0)\,\hat{u}(0)}_{= 0} - \int_0^{\,l} \frac{N(x)\,\hat{N}(x)}{EA}\,dx = 0
\end{align}
Die Arbeiten an den beiden Enden des Stabes sind Null, weil $N(l) = 0$ und $\hat{u}(0)= 0$ und wir d\"{u}rfen f\"{u}r $- EA\,u''(x) = p(x)$ setzen, weil wir wissen das die L\"{a}ngsverschiebung die Differentialgleichung in (\ref{Eq10}) erf\"{u}llt (der Ingenieur w\"{u}rde sagen, weil der Stab im Gleichgewicht ist).

Somit reduziert sich (\ref{Eq9}) auf
\begin{align}
\Pi(u + \hat{u}) -\Pi(u) = \frac{1}{2}\, a(\hat{u},\hat{u}) = \frac{1}{2}\, \int_0^{\,l} \frac{\hat{N}^2(x)}{EA}\,dx > 0
\end{align}
und dieser Ausdruck ist immer gr\"{o}{\ss}er null, solange $\hat{N} \neq 0$. In jedem Nachbarpunkt $u + \hat{u} $ ist also die potentielle Energie gr\"{o}{\ss}er als im Punkt $u$, d.h. $\Pi(u)$ muss wirklich der tiefste Punkt sein.

Man beachte, dass wir auf diese Art und Weise das Problem umgangen haben, die potentielle Energie differenzieren zu m\"{u}ssen, wie das bei normalen Minimax-Aufgaben usus ist.

Dasselbe, was wir bei der Feder \"{u}ber das Minimum und Maximum der potentiellen Energie gesagt haben, gilt auch hier. In der tiefsten Lage, in der Gleichgewichtslage $u(x)$, ist wegen
\begin{align}
G(u,u) = \int_0^{\,l} p\,u\,dx - \int_0^{\,l} \frac{N^2}{EA}\,dx = 0
\end{align}
die potentielle Energie negativ
\begin{align}
\Pi(u) = \frac{1}{2}\, \int_0^{\,l} \frac{N^2}{EA}\,dx - \int_0^{\,l} p\,u\,dx = \frac{1}{2}\,\int_0^{\,l} p\,u\,dx - \int_0^{\,l} p\,u\,dx = - \frac{1}{2}\,\int_0^{\,l} p\,u\,dx
\end{align}
denn das letzte Integral ist die Eigenarbeit der Kr\"{a}fte $p$ auf ihren Wegen $u$ und diese Eigenarbeit ist, wenn wir nicht eine ganz exotische Mechanik betreiben, immer positiv.

Wenn wir dagegen dem Stab einer Verformung aufzwingen, etwa das rechte Ende um eine Strecke $\bar{u} $ ziehen, dann m\"{u}ssen wir eine L\"{a}ngsverschiebung $ u(x) $ finden, die den Gleichungen
\begin{align}
- EA\,u''(x) = 0 \qquad u(0) = 0 \qquad u(l) = \bar{u}
\end{align}
gen\"{u}gt.

In der potentiellen Energie kommt nun keine \"{a}u{\ss}ere Kraft vor, und daher reduziert sich die potentielle Energie auf den positiven Ausdruck
\begin{align}
\Pi(u) = \frac{1}{2}\, \int_0^{\,l} \frac{N^2}{EA}\,dx
\end{align}
Wenn wir jetzt die potentielle Energie zum Minimum machen wollen, dann bedeutet das wirklich, dass wir die Funktion $ u(x) $ finden m\"{u}ssen, die $\Pi(u) $ m\"{o}glichst nahe an Null r\"{u}ckt. Der Stab will also mit m\"{o}glichst wenig Anstrengung der Bewegung $\bar{u} $ nachgeben.

\subsubsection{Der Formalismus}
Der mathematische Hintergrund ist der folgende: Alle Funktionen $u(x)$, die an dem Wettbewerb teilnehmen, m\"{u}ssen an den Enden die Werte $u(0) = 0$ und $u(l) = \bar{u}$ haben. Wir nennen diesen Raum den Raum $ V $. Ihn kann man sich so entstanden denken, dass man zu einer fest gew\"{a}hlten Funktionen $v(x)$ mit den Eigenschaften $v(0) = 0$ und $v(l) = \bar{u}$ lauter Funktionen $u$ mit der Eigenschaft $u(0) = 0$ und $u(l) = 0$ hinzu addiert.

Beim Nachweis
\begin{align}
\Pi(u + \hat{u}) - \Pi(u) =  a(u,\hat{u}) + \frac{1}{2}\, a(\hat{u},\hat{u}) > 0
\end{align}
ist $\hat{u}$ dann eine solche Testfunktion, $\hat{u}(0) = \hat{u}(l) = 0$. Wegen
\begin{align}
G(u,\hat{u}) = \int_0^{\,l} 0\cdot \hat{u}\,dx + N(l)\cdot 0 - N(0)\cdot 0 - a(u,\hat{u}) = - a(u,\hat{u}) = 0
\end{align}
reduziert sich das aber auf
\begin{align}
\Pi(u + \hat{u}) - \Pi(u) =  \frac{1}{2}\, a(\hat{u},\hat{u}) > 0
\end{align}
also einen positiven Ausdruck und damit ist alles gezeigt, dass $u$ die potentielle Energie in der Tat zum Minimum macht.\\

%%%%%%%%%%%%%%%%%%%%%%%%%%%%%%%%%%%%%%%%%%%%%%%%%%%%%%%%%%%%%%%%%%%%%%%%%%%%%%%%%%%%%%%%%%%%%%%%%%%
\section{Der Balken}
Nachdem wir also jetzt zuerst am Beispiel einer Feder und eines Stabes die Energie-und Arbeitsprinzipe formuliert haben, wollen wir nun dasselbe auch f\"{u}r Balken, dem vielleicht wichtigsten statischen Element, tun.

Das sch\"{o}ne an den Arbeitsprinzipien ist, dass die Vorgehensweise immer dieselbe ist und auch die mathematischen Strukturen immer dieselben sind. Wir beginnen mit einer Differentialgleichung, die den Zusammenhang zwischen der \"{a}u{\ss}eren Kraft und der Verformung des Bauteils beschreibt und leiten die erste Greensche Identit\"{a}t f\"{u}r diese Differentialgleichung her. All die dann folgenden Schritte sind im Grunde identisch mit den Schritten, die wir bei der Feder bzw. bei dem Stab gemacht haben. Und es wird sich zeigen, dass auch bei Fl\"{a}chentragwerken die Logik immer dieselbe ist.\\

Wie ist das, wenn federende Lager vorhanden sind? Dann sind die $\delta w $ (in der Regel) in den Lagern nicht null und ein solches Lager im Punk $x_L$ mit der Federsteifigkeit $c $ liefert somit einen Beitrag
\begin{align} \label{Eq22}
\underbrace{c\,w(x_L)}_{Kraft}\,\underbrace{\delta w(x_L)}_{Weg}
\end{align}
Ist das nun virtuelle innere Energie oder virtuelle \"{a}u{\ss}ere Arbeit? Es ist
virtuelle innere Energie, wird also auf der Seite von $\delta A_i $ verbucht. Die Energie ist positiv, wenn die Zusammendr\"{u}ckung des Lagers, $w(x_L) $, dieselbe Richtung hat, wie die virtuelle Verr\"{u}ckung $\delta w(x_L) $.

Man kann aber (\ref{Eq22}) auch mit der Lagerkraft $F$ schreiben, also mit der  Kraft, mit der der Boden die Feder st\"{u}tzt
\begin{align} \label{Eq22}
-F\,\delta w(x_L) \,.
\end{align}
Das Minus ist der Tatsache geschuldet, dass $F $ und $w(x_L) $ immer entgegengesetzte Richtungen haben.

Im Falle des federnd gelagerten Tr\"{a}gers, s. Bild \ref{S8}, gibt es also zwei Schreibweisen f\"{u}r $\delta A_a = \delta A_i$
\begin{align}
\int_0^{\,l} p\,\delta w\,dx = \int_0^{\,l} \frac{M\,\delta M}{EI}\,dx + c\,w(l)\,\delta w(l)
\end{align}
oder
\begin{align}
\int_0^{\,l} p\,\delta w\,dx = \int_0^{\,l} \frac{M\,\delta M}{EI}\,dx - F\,\delta w(l)
\end{align}
Man gewinnt am besten Klarheit \"{u}ber das Problem, wenn man die Steifigkeitsmatrix der Feder zur Hilfe nimmt
\begin{align}
\vek  K\,\vek u = \vek f\,.
\end{align}
am Boden ist $u_2 = 0 $ und so reduziert sich alles auf
\begin{align}
k\,u_1 = f_1 \qquad -k\,u_1 = f_2\,.
\end{align}
\\

Das ist die Mohrsche Arbeitsgleichung. In der Literatur wird zum Beweis dieser Formel an das {\em Prinzip der virtuellen Kr\"{a}fte\/} appelliert, das Ergebnis durch Anwendung eines mechanischen Prinzips hergeleitet.

Das {\em Prinzip der virtuellen Kr\"{a}fte\/} besagt das folgende: wenn $w$ die Biegelinie eines Balkens ist, (zu vorgegebener \"{a}u{\ss}erer Belastung), das unter der Einwirkung von \"{a}u{\ss}eren Kr\"{a}ften im Gleichgewicht ist


 kann man Durchbiegungen berechnen. Auch diese Gleichung basiert auf der ersten Greenschen Identit\"{a}t und zwar auf dem {\em Prinzip der virtuellen Kr\"{a}fte\/}
\begin{align}
G(\delta w^*,w) = \int_0^{\,l} EI\,\delta w^{*IV}\,w\,dx + [\delta V^*\,w - M*\,\delta w'] - \int_0^{\,l} \frac{M\,M*}{EI} dx = 0
\end{align}
Das erste Argument ist also $\delta u*$ und erst das zweite Argument ist die Biegelinie des Tr\"{a}gers.

Anschaulich geschieht das Folgende: Man belastet eine Kopie des Tr\"{a}gers mit einer Einzelkraft $\bar{P} = \bar{1}$ (in Richtung der gesuchten Verschiebung). Die Arbeit, die diese 'virtuelle' Kraft auf dem Weg $w(x)$ am Ort von $\bar{P}$ leistet, ist gleich der virtuellen inneren Energie zwischen $w*$, der Biegelinie zu $\bar{P} = \bar{1}$, und der Biegelinie $w$ des Tr\"{a}gers.\\

So weit die verbale Beschreibung, aber mit Begriffen l\"{a}sst sich nichts beweisen, dazu muss man Mathematik machen.
\\

\begin{align}
A\,\delta w(x_a) &=  [V\, \delta w - \ldots]_{x_a}^{x_b} \\
B\,\delta w(x_b) &=  [\ldots + V\,\delta w]_{x_a}^{x_b} + [V\,\delta w + \ldots]_{x_b}^{x_c}\\
P\,\delta w(x_P) &=  [\ldots + V\,\delta w]_{x_b}^{x_P} + [V\,\delta w + \ldots]_{x_P}^{x_c}\\
C\,\delta w(x_c) &=  [\ldots + V\,\delta w]_{x_b}^{x_c}
\end{align}
\\

Weil nun f\"{u}r kleine Winkel der Tangens und der Winkel nahezu gleich gro{\ss} sind
\begin{align}
\tan\,\Np \cong \Np
\end{align}
wird $\Np = \tan\,\Np$ gesetzt, was dann zu dem merkw\"{u}rdigen f\"{u}hrt, dass der eigentlich dimensionslose Tangens $w'$ pl\"{o}tzlich die Dimension Rad hat. Kaum ein Autor, der es wagt $w'$ ohne Dimension zu schreiben, wie es richtig w\"{a}re.
Denn es ist nicht der Drehwinkel $\Np $, der zu dem Biegemoment konjugiert ist, sondern der Tangens dieses Winkels
\begin{align}
w'(x) = \tan\,\Np(x)\,.
\end{align}
In der ersten Greensche Identit\"{a}t steht $M \cdot w'$ und nicht $M \cdot \Np$.
Autoren trennen hier nicht sauber. Besonders bei der Berechnung von Einflusslinien kommen so ganz merkw\"{u}rdige und h\"{o}chst irritierende Effekte zustande. Wir werden dar\"{u}ber sp\"{a}ter noch mehr zu sagen haben.



\colorbox{hellgrau}{Box mit hellgrauem Hintergrund}

\colorbox{hellgrau}{\parbox{0.8\textwidth}{Im Prinzip macht der Tragwerkskplaner nichts anders, nur vereinfacht er das Anschreiben der Identit\"{a}ten ganz wesentlich. Zun\"{a}chst gibt es bei ihm nur ein $u$ und ein $w$, denn man wei{\ss} ja automatisch welches $u_i $ oder $w_i $ gemeint ist, wenn man auf den oder den Stiel bzw. Riegel zeigt. Dasselbe gilt f\"{u}r die virtuellen Verr\"{u}ckungen $\delta u$ und $\delta w$. Dann hat der Tragwerksplaner einen intuitiven Begriff von dem, was \"{a}u{\ss}ere Arbeit ist, und so kann er die Summe der virtuellen \"{a}u{\ss}eren Arbeiten anschreiben, ohne viel Mathematik betreiben zu m\"{u}ssen}}\\
\\

%%%%%%%%%%%%%%%%%%%%%%%%%%%%%%%%%%%%%%%%%%%%%%%%%%%%%%%%%%%%%%%%%%%%%%%%%%%%%%%%%%%%%%%%%%%%%%%%%%%
\section{Rahmen (praktisch)}
. Zu jeder L\"{a}ngsverschiebung $u_i $ und Durchbiegung $w_i $ eines Stiels oder Riegels geh\"{o}rt ja eine eigene Identit\"{a}t und erst ihre Summe repr\"{a}sentiert dann die ganze Statik des Rahmens.\\

Im Prinzip macht der Tragwerkskplaner nichts anders, nur vereinfacht er das Anschreiben der Identit\"{a}ten ganz wesentlich. Zun\"{a}chst gibt es bei ihm nur ein $u$ und ein $w$, denn man wei{\ss} ja automatisch welches $u_i $ oder $w_i $ gemeint ist, wenn man auf den oder den Stiel bzw. Riegel zeigt. Dasselbe gilt f\"{u}r die virtuellen Verr\"{u}ckungen $\delta u$ und $\delta w$. Dann hat der Tragwerksplaner einen intuitiven Begriff von dem, was \"{a}u{\ss}ere Arbeit ist, und so kann er die Summe der virtuellen \"{a}u{\ss}eren Arbeiten anschreiben, ohne viel Mathematik betreiben zu m\"{u}ssen
\begin{align}
\delta A_a = P\,\delta w(x_P) + \int p\,\delta w\,dx + \ldots
\end{align}
Man sieht ja auf dem Plan, wo die Streckenlast $p$ wirkt, und deswegen kann man die Integrationsgrenzen weglassen, und so wird das Gesicht der \"{a}u{\ss}eren Arbeit $\delta A_a $ einfach durch die Verteilung der Lasten auf dem Rahmen diktiert.

Genauso geht der Tragwerkskplaner mit der virtuellen inneren Energie vor. Weil es nur noch ein $u$ bzw. $w$ gibt, gibt es auch nur noch eine Normalkraft $N$ und ein Moment $M$ in allen Stielen und Riegeln, so dass sich die virtuelle innere Energie zu
\begin{align}
\delta A_i = \int \frac{N\,\delta N}{EA}\,dx + \int \frac{M\,\delta M}{EI}\,dx
\end{align}
ergibt. Die fehlenden Integrationsgrenzen weisen darauf hin, dass \"{u}ber alle Stiele und Riegel zu integrieren ist. Feldweise sind dabei die lokalen Steifigkeiten, $EA $ und $EI $ einzusetzen.\\

Dies gilt auch dann, wenn federnde Lager vorhanden sind. Eine Feder ist ja ein zus\"{a}tzlicher Stiel (\"{a}hnlich einem Pendelstab) und der Punkt, in dem die Feder den Balken st\"{u}tzt, ist ein innenliegender Knoten. Das eigentliche Lager liegt tiefer, es ist der Punkt, in dem sich die Feder auf den Baugrund abst\"{u}tzt.

Das Tragwerk besteht jetzt aus Balken und Feder und dementsprechend muss man die virtuelle innere Energie beider Bauteile mitnehmen
\begin{align}
G(w, \delta w) + G(u, \delta u) = \int_0^{\,l} p(x)\,w(x)\,dx  - \int_0^{\,l} \frac{M\,\delta M}{EI}\,dx - \delta u\,k\,u = 0\,.
\end{align}
Wir haben hier die Randarbeiten gleich weggelassen, weil sie sich zu Null ergeben und ausgenutzt, dass man die virtuelle innere Energie der Feder als $\delta u\,k\,u $ schreiben kann,
wenn $u $ und $\delta u $ die Bewegungen am Kopf der Feder sind.

Auf dieses Ergebnis kommt man wie folgt: Die Federsteifigkeit eines Stabes ist $k = EA/l$ und daher ist umgekehrt das $EA $, das zu einer Feder geh\"{o}rt, $EA = k\,l$. Die Normalkraft in der Feder ist
$N\,= k\,u $, und es ist $\delta N\,= k\,\delta u$, so dass
\begin{align}
\int_0^{\,l} \frac{N\,\delta N}{EA}\,dx = \int_0^{\,l} \frac{k\,u \cdot k\,\delta u}{k\,l} \,dx = \delta u\,k\,u\,.
\end{align}
Man kann dieses Ergebnis nat\"{u}rlich auch aus der Steifigkeitsmatrix der Feder ableiten.
\\

\begin{framed}
  \begin{align*}
    \mbox{"`}1\mbox{"'} \cdot \delta_{ik}
    = &
    \underbrace{\int\frac{M_{i}\,M_{k}}{EI}\;}_{\text{Biegemomente}}
    + \underbrace{\int\frac{N_{i}\,N_{k}}{EA}\;x}_{\text{Normalkr\"{a}fte}}
    \\[1em]
    &
    + \underbrace{\sum_m \frac{F_{mi}F_{mk}}{c_{m}}}_{\text{Normalkraftfedern}}
    + \underbrace{\sum_m \frac{M_{mi}M_{mk}}{c_{\varphi m}}}_{\text{Biegemomentenfedern}}
    \\[2em]
    &
    + \underbrace{\int N_i\,\alpha_T\,T\;}_{\text{Gleichm. Erw\"{a}rmung}}
    + \underbrace{\int M_i\,\alpha_T\,\frac{\Delta T}{h}\;}_{\text{Ungleichm. Erw\"{a}rmung}}
    - \underbrace{\sum_n F_{ni}\,\delta^0_{n}}_{\text{Lagerverschiebungen}}
    - \underbrace{\sum_n M_{ni}\,\varphi^0_{n}}_{\text{Lagerverdrehungen}}
  \end{align*}
\end{framed}


\colorbox{hellgrau}{\parbox{0.8\textwidth}{Die Arbeitss\"{a}tze der Statik beruhen auf diesen Identit\"{a}ten.}}\\


\colorbox{hellgrau}{\parbox{0.8\textwidth}{Die Arbeitss\"{a}tze der Statik beruhen auf diesen Identit\"{a}ten.}}\\

Die erste Greenschen Identit\"{a}t erlaubt .
\begin{align}
G(w, \delta w) = 0 \qquad G(w,w) = 0 \qquad G(\delta w^*, w) = 0\,.
\end{align}
Diese entsprechen dem Prinzip der virtuellen Verr\"{u}ckungen, dem Energieerhaltungssatz und dem {\em Prinzip der virtuellen Kr\"{a}fte\/}
\begin{align}
\delta A_a - \delta A_i = 0 \qquad A_a - A_i = 0 \qquad \delta A_a^* - \delta A_i^* = 0\,.
\end{align}\\

So gilt f\"{u}r die Lagerkr\"{a}fte des Tr\"{a}gers in Bild \ref{U204},
\begin{align}
A - V_l = 0\, \quad  - V_l + B + V_r = 0 \quad - V_l + C = 0
\end{align}
und f\"{u}r die Spr\"{u}nge links und rechts von $P$ und $M$
\begin{align}
\uparrow\,- V_l + \downarrow\,P + \downarrow\,V_r = 0\qquad\qquad  \curvearrowright\, M_l + \curvearrowright\,M - \curvearrowleft\,M_r = 0\,,
\end{align}
so dass bei der Addition der Randarbeiten an den Intervallgrenzen die folgenden Arbeiten \"{u}brig bleiben
\begin{align}\label{Eq21}
A\,\delta w(x_1) + B\,\delta w(x_3) + C\,\delta w(x_6) + P\,\delta w(x_2) + M\,\delta w'(x_3)\,.
\end{align}
Diese Summe plus der virtuellen Arbeit der Streckenlast ist die gesamte virtuelle \"{a}u{\ss}ere Arbeit
\begin{align}
\delta A_a = \text{(\ref{Eq21})} + \int_{x_5}^{\,x_6} p\,\delta w\,dx\,.
\end{align}




Zum Abschluss nun noch die Anwendung des Prinzips der virtuellen Kr\"{a}fte auf diesen Tr\"{a}ger.

In Bild \ref{S4} d wirkt im Viertelspunkt $x_0$ eine Einzelkraft $\bar{P} = 1$ und erzeugt die Biegelinie $\delta w^*$. Die Biegelinie $w(x)$ des urspr\"{u}nglichen Lastfalls, Bild \ref{S4} a, ist eine zul\"{a}ssige virtuelle Verr\"{u}ckung f\"{u}r den Lastfall $\bar{P} = 1$, und so ergibt die Anwendung der ersten Greenschen Identit\"{a}t das Resultat
\begin{align}
G(\delta w^*,w) = \underbrace{\phantom{\int_0^{\,l} }\bar{P} \cdot w(x_0)}_{\delta A_a^*} - \underbrace{\int_0^{\,l} \frac{\delta M^*\,M}{EI}\,dx}_{\delta A_i^*} = 0
\end{align}
oder
\begin{align}
w(x_0) = \int_0^{\,l} \frac{\delta M^*\,M}{EI}\,dx\,,
\end{align}
was der Ingenieur als eine Anwendung des Prinzips der virtuellen Kr\"{a}fte interpretiert.\\

%----------------------------------------------------------------------------------------------------------
\begin{figure}[tbp]
\centering
\if \bild 2 \sidecaption \fi
\includegraphics[width=0.9\textwidth]{\Fpath/S8}
\caption{Federndes Lager} \label{S8}
%
\end{figure}%
%----------------------------------------------------------------------------------------------------------\\


Im Grunde empfiehlt es sich sowieso, vor der Formulierung der Identit\"{a}ten in Gedanken alle Lager zu entfernen, denn dann kann man den Balken beliebige virtuelle Verr\"{u}ckungen $\delta w$ erteilen ohne  fragen zu m\"{u}ssen, ob denn die virtuelle Verr\"{u}ckung zul\"{a}ssig sei. Das Entfernen der Lager hat auch noch den positiven Effekt, dass die Lagerkr\"{a}fte als \"{a}u{\ss}ere Kr\"{a}fte, sie halten den Balken ja in  der Schwebe, sichtbar werden.
\\

Die wesentlichen Informationen bei der Formulierung der ersten Greenschen Identit\"{a}t kommen aus den Randarbeiten. In der einen Variante steht dort
\begin{align}
G(w, \delta w) = \ldots + [V\,\delta w - M\,\delta w']_{@0}^{@l} - \ldots
\end{align}
und in der anderen Variante
\begin{align}
G(\delta w, w) = \ldots + [\delta V\, w - \delta M\, w']_{@0}^{@l} - \ldots
\end{align}
In der ersten Variante kann man durch geeignete Wahl von $\delta w$ bzw. $\delta w'$ Informationen \"{u}ber die Randwerte der Querkraft bzw. des Moments gewinnen, typischerweise die Lagerkr\"{a}fte, $\delta w = 1$ bzw. $\delta w' = 1$ im Lager.

In der zweiten Variante kann man durch geeignete Wahl von $\delta V $ bzw. $\delta M $ Informationen \"{u}ber die Weggr\"{o}{\ss}en an den Intervallgrenzen bekommen. Die Verl\"{a}ufe $\delta V$ und $\delta M$ geh\"{o}ren zu Einzelkr\"{a}ften bzw. Einzelmomenten mit denen man die Durchbiegungen oder die Verdrehung in einem bestimmten Punkt berechnen will. Weil in einem solchen Fall die zugeh\"{o}rige Biegelinie nicht mehr in $C^4 $ liegt, muss man den Tr\"{a}ger in zwei Teile teilen und so entstehen die Intervallgrenzen.


%%%%%%%%%%%%%%%%%%%%%%%%%%%%%%%%%%%%%%%%%%%%%%%%%%%%%%%%%%%%%%%%%%%%%%%%%%%%%%%%%%%%%%%%%%%%%%%%%%%
\subsection{Der Umgang mit den Identit\"{a}ten in der Praxis}

%%%%%%%%%%%%%%%%%%%%%%%%%%%%%%%%%%%%%%%%%%%%%%%%%%%%%%%%%%%%%%%%%%%%%%%%%%%%%%%%%%%%%%%%%%%%%%%%%%
\subsection{Rahmen}

Die Greenschen Identit\"{a}ten bilden in der Tat den Hintergrund eines gro{\ss}en Teils des Rechnens in der Statik, aber es w\"{a}re viel zu m\"{u}hsam die Identit\"{a}ten abschnittsweise f\"{u}r  $u$ und $w$, zu formulieren, eventuelle Sprungterme, die aus Einzelkr\"{a}ften oder Einzelmomenten resultieren, an den Intervallgrenzen zu ber\"{u}cksichtigen und das Ganze zu einem geschlossenen Ausdruck zu summieren.
%----------------------------------------------------------------------------------------------------------
\begin{figure}[tbp]
\centering
\if \bild 2 \sidecaption \fi
\centering
\includegraphics[width=0.9\textwidth]{\Fpath/S1}
\caption{Abgewinkelter Tr\"{a}ger} \label{S1}
\end{figure}%
%----------------------------------------------------------------------------------------------------------

Ingenieure haben sich daf\"{u}r das Prinzip der virtuellen Verr\"{u}ckung in und das {\em Prinzip der virtuellen Kr\"{a}fte\/} so zurechtgelegt, dass man all diese Schwierigkeiten vermeidet.

Der Rahmen in Bild \ref{S1} besteht aus zwei Balken und jeder Balken kann sich in seiner Achsrichtung $(u)$ und quer dazu $(w)$ verformen, so dass wir es mit vier Funktionen $u_1, w_1, u_2, w_2$ zu tun haben, die die Verformungen des Rahmens beschreiben und die innere Energie enth\"{a}lt Beitr\"{a}ge von allen vier Funktionen
\begin{align}
A_i = \frac{1}{2}\,\int_0^{\,l_1} \frac{N_1^2}{EA}\,dx + \frac{1}{2}\,\int_0^{\,l_1} \frac{M_1^2}{EI}\,dx + \frac{1}{2}\,\int_0^{\,l_2} \frac{N_2^2}{EA}\,dx + \frac{1}{2}\,\int_0^{\,l_2} \frac{M_2^2}{EI}\,dx
\end{align}
In der Statik schreiben wir aber viel k\"{u}rzer
\begin{align}
A_i = \frac{1}{2}\,\int_0^{\,l} \frac{N^2}{EA}\,dx + \frac{1}{2}\,\int_0^{\,l} \frac{M^2}{EI}\,dx
\end{align}
Genauso ist es mit der \"{a}u{\ss}eren Arbeit
\begin{align}
A_a  = \frac{1}{2}\,P_1\,u_1(x_1) + \frac{1}{2}\,\int_0^{\,x_2} p\,w_2\,dx + \frac{1}{2}\,P_2\,w_2(x_2)
\end{align}
wo die Unterscheidung zwischen $u_1$ und $u_2$ bzw. $w_1$ und $w_2$ weggelassen wird
\begin{align}
A_a = \frac{1}{2}\,P_1\,u(x_1) + \frac{1}{2}\,\int_0^{\,l} p\,w\,dx + \frac{1}{2}\,P_2\,w(x_2)
\end{align}
und die genauen Integrationsgrenzen, wie \"{u}blich, dem Bild entnommen werden m\"{u}ssen.
\\

Das Prinzip vom Minimum der potentiellen Energie fasst nun diese Beobachtungen wie folgt zusammen: Die Auslenkung $ u $ der Feder unter der Wirkung der Kraft $ f $ macht die potentielle Energie der Feder zum Minimum. Wenn man also nur lange genug Zufallszahlen $ u $ in die Funktion $\Pi(u)$ einsetzen w\"{u}rde, sich eine Liste der Wert $\Pi(u)$ machen w\"{u}rde, dann w\"{u}rde man automatisch zu der gesuchten Gleichgewichtslage $u$ der Feder gef\"{u}hrt.\\



Der Stab in Bild 1 a tr\"{a}gt eine Streckenlast und der Stab in Bild 1 b eine Einzelkraft. Der Satz von Bett besagt nun, dass Arbeiten, die Streckenlasten im Bild 1H auf den Wegen der Verformungen des Systems in Bild 1B leisten, genauso gro{\ss} ist, wie die Arbeit, die die Einzelkraft in Bild 1B auf den Wegen des Stabes in Bild 1A leisten
\begin{align}
A_{1,2} = \int_0^{\,l} p(x)\,u_2(x)\,dx = P \,u_1(l) = A_{2,1}\,.
\end{align}
Der {\em Satz von Betti\/} beruht auf der zweiten Greenschen Identit\"{a}t, die man einfach durch Spiegelung der ersten Identit\"{a}t erh\"{a}lt
\begin{align}
\text{\normalfont\calligra B\,\,}(u,\hat{u}) = \text{\normalfont\calligra G\,\,}(u,\hat{u}) - \text{\normalfont\calligra G\,\,}(\hat{u}.u) = 0 - 0 = 0
\end{align}
oder
\begin{align}
B(u,\hat{u}) &= \int_0^{\,l} - EA\,u''(x)\,\hat{u}(x)\,dx + [N\,\hat{u}]_{@0}^{@l} \nn \\
&- [u\,\hat{N}]_{@0}^{@l} - \int_0^{\,l} u(x)\,(- EA\,\hat{u}''(x))\,dx = 0
\end{align}
In der zweiten Zeile wiederholt sich also die erste Zeile, nur dass eben die Pl\"{a}tze von $ u $ und $ \hat{u} $ vertauscht sind. Diese Identit\"{a}t ist g\"{u}ltig f\"{u}r alle Paare von Funktionen $u$ und $\hat{u}$, die im Integrationsintervall in $C^2$ liegen.

Wir formulieren und diesen {\em Satz von Betti\/} mit den beiden Verschiebungen $ u_1 $ und $ u_2 $. Die linke Verschiebung (Streckenlast) gen\"{u}gt den Gleichungen
\begin{align}
- EA\,u_1''(x) = p(x) \qquad u_1(l) = 0 \qquad N_1(l) = 0
\end{align}
und die rechte Verschiebung (Einzelkraft) den Gleichungen
\begin{align}
- EA\,u_2''(x) = 0 \qquad u_2(0) = 0 \qquad N_2(l) = P\,.
\end{align}
Beide L\"{o}sungen liegen in $C^2$ und somit ist die zweite Greensche Identit\"{a}t anwendbar
\begin{align}
B(u_1,u_2) = \int_0^{\,l} p(x)\,u_2(x)\,dx - P\,u_1(l) = 0
\end{align}
oder umgestellt
\begin{align}
A_{1,2} = \int_0^{\,l} p(x)\,u_2(x)\,dx = P\,u_1(l) = A_{2,1}\,,
\end{align}
was genau der Inhalt des Satzes von Betti ist.
\\

Wie sieht es in einem solchen Falle mit dem Prinzip der virtuellen Verr\"{u}ckungen bzw. dem {\em Prinzip der virtuellen Kr\"{a}fte\/} aus?\\

Dass zul\"{a}ssige virtuelle Verr\"{u}ckungen im rechten Lager nicht null sein m\"{u}ssen, kann man wie folgt verstehen: der Lastfall Lagersenkung ist identisch mit einem Lastfall, bei dem am rechten Ende eine Einzelkraft $B$ den Tr\"{a}ger, jetzt als Kragtr\"{a}ger gedacht, nach unten zieht, und die zul\"{a}ssigen virtuellen Verr\"{u}ckungen an dem Kragtr\"{a}ger m\"{u}ssen nur die Lagerbedingungen $\delta w(0) = \delta w'(0) = 0$ erf\"{u}llen.

Wie ist das aber nun, wenn die (zul\"{a}ssige) virtuelle Verr\"{u}ckung im rechten Lager zuf\"{a}llig gerade null ist, was dann also
\begin{align} \label{Eq31}
G(w, \delta w) &= - \int_0^{\,l} \frac{M \delta M}{EI}\,dx = -\delta A_i = 0
\end{align}
zur Folge hat. Um dieses Ergebnis zu verstehen, argumentieren wir wie folgt: Ent\-weder ist $ \delta w'(l) = 0$ oder $ \delta w'(l) \neq 0$. Im ersten Fall kann $ \delta w(x)$ nur zwischen den Lagern von Null verschieden sein, da ja $\delta w$ in den Lagern 'stumm' ist. Da aber keine Streckenlast vorhanden ist, die auf dem Weg $\delta w(x)$ eine Arbeit leisten k\"{o}nnte, ist $\delta A_a = 0$. Im zweiten Falle k\"{o}nnte das rechte Balkenendmoment $M(l)$ auf der Drehung $\delta w'(l)$ eine Arbeit leisten, aber da das Moment null ist, ist auch diese Arbeit null und somit gilt auch in diesem Fall $\delta A_a = 0$ und (\ref{Eq31}) ist best\"{a}tigt.
\\
Das ist das Grundschema bei der Anwendung von Einflussfunktionen. Wir kennen die Belastung $p = EI\,w^{IV}$ auf einen Tr\"{a}ger, kennen also das Bild von $w$ unter der Abbildung $EI\,w^{IV}$ und wollen nun wissen, wie gro{\ss} die Durchbiegung $w$ an einer gewissen Stelle $x$ ist.\\

Es sind ja wieder die Randarbeiten, die die interessierenden Werte $M(x) $ und $V(x) $ oder $N(x) $ liefern und jetzt muss also das, was konjugiert ist zu den Kraftgr\"{o}{\ss}en, an den beiden Intervallgrenzen, links und rechts vom Aufpunkt springen, damit sich das gew\"{u}nschte Resultat ergibt
\begin{align}
M(x) \cdot (w'(x_{-}) - w'(x_+)) = M(x) \cdot 1\\
V(x) \cdot (w(x_{-}) - w(x_+)) = V(x)\cdot 1\\
N(x) \cdot (u(x_{-}) - u(x_+)) = N(x) \cdot 1
\end{align}
Ein Knick, also ein Sprung in der Neigung der Tangente, liefert die Einflussfunktionen f\"{u}r $M(x) $ und ein Sprung in der Biegelinie bzw. der L\"{a}ngsverschiebung liefert die Einflussfunktionen f\"{u}r $V(x) $ bzw. $N(x)$.

Wie das im Detail geht, wollen wir am Beispiel des Tr\"{a}gers in Bild X zeigen. Zun\"{a}chst wird der Tr\"{a}ger dupliziert, also eine getreue Kopie des Tr\"{a}gers erstellt.\\

Und wenn man dabei einen Fehler macht, wenn man ein falsches Signal aussendet, weil vielleicht die Spreizung nicht genau 1 ist, oder die Ausbreitung der Welle auf k\"{u}nstliche Hindernisse st\"{o}{\ss}t (falsch angesetzte Steifigkeiten), dann ist auch das Signal, das den Fu{\ss}punkt der Last erreicht falsch und dann ist auch die Schnittkraft falsch.



Der erste Schritt bei der Berechnung von Einflussfunktionen f\"{u}r Schnittkr\"{a}fte wie auch Lagerkr\"{a}ften ist, dass man die Kraft sichtbar macht, also zu einer \"{a}u{\ss}eren Kraft macht.
Dies erreicht man durch den Einbau eines entsprechenden Gelenks, das den Kraftfluss unterbricht und dazu zwingt mit einem \"{a}u{\ss}eren Kr\"{a}ftepaar das Gleichgewicht wieder herzustellen.

Diesem so modifizierten System erteilt man dann eine virtuelle Verr\"{u}ckung derart, dass die beiden Kr\"{a}fte links und rechts vom Gelenk den Weg $(-1)$ zur\"{u}ck legen.
Das f\"{u}hrt z.B. bei der Berechnung einer Einflussfunktion f\"{u}r eine Querkraft zu einem Beitrag
\begin{align}
V(x) \cdot 1
\end{align}
in der Bilanz $A_{1,2} = A_{2,1}$ oder, wenn man den Term auf eine Seite alleine bringt, auf die Einflussfunktion
\begin{align}
V(x) \cdot 1 = \ldots\,.
\end{align}\\


In der Praxis werden Einflussfunktionen nat\"{u}rlich nicht mehr mit dem Kraftgr\"{o}{\ss}enverfahren berechnet, sondern es werden FE-Programme benutzt und in diese sind die Routinen zur Berechnung der Einflussfunktionen fertig eingebaut. Welche Technik dabei zum Einsatz kommt, diskutieren wir in Abschnitt X.\\


Der Symmetrie der Wechselwirkungsenergie
\beq
 a(u,\hat{u})  = a(\hat{u},u)
\eeq
entspricht die Symmetrie der Steifigkeitsmatrix $\vek K$ und daher folgt, dass wenn $u$ und $G$ die Variationsl\"{o}sungen des {\em primalen\/} und des {\em dualen\/} Problems sind
\begin{align}
u \in \mathcal{V} \qquad a(u,v) &= (p,v) \qquad \forall\,v \in \mathcal{V} \qquad \mbox{{\em primal problem\/}} \\
G \in \mathcal{V} \qquad a(G,v) &= (\delta,v) \qquad \forall\,v \in \mathcal{V} \qquad \mbox{{\em dual problem\/}}
\end{align}
dies das Ergebnis
\beq
u(\vek x) = (p,G) = (\delta,u) \,.
\eeq
impliziert. Glg. (\ref{Eq41}) ist dasselbe Ergebnis 'ausgeschrieben'.



In einigen B\"{u}chern wird der Aufpunkt mit dem Buchstaben $\xi$ bezeichnet und man hat so den Buchstaben $x$ als Integrationsvariable frei, aber dann wird die Durchbiegungen eine Funktion von $\xi$, was auch ungewohnt ist
\beq
u(\xi) = \int_0^{\,l} G(x, \xi)\,p(x)\,dx \,.
\eeq
Wir ziehen daher die Kombination $x$ und $y$ vor.\\

%%%%%%%%%%%%%%%%%%%%%%%%%%%%%%%%%%%%%%%%%%%%%%%%%%%%%%%%%%%%%%%%%%%%%%%%%%%%%%%%%%%%%%%%%%%%%%%%%%%
\section{Was finite Elemente nicht sind}
Viele Ingenieure verstehen finite Elemente in etwa so: Man unterteilt eine Scheibe in kleine dreiecksf\"{o}rmige Elemente, die in den Knoten zusammenh\"{a}ngen, und man ersetzt die Belastung durch \"{a}quivalente Knotenkr\"{a}fte $f_i$ und bestimmt die Knotenverformungen $u_i $ so, dass in  allen Knoten Gleichgewicht herrscht
\begin{align}
\vek K\,\vek u = \vek f\,.
\end{align}
Diese Erkl\"{a}rung wird dann noch meist begleitet von einem Bild, wo man das urspr\"{u}ngliche Tragwerk sieht und daneben das FE Modell, s. Bild X.

Das ist sehr eing\"{a}ngig und deswegen ist das Modell wohl sehr popul\"{a}r, aber die Gleichung $\vek K\,\vek u = \vek f$ ist keine Gleichgewichtsbedingung und die Belastung wird auch nicht in die Knoten reduziert. Ebenso h\"{a}ngen die Elemente nicht nur in den Knoten zusammen, und daher tauschen sie \"{u}ber ihren ganzen Umfang ihre Kr\"{a}fte mit den Nachbarelementen aus.

Was die Situation so schwierig macht ist, dass man bei stabartigen Tragwerken durchaus so argumentieren kann. Dann sind die  Stiele und Riegel die einzelnen finiten Elemente und in den Knoten wirken echte Knotenkr\"{a}fte, weil die Methode der finiten Elemente bei Stabtragwerken (in Standardf\"{a}llen) im Grunde mit dem Drehwinkelverfahren identisch ist.

Wenn die Tr\"{a}ger aber ihren Querschnitt ver\"{a}ndern, also zum Beispiel gevoutet sind, dann ist auch das FE-Modell nur eine N\"{a}herung, weil es ja solche 'Feinheiten' in der Regel ignoriert.


\subsubsection{}
\subsubsection{Resum\'{e}}
Die erste und zweite Greensche Identit\"{a}t (Betti) kann man als unterschiedliche Bilanzen von acht Arbeiten lesen
\begin{align}
A_i = A_a \qquad \delta A_i = \delta A_a \qquad \delta A_i^* = \delta A_a^* \qquad A_{1,2} = A_{2,1} \qquad \text{(Betti)}
\end{align}
Damit, dass man eine Differentialgleichung w\"{a}hlt oder setzt, die das Bauteil regiert, legt man die Form der Gleichgewichtsbedingungen und s\"{a}mtliche Arbeits- und Energieprinzipe f\"{u}r das Bauteil fest. Diese folgen durch partielle Integration aus der Differentialgleichung, holen also im Grunde nur das aus der Differentialgleichung heraus, was implizit bei ihrer Herleitung in sie gesteckt wurde.

Das Rechnen der Statik besteht zu einem gro{\ss}en Teil auf der Anwendung der Greenschen Identit\"{a}ten. Um diese Identit\"{a}ten nun dem Ingenieur schmackhafter zu machen, hat man Prinzipe erfunden, das {\em Prinzip der virtuellen Kr\"{a}fte\/}, das Prinzip der virtuellen Verr\"{u}ckungen, den Energieerhaltungssatz usw. Dieses sind im Grunde nur verbale Umschreibungen der Greenschen Identit\"{a}ten, in Worte gefasste Zusammenfassungen.

Vom didaktischen Standpunkt aus ist das voll gelungen. Man w\"{u}sste es nicht besser zu machen, denn ohne diese Prinzipe w\"{u}rde man sich bei der statischen Untersuchung von Stockwerkrahmen mittels Arbeitsprinzipen (Mohr, Betti, etc.) hoffnungslos in der Formulierung der Greenschen Identit\"{a}ten verstricken.
\\

Wenn ein Vektor $\vek u $ das Gleichungssystem $\vek K\vek u = \vek f$ l\"{o}st, dann kann man das System auf beiden Seiten skalar mit einem Vektor $\vek \delta \vek u$ multiplizieren
\begin{align}
\vek K\vek u = \vek f \qquad \Rightarrow \qquad \vek \delta\vek u^T\,\vek K\,\vek u = \vek \delta \vek u^T\,\vek f
\end{align}




Angenommen $g(y,x)$ sei die Durchbiegung eines Balkens in einem festen Punkt $x$, wenn in einem abliegenden Punkt $y$ eine Einzelkraft angreift, dann ist
\begin{align}
w(x) = \int_0^{\,l} g(y,x)\,p(y)\,dy
\end{align}
die Durchbiegung im Punkt $x$, wenn eine Streckenlast $p$ wirkt.

Denn in Gedanken kann man die Streckenlast $p$ in eine Schar von lauter kleinen Einzelkr\"{a}ften $dP = p\,dy$ zerlegen.

Wenn ein Vektor $\vek u$ das Gleichungssystem
\begin{align}
\vek K\,\vek u = \vek f
\end{align}
l\"{o}st, dann ist
\begin{align}
\delta \vek u^T\,\vek K\,\vek u = \vek \delta \vek u^T\,\vek f
\end{align}
f\"{u}r alle Vektoren $\delta \vek u$.

Was wir oben an ganz elementaren Beispielen durchexerziert haben, wiederholt sich in der Folge mit komplexeren Tragwerken. Das Rechnen in der Statik ist zu einem gro{\ss}en Teil eine Anwendung dieser Identit\"{a}ten und zwar genau genommen, der ersten Greenschen Identit\"{a}t.

Immer dann, wenn, wie in der Statik der Kontinua, die Gleichgewichtslage eines Tragwerks von einer oder mehreren (Stiele, Riegel) Funktionen beschrieben wird, bildet die erste Greensche Identit\"{a}t das Kernst\"{u}ck.

Wir werden in der Folge Gelegenheit haben dieses zu vertiefen. Hier an dieser Stelle sei nur noch einmal betont, wie wichtig diese Identit\"{a}ten f\"{u}r das Rechnen in der Statik sind. Es sei dies auch noch einmal zum Anlass genommen, darauf hinzuweisen, dass die Arbeits- und Energieprinzipe der Statik und Mechanik mathematischer Natur sind. Anders gesagt: das Prinzip der virtuellen Verr\"{u}ckungen ist kein Naturgesetz, sondern umgekehrt
\subsection{Skalarprodukt}
Beginnen wir mit zwei Vektoren, einem Verschiebungsvektor $\vek u $ und einem Kraftvektor $\vek f $. Der Vektor $\vek u = \{u_1,u_2,u_3\}^T$ enth\"{a}lt die Durchbiegungen der drei Innenknoten des Seils und der Vektor $\vek f =  \{f_1,f_2,f_3\}^T$ ist die Liste der Knotenkr\"{a}fte.

Die Arbeit, die die Kr\"{a}fte $\vek f $ auf den Wegen $\vek u $ leisten, ist das Skalarprodukt
zwischen den beiden Vektoren
\begin{align}
\vek u \dotprod \vek f = \vek u^T\,\vek f = u_1\,f_1 + u_2\,f_2 + u_3\,f_3 = |\vek u| \,|\vek f| \,\cos\,\Np
\end{align}
F\"{u}r dieses gibt es wie man sieht, zwei verschiedene Schreibweisen, einmal die klassische mit Punkt, $\vek u \dotprod \vek f$, und dasselbe auch noch mal in Matrizenschreibweise
\begin{align}
\vek u^T\,\vek f = [u_1, u_2, u_3] \cdot \left [\barr{c}  f_1 \\  f_2 \\ f_3\earr \right ]
\end{align}
als das Produkt eines Zeilenvektors  mit einem Spaltenvektor.

Der Zusammenhang zwischen dem Verschiebungsvektor $\vek u $ und dem Kraftvektor $\vek f $ wird in der linearen Statik durch eine Steifigkeitsmatrix $\vek K $ beschrieben
\begin{align}
\vek K\,\vek u = \vek f\,.
\end{align}
Im Falle eines Seils, das zwischen zwei Hausw\"{a}nden h\"{a}ngt, s. Bild X, ist das zum Beispiel die Matrix
\beq\label{Eq176}
 \left[\barr{r r r} 2 & - 1 & 0 \\ - 1 & 2 & -1 \\ 0 & -1 & 2 \earr\right]
\,\left[\barr{c} u_1 \\u_2 \\ u_3 \earr \right] = \left[\barr{c} f_1 \\ f_2  \\
f_3  \earr \right]\,.
\eeq
Damit  ist praktisch schon die ganze Statik beschrieben. Das Seil wird mit Knotenkr\"{a}ften $\vek f$  belastet,  und die Knoten geben nach, sie senken sich, Knotenvektor $\vek u $.

Nun wollen wir die Arbeits- und Energieprinzipe des Seils formulieren. Das Transponierte eines Spaltenvektors ist ein Zeilenvektor und umgekehrt. Das Transponierte einer reellen Zahl wie $\pi$ jedoch, ist dieselbe Zahl, $\pi^T = \pi$.

Im folgenden seien $\vek u$ und $\vek \delta \vek u$ zwei beliebige Vektoren.  Im ersten Schritt multiplizieren wir die Steifigkeitsmatrix $\vek K $ mit dem Vektor $\vek u $ und diesen Vektor dann skalar mit dem zweiten Vektor, $\vek \delta \vek u^T\,\vek K\,\vek u$. Das Ergebnis ist also eine Zahl,  wir nennen sie $a$, die man transponieren kann, ohne die Zahl zu \"{a}ndern
\begin{align}
a^T = a
\end{align}
Der folgende Ausdruck ist daher eine Identit\"{a}t
\begin{align}
G(\vek u,\vek \delta \vek u) =  \vek u^T \,\vek K\,\vek \delta \vek u- \vek \delta \vek u^T\vek K\,\vek u = a^T - a = 0\,.
\end{align}
Auf dieser Identit\"{a}t beruhen die Arbeits- und Energieprinzipe des Seils. Um das recht w\"{u}rdigen zu k\"{o}nnen, ben\"{o}tigen wir noch zwei Definitionen: Die innere Energie des Seils ist der Ausdruck
\begin{align}
A_i = \frac{1}{2}\, \vek u^T\,\vek K\,\vek u
\end{align}
und die potentielle Energie des Seils ist die Funktion
\begin{align}
\Pi(\vek u)= \frac{1}{2}\, \vek u^T\,\vek K\,\vek u - \vek f^T\,\vek u
\end{align}
Sie ist eine Funktion der Knotenverschiebungen $u_i$ und ihre Ableitungen nach den $u_i$  bilden den Vektor
\begin{align}
\nabla \Pi(\vek u) = \{\frac{\partial \Pi}{\partial u_1}, \frac{\partial \Pi}{\partial u_2}, \frac{\partial \Pi}{\partial u_3}\}^T
\end{align}
Ihn nennt man den Gradienten von $\Pi(\vek u)$. Man findet leicht, dass
\begin{align}
\nabla \Pi(\vek u) = \vek K\,\vek u - \vek f
\end{align}
Zum Vergleich betrachte man die Funktion
\begin{align}
\Pi(u) = \frac{1}{2}\, k\,u^2 - f\,u
\end{align}
und leite sie nach $u$ ab
\begin{align}
\frac{d\Pi}{du} = k\,u - f
\end{align}
So versteht man, warum der Gradient $\nabla\,\pi(\vek u)$ die obige Gestalt hat.
\\

Oder $w_1(x)$ und $w_2(x)$ seien die Biegelinien eines gelenkig gelagerten Einfeldtr\"{a}gers in zwei verschiedenen Lastf\"{a}llen
\begin{align}
EI\,w_1^{IV}(x) = p_1(x) \qquad  EI\,w_2^{IV}(x) = p_2(x)\,.
\end{align}
Multiplikation und Integration \"{u}ber Kreuz ergibt
\begin{align}
\int_0^{\,l} w_2(x)\,EI\,w_1^{IV}(x) \,dx \qquad \int_0^{\,l} w_1(x)\,EI\,w_2^{IV}(x) \,dx\,.
\end{align}
Um jetzt weiter zu kommen, m\"{u}ssen wir partielle Integration anwenden
\begin{align}
\int_0^{\,l} w_2(x)\,EI\,w_1^{IV}(x) \,dx = \int_0^{\,l} EI\,w_2''\,w_1''\,dx
\end{align}\\

Dem Tragwerkplaner w\"{u}rde es viel zu lange dauern, bei einem dreist\"{o}ckigen Stockwerkrahmen all diese Identit\"{a}ten einzeln anzuschreiben. Daher gibt es bei ihm nur ein $u$ und ein $w$, denn man wei{\ss} ja automatisch welches $u_i $ oder $w_i $ gemeint ist, wenn man auf den oder den Stiel bzw. Riegel zeigt. Dasselbe gilt f\"{u}r die virtuellen Verr\"{u}ckungen $\textcolor{red}{\delta u}$ und $\textcolor{red}{\delta w}$. Und dann hat der Tragwerksplaner einen intuitiven Begriff von dem, was \"{a}u{\ss}ere Arbeit ist, und so kann er die Summe der virtuellen \"{a}u{\ss}eren Arbeiten anschreiben, ohne viel Mathematik betreiben zu m\"{u}ssen
\begin{align}\label{Eq33}
\textcolor{red}{\delta A_a} = P\cdot\textcolor{red}{\delta w(x_P)} + \int p_z(x)\,\textcolor{red}{\delta w(x)}\,dx + \ldots
\end{align}
Man sieht ja auf dem Plan, wo die Streckenlasten $p_z$ und $p_x$ wirken, und deswegen kann man die Integrationsgrenzen weglassen, und so wird das Gesicht der \"{a}u{\ss}eren Arbeit $\textcolor{red}{\delta A_a} $ einfach durch die Verteilung der Lasten auf dem Rahmen diktiert.

Genauso vereinfacht der Tragwerkskplaner die virtuelle innere Energie. Weil es nur noch ein $u$ bzw. $w$ gibt, gibt es auch nur noch eine Normalkraft $N$ und ein Moment $M$ in allen Stielen und Riegeln, so dass sich die virtuelle innere Energie zu
\begin{align}
\textcolor{red}{\delta A_i} = \int \frac{N\,\textcolor{red}{\delta N}}{EA}\,dx + \int \frac{M\,\textcolor{red}{\delta M}}{EI}\,dx
\end{align}
ergibt.\\

Die fehlenden Integrationsgrenzen weisen darauf hin, dass \"{u}ber alle Stiele und Riegel zu integrieren ist. Feldweise sind dabei die lokalen Steifigkeiten, $EA $ und $EI $, einzusetzen.

Diese Vereinfachung macht den Umgang mit den Identit\"{a}ten sehr bequem. In der Statik reduziert man die Beitr\"{a}ge zu den Greenschen Identit\"{a}ten auf die folgenden acht Arbeiten
\begin{align}
A_a\,, A_i \qquad \delta A_i\,, \delta A_a \qquad \delta A_i^*\,, \delta A_a^* \qquad A_{1,2}\,, A_{2,1}\,\,\text{(Betti)}
\end{align}
und es ist dann der Sorgfalt des Ingenieurs \"{u}berlassen, die \"{a}u{\ss}eren und inneren Arbeiten richtig einzusortieren und zu bilanzieren. Es m\"{o}gen noch so viele St\"{a}be und Riegel sein, die einen Rahmen bilden, jedes Bauteil mit seiner eigenen Greenschen Identit\"{a}t, aber am Schluss reduzieren sich alle Beitr\"{a}ge in den Greenschen Identit\"{a}ten auf die obigen Arbeiten.

Dass diese Regel so einfach ist, beruht darauf, dass jede Identit\"{a}t f\"{u}r sich null ist und damit auch die Summe aller Identit\"{a}ten
\begin{align}
0 + 0 + 0 + \ldots + 0 = 0
\end{align}
oder wenn man das ganze, wie ein Buchhalter nach Soll und Haben, nach au{\ss}en und innen, verteilt
\begin{align}
\delta A_a = \delta A_i \qquad A_a = A_i \qquad \delta A_a^* = \delta A_i^*\,.
\end{align}




%----------------------------------------------------------------------------------------------------------
\begin{figure}[tbp]
\centering
\if \bild 2 \sidecaption \fi
\includegraphics[width=1.0\textwidth]{\Fpath/S4}
\caption{Durchlauftr\"{a}ger} \label{S4}
%
\end{figure}%
%----------------------------------------------------------------------------------------------------------\\

%%%%%%%%%%%%%%%%%%%%%%%%%%%%%%%%%%%%%%%%%%%%%%%%%%%%%%%%%%%%%%%%%%%%%%%%%%%%%%%%%%%%%%%%%%%%%%%%%%
{\textcolor{blau2}{\subsection{Beispiel}}}
Bei der Anwendung der Greenschen Identit\"{a}ten auf die Biegelinie des Durchlauftr\"{a}gers in Bild \ref{S4} a muss man, wie oben erl\"{a}utert, an jedem Zwischenlager und im Punkt $x_P$ anhalten, weil man \"{u}ber Lagerkr\"{a}fte und Einzelkr\"{a}fte nicht einfach hinweg integrieren kann. Die gesamte Identit\"{a}t ist also eine Summe von drei Termen
\begin{align}
G(w,\textcolor{red}{\delta w}) &= G(w_1, \textcolor{red}{\delta w})_{(x_a,x_b)} + G(w_2,\textcolor{red}{\delta w})_{(x_b,x_P)} + G(w_3,\textcolor{red}{ \delta w})_{(x_P,x_c)}\nn \\
&= 0 + 0 + 0 = 0\,,
\end{align}
oder
\begin{align}
G(w,\textcolor{red}{\delta w}) &= \int_0^{\,l} p\,\textcolor{red}{\delta w(x)}\,dx + A\cdot\textcolor{red}{\delta w(x_a)} + B\cdot\textcolor{red}{\delta w(x_b)} + C\cdot\textcolor{red}{\delta w(x_c)} + P\cdot\textcolor{red}{\delta w(x_P)}\nn \\
& - \int_0^{\,l} \frac{M\,\textcolor{red}{\delta M}}{EI}\,dx = 0\,.
\end{align}
Die Einzelarbeiten $ A\,\textcolor{red}{\delta w(x_a)}$ etc. kommen dabei aus den eckigen Klammern, sind also die \"{u}brig gebliebenen Randarbeiten an den Balkenenden bzw. die aufaddierten Randarbeiten an den \"{U}bergangsstellen, wie z.B. im Punkt $x_b$
\begin{align}
[\ldots + V\,\textcolor{red}{\delta w}]_{x_a}^{x_b} + [V\,\textcolor{red}{\delta w} + \ldots]_{x_b}^{x_c} = (V(x_b^-) - V(x_b^+))\,\textcolor{red}{\delta w(x_b)} = B\,\textcolor{red}{\delta w(x_b)}\,.
\end{align}
Setzt man f\"{u}r $\textcolor{red}{\delta w(x)}$ die Biegelinie selbst, so erh\"{a}lt man (nach Multiplikation mit dem Faktor $1/2$) den Energieerhaltungssatz
\begin{align}
\frac{1}{2}\, G(w,w) = \underbrace{\frac{1}{2}\,\int_{0}^{\,l} p(x)\,w_1(x)\,dx + \frac{1}{2}\, P\,w_2(x_P)}_{A_a} - \underbrace{\frac{1}{2}\,\int_0^{\,l} \frac{M^2}{EI}\,dx}_{A_i} = 0\,.
\end{align}
Ist $\textcolor{red}{\delta w(x)}$ eine zul\"{a}ssige virtuelle Verr\"{u}ckung, also $\textcolor{red}{\delta w = 0}$ in den Lagern, dann ergibt sich
\begin{align}
G(w,\textcolor{red}{\delta w}) = \underbrace{\int_{0}^{\,l} p(x)\,\textcolor{red}{\delta w(x)}\,dx + P\, \textcolor{red}{\delta w(x_P)}}_{\delta A_a} - \underbrace{\int_0^{\,l} \frac{M\, \textcolor{red}{\delta M}}{EI}\,dx}_{\delta A_i} = 0\,.
\end{align}
Wenn die virtuelle Verr\"{u}ckung $\delta w(x)$ in den Lagern nicht null ist, (was ja mathematisch  durchaus zul\"{a}ssig ist), dann leisten auch die Lagerkr\"{a}fte  Arbeiten
\begin{align}
G(w,\textcolor{red}{\delta w}) &= \int_{0}^{\,l} p(x)\,\textcolor{red}{\delta w(x)}\,dx  + P \cdot \textcolor{red}{\delta w(x_P)} + A\cdot\textcolor{red}{\delta w(x_a)} + B\cdot\textcolor{red}{\delta w(x_b) }\nn \\ &+ C\cdot\textcolor{red}{\delta w(x_c)}
 - \int_0^{\,d} \frac{M\,\textcolor{red}{ \delta M}}{EI}\,dx = 0\,.
\end{align}
Was an \"{a}u{\ss}erer Arbeit in den Lagern hinzukommt, wird nat\"{u}rlich  dadurch kompensiert, dass sich auch $\textcolor{red}{\delta M} $ \"{a}ndert, aber insgesamt bleibt die Bilanz null.


Auch die Translation $\textcolor{red}{\delta w(x) = 1}$ ist offiziell keine zul\"{a}ssige virtuelle Verr\"{u}ckung, aber das disqualifiziert sie nicht. Im Gegenteil, ihre Anwendung liefert das  Gleichgewicht der vertikalen Kr\"{a}fte
\begin{align}
G(w,\textcolor{red}{1}) &= \int_{0}^{\,l} p(x)\,dx + P \cdot \textcolor{red}{1}  + A\cdot \textcolor{red}{1} + B\cdot \textcolor{red}{1} + C\cdot \textcolor{red}{1}  = 0
\end{align}
und die Drehung $\textcolor{red}{\delta w(x)} = x$, die weder klein noch zul\"{a}ssig ist, kontrolliert das Moment um das linke Lager
\begin{align}
G(w,\textcolor{red}{x}) &=\int_{0}^{\,l} p(x)\,\textcolor{red}{x}\,dx + P \cdot \textcolor{red}{x_P} + B\cdot \textcolor{red}{x_b} + C\cdot \textcolor{red}{x_c}  = 0\,.
\end{align}
Die Crux mit zul\"{a}ssigen und nicht zul\"{a}ssigen virtuellen Verr\"{u}ckungen kann man am einfachsten so umgehen, dass man prinzipiell alle Lager entfernt und durch das Anbringen der Lagerkr\"{a}fte den Balken in der Schwebe h\"{a}lt. Dann sind alle virtuellen Verr\"{u}ckungen, wenn sie nur hinreichend glatt sind,  $\textcolor{red}{\delta w} \in C^2(0,l) $, zul\"{a}ssige virtuelle Verr\"{u}ckungen.\\


Auf die Interpretation von $\vek u^T\,\vek K\,\vek  u$ als innerer virtueller Energie kommt man wie folgt: Die Federsteifigkeit eines Stabes ist $k = EA/l$ und daher ist umgekehrt das $EA $, das zu einer Feder geh\"{o}rt, $EA = k\,l$. Der Einfachheit halber sei nur ein Ende der Feder beweglich $(u)$. Die Normalkraft in der Feder ist
$N\,= k\,u $, und es ist $\textcolor{red}{\delta N\,= k\,\delta u}$, so dass
\begin{align}
\int_0^{\,l} \frac{N\,\textcolor{red}{\delta N}}{EA}\,dx = \int_0^{\,l} \frac{k\,u \cdot k\,\textcolor{red}{\delta u}}{k\,l} \,dx = \textcolor{red}{\delta u}\,k\,u\,.
\end{align}
Die virtuelle innere Energie in dem Tragwerk ist jetzt also
\begin{align}
\delta A_i = \int \frac{M\,\textcolor{red}{\delta M}}{EI}\,dx + \int \frac{N\,\textcolor{red}{\delta N}}{EA}\,dx + \textcolor{red}{\vek \delta u^T}\,\vek K\,\vek u
\end{align}
und sinngem\"{a}{\ss} sind $\textcolor{red}{\delta A_i^*}$ und $A_i $ zu erweitern
\begin{align}
\textcolor{red}{\delta A_i^*} = \ldots + \vek u^T\,\vek K\,\textcolor{red}{\vek\delta \vek  u^*} \qquad A_i = \ldots + \frac{1}{2}\,\vek  u^T\,\vek K\,\vek u\,.
\end{align}
Bei Federn, die sich mit einem Ende auf den Boden abst\"{u}tzen, ist ein Freiheitsgrad gesperrt und es verbleibt nur ein Freiheitsgrad, den wir $u $ nennen wollen.
Die Identit\"{a}t erlaubt verschiedene Schreibweisen
\begin{align}
G(u, \textcolor{red}{\delta u}) = \textcolor{red}{\delta u}\,f - u\,k\,\textcolor{red}{\delta \,u} = \textcolor{red}{\delta u}\,f - \frac{f\,\textcolor{red}{\delta f}}{k} = \textcolor{red}{\delta A_a} - \textcolor{red}{\delta A_i} = 0\,.
\end{align}
In der Formulierung als Prinzip der virtuellen Kr\"{a}fte
\begin{align}
G(\textcolor{red}{\delta u^*}, u) = u\,\textcolor{red}{\delta f^*} - \textcolor{red}{\delta u^*}\,k\,u =  u\,\textcolor{red}{\delta \,f^*} - \frac{\textcolor{red}{\delta \,f^*}\, f}{k} = \textcolor{red}{\delta A_a^*} - \textcolor{red}{\delta A_i^*} = 0
\end{align}
bedeutet das also
\begin{align}
\textcolor{red}{\delta A_i^*} = \frac{\textcolor{red}{\delta \,f^*}\, f}{k}= \textcolor{red}{\delta f^*} \times \frac{f}{k} = \text{Kraft}^*\,\times\,\text{Weg}\,.
\end{align}
Wir stellen uns nun vor, dass wir an einer Stelle des Tragwerks eine Einzelkraft $\textcolor{red}{\bar{P} = \bar{1}}$ aufbringen, um mit dem Prinzip der virtuellen Kr\"{a}fte eine Verformung $\delta $ aus einem Lastfall $p$ zu berechnen. Diese Kraft $\textcolor{red}{\bar{P} = \bar{1}}$ verursacht eine Kraft $\textcolor{red}{\delta f^*}$ in der Feder.

Diese Kraft mal der Zusammendr\"{u}ckung $u = f/k$ (aus dem Lastfall $p$) ist der Beitrag der Feder zur Arbeitsgleichung
\begin{align}
\textcolor{red}{\delta A_a^*} = \textcolor{red}{\bar{1}} \cdot \delta  = \int \frac{\textcolor{red}{\bar{M}}\,M}{EI}\,dx + \int \frac{\textcolor{red}{\bar{N}}\,N}{EA}\,dx + \frac{\textcolor{red}{\delta f^*} \,f}{k}= \textcolor{red}{\delta A_i^*}\,.
\end{align}
Er ist positiv, wenn $\textcolor{red}{\delta f^*} $ und $f $ dasselbe Vorzeichen haben.
\\

%%%%%%%%%%%%%%%%%%%%%%%%%%%%%%%%%%%%%%%%%%%%%%%%%%%%%%%%%%%%%%%%%%%%%%%%%%%%%%%%%%%%%%%%%%%%%%%%%%%
{\textcolor{blau2}{\section{Vereinfachung}}}
Die vielen Unterbrechungen an den Zwischenlagern und der Vorzeichenwechsel zwischen der linken und der rechten Querkraft und ebenso den Biegemomenten links und rechts an den Intervallgrenzen f\"{u}hren schnell dazu, dass man den \"{U}berblick verliert.

Bevor man jetzt anf\"{a}ngt und sich m\"{u}ht herauszufinden, wie denn die Lagerkraft zu den Querkr\"{a}ften steht, ob die Lagerkraft positiv oder negativ ist etc. empfiehlt es sich wie folgt vorzugehen:

Die erste Greensche Identit\"{a}t wird so angeschrieben, als ob alle Lagerkr\"{a}fte dieselbe Richtung h\"{a}tten wie die positiven virtuellen Verr\"{u}ckungen
\begin{align}
G(w, \textcolor{red}{\delta w}) &= M_A \cdot \textcolor{red}{\delta w'(x_1)} + A\cdot \textcolor{red}{\delta w(x_1)} + P\cdot \textcolor{red}{\delta w(x_2)} + M \cdot \textcolor{red}{\delta w'(x_3)}\nn \\
&+ B \cdot \textcolor{red}{\delta w(x_4)} + \int_{x_5}^{\,x_6} p(x)\,\textcolor{red}{\delta w(x)}\,dx + C\cdot\textcolor{red}{\delta w(x_6)} = 0\,.
\end{align}
Die positiven Weggr\"{o}{\ss}en legen also die positiven Richtungen fest. Im zweiten Schritt werden die Lagerkr\"{a}fte eingesetzt und zwar wie sie wirklich wirken, also mit Vorzeichen.
So erh\"{a}lt man automatisch das richtige Ergebnis.

In der Regel wird man zudem nur zul\"{a}ssige virtuelle Verr\"{u}ckungen benutzen, wir haben diese Einschr\"{a}nkung hier
bewusst nicht gemacht, und dann sind die Arbeiten in den Lagern sowieso null.\\


Weil das so ist, bringen wir jedem Student im ersten Semester Statik bei, dass in der linearen Statik Drehungen durch Bewegungen l\"{a}ngs der Tangenten an den Drehkreis ersetzt werden. Hier passt sich gezwungenerma{\ss}en die Statik der Mathematik an und opfert die Anschauung der Mathematik.
\\

Genau so kommen die folgenden Identit\"{a}ten zustande. Sie beruhen einfach nur auf partieller Integration, also identischen Umformungen, und wir gehen daher kein Risiko ein, wenn wir behaupten dass diese f\"{u}r alle Paare von Funktionen null sind.
\\

Nun ist es ja so, dass die Randarbeiten $[V\,\textcolor{red}{\delta w} - M\,\textcolor{red}{\delta w'}]$ alternierende Vorzeichen haben, was die Behandlung von Lagersenkungen theoretisch etwas kompliziert macht. Jedesmal m\"{u}sste man sich neu \"{u}berlegen mit welchem Vorzeichen man den Beitrag versieht und welches Vorzeichen er dann schlie{\ss}lich hat, wenn man ihn auf den rechte Seite bringt.

Man kann das ganze Problem aber umgehen, wenn man sich auf den statischen Gehalt konzentriert. Die Terme in der eckigen Klammer sind \"{a}u{\ss}ere Arbeiten, und damit sind sie im Endergebnis gleich den Arbeiten, die die virtuellen Lagerkr\"{a}fte $\textcolor{red}{\bar{F}}$ bzw. $\textcolor{red}{\bar{M}}$ (Momente) auf den vorgegebenen Lagersenkungen $w_\Delta$ und Lagerverdrehungen $\tan \Np_\Delta$ leisten. Die Bilanz lautet also
\\



Die Begriffe virtuelle Verr\"{u}ckungen und virtuelle Kr\"{a}fte sind f\"{u}r den Anf\"{a}nger eher verwirrend, weil er meint dies w\"{a}ren spezielle Zust\"{a}nde. Insbesondere wenn von den virtuellen Verr\"{u}ckungen oder den virtuellen Kr\"{a}ften ausdr\"{u}cklich verlangt wird, dass sie 'klein' sein m\"{u}ssen.

Virtuell soll ausdr\"{u}cken, dass diese Zust\"{a}nde gedachte Zust\"{a}nde sind, Gedankenexperimente. Ansonsten sind es ganz normale Lastf\"{a}lle, wie der Originallastfall auch. Nur dass sie eben so ausgew\"{a}hlt werden, dass man bei der Formulierung von $G(w, \textcolor{red}{\delta w}) = 0 $ oder $G(\textcolor{red}{\delta w^*},  w) = 0 $ die Informationen erh\"{a}lt, die man sucht.

Die Notation $\textcolor{red}{\delta w^*}$ ist nicht gl\"{u}cklich. Virtuell ist doch gedacht und warum wird das eine ohne Stern und das andere aber mit Stern geschrieben? Wichtig ist doch nur die Reihenfolge in der ersten Greenschen Identit\"{a}t. Was kommt zuerst, erst $w$ und dann $\textcolor{red}{\delta w} $ oder erst $\textcolor{red}{\delta w}$ und dann $w$?
\\

%--------------------------------------------------------------------------------------------------
\subsubsection*{Querkraft}
Die linke Querkraft kann sich nicht bewegen, weil sie auf dem Kragtr\"{a}ger sitzt, so dass die rechte Querkraft den ganzen Weg alleine gehen muss. Dies erreicht man, indem man das Gelenk so spreizt, dass
der Teil hinter dem Gelenk auf der ganzen L\"{a}nge um eine L\"{a}ngeneinheit nach unten gleitet.

\"{U}bertr\"{a}gt man diese Bewegung auf das Original, dann leisten dort die \"{a}u{\ss}eren Kr\"{a}fte die Arbeit $A_{1,2} = - V\cdot \textcolor{blau2}{1} + P \cdot \textcolor{blau2}{1}$ w\"{a}hrend
die Arbeiten an der Kopie null sind, $A_{2,1} = 0$, weil keine Kr\"{a}fte n\"{o}tig sind, um den Teil hinter dem Gelenk abzusenken
\begin{align}
A_{1,2} = - V\cdot \textcolor{blau2}{1} + P \cdot \textcolor{blau2}{1} = A_{2,1} = 0
\end{align}
 oder
\begin{align}
V = P\,.
\end{align}
%--------------------------------------------------------------------------------------------------
\subsubsection*{Moment}
Man baut in die Kopie ein Gelenk ein und kippt den \"{a}u{\ss}eren Schenkel des Kragtr\"{a}ger so nach oben, dass er einen Winkel $\textcolor{blau2}{\tan \Np = 1} $ mit der Horizontalen bildet. Damit wird der Forderung Gen\"{u}ge getan, dass die beiden Momente $M(x)$ insgesamt einen Weg von $\textcolor{blau2}{(-1)}$ zur\"{u}cklegen. Das linke Moment sitzt fest auf dem unverr\"{u}ckbaren Kragtr\"{a}ger und so muss das Moment rechts vom Gelenk den Weg alleine gehen.

Zum Spreizen des Gelenks sind keine Kr\"{a}fte n\"{o}tig und so ist $A_{2,1} = 0$. Insgesamt erh\"{a}lt man somit f\"{u}r den {\em Satz von Betti\/}
\begin{align}
A_{1,2} = - M\,\textcolor{blau2}{\tan\,\Np} - P\,\textcolor{blau2}{\tan\,\Np} \cdot 0.5\,l = A_{2,1} = 0
\end{align}
oder wegen $\textcolor{blau2}{\tan\,\Np = 1}$
\begin{align}
M = - P \cdot 0.5\,l\,.
\end{align}
%--------------------------------------------------------------------------------------------------
\subsubsection*{Normalkraft}
Jetzt wird ein Normalkraftgelenk eingebaut und um eine L\"{a}ngeneinheit gespreizt. Wieder ist es so, dass nur die rechte Normalkraft sich bewegen kann und sie daher den vollen Weg $\textcolor{blau2}{-1}$ gehen muss. Das Spreizen des Normalkraftgelenks am rechten System erfordert keinerlei Kr\"{a}fte und somit ergibt sich der {\em Satz von Betti\/} zu
\begin{align}
A_{1,2} = - N \cdot \textcolor{blau2}{1} + P\cdot \textcolor{blau2}{1} = A_{2,1} = 0
\end{align}
oder
\begin{align}
N = P\,.
\end{align}\\

Was in der linearen Algebra die Einheitsvektoren $\vek e_i$ sind, sind in der Ana\-lysis die Dirac-Deltas. Mittels der Dirac-Deltas kann man die Einflussfunktion f\"{u}r die Durchbiegung $w$ eines Tr\"{a}gers in einem Punkt $x$ als die L\"{o}sung der Differentialgleichung
\begin{align}
EI\,\frac{d^4}{dy^4}\,G_0(y,x) = \delta(y-x)
\end{align}
interpretieren.

Die Formulierung des Satzes von Betti wird dann sehr einfach
\begin{align}
B(G_0,w) = \int_0^{\,l} \delta(y-x)\,w(y)\,dy - \int_0^{\,l} G_0(y,x)\,p(y)\,dy = 0
\end{align}
oder
\begin{align}
w(x) = \int_0^{\,l} G_0(y,x)\,p(y)\,dy \,.
\end{align}
Wir wiederholen hier das, was wir oben schon gesagt haben: Das Operieren mit Dirac-Delta is ein sehr suggestiver Kalk\"{u}l, aber man muss deutlich sehen, dass es ein {\em symbolisches Rechnen\/} ist und bleibt. Mit dem Dirac-Delta kann man sehr einfach die vielen Effekte, die hinter einer Einzelkraft stehen, wie die Zweiteilung der L\"{o}sung, der Sprung in der Querkraft, etc., umgehen, aber auf der anderen Seite darf man nicht vergessen, dass man nur auf dem Weg
\begin{align}
B(G_0,w) = B(G_{@0}^{@l},w)_{(0,x)} + B(G_{@0}^{@l},w)_{(x,l)} = 0 + 0
\end{align}
die Ergebnisse begr\"{u}nden kann. Anders gesagt: Wenn man wei{\ss}, was herauskommt, kann man die Abk\"{u}rzung nehmen, aber vorher muss man wissen, was eigentlich herauskommt...
\\

%%%%%%%%%%%%%%%%%%%%%%%%%%%%%%%%%%%%%%%%%%%%%%%%%%%%%%%%%%%%%%%%%%%%%%%%%%%%%%%%%%%%%%%%%%%%%%%%%%%
{\textcolor{blau2}{\subsection{Archimedes' Hebel}}}
Archimedes wusste, wie so etwas passieren kann und er behauptete auch prompt, dass er die Welt aus den Angeln heben k\"{o}nnte, wenn man ihm nur einen festen Punkt g\"{a}be, siehe Bild \ref{Diss15}\,a.\\

{\em Die Kraft seiner Hand mal der L\"{a}nge des Hebels ist gleich dem Moment der Erde um den festen Punkt.\/}\\

Dort, wo bei Archimedes die Erde sitzt, platzieren wir ein Lager und wir fragen, welche Lagerkraft $R_A$ ist notwendig, um dem Druck von Archimedes Hand zu widerstehen.

Die Einflussfunktionen f\"{u}r die Lagerkraft $R_A$ = die Verschiebungsfigur, die sich einstellt wenn man das linke Lager entfernt, und dort den Balken um eine L\"{a}ngeneinheit nach unten dr\"{u}ckt

In der linearen Statik und auch Mechanik sind Drehungen Pseudodrehungen: Der Abstand $x$ eines Punktes von dem Drehpunkt und die vertikale Verschiebung $y$ bilden ein rechteckiges Dreieck
\beq
\tan \Np = \frac{y}{x}
\eeq
mit $\Np$ als dem Drehwinkel. Das Lager geht den Weg
$y = 1$ und daher muss der Hebel eine $90^0$ Drehung vollf\"{u}hren, wenn der Abstand $h$ zwischen den beiden Punkten infinitesimal klein wird
\beq
\tan \Np = \lim_{h \to 0} \frac{1}{h} = \infty
\eeq
und dies bedeutet, dass eine unendlich gro{\ss}e Kraft
\begin{equation}
R_A = \lim_{h \to 0} \frac{1}{h} \,l\,P
\end{equation}
notwendig ist, um selbst der kleinsten Kraft $P$ am anderen Ende des Tr\"{a}gers das Gleichgewicht zu halten.\\

%%%%%%%%%%%%%%%%%%%%%%%%%%%%%%%%%%%%%%%%%%%%%%%%%%%%%%%%%%%%%%%%%%%%%%%%%%%%%%%%%%%%%%%%%%%%%%%%%%%
{\textcolor{blau2}{\section{Betti extended}}}
Die Resultate in diesem Kapitel beruhen auf einem Satz, den wir {\em Betti extended\/} nennen. Er ist eine Erweiterung des Satzes von Betti auf finite Elemente.

Die L\"{a}ngsverschiebung $u(x)$ eines Stabes in einem Punkt $x$ kann gem\"{a}{\ss} dem {\em Satz von Betti\/} auf zwei Arten dargestellt werden
\begin{align} \label{Eq47}
u(x) = \int_0^{\,l} \delta_0(y-x)\,u(y)\,dy = \int_0^{\,l} G_0(y,x)\,p(y)\,dy\,.
\end{align}
Man kann die Funktion $u$ mit dem Dirac-Delta \"{u}berlagern oder die Belastung $p$ mit der durch die Einzelkraft ausgel\"{o}sten L\"{a}ngsverschiebung $G_0(y,x)$. Beide Formulierungen liefern dasselbe Resultat.

Der {\em Satz von Betti\/} extended besagt nun, dass man in (\ref{Eq47}) die beiden Funktionen $u$ und $G_0(y,x)$ durch ihre Projektionen auf $V_h$ ersetzen darf, also durch die FE-L\"{o}sungen $u_h$ bzw. $G_h(y,x)$
\begin{align} \label{Eq47}
u_h(x) = \int_0^{\,l} \delta(y-x)\,u_h(y)\,dy = \int_0^{\,l} G_h(y,x)\,p(y)\,dy
\end{align}
und dass das Ergebnis dann die Verschiebung $u_h(x)$ der FE-L\"{o}sung in dem Punkt $x$ ist.

Allgemeiner gefasst besagt der {\em Satz von Betti extended\/} das folgende: Ist $J(u)$ ein lineares Funktional, das die Darstellung
\begin{align}
J(u) = \int_0^{\,l} \delta(y-y)\,u(y)\,dy = \int_0^{\,l} G(y,x)\,p(y)\,dy
\end{align}
hat, dann erh\"{a}lt man den Wert $J(u_h)$, also den Wert des Funktionals bezogen auf die FE-L\"{o}sung, mit den beiden \"{a}quivalenten Formulierungen
\begin{align}
J(u_h) = \int_0^{\,l} \delta(y-y)\,u_h(y)\,dy = \int_0^{\,l} G_h(y,x)\,p(y)\,dy\,.
\end{align}
Auf diesem Satz beruht die Technik der Einflussfunktionen f\"{u}r finite Elemente.

So wie die exakte Einflussfunktion $G(y,x)$ durch \"{U}berlagerung mit $p$ den exakten Wert
\begin{align}
J(u) = \int_0^{\,l} G_h(y,x)\,p(y)\,dy
\end{align}
liefert, so liefert die gen\"{a}herte Einflussfunktion $G_h(y,x)$ den Wert $J(u_h)$ f\"{u}r die FE-L\"{o}sung
\begin{align}
J(u_h) = \int_0^{\,l} G_h(y,x)\,p(y)\,dy\,.
\end{align}
Fassen wir das im gr\"{o}{\ss}eren Rahmen. Die Technik der finiten Elemente besteht darin, dass wir die exakte L\"{o}sung auf $V_h$ projizieren,
\begin{align}
a(u-u_h,\Np_i) = 0 \qquad i = 1,2,\ldots n\,.
\end{align}
Parallel dazu projizieren wir aber auch alle Einflussfunktionen auf $V_h$
\begin{align}
a(G-G_h,\Np_i) = 0 \qquad i = 1,2,\ldots n\,.
\end{align}
Und der Zusammenhang zwischen $u$ und den diversen Einflussfunktionen $G(y,x)$
\begin{align}
J(u) = \int_0^{\,l} G(y,x)\,p(y)\,dy
\end{align}

\"{u}bertr\"{a}gt sich automatisch auf die N\"{a}herungen $u_h$ und $G_h(y,x)$, d.h. der Zusammenhang geht nicht verloren, denn
\begin{align}
J(u_h) = \int_0^{\,l} G_h(y,x)\,p(y)\,dy
\end{align}

Wenn man jetzt die Einflussfunktionen trotzdem berechnen will, dann steht man vor einem Problem, denn die zur FE-Einflussfunktion geh\"{o}rigen \"{a}quivalenten Knotenkr\"{a}fte $j_i $ sind alle null
\begin{align}
j_i = J(\vek \Np_i) = 0\,.
\end{align}
Wenn wir jetzt trotzdem versuchen, die Einflussfunktionen mit finiten Elementen zu berechnen, dann sind die \"{a}quivalenten Knotenkr\"{a}fte $f_i$ gleich den Lagerkr\"{a}ften, die zu den Verschiebungsfeldern
$\vek \Np_i$ geh\"{o}ren. Diese werden wie oben berechnet. Es sei $\vek \Np_X$ das Verschiebungsfeld, bei dem sich das Lager um eine L\"{a}ngeneinheit in vertikaler Richtung absenkt.

Die \"{a}quivalenten Knotenkr\"{a}fte $f_i$, die die (gen\"{a}herte) Einflussfunktion generieren, sind dann die Wechselwirkungsenergie zwischen den Feldern $\vek \Np_i$ und dem Feld $\vek \Np_X$
\begin{align}
a(\vek \Np_i,\vek \Np_X) = f_i \cdot 1\,.
\end{align}
Der Einflussfunktion, die die $f_i$ generieren, fehlt wieder, wie oben, das St\"{u}ck $\vek \Np_X$. Wenn man es dazu addiert, hat man die ganze Einflussfunktion.\\



%%%%%%%%%%%%%%%%%%%%%%%%%%%%%%%%%%%%%%%%%%%%%%%%%%%%%%%%%%%%%%%%%%%%%%%%%%%%%%%%%%%%%%%%%%%%%%%%%%%
{\textcolor{blau2}{\section{Einflussfunktionen f\"{u}r \"{a}quivalente Knotenkr\"{a}fte}}}
Eine Belastung wird in einen Knoten reduziert, indem man die Belastung mit der Einheitsverschiebung des Knotens \"{u}berlagert
\begin{align}
f_i = \int_0^{\,l} p(x)\,\Np_i(x)\,dx\,.
\end{align}
Die Einflussfunktionen f\"{u}r die Knotenkr\"{a}fte sind also identisch mit den Einheitsverschiebungen der Knoten. So weit, wie sich eine Einheitsverschiebung erstreckt, soweit reicht der Einfluss eines Knotens. Je kleiner die Elemente werden, um so geringer wird also auch die Knotenkraft $f_i $, die auf einen einzelnen Knoten entf\"{a}llt.

Was wir bei einer FE-Analyse als Knotenkr\"{a}fte $f_i $ bezeichnen, sind eigentlich \"{a}quivalente Knotenkr\"{a}fte, also nicht echte Knotenkr\"{a}fte im physikalischen Sinne, sondern es sind 'Rechenpfennige'. Wenn in einem Knoten eine horizontale  Knotenkraft $f = 10$ kNm wirkt, dann bedeutet es, dass in der N\"{a}he des Knotens Lasten so verteilt sind dass sie bei einer horizontalen Auslenkung des Knotens um eine L\"{a}ngeneinheit die Arbeit $f = 10$ kNm leisten.

Wie genau diese Kr\"{a}fte verteilt sind, wo sie konzentriert sind und wo nicht, das bleibt offen und so gibt es mehrere Anordnungen von Kr\"{a}ften die alle dieselben \"{a}quivalente Knotenkr\"{a}fte aufweisen.

Die \"{a}quivalente Knotenkr\"{a}fte sind also ein K\"{u}rzel f\"{u}r die Arbeiten, die die reale Belastung auf den Einheitsverformungen der Knoten leisten. Ein FE-Programm denkt und rechnet in Arbeiten, $\vek K\,\vek u = \vek f$. Die $u_i $ sind (dimensionslose) Knotenverschiebungen, aber die $f_i $ sind Arbeiten.


\begin{remark}
Der Vektor $\vek r$ ist schwach besetzt, weil ja nur die Ansatzfunktionen der Knoten, die dem Lagerknoten direkt gegen\"{u}berliegen, eine \"{a}quivalente Lagerkraft $r_i = a(\Np_X,\Np_i)$ in dem Lager aufweisen. Die weiter abliegenden $\Np_i$ '\"{u}berlappen' sich nicht mit $\Np_X$.
\end{remark}

Wir k\"{o}nnen nur gen\"{a}herte Einflussfunktionen erzeugen, aber es gilt, dass der Fehler in den Einflussfunktionen orthogonal zur Belastung ist. Anders gesagt die Abweichung zwischen $\sigma_{xx}^h$ und $\sigma_{xx}$ ist in jedem Punkt Null, obwohl der Fehler in der Einflussfunktion, $G(\vek y,\vek x) - G_h(\vek y,\vek x)$, nicht null ist!



Die physikalisch echte Momentenbedingung unter Benutzung des Drehkreises erf\"{u}llt der Balken dagegen nicht!


Im folgenden wollen wir die Berechnung von Einflussfunktionen f\"{u}r Schnitt\-gr\"{o}{\ss}en etwas systematischer fassen.
\begin{alignat}{2}
\textcolor{blau2}{G_1(y,x)} &= \text{Einflussfunktion f\"{u}r $N(x)$} &&= \phantom{-}EA\,u'(x)\nn \\
\textcolor{blau2}{G_2(y,x)} &= \text{Einflussfunktion f\"{u}r $M(x)$} &&= -EI\,w''(x)\nn \\
\textcolor{blau2}{G_3(y,x)} &= \text{Einflussfunktion f\"{u}r $V(x)$} &&= -EI\,w'''(x)\nn
\end{alignat}
Der Index 1, 2 und 3 korrespondiert dabei der Ordnung der Ableitung (letzte Spalte).

Weil die Funktionen dadurch entstehen, dass man ein Gelenk spreizt, muss man den {\em Satz von Betti\/} f\"{u}r den linken und rechten Teil
separat formulieren und dann addieren.

Die Verformung $w(x)$ oder $u(x)$ f\"{u}r die man eine Einflussfunktion sucht, hei{\ss}t auch die {\em prim\"{a}re L\"{o}sung\/} und die Einflussfunktion hei{\ss}t dementsprechend die {\em duale L\"{o}sung\/}.
\\

Zu der Balkendifferentialgleichung $EI\,w^{IV}(x) = p(x) $, die ja eine Differentialgleichung vierter Ordnung ist, geh\"{o}ren vier Dirac-Deltas und zu der Stab-Gleichung $- EA\,u''(x) = p(x)$ und allen anderen Differentialgleichungen zweiter Ordnung geh\"{o}ren zwei Dirac-Deltas, s. Bild \ref{VierBeam14},
\begin{align}
&\delta_0(y-x) \qquad \text{Einzelkraft}\\
&\delta_1(y-x) \qquad \text{Verschiebungssprung in Achsrichtung}
\end{align}
die jeweils den konjugierten Wert liefern
\begin{align}
\int_0^{\,l} \delta_0(y-x)\,u(y)\,dy = u(x) \qquad \int_0^{\,l} \delta_1(y-x)\,u(y)\,dy = N(x)\,.
\end{align}


F\"{u}r das Verst\"{a}ndnis ist es wichtig zu sehen, dass der {\em Satz von Betti\/} immer mit zwei Systemen operiert, dem Originaltr\"{a}ger und einer $1:1$ Kopie. In beide Tr\"{a}ger wird das Gelenk eingebaut, aber
nur die Kopie wird gespreizt. Und zwar so, dass, wenn man diese Bewegung auf den Originaltr\"{a}ger \"{u}bertr\"{a}gt, die Kraftgr\"{o}{\ss}en links und rechts von dem Gelenk in der Summe den Weg $(-1)$ zur\"{u}ck legen, s. Bild \ref{S17}.\\

Die Berechnung von Einflussfunktionen durch den Einbau von Gelenken und deren Spreizung, s. Bild \ref{1GreenF226}, wird in der Literatur manchmal mit dem Namen {\em Mueller-Breslau\/} verbunden, obwohl es ja doch eigentlich der {\em Satz von Betti\/} ist.\\



Beim Rechnen in der Statik kommt {\em Meter\/} vor und nat\"{u}rlich auch {\em Newton\/}, aber {\em rad\/} kommt (normalerweise) gar nicht vor. Der Tangens, ja, der kommt auch st\"{a}ndig vor, aber der hat keine Dimension...

Nat\"{u}rlich kann man jetzt einwenden, dass die Autoren, die dem Tangens $\delta$ die Dimension {\em rad\/} geben, dies in der Absicht tun, das Ergebnis f\"{u}r den Leser in eine ihm gewohnte Dimension zu \"{u}bersetzen
\begin{align}
\delta = \tan\,\Np \simeq \Np\,,
\end{align}
aber wir haben den Eindruck, dass viele Autoren wirklich meinen, dass das $\delta$ ein Winkel sei.\\

Neue Formulierungen, neue Differentialgleichung sollte man also zuerst mit den trigonometrischen Funktionen testen, weil diese ja eine Basis des Funktionenraums \"{u}ber $(0,l)$ sind.
\\

Denn jeder, der davon \"{u}berzeugt ist, dass er mittels mechanischer Prinzipe mathematische Ergebnisse voraussagen kann, muss doch erkl\"{a}ren, wie das gehen soll, wie man mit S\"{a}tzen aus einem Gebiet {\em A\/} (Mechanik) Resultate im Gebiet {\em B\/} (Mathematik) herleiten kann.\\


%%%%%%%%%%%%%%%%%%%%%%%%%%%%%%%%%%%%%%%%%%%%%%%%%%%%%%%%%%%%%%%%%%%%%%%%%%%%%%%%%%%%%%%%%%%%%%%%%%%
{\textcolor{blau2}{\section{Durchlauftr\"{a}ger}}}
Das Thema Einzelkr\"{a}fte leitet \"{u}ber zu Durchlauftr\"{a}gern, bei denen ja in den Lagern Einzelkr\"{a}fte den Tr\"{a}ger st\"{u}tzen.

Wenn die virtuellen Verr\"{u}ckungen zul\"{a}ssig sind, dann reduziert sich das Prinzip der virtuellen Verr\"{u}ckungen auf die einfache Gleichung
\begin{align}
\delta A_a = \int_0^{\,l} p \,\textcolor{red}{\delta w}\,dx = \int_0^{\,l} \frac{M \textcolor{red}{\delta M}}{EI}\,dx = \delta A_i\,.
\end{align}


Der Effekt, auf denen es uns jetzt ankommt, sind die Arbeiten, die die Lagerkr\"{a}fte leisten, wenn man eine beliebige virtuelle Verr\"{u}ckungen $ \textcolor{red}{\delta w}$ ansetzt. Wir nehmen uns ein beliebiges Zwischenlager vor, s. Bild \ref{U19}. Die Querkr\"{a}fte im Anschnitt zum Lager nennen wir $V^-$ und $V^+$, entsprechend den Seiten auf denen sie wirken. Hei{\ss}t die Lagerkraft $B$ dann gilt
\begin{align}
B  = V^+ - V^-\,.
\end{align}
Wenn wir nun das Lager  um das Ma{\ss} $\textcolor{red}{\delta w} $ verr\"{u}cken, dann ergibt sich die virtuelle \"{a}u{\ss}ere Arbeit der Lagerkraft zu
\begin{align}
B \cdot \textcolor{red}{\delta w}\,,
\end{align}
wo also die Lagerkraft in derselben Richtung positiv gez\"{a}hlt wird, wie die virtuelle Verr\"{u}ckung.

Dieses Ergebnis stimmt mit dem Resultat \"{u}berein, das man erh\"{a}lt wenn man feldweise die erste Greensche Identit\"{a}t formuliert und addiert.
\\

Das einfachste Beispiel hierf\"{u}r ist ein gelenkig gelagerter Einfeldtr\"{a}ger
\begin{align}
\text{\normalfont\calligra G\,\,}(w, \textcolor{red}{\delta w}) &= \int_0^{\,l} p(x)\,\textcolor{red}{\delta w(x)}\,dx + [V\,\textcolor{red}{\delta w} - M\,\textcolor{red}{\delta w'}]_{@0}^{@l} - \int_0^{\,l} \frac{M\,\textcolor{red}{\delta M}}{EI}\,dx = 0\,,
\end{align}
denn wegen $M(0) = M(l) = 0 $ und $\textcolor{red}{\delta w(0) = \delta w(l) = 0}$ folgt leicht
\begin{align}
[V\,\textcolor{red}{\delta w} - M\,\textcolor{red}{\delta w'}]_{@0}^{@l} &= V(l)\,\textcolor{red}{\delta w(l)} - M(l)\,\textcolor{red}{\delta w'(l)} - V(0)\,\textcolor{red}{\delta w(0)} + M(0)\,\textcolor{red}{\delta w'(0)}\nn \\
&= V(l) \cdot 0 - 0\cdot \textcolor{red}{\delta w'(l)} - V(0) \cdot 0 + 0 \cdot \textcolor{red}{\delta w'(0)} = 0\,,
\end{align}
so dass sich die erste Greensche Identit\"{a}t in der Tat auf den Ausdruck
\begin{align}
\text{\normalfont\calligra G\,\,}(w, \delta w) &= \int_0^{\,l} p(x)\,\textcolor{red}{\delta w(x)}\,dx  - \int_0^{\,l} \frac{M\,\textcolor{red}{\delta M}}{EI}\,dx = 0
\end{align}
reduziert, in dem keine Randarbeiten mehr vorkommen.



Das einfachste Beispiel hierf\"{u}r ist ein gelenkig gelagerter Einfeldtr\"{a}ger
\begin{align}
\text{\normalfont\calligra G\,\,}(w, \textcolor{red}{\delta w}) &= \int_0^{\,l} p(x)\,\textcolor{red}{\delta w(x)}\,dx + [V\,\textcolor{red}{\delta w} - M\,\textcolor{red}{\delta w'}]_{@0}^{@l} - \int_0^{\,l} \frac{M\,\textcolor{red}{\delta M}}{EI}\,dx = 0\,,
\end{align}
denn wegen $M(0) = M(l) = 0 $ und $\textcolor{red}{\delta w(0) = \delta w(l) = 0}$ folgt leicht
\begin{align}
[V\,\textcolor{red}{\delta w} - M\,\textcolor{red}{\delta w'}]_{@0}^{@l} &= V(l)\,\textcolor{red}{\delta w(l)} - M(l)\,\textcolor{red}{\delta w'(l)} - V(0)\,\textcolor{red}{\delta w(0)} + M(0)\,\textcolor{red}{\delta w'(0)}\nn \\
&= V(l) \cdot 0 - 0\cdot \textcolor{red}{\delta w'(l)} - V(0) \cdot 0 + 0 \cdot \textcolor{red}{\delta w'(0)} = 0\,,
\end{align}
so dass sich die erste Greensche Identit\"{a}t in der Tat auf den Ausdruck
\begin{align}
\text{\normalfont\calligra G\,\,}(w, \delta w) &= \int_0^{\,l} p(x)\,\textcolor{red}{\delta w(x)}\,dx  - \int_0^{\,l} \frac{M\,\textcolor{red}{\delta M}}{EI}\,dx = 0
\end{align}
reduziert, in dem keine Randarbeiten mehr vorkommen.


Die Arbeiten $\textcolor{red}{\bar{F}_{k}}\,w_{\Delta\,k}$ sind positiv, wenn $\textcolor{red}{\bar{F}_k}$ und $w_{\Delta\,k}$ gleichgerichtet sind. Analoges gilt f\"{u}r die Momente und die Verdrehungen.
\\

%%%%%%%%%%%%%%%%%%%%%%%%%%%%%%%%%%%%%%%%%%%%%%%%%%%%%%%%%%%%%%%%%%%%%%%%%%%%%%%%%%%%%%%%%%%%%%%
{\textcolor{blau2}{\section{Die Arbeitsgleichung}}}
Die Mohrsche Arbeitsgleichung hat in ihrer elementarsten Form die Gestalt
\begin{align}
\textcolor{red}{\bar{1}}\cdot\delta = \int_0^{\,l} \frac{M\,\textcolor{red}{\bar{M}}}{EI}\,dx\,.
\end{align}
In den Lehrb\"{u}chern wird zum Beweis der Formel auf das Prinzip der virtuellen Kr\"{a}fte verwiesen.

Nun ist die Formel nat\"{u}rlich ein St\"{u}ck Mathematik, und sie wird nicht nur in der Statik verwendet, sondern auch zum Beispiel in der Meteorologie, in der Thermodynamik und vielen anderen Zweigen der Physik.  Daher wollen wir an dieser Stelle einmal einen mathe\-matischen Beweis f\"{u}r diese Formel f\"{u}hren, allerdings in der Sprache des Ingenieurs.

Zun\"{a}chst erstellt man eine Kopie des Tr\"{a}gers und belastet diese Kopie in Richtung der gesuchten Verformung mit einer Kraft $\textcolor{red}{\bar{P} = \bar{1} }$. Weil die Querkraft in der Balkenmitte springt
\begin{align}
\textcolor{red}{\bar{V}_L - \bar{V}_R = \bar{P}}\,,
\end{align}
muss man die Biegelinie aus zwei Teilen, $\textcolor{red}{\bar{w}_L} $ und $\textcolor{red}{\bar{w}_R} $, zusammensetzen. Diese beiden Teile sind homogene L\"{o}sungen der Balkengleichung (keine Belastung auf der freien Strecke) und in der Mitte des Balkens gehen sie, bis auf den Sprung in der Querkraft, stetig ineinander \"{u}ber, $\textcolor{red}{\bar{w}_L = \bar{w}_R}$, $\textcolor{red}{\bar{w}_L' = \bar{w}_R'}$ und $\textcolor{red}{\bar{M_L} = \bar{M_R}}$. F\"{u}r den ersten Abschnitt ergibt sich, wir integrieren bis $x = l/2$,
\begin{align}
\text{\normalfont\calligra G\,\,}(\textcolor{red}{\bar{w}_L},w)_{(0,x)} &= \textcolor{red}{\bar{V}_L}(x)\,w(x) - \textcolor{red}{\bar{M}_L(x)}\,w'(x) - \int_0^{\,x} \frac{M\,\textcolor{red}{\bar{M}_L}}{EI}\,dx = 0
\end{align}
und analog f\"{u}r den zweiten Abschnitt
\begin{align}
\text{\normalfont\calligra G\,\,}(\textcolor{red}{\bar{w}_R},w)_{(x,l)} &= -\textcolor{red}{\bar{V}_R(x)}\,w(x) + \textcolor{red}{\bar{M}_R(x)}\,w'(x)  - \int_{x}^{\,l} \frac{M\,\textcolor{red}{\bar{M}_L}}{EI}\,dx = 0\,.
\end{align}
In der Mitte ist $\textcolor{red}{\bar{M}_L = \bar{M}_R}$ und $w'$ ist stetig, so dass sich die Arbeitsbeitr\"{a}ge der Momente aufheben, und es verbleibt
\begin{align}
\text{\normalfont\calligra G\,\,}(\textcolor{red}{\bar{w}_L},w)_{(0,x)} + G(\textcolor{red}{\bar{w}_R)},w)_{(x,l)} = \textcolor{red}{\bar{P}}\,w(x) - \int_0^{\,l} \frac{\textcolor{red}{\bar{M}}\,M}{EI}\,dx = 0
\end{align}
oder
\begin{align}
w(x) =  \int_0^{\,l} \frac{\textcolor{red}{\bar{M}}\,M}{EI}\,dx\,,
\end{align}
was die Arbeitsgleichung ist.\\

Nun k\"{o}nnte man argumentieren, gut, jetzt haben wir die Lager weggenommen und dann ist der Begriff zul\"{a}ssig und nicht zul\"{a}ssig hinf\"{a}llig, aber warum m\"{u}ssen 'mit Lager' die virtuellen Verr\"{u}ckungen klein sein und wenn man die Lager wegnimmt, dann ist jede Gr\"{o}{\ss}e, jeder Ausschlag von $\textcolor{red}{\delta w}$ erlaubt?

Darauf k\"{o}nnte man erwidern, dass ja hier mit Starrk\"{o}rperbewegungen $\textcolor{red}{\delta w = a\,x + b}$ gearbeitet wird, und damit die virtuelle innere Energie sowieso null ist. Aber niemand hindert uns daran eine virtuelle Verr\"{u}ckungen $\textcolor{red}{\delta w(x)} $ zu w\"{a}hlen, die keine Starrk\"{o}rperbewegung ist, die gro{\ss} ist und trotzdem stimmt die Bilanz $\textcolor{red}{\delta A_a - \delta A_i = 0}$.
\\
Der eigentliche sch\"{o}pferische Akt dabei ist aber nicht die partielle Integration des Arbeitsintegrals, sondern die Wahl  der virtuellen Verr\"{u}ckung $\delta w $,   der Testfunktion. Wir testen ein A, die Streckenlast, indem wir es gegen ein B, die virtuelle Verr\"{u}ckung, halten. Wie A auf B reagiert, ist Hinweis darauf, wie A beschaffen ist.
\\


Um die Breite eines Schranks zu messen, halten wir einen Zollstock dagegen, das Gewicht eines K\"{o}rpers finden wir, indem wir ihn anheben, etc. Immer sind es zwei Dinge, die miteinander in Wechselwirkung stehen und daher ist die Dualit\"{a}t der Kernbegriff.\\

{\textcolor{blau2}{\section{Das Ziel}}
Wir wollen in diesem Kapitel die Arbeits- und Energieprinzipe der Statik herleiten und zeigen, auf welchen Grundlagen sie beruhen.

%----------------------------------------------------------------------------------------------------------
\begin{figure}[tbp]
\includegraphics[width=1.0\textwidth]{\Fpath/U16}
%\caption{Balken und m\"{o}gliche virtuelle Verr\"{u}ckungen, die auch gross sein d\"{u}rfen! Jeder Faktor ist zul\"{a}ssig, $\delta w = 10^{10}\,\sin(\pi\,x/\ell)$} \label{U16}
\end{figure}%
%----------------------------------------------------------------------------------------------------------

Wir beginnen mit einem einfachen Beispiel, dem Balken in Bild \ref{U16} a. In der Gleichgewichtslage ist gem\"{a}{\ss} dem Energieerhaltungssatz die innere Energie gleich der \"{a}u{\ss}eren Arbeit, die die Streckenlast auf dem eigenen Wege verrichtet
\begin{align}
\frac{1}{2}\,\int_0^{\,l} \frac{M^2}{EI}dx = \frac{1}{2}\,\int_0^{\,l} p\,w\,dx.
\end{align}
Wieso ist das so? Warum sind die beiden Integrale gleich?

Wenn wir dem Balken dann eine zul\"{a}ssige virtuelle Verr\"{u}ckung $\delta w $ erteilen, s. Bild \ref{U16} b, das ist eine Verr\"{u}ckung, die mit den Lagerbedingungen des Tr\"{a}gers vertr\"{a}glich ist,  $\delta w(0) = \delta w(l) = 0$, wie z.Bsp.
\begin{align}
\delta w(x) = \sin \frac{\pi\,x}{l}\,,
\end{align}
dann finden wir, dass die virtuelle \"{a}u{\ss}ere Arbeit genauso gro{\ss} ist, wie die virtuelle innere Energie
\begin{align}
\delta A_a = \int_0^{\,l} p\,\delta w\,dx = \int_0^{\,l} \frac{M\,\delta M}{EI}\,dx = \delta A_i\,.
\end{align}
Und das gilt nicht nur f\"{u}r dieses $\delta w = \sin x\,\pi/l$, sondern f\"{u}r {\em alle\/} nur vorstellbaren zul\"{a}ssigen virtuellen Verr\"{u}ckungen, also etwa alle Funktionen $\delta w$ in Bild \ref{U16} b. Warum sind wir dessen so sicher?

Jede Last l\"{o}st einen Wettbewerb aus, denn auf $V$, das sind alle Funktionen, die die Lagerbedingungen einhalten, geht es, sobald die Belastung auf den Balken gebracht wurde, darum das Minimum der potentiellen Energie
\begin{align}
\Pi(w) = \frac{1}{2}\,\frac{M^2}{EI}\,dx - \int_0^{\,l} p\,w\,dx
\end{align}
zu finden. Der Sieger dieses Wettbewerbes ist die Biegelinie des Balkens.

Und wenn wir nachrechnen, so finden wir, dass die potentielle Energie der Biegelinie {\em  negativ\/} ist
\begin{align}
\Pi(w) = -\frac{1}{2}\, \int_0^{\,l} p\,w\,dx\,,
\end{align}
denn das Integral f\"{u}r sich ist positiv, weil $p$ und $w$ in die gleiche Richtung zeigen.
%----------------------------------------------------------------------------------------------------------
\begin{figure}[tbp]
\centering
\if \bild 2 \sidecaption \fi
\includegraphics[width=.6\textwidth]{\Fpath/SECONDORDER}
\caption{Theorie II. Ordnung}
\label{SecondOrder}%
\end{figure}%
%----------------------------------------------------------------------------------------------------------

Das Minimum der potentielle Energie bedeutet also {\em nicht\/} m\"{o}glichst wenig Aufwand, die potentielle Energie $\Pi(w)$ m\"{o}glichst  dicht an Null r\"{u}cken, sondern das Gegenteil: m\"{o}glichst weit weg von null, den Abstand $|\Pi(w)|$ m\"{o}glichst gro{\ss} machen! Wenn man etwas, was negativ ist, kleiner macht, dann r\"{u}ckt man es weiter weg von Null, man vergr\"{o}{\ss}ert den Betrag $|\Pi(w)|$ der potentiellen Energie.

Das Prinzip vom Minimum der potentiellen Energie ist also eigentlich ein {\em Maximumprinzip\/}, zumindest in Lastf\"{a}llen $g$ oder $p$.

Und wenn wir die Belastung verdoppeln $p \to 2\,p$, und damit auch die Auslenkung des Balkens, $w \to 2\,w$, dann {\em vervierfacht\/} sich die potentielle Energie
\begin{align}
 - \frac{1}{2}\,\int_0^{\,l} 2\,  p\,\,2 \,w\,dx = 4\,\Pi(w)\,.
\end{align}
Zum Abschluss noch eine Bemerkung zur Theorie II. Ordnung.
Wenn man ein Tragwerk nach Theorie II. Ordnung berechnet, dann stellt man das Gleichgewicht am verformten Tragwerk auf, so hei{\ss}t es. Aber durch die Zusammendr\"{u}ckung des Kragtr\"{a}gers in L\"{a}ngsrichtung, s. Bild \ref{SecondOrder}, wird der Hebelarm kleiner, und diese Verk\"{u}rzung findet keine Ber\"{u}cksichtigung bei der Berechnung des Einspannmoments
\begin{align}
M = P \cdot l  + H \cdot w(l) \qquad \text{?}
\end{align}
denn eigentlich m\"{u}sste man f\"{u}r $l$ die leicht verk\"{u}rzte L\"{a}nge
\begin{align}
l' = l - \frac{H\,l}{EA}\,
\end{align}
setzen.

Weil das ein grunds\"{a}tzlicher 'Defekt' der Theorie II. Ordnung ist, kann man---genau genommen---das Gleichgewicht eines Rahmens, der nach Theorie II. Ordnung berechnet wurde, nicht kontrollieren, weil  die Knotenverformungen auf den L\"{a}ngen $l'$ beruhen, die Haltekr\"{a}fte aber mit den 'vollen' L\"{a}ngen $l$ berechnet werden.
\\

Beginnen wir mit einer {\em statisch bestimmt gelagerten\/} Scheibe. Am rechten Rand, in der H\"{o}he $h$, greift eine Einzelkraft $P$ an. Die Einflussfunktion f\"{u}r die vertikale Lagerkraft im rechten Rollenlager ist eine Rotation der ganzen Scheibe um das linke Lager und zwar so, dass sich das rechte Lager um einen Meter nach unten bewegt. Es ist nun evident, dass die Einflussfunktion nicht davon abh\"{a}ngt, wie dick die Backsteine sind. Genauso ist es f\"{u}r die Berechnung der Lagerkraft aus $P$ irrelevant, wieviele Schichten von Backsteinen noch \"{u}ber die H\"{o}he $h$ hinaus folgen.

Eine \"{a}hnliche Situation liegt bei einem 3-Gelenk-Bogen vor. Die Lagerkr\"{a}fte des Bogens h\"{a}ngen nur von der Lage der drei Gelenke zueinander ab, aber nicht davon, welche Form die beiden Bogenh\"{a}lften dazwischen haben.\\

%%%%%%%%%%%%%%%%%%%%%%%%%%%%%%%%%%%%%%%%%%%%%%%%%%%%%%%%%%%%%%%%%%%%%%%%%%%%%%%%%%%%%%%%%%%%%%%%%%%
{\textcolor{blau2}{\section{Einflussfunktionen ohne Einbau von Gelenken}}}
Den Einbau von Gelenken bei Stabtragwerken kann man umgehen, wenn man den Aufpunkt auf ein sehr kurzes Element legt. In dem Element selbst ist die Einflussfunktion nur eine N\"{a}herung, aber in allen anschlie{\ss}enden Elementen ist sie deckungsgleich mit der exakten Einflussfunktion.

Man sieht das sehr sch\"{o}n in den Bildern \ref{1GreenF173} und \ref{U37}, wo \"{a}quivalente Knotenkr\"{a}fte $j_i $ die Einflussfunktionen f\"{u}r das Biegemoment bzw. die Querkraft in dem Element erzeugen.

Bei Fl\"{a}chentragwerken gilt das \"{u}brigens nicht.
\\

, denn die Lagerkr\"{a}fte $f_i$ berechnet ein FE-Programm mit dem Prinzip der virtuellen Verr\"{u}ckungen, d.h. durch \"{U}berlagerung der Spannungen mit den Verzerrungen der virtuellen Verr\"{u}ckungen
\begin{align}
f_i = \int_{\Omega} \sigma_{ij} \,\varepsilon_{ij}\,\,d\Omega
\end{align}
Im letzten Punkt der Lager werden die Spannungen zwar unendlich gro{\ss}, aber das Integral, also die \"{U}berlagerung der Spannungen mit den Verzerrungen aus den Knoteneinheitsverformungen ist endlich, s. Abschnitt \ref{Punktlager}.

Dass das so sein muss, sieht man auch im Lastangriffspunkt. Angenommen wir machen das letzte Element in der oberen rechten Ecke immer kleiner, dann werden die Spannungen in dem Element immer gr\"{o}{\ss}er, aber das Arbeitsintegral
\begin{align}
10\,\text{kN} \cdot 1\,\text{m} = \int_{\Omega} \sigma_{ij}\, \varepsilon_{ij}\,\,d\Omega
\end{align}
beh\"{a}lt seinen Wert bei. Wenn der Integrand beschr\"{a}nkt w\"{a}re und die Gr\"{o}{\ss}e des Elementes ging gegen null, dann w\"{u}rde auch das Ergebnis gegen null gehen. Er muss also gerade so gegen unendlich gehen, dass die schrumpfende Fl\"{a}che des Elements durch ein Anwachsen der Spannungen ausgeglichen wird.

Das ist, vereinfacht gesagt der Grund, warum man Lagerkr\"{a}fte auch in singul\"{a}ren Punkten mit finiten Elementen ausrechnen kann.\\

Im Grunde ist es ein Problem der Mechanik. Das Thema Einzelkr\"{a}fte l\"{a}sst sich bei Scheiben nicht durchhalten, weil echte Einzelkr\"{a}fte das Material einer Scheibe zum Flie{\ss}en bringen w\"{u}rden und somit die Kr\"{a}fte einfach vom Bildschirm verschwinden w\"{u}rden.\\

Welches wissenschaftliche Werk hat sich je mit dieser Frage besch\"{a}ftigt? Alle sprechen von klein und infinitesimal klein, nur wenn man nachfragt, 'Wie klein?', dann kommt keine Antwort.\\


Das linke Integral, $\delta A_a$, und das rechte Integral, $\delta A_i$, sind Zahlen, sie geh\"{o}ren also ganz in das Gebiet der Mathematik. Man kann doch nicht die Gleichheit zweier Zahlen mit einem Prinzip der Mechanik 'beweisen'.

Es sind f\"{u}nf (!) Funktionen, $w, \delta w, p, M, \delta M$, deren Integrale hier bilanziert werden. Und der Ingenieur ist k\"{u}hn genug zu behaupten, dass . Wieso sollen sich
\\

Und die Grenze wird schon gleich am Anfang \"{u}berschritten, denn die Gleichgewichtsbedingungen der klassischen Statik beruhen auf dem Begriff der Starrk\"{o}rperbewegungen, also Translationen und Pseudodrehungen.
Wenn man nur genug Stellen nach dem Komma mitnimmt

Begriffe wie klein und infinitesimal klein geh\"{o}ren in das Gebiet der N\"{a}herungsrechnung, aber die Statik l\"{o}st die Gleichungen exakt! Noch die hundertste Stelle nach dem Komma ist exakt.
\\

Man kann doch nicht Beweise dadurch f\"{u}hren, dass man den einen Term $\delta A_a$ nennt und den anderen $\delta A_i$ und dann aus dem Prinzip der virtuellen Verr\"{u}ckungen schlie{\ss}en, dass die beiden gleich sein m\"{u}ssen. Dann ist auch $0 = 1$, denn die \"{a}u{\ss}ere virtuelle Arbeit ist zuf\"{a}llig gerade 0 und die innere virtuelle Arbeit ist 1 und wegen $\delta A_a = \delta A_i$ ist also 0 = 1.\\

 eben nur auf partieller Integration beruht. Es sind genau drei Schritte, die zu diesem Ergebnis f\"{u}hren.
\begin{enumerate}
  \item Die Biegelinie gen\"{u}gt den Gleichungen
\begin{align}
EI\,w^{IV}(x) = p(x) \qquad w(0) = w(l) = 0 \qquad M(0) = M(l) = 0\,.
\end{align}
  \item Die virtuelle Verr\"{u}ckung ist 'stumm' an den Balkenenden, $\delta w(0) = \delta w(l) = 0$.
  \item Die erste Greensche Identit\"{a}t (partielle Integration) garantiert, dass f\"{u}r Funktionen $w \in C^4(0,l)$  und $\textcolor{red}{\delta w } \in C^2(0,l)$ der folgende Ausdruck null ist
\begin{align}
\text{\normalfont\calligra G\,\,}(w,\textcolor{red}{\delta w }) &= \int_0^{\,l} EI\,w^{IV}(x)\,\textcolor{red}{\delta w }\,dx + [V\,\textcolor{red}{\delta w } - M\,\textcolor{red}{\delta w '}]_{@0}^{@l} - \int_0^{\,l} \frac{M\,\textcolor{red}{\delta M}}{EI}\,dx \nn\\
&=  \int_0^{\,l} EI\,w^{IV}(x)\,\textcolor{red}{\delta w }\,dx - \int_0^{\,l} \frac{M\,\textcolor{red}{\delta M}}{EI}\,dx = 0\,,
\end{align}
\end{enumerate}
Diese drei Schritte erlauben den Schluss $\delta A_a = \delta A_i$ und zwar f\"{u}r beliebig gro{\ss}e virtuelle Verr\"{u}ckungen $\delta w(x)$
\begin{align}
\text{\normalfont\calligra G\,\,}(w,\textcolor{red}{\delta w }) &= \int_0^{\,l} EI\,w^{IV}(x)\,\textcolor{red}{\delta w }\,dx - \int_0^{\,l} \frac{M\,\textcolor{red}{\delta M}}{EI}\,dx = 0\,,
\end{align}

Man kann eigentlich nur Mitleid mit den Studenten haben, die so etwas verstehen m\"{u}ssen.

Die Folge ist eigentlich, dass die Statik durch solche 'Lehrs\"{a}tze' in eine unn\"{o}tige Schieflage kommt. Der Aberglaube, so ist man versucht zu sagen, dominiert die Statik.\\
Alle Biegelinien vom Typ $w = a\,x + b$ sind 'Null-Momenten-Linien' und deswegen m\"{u}ssen die \"{a}u{\ss}eren Kr\"{a}fte orthogonal zu all diesen Starrk\"{o}rperbewegungen sein.
\\

Mit diesen Zitaten wollen wir darauf aufmerksam machen, dass das {\em Rechnen\/} in der Statik mathematischen Gesetzen unterliegt---und nur mathematischen Gesetzen. Castigliano hat bei der Herleitung seines Satzes vom Fachwerk (wo der Satz gilt) auf den elastischen K\"{o}rper (wo er nicht gilt) geschlossen. Er hat vergessen, dass man mathematische Probleme nicht mit den Gesetzen der Mechanik beweisen kann

Mathematik und Mechanik sind also zwei getrennte Gebiete und man kann nicht mit Mitteln der Mechanik einen mathematischen Beweis f\"{u}hren.
, wie sie sich in der ersten Greenschen Identit\"{a}t darstellt
\begin{align}
\text{\normalfont\calligra G\,\,}(w,\textcolor{red}{\hat{w}}) = \int_0^{\,l} EI\,w^{IV}(x)\,\textcolor{red}{\hat{w}}\,dx + [V\,\textcolor{red}{\hat{w}} - M\,\textcolor{red}{\hat{w}'}]_{@0}^{@l} - \int_0^{\,l} \frac{M\,\textcolor{red}{\hat{M}}}{EI}\,dx = 0\,,
\end{align}
Die Starrk\"{o}rperbewegungen der Balkenstatik sind alle Biegelinien $\hat{w} = a\,x + b$ mit 'Null-Momenten', $- EI\,\hat{w}'' = 0$, und die erste Greensche Identit\"{a}t schreibt zwingend vor, dass die \"{a}u{\ss}eren Kr\"{a}fte orthogonal sein m\"{u}ssen, zu all diesen Funktionen $\hat{w} = a\,x + b$. Das ist gerade die Summe der vertikalen Kr\"{a}fte
\begin{align}
\text{\normalfont\calligra G\,\,}(w,\textcolor{red}{1}) = \int_0^{\,l}  EI\,w^{IV}(x)\cdot\textcolor{red}{1}\,dx + V(l) \cdot\textcolor{red}{1} - V(0)\cdot\textcolor{red}{1} = 0
\end{align}
bzw. die Summe der Momente um den linken Anfangspunkt $x = 0$
\begin{align}
\text{\normalfont\calligra G\,\,}(w,\textcolor{red}{x}) = \int_0^{\,l} EI\,w^{IV}(x)\,\textcolor{red}{x}\,dx + V(l) \cdot\textcolor{red}{l} - M(l) \cdot\textcolor{red}{1} + M(0) \cdot\textcolor{red}{1}= 0\,,
\end{align}
Genauso ist ein Fachwerk im Gleichgewicht, $\vek K\,\vek u = \vek f$, (das ist jetzt die nicht-reduzierte Steifigkeitsmatrix, inklusive den Lagerknoten), wenn die Knotenkr\"{a}fte $\vek f$ orthogonal sind zu allen Starrk\"{o}rperbewegungen $\vek u_0$ des Fachwerks und die Starrk\"{o}rperbewegungen sind Translationen und Pseudodrehungen
\begin{align}
\vek u_0 = \vek a + \vek \omega \times \vek x
\end{align}
Mittels partieller Integration zeigt man, dass das Arbeitsintegral
Wer sagt denn, dass $\delta A_i$ die \"{U}berlagerung der Biegemomente ist? Vielleicht gibt es ja noch eine genauere Formulierung, wo dann auch noch die Durchbiegungen \"{u}berlagert werden
\begin{align}
\delta A_i = \int_0^{\,l} (\frac{M\,\delta M}{EI} + w\,\delta\,w)\,dx
\end{align}
Warum nicht so? Weil dann die Glg. X nicht mehr stimmt, die im \"{U}brigen auf partieller Integration beruht.

Autoren sind virtuos in der Handhabung der Energieprinzipe, aber uns scheinen die Beweise oft nur darauf zu beruhen, dass der Autor den einen Term $\delta A_a$ nennt und den anderen $\delta A_i$ und weil das Prinzip der virtuellen Verr\"{u}ckung besagt, dass die virtuellen \"{a}u{\ss}eren Arbeiten und inneren Arbeiten gleich sein m\"{u}ssen, folgt $\delta A_a = \delta A_i$. Beweis gelungen!

Aber dann beweisen wir auch, dass 0 dasselbe ist, wie 1. Denn die \"{a}u{\ss}ere virtuelle Arbeit ist zuf\"{a}llig gerade 0 und die innere virtuelle Arbeit ist 1 und wegen $\delta A_a = \delta A_i$ ist 0 = 1.

Das ist nat\"{u}rlich jetzt spa{\ss}haft gemeint, aber uns kommen viele 'Beweise' in den Statikb\"{u}chern so vor, weil st\"{a}ndig auf die Energieprinzipe Bezug genommen wird. Aus dem und dem Prinzip folgt, dass die beiden Integrale gleich sein m\"{u}ssen, etc.

Nur, Castiglianos Theorem gilt nicht f\"{u}r Scheiben und auch nicht f\"{u}r elastische K\"{o}rper, weil bei diesen Bauteilen die Verformungen aus einer Einzelkraft $P = 1$, das ist die Hilfsgr\"{o}{\ss}e mit der Castigliano operiert, unendlich gro{\ss} sind.


Das Problem ist das folgende...

nicht unter einen Generalverdacht stellen...

Etwas 'wackeliger' wird die Geschichte, wenn wir auf Wikipedia den Satz von Castigliano nachlesen:

{\em Die partielle Ableitung der in einem linear elastischen K\"{o}rper gespeicherten Form\"{a}nderungsenergie nach der \"{a}u{\ss}eren Kraft ergibt die Verschiebung $v_k$ des Kraftangriffspunktes in Richtung dieser Kraft\/}.

Aber die Form\"{a}nderungsenergie eines elastischen K\"{o}rpers, der eine Einzelkraft tr\"{a}gt, ist unendlich gro{\ss}, es macht also keinen Sinn eine Ableitung berechnen zu wollen und auch die Verschiebung $v_k$ des Kraftangriffspunktes ist unendlich gro{\ss}.

Mit diesen Zitaten wollen wir darauf aufmerksam machen, dass das {\em Rechnen\/} in der Statik mathematischen Gesetzen unterliegt---und nur mathematischen Gesetzen. Castigliano hat bei der Herleitung seines Satzes vom Fachwerk (wo der Satz gilt) auf den elastischen K\"{o}rper (wo er nicht gilt) geschlossen. Er hat vergessen, dass man mathematische Probleme nicht mit den Gesetzen der Mechanik beweisen kann. \\

In der ersten Greenschen Identit\"{a}t paart man zwei gleichberechtigte Funktionen, und es h\"{a}ngt nur vom Geschick des Aufstellers ab, die zweite Funktion so zu w\"{a}hlen,
etwa $\textcolor{red}{\delta w(x)}= \sin x$, dass er die Informationen bekommt, die er sucht.

Aber der Sinus ist nat\"{u}rlich genauso real (und nicht 'virtuell') wie die Biegelinie $w$ des Balkens, so dass beide in der Formulierung
\begin{align}
G(w,\textcolor{red}{\sin x}) = 0\,.
\end{align}
gleichberechtigt nebeneinander stehen.

%----------------------------------------------------------------------------------------------------------
\begin{figure}[tbp]
\includegraphics[width=0.7\textwidth]{\Fpath/U58}
%\caption{Stablement und virtuelle Verr\"{u}ckung} \label{U58}
\end{figure}%
%----------------------------------------------------------------------------------------------------------

Aber bei Fachwerken interessieren uns nicht so sehr die Ansatzfunktionen, als vielmehr die einzelnen Stabelemente und ihre Steifigkeitsmatrizen aus denen wir die Gesamtsteifigkeitsmatrix aufbauen.


Das ist so, wie wenn man die Biegelinie $w(x)$ eines Balkens unter einer Gleichlast berechnen soll und als N\"{a}herung eine Sinuswelle w\"{a}hlt, $w_h(x) = \sin (\pi x/l)$. Eingesetzt in die Balkengleichung ergibt sich
\begin{align}
EI\,\frac{d^4}{dx^4} \sin \frac{\pi x}{l} = \frac{\pi^4}{l^4}\,\sin \frac{\pi x}{l} = p_h(x)
\end{align}
als der Lastfall, den man eigentlich gel\"{o}st hat.
\\
Die Kr\"{a}fte $\vek p_h$ und $\vek s_h$ werden von einem FE-Programm nicht ausgegeben, weil sie in der Regel so 'merkw\"{u}rdig' aussehen, s. Bild  \ref{U28}, dass ein Anwender, der mit der Theorie der finiten Elemente nicht vertraut ist, irritiert w\"{a}re.\\

\begin{remark}
Bei einer Scheibe ist die Belastung ein Vektorfeld $\vek p = \{p_x,p_y\}^T$ und auch die Verformungen messen sich nach horizontalen und vertikalen Anteilen, $\vek u = \{u_x, u_y\}^T$. Daher ist die virtuelle \"{a}u{\ss}ere Arbeit ein Skalarprodukt wie
\begin{align}
\delta A_a = \int_{\Omega} \vek p^T\,\vek  \delta \vek u \,d\Omega = \int_{\Omega} (p_x\,\delta u_x + p_y\,\delta u_y)\,d\Omega\,.
\end{align}
\end{remark}


%%%%%%%%%%%%%%%%%%%%%%%%%%%%%%%%%%%%%%%%%%%%%%%%%%%%%%%%%%%%%%%%%%%%%%%%%%%%%%%%%%%%%%%%%%%%%%%%%%%%
%{\textcolor{blau2}{\section{Die Ambivalenz der finiten Elemente}}}

Zuerst war die Methode der finiten Elemente nur ein numerisches Werkzeug, dann ist aber mit ihr ein neuer L\"{o}sungsbegriff in die Statik gekommen ist. Der Begriff der Variationsl\"{o}sung hat zunehmend in der Statik den klassischen L\"{o}sungsbegriff verdr\"{a}ngt und die sogenannten \"{a}quivalente Knotenkr\"{a}fte, die ja eigentlich nur ein bequemes Werkzeug  zur Umsetzung der Numerik sind, haben zunehmend in der Statik ein Eigenleben entwickelt und werden wie reale statische Objekte behandelt.

Die Wandscheibe in Bild \ref{U28} st\"{u}tzt sich auf zwei Punktlager ab, um die Punktlast zu tragen. Vom Standpunkt der Mathematik und der Mechanik aus ist das ein schlecht gestelltes Problem, weil, wenn wir klassisch denken, die Spannungen und Verformungen der Scheibe in dem Lastangriffspunkt bzw. den Lagerpunkten unendlich gro{\ss} werden. Je feiner man das Netz macht, um so gr\"{o}{\ss}er werden die 'Ausrei{\ss}er'.

Der Ingenieur denkt aber anders, er denkt zun\"{a}chst in \"{a}quivalenten Knotenkr\"{a}ften. Er will erst einmal wissen, wie sich die Belastung auf die Lager verteilt.

Weil die Scheibe statisch bestimmt gelagert ist, hat die Tatsache, dass die Spannungen unendlich gro{\ss} werden, keinen Einfluss auf die Lagerkr\"{a}fte.
Finite Elemente f\"{u}hren auf das Gleichungssystem
\begin{align}
\vek K\,\vek  u = \vek f\,.
\end{align}
Streichen wir nun nicht die Zeilen und Spalten, die zu gesperrten Freiheitsgraden $u_i = 0$ geh\"{o}ren, dann muss der Vektor $\vek f$ der \"{a}quivalenten Knotenkr\"{a}fte, der jetzt auch die Lagerkr\"{a}fte umfasst, orthogonal sein zu allen Knotenvektoren $\vek u_0$, die zu Translationen und (Pseudo)Rotationen der Scheibe geh\"{o}ren, also muss gelten
\begin{align}
- f_1 + 10 = - 10 + 10 = 0 \qquad f_2 + f_4 = 10 - 10 = 0\,,
\end{align}
wenn wir die Numerierung in Bild \ref{U28} c zu Grunde legen.

Bei statisch unbestimmten Tragwerken wird im allgemeinen die Verteilung der Belastung auf die Lager von der Feinheit des Netzes abh\"{a}ngen.


Zuerst war die Methode der finiten Elemente nur ein numerisches Werkzeug, dann ist aber mit ihr ein neuer L\"{o}sungsbegriff in die Statik gekommen ist. Der Begriff der Variationsl\"{o}sung hat zunehmend in der Statik den klassischen L\"{o}sungsbegriff verdr\"{a}ngt und die sogenannten \"{a}quivalente Knotenkr\"{a}fte, die ja eigentlich nur ein bequemes Werkzeug  zur Umsetzung der Numerik sind, haben zunehmend in der Statik ein Eigenleben entwickelt und werden wie reale statische Objekte behandelt.


\begin{align}
10\,\text{kNm} &= \int_{\Omega} \vek p_h^T\,\vek \Np_i\,\,d\Omega + \int_{\Gamma} \vek s_h^T\,\vek \Np_i\,ds \nn \\
&= [F/L^2] \cdot [L] \cdot [L^2] + [F/L] \cdot [L] \cdot [L]\,.
\end{align}

Es gibt nat\"{u}rlich noch viele andere Werte, die eventuell interessieren k\"{o}nnen, aber zu ihrer Berechnung muss man dann auf Einflussfunktionen oder andere Techniken zur\"{u}ckgreifen. Nur die kanonischen Werte bekommt man sozusagen auf dem Tablett (der ersten Greenschen Identit\"{a}t) serviert.


%%%%%%%%%%%%%%%%%%%%%%%%%%%%%%%%%%%%%%%%%%%%%%%%%%%%%%%%%%%%%%%%%%%%%%%%%%%%%%%%%%%%%%%%%%%%%%%%%%%
%{\textcolor{blau2}{\section{Der Pfad vom Aufpunkt zum Lastangriffspunkt}}}
So viel Weg, wie von der Spreizung des Aufpunkts im Lastangriffspunkt ankommt, ist gleich dem Einfluss, den die Last auf die Spannung im Aufpunkt hat. Somit stellt sich die Frage, wie kommunizieren die beiden miteinander. Oder anders gefragt: Wie und auf welchen Wegen erreicht die Spreizung den Fusspunkt der Last.

Um den lokalen Anteil
Um die Einflussfunktionen f\"{u}r das Biegemoment in dem Punkt $x$ zu berechnen, unterteilen wir den Balken in zwei Elemente, der L\"{a}nge $\ell_a = x$ bzw. $\ell_b = \ell - x$, die in dem Aufpunkt gelenkig miteinander verbunden sind. Mit den Bezeichnungen des Bildes \ref{U24} ergibt sich dann das Gleichungssystem zu
\begin{align}
\left[ \barr {r @{\hspace{4mm}}r @{\hspace{4mm}}r
}
      k_{11} & k_{12} & k_{13}  \\
      k_{21} & k_{22} & 0  \\
      k_{31} & 0 & k_{33}
    \earr \right] \left[ \barr {r} u_1 \\ u_2 \\ u_3 \earr \right] =  \left[ \barr {r} 0 \\ -M \\ M \earr \right]\,,
\end{align}
wobei
\begin{align}
k_{11} = \frac{12}{l_a^3} + \frac{12}{l_b^3} \quad k_{12} = - \frac{6\,EI}{l_a^2} \quad k_{13} =  - \frac{6\,EI}{l_b^2} \quad k_{22} = \frac{4\,EI}{l_a}\quad k_{33} = \frac{4\,EI}{l_b}\,.
\end{align}
Zuerst setzt man $M = 1$ und skaliert dann anschlie{\ss}end $M$ so, dass sich die gew\"{u}nschte Spreizung $\tan\,\Np_l + \tan\,\Np_r = - u_2 + u_3 = 1$ ergibt.

Man k\"{o}nnte die Spreizung auch mit einer Knotenkraft $f_1 = P$ statt mit den beiden Momenten erzielen, aber dann w\"{a}re $A_{2,1}$ nicht null. Das ist zwar auch kein Problem, aber so ist es einfacher.


Bei der Berechnung eines Durchlauftr\"{a}gers mit dem Drehwinkelverfahren werden erst alle Knoten festgehalten und dann die Knoten einzeln gel\"{o}st und ausgeglichen. Am Ende des Ausgleichs kennt man die Verdrehungen der Knoten, das Tragwerk ist, wie man sagt, geometrisch bestimmt.
Einem solchen Knotenausgleich folgt dann noch ein zweiter Schritt, bei dem die Schnittkr\"{a}fte zwischen den Knoten berechnet werden.
Dabei k\"{o}nnen aber die einzelnen Felder abschnittsweise wie ein fest eingespannter Balken behandelt werden, was die Berechnung sehr vereinfacht.

\"{U}bertragen auf Platten bedeutet das: wenn man wei{\ss}, wie sich die R\"{a}nder einer Platte verdrehen und durchbiegen, dann kann man die Schnittgr\"{o}{\ss}en im Innern der Platte berechnen. Das ist die L\"{o}sungsstrategie der Methode der Randelemente.
\\

Die Regel ist also: In der ersten Greenschen Identit\"{a}t $\text{\normalfont\calligra G\,\,}(w,\textcolor{red}{\delta w}) = \delta A_a - \delta A_i = 0$ werden bei der Formulierung des Teils $\delta A_a$ Kr\"{a}fte mit Wegen gepaart und die Kr\"{a}fte kommen von der ersten Funktion und die Wege von der zweiten, wie wir das in (\ref{Eq61}) und (\ref{Eq62}) angedeutet haben.

Mit den Werten
\begin{align}
EA = 1.0 \cdot 10^6 \,\text{kN},\,EA_c = 2 \cdot EA\qquad  l = 3,\,l_e = 1\, \qquad p = 1000\,\text{kN}/\text{m}
\end{align}
ergibt sich
\begin{align}
u_1^c &= 2.52\cdot 10^{-3}\text{m},\,  u_2^c = 3.25 \cdot 10^{-3}\text{m},\,  u_3^c = 3.74 \cdot 10^{-3}\text{m} \\
 f^+ &= \pm (3.25 - 2.52)\cdot 10^{-3}\,\text{m} \,1.0\cdot 10^6\,\text{kN} = 750\,\text{kNm}
\end{align}
und diese Belastung, s. Bild \ref{U99} c, ergibt an dem urspr\"{u}nglichen Stab dieselbe Verformung.

%-----------------------------------------------------------------
\begin{figure}[tbp]
\centering
\includegraphics[width=0.9\textwidth]{\Fpath/S27}

\label{S27}
%
\end{figure}%
%-----------------------------------------------------------------


Die Gr\"{u}nde, warum Homogenisierungsmethoden erfolgreich sind, sind also:
\begin{itemize}
  \item Die $f_i^+$ sind Gleichgewichtskr\"{a}fte.
  \item Die Fernwirkung der $f_i^+$ tendieren gegen null.
\end{itemize}

Wir argumentieren wie folgt: Steifigkeits\"{a}nderungen k\"{o}nnen durch die Wirkung von Gleichgewichtskr\"{a}ften $f^+$ beschrieben werden. Wenn das betroffene Element nicht zu gro{\ss} ist, dann liegen die Angriffspunkte dieser Kr\"{a}fte relativ dicht beieinander. Die Effekte, die diese Kr\"{a}fte in dem Trag\-werk  bewirken, h\"{a}ngt nun von der Gestalt der Einflussfunktionen ab, die zu dem Effekt geh\"{o}rt, den wir studieren wollen.\\

%%%%%%%%%%%%%%%%%%%%%%%%%%%%%%%%%%%%%%%%%%%%%%%%%%%%%%%%%%%%%%%%%%%%%%%%%%%%%%%%%%%%%%%%%%%%%%%%%%%
%{\textcolor{blau2}{\section{$N$-, $V$- oder $M$-Gelenke}}}
In einem $N$-Gelenk springt die L\"{a}ngsverschiebung $u$, in einem Querkraftgelenk springt die Durchbiegung und in einem Momentengelenk springt die Tangente an die Biegelinie.


Aber wie geht das---$\delta(x)$ ist beliebig klein, also frei w\"{a}hlbar und die $1$ ist konstant? Dann muss doch $\eta(x)$ von $\delta(x)$ abh\"{a}ngen oder \"{a}ndert sich mit dem $\delta(x)$ auch die 1? Dann sollte der Autor doch besser $\Delta\,\Np$ schreiben, statt 1.


Wenn der oben zitierte Autor einen Beweis gef\"{u}hrt h\"{a}tten, dann h\"{a}tte er gemerkt, dass die virtuellen Verr\"{u}ckungen {\em beliebig gro{\ss}\/} sein k\"{o}nnen und dass alle Einflussfunktionen (= kinematische Ketten) Ausschl\"{a}ge aufweisen, die alles andere als klein sind, wie man in jedem Tabellenwerk nachlesen kann.


Nat\"{u}rlich ist die N\"{a}he der Mathematik zur Mechanik f\"{u}r die Mathematik immer sehr fruchtbar gewesen (wie umgekehrt auch), aber die N\"{a}he liefert Ideen, f\"{u}hrt zu Vermutungen, zu neuen mathematischen S\"{a}tzen, die aber mathematisch bewiesen werden m\"{u}ssen.

Aber dieses Vermischen von Mechanik und Mathematik zieht sich durch die ganze Statik. Didaktisch ist das sicherlich sinnvoll, aber auf der anderen Seite darf man ein Resultat wie $0 = 1$ nicht automatisch deswegen f\"{u}r richtig halten, weil $\delta A_a = 0$ ist und $\delta A_i = 1$ und ja das Prinzip der virtuellen Verr\"{u}ckungen in der Statik gilt.\\


Wenn $3\,x = 12$ ist, dann kann man die Gleichung mit {\em beliebig kleinen oder gro{\ss}en \/} Zahlen $\delta u$ multiplizieren
\begin{align}
\delta u \cdot 3\,x = 12 \cdot \delta u\,.
\end{align}
Warum soll das nur f\"{u}r infinitesimal kleine $\delta u$ richtig sein?\\

%%%%%%%%%%%%%%%%%%%%%%%%%%%%%%%%%%%%%%%%%%%%%%%%%%%%%%%%%%%%%%%%%%%%%%%%%%%%%%%%%%%%%%%%%%%%%%%%%%%
%{\textcolor{blau}{\section{Finite Elemente und Projektion}}}\index{Projektion}

Wenn man einen Vektor $\vek v$, der schr\"{a}g in den Raum zeigt, auf die $x-y$-Ebene projiziert, dann entsteht sein Schattenbild $\vek v_h$.
Um von $\vek v_h$ wieder zur Spitze von $\vek v$ zu kommen, m\"{u}ssen wir einen senkrechten Vektor $\vek e$ zu $\vek v_h$ addieren
\begin{align}
\vek v = \vek v_h + \vek e\,.
\end{align}
Bemerkenswert hieran ist, dass der Vektor  $\vek e$ senkrecht auf der  $x-y$-Ebene steht. Das ist gleichbedeutend damit, dass es in der Ebene keine bessere N\"{a}herung f\"{u}r  $\vek v$ gibt, als den Vektor $\vek v_h$. Ebenso gilt, dass alle Vektoren, die \"{u}ber dem Vektor $\vek v$ liegen, in dem Sinne, dass das Lot von ihrer Spitze auf die $x-y$-Ebene in denselben Punkt f\"{a}llt, die Spitze von $\vek v_h$, denselben Schatten haben. Die Sonne, wenn sie denn genau von oben scheint, erzeugt nur einen Schatten f\"{u}r alle diese Vektoren.

ragt, auf die drei Koordinaten ebenen projiziert, dann kann man den Vektor aus seinen 'Schatten' rekonstruieren. Auch wenn der Bauzeichner Risse anfertigt, dann sind das Projektionen auf die verschiedenen Koordinatenebenen.

\"{A}hnlich gehen die finiten Elemente vor. Sie projizieren die exakte L\"{o}sung auf den Ansatzraum $V_h$. Die FE-L\"{o}sung $u_h$ ist der Schatten
der exakten L\"{o}sung. Nur wird der Abstand anders gemessen, als bei Vektoren.

Die FE-L\"{o}sung ist die Funktion $u_h$ in $V_h$
Die Koordinaten eines Vektors $\vek v$ sind die Projektionen des Vektors auf die drei Koordinatenachsen
\begin{align}
x_i = \vek v^T\,\vek e_i\,.
\end{align}
Wenn man den Vektor in die Ebene projiziert,
dan kann man den Vektor aus den drei Richtungen wieder generieren
\begin{align}
\vek  v = x_1 \, \vek e_1 + x_2 \, \vek e_2 + x_3 \, \vek e_3
\end{align}\\

 Das ist kein Widerspruch zu der Tatsache, dass die $\vek p_i$ Gleichgewichtskr\"{a}fte sind, denn durch den Schnitt wird das Gleichgewicht der $\vek p_i$ gest\"{o}rt.

Die Gegenkr\"{a}fte ziehen und dr\"{u}cken an den Elementen direkt am Rand und diese stabilisieren sich \"{u}ber ihre Festhaltung auf dem Rand.
Weil die $\vek p_i$ Gleichgewichtskr\"{a}fte sind, folgt, dass man die Lagerkr\"{a}fte nur braucht, um die Elemente direkt neben dem Rand festzuhalten.

Das ist nicht so \"{u}berraschend, wie es sich vielleicht anh\"{o}rt.

In Gedanken schneide man das innere Netz einer Scheibe heraus.

Symbolisch ausgedr\"{u}ckt hat man
\begin{align}
\int_0^{\,l} p_1\,u_{2}\,dx = \int_0^{\,l} p_2\,u_{1}\,dx \qquad \Rightarrow \qquad \int_0^{\,l} p_1\,\underset{\uparrow}{u_2^h}\,dx = \int_0^{\,l} p_2\,\underset{\uparrow}{u_{1@h}}\,dx
\end{align}


Man beachte, dass jetzt acht Funktionen im Spiel sind, vier Verschiebungen und vier Lasten
\begin{align}
u_1, u_2, u_{1@h}, u_2^h, \qquad  p_1, p_2, p_{1@h}, p_{2@h}\,,
\end{align}
und somit zwei 'normale' Paarungen m\"{o}glich sind
\begin{align}
\int_0^{\,l} u_1 \,p_2 dx = \int_0^{\,l} u_2\,p_1\,dx \qquad \int_0^{\,l} u_{1@h} \,p_{2@h}\,dx = \int_0^{\,l} u_2^h\,p_{1@h}\,dx\,,
\end{align}
aber zus\"{a}tzlich nun auch die neue Paarung
\begin{align}
\int_0^{\,l} p_1\,u_2^h\,dx = \int_0^{\,l} p_2\,u_{1@h}\,dx\,.
\end{align}\\

Die Einflussfunktion f\"{u}r die horizontale Verschiebung eines Knotens ist die Reaktion der Scheibe auf eine horizontal gerichtete Einzelkraft der Gr\"{o}{\ss}e $P = 1$, also ein Dirac-Delta, siehe Bild  \ref{U129} a. Wenn man diesen Lastfall mit finiten Elementen l\"{o}st, und sich den FE-Lastfall anschaut, der zu diesem Lastfall geh\"{o}rt, dann erh\"{a}lt man das Bild \ref{U129} b. Wir nennen diese Kr\"{a}fte $\delta_h(\vek y,\vek x)$.

W\"{a}hrend das Dirac-Delta symbolisch zu nehmen ist, ist das gen\"{a}hrte Dirac-Delta eine Funktion, die man plotten kann, die man auf dem Bildschirm darstellen kann, s. Bild \ref{U129} b.

%%%%%%%%%%%%%%%%%%%%%%%%%%%%%%%%%%%%%%%%%%%%%%%%%%%%%%%%%%%%%%%%%%%%%%%%%%%%%%%%%%%%%%%%%%%%%%%%%%%
%{\textcolor{blau2}{\section{Matrizenstatik}}}\index{Matrizenstatik}\label{Matrizenstatik}
Zur Vorbereitung auf die rechnerische Umsetzung der Berechnung von Einflussfunktionen mit finiten Elementen wollen wir kurz die Grundlagen der Matrizenstatik rekapitulieren.

Das elementarste Beispiel f\"{u}r Matrizenstatik ist die Berechnung eines Fachwerkes mit finiten Elementen, die auf das Gleichungssystem
f\"{u}hrt.

Die Einheitsverformungen $\Np_i(x)$ eines Fachwerkes haben am Ort von $u_i$ und in Richtung des Freiheitsgrades $u_i$ den Wert Eins, $u_i = 1$, und den Wert Null an allen anderen Stellen,  $u_j = 0$, und somit ist die Verformungsfigur
\begin{align}
u(x) = \sum_i u_i\,\Np_i(x)
\end{align}
die Summe der---mit den $u_i$ gewichteten---Einheitsverformungen $\Np_i(x)$.

Die Matrix $\vek K $ in (\ref{Eq67}) ist die sogenannte reduzierte Steifigkeitsmatrix, wenn man in der nicht reduzierten Matrix $ \vek K_S$ die Zeilen und Spalten streicht, die zu gesperrten Freiheitsgraden geh\"{o}ren.

Die nicht reduzierte Matrix $\vek K_S $, die Systemmatrix, ist singul\"{a}r, weil kein Lager da ist, um das Fachwerk festzuhalten. Das kommt erst beim \"{U}bergang von $\vek K_S $ zu $\vek K$. Die Vektoren $\vek u_0$, die im Kern der Matrix $ \vek K_S $ liegen, sind gerade die Starrk\"{o}rperbewegungen des ungelagerten Fachwerks. Sie haben also die Gestalt
\begin{align}
\vek u_0 = \vek a + \vek b \times \vek x\,.
\end{align}
Wobei der Vektor $\vek a $ eine Translation beschreibt und der Vektor $\vek b $ die Drehachse (samt Drehwinkel, $\Np = |\vek b|$) darstellt, um den das Fachwerk gedreht wird.


Im Kern liegen bedeutet, dass die Vektoren $\vek u_0 $ auf den Nullvektor $\vek 0$ abgebildet werden, $\vek K_S\,\vek u_0 = \vek 0$. Aus der Identit\"{a}t
\begin{align}
B(\vek u,\vek u_0) = \vek u_0^T\,\vek K\,\vek u - \vek u^T\,\vek K\,\vek u_0 = 0
\end{align}
folgt, dass der Vektor $\vek f$ der Knotenkr\"{a}fte zu den Vektoren $\vek u_0 $ orthogonal ist
\begin{align}
\vek u_0^T \,\vek f = 0\,.
\end{align}
Damit ist garantiert, dass die Vektoren $\vek f $ den Gleichgewichtsbedingungen gen\"{u}gen, denn w\"{a}hlt man als Vektor $\vek u_0$ eine Translation in horizontaler oder vertikaler Richtung, dann ist das das Gleichgewicht der Knotenkr\"{a}fte in horizontaler bzw. vertikaler Richtung
\begin{align}
\sum H = 0 \qquad \sum V = 0
\end{align}
 und w\"{a}hlt man als Vektor $\vek u_0 $ eine Pseudorotation, dann entspricht dies der Momentensumme
\begin{align}
\sum M = 0
\end{align}
um den Nullpunkt des Koordinatensystems.

Wegen $\vek K\,\vek u_0 = \vek 0$ muss im \"{U}brigen jede Zeile orthogonal zu den Vektoren $\vek u_0$ sein, also 'im Gleichgewicht' sein.\\


\hspace*{-12pt}\colorbox{hellgrau}{\parbox{0.98\textwidth}{Die Zeilen (= Spalten) einer nicht reduzierten Steifigkeitsmatrix sind orthogonal zu den Vektoren $\vek u_0 = \vek  a + \vek b \times \vek x$}}\\

Insbesondere, wenn es nur einen Typ von Starrk\"{o}rperbewegung gibt, wie bei einem Stab, $\vek u_0 = \vek a$ (eine Verschiebung nach links oder rechts), ist in jeder Zeile die Summe der Eintr\"{a}ge null.\\

Nach der $h$-Vertauschungsregel gilt jedoch
\begin{align}
S_h  = \int_{\Omega} G_h(\vek y,\vek x)\,p(\vek y)\,d\Omega_{\vek y} =  \int_{\Omega} G(\vek y,\vek x)\,p_h(\vek y)\,d\Omega_{\vek y}\,,
\end{align}
was besagt, dass $S_h$ mit der Knotenkraft $f_i$ im Ausdruck identisch ist.

Bevor wir dies diskutieren, kehren wir noch einmal zu dem Fachwerk zur\"{u}ck.\\

%\subsubsection*{Einflussfunktion f\"{u}r eine Lagerkraft---statisch}
Die Einflussfunktion f\"{u}r eine Lagerkraft ist die Verformungsfigur des Fachwerks, wenn der Lagerknoten um 1 Meter (das ist rein rechnerisch) ausgelenkt wird, s. Bild \ref{U85}.
%-----------------------------------------------------------------
\begin{figure}[tbp]
\centering
\includegraphics[width=1.0\textwidth]{\Fpath/U87}
%\caption{Durchlauftr\"{a}ger, \textbf{a)} FE-Modell, \textbf{b)} Einflussfunktion f\"{u}r die Lagerkraft, \textbf{c)} Ansatzfunktionen und Einheitsverformung des Lagerknotens} \label{U87}
%
\end{figure}%
%-----------------------------------------------------------------
Der gesperrte Lagerknoten entspreche dem Freiheitsgrad $u_k$. Der Ingenieur behandelt das Problem wie folgt: Er bringt die zu $u_k$ geh\"{o}rige Spalte $\vek s_k$ der Steifigkeitsmatrix $\vek K_{+1}$ auf die rechte Seite und l\"{o}st das System
\begin{align}
\vek K\,\vek g = - \vek s_k\,.
\end{align}
Die Matrix $\vek K$ ist die reduzierte Steifigkeitsmatrix und $\vek K_{+1}$ ist der Vorg\"{a}nger, bei dem die zu $u_k$ geh\"{o}rige Spalte noch nicht gestrichen wurde.

Die Verformungsfigur
\begin{align}
g(x,y) = \sum_i g_{i}\,\Np_i(y)
\end{align}
ist dann die gesuchte Einflussfunktion und
\begin{align}
R_k = \vek g^T\,\vek f
\end{align}
ist die zu einem Lastfall $\vek f$ geh\"{o}rige Lagerkraft $R_k$ in dem festgehaltenen Knoten in Richtung des Freiheitsgrades $u_k$.

\begin{remark}
Die Eintr\"{a}ge in Spalte $k$ der nicht reduzierten Steifigkeitsmatrix ($k$ = gesperrter Freiheitsgrad des Lagers) sind die Integrale, s. Bild \ref{U87} c, der Einheitsverformungen
\begin{align}
k_{ik} = \int_0^{\,l} EI\,\Np_i''\,\Np_k''\,dx = \delta A_i(\Np_i,\Np_k) = \delta A_a(\Np_i,\Np_k) = \text{Lagerkraft aus $\Np_i$}
\end{align}
\end{remark}

%\subsubsection*{Einflussfunktion f\"{u}r eine Lagerkraft---mathematisch}
Versuchen wir dasselbe Ergebnis mathematisch herzuleiten: Eine Lagerkraft $R_k$ (in Richtung des gesperrten Freiheitsgrades $u_k = 0$) ist ein Funktional
\begin{align}
R_k = \mathcal{J}(\vek u)\,,
\end{align}
und die Einflussfunktion f\"{u}r $R_k$, s. Bild \ref{U87}, erh\"{a}lt man, wenn man als Knotenkr\"{a}fte $j_{i}$ die Lagerkr\"{a}fte der Einheitsverformungen wirken l\"{a}sst
\begin{align}
j_{i} = R_k(\Np_i)\,.
\end{align}
Der Vektor der Knotenverschiebungen $\vek g$ (das sind die $u_i$ der Einflussfunktion) ist dann die L\"{o}sung des Systems
\begin{align}
\vek K\,\vek g = \vek j\,.
\end{align}
Wenn also in einem Lastfall $\vek f$ das Fachwerk die Knotenverformungen $\vek u$ aufweist, dann ist
\begin{align}
R_k = \sum_i\,u_i\,j_{i} = \sum_i\,u_i\,R_k(\Np_i) = \vek u^T\,\vek j = \vek j^T\,\vek K^{-1}\,\vek f = \vek g^T\,\vek f
\end{align}
die Lagerkraft $R_k$ in diesem Lastfall.

Die Berechnung der Zahlen $j_i = R_k(\Np_i)$ kann man sich wie folgt zurechtlegen. Es sei $\vek K_{+1}$ die oben erw\"{a}hnte Steifigkeitsmatrix des Fachwerks.

Die Kr\"{a}fte, die n\"{o}tig sind, um die Verformung $u_k = 1$ und $u_i = 0$ sonst, zu erzeugen, sind identisch mit der Spalte $\vek s_k \times (-1)$ der Matrix $\vek K_{+1}$. (Minus, weil wir den Vektor auf die rechte Seite bringen). Wegen des Satzes von Betti sind die Eintr\"{a}ge $s_{ki}$ (Zeile $i$) in dem Vektor $\vek s_k$ gleich die Lagerkr\"{a}fte $R_k(\Np_i)$ und somit ist der Vektor $\vek g_k$ die L\"{o}sung des Systems
\begin{align}
\vek K\,\vek g = - \vek s_k\,,
\end{align}
was dasselbe Ergebnis wie zuvor ist.\\

Platten k\"{o}nnen schubstarr (Kirchhoffplatte) oder schubweich (Reissner-Mindlin) gerechnet werden. Die Kirchhoffplatte ist die Erweiterung des klassischen Biegebalkens (Euler-Bernoulli) in die $y$-Richtung. Die Unterschiede in den Ergebnissen sind relativ gering. Wir wollen uns daher im Folgenden mit der Kirchhoffplatte besch\"{a}ftigen.


%-----------------------------------------------------------------
\begin{figure}[tbp]
\centering
\includegraphics[width=0.8\textwidth]{\Fpath/U85}
%\caption{Einflussfunktion f\"{u}r eine Lagerkraft in einem Fachwerk} \label{U85}
%
\end{figure}%
%-----------------------------------------------------------------

Denn bei der $\sum H$, $\sum V$ und der $\sum M$ geht e

Das merkw\"{u}rdige ist aber, dass, weil die Scheibe statisch bestimmt gelagert ist, die Tatsache, dass die Spannungen unendlich gro{\ss} werden, keinen Einfluss auf die Lagerkr\"{a}fte hat.

Bei statisch unbestimmten Tragwerken wird im allgemeinen die Verteilung der Belastung auf die Lager von der Feinheit des Netzes abh\"{a}ngen.
Dadurch, dass man eine Scheibe in finite Elemente unterteilt, verl\"{a}sst man praktisch den Boden der Elastizit\"{a}tstheorie und rechnet mit einem
'Metamodell', das sich der Ingenieur, \"{a}hnlich wie ein Fachwerk, aus Scheibenelementen zusammengesetzt denkt. Und pl\"{o}tzlich kann man auch bei Scheiben, der Elastizit\"{a}tstheorie zum Trotz, mit Einzelkr\"{a}ften (= Knotenkr\"{a}ften) rechnen.

Das Vehikel f\"{u}r den \"{U}bergang zwischen dem Metamodell und dem Originalmodell ist der Begriff der Arbeit. Eine Knotenkraft $f_i = 10 $ kNm in einem festgehaltenen Knoten bedeutet aber trotzdem nicht, dass dort eine Einzelkraft von $10$ kN angreift, sondern vielmehr, dass die Wolke von Fl\"{a}chen- und Kantenkr\"{a}ften, die die Scheibe dort st\"{u}tzt, bei einer Auslenkung des Knotens um eine L\"{a}ngeneinheit die Arbeit $1 \cdot 10$ kNm leistet.



Der Unterschied zwischen der Elastizit\"{a}tstheorie und dem Metamodell ist, dass man keine Einflussfunktionen f\"{u}r nicht existierende Lagerkr\"{a}fte in Punktlagern berechnen kann, aber sehr wohl Einflussfunktionen f\"{u}r die \"{a}quivalenten Knotenkr\"{a}fte $f_i$ in solchen Lagern.
\\

%%%%%%%%%%%%%%%%%%%%%%%%%%%%%%%%%%%%%%%%%%%%%%%%%%%%%%%%%%%%%%%%%%%%%%%%%%%%%%%%%%%%%%%%%%%%%%%%%%%
%{\textcolor{blau2}{\section{Punktlager bei Scheiben}}}\index{Punktlager bei Scheiben}\label{FE-LagerScheibe}
Punktlager---mathematisch unendlich feine, unendlich d\"{u}nne Nadeln---k\"{o}nnen, wenn man der Elastizit\"{a}tstheorie folgt, eine Scheibe nicht festhalten.   Die Lagerkraft ist in allen Lastf\"{a}llen null.
Andererseits erh\"{a}lt man aber mit finiten Elementen doch sinnvolle Ergebnisse in Punktlagern. Wie geht das?

\begin{remark}
Es ist eine der Merkw\"{u}rdigkeiten dieser  'technischen' Scheibentheorie, wenn wir die Mischung aus Elastizit\"{a}tstheorie und finiten Elementen einmal so bezeichnen wollen, dass auf der Au{\ss}enseite der Scheibe, also in den Lagern und l\"{a}ngs den R\"{a}ndern, alles den Ingenieurvorstellungen entspricht, aber sobald man auf die andere Seite des Punktlagers wechselt, die Spannungen tendenziell unendlich gro{\ss} werden.
\end{remark}



%%%%%%%%%%%%%%%%%%%%%%%%%%%%%%%%%%%%%%%%%%%%%%%%%%%%%%%%%%%%%%%%%%%%%%%%%%%%%%%%%%%%%%%%%%%%%%%%%%%
%{\textcolor{blau2}{\section{Linienlager bei Scheiben}}}\label{Linienlager bei Scheiben}
Wir hatten oben \"{u}ber Einflussfunktionen f\"{u}r die $f_i$ in den Lagerknoten gesprochen.
Festgehaltenen Knoten in Reihe stellen ein Linienlager dar und die Knotenkr\"{a}fte $f_i$ sind die Lagerkr\"{a}fte, die oft von den Programmen in verteilte Kr\"{a}fte (Linienkr\"{a}fte) umgerechnet werden und so ausgegeben werden.

Wie berechnet ein FE-Programm die Knotenkr\"{a}fte $f_i$ in den Lagerknoten? \\
\begin{itemize}
  \item Es erweitert den Vektor $\vek u$ zun\"{a}chst um die zuvor gestrichenen $u_i = 0$ in den Lagerknoten, $\vek u \to \vek u_{G}$,
  \item und multipliziert die nicht-reduzierte, globale Steifigkeitsmatrix $\vek K_{G}$ mit dem vollen Vektor $\vek u_{G}$,
  \item die Eintr\"{a}ge $f_i$ in dem Vektor $\vek f_{G} = \vek K_{G}\,\vek u_{G}$, die zu den gesperrten Freiheitsgraden geh\"{o}ren, sind die Knotenkr\"{a}fte in den Lagern.
\end{itemize}

Das erkl\"{a}rt, was programmtechnisch geschieht, aber nicht, wieso eine Knotenkraft $f_i$ gerade den Wert $f_i = 123.45$ kNm hat. Das Resultat beruht auf den Einflussfunktionen.

Betrachten wir ein Rollenlager wie in Bild \ref{U141} mit einer Spannungsverteilung $\sigma_{yy}^h$ in der Lagerfuge. Das Mittel betrage $\bar{\sigma}_{yy}^h$. Die Einheitsverformungen der vier Lagerknoten in vertikaler Richtung seien die vier Dachfunktionen $\Np_i(x), \,i = 1,3, 5, 7$. Die vier Knotenkr\"{a}fte
\begin{align}
f_i = \int_0^{\,l} \sigma_{yy}^h(x)\,\Np_i(x)\,dx \qquad i = 1,3,5,7
\end{align}
sind die \"{U}berlagerung von $\sigma_{yy}^h$ mit den vier Wegen $\Np_i(x)$.

Jeden einzelnen Wert $\sigma_{yy}^h(x)$ hat das FE-Programm mit der zugeh\"{o}rigen Einflussfunktion berechnet, also einer Spreizung des Punktes in vertikaler Richtung. Das ist nun sicherlich ein diffiziles Problem, das aber sehr viel einfacher wird, wenn man mit dem Mittelwert $\bar{\sigma}_{yy}$ der Spannungen rechnet, weil die zugeh\"{o}rige Einflussfunktion einfach dadurch entsteht, dass man das Lager durchschneidet und dann den Teil oberhalb um einen Meter (nur rechnerisch) {\em als Ganzes\/} nach oben dr\"{u}ckt. Diese Bewegung l\"{a}sst sich viel besser ann\"{a}hern, als eine Serie von einzelnen Punktversetzungen und deswegen d\"{u}rfte die Resultierende $R_h$ der Knotenkr\"{a}fte $f_i$ relativ genau sein
\begin{align}
R_h = f_1 + f_3 + f_5 + f_7 = \bar{\sigma}_{yy}^h \cdot \int_0^{\,l} (\Np_1 + \Np_3 + \Np_5 + \Np_7)\,dx\,.
\end{align}
Aber schon die einzelne Knotenkraft $f_i$ profitiert von diesem Effekt, weil jedes $f_i$ ja selbst schon ein gewichtetes Mittel ist
\begin{align}
f_i = \int_0^{\,l} \sigma_{yy}^{h}(x)\,\Np_i(x)\,dx \simeq \bar{\sigma}_{yy}^{(i)}\cdot\int_0^{\,l} \Np_i(x)\,dx\,,
\end{align}
also in etwa dem Mittelwert $\sigma_{yy}^{(i)} = const.$ von $\sigma_{yy}^{h}(x)$ im Bereich von $\Np_i$ entspricht und dieser Werte gewichtet mit dem Integral von $\Np_i(x)$.
\\

Wir k\"{o}nnten dieses Ergebnis jetzt verallgemeinern und statuieren: Nur die Elemente direkt neben dem Rand erzeugen die Lagerkr\"{a}fte.
Das ist aber eigentlich evident, wie man sieht, wenn man den inneren Teil einer Scheibe herausschneidet und nur die Elemente am Rand stehen l\"{a}sst. An den Schnittkanten muss man Haltekr\"{a}fte anbringen, um das Gleichgewicht wieder herzustellen. Und diese Haltekr\"{a}fte sind gerade so gro{\ss}, wie die Belastung, die auf den inneren Teil wirkt. Wenn der Schnittkreis so gro{\ss} w\"{a}re, dass die ganze Scheibe hineinpassen w\"{u}rde, dann w\"{a}re in der Tat (\ref{Eq70}) ma{\ss}gebend.\\

\begin{align}
f_i^h &= \int_{x_a}^{\,x_b} p_h \,\Np_i\,dx \qquad (x_a,x_b) = \text{Tr\"{a}ger von $\Np_i$}\nn \\
&= \underbrace{\int_{x_a}^{\,x_b} EI\,w_h^{IV}\,\Np_i\,dx + [V_h \,\Np_i - M_h\,\Np_i']_{x_a}^{x_b} }_{\delta A_a} = \underbrace{\int_{x_a}^{\,x_b} \frac{M_h\,M_i}{EI}\,dx}_{\delta A_i}\nn \\
&= \int_{x_a}^{\,x_b} \sum_j\,EI\,\Np_j''\,\Np_i''\,dx \,u_j= \sum_j\,a(\Np_j,\Np_i)\,u_j = \sum_j\,k_{ij}\,u_j\,.
\end{align}
Das Arbeitsintegral $(p_h,\Np_i)$ in der ersten Zeile ist dabei symbolisch zu nehmen. Es ist eine Kurzform f\"{u}r das $\delta A_a$ in der zweiten Zeile. Wegen $\delta A_a = \delta A_i$ kann man $f_i^h$ durch die innere Arbeit $\delta A_i$ ausdr\"{u}cken kann. So kommt die Steifigkeitsmatrix in die Gleichung hinein, $\vek K\,\vek u = \vek f_h$, was $\vek f_h = \vek f$ bedeutet.

Der Tr\"{a}ger der Funktion $\Np_i(x)$ ist der Teil der $x$-Achse, in dem $\Np_i$ nicht konstant null ist, wo also $\Np_i(x)$ 'lebt'.
\\

Stimmt dieser Verlauf im Fall der Platte mit der Kurve \"{u}berein, die man erhalten w\"{u}rde, wenn man 'von Innen k\"{a}me', also den Kirchhoffschub im Abstand von 10 cm vom Rand ausrechnet und dann die letzten 10 cm durch Extrapolation \"{u}berbr\"{u}ckt? Nein, in der Regel sind die beiden Kurven nicht gleich. Die von Innen extrapolierten Lagerkr\"{a}fte d\"{u}rften auch relativ schlecht sein, weil die rechnerischen Querkr\"{a}fte sehr schwankend sind und eigentlich nur in der Elementmitte halbwegs passabel sind.
\\

\subsubsection*{Was geht und was nicht geht}

Eine Scheibe so zu st\"{u}tzen, dass die Verschiebungen $u_i$ in einem Punkt null sind---das geht unter zuhilfenahme von Fl\"{a}chen- und Linienkr\"{a}ften.
Das Arbeits\"{a}quivalent dieser Kr\"{a}fte ist die \"{a}quivalente Knotenkraft $f_i$ im Ausdruck
\begin{align}
f_i = \int_{\Omega} \vek p_h \dotprod \vek \Np_i\,d\Omega + \int_{\Gamma} \vek  s_h\dotprod \vek \Np_i\,ds\,.
\end{align}
(Die $\vek s_h$ sind die Linienkr\"{a}fte auf den Elementkanten $\Gamma$ in der N\"{a}he des Lagerknotens).

Was aber nicht geht, ist null Lagerverschiebungen + punktf\"{o}rmige Lagerkraft. Das kann die Elastizit\"{a}tstheorie nicht und ein FE-Programm kann es noch weniger, weil eine echte Punktkraft das Material zum Flie{\ss}en bringen w\"{u}rde.\\

Bei statisch unbestimmten Tragwerken werden die Winkel $\Np_L$ und $\Np_R$ nat\"{u}rlich direkt am Gelenk gemessen. Es sind dann die Stabdrehwinkel der Endtangenten.

Bei diesen Abmessungen gilt im \"{U}brigen
\begin{align}
\Np_L = 36.86^0 \qquad \Np_R = 14.0^0\,,
\end{align}
und der Winkel zwischen den beiden Schenkeln betr\"{a}gt somit $50.86^0$ und der Tangens dieses Winkels ist 1.23.

Damit aus $\delta\,\Np = 1$ etwas vern\"{u}nftiges wird, muss man es so interpretieren:  $\delta\,\Np = 1$ steht f\"{u}r $\tan\,\Np_L + \tan\,\Np_R = 1$. Anders kommt man \"{u}ber die H\"{u}rde nicht hinweg, denn es gibt kein Additionstheorem f\"{u}r den Tangens der Art, dass
\begin{align}
\tan(\Np_L + \Np_R) \overset{?}{=} \tan\,\Np_L + \tan\,\Np_R\,.
\end{align}


In diesem Kapitel besch\"{a}ftigen wir uns mit Einflussfunktionen, ihrer Herleitung und ihrer Anwendung in der Statik.

Mathematisch beruhen Einflussfunktionen auf dem {\em Satz von Betti\/}, also der Tatsache, dass in der linearen Statik die Differentialgleichungen selbstadjungiert\index{selbstadjungiert} sind.
\\

Das Kernst\"{u}ck der Statik der Kontinua\index{Statik der Kontinua} sind die Differentialgleichungen, die den Zusammenhang zwischen der Belastung und den Verformungen beschreiben. Was der einzelnen Differentialgleichung eigen ist, was jeweils speziell an ihr ist, kommt bei der Formulierung der ersten Greenschen Identit\"{a}ten ans Licht. Auf diesen Identit\"{a}ten beruhen im Grunde alle Arbeits- und Energieprinzipe der Statik der Kontinua.



Uns kommt es hier haupts\"{a}chlich darauf an, zu sehen, dass bei solchen Formulierungen die positive Richtung der Einzelkr\"{a}fte und Lagerkr\"{a}fte mit der positiven Richtung der Weggr\"{o}{\ss}en \"{u}bereinstimmt. Das ist jetzt nicht einfach gesetzt, sondern das ergibt sich automatisch, wenn man die erste Greensche Identit\"{a}t abschnittsweise formuliert und dann addiert, $0 + 0 + 0 + 0 + 0 = 0$.

Betrachten wir das Lager $B$, s. Bild \ref{U19}, und die zugeh\"{o}rige Lagerkraft
\begin{align}
B  = V^- - V^+\,.
\end{align}
Wenn wir das Lager  um das Ma{\ss} $\textcolor{red}{\delta w} $ verr\"{u}cken und die erste Greensche Identit\"{a}t an dem Durchlauftr\"{a}ger formulieren, dann ergibt sich aus der virtuellen \"{a}u{\ss}eren Arbeit der beiden Querkr\"{a}fte links und rechts vom Lager,
\begin{align}
[... V^-\textcolor{red}{\delta w} ]^{x_4} + [V^+\textcolor{red}{\delta w}  \ldots]_{x_4} =  V^- \textcolor{red}{\delta w} - V^+ \textcolor{red}{\delta w} = ( V^- - V^+) \,\textcolor{red}{\delta w}
\end{align}
die Arbeit der Lagerkraft zu
\begin{align}
B \cdot \textcolor{red}{\delta w}\,,
\end{align}
wo also die Lagerkraft in derselben Richtung positiv gez\"{a}hlt wird, wie die virtuelle Verr\"{u}ckung.\\

Wenn wir Risse von einem Haus anfertigen, dann sind
Und alle diese 'virtuellen Verr\"{u}ckungen' sind real, nicht blo{\ss} gedacht.

Immer wenn ein Ingenieur von virtuellen Verr\"{u}ckungen spricht, muss man genau hinh\"{o}ren, weil der Ingenieur dann leider zu oft geneigt ist, Mathematik und Mechanik zu vermengen. Manchmal kommen die Argumente aus der Mathematik, und manchmal kommen sie aber aus der Mechanik und dann sind sie nicht hilfreich.

Argumente aus der Statik sind gut geeignet, um den mechanischen Hintergrund zu erl\"{a}utern, um das Verst\"{a}ndnis zu vertiefen, aber sie haben eigentlich keine Beweiskraft, weil eben
\begin{align}
\text{\normalfont\calligra G\,\,}(w,\textcolor{red}{\delta w}) = 0\,,
\end{align}
ein mathematisches Resultat ist.

Vielleicht sollten wir uns damit zufrieden geben, zu betonen, dass virtuelle Verr\"{u}ckungen einfach Testfunktionen sind und dass die einzige Restriktion, der sie unterlegen, ist, dass sie hinreichend glatt sind, damit man sie partiell integrieren kann.

\"{A}hnliches kann man \"{u}ber die virtuellen Kr\"{a}fte sagen. Sie sind weder nur gedacht noch klein, sondern sie werden durch Differentiation
aus der Testfunktion $\delta w^* $  hergeleitet, die bei der Formulierung des Prinzips der virtuellen Kr\"{a}fte jetzt die erste Stelle in der ersten Greenschen Identit\"{a}t einnimmt
\begin{align}
\text{\normalfont\calligra G\,\,}(\textcolor{red}{\delta w^*},w) = 0\,.
\end{align}

%----------------------------------------------------------------------------------------------------------
\begin{figure}[tbp]
\centering
\if \bild 2 \sidecaption \fi
\includegraphics[width=0.6\textwidth]{\Fpath/S5}
\caption{Vorzeichenregelung} \label{S5}
%
\end{figure}%
%----------------------------------------------------------------------------------------------------------
\end{document}
%%%%%%%%%%%%%%%%%%%%%%%%%%%%%%%%%%%%%%%%%%%%%%%%%%%%%%%%%%%%%%%%%%%%%%%%%%%%%%%%%%%%%%%%%%%%%%%%%%%
{\textcolor{blau2}{\section{Das Vorzeichen der Balkenendkr\"{a}fte}}}\index{Vorzeichen der Balkenendkr\"{a}fte}
In den Greenschen Identit\"{a}ten steckt sehr viel Statik, auf die wir im folgenden noch n\"{a}her eingehen wollen.

Wenn man sich einmal die M\"{u}he macht und die Randarbeiten in der ersten Greenschen Identit\"{a}t des Balkens in voller L\"{a}nge ausschreibt
\begin{align}\label{Eq24}
\text{\normalfont\calligra G\,\,}(w, w) &= \int_0^{\,l} p(x)\,w(x)\,dx + V(l)\,w(l) - M(l)\,w'(l) \nn \\
&- V(0)\,w(0) + M(0)\,w'(0) - \int_0^{\,l} \frac{M^2}{EI}\,dx = 0\,,
\end{align}
dann f\"{a}llt das alternierende Vorzeichen der Randarbeiten auf. Das ist nat\"{u}rlich der partiellen Integration geschuldet. Es verwundert dabei aber doch, dass die Arbeitsbeitr\"{a}ge
immer das richtige Vorzeichen haben, wenn man Bild \ref{S5} zu Grunde legt. Als ob der Ingenieur geahnt h\"{a}tte, wie er die Richtung der positiven Schnittkr\"{a}fte zu w\"{a}hlen h\"{a}tte.

Aber es ist nat\"{u}rlich umgekehrt, der Ingenieur hat erst das Arbeitsintegral
\begin{align}
\int_0^{\,l} EI\,w^{IV}(x)\,w(x)\,dx
\end{align}
partiell integriert, und dann hat er gewusst, wie er die positiven Richtungen w\"{a}hlen muss.

Man kann es aber auch so sehen: Die partielle Integration holt das aus der Differentialgleichung heraus, was bei ihrer Herleitung explizit oder implizit hineingesteckt wurde.
\\

Wenn diese auch jedem Ingenieur klar ist, 'mit welcher virtuellen Kraft will man eine Querkraft berechnen?'

Die Durchbiegung eines Balkens kann man also auf zwei Arten berechnen
\begin{align}
1 \cdot w(x) = \int_0^{\,l} \frac{M\,\bar{M}}{EI}\,dx = \int_0^{\,l} G_0(y,x)\,p(y)\,dy
\end{align}
indem man $M$ mit $\bar{M}$ \"{u}berlagert (Prinzip der virtuellen Kr\"{a}fte) oder indem man die Durchbiegung $G_0(y,x)$ aus der Kraft $\bar{P} = 1$ mit der Belastung \"{u}berlagert ({\em Satz von Betti\/}).

Das Moment $M(x)$ in einem Punkt $x$ kann man aber nur mit dem {\em Satz von Betti\/} berechnen
\begin{align}
M(x) = \int_0^{\,l} G_2(y,x)\,p(y)\,dy
\end{align}
indem man an der Stelle $x$ den Balken 'knickt', Biegelinie $G_2(y,x)$ , und diese Biegelinie mit der Belastung \"{u}berlagert.

Wenn man trotzdem versucht $M(x)$ mit dem Prinzip der virtuellen Kr\"{a}fte zu berechnen
\begin{align}
M(x) = \int_0^{\,l} \frac{M\,\bar{M}}{EI}\,dx
\end{align}
dann erleidet man Schiffbruch, weil der Momentenverlauf $\bar{M}$, der zu der Biegelinie $G_2$ 'mit Knick' geh\"{o}rt, nicht berechnet werden kann. An dem Knick scheitert man.\\

Interessanter wird diese Gleichung, wenn man die Terme partiell integriert, (sofern Funktionen im Spiel sind) oder aus ihr den umgekehrten Schluss zieht, wie das eine Marktfrau macht.
%----------------------------------------------------------------------------------------------------------
\begin{figure}[tbp]
\centering
\if \bild 2 \sidecaption \fi
\includegraphics[width=0.9\textwidth]{\Fpath/U100}
\caption{Verdrehung einer Waage} \label{U100}
%
\end{figure}%
%----------------------------------------------------------------------------------------------------------

Wenn eine Waage\index{Waage} im Gleichgewicht ist, s. Bild \ref{U100},
\begin{align}
P_L \cdot h_L = P_R \cdot h_R\,,
\end{align}
dann kann man die Gleichung mit beliebigen Zahlen $\textcolor{red}{x = \tan\,\Np}$ multiplizieren
\begin{align}
P_L \cdot h_L \cdot \textcolor{red}{\tan\,\Np}= P_R \cdot h_R\cdot \textcolor{red}{\tan\,\Np}\,,
\end{align}
was bedeutet, dass bei einer beliebigen Drehung $\textcolor{red}{\Np}$ des Waagebalkens die beiden Gewichte $P_L$ und $P_R$ dieselbe Arbeit leisten, denn die Wege, die die beiden Gewichte gehen, sind ja gerade
\begin{align}
\delta w_L = h_L \cdot \textcolor{red}{\tan\,\Np }\qquad \delta w_R = h_R\cdot \textcolor{red}{\tan\,\Np}\,.
\end{align}
Die Marktfrau st\"{o}{\ss}t die Waage leicht an und wenn die Waage in jeder  Lage stehen bleibt, dann herrscht Gleichgewicht. Die Marktfrau schlie{\ss}t also aus dem Prinzip der virtuellen Verr\"{u}ckungen auf das Gleichgewicht
\begin{align}
P_L \cdot h_L \cdot \textcolor{red}{\tan\,\Np}= P_R \cdot h_R\cdot \textcolor{red}{\tan\,\Np} \qquad \Rightarrow \qquad P_L \cdot h_L = P_R \cdot h_R\,.
\end{align}
%----------------------------------------------------------------------------------------------------------
\begin{figure}[tbp]
\centering
\if \bild 2 \sidecaption \fi
\includegraphics[width=0.8\textwidth]{\Fpath/U65}
\caption{Auslenkung $y'$ bei einer echten Drehung und Auslenkung $y$ bei einer Pseudodrehung} \label{U65}
%
\end{figure}%
%----------------------------------------------------------------------------------------------------------

\begin{remark}
Wir haben hier mit Pseudodrehungen gearbeitet $h = x\cdot \textcolor{red}{\tan\,\Np}$. Wenn man reale Drehungen zu Grunde legt, so wie sie die Marktfrau sieht, dann betr\"{a}gt die Auslenkung der Gewichte $\bar{h} = x\cdot\textcolor{red}{\sin\,\Np}$, ist also etwas kleiner, s. Bild \ref{U65}. Mathematisch ist es jedoch irrelevant, ob der Faktor auf beiden Seiten der Gleichung $\textcolor{red}{\tan\,\Np}$ lautet oder $\textcolor{red}{\sin\,\Np}$, da er sich wegk\"{u}rzt.
\end{remark}


Mit Blick auf ihre gemeinsame Mitte, die Einflussfunktionen, kann man die beiden Methoden etwa wie folgt charakterisieren. \\

\begin{itemize}
  \item Finite Elemente: Gen\"{a}herte Einflussfunktion, aber exakte Daten (Belastung $p$)
  \item Randelemente: Exakte Fundamentall\"{o}sung, aber gen\"{a}herte Randwerte
\end{itemize}

Die Erweiterung der partiellen Integration auf zwei und drei Dimensionen
\begin{align}
\int_{\Omega} u,_i \,v\,d\Omega = \int_{\Gamma} u\,n_i\,ds - \int_{\Omega} u\,v,_i\,d\Omega
\end{align}
impliziert, dass das Integral der Spannungen $\sigma_{xx} = E\,(\varepsilon_{xx} + \nu\,\varepsilon_{yy}) = E\,(u_1,_1 + \nu\,u_2,_2)$ in einer allseits festgehaltenen Scheibe, $u_1 = u_2 = 0$ auf dem Rand $\Gamma$, null ist
\begin{align}
\int_{\Omega} E\,(u_1,_1 + \nu\,u_2,_2) \cdot 1\,d\Omega = \int_{\Gamma} E\,(u_1\,n_1 + \nu\,u_2\,n_2)\,ds = 0\,.
\end{align}
Analog zeigt man das f\"{u}r $\sigma_{yy}$ und $\sigma_{xy}$ und f\"{u}r die Momente einer eingespannten Platte.\\

Die Ingenieure sind immer versucht, die Statik 'gerade zu r\"{u}cken', also die Tatsache, dass man in der Balkenstatik die Drehungen durch Pseudodrehungen ersetzt, damit zu rechtfertigen, dass f\"{u}r kleine Drehwinkel $\Np$ ja der Tangens ungef\"{a}hr gleich dem Winkel selbst ist und daher die Statik eigentlich keinen Fehler begeht.

Auf demselben Wege kommt die Forderung in die Statik hinein, dass die virtuellen Verr\"{u}ckungen 'klein' sein m\"{u}ssen, weil dann nicht auff\"{a}llt, dass man sich auf der Tangente statt auf dem Drehkreis bewegt, s. Bild \ref{U14}.

Aber bei der Statik handelt es sich nicht um N\"{a}herungsrechnung, sondern die Statik rechnet {\em exakt\/}. Sie hat diese 'Korrekturen' nicht n\"{o}tig. Wenn man hinter dem Dezimalpunkt nur weit genug nach rechts geht, dann weicht die Tangente doch irgendwann vom Drehkreis ab und man kommt so zu dem Schluss: {\em Tangente ist richtig, Drehkreis ist falsch\/}.

Und wenn man der Tangente folgt, dann ist es ohne Belang, ob die virtuelle Verr\"{u}ckung gro{\ss} oder klein ist. Sie kann {\em jeden\/} Wert haben.

Der 'Grundfehler' aus der Sicht des Ingenieurs ist, wenn man so will, die Tatsache, dass in der ersten Greenschen Identit\"{a}t die Momentensumme $M = 0$ auf Pseudodrehungen beruht. Aber das muss man akzeptieren. Das geh\"{o}rt zur Balkengleichung eben dazu (und eigentlich zur ganzen linearen Mechanik).

\hspace*{-12pt}\colorbox{hellgrau}{\parbox{0.98\textwidth}{Der Ingenieur stellt die Differentialgleichung $EI\,w^{IV}(x) = p(x)$ auf, aber wie die dazu passenden Gleichgewichtsbedingungen aussehen, steht nicht in seinem Belieben. Das entscheidet allein die Mathematik.}}\\

Man kann daher nicht mitten im Galopp pl\"{o}tzlich wieder auf richtige Drehungen umschwenken und so die Statik in die N\"{a}he der N\"{a}herungsrechnung r\"{u}cken. {\em Die Statik rechnet exakt, sie l\"{o}st die gestellten Aufgaben exakt!\/}

Die Einflussfunktionen von statisch bestimmten Tragwerken sind kinematische Ketten und die Punkte bewegen sich dabei nicht auf Kreisb\"{o}gen um den Drehpol, sondern auf Tangenten an die B\"{o}gen, denn der Abstand $x$ vom Drehpol und die Auslenkung $y$ bilden einen rechten Winkel
\begin{align}
\tan\,\Np = \frac{y}{x}\,.
\end{align}
Nur so erh\"{a}lt man die richtigen Einflussfunktionen und damit die korrekten Schnittgr\"{o}{\ss}en.
Das wird von vielen Ingenieure \"{u}bersehen, die immer wieder versuchen die Mathematik an die Statik zur\"{u}ckzubinden, was aus Gr\"{u}nden der Anschauung sicherlich sinnvoll und auch geboten ist. Ja unbedingt notwendig ist, um das statische Gef\"{u}hl ...\\

 Von diesen Bilanzen, oder sollen wir besser Invarianten sagen,
\begin{align}
\text{DGL} + \Omega + \Gamma + \text{part. Integration} = 0\,.
\end{align}

lebt die Mechanik und die Statik. Die Statik der Kontinua ist im Grunde

Analysis ist also nicht nur Ableitungen, wie hier Gebiet, Rand, Funktion und die  Ableitungen zu einer Einheit verschmelzen. das deutlich macht, was f\"{u}r m\"{a}chtige wieso Differentialgleichungen

 Ein Balken hat eine L\"{a}nge, eine Platte hat eine gewisse Ausdehnung. Die obigen Identit\"{a}ten sind im Grunde nur die Erweiterung der Regeln der partiellen Integration auf

Daher kann man eine Funktion $u$ und ihre Ableitungen im integralen Sinn auf dem Gebiet $\Omega$ messen, und die Gesamtbilanz ist null.
Das ist die Symbiose von Gebiet und Funktion. Zu jedem Gebiet $\Omega$ und zu jeder auf $\Omega$ definierten Funktion
geh\"{o}rt eine Bilanzgleichung.

Die obigen Identit\"{a}ten sind {\em Invarianten\/} der Differential- und Integralrechnung.

Die Regeln der partiellen Integration besagen nun, dass\\

\begin{itemize}
  \item das Gleichgewicht
  \item das Prinzip der virtuellen Verr\"{u}ckungen
  \item das Prinzip der virtuellen Kr\"{a}fte
  \item der Energieerhaltungssatz
  \item der Satz von Betti
\end{itemize}\\

%---------------------------------------------------------------------------------
\begin{figure}[tbp]
\centering
\if \bild 2 \sidecaption \fi
\includegraphics[width=0.8\textwidth]{\Fpath/U131}
  \caption{Quadratplatte 8 m $\times $ 8 m, FE-Einflussfunktion f\"{u}r das Biegemoment $m_{xx}$ im St\"{u}tzenanschnitt, \textbf{ a)} 3-D Darstellung, \textbf{ b)} L\"{a}ngsschnitt}
  \label{U131}
%
\end{figure}
%---------------------------------------------------------------------------------\\

 l\"{a}ngs
\bfo\label{Phi1Bis4}
\Np_1^e(x) = \frac{1 - x}{l} \qquad  \Np_2^e(x) = \frac{x}{l}
\efo
und quer
\bfo\label{Phi1Bis4}
\parbox{5cm}{
\bfo
\Np_1^e(x) &=& 1 - \frac{3x^2}{l^2} + \frac{2x^3}{l^3} \nn \\
\Np_2^e(x) &=& - x + \frac{2x^2}{l} - \frac{x^3}{l^2} \nn
\efo
}
\parbox{5cm}{
\bfo
\Np_3^e(x) &=& \frac{3x^2}{l^2} - \frac{2x^3}{l^3}\nn \\
\Np_4^e(x) &=& \frac{x^2}{l} - \frac{x^3}{l^2}\,.\nn  \label{Einheitsverformungen}
\efo
}
\efo



%%%%%%%%%%%%%%%%%%%%%%%%%%%%%%%%%%%%%%%%%%%%%%%%%%%%%%%%%%%%%%%%%%%%%%%%%%%%%%%%%%%%%%%%%%%%%%%%%%%
{\textcolor{blau2}{\section{Drehungen}}}\index{Drehungen}
Eine Merkw\"{u}rdigkeit weist die Balkenstatik auf und zwar sind das Drehungen, genauer gesagt Starrk\"{o}rperdrehungen. So ist die Einflussfunktion f\"{u}r das Biegemoment in einem Kragtr\"{a}ger eine solche Drehung der rechten H\"{a}lfte des Kragtr\"{a}ger um 45$^0$ nach oben. Wenn man auf dem Kragtr\"{a}ger nur weit genug nach au{\ss}en geht, also den Kragtr\"{a}ger nur lang genug macht, dann kann man in dem Kragtr\"{a}ger beliebig gro{\ss}e Momente erzeugen.

Verl\"{a}ngert man den Strahl bis zum Fixsternhimmel, so entstehen aus winzigen Drehungen unendlich gro{\ss}e Verschiebungen.
Richtet man einen sehr starken Laserstrahl auf den Mond, dann bewegen sich die Lichtpunkte auf dem Mond bei einer winzigen Drehung des Lasers mit einer Geschwindigkeit, die gr\"{o}{\ss}er ist als die Lichtgeschwindigkeit. (Was kein Widerspruch zur Relativit\"{a}tstheorie von Einstein ist).\\




die Statik kennt keine Einflussfunktionen f\"{u}r die \"{a}quivalenten Lagerkr\"{a}fte $f_i$ bei Fl\"{a}chentragwerken
\begin{align}
f_i \overset{?}{=}  \int_{\Omega} G(\vek y,\vek x)\,p(\vek y)\,d\Omega_{\vek y}\,.
\end{align}
Die $f_i$ geh\"{o}ren ja nach der Vorstellung des Ingenieurs zu Punktlagern, aber man kann Fl\"{a}chentragwerke (mit der oben erw\"{a}hnten Ausnahme) nicht auf Punktlager stellen.
Die $f_i$ geh\"{o}ren also ganz in das FE-Modell

Die Sache wird dadurch verkompliziert, dass Ingenieure sich inzwischen daran gew\"{o}hnt haben, die Knotenkr\"{a}fte $f_i$ in den Lagerknoten als die Lagerkr\"{a}fte einer Platte oder Scheibe anzusehen. Sie haben kein Problem damit, gedanklich die klassische Statik mit dem Model 'als ob' zu mischen.

Um die $f_i$ am Leben zu erhalten, muss man also das Modell wechseln und

Das ist ein Punkt, wo das FE-Modell dem Denken des Ingenieurs viel n\"{a}her ist, als die Theorie, die gleich alles auf die 'Spitze' treibt und Punktlager unendlich d\"{u}nnen Nadeln gleichsetzt, was nat\"{u}rlich kein Material aush\"{a}lt.

F\"{u}r eine Luftst\"{u}tze kann man keine Einflussfunktion berechnen, sie ist identisch null, aber f\"{u}r die Knotenkr\"{a}fte $f_i$ in einem Punktlager schon. Diese Zweigleisigkeit w\"{u}rde sich erst aufl\"{o}sen, wenn man die Maschenweite $h$ der Elemente gegen Null gehen lassen w\"{u}rde, dann w\"{u}rde auch die Einflussfunktion f\"{u}r die Knotenkraft gegen null gehen, aber so weit geht man in der Praxis nicht.

Dies ist ein Punkt, wo die Maschenweite $h$ \"{u}ber ihren reinen Zahlenwert hinaus, auch den Charakter des Modells steuert
\begin{itemize}
  \item $h > 0$, ein Linienlager auf einer unter Umst\"{a}nden sehr kleinen Fl\"{a}che $h \times d$ ($d$ = St\"{a}rke der Scheibe)
  \item $h \to 0$, eine unendlich d\"{u}nne Nadel als Lager\,.
\end{itemize}
Das hat man sonst bei finiten Elementen nicht.

\\

Bei Stabtragwerken kommen Einzelkr\"{a}fte, Einzelmomente in nat\"{u}rlicher Weise vor und daher ordnen sich Knotenkr\"{a}fte und Knotenmomente ohne Probleme in die Statik ein. Numerik und Statik gehen parallel.

Bei Scheiben und Platten sind die Knotenkr\"{a}fte $f_i$ jedoch nur Rechengr\"{o}{\ss}en, 'eins im Sinn', die stellvertretend die Wirkungen der Fl\"{a}chen- oder Linienkr\"{a}fte beschreiben, die in der Umgebung des Knotens stehen. Es sind Arbeits\"{a}quivalente. Eine \"{a}quivalente Knotenkraft von $f_i = 10$ kNm signalisiert, dass die in der N\"{a}he des Knotens verteilten Kr\"{a}fte bei einer Auslenkung des Knotens um 1 m die Arbeit $10$ kNm leisten w\"{u}rden, also soviel, wie eine einzelne Kraft von $f_i = 10$ kN leisten w\"{u}rde, die im Knoten steht.
\\

An Hand des Systems $\vek K_{G}\,\vek u_{G} = \vek f_{G}$ berechnet ein FE-Programm die $f_i$ in den Lagerknoten und daraus dann durch Interpolation der Knotenwerte die Lagerkraft der Platte bzw. der Scheibe l\"{a}ngs des Randes.

Diese Vorgehensweise vermeidet auch die Frage, wie man denn eine Einflussfunktion f\"{u}r eine Lagerkraft in einem Punkt berechnet, denn wie wollte man einen Punkt des Randes um eine L\"{a}ngeneinheit absenken aber den Rest stehen lassen?

Das geht nur so, dass man von Innen kommt, also den Kirchhoffschub in 10 cm Abstand vom Rand berechnet und dann auf den  Rand extrapoliert. Wenn man das f\"{u}r alle Punkte l\"{a}ngs des Randes macht, dann erh\"{a}lt man die Lagerkr\"{a}fte l\"{a}ngs des Randes.

Wir wollen hier nicht untersuchen, inwieweit die so von Innen auf den Rand extrapolierten Lagerkr\"{a}fte mit dem Verlauf \"{u}bereinstimmen, den man aus den $f_i$ erh\"{a}lt, und uns lieber damit besch\"{a}ftigen, wie man Einflussfunktionen f\"{u}r die $f_i$ berechnen kann.\\

%%%%%%%%%%%%%%%%%%%%%%%%%%%%%%%%%%%%%%%%%%%%%%%%%%%%%%%%%%%%%%%%%%%%%%%%%%%%%%%%%%%%%%%%%%%%%%%%%%%
{\textcolor{blau2}{\section{Goal oriented adaptive refinement}}\index{Betti und finite Differenzen}
Zu Beginn dieses Kapitels haben wir, s. Bild \ref{U122}, die Durchbiegung eines Seils in einem Punkt berechnet, der zwischen zwei Knoten lag. Dies zwang uns dazu mit einer gen\"{a}herten Einflussfunktion $G_h(y,x)$ zu rechnen, was eine Abweichung in der berechneten Durchbiegung zur Folge hatte
\begin{align}
w(x) - w_h(x) = \int_0^{\,l} (G(y,x) - G_h(y,x))\,p(y)\,dy\,.
\end{align}
Hier ist $p$ die Original-Belastung auf dem Seil. Wie wir wissen, ersetzt das FE-Programm aber die Streckenlast $p$ durch Einzelkr\"{a}fte $f_i$ in den Knoten, die den FE-Lastfall $p_h$ darstellen. Um jetzt nicht zu weit ausholen zu m\"{u}ssen, nehmen wir der Einfachheit halber an, dass der FE-Lastfall $p_h$  aus einem auf und ab von Streckenlasten $p_h$ besteht. (Was in diesem Fall ja nicht richtig ist, aber auch mit Knotenkr\"{a}ften bleibt die Logik dieselbe).

Nun kann man zeigen, dass sich die obige Formel nicht \"{a}ndert, wenn man f\"{u}r $p$ die Differenz $p - p_h$ setzt
\begin{align} \label{Eq85}
w(x) - w_h(x) = \int_0^{\,l} (G(y,x) - G_h(y,x))\,(p(y) - p_h(y))\,dy\,.
\end{align}
Das Integral, die \"{a}u{\ss}ere Arbeit, kann man durch die gleich gro{\ss}e innere  Arbeit
\begin{align} \label{Eq86}
w(x) - w_h(x) &= H \int_0^{\,l} (G'(y,x) - G_h'(y,x))\,(w'(y) - w_h'(y))\,dy\nn \\
 &= a(G-G_h,w-w_h)
\end{align}
ersetzen. Nach ein paar weiteren Schritten f\"{u}hrt dies auf die Absch\"{a}tzung
\begin{align}
|w(x) - w_h(x)| \leq c \cdot a(G-G_h,G-G_h)\cdot a(w-w_h,w-w_h)
\end{align}
mit einer Konstanten $c$. Die Botschaft dieser Ungleichung ist, dass der Fehler beschr\"{a}nkt ist durch die inneren Energie des Fehlers $G-G_h$ in der Einflussfunktion und der inneren Energie des Fehlers $w - w_h$ der FE-L\"{o}sung, was im Grunde bedeutet: Je schlechter sich die Einflussfunktion bzw. die exakte L\"{o}sung ann\"{a}hern l\"{a}sst, um so gr\"{o}{\ss}er ist der Fehler auf der linken Seite---was evident klingt.
\\

Dieses Buch hat daher die Absicht zu zeigen, wie die Arbeits- und Energieprinzipe der Statik

\begin{itemize}
  \item das Prinzip der virtuellen Verr\"{u}ckungen
  \item der Energieerhaltungssatz
  \item das Prinzip der virtuellen Kr\"{a}fte
  \item der Satz von Betti
\end{itemize}
sich aus der Mathematik entwickeln, weil diese S\"{a}tze keine Naturgesetze sind, sonder sie in allen ihren Ausformungen nur auf Mathematik beruhen. Wenn man zu ihren Quellen will, dann muss man die mathematischen Grundlagen dieser S\"{a}tze und Prinzipe studieren.\\

Wir wollen aus den Ingenieuren keine Mathematiker machen, aber sie sollen doch an einem Punkt die Gelegenheit bekommen zu verstehen, warum virtuelle Verr\"{u}ckungen oder virtuelle Kr\"{a}fte nicht klein sein m\"{u}ssen. Warum das Rechnen in der Statik auf mathematischen Gesetzen beruht und nicht auf Naturgesetzen.\\

%%%%%%%%%%%%%%%%%%%%%%%%%%%%%%%%%%%%%%%%%%%%%%%%%%%%%%%%%%%%%%%%%%%%%%%%%%%%%%%%%%%%%%%%%%%%%%%%%%%
{\textcolor{blau2}{\section{Gl\"{a}tten von Oszillationen}}
Eines der popul\"{a}rsten Ver\"{o}ffentlichungen zu dem Thema finite Elemente ist der Aufsatz von Zhu und Zienkiewicz, \cite{Z2}, der zeigt, wie man Oszillationen in den Spannungen gl\"{a}tten kann.

Der umgekehrte Schluss lautet dann, wenn die Spannungen nicht oszillieren, dann sind sie 'richtig'. Und das kann ein Trugschluss sein, wie in \cite{Babuska5} gezeigt wurde. Pollution kann die Spannungen in einem Bauteil in eine gewisse Richtung verf\"{a}lschen, ohne dass es dabei zu Oszillationen kommt. Es werden einfach nur alle Werte um einen gewissen Betrag nahezu gleichm\"{a}{\ss}ig angehoben oder gesenkt.

Und der Ingenieur ahnt nicht, dass diese glatten Spannungen einen systematischen Fehler aufweisen.

%----------------------------------------------------------------------------------------------------------
\begin{figure}[tbp]
\centering
\if \bild 2 \sidecaption \fi
\includegraphics[width=1.0\textwidth]{\Fpath/S2}
\caption{Anwendung des Prinzips der virtuellen Kr\"{a}fte} \label{S2}
%
\end{figure}%
%----------------------------------------------------------------------------------------------------------

%----------------------------------------------------------------------------------------------------------
\begin{figure}[tbp]
\centering
\if \bild 2 \sidecaption \fi
\includegraphics[width=0.6\textwidth]{\Fpath/S11}
\caption{Anwendung des Prinzips der virtuellen Verr\"{u}ckungen zur Berechnung der Lagerkraft $A$} \label{S11}
%
\end{figure}%
%----------------------------------------------------------------------------------------------------------\\

Die Einflussfunktion f\"{u}r eine Durchbiegung $w(x)$ ist gleich der Biegelinie, die von einer Einzelkraft $P = 1$ erzeugt wird, die in dem Aufpunkt $x$ angreift und die Einflussfunktion f\"{u}r eine Verdrehung $w'(x)$ ist die Biegelinie, die von einem Moment $M = 1$ erzeugt wird,
das in dem Aufpunkt $x$ angreift, s. Bild \ref{U150}.

Es ist also immer die zur Weggr\"{o}{\ss}e konjugierte Kraftgr\"{o}{\ss}e, die die Einflussfunktion produziert. Diese einfache Regel basiert auf dem Satz von Betti.\\

Um die Querkraft im Punkt $x$ des Tr\"{a}gers in Bild  \ref{U164} zu berechnen, bauen wir im Punkt $x$ ein Querkraftgelenk ein, unterbrechen also den Kraftfluss und m\"{u}ssen dies dadurch korrigieren, dass wir jetzt von au{\ss}en zwei gegengleiche Querkr\"{a}fte $V(x)$ wirken lassen. Dann erteilen wir dem so modifizierten Tr\"{a}ger eine Bewegung derart, dass die beiden Querkr\"{a}fte insgesamt den Weg $(-1)$ zur\"{u}cklegen, sie also die Arbeit
\begin{align}
V(x)\,(\textcolor{blau2}{-1})
\end{align}
leisten. Bei dieser Bewegung leistet die \"{a}u{\ss}ere Belastung die Arbeit
\begin{align}
P\cdot \textcolor{blau2}{w}
\end{align}
und gem\"{a}{\ss} dem Satz von Betti muss die Summe dieser beiden Arbeiten null sein, weil, wie wir sp\"{a}ter sehen werden, $A_{2,1}$ immer null ist,
\begin{align}
A_{1,2} = V(x)\cdot (\textcolor{blau2}{-1}) + P\cdot\textcolor{blau2}{w} = 0 \qquad (\leftarrow \,\,\,\,A_{2,1})
\end{align}
oder
\begin{align}
V(x) \cdot \textcolor{blau2}{1} = P\,\textcolor{blau2}{w}\,.
\end{align}\\

%----------------------------------------------------------
\begin{figure}[tbp]
\centering
\includegraphics[width=1.0\textwidth]{\Fpath/1GREENF73D}
\caption{Wie die Einflussfunktion f\"{u}r das Biegemoment $M$ \"{u}ber den Tr\"{a}ger wandert und  dabei im Grunde ihre Gestalt beibeh\"{a}lt (Gleichlast)}
\label{1GreenF73}%
%
\end{figure}%%
%----------------------------------------------------------

%----------------------------------------------------------
\begin{figure}[tbp]
\centering
\includegraphics[width=1.0\textwidth]{\Fpath/1GREENF74D}
\caption{Wie die Einflussfunktion f\"{u}r die Querkraft $V$ \"{u}ber den Tr\"{a}ger wandert und dabei im Grunde ihre Gestalt beibeh\"{a}lt (Gleichlast) }
\label{1GreenF74}%
%
\end{figure}%%
%----------------------------------------------------------\\

%%%%%%%%%%%%%%%%%%%%%%%%%%%%%%%%%%%%%%%%%%%%%%%%%%%%%%%%%%%%%%%%%%%%%%%%%%%%%%%%%%%%%%%%%%%%%%%%%%%
{\textcolor{blau2}{\section{Varianten im Entwurf}}}
Wenn man einen Backstein wegnimmt, \"{a}ndert sich dann die Einflussfunktion oder nicht? Wenn sie sich nicht \"{a}ndert, dann kann man den Backstein weglassen. Das ist---etwas (stark) verk\"{u}rzt---die Frage, die wir hier anschneiden wollen.

Weil Einflussfunktionen f\"{u}r Lager- und Schnittkr\"{a}fte in statisch bestimmten Tragwerken kinematische Ketten sind, ist es klar, dass die Lagerkr\"{a}fte nicht davon abh\"{a}ngen, wie die Bauteile dimensioniert sind. \"{A}hnliches gilt f\"{u}r die Schnittkr\"{a}fte, obwohl hier die spezielle Gestalt eines Tr\"{a}gers einen Einfluss haben kann. In einem Fachwerktr\"{a}ger spaltet sich eine Querkraft anders auf, als in einem geraden Biegebalken, aber vom Prinzip her
bestimmt nur die Kinematik die Gr\"{o}{\ss}e der Schnittkr\"{a}fte.
%--------------------------------------------------------------------------------------
\begin{figure}
\centering
\includegraphics[width=0.7\textwidth]{\Fpath/U26}
\caption{Statisch bestimmtes Fachwerk, die Stabkr\"{a}fte h\"{a}ngen nur von der Geometrie des Fachwerkes ab, aber nicht von den Steifigkeiten $EA_i$ der St\"{a}be}
\label{U26}%
%
\end{figure}%
%--------------------------------------------------------------------------------------

%%%%%%%%%%%%%%%%%%%%%%%%%%%%%%%%%%%%%%%%%%%%%%%%%%%%%%%%%%%%%%%%%%%%%%%%%%%%%%%%%%%%%%%%%%%%%%%%%%%
{\textcolor{blau2}{\subsection{Statisch bestimmtes Fachwerk}}}

Die Situation l\"{a}sst sich am besten an Hand eines Fachwerks illustrieren.
Viele Probleme in der Mechanik m\"{u}nden bei der Behandlung mit finiten Elementen in dem dreifachen Produkt
\beq
\vek K_{n \times n}\,\vek u_{n \times 1} = \vek A^T_{n \times m}\,\vek C_{m \times m}\,\vek A_{m \times n}\,\vek u_{n \times 1} = \vek f_{n \times 1}\,,
\eeq
wobei $\vek A$ eine rechteckige Matrix ist und $\vek C$ eine quadratische Matrix, die von den Materialparametern abh\"{a}ngt, \cite{Strang4}.

In einem Fachwerk aus $m$ St\"{a}ben sind die Komponenten des Vektor $\vek A\,\vek u$ die Dehnungen $\varepsilon_i = u_i'$ der $m$ Fachwerkst\"{a}be und die Matrix $\vek C$ ist eine Diagonalmatrix mit den Eintr\"{a}gen $c_{ii} = EA_i$, einer f\"{u}r jeden Stab, und der Vektor
\beq
\vek C\vek A\,\vek u = \vek n = \{N_1, N_2, \ldots N_m\}^T
\eeq
ist die Liste der Normalkr\"{a}fte $N_i$ in den Fachwerkst\"{a}ben und das Gleichungssystem
\beq
\vek A^T\,\vek n = \vek f
\eeq
formuliert die Gleichgewichtsbedingungen in den Knoten. Wenn die Matrix $\vek A$ quadratisch ist wie in einem statisch bestimmten Fachwerk, dann reichen die Gleichgewichtsbedingungen alleine aus, um die Stabkr\"{a}fte zu berechnen
\beq
\vek n = (\vek A^T)^{-1} \,\vek f
\eeq
und diese wiederum bestimmen die Knotenverschiebungen
\beq
\vek u = (\vek C\,\vek A)^{-1} \,\vek n\,.
\eeq
Weil die Matrix $\vek A$ nur von der Geometrie des Fachwerkes abh\"{a}ngt, also der L\"{a}nge  $l_i$ der St\"{a}be und ihren Winkeln $\alpha_i$ gegen\"{u}ber der Horizontalen, folgt, dass in einem statisch bestimmten Fachwerk die Einflussfunktionen f\"{u}r die Normalkr\"{a}fte, die ja die Spalten der Matrix $(\vek A^T)^{-1}$ bilden, nur von der Geometrie des Fachwerkes abh\"{a}ngen, aber nicht von den Steifigkeiten $EA_i$ der einzelnen Elemente, s. Bild \ref{U26}.

Anders ist es bei den Knotenverschiebungen. Ihre Einflussfunktionen h\"{a}ngen von der Matrix $\vek C$ ab.\\

\hspace*{-12pt}\colorbox{hellgrau}{\parbox{0.98\textwidth}{In statisch bestimmten Tragwerke h\"{a}ngen die Einflussfunktionen f\"{u}r die Schnitt- und Lagerkr\"{a}fte nicht von den Steifigkeiten der einzelnen Bauteile ab. Die Einflussfunktionen f\"{u}r Verformungen dagegen schon.}}\\

%%%%%%%%%%%%%%%%%%%%%%%%%%%%%%%%%%%%%%%%%%%%%%%%%%%%%%%%%%%%%%%%%%%%%%%%%%%%%%%%%%%%%%%%%%%%%%%%%%%
{\textcolor{blau2}{\subsection{Scheibe}}}
Bei statisch bestimmten Tragwerken bilden die Einflussfunktionen kinematische Ketten und daraus folgt, dass die Gestalt einer Einflussfunktion
nicht von den Steifigkeiten der einzelnen Bauteilen abh\"{a}ngt.

Daraus m\"{u}ssen wir schlie{\ss}en, dass eine Scheibe hochgradig statisch unbestimmt ist, denn die Gestalt der Einflussfunktion f\"{u}r eine Spannung $\sigma_{xx}$, die ja durch eine Spreizung des Aufpunktes ausgel\"{o}st wird, h\"{a}ngt deutlich von der Steifigkeit der Scheibe ab.

Eine solche Einflussfunktion ist ein 'Gesamtkunstwerk', an deren Auspr\"{a}gung und Gestalt die ganze Scheibe beteiligt ist. Jede noch so kleine Modifikation in einer Steifigkeit $k_{ij}$ \"{a}ndert die Einflussfunktion ab, denn die Zeilen (= Spalten) der inversen Steifigkeitsmatrix $\vek K^{-1}$ sind ja die Knotenverschiebungen  der Einflussfunktionen.\\

%---------------------------------------------------------------------------------
\begin{figure}[tbp]
\centering
\if \bild 2 \sidecaption \fi
\includegraphics[width=0.8\textwidth]{\Fpath/U132}
  \caption{Wandscheibe, FE-Einflussfunktion f\"{u}r die Spannung $\sigma_{xx}$ in einem Punkt nahe dem unteren Rand, \textbf{ a)} Aufriss, \textbf{ b)} Knotenverschiebungen (Vektoren $\vek g$) aus der Einflussfunktion, $\sigma_{xx} = \vek g^T\,\vek f$}
  \label{U132}
%
\end{figure}
%---------------------------------------------------------------------------------\\

%----------------------------------------------------------------------------------------------------------
\begin{figure}[tbp]
\centering
\if \bild 2 \sidecaption \fi
\includegraphics[width=1.0\textwidth]{\Fpath/U43}
\caption{Das Rohr mit vier Durchmessern muss in vier Abschnitte unterteilt werden. F\"{u}r jeden Abschnitt wird separat die erste Greensche Identit\"{a}t aufgestellt und dann werden die Identit\"{a}ten addiert, 0 + 0 + 0  + 0 = 0} \label{U43}
%
\end{figure}%
%----------------------------------------------------------------------------------------------------------

%%%%%%%%%%%%%%%%%%%%%%%%%%%%%%%%%%%%%%%%%%%%%%%%%%%%%%%%%%%%%%%%%%%%%%%%%%%%%%%%%%%%%%%%%%%%%%%%%%%
{\textcolor{blau2}{\section{Unterschiedliche Steifigkeiten}}}\index{unterschiedliche Steifigkeiten}
Wenn sich die Steifigkeiten l\"{a}ngs eines Tr\"{a}gers \"{a}ndern, wie in Bild \ref{U43}, dann kann man die erste Greensche Identit\"{a}t nur abschnittsweise anschreiben. Weil aber die Randarbeiten an den Intervallgrenzen wegen $u_L = u_R$ und $N_L = N_R$ bei der Addition der Identit\"{a}ten wegfallen, bleibt am Schluss der Ausdruck
\begin{align}
\text{\normalfont\calligra G\,\,}(u,u) &= \int_0^{\,x_1} \frac{N^2}{EA_1}\,dx + \int_{\,x_1}^{x_2} \frac{N^2}{EA_2}\,dx +\int_{\,x_2}^{x_3} \frac{N^2}{EA_3}\,dx + \int_{\,x_3}^{x_4} \frac{N^2}{EA_4}\,dx \nn \\
&-P\,u(l) = 0\,.
\end{align}
\"{u}brig, der, bis auf den Faktor $1/2$, dem Energieerhaltungssatz entspricht.

%---------------------------------------------------------------------------------
\begin{figure}
\centering
{\includegraphics[width=0.9\textwidth]{\Fpath/U104}}
  \caption{\textbf{ a)} Unterteilung eines Stabes in f\"{u}nf lineare Elemente
  \textbf{ b-f)} die Verschiebungen sind die Spalten der inversen Steifigkeitsmatrix (alle Werte mal $l/(EA)$).}
  \label{U104}
%
\end{figure}%
%---------------------------------------------------------------------------------
\\


Ist diese Kraft nun gleich $R_{FE} + R_{X}$ oder nur gleich $R_{FE}$? Es ist die volle St\"{u}tzenkraft, also $R_h = R_{FE} + R_{X}$, wenn wir die Bezeichnungen des obigen Beispiels w\"{a}hlen. Dies sieht man, wenn man die {\em $h$-Vertauschungsregel\/} anwendet
\begin{align}\label{Eq99}
R_h = \int_{\Omega} G_h(\vek y,\vek x)\,p(\vek y)\,d\Omega_{\vek y} = \int_{\Omega} G(\vek y,\vek x)\,p_h(\vek y)\,d\Omega_{\vek y}\,,
\end{align}
denn gem\"{a}{\ss} dem zweiten Integral ist die Kraft $R_h$ gleich der mit der exakten Einflussfunktion $G(\vek y,\vek x)$ ermittelten St\"{u}tzenkraft des FE-Lastfalls $p_h$, und daher fehlt nichts.

Dabei ist aber ein kleiner Trick im Spiel. Das $p_h$ in (\ref{Eq99}) ist der FE-Lastfall des Systems {\em ohne St\"{u}tze\/}, weil ja am Anfang die St\"{u}tze weggenommen wurde, um die Platte dort um 1 Meter nach unten dr\"{u}cken zu k\"{o}nnen.\\

Betrachten wir eine Platte, in deren Mitte eine starre St\"{u}tze steht. Der Knoten habe in vertikaler Richtung den Freiheitsgrad $u_i$, s. Bild \ref{U142}. Es sei $\Np_i(\vek x)$ die Einheitsverformung, die zu dem Freiheitsgrad $u_i$  geh\"{o}re. Dann ist $f_i$ die Summe
\begin{align}
\sum_j k_{i j} u_j = f_i\,.
\end{align}
Diese Gleichung entspricht der Bilanz
\begin{align}
\delta A_i(w_h,\Np_i) = \delta A_a(p_h,\Np_i)\,,
\end{align}
was die umgestellte erste Greensche Identit\"{a}t ist
\begin{align}
\text{\normalfont\calligra G\,\,}(w_h,\Np_i) = \delta A_a(p_h,\Np_i) - \delta A_i(w_h,\Np_i) = 0\,.
\end{align}
Wir nennen diese St\"{u}tzkraft $R_{FE}$. Zu ihr muss nun noch die St\"{u}tzkraft $R_X$ addiert werden, die n\"{o}tig ist, um der Kraft, die direkt in die St\"{u}tze flie{\ss}t, das Gleichgewicht zu halten
\begin{align}
R_X = -\int_{\Omega} p\,\Np_i\,d\Omega\,,
\end{align}
so dass sich die gesamte St\"{u}tzkraft zu
\begin{align}
R_h = R_{FE} + R_{X}
\end{align}
ergibt.

All das gilt nat\"{u}rlich auch sinngem\"{a}{\ss} f\"{u}r die R\"{a}nder von Scheiben und Platten, wo randnahe Lasten direkt in die Lager reduziert werden. Sie gehen in den Vektor $\vek f_{G}$, der dem System $\vek K_{G}\,\vek u_{G} = \vek f_{G}$ zu Grunde liegt, nicht ein. Sie m\"{u}ssen 'von Hand' dazu addiert werden.\\

Mit finiten Elementen erh\"{a}lt man nat\"{u}rlich nur eine N\"{a}herung $G_h(\vek y,\vek x)$ f\"{u}r die Biegefl\"{a}che  und so ist auch die FE-St\"{u}tzenkraft
\begin{align}
R_h = \int_{\Omega} G_h(\vek y,\vek x)\,p(\vek y)\,d\Omega_{\vek y}
\end{align}
nur eine N\"{a}herung.

Auch dieser Einflussfunktion fehlt der Anteil $\Np_X(\vek y)$, weil diese Funktion ja nicht in $V_h$ liegt. Gedanklich ist es aber ein leichtes, die Funktion zu $G_h(\vek y,\vek x)$  hinzu zu addieren.
\\

Schlie{\ss}lich und endlich l\"{a}uft das Ganze darauf hinaus, dass man die Einflussfunktion f\"{u}r die \"{a}quivalente Lagerkraft in dem Knoten einer Platte oder Scheibe so berechnet, wie man das naiverweise vermuten w\"{u}rde: Man spreizt den Spalt zwischen dem Lagerknoten und seinem Widerlager (Wand, St\"{u}tze) um 1 Meter entgegen der Richtung der gesuchten Lagerkraft. Die Verformungsfigur der Platte oder Scheibe, die sich dabei einstellt,
\begin{align}
G_h(\vek y,\vek x)
\end{align}
ist die Einflussfunktion, s. Bild \ref{U180}, f\"{u}r die Lagerkraft $R_h = R_{FE} + R_X$, also die vollst\"{a}ndige Lagerkraft, inklusive dem Anteil $R_X$, der dem Anteil der Belastung entspricht, der direkt in das Lager reduziert wird.\\

%---------------------------------------------------------------------------------
\begin{figure}
\centering
\if \bild 2 \sidecaption \fi
\includegraphics[width=1.0\textwidth]{\Fpath/1GREENF20}
\caption{FE-Einflussfunktion f\"{u}r die Durchbiegung eines Seils. Leicht unterschiedliche Lage \textbf{ a)} und \textbf{ b)} des Aufpunktes $x$. Die Einflussfunktion \"{a}ndert sich praktisch nicht.}
\label{1GreenF20}%
%
\end{figure}%
%---------------------------------------------------------------------------------

Bei Verschiebungen $u(\vek x)$ gibt es keinen Sprung. Die \"{a}quivalenten Knotenkr\"{a}fte sind die Verschiebungen $\Np_i(x)$ der Ansatzfunktionen und wenn man das Element wechselt, dann \"{a}ndern sich diese Knotenkr\"{a}fte nicht sprungweise.

\\

Die Berechnung von Einflussfunktionen
Oben haben wir die Einflussfunktion f\"{u}r eine St\"{u}tzenkraft berechnet, indem wir einfach die Einflussfunktion f\"{u}r die Absenkung des St\"{u}tzenkopfs mit der St\"{u}tzensteifigkeit $k$ multipliziert haben.

Bei Lagerknoten von Aussen- oder Innenw\"{a}nden l\"{a}sst sich diese Technik nicht anwenden, weil solche W\"{a}nde nicht als Abfolge von St\"{u}tzen gedacht werden k\"{o}nnen. Hier kommen wir anders zum Ziel.

%%%%%%%%%%%%%%%%%%%%%%%%%%%%%%%%%%%%%%%%%%%%%%%%%%%%%%%%%%%%%%%%%%%%%%%%%%%%%%%%%%%%%%%%%%%%%%%%%%%
{\textcolor{blau2}{\section{Kopplung Wand---Scheibe}}
Bei der sogenannten {\em Positionsstatik\/}, wo man jeden Unterzug, jede Deckenplatte f\"{u}r sich alleine untersucht, wird man die Auflagerung einer Deckenplatte auf die W\"{a}nde durch Federn simulieren und dann kann man, wie oben gezeigt, sehr einfach die Einflussfunktionen f\"{u}r die \"{a}quivalenten Knotenkr\"{a}fte in diesen Federn berechnen.

Bei einer {\em 3-D Statik\/}, m\"{u}sste man jedoch anders vorgehen.\\

'zerrei{\ss}t', wenn man also einen Balken
aber die Momente und auch die Kr\"{u}mmungen gehen nur wie $-\ln\,r$ gegen Unendlich und dieses Quadrat ist noch me{\ss}bar
\begin{align}
\int_0^{\,2\,\pi} \int_0^{\,1} \ln^2\,r\,dr\,d\Np \leq \infty\,.
\end{align}

Das k\"{o}nnen wir nicht direkt tun, weil $G$ keine $C^1$ Funktion ist. Wir m\"{u}ssen also eine $\varepsilon$-Umgebung des Aufpunktes aussparen
\begin{align}
\text{\normalfont\calligra G\,\,}(G, G)_{\Omega_\varepsilon} = \int_{\Omega_\varepsilon} - \Delta G\,G\,d\Omega_{\vek y} + \int_{\Gamma}
\end{align}
und dann den Grenzprozess
\begin{align}
\lim_{\varepsilon \to 0} \text{\normalfont\calligra G\,\,}(G, G)_{\Omega_\varepsilon}
\end{align}
analysieren.

In dem gelochten Gebiet $\Omega_\varepsilon$ ist $- \Delta G = 0$, auf dem Rand $\Gamma$ ist $G = 0$, so dass von der \"{a}u{\ss}eren Arbeit nur das Integral \"{u}ber den Rand $\Gamma_{N_\varepsilon}$ des Lochs um den Aufpunkt verbleibt und dessen Grenzwert
\begin{align}
\lim_{\varepsilon \to 0} \int_{\Gamma_{N_\varepsilon}} \nabla G \dotprod  \vek n \,G\,ds = 1 \cdot \infty
\end{align}
ist unendlich gro{\ss}, genauso wie die innere Arbeit
\begin{align}
A_i = \lim_{\varepsilon \to 0} \int_{\Omega_\varepsilon } \nabla G \dotprod  \nabla G \,d\Omega = \lim_{\varepsilon \to 0}\int_0^{\,2 \pi} \int_\varepsilon^{\,R} \frac{1}{r^2} r\,dr\,d\Np = \infty\,,
\end{align}
denn das Integral
\begin{align}
\int_0^{\,1} \frac{1}{r}\,dr = \infty
\end{align}
ist unendlich.

In der Praxis macht man sich nat\"{u}rlich nicht die M\"{u}he all diese Grenzprossese genau nachzuvollziehen, sondern man schaut nur auf die innere Energie.

Die Funktionen, bei denen man vermutet, dass sie unendlich gro{\ss}e Energie haben, sind die Einflussfunktionen f\"{u}r Verschiebungen oder Spannungen, etc. Diese haben alle die Struktur
\begin{align}
r^2\,\ln r f(\Np)\qquad \ln r\,\,f(\Np) \qquad \frac{1}{r^n}\,f(\Np)
\end{align}
wobei die Potenz $n$ der St\"{a}rke der Singularit\"{a}t entspricht.\\

Die Methode des {\em goal oriented adaptive refinement\/} basiert auf diesem Fehlersch\"{a}tzer. Dabei wird das Netz adaptiv so verfeinert, dass die beiden Fehler m\"{o}glichst klein werden, \cite{Ha5}.\\

Das ist auch das mathematische Gegenargument zu dem Argument des Ingenieurs, dass das Material kl\"{u}ger sei, denn die Singularit\"{a}ten schlagen direkt bis zu den Einflussfunktionen durch und verf\"{a}lschen somit die Ergebnisse.

Wir w\"{u}rden uns aber trotzdem auf der Seite des Ingenieurs halten, weil in der Regel im Bauwesen viele andere Effekte die Genauigkeit eines FE-Modells bestimmen und man m\"{u}sste schon in einem theoretischen 'Reinraum' operieren, wenn man sich \"{u}bertriebene Gedanken \"{u}ber den Einfluss von Singularit\"{a}ten auf die FE-L\"{o}sung machen wollte, sind doch die Genauigkeitsanforderungen im Bauwesen wesentlich geringer als z.B. im Maschinenbau. Gleichwohl kann es auch in Standardf\"{a}llen, siehe die Berechnung der Einflussfunktion f\"{u}r $N_{yx}$ in Bild \ref{U200}, Probleme mit der Genauigkeit geben. Man muss also mitdenken!

\\

%%%%%%%%%%%%%%%%%%%%%%%%%%%%%%%%%%%%%%%%%%%%%%%%%%%%%%%%%%%%%%%%%%%%%%%%%%%%%%%%%%%%%%%%%%%%%%%%%%%
{\textcolor{blau2}{\section{Modellfehler und numerischer Fehler}}}
Wenn man einen konischen Schaft mit einem konstanten, mittleren Durchmesser rechnet, dann begeht man einen {\em Modellfehler\/}. Wenn man die L\"{a}ngsverschiebung in dem FE-Modell des (gleichf\"{o}rmigen) Schafts durch einen Polygonzug (= lineare Elemente) ann\"{a}hert, dann begeht man einen weiteren Fehler, den {\em numerischen Fehler\/}, so dass sich die folgende Kette von Fehlern ergibt
\begin{align}
u_{exakt} - u_{h} = \underbrace{u_{exakt} - u_{uniform}}_{Modellfehler} +  \underbrace{ u_{uniform} - u_h}_{numerischer Fehler}\,.
\end{align}
In der englischsprachigen Literatur lauten die daran ankn\"{u}pfenden Untersuchungen {\em verification and validation\/}. Verification untersucht die Gr\"{o}{\ss}e des numerischen Fehlers und validation fragt, ob \"{u}berhaupt die richtigen Gleichungen gel\"{o}st wurden. Andernfalls hat man auch auf einem noch so feinen Netz keine Chance, in die N\"{a}he der exakten L\"{o}sung zu kommen.\\

{\textcolor{blau2}{\subsubsection*{Modellfehler}}}
Den Ingenieur interessiert vor allem der Modellfehler, weil man immer das Tragwerk vereinfachen muss, 'um es in den Rechner zu bekommen'. Hier kann der Mathematiker wenig helfen. Nur der Ingenieur kann die Vereinfachungen im Modell rechtfertigen oder auf kritische Punkte im Modell hinweisen.

Der Modellfehler ist eigentlich der Ingenieurfehler schlechthin. Eigentlich interessiert den Ingenieur nur dieser Fehler. Welche Auswirkungen hat es, wenn man eine St\"{u}tze im dritten Geschoss entfernt, oder wie lagern sich die Kr\"{a}fte um, wenn man eine Wand verr\"{u}ckt?

Die Untersuchungen zu den Auswirkungen von Steifigkeits\"{a}nderungen in Kapitel \ref{Steifigkeits\"{a}nderungen} m\"{o}gen bei der Beantwortung solcher Fragen hilfreich sein.

{\textcolor{blau2}{\subsubsection*{Numerische Fehler}}}
Dagegen ist der numerische Fehler relativ gut erforscht. Er setzt sich aus zwei Anteilen zusammen

\begin{enumerate}
  \item dem Interpolationsfehler, wenn man also Kurven und Fl\"{a}chen st\"{u}ckweise durch Polynome ann\"{a}hert
  \item dem Fehler, der darauf beruht, dass man ja die Knotenwerte der Kurve, der Fl\"{a}che, die man interpolieren will, nicht kennt und sich durch L\"{o}sen des  Systems $\vek K\,\vek u = \vek f$ erst N\"{a}herungen daf\"{u}r beschaffen muss.
\end{enumerate}

Der numerische Fehler h\"{a}ngt
\begin{itemize}
  \item von der Maschenweite $h$ ab
  \item davon, wie glatt die L\"{o}sung ist, die man approximieren will. Die Biegelinie eines Balkens im LF $g$ ist glatter, als die Biegelinie unter einer Einzelkraft, weil letztere einen Knick in der zweiten Ableitung ($M = - EI\,w''$) und einen Sprung in der dritten Ableitung ($V = - EI\,w'''$) aufweist,
  \item von der Ordnung der Ableitungen ab. Je h\"{o}her die Ableitung der Zielgr\"{o}{\ss}e,  um so  gr\"{o}{\ss}er ist der Fehler. Das ergibt z.B. bei einer Platte die folgende Reihenfolge:
  \begin{itemize}
    \item $w$
    \item $w,_x$; $w,_y$
    \item $m_{xx}$, $m_{xy}$, $m_{yy}$
    \item $q_x$, $q_y$
  \end{itemize}
  \item Einflussfunktionen, die differenzieren, ungerade Ableitungen des Zielwertes, $w,_x$ und $w,y$ bzw. $q_x$ und $q_y$, sind schlechter zu approximieren als solche, die integrieren, gerade Ableitungen des Zielwertes, $w$  (0-te Ableitung) und die Momente $m_{xx}, m_{xy}, m_{yy}$ (zweite Ableitung).
\end{itemize}

Einen wichtigen Schluss kann man aus diesen Bemerkungen aber ziehen, n\"{a}mlich dass St\"{u}tzenkr\"{a}fte von einem FE-Programm gut angen\"{a}hert werden k\"{o}nnen, weil die Einflussfunktion f\"{u}r eine St\"{u}tzenkraft ja dadurch entsteht, dass man die St\"{u}tze wegnimmt und die Platte an dieser Stelle um eine L\"{a}ngeneinheit nach unten dr\"{u}ckt. Die Einflussfunktion ist praktisch die zu Eins normierte Einflussfunktion f\"{u}r die Durchbiegung am Ort der St\"{u}tze, wenn die St\"{u}tze fehlt. Eine solche Biegefl\"{a}che kann man jedoch mit einem FE-Programm gut ann\"{a}hern.

\"{A}hnliches gilt f\"{u}r frei stehende W\"{a}nde, weil die Einflussfunktion f\"{u}r die \"{u}ber die Wandl\"{a}nge integrierten St\"{u}tzkr\"{a}fte dadurch entsteht, dass man die Wand als ganzes um 1 m absenkt. Auch diese Bewegung kann man schon auf einem relativ groben Netz gut darstellen.




%-----------------------------------------------------------------
\begin{figure}[tbp]
\centering
\includegraphics[width=0.9\textwidth]{\Fpath/U235}
\caption{Schw\"{a}chung zweier Riegel und die dadurch ausgel\"{o}sten Kr\"{a}fte und Momente $f_i^+$}
\label{U235}
%
\end{figure}%
%----------------------------------------------------------------- \\

{\textcolor{blau2}{\subsubsection*{Der Zusammenhang mit den Einflussfunktionen}}}

Eine \"{A}nderung der L\"{a}ngssteifigkeit $EA$ in einem Element, f\"{u}hrt, wie wir gesehen haben, zu zwei zus\"{a}tzlichen Knotenkr\"{a}ften $\pm f_i^+$ an den Enden des Stabes. Das ist aber eine \"{a}hnliche Situation, wie bei der Berechnung der Einflussfunktion f\"{u}r die Normalkraft $N(x)$ in dem Stab, wo ja ebenfalls zwei gegengleiche Knotenkr\"{a}fte $\pm EA/l_e$ die Knoten belasten.

Verglichen mit den $\pm EA/l_e$ sind die $f_i^+$ klein, aber das ist ein einfacher Skalenfaktor und so kann man doch an Hand der Einflussfunktion f\"{u}r $N$ einen Eindruck gewinnen, wie weit die $f_i^+$ ausstrahlen.

In analoger Weise kann man bei einem Balken argumentieren. Eine \"{A}nderung von $EI $ f\"{u}hrt zu Zusatzkr\"{a}ften $f_1^+$ und $f_3^+$ und Zusatzmomenten $f_2^+$ und $f_4^+$ kNm, die man in ein symmetrisches und antimetrisches Paar aufspalten kann
und so bekommen sie \"{A}hnlichkeit mit den Knotenkr\"{a}ften $j_i$, die die Einflussfunktionen erzeugen. Die $j_i$ sind ja auch Gleichgewichtslasten.
\\

, sinngem\"{a}{\ss} also mit der Formel (hier in einer etwas symbolischen Form)
\begin{align}\label{Eq71}
w(x) = \int_0^{\,l} G_0(y,x)\,p(y)\,dy + \sum_i (X_i\,G_0'^L(y_i,x) - X_i\,G_0'^R(y_i,x))\,.
\end{align}
$G_0' (= \tan \Np)$ ist die Ableitung der Einflussfunktion, passend zu den Momenten $X_i$. Wenn die $X_i$ andere Gr\"{o}{\ss}en sind, dann muss man nat\"{u}rlich andere Werte von $G_0(y_i,x)$ abgreifen.

Diese Zweiteilung in der Summe ist n\"{o}tig, weil ja das Moment $X_i$ links nicht mit demselben Faktor $G_0'(y_i,x)$ 'weitergeleitet' wird, wie das Moment $X_i$ rechts. Die Einzelkraft $P = 1 $, die die Einflussfunktion f\"{u}r die Durchbiegung im Punkt $x $ erzeugt,  s. Bild \ref{U101}, bewirkt ja unterschiedliche Verdrehungen in dem Gelenk und das bedeutet umgekehrt, dass das Moment $X_i $ links vom Gelenk eine andere Durchbiegung im Aufpunkt $x$ erzeugt, als das gleich gro{\ss}e Moment $X_i $ rechts vom Gelenk. \\

Mit der modifizierten rechten Seite
\begin{align}
\vek K\,\vek u_c = \vek f + \vek f^+
\end{align}
kann man am Modell $\vek K$ die Knotenverformungen $\vek u_c$ des Modells $\vek K_c$ berechnen, aber nur die Knotenverformungen. Um aus $\vek u_c$ die Schnittkr\"{a}fte in den einzelnen Elementen zu berechnen, muss man nat\"{u}rlich die Steifigkeiten $EA_c$ und $EI_c$ ansetzen, die die Elemente in dem Modell $\vek K_c$ haben
\begin{align}
M_h^c(x) = - EI_c\,w_h''(x)\,.
\end{align}
Aber es gilt:

\hspace*{-12pt}\colorbox{hellgrau}{\parbox{0.98\textwidth}{Steifigkeits\"{a}nderungen bei statisch bestimmten Tragwerken bewirken keine Umlagerung der Kr\"{a}fte, sondern nur eine \"{A}nderung der Verformungen.}}\\

%-----------------------------------------------------------------
\begin{figure}[tbp]
\centering
\includegraphics[width=0.9\textwidth]{\Fpath/U99}
\caption{\"{A}nderung der L\"{a}ngssteifigkeit im mittleren Element,  \textbf{ a)} Originalsystem und Belastung,  \textbf{ b)} Einflussfunktion f\"{u}r $u(l)$,  \textbf{ c)} neue und alte L\"{o}sung im Vergleich}
\label{U99}
%
\end{figure}%
%-----------------------------------------------------------------



%%%%%%%%%%%%%%%%%%%%%%%%%%%%%%%%%%%%%%%%%%%%%%%%%%%%%%%%%%%%%%%%%%%%%%%%%%%%%%%%%%%%%%%%%%%%%%%%%%%
\textcolor{blau2}{\section{Elementares Beispiel am Stab}}
Dieselben \"{U}berlegungen gelten sinngem\"{a}{\ss} f\"{u}r \"{A}nderungen in der L\"{a}ngssteifigkeit $EA $ eines Stabes. Um den vorgelegten Stab an das Tragwerk anzuschlie{\ss}en, brauchen wir gegengleiche Knotenkr\"{a}fte $f_i^+$ oder $N_a^+$ und $N_b^+$
\begin{align}
N_a^+ = - N_b^+
\end{align}
und so folgt, dass der Einfluss dieser beiden Kr\"{a}fte
\begin{align}
&N_a^+ \cdot G(y_a,x) + N_b^+ \cdot G(y_b,x) = N_a^+ \cdot (G(y_a,x) - G(y_b,x)) \nn\\
&\simeq N_a^+ \,G'(y_a,x)\,l_e
\end{align}
proportional zur Relativverschiebung der beiden Stabenden unter der Wirkung der Einflussfunktion ist. Auch dieser Unterschied d\"{u}rfte vernachl\"{a}ssigbar sein, wenn der Aufpunkt weit genug weg liegt.\\

\hspace*{-12pt}\colorbox{hellgrau}{\parbox{0.98\textwidth}{\"{A}nderungen der L\"{a}ngssteifigkeit, $EA + \Delta EA $, in einem Element f\"{u}hren zu gegengleichen Zusatzkr\"{a}ften $f_i^+$ (in Achsrichtung) an den Elementenden.}}\\
Eine konstante Streckenlast $p$ zieht an einem Stab, s. Bild \ref{U99} a. Die \"{U}berlagerung der Einflussfunktion $G_0(l,y)$ f\"{u}r $u(l)$ mit der Belastung, s. Bild \ref{U99} b, ergibt
\begin{align}
u(l) = \int_0^{\,l} G_0(y,l)\,p(y)\,dy\,.
\end{align}
Eine Erh\"{o}hung der L\"{a}ngssteifigkeit, $EA_c > EA$, in dem zweiten Element kann durch zwei gegengleiche Knotenkr\"{a}fte $\pm f^+$ kompensiert werden, s. Bild \ref{U99} c, und somit lautet die neue L\"{a}ngsverschiebung am Stabende (berechnet mit der 'alten' Einflussfunktion)
\begin{align}
u_c(l) &= \int_0^{\,l} G_0(y,l)\,p(y)\,dy + f^+ G_0(x_a,l) - f^+ G_0(x_b,l)\nn \\
&= u(l) + f^+ (G_0(x_a,l) - G_0(x_b,l)) = u(l) - f^+ G_0'(x_a,l)\,l_e\,.
\end{align}
Die Differenz $u_c(l) - u(l)$ ist also proportional zu den Faktoren $f^+$, $G_0' = 1/EA$ und der L\"{a}nge $l_e$ des Elements. Die Abh\"{a}ngigkeit von $l_e$ best\"{a}tigt die Vermutung, dass die Effekte umso eher zu vernachl\"{a}ssigen sind, je n\"{a}her die beiden Kr\"{a}fte $f_i^+$ beieinander liegen, aber auch die Steigung $G_0' = 1/EA = 10^{-6}$ ist, wenn wir einen realistischen Wert f\"{u}r $EA$ zu Grunde legen, sehr klein. Eine kleine Steigung $G_0'$ bedeutet ja, dass sich die Werte der Einflussfunktion $G_0$ in den Fusspunkten der beiden gegengleichen Kr\"{a}fte $\pm f^+$ kaum unterscheiden.
%-----------------------------------------------------------------
\begin{figure}[tbp]
\centering
\includegraphics[width=0.7\textwidth]{\Fpath/U111}
\caption{Elementendkr\"{a}fte und Einflussfunktion $g(y)$ auf dem Element}
\label{U111}
%
\end{figure}%
%-----------------------------------------------------------------


Mit den Zahlen
\begin{align}
EA = 1.0 \cdot 10^6 \,\text{kN},\,EA_c = 2 \cdot EA\qquad  l = 3,\,l_e = 1\, \qquad p = 10\,\text{kN}/\text{m}
\end{align}
ergibt sich z.B.
\begin{align}
u_1^c &= 2.52\cdot 10^{-5}\,\text{m},\,  u_2^c = 3.25 \cdot 10^{-5}\,\text{m},\,  u_3^c = 3.74 \cdot 10^{-5}\,\text{m} \\
 f^+ &= \pm (3.25 - 2.52)\cdot 10^{-5}\,\text{m}\cdot 1.0\cdot 10^6\,\text{kN} = \pm 7.5\,\text{kNm}
\end{align}
und die Streckenlast $p$ plus den $f^+$, s. Bild \ref{U99} c, erzeugt an dem Original dieselben Knotenverschiebungen, wie $p$ alleine an dem verst\"{a}rkten Stab.
\\

Bleibt noch das Prinzip der virtuellen Verr\"{u}ckungen als (scheinbare) Alternative zum Satz von Betti, um Kraftgr\"{o}{\ss}en zu berechnen. Aber alle Einflussfunktionen f\"{u}r Kraftgr\"{o}{\ss}en, die auf dem Prinzip der virtuellen Verr\"{u}ckungen basieren, sind im Grunde Anwendung des Satzes von Betti, wie das Bild \ref{U156} demonstriert, wo, wie in Bild \ref{U148}, die Gr\"{o}{\ss}e der Lagerkraft $A$ mit dem Satz von Betti berechnet wird.

Hierzu wiederholt man im Bild \ref{U156} b das urspr\"{u}ngliche System, entfernt aber alle Lager, so dass der gewichtslose Balken frei schwebt. Eine Verdrehung um das rechte Ende
\begin{align}
w_2(x) = 1 - \frac{x}{l}
\end{align}
erfordert also keinerlei Kr\"{a}fte am System 2. Das System 1 ist der Balken links mit der Streckenlast und der Biegelinie $w(x) = w_1(x)$. Die Arbeit der Lasten am System 1 auf den Wegen des Systems 2 ist gem\"{a}{\ss} Betti gleich der Arbeit der (nicht vorhandenen) Lasten am System 2 auf den Wegen $w_1(x)$ und so folgt
\begin{align}
A_{1,2} = - A\cdot 1 + \int_0^{\,l} p(x)\,w_2(x)\,dx = \int_0^{\,l} 0\cdot w_2(x)\,dx = A_{2,1} = 0
\end{align}
oder
\begin{align}
A_{1,2} = \int_0^{\,l} p(x)\,w_2(x)\,dx\,.
\end{align}
Dass die Arbeit $A_{2,1} = 0$ ist, ist kein Zufall, sondern das ist bei der Berechnung von Einflussfunktionen f\"{u}r Kraftgr\"{o}{\ss}en mit dem Satz von Betti immer so---auch bei statisch unbestimmten Tragwerken.\\

Bei einem (statisch unbestimmten) Durchlauftr\"{a}ger braucht man nat\"{u}rlich Kr\"{a}fte, um einen Versatz der Gr\"{o}{\ss}e Eins (= Einflussfunktion f\"{u}r $V(x)$) zu erzeugen, da diese Kr\"{a}fte aber gegengleich sind, ist die Arbeit, die sie auf der Durchbiegung $w_1(x)$ leisten, null
\begin{align}
 A_{2,1} = (V_L(x) - V_R(x)) \cdot w_1(x) = 0\,.
\end{align}

\begin{remark}
Bei Drehfedern
\begin{align}
M = k_{\Np} \cdot \Np
\end{align}
hat man diese Zweideutigkeit, $\Np$ oder $\tan\,\Np$, pur, denn oft wird nicht eindeutig gesagt, welche Einheit die Drehsteifigkeit hat
\begin{align}
k_{\Np} = \left \{ \begin{array}{l } {\displaystyle  \text{{\em Moment\/}}/\text{{\em Drehwinkel\/}}  }      \\
{\displaystyle \text{{\em Moment\/}}/\text{{\em Tangens des Drehwinkels\/}}}\,.
\end{array} \right.
\end{align}
nur wegen $\tan \Np \simeq \Np$ f\"{a}llt das nicht auf. Meist ist es aber der Tangens, insbesondere, wenn es sich um den Einspanngrad handelt.
\end{remark}

%-----------------------------------------------------------------
\begin{figure}
\centering
\if \bild 2 \sidecaption \fi
{\includegraphics[width=0.7\textwidth]{\Fpath/U140}}
\caption{Bestimmung von $u_c$ des Systems $S_c$ am System $S$ mit konstantem $EA$}
\label{U140}%
%
\end{figure}%
%-----------------------------------------------------------------

%%%%%%%%%%%%%%%%%%%%%%%%%%%%%%%%%%%%%%%%%%%%%%%%%%%%%%%%%%%%%%%%%%%%%%%%%%%%%%%%%%%%%%%%%%%%%%%%%%%
\textcolor{blau2}{\subsection{Direkte Berechnung als weitere M\"{o}glichkeit}}
{\em Das geht so nicht!\/}

Eine weitere Variante, wie man sich die Dinge zurecht legen kann, ist in Bild \ref{U140} dargestellt, wo ein Stab mit zwei unterschiedlichen Steifigkeiten, $EA_1 = 1$ und $EA_2 = 2$, gezogen wird (System $S_c$).

Man l\"{o}st den Lastfall erst am System $EA_1 = EA_2 = 1$ (System $S$) und geht mit der L\"{o}sung $u$ in das System $S_c$. Das ergibt die Kr\"{a}fte in Bild \ref{U140} c, die aber nicht den Lastfall darstellen, den man l\"{o}sen wollte und die auch nicht im Gleichgewicht sind. Um das zu korrigieren l\"{o}st man am System $S$  einen Zusatzlastfall, s. Bild \ref{U140} e, und addiert dessen L\"{o}sung $u_+$ zur L\"{o}sung $u$, und am Ende hat man $u_c = u + u_+$.

%----------------------------------------------------------------------------
\begin{figure}
\centering
{\includegraphics[width=1.0\textwidth]{\Fpath/U257}}
  \caption{Einflussfunktion f\"{u}r die Eckverschiebung, \textbf{a)} Verformung der Scheibe, \textbf{b)} die Knotenverschiebungen $\vek g_i$ der Einflussfunktion, Knotenkr\"{a}fte, die in Richtung der $\vek g_i$ wirken, haben maximalen Einfluss und Kr\"{a}fte, die senkrecht auf den $\vek g_i$ stehen keinen Einfluss} \label{U257}
\end{figure}\\


%----------------------------------------------------------------------------
\begin{figure}
\centering
{\includegraphics[width=1.0\textwidth]{\Fpath/U79}}
  \caption{Plot der Knotenvektoren  $\vek g_i$ des Funktionals $J(u_h) = \sigma_{xx}$ einer punktgelagerten Scheibe. Wenn die Knotenkr\"{a}fte in Richtung der Pfeile weisen, erzielen sie die gr\"{o}{\ss}tm\"{o}glichen Wirkung hinsichtlich der Spannung $\sigma_{xx}$ im Aufpunkt. Links oberhalb des Aufpunktes liegt anscheinend ein Lagrange-Punkt}
  \label{U79}
\end{figure}

%----------------------------------------------------------------------------------------------------------
\begin{figure}[tbp]
\centering
\if \bild 2 \sidecaption \fi
\includegraphics[width=.6\textwidth]{\Fpath/U107}
\caption{Starrer Stempel auf Halbraum. An den Kanten des Stempels werden die Spannungen
unendlich gro{\ss}, weil dort die Verzerrungen im Boden unendlich gro{\ss} sind} \label{U107}
\end{figure}%%
%----------------------------------------------------------------------------------------------------------
\\

, wie etwa im Fall des B\"{u}rogeb\"{a}udes in Bild \ref{U203}, wo die Energiebilanz
\begin{align}
\frac{1}{2}\,P \cdot \Delta u = \frac{1}{2}\,\sum_i\, [ \int_0^{\,l_i} (\frac{M_i^2}{EI}+ \frac{N_i^2}{EA})\,dx \,]
\end{align}
sich nur durchhalten l\"{a}sst, wenn die $M_i$ und $N_i$ gegen Null tendieren, wenn die Zahl der Stiele und Riegel w\"{a}chst. (?) Richtig ?

%---------------------------------------------------------------------------------
\begin{figure}
\centering
{\includegraphics[width=.7\textwidth]{\Fpath/U203}}
\caption{Einzelkraft an Geb\"{a}udeecke (Stockwerkrahmen) }
\label{U203}%
%
\end{figure}%
%---------------------------------------------------------------------------------\\

\subsection{Anschauliche  Herleitung der Gleichungen}

Die Situation: In dem Dreifeldtr\"{a}ger in Bild \eqref{VerySimple1} a \"{a}ndert sich die Steifigkeit, $EI \to EI + \Delta EI$ in der Mitte des zweiten Feldes auf einem Teilst\"{u}ck $[x_a, x_b]$. Wir wollen voraussagen, welche \"{A}nderungen sich daraus f\"{u}r die Durchbiegung am Kragarmende ergibt.\\

Zun\"{a}chst sei daran erinnert, dass man am urspr\"{u}nglichen Tragwerk die Durchbiegung am Kragarmende wie folgt berechnen kann: Man bringt zun\"{a}chst eine Kraft $P = 1$ in Richtung der gesuchten Durchbiegung auf. Die zugeh\"{o}rige Biegelinie $G$ (= Greensche Funktion) ist die Einflussfunktion f\"{u}r die Durchbiegung am Kragarmende, d.h. die \"{U}berlagerung von $G$ mit der Streckenlast $p$ ergibt die Durchbiegung
\bfo
w(l) = \int_0^{\,l} G\,p\,dx \,.
\efo
Nach dem Satz von Mohr kann man aber auch statt dessen die Momente $M$ aus der Belastung und $M_G$ aus der Einflussfunktion miteinander \"{u}berlagern
\bfo
w(l) = \int_0^{\,l} \frac{M\,M_G}{EI}\,dx = \int_0^{\,l} EI\,w''\,G''\,dx\,.
\efo
Dies, wie gesagt, nur zur Erinnerung. \\

Der Index $c$ (= {\em changed\/}) bezeichne im Folgenden die Gr\"{o}{\ss}en, die sich auf das ver\"{a}nderte Tragwerk beziehen.\\

Wir benutzen die folgende Logik:\\
\begin{enumerate}
  \item Ist ein Tragwerk im Gleichgewicht, dann sind bei jeder virtuellen Verr\"{u}ckung die virtuellen \"{a}u{\ss}eren Arbeiten gleich den virtuellen inneren Arbeiten
\bfo
\delta A_a = \delta A_i\,.
\efo
  \item Dies gilt auch f\"{u}r das modifizierte Tragwerk
\bfo
\delta A_a^c = \delta A_i^c\,.
\efo
  \item Nachdem sich aber die Belastung nicht \"{a}ndert, m\"{u}ssen bei gleicher virtueller Verr\"{u}ckung der beiden Tragwerke die virtuellen \"{a}u{\ss}eren Arbeiten gleich gro{\ss} sein
\bfo
\delta A_a = \delta A_a^c\,,
\efo
und wegen des Prinzips der virtuellen Verr\"{u}ckungen, m\"{u}ssen daher auch die virtuellen inneren Arbeiten gleich gro{\ss} sein
\bfo
\delta A_i = \delta A_a = \delta A_a^c = \delta A_i\,.
\efo
\end{enumerate}
Der Anschaulichkeit halber wollen wir $\delta A_a$ und $\delta A_i$ mit 'Argumenten' schreiben, also
\bfo
\delta A_a(w,G)  \qquad \delta A_i(w,G)
\efo
Vor dem Komma steht die Biegelinie des Systems und hinter dem Komma die virtuelle Verr\"{u}ckung.
%----------------------------------------------------------------------------------------------------------
\begin{figure}[tbp]
\includegraphics[width=1.0\textwidth]{\Fpath/VERYSIMPLE1}
\caption{Berechnung der \"{A}nderung der Durchbiegung am Kragarmende infolge einer Steifigkeits\"{a}nderung im zweiten Feld.} \label{VerySimple1}
\end{figure}%
%----------------------------------------------------------------------------------------------------------

Die oben eingef\"{u}hrte Biegelinie $G$ aus der Einzelkraft \"{u}bernimmt nun die Rolle der virtuellen Verr\"{u}ckung. 'Wackelt' man also mit der Greenschen Funktion $G$ an dem urspr\"{u}nglichen System dann gilt
\bfo
\delta A_i(w,G) = \int_0^{\,l} EI\,G''\,w''\,dx = \int_0^{\,l} G\,p\,dx = \delta A_a(w,G)
\efo
und am modifizierten System bei derselben virtuellen Verr\"{u}ckung
\bfo
\delta A_i^c(w_c,G) = \int_0^{\,l} EI_c\,G''\,w_c''\,dx = \int_0^{\,l} G\,p\,dx = \delta A_a^c(w_c,G)\,.
\efo
Nun ist die Biegesteifigkeit $EI_c$ am modifizierten System bis auf den Abschnitt $[x_a,x_b]$ mit dem urspr\"{u}nglichen $EI$ identisch und somit folgt
\bfo
\delta A_i^c(w_c,G) = \int_0^{\,l} EI_c\,G''\,w_c''\,dx = \int_0^{\,l} EI\,G''\,w_c''\,dx + \int_{x_a}^{\,x_b} \Delta EI\,G''\,w_c''\,dx\,.
\efo
Das erste Integral auf der rechten Seite ist gerade $w_c(l)$, s. Anhang, und damit folgt
\bfo
\delta A_i^c(w_c,G) = w_c(l) + \int_{x_a}^{\,x_b} \Delta EI\,G''\,w_c''\,dx
\efo
und wegen
\begin{align}
\delta A_i^c(w_c,G) &=  w_c(l) + \int_{x_a}^{\,x_b} \Delta EI\,G''\,w_c''\,dx \nn \\
&= \delta A_a^c(w_c,G) = \delta A_a(w,G) = \delta A_i(w,G) = w(l)\nn
\end{align}
schlie{\ss}lich das zentrale Ergebnis
\bfo
w_c(l) - w(l) = - \int_{x_a}^{\,x_b} \Delta EI\,G''\,w_c''\,dx\,.
\efo
Nun soll noch eine Vereinfachung vorgenommen werden. Um diese Formel auszuwerten, muss man die beiden Funktionen $G$ und $w_c$ kennen. Die Biegelinie $w_c$ muss an dem modifizierten Tragwerk berechnet werden. Wenn man aber die Gleichungen f\"{u}r das modifizierte Tragwerk aufstellt, dann braucht man diese Formel aber nicht mehr, denn dann erh\"{a}lt man alle interessierenden Ergebnisse automatisch.

Wir ersetzen daher $w_c$ im Abschnitt $[x_a, x_b]$ durch die urspr\"{u}ngliche Biegelinie $w$ und kommen so zu der N\"{a}herungsformel
\bfo
w_c(l) - w(l) \simeq - \int_{x_a}^{\,x_b} \Delta EI\,G''\,w''\,dx\,.
\efo
Setzt man noch
\bfo
M =  - EI\,w'' \qquad M_G = - EI\,G''\,,
\efo
so erh\"{a}lt man schlie{\ss}lich das Ergebnis
\bfo
w_c(l) - w(l) \simeq - \frac{\Delta EI}{EI}\int_{x_a}^{\,x_b} \frac{M\,M_G}{EI}\,dx\,.
\efo

{\bf Anhang\/}\\

'Wackelt' man mit der Biegelinie $w_c$ am System mit der Einzellast $P = 1$, dann muss dabei die \"{a}u{\ss}ere virtuelle Arbeit $\delta A_a(G,w_c) = P \times w_c(l)$ gleich der inneren virtuellen Arbeit sein, also
\bfo
\delta A_a(G,w_c) = 1 \times  w_c(l) = \int_0^{\,l} EI\,G''\,w_c''\,dx = \delta A_i(G,w_c)\,.
\efo

Was wir hier am Durchlauftr\"{a}ger erl\"{a}utert haben, gilt sinngem\"{a}{\ss} f\"{u}r alle Tragwerke.
Drei Faktoren bestimmen, welche Auswirkung eine \"{A}nderung $\Delta EI$ der Steifigkeit $EI$ auf eine Gr\"{o}{\ss}e $O$ hat
\bfo
O_c - O = \frac{\Delta EI}{EI}\int_{x_a}^{\,x_b} \frac{M \, M_G}{EI} \,dx = \frac{\Delta EI}{EI} \cdot \delta A_i\,.
\efo
Man \"{u}berlagert eine {\em  Wichtungsfunktion\/}, n\"{a}mlich $M_G$, mit dem Moment $M$ aus der Belastung und wichtet das Ganze dann noch einmal mit dem Faktor $\Delta EI/EI$, der ein Ma{\ss} f\"{u}r die relative \"{A}nderung der Steifigkeit ist.

Noch allgemeiner gesagt ist die Formeln von der Bauart
\bfo
O_c - O = \mbox{rel. Steifigkeits\"{a}nderung} \times \mbox{virt. innere Energie}\,.
\efo

%%%%%%%%%%%%%%%%%%%%%%%%%%%%%%%%%%%%%%%%%%%%%%%%%%%%%%%%%%%%%%%%%%%%%%%%%%%%%%%%%%%%%%%%%%%%%%%%%%%
\textcolor{blau2}{\section{Sensitivit\"{a}tsanalyse}}
Es gibt nun noch ein nah verwandtes Thema, bei dem die Berechnung von Einflussfunktionen nach {\em Mohr\/} eine gro{\ss}e Rolle spielt. In einem Rahmen \"{a}ndere sich die Biegesteifigkeit $EI$ in einem Stiel oder Riegel, $EI \to EI + \Delta EI$, dann kann man zeigen, dass die \"{A}nderung einer interessierenden Weg- oder Schnittgr\"{o}{\ss}e, z.B. des Moments $M + \Delta M$ an einer Stelle $x$, gleich dem Integral
\begin{align}
\Delta M(x) = \frac{\Delta EI}{EI} \int_0^{\,l} \frac{\bar{M}\,M_c}{EI}\,dx
\end{align}
ist. Hier ist $\bar{M}$ das Moment, das zur Einflussfunktion geh\"{o}rt und $M_c$ ist das Lastmoment in dem betroffenen Bauteil---nach der Steifigkeits\"{a}nderung---und es gilt:\\

\hspace*{-12pt}\colorbox{hellgrau}{\parbox{0.98\textwidth}{Bei der Auswertung wird nur \"{u}ber das betroffene Element integriert!}}\\

Was bedeutet das? Angenommen im 3. Stock reduziert sich in einem Riegel die Biegesteifigkeit um die H\"{a}lfte, $\Delta EI = 0.5 \cdot EI$. Welche Auswirkungen hat das auf das Biegemoment $M(x)$ in einem Riegel im 5. Stock? Theoretisch geht man wie folgt vor:
\begin{enumerate}
  \item Man ermittelt das Biegemoment aus der Last in dem Riegel im 3. Stock. Das ist der Verlauf $M_c(x)$. Der Index $c$ soll darauf hindeuten, dass es das Moment {\em nach\/} der Reduktion der Steifigkeit in dem Riegel ist.
  \item Man stellt die Einflussfunktion f\"{u}r das Biegemoment $M(x)$ im 5. Stock auf, $x$ sei die Mitte des interessierenden Riegels, und bestimmt den Verlauf, den die Momente $\bar{M}_M(x)$ aus dieser Einflussfunktion in dem Riegel im 3. Stock haben.
  \item Im letzten Schritt \"{u}berlagert man diese beiden Momente
\begin{align}
\Delta M(x) = 0.5 \int_{Riegel\,\,im\,\,3.\,\,Stock} \frac{\bar{M}_M\,M_c}{EI}\,dx\,.
\end{align}
   Das Ergebnis ist die \"{A}nderung $\Delta M(x)$ des Biegemomentes im Riegel im 5. Stock.
\end{enumerate}

Wenn man wissen will, wie sich die Querkraft $V(x)$ in der Riegelmitte im 5. Stock \"{a}ndert, dann bestimmt man die Einflussfunktion f\"{u}r $V(x)$ und \"{u}berlagert deren Moment $\bar{M}_V(x)$ im Riegel im 3. Stock mit dem Lastmoment
\begin{align}
\Delta V(x) = 0.5 \int_{Riegel\,\,im\,\,3.\,\,Stock} \frac{\bar{M}_V\,M_c}{EI}\,dx\,,
\end{align}
und so durch alle interessierenden Gr\"{o}{\ss}en.

Wenn in einem betroffenen Bauteil $\bar{M}$ und $M_c$ beide gro{\ss} sind, dann haben \"{A}nderungen $EI \to EI + \Delta EI$ sp\"{u}rbaren Einfluss, wenn dagegen eine oder beide Gr\"{o}{\ss}en klein sind, dann ist der Einfluss eher vernachl\"{a}ssigbar.

Das ganze ist nat\"{u}rlich theoretisch, weil man das Biegemoment $M_c(x)$ in dem Riegel im 3. Stock nach der Steifigkeits\"{a}nderung braucht und um das zu bestimmen, muss man den Rahmen mit der ge\"{a}nderten Steifigkeit, $EI \to 0.5 \cdot EI$ im Riegel im 3. Stock, neu durchrechnen. Dann wei{\ss} man aber nat\"{u}rlich auch, wie gro{\ss} die \"{A}nderung im 5. Stock ist.

Um daraus ein praktisches Werkzeug zu machen, muss man den Verlauf von $M_c(x)$ durch eine Sch\"{a}tzung ersetzen
\begin{align}
M_c(x) \simeq \alpha \cdot M(x)\,,
\end{align}
wobei die Wahl des Parameters $\alpha$ dem Geschick und der Erfahrung des Ingenieurs \"{u}berlassen bleibt
\begin{align}
\Delta M(x) \simeq \frac{\Delta EI}{EI} \int_0^{\,l} \frac{\bar{M}\,\alpha @ M}{EI}\,dx \qquad \alpha @ M \simeq M_c\,.
\end{align}
F\"{u}r Hinweise darauf, wie man den Faktor $\alpha$ geeignet absch\"{a}tzt, s. \cite{Ha5} Kapitel 3.8 {\em sensitivity analysis\/}.

Bei dem Rahmen in Bild   \ref{U186} soll der Einfluss einer Steifigkeits\"{a}nderung in einem der Riegel und Stiele des Rahmens auf das Fusspunktsmoment im linken Stiel \"{u}berschl\"{a}gig erfasst werden.


In Bild \ref{U186} a sieht man die Momente $M$ aus dem Wind und in Bild \ref{U186} b die Momente $\bar{M}$ der Einflussfunktion f\"{u}r das Fu{\ss}punktsmoment. In Gedanken kann man nun von Bauteil zu Bauteil gehen und so abw\"{a}gen, welchen Effekt eine \"{A}nderung $EI \to EI + \Delta EI$ in dem Stiel oder Riegel auf das Fu{\ss}punktsmoment haben wird. Es sind vor allem, wie man sieht, die beiden unteren Stiele, deren Steifigkeiten $EI$ am wichtigsten f\"{u}r die Gr\"{o}{\ss}e des Fu{\ss}punktsmoments sind.



%----------------------------------------------------------------------------------------------------------
\begin{figure}[tbp]
\includegraphics[width=1.0\textwidth]{\Fpath/VERYSIMPLE2}
\caption{Windlast auf einen Rahmen, {\bf a} Momente aus $p$, {\bf b}
Einflussfunktion f\"{u}r das Moment im Fusspunkt des rechten Stiels, {\bf c}
Momente aus der Einflussfunktion} \label{VerySimple2}
\end{figure}%
%----------------------------------------------------------------------------------------------------------

\begin{remark}
Diese 'Schaukellogik'\index{Schaukellogik} macht auch deutlich, dass der naive Gleichgewichtsbegriff, der von der Gleichheit der Kr\"{a}fte ausgeht, die links und rechts am Seil ziehen, s. Bild \ref{U220}, nur ein Spezialfall eines allgemeineren und weiter reichenden Gleichgewichtsbegriffs ist: Ein Getriebe, eine 'Schaukel' an der nicht notwendig gleich gro{\ss}e Kr\"{a}fte, $\sum H \neq 0, \sum V \neq 0$, angreifen, ist im Gleichgewicht, wenn die Kr\"{a}fte bei einer virtuellen Verr\"{u}ckung die gleiche Arbeit leisten. Dem \"{U}bersetzungsverh\"{a}ltnis des 'Getriebes', man denke an einen Flaschenzug, kommt also eine ma{\ss}gebende Rolle.

Weil man jede Scheibe durch Entfernen der Lager zu einem 'Ein-St\"{u}ck-Getriebe' machen kann, gilt das auch f\"{u}r starre K\"{o}rper. Nur ist es so, dass es bei diesen nur die Starrk\"{o}rperbewegungen $\vek u_0 = \vek a + \vek b \times \vek x$ als m\"{o}gliche Bewegungen gibt, woraus dann die bekannten Gleichgewichtsbedingungen  $\sum H = 0, \sum V = 0, \sum M = 0$ folgen.
\end{remark}


Es gibt aber noch eine weitere Methode der Mittelbildung und zwar das Umrechnen der Resultate in \"{a}quivalente Knotenkr\"{a}fte, wie es die FE-Programme in Punktlagern machen. Das ist so \"{a}hnlich wie beim Gl\"{a}tten einer Funktion $f(x) \to \bar{f}(x)$ mittels einer Gewichtsfunktion $\Np(x)$
\begin{align}
\bar{f}(x) = \int_0^{\,l} f(x)\,\Np(x)\,dx\,.
\end{align}
Ein Punktlager wird von einem FE-Programm ja nicht als Punktlager gerechnet, in dem Sinne, dass dort eine echte Einzelkraft die Scheibe st\"{u}tzt, sondern die Haltekr\"{a}fte sind vielmehr Fl\"{a}chen- und Linienkr\"{a}fte (letztere zwischen den Elementen), die prim\"{a}r \"{u}ber die Elemente verteilt sind, die den Lagerknoten enthalten. Aus der Ferne m\"{o}gen diese Kr\"{a}fte den Eindruck einer einzelnen Haltekraft vermitteln, aber statisch gelingt die Umwandlung in \"{a}quivalente Knotenkr\"{a}fte (horizontal und vertikal) nur, wenn man den Knoten um eine L\"{a}ngeneinheit in die entsprechende Richtung verr\"{u}ckt und die Arbeit z\"{a}hlt, die die Haltekr\"{a}fte dabei leisten.\\

Jetzt kann man fragen, wozu braucht man dann noch die Greenschen Identit\"{a}ten? Nun die Identit\"{a}ten bilden die Grundlage der Statik der Kontinua. An den Identit\"{a}ten kann man ablesen, was als \"{a}u{\ss}ere Arbeit z\"{a}hlt, ablesen, dass $M$ und $\tan \Np$ zueinander konjugiert oder dass die Querkraft $ V = - EI\,w'''$ lautet, und dass sie zu $w$ konjugiert ist, etc. Der Ingenieur macht das automatisch richtig, weil man ihm das so beigebracht hat, aber die Regeln daf\"{u}r, die stehen in den Greenschen Identit\"{a}ten.
\\

%%%%%%%%%%%%%%%%%%%%%%%%%%%%%%%%%%%%%%%%%%%%%%%%%%%%%%%%%%%%%%%%%%%%%%%%%%%%%%%%%%%%%%%%%%%%%%%%%%%
{\textcolor{blau2}{\section{Die Elemente $k_{ij}$ einer Steifigkeitsmatrix}}}\index{Elemente einer Steifigkeitsmatrix}
Wir wollen uns kurz die beiden Bedeutungen klar machen, die das Elemente $k_{ij}$ einer Steifigkeitsmatrix haben kann. Wegen $\delta A_i = \delta A_a$ kann $k_{ij}$ als (virtuelle) innere Arbeit gelesen werden oder als (virtuelle) \"{a}u{\ss}ere Arbeit.

%---------------------------------------------------------------------------------
\begin{figure}
\centering
{\includegraphics[width=0.8\textwidth]{\Fpath/U70}}
  \caption{Stab, Situation in der N\"{a}he des Knotens $i$, \textbf{ a)} Einheitsverformungen (L\"{a}ngsverschiebungen, hier als Funktionen nach oben abgetragen)
  \textbf{ b)} Kr\"{a}fte, die die Einheitsverformung $\Np_i(x)$ bewirken}
  \label{U70}
%
\end{figure}
%---------------------------------------------------------------------------------
Beginnen wir mit der inneren Energie. Das Element $k_{ij}$ ist zun\"{a}chst definiert als die Wechselwirkungsenergie zwischen den beiden Einheitsverformungen $\Np_i(x)$ und $\Np_j(x)$, s. Bild \ref{U70},
\begin{align}
k_{ij} = a(\Np_i,\Np_j) = \int_0^{\,l} EA\,\Np_i'(x)\,\Np_j'(x)\,dx \qquad \text{(Stab)}\,.
\end{align}
F\"{u}r das weitere nehmen wir nun die erste Greensche Identit\"{a}t zwischen $\Np_i(x)$ und $\Np_j(x)$ zu Hilfe und beachten, dass wir die Formulierung in Teilen vornehmen m\"{u}ssen, weil in dem Knoten  $x_i$  eine Einzelkraft $A$ angreift. Das ist die Kraft, die die Einheitsverschiebung $u_i = 1$ des Knotens bewirkt. Ferner wissen wir, dass die $\Np_i(x)$ homogene L\"{o}sungen der Stabgleichung sind, $- EA\,\Np_i'' = 0$. Es ergibt sich somit
\begin{align}
\text{\normalfont\calligra G\,\,}(\Np_i,\Np_j) &= \text{\normalfont\calligra G\,\,}(\Np_i,\Np_j)_{(0,x_i)} +
\text{\normalfont\calligra G\,\,}(\Np_i,\Np_j)_{(x_i,l)} \nn \\
&= \underbrace{B \cdot\Np_j(x_{i-1}) + A\cdot \Np_j(x_i) + C\cdot\Np_j(x_{i+1})}_{\delta A_a} -  k_{ij} = 0\,.
\end{align}
Die innere Arbeit $k_{ij}$ ist also gleich der \"{a}u{\ss}eren Arbeit, die die Kr\"{a}fte $A, B, C$ auf den Knotenbewegungen
der Verr\"{u}ckung $\Np_j(x)$ leisten.

Insbesondere ist
\begin{alignat}{3}
k_{i,\,i-1} &= B \cdot \Np_{i-1}(x_{i-1}) &&= B \cdot 1 = - \frac{EA}{l_e} \\
k_{i\,i} &= A \cdot \Np_{i}(x_{i}) &&= A \cdot 1 = 2\, \frac{EA}{l_e} \\
k_{i,\,i+1} &= C \cdot \Np_{i+1}(x_{i+1}) &&= C \cdot 1 = - \frac{EA}{l_e}\,.
\end{alignat}
Alle anderen $k_{ij}$ in der Zeile $i$ sind Null, weil die weiter abliegenden Funktionen $\Np_j(x)$ nicht dazu kommen, die drei Kr\"{a}fte $A, B, C$ zu verschieben.

Diese Resultate kann man nun wie folgt zusammenfassen:

\begin{itemize}
\item $k_{ij}$ ist eine innere Energie
\begin{align}
\delta A_i = \int_0^{\,l} EA\,\Np_i'(x)\,\Np_j'(x)\,dx = \int_0^{\,l} N_i(x)\,d\,\Np_j\,.
\end{align}
Es ist die Arbeit, die die Normalkraft $N_i(x) = EA\,\Np_i'(x)$ auf den Wegen $d\,\Np_j$ leistet.

\item Wegen $\delta A_i = \delta A_a$ ist $k_{ij}$ aber auch gleich der virtuellen \"{a}u{\ss}eren Arbeit, die die Knotenkr\"{a}fte $A, B, C$ bei der Verschiebung $\Np_j$ der Knoten leisten.
\item Die Elemente $k_{ii}$ auf der Hauptdiagonalen sind die Kr\"{a}fte, die man f\"{u}r
\begin{align}
u_i = 1 \qquad u_j = 0 \qquad j \neq i\,.
\end{align}
braucht.
\item Die Elemente $k_{ij}$ (in derselben Zeile) sind die 'Haltekr\"{a}fte', also die Kr\"{a}fte, die n\"{o}tig sind, um die durch $u_i = 1$ ausgel\"{o}ste Bewegung an den n\"{a}chsten Knoten zum Stillstand zu bringen, $u_j = 0$.
\end{itemize}
\begin{remark}
Bei 'echten' finiten Elementen, also bei $2-D$ und $3-D$ Problemen sind die treibenden und die haltenden Kr\"{a}fte keine Einzelkr\"{a}fte, sondern 'Wolken' von Fl\"{a}chenkr\"{a}ften und Kantenkr\"{a}ften auf den Kanten zwischen dem Element, auf dem der Knoten $\vek x_i$ liegt und den Nachbarelementen.

Wir werden weiter unten im Text die treibenden und haltenden Kr\"{a}fte, die zur Einheitsverformung $\Np_i(x)$ geh\"{o}ren {\em shape forces\/} nennen und mit $\vek p_i$ bezeichnen. Das k\"{o}nnen Fl\"{a}chenkr\"{a}fte, Linienkr\"{a}fte,  Einzelkr\"{a}fte etc. sein, alles kann in $\vek p_i$ vertreten sein. Der Ausdruck
\begin{align}
f_{ij} = \delta A_a(\vek p_i,\vek \Np_j)
\end{align}
soll dann die Arbeit sein, die die Kr\"{a}fte $\vek p_i$ auf den Wegen $\vek \Np_j$ leisten. Es ist nat\"{u}rlich $k_{ij} = f_{ij}$ gem\"{a}{\ss} dem Motto 'innen = au{\ss}en'. Die $f_j$ sind hier doppelt-indiziert, weil sie aus verschiedenen Lastf\"{a}llen $\vek p_i$ kommen.
In dieser Notation gilt f\"{u}r die Kr\"{a}fte $A, B, C$ in Bild \ref{U70}
\begin{align}
A &= f_{ii} = \delta A_a(\vek p_i,\vek \Np_i) \quad B = f_{i,i-1} =\delta A_a(\vek p_i,\vek \Np_{i-1})\nn \\
C &= f_{i,i-1} =\delta A_a(\vek p_i,\vek \Np_{i+1})\,.
\end{align}
  \end{remark}

%%%%%%%%%%%%%%%%%%%%%%%%%%%%%%%%%%%%%%%%%%%%%%%%%%%%%%%%%%%%%%%%%%%%%%%%%%%%%%%%%%%%%%%%%%%%%%%%%%%
{\textcolor{blau2}{\section{Reduktion in die Knoten}}}\index{Reduktion in die Knoten}
Die Knoteneinheitsverformungen $\Np_i$ bestimmen die \"{a}quivalenten Knotenkr\"{a}fte $f_i$
\begin{align} \label{Eq59}
f_i = \int_0^{\,l} p(x)\,\Np_i(x)\,dx\,.
\end{align}
Mit den $\Np_i$ kann man also die Belastung, wie man sagt, in die Knoten reduzieren.

Insbesondere erh\"{a}lt man also die Festhaltekr\"{a}fte $\times (-1)$ am beidseitig eingespannten Balkenelement durch \"{U}berlagerung der Belastung mit den Elementeinheitsverformungen $\Np_i^e(x)$, s. (\ref{Eq219}),
\begin{align}\label{Eq60}
f_i^e = \int_0^{\,l} p(x)\,\Np_i^e(x)\,dx\,.
\end{align}
 Die so berechneten $f_i^e$ sind die 'aktiven', die 'treibenden' Knotenkr\"{a}fte, w\"{a}hrend ja die Festhaltekr\"{a}fte, wie ihr Name schon sagt, diesen Kr\"{a}ften das Gleichgewicht halten und deswegen der Faktor $(-1)$.

 Die $f_i$ in (\ref{Eq59}) (ohne den Index $e$) enthalten die Beitr\"{a}ge von dem Element links und rechts von dem Knoten, weil sich ja die $\Np_i(x)$ \"{u}ber
 die ganze Umgebung des Knotens erstrecken, w\"{a}hrend die $f_i^e$ nur auf ein Element schauen, deswegen der Index $e$. Die $f_i^e$ findet man in den Handb\"{u}chern f\"{u}r die gebr\"{a}uchlichsten Lastf\"{a}lle tabelliert, aber im Grunde deckt (\ref{Eq60}) die ganze Vielfalt an m\"{o}glichen Lastf\"{a}llen ab.

 Wirkt eine Einzelkraft $P$ bzw. ein Moment $M$ in einem Punkt $x$, dann erh\"{a}lt man die zugeh\"{o}rigen $f_i^e$ einfach durch Auswertung im Punkt
 \begin{align}
 f_i^e = \Np_i^e(x) \cdot P \qquad  f_i^e = {\Np_i^e} '(x) \cdot M\,.
 \end{align}

\begin{align}
\vek K^{(-1)}\,\vek \Delta \vek K = \left[ \barr {r @{\hspace{2mm}}r }
      2 & -1  \\
      -1 & 2
     \earr \right]\left \,\left[ \barr {r @{\hspace{2mm}}r }
      1 & -1  \\
      -1 & 1
     \earr \right] = \left[ \barr {r @{\hspace{2mm}}r }
      0.33 & -0.33  \\
      -0.33 & 0.33
     \earr \right]
\end{align}

%----------------------------------------------------------------------------------------------------------
\begin{figure}[tbp]
\centering
\if \bild 2 \sidecaption \fi
\includegraphics[width=0.7\textwidth]{\Fpath/U267}
\caption{Moment einer Kraft um einen Punkt, $|\vek M|$ = Arbeit von $\vek P$ bei einer (Pseudo)-Drehung} \label{U267}
%
\end{figure}%
%----------------------------------------------------------------------------------------------------------

\begin{remark}
Das Operieren mit Pseudodrehungen ist keine Eigenheit der Stabstatik, sondern sie steckt schon von Anfang an in der Mechanik. Das Moment $\vek M = \vek r \times \vek P$ einer Kraft $\vek P$ um einen Punkt $\vek x$ ist dem Betrage nach, $
|\vek M| =  | \vek r | \cdot | \vek P| \cdot \sin\,\Np
$,
gleich der Arbeit, die die Kraft $\vek P$ bei einer Pseudodrehung leistet, s. Bild \ref{U267}.
\end{remark}

%---------------------------------------------------------------------------------
\begin{figure}
\centering
{\includegraphics[width=.7\textwidth]{\Fpath/U157}}
\caption{Einflussfunktion f\"{u}r die Querkraft $V$ in einem Seil. Auf der Strecke zwischen den beiden Kr\"{a}ften ist die Energie sehr gro{\ss}. In der Grenze, $\Delta x \to 0$, bleibt davon nichts \"{u}brig, weil alle Energie dazu benutzt wird, dass Seil auseinander zu rei{\ss}en (Versatz in der Biegelinie). Die Vorspannkraft $H$ spielt die Rolle der Steifigkeit beim Seil }
\label{U157}%
%
\end{figure}%
%---------------------------------------------------------------------------------

Wie ist das nun bei Einflussfunktionen? Bei Einflussfunktionen f\"{u}r Weggr\"{o}{\ss}en sind die 'treibenden Kr\"{a}fte' Einzelkr\"{a}fte $P = 1$ oder Momente $M = 1$. Es ist nicht viel Energie, die in das Tragwerk flie{\ss}t. Aber bei Einflussfunktionen f\"{u}r Schnittgr\"{o}{\ss}en scheint das anders. Die Kr\"{a}fte und Momente, die einen Knick oder Verschiebungssprung hervorrufen, sind unendlich gro{\ss}, aber trotzdem ist die Energie in dem Tragwerk endlich.

Das R\"{a}tsel l\"{o}st sich, wenn man sich das Bild \ref{U157} anschaut. Die unendlich gro{\ss}e Energie der Kr\"{a}fte $1/\Delta x$ steckt am Schluss in dem Versatz $[[w]] = 1$, ist sozusagen 'eingefroren', w\"{a}hrend die restliche Biegelinie eine endliche Energie hat
\begin{align}
A_i = \frac{1}{2}\,\int_0^{\,x} H\,(w')^2\,dx + \frac{1}{2}\,\int_x^{\,l} H\,(w')^2\,dx\,.
\end{align}

Das waren nun alles Ableitungen des Kerns $G_0(y,x)$ nach $x$. Es gibt aber auch Ableitungen des Kerns $G_0(y,x)$ nach $y$ und zwar, wenn
z.B. ein Einzelmoment $M$ angreift. Dann muss man die Ableitung von $G_0(y,x)$ nach $y$ mit $M$ multiplizieren
\begin{align}
w(x) = \frac{d}{dy}\,G_0(y,x)\,M\,,
\end{align}
denn ein Moment $M$ gleicht zwei dicht nebeneinander stehenden Einzelkr\"{a}ften
\begin{align}
\pm P = \frac{1}{\Delta y} \cdot M\,,
\end{align}
deren Abstand $\Delta y$ gegen Null geht
\begin{align}
w(x) = \lim_{\Delta y \to 0}\, (G_0(y + 0.5\,\Delta y,x)  -  G_0(y - 0.5\,\Delta y,x)) \cdot \frac{1}{\Delta y} \cdot M\,.
\end{align}\\

Man kann es auch so verstehen: Die klassische Statik kennt Einflussfunktionen f\"{u}r Lagerkr\"{a}fte oder Einspannmomente bei Stabtragwerken, aber sie kennt keine Einflussfunktionen f\"{u}r \"{a}quivalente Knotenkr\"{a}fte, weil dieser Begriff in der klassischen Statik direkt nicht vorkommt und solche Knotenkr\"{a}fte nur 'Rechengr\"{o}{\ss}en' sind.
Zur Illustration dieses Ph\"{a}nomens ist in Bild \ref{U113} die Situation dargestellt, dass sich in einem finiten Element einer Wandscheibe die Steifigkeit \"{a}ndert und somit in den vier Knoten des Elementes Zusatzkr\"{a}fte $f_i^+$ angreifen. Diese Kr\"{a}fte m\"{u}sste man also zur rechten Seite $\vek K\,\vek u = \vek f$ dazu addieren, um mit der urspr\"{u}nglichen Steifigkeitsmatrix den Verschiebungsvektor $\vek u_c$ des ge\"{a}nderten Modells zu berechnen, $\vek K\,\vek u_c = \vek f + \vek f^+$.
\\

Wenn wir das formalisieren, dann setzt der Ingenieur die L\"{o}sung aus $\vek K\,\vek u = \vek f$ und $\vek K_c\,\vek u_x = \vek f_x$ zusammen, wobei $\vek f_x$ der Lastvektor ist, der nur die umgedrehte St\"{u}tzenkraft enth\"{a}lt und $\vek u_x$ ist die zugeh\"{o}rige L\"{o}sung am modifizierten System, Matrix $\vek K_c$.

Zur Berechnung von $\vek u_c$
\begin{align}
\vek K\,\vek u_c = \vek f + \vek f^+
\end{align}
ben\"{o}tigt man den Vektor $\vek u_c$ {\em nach\/} der Steifigkeits\"{a}nderung, denn $\vek f^+ = \vek \Delta \,\vek K\,\vek u_c$.
%-----------------------------------------------------------------
\begin{figure}[tbp]
\centering
\includegraphics[width=0.9\textwidth]{\Fpath/U276}
\caption{Riegel mit Streckenlast \textbf{ a)} Verformungen  \textbf{ b)} nach Ausfall der St\"{u}tze \textbf{ c)} Berechnung von $\vek u_c$ am urspr\"{u}nglichen System \textbf{ d)} Ergebnis des Ingenieurs}
\label{U276}
%
\end{figure}%
%-----------------------------------------------------------------

Die Idee liegt nahe f\"{u}r $\vek u_c$ den alten Vektor $\vek u$ zu setzen, also mit einem Vektor
\begin{align}
\vek f_x = \vek \Delta \,\vek K\,\vek u
\end{align}
zu rechnen. Das ist im Grunde die Vorgehensweise des Ingenieurs. Wenn eine St\"{u}tze ausf\"{a}llt, dann dreht er die St\"{u}tzkraft um und bringt sie als Zusatzbelastung auf das Tragwerk auf. Das ist die Kraft $\vek f_x$. Der Ingenieur ermittelt also $\vek u_c$ n\"{a}herungsweise aus der Gleichung
\begin{align}
\vek K\,\vek u_c \simeq \vek f + \vek f_x\,.
\end{align}
Genauer gesagt, er l\"{o}st $\vek K\,\vek u = \vek f$ und addiert zu $\vek u$ die L\"{o}sung des Systems $\vek K\vek u_x = \vek f_x$, was in der Summe
$\vek  u_c \simeq \vek u + \vek u_x$ entspricht.
%-----------------------------------------------------------------
\begin{figure}[tbp]
\centering
\includegraphics[width=0.9\textwidth]{\Fpath/U186}
\caption{Momente $M$ \textbf{ a)} aus Wind und \textbf{ b)} Momente $\bar{M}$ der Verdrehung des Fu{\ss}punktes, Einflussfunktion f\"{u}r Fu{\ss}punktsmoment. Stabweise wird das Integral von $M$ und $\bar{M}$ mit dem Quotienten $\Delta EI/EI$ gewichtet und bestimmt so den Einfluss, den eine \"{A}nderung $\Delta EI$ in dem Stab auf das Fu{\ss}punktsmoment hat. Genau genommen m\"{u}sste man $M$ durch $M_c$ ersetzen, aber dies kann durch einen Korrekturfaktor, $M_c \simeq \alpha M$ n\"{a}herungsweise ausgeglichen werden}
\label{U186}
%
\end{figure}%
%-----------------------------------------------------------------

Betrachten wir ein Beispiel! In dem Rahmen in Bild \ref{U276} f\"{a}llt die Mittelst\"{u}tze aus und die Absenkung in der Riegelmitte betr\"{a}gt danach $u_c = 750$\,mm, was rechnerisch die Kraft $f^+$ ergibt\footnote{Das zweite Minus ist dem Verlust der Steifigkeit geschuldet.}
\begin{align}
f^+ = - (- \frac{EA}{l}) u_c = \frac{EA}{l}\,u_c = \frac{1.07\cdot 10^6}{4} \,0.75\,\text{m} = 200\,000\,\text{kN}\,.
\end{align}
Diese wird als Zusatzkraft auf den Riegel des alten Systems aufgebracht
\begin{align}
\vek K\,\vek u_c = \vek f + \vek f^+
\end{align}
und dabei ergibt sich genau der richtige Wert $u_c$ f\"{u}r die Durchbiegung des Riegels am modifizierten System.

Der Ingenieur setzt einfach die St\"{u}tzkraft als Riegellast auf den Rahmen, $f_x = 225$ kN, und kommt so auf den Wert $u_c \simeq u + u_x = 0 + 674$\,mm statt exakt $u_c = 750$\,mm. F\"{u}r eine Absch\"{a}tzung von Effekten, die durch Steifigkeits\"{a}nderungen hervorgerufen werden, ist diese Methode also ausreichend, man bringt einfach die St\"{u}tzenkraft auf das modifizierte System auf.
\\

%%%%%%%%%%%%%%%%%%%%%%%%%%%%%%%%%%%%%%%%%%%%%%%%%%%%%%%%%%%%%%%%%%%%%%%%%%%%%%%%%%%%%%%%%%%%%%%%%%%
{\textcolor{blau2}{\section{Zusammenfassung}}}
In diesem Kapitel haben wir die folgenden Punkte behandelt:
\begin{itemize}
  \item Differentialgleichungen der Stabstatik
  \item Die zugeh\"{o}rigen Greenschen Identit\"{a}ten
  \item Die Herleitung des Prinzips der virtuellen Verr\"{u}ckungen
\end{itemize}

%---------------------------------------------------------------------------------
\begin{figure}
\centering
{\includegraphics[width=0.8\textwidth]{\Fpath/U278}}
  \caption{In diesem Bild kommt die 'Doppelb\"{o}digkeit' der Statik mit finiten Elementen sehr gut zum Ausdruck: der Ingenieur hat kein arges, die \"{a}quivalenten Knotenkr\"{a}ften in den Lagerknoten f\"{u}r real zu nehmen}
  \label{U278}
%
\end{figure}
%---------------------------------------------------------------------------------

Auch bei jeder statisch bestimmt gelagerten Scheibe kann man diesen \"{U}bergang in den Lagerknoten erleben, s. Bild \ref{U278}. Wir wissen ja, was an Lagerkraft herauskommen muss, und so haben wir gar kein Problem damit, die $f_i$ in den Lagern f\"{u}r real zu nehmen. Im Innern der Scheibe, speziell in den Elementen, die dem Lagerknoten gegen\"{u}ber liegen, wirkt dagegen ein konfuses Durcheinander von Fl\"{a}chen- und Linienkr\"{a}ften. \\

Wenn man die Belastung auf einem Kragtr\"{a}ger in den Endknoten reduziert, dann mag das einem Laien sehr ungenau erscheinen, aber der Ingenieur wei{\ss}, dass die Momente des Knotenlastfalls eine ganz gute N\"{a}herung sind.

Es hat sich doch ein gro{\ss}er Ballast an S\"{a}tzen und Prinzipien angeh\"{a}uft, der oft mehr verdunkelt als erhellt. Wir brauchen den spitzen Bleistift um ohne Umwege die Ergebnisse direkt aus den Regeln der partiellen Integration (das ist die einzige Mathematik, die wir in diesem Buch ben\"{o}tigen) herleiten zu k\"{o}nnen. Wir haben uns aber bem\"{u}ht dem legitimen Wunsch nach anschaulichen Beispielen gerecht zu werden.

Wir haben mehrere e-Mails mit dem Autor gewechselt, weil wir der Meinung sind, dass man mathematische Ergebnisse (und Einflussfunktionen sind nun einmal Formeln) nur auf mathematischem Wege herleiten kann, w\"{a}hrend der Autor, im Glauben seinen Studenten damit einen Gefallen zu tun, die Anschauung zu Hilfe nahm, aber Anschauung und Balkenstatik nicht zur Deckung bringen konnte (Stichwort: Starrk\"{o}rperdrehungen).


Was umso mehr Wunder nimmt, weil ja gerade in der Statik Anschauung und das statische Gef\"{u}hl Hand in Hand gehen. \\

%---------------------------------------------------------------------------------
\begin{figure}
\centering
{\includegraphics[width=1.0\textwidth]{\Fpath/U286}}
\caption{Fachwerk \textbf{ a)} Minimall\"{o}sung \textbf{ b)} stark ausgefacht}
\label{U286}%
%
\end{figure}%
%---------------------------------------------------------------------------------

Anders gesagt, wenn die Zusatzbelastung gro{\ss}e Wege geht, ihre Eigenarbeit gro{\ss} ist, dann muss man genau hinschauen, w\"{a}hrend man in
allen anderen F\"{a}llen davon ausgehen kann, dass die Effekte 'versickern'.
\end{remark}

Im Grunde herrscht ein delikates Gleichgewicht zwischen innerer und \"{a}u{\ss}erer Energie bzw. Arbeit, $A_i = A_a$. Bei dem Fachwerk in Bild Bild \ref{U286} a ist die Bilanz einfach, weil nur \"{u}ber drei St\"{a}be integriert wird
\begin{align}
A_i = \sum_{j = 1}^3\,EA\int_0^{\,l_j} (u_j')^2\,dx = P\,u\,.
\end{align}
Je mehr St\"{a}be hinzukommen, s. Bild \ref{U286} b, \"{u}ber um so mehr St\"{a}be wird integriert und so k\"{o}nnte man meinen, dass wegen $A_i = A_a$ auch die Auslenkung w\"{a}chst. Aber das Gegenteil ist nat\"{u}rlich der Fall, $A_i$ und $A_a$ nehmen ab. Das Mehr an St\"{a}ben \"{u}ber die integriert wird, wird dadurch kompensiert, dass die Dehnungen $\varepsilon_j = u_j'$ der St\"{a}be kleiner werden. Die L\"{a}ngsverschiebungen $u_j(x)$ sind ja linear (keine \"{a}u{\ss}eren Kr\"{a}fte zwischen den Knoten) und so h\"{a}ngen die Dehnungen $u_j'$ nur von den Knotendifferenzen ab, aber diese Differenzen werden mit zunehmender Zahl von St\"{a}ben 'schneller' kleiner als die L\"{a}nge der Strecke, \"{u}ber die integriert wird, zunimmt.
\\

\begin{remark}
Wir haben es nun schon \"{o}fter erw\"{a}hnt. Betti und Mohr sind zwei Seiten einer Medaille. Mittels partieller Integration kommt man von Mohr
\begin{align}
\text{(Mohr)} \qquad m_{xx}^h(\vek x) = a(G_h,w_h) = G_h(\vek x,\vek x) \cdot P \qquad \text{(Betti)}\,.
\end{align}
zu Betti und umgekehrt. Der Punktwert $G_h(\vek x,\vek x)$ ist so 'falsch', wie $a(G_h,w_h)$ 'falsch' ist. Ein Nadel\"{o}hr zu treffen klingt dramatischer als ein Integral wie $a(G_h,w_h)$ auszuwerten, aber es ist kein Unterschied.
\end{remark}
\\

Bei der sogenannten 3-D Statik ist das anders, aber diese leidet darunter, dass die Ergebnisse leicht un\"{u}bersichtlich werden, man nicht mehr wei{\ss}, 'wo die Kr\"{a}fte hinflie{\ss}en', weil die sogenannte 'Knopfdruckstatik' den Tragwerksplaner viel weniger zum Mitdenken anregt, als die Positionsstatik.

Wir wollen das F\"{u}r und Wider hier nicht wiederholen. Der angenehme Nebeneffekt der 3-D Statik ist auf jeden Fall, dass Steifigkeits\"{a}nderungen oder allgemeiner \"{A}nderungen im Entwurf sich leichter nachvollziehen lassen, als bei der Positionsstatik.\\

%%%%%%%%%%%%%%%%%%%%%%%%%%%%%%%%%%%%%%%%%%%%%%%%%%%%%%%%%%%%%%%%%%%%%%%%%%%%%%%%%%%%%%%%%%%%%%%%%%%%%%%
\textcolor{blau2}{\section{Konstruktive Fragen}}
Bevor es finite Elemente (und Randelemente) gab, gab es die {\em Positionsstatik\/}, wo man ein Tragwerk in einzelne Positionen einteilte und diese einzeln berechnete. Die Ber\"{u}cksichtigung der Steifigkeiten der angrenzenden Bauteile ist dabei nur n\"{a}herungsweise m\"{o}glich.

Das Thema dieses Abschnitts soll daher die Frage sein: Wie kann man bei der Positionsstatik nachtr\"{a}gliche Steifigkeits\"{a}nderungen ber\"{u}cksichtigen, oder, was eine \"{a}hnliche Fragestellung ist, wie kann man den Effekt von zu groben Annahmen bez\"{u}glich der Steifigkeiten absch\"{a}tzen?

Das grunds\"{a}tzliche Werkzeug haben wir schon in Abschnitt XX diskutiert. Zu jedem interessierenden Wert $J(w)$, einer Schnittgr\"{o}{\ss}e, einer Lagerkraft, einer Verformung, gibt es eine Einflussfunktion
\begin{align}
J(w) = \int_0^{\,l} G(y,x)\,p(y)\,dy
\end{align}
und die entscheidende Frage ist daher, wie sich der Kern $G(y,x)$ dieser Einflussfunktion mit den Steifigkeiten \"{a}ndert.

Man muss diese Frage nat\"{u}rlich auch unter dem Aspekt betrachten, wieweit strahlen denn eigentlich die Effekte, die die Belastung verursacht aus? Wir haben oben festgehalten, dass die Einflussfunktionen je nach Ordnung der Ableitung der Zielgr\"{o}{\ss}e im Aufpunkt unterschiedliches Abklingenverhalten haben.

Es kann sich dabei auch nur mehr um eine qualitative Untersuchung handeln, als eine quantitative. Es gilt also die Effekte abzusch\"{a}tzen, abzusch\"{a}tzen ob eine Neuberechnung n\"{o}tig oder sinnvoll ist.


Die FE-Einflussfunktion $G_2(\vek y,\vek x)$ ist ja nicht exakt, sondern eine N\"{a}herung, der die Spitze $G_2(\vek x,\vek x) = \infty$ von dem FE-Programm abgeschnitten wird. Das verdeutlicht, wie verletzlich der Wert $m_{xx}$ \"{u}ber der St\"{u}tze ist.

Wechseln wir von gew\"{o}hnlichen Differentialgleichungen zu partiellen Differentialgleichungen, dann gilt all dies nat\"{u}rlich analog. Jede hinreichend regul\"{a}re Funktion, $u \in C^2(\Omega)$, gen\"{u}gt der Bilanz
\begin{align}\label{Eq90}
\text{\normalfont\calligra G\,\,}(u, u) = \int_{\Omega} - \Delta u\,u\,\,d\Omega + \int_{\Gamma} \frac{\partial u}{\partial n}\,u\,ds - \int_{\Omega} \nabla u \dotprod \nabla u\,d\Omega = 0\,,
\end{align}
gleichg\"{u}ltig wie $u$ aussieht oder wie kurvenreich der Rand $\Gamma$ ist. Das ist schon ein beeindruckendes Resultat und zeigt, welche analytische Kraft in der partiellen Integration steckt.\\

Wenn eine Str\"{o}mung divergenzfrei ist (keine Quellen oder Senken im Innern), dann muss zu jedem Zeitpunkt die Menge an Fl\"{u}ssigkeit, die in ein Kontrollvolumen $\Omega$ hineinflie{\ss}t, in gleicher Gr\"{o}{\ss}e auch wieder herausflie{\ss}en. Dieses intuitiv evidente Resultat beruht auf dieser Identit\"{a}t, $\text{\normalfont\calligra G\,\,}(u, 1) = 0$. In der Statik sind die Lasten die Quellen und wenn zwischen zwei Stabenden keine Lasten vorhanden sind, dann m\"{u}ssen die Schnittkr\"{a}fte an den beiden Enden des Stabes in der Summe null sein. \\

\hspace*{-12pt}\colorbox{hellgrau}{\parbox{0.98\textwidth}{Der Ingenieur stellt die Differentialgleichung $EI\,w^{IV}(x) = p(x)$ auf, aber wie die dazu passenden Gleichgewichtsbedingungen aussehen, steht nicht in seinem Belieben. Das entscheidet allein die Mathematik.}}\\

Man kann daher nicht mitten im Galopp pl\"{o}tzlich wieder auf richtige Drehungen umschwenken und so die Statik in die N\"{a}he der N\"{a}herungsrechnung r\"{u}cken. {\em Die Statik rechnet exakt, sie l\"{o}st die gestellten Aufgaben (im Rahmen der zu Grunde gelegten Annahmen) exakt!\/}\\

 genannt werden\footnote{Das $L$ steht f\"{u}r den franz\"{o}sischen Mathematiker Lebesgue\index{Lebesgue-Integral}},

 \begin{enumerate}
\item Das Ergebnis ist exakt, wenn die Einflussfunktion $G(y,x)$ mit den Ansatzfunktionen dargestellt werden kann.
\item Wenn das nicht m\"{o}glich ist, dann benutzt das FE-Programm eine N\"{a}herung, aber dann ist das Ergebnis im allgemeinen auch nur eine N\"{a}herung.
\end{enumerate}
Wenn der Benutzer die Durchbiegung des Seils in einem Punkt $x$ wissen, will, dann geht das FE-Programm wie folgt vor: Es ermittelt die Einflussfunktion $G(y,x)$ f\"{u}r diesen Punkt und \"{u}berlagert $G(y,x)$ mit der Belastung $p$ wie in (\ref{Eq108}). Wenn $G(y,x)$ exakt ist, dann ist auch das Ergebnis $w(x)$ exakt. Wenn sich jedoch die Einflussfunktion nicht mit den $\Np_i(x)$ darstellen l\"{a}sst, dann benutzt das FE-Programm eine N\"{a}herung $G_h(y,x) \sim G(y,x)$ und erh\"{a}lt aber nat\"{u}rlich dann auch nur einen gen\"{a}herten Wert $w_h(x) \sim w(x)$
\begin{align}
w_h(x) = \int_0^{\,l} G_h(y,x)\,p(y)\,dy\,.
\end{align}
Das ist die Logik der finiten Elemente.
\\
%----------------------------------------------------------------------------------------------------------
\begin{figure}[tbp]
\centering
\if \bild 2 \sidecaption \fi
\includegraphics[width=1.0\textwidth]{\Fpath/U74}
\caption{Vergleich einer FE-L\"{o}sung mit der exakten L\"{o}sung} \label{U74}
\end{figure}%
%----------------------------------------------------------------------------------------------------------
Die FE-L\"{o}sung eines Stabes, s. Bild \ref{U74}, mag einen solchen Vergleich zwischen $p$ und $p_h$ verdeutlichen. Weil die finiten Elemente die Lasten in den Knoten konzentrieren, ist der Fehler auf der Lastseite gro{\ss}, wo im Original Linienkr\"{a}fte $p = 10$ kN/m wirken, sind im FE-Lastfall keine Kr\"{a}fte vorhanden. Dagegen ist der Fehler in den Normalkr\"{a}ften deutlich kleiner, s. Bild \ref{U74} d, und wenn man $u$ und $u_h$ vergleicht, dann wird der Fehler noch mal kleiner,  s. Bild \ref{U74} b. \\

Es ist diese Ambivalenz, die die Diskussion von FE-Ergebnissen so schwierig macht. Der Ingenieur ist  ein Meister in diesem 'Seitenwechsel'. Er hat kein Problem damit, die $f_i$ einmal als nur gedacht und einmal als real zu nehmen. Es ist nur eine der unz\"{a}hligen Unsch\"{a}rfen und N\"{a}herungen mit denen er tagt\"{a}glich in seiner Arbeit konfrontiert ist.
\\
In der Mathematik nennt man ein Integral von zwei Funktionen ein $L_2$-Skalarprodukt\index{$L_2$-Skalarprodukt}, was aus der Sicht der Statik eine sinnvolle Bezeichnung ist, weil eigentlich alle Integrale in der Statik die \"{U}berlagerung einer Kraftgr\"{o}{\ss}e mit einer Weggr\"{o}{\ss}e sind, sie also Arbeitsintegrale sind.

%----------------------------------------------------------------------------------------------------------
\begin{figure}[tbp]
\centering
\if \bild 2 \sidecaption \fi
\includegraphics[width=1.0\textwidth]{\Fpath/U41}
\caption{Zwei Sichtweisen: Lagersenkung oder Kragarmbelastung}
\label{U41}
\end{figure}%
%----------------------------------------------------------------------------------------------------------

%%%%%%%%%%%%%%%%%%%%%%%%%%%%%%%%%%%%%%%%%%%%%%%%%%%%%%%%%%%%%%%%%%%%%%%%%%%%%%%%%%%%%%%%%%%%%%%%%%%
{\textcolor{blau2}{\subsection{Zwei Sichten auf dieselbe Sache}}}
Bei der Lagersenkung des Balkens in Bild \ref{U41} a berechnen wir f\"{u}r die potentielle Energie den Wert
\begin{align}
\Pi(w)= \frac{1}{2}\,\int_0^{\,l} \frac{M^2}{EI}\,dx = \frac{1}{2}\,V(l)\,w_\Delta\,,
\end{align}
wenn wir die erste Greensche Identit\"{a}t
\begin{align}
\frac{1}{2}\,\text{\normalfont\calligra G\,\,}(w,w) = \frac{1}{2}\,V(l)\,w_\Delta - \frac{1}{2}\,\int_0^{\,l} \frac{M^2}{EI}\,dx = 0
\end{align}
zu Hilfe nehmen.

Wenn man einen gleich langen Kragtr\"{a}ger an seinem freien Ende gerade so stark belastet, dass eine Kraft $P$ dieselbe Durchbiegung $w_\Delta$ generiert, dann gilt
\begin{align}
\Pi(w) = \frac{1}{2}\int_0^{\,l} \frac{M^2}{EI}\,dx - P\,w_\Delta = - \frac{1}{2}\, P\,w_\Delta \,.
\end{align}
Nat\"{u}rlich ist $P = V(l)$ und so stimmen die beiden Ausdr\"{u}cke, abgesehen von dem Vorzeichen, zahlenm\"{a}{\ss}ig \"{u}berein.

Wir sehen an diesem Beispiel aber auch, dass die potentielle Energie kein 'absoluter Wert' ist, sondern dass sie, je nach Sichtweise, einmal positiv und einmal negativ sein kann, wenn nat\"{u}rlich auch der Betrag sich nicht \"{a}ndert.


Und noch eine Bemerkung. Es ist gar nicht hilfreich, wenn behauptet wird $\delta w$ und $\delta \Np$ seien infinitesimal klein, aber ihr Verh\"{a}ltnis sei endlich. Das $\delta w$  in Bild \ref{U13} ist gesch\"{a}tzte 0.3 m und $\tan \Np_l$ und $\tan \Np_r$ kann man sicherlich auch nicht infinitesimal klein nennen.
\\

\begin{remark}
Dass in (\ref{Eq101}) auf der rechten Seite eine 1 steht, wo in Abschnitt \ref{Chap3}.\ref{General} an entsprechender Stelle eine -1 stand, ist dem Umstand geschuldet, dass die Lagerkraft $R$ (= Federkraft) hier nach oben zeigt, w\"{a}hrend sie in Abschnitt \ref{Chap3}.\ref{General} konsequent in die Richtung $f_i$ zeigt.

Da dann die Einflussfunktion wie 'auf den Kopf' gestellt aussieht, haben wir hier $R$ nach oben zeigen lassen.\\
\end{remark}


Wenn der Knoten ein starres Lager ist, dann muss man, wie oben, zur Lagerkraft der FE-L\"{o}sung noch den Anteil $R_d$ hinzu addieren, der direkt in den Lagerknoten reduziert wurde.

%%%%%%%%%%%%%%%%%%%%%%%%%%%%%%%%%%%%%%%%%%%%%%%%%%%%%%%%%%%%%%%%%%%%%%%%%%%%%%%%%%%%%%%%%%%%%%%%%%%
{\textcolor{blau2}{\section{Zusammenfassung}}\label{Zusammenfassung}
Es sei $f_i$ die Lagerkraft in einem festgehaltenen Knoten. Den Wert von $f_i$ erh\"{a}lt man nach der L\"{o}sung des Systems $\vek K\,\vek u = \vek f$ aus dem nicht-reduzierten System\footnote{bevor also die Zeilen und Spalten gestrichen werden, die zu Lagerknoten $u_i = 0$ geh\"{o}ren}\index{$\vek K_G$}\index{nicht-reduzierte Steifigkeitsmatrix}
\begin{align}
\vek K_G\,\vek u_G = \vek f_G\,.
\end{align}
Zu $f_i$ muss man noch den Anteil hinzu addieren, der direkt in das Lager reduziert wurde.

F\"{u}r jedes $f_i$ gibt es eine Einflussfunktion
\begin{align}
G_h(\vek y,\vek x) = \sum_j\,g_j(\vek x)\,\Np_j(\vek y)\,,
\end{align}
deren Knotenwerte $g_j$ die L\"{o}sung des Systems
\begin{align}
\vek K\,\vek g = \vek j
\end{align}
sind. Hierbei ist der Vektor $\vek j$ identisch mit der Spalte $i$ der Steifigkeitsmatrix $\vek K_G$, allerdings auf die L\"{a}nge $n$ gek\"{u}rzt, d.h. die zu gesperrten Freiheitsgraden geh\"{o}rigen Zeilen werden gestrichen.

Die vollst\"{a}ndige Einflussfunktion, die den Anteil mit umfasst, der direkt in den Knoten flie{\ss}t, lautet
\begin{align} \label{Eq104}
G_h(\vek y,\vek x) = \sum_j\,g_j(\vek x)\,\Np_j(\vek y) + \Np_i(\vek y)\,.
\end{align}
Ist der Knoten ein elastisches Lager, dann erh\"{a}lt man die Einflussfunktion f\"{u}r $f_i$, indem man eine Kraft $P = 1$ auf den Knoten setzt, und die dabei entstehende Biegefl\"{a}che mit der Steifigkeit $k$ des Lagers multipliziert. Eine Korrektur wie in (\ref{Eq104}) ist nicht notwendig.
\\

%%%%%%%%%%%%%%%%%%%%%%%%%%%%%%%%%%%%%%%%%%%%%%%%%%%%%%%%%%%%%%%%%%%%%%%%%%%%%%%%%%%%%%%%%%%%%%%%%%%
{\textcolor{blau2}{\section{Generalisierung}}\label{General}

Nehmen wir an, ein Ansatz besteht aus $n $ Ansatzfunktionen $\Np_i$ und irgendwo sei ein festes Lager, das wir uns als einzelnen Knoten vorstellen k\"{o}nnen. Hinter jeder Ansatzfunktion $\Np_i$ steht ein gewisser Satz von Kr\"{a}ften, die {\em shape forces\/} $p_i$, die dem Tragwerk die Verformung $\Np_i$ aufzwingen. Das Lager  h\"{a}lt dagegen und so geh\"{o}rt zu jedem $\Np_i$ eine \"{a}quivalente Lagerkraft $j_i$  in dem Lagerknoten, die gleich der Wechselwirkungsenergie zwischen $\Np_i$  und der Einheitsverformung $\Np_X$ ($X$ ist ein Index) des Lagerknotens ist
\begin{align}\label{Eq99}
a(\Np_X,\Np_i) = 1 \cdot j_i\,.\nn
\end{align}

\hspace*{-12pt}\colorbox{hellgrau}{\parbox{0.98\textwidth}{Die $j_i$ sind wegen $a(\Np_X,\Np_i) = k_{X i}$ identisch mit den Elementen der Zeile $X$ der nicht-reduzierten Steifigkeitsmatrix $\vek K_{G}$.}}\\

Hier ergibt sich eine kleine Schwierigkeit mit den Indices. Die Zeile $X$ von $\vek K_{G}$ enth\"{a}lt $N$ Eintr\"{a}ge, wenn $N \times N$ die Gr\"{o}{\ss}e von $\vek K_{G}$ ist. Der Vektor $\vek j$, mit dem wir im folgenden operieren, hat aber nur $n$ Eintr\"{a}ge, weil die Eintr\"{a}ge, die zu gesperrten Freiheitsgraden geh\"{o}ren, gestrichen wurden.

Zu einer $n$-gliedrigen FE-L\"{o}sung $u_h = u_1\,\Np_1(x) + u_2\,\Np_2(x) + \ldots$ geh\"{o}rt demnach die Lagerkraft
\begin{align}
R = \vek u^T\,\vek j = u_1\,j_1 + u_2\,j_2 + \ldots + u_n\,j_n\,.
\end{align}
Diese Lagerkraft muss nun aber auch gleich dem Skalarprodukt $\vek f^T\,\vek g$ sein, also dem Produkt aus den \"{a}quivalenten Knotenkr\"{a}ften $\vek f$ des Lastfalls und den Knotenverschiebungen $\vek g$ der Einflussfunktion f\"{u}r die Lagerkraft,
\begin{align}
R = \vek u^T\,\vek j = \vek f^T\,\vek g\,,
\end{align}
und damit folgt, dass das einzelne $g_i$ die Lagerkraft im Lastfall $\vek f = \vek e_i$ ($i$-ter Einheitsvektor) sein muss.
\begin{align}
R (\text{im LF $\vek e_i$}) = \vek u^T (\text{im LF $\vek e_i$})\,\vek j = \vek e_i^T \vek g = g_i\,.
\end{align}

Die Knotenverschiebungen in den Lastf\"{a}llen $\vek f = \vek e_i$ sind aber nun gerade die Zeilen (= Spalten) der Inversen $\vek K^{-1}$ und so folgt weiter, dass
\begin{align}
\vek g = \vek K^{-1}\,\vek j\,
\end{align}
der Vektor der Knotenwerte der FE-Einflussfunktion ist
\begin{align}
G_h(y,x) = \sum_{i = 1}^n g_i(x)\,\Np_i(y)\,.
\end{align}
Das $ x$ ist hier die Koordinate des Knotens.

Die vollst\"{a}ndige Einflussfunktion, die den Anteil mit umfasst, der direkt in einen Knoten $i$ flie{\ss}t, lautet
\begin{align} \label{Eq104}
G_h(y,x) = \sum_j\,g_j(x)\,\Np_j(y) + \Np_i(y)\,.
\end{align}
Ist der Knoten ein elastisches Lager, dann erh\"{a}lt man die Einflussfunktion f\"{u}r $f_i$, indem man eine Kraft $P = 1$ auf den Knoten setzt, und die dabei entstehende Biegefl\"{a}che mit der Steifigkeit $k$ des Lagers multipliziert. Eine Korrektur wie in (\ref{Eq104}) ist nicht notwendig.

%%%%%%%%%%%%%%%%%%%%%%%%%%%%%%%%%%%%%%%%%%%%%%%%%%%%%%%%%%%%%%%%%%%%%%%%%%%%%%%%%%%%
{\textcolor{blau2}{\section{Positionsstatik und 3-D Berechnung}}
Bei der sogenannten {\em Positionsstatik\/} wird jeder Unterzug, jede Deckenplatte f\"{u}r sich allein berechnet. Unter Umst\"{a}nden m\"{o}chte man aber die Werte $f_i$ in den Lagerknoten einer Platte mit den Ergebnissen einer 3-D Berechnung vergleichen.

Nun ist es aber so, dass bei einer 3-D Berechnung nicht die Anschnittkr\"{a}fte---in der Stabstatik w\"{a}ren das die Balkenendkr\"{a}fte---ausgegeben werden, sondern nur die Knotenkr\"{a}fte $f_i$, also die Summe \"{u}ber alle Anschlusskr\"{a}fte. Die Frage ist daher, wie man die  Anschlusskr\"{a}fte berechnen kann.

Das geht im Grunde wie in der Stabstatik. Wenn das Gleichungssystem
\begin{align} \label{Eq105}
\vek K\vek u = \vek f + \vek p
\end{align}
gel\"{o}st ist, dann kennt man die $u_i$ an jedem Knoten. Diese kann man nun stabweise in das lokale Koordinatensystem der angeschlossenen St\"{a}be umrechnen $u_i \to u_i^e$ und dann an Hand der Beziehung
\begin{align}\label{Eq106}
\vek K^e\,\vek u^e = \vek f^e + \vek p^e
\end{align}
die Balkenendkr\"{a}fte $f_i^e$ an jedem einzelnen Stab berechnen. Die $p_i^e$ sind die Auflagerdr\"{u}cke (= Festhaltekr\"{a}fte $\times (-1)$) am Stab aus der Belastung.

Bei einer Platte macht man sinngem\"{a}{\ss} dasselbe. Man multipliziert die Steifigkeitsmatrix $\vek K^p$ der Platte (nur der Platte!) mit den zur Platte geh\"{o}rigen Anteilen $\vek u^p$ des globalen Vektors $\vek u_G$ und erh\"{a}lt so die $f_i^p$ am Rand der Platte
\begin{align}
\vek K^p\,\vek u^p = \vek f^p
\end{align}
und diese kann man dann mit den Ergebnissen aus der Positionsstatik vergleichen.

Wenn man einmal annimmt, dass die Steifigkeitsmatrix $\vek K^p$ der Platte aus dem 3-D Modell und die Matrix $\vek K^{pos}$ aus der Positionsstatik nicht allzusehr voneinander abweichen, dann sind die Unterschiede in den $f_i$ auf die Unterschiede in den Knotenverformungen zwischen dem 3-D Modell und der Positionsstatik zur\"{u}ckzuf\"{u}hren.\\

\begin{remark}
Die $f_i$ in (\ref{Eq105}) sind die Kr\"{a}fte, die direkt in den Knoten des Rahmens angreifen und die $p_i$ sind die in jedem Knoten aufsummierten Auflagerdr\"{u}cke aus der verteilten Belastung links und rechts vom Knoten.

Die $f_i^e$ in (\ref{Eq106}) dagegen sind Balkenendkr\"{a}fte und keine Knotenkr\"{a}fte. In der Literatur wird leider derselbe Buchstabe f\"{u}r diese unterschiedlichen Kr\"{a}fte benutzt---einmal ist man am Balkenende und einmal im Knoten.
\end{remark}

%%%%%%%%%%%%%%%%%%%%%%%%%%%%%%%%%%%%%%%%%%%%%%%%%%%%%%%%%%%%%%%%%%%%%%%%%%%%%%%%%%%%%%%%%%%%%%%%%%%
\textcolor{blau2}{\section{Durchlauftr\"{a}ger}}

Wir wollen noch einige Dinge erg\"{a}nzen, die speziell die Stabstatik betreffen.
%-----------------------------------------------------------------
\begin{figure}[tbp]
\centering
\includegraphics[width=0.8\textwidth]{\Fpath/U232}
\caption{Anpassung eines Balkenelements an die Biegelinie des Tr\"{a}gers. Die Absenkung auf das Niveau des Tr\"{a}gers erfordert keine Kr\"{a}fte, nur die Verdrehung der Endtangenten des Elements ($- - - $) erfordert Momente $f^+$}
\label{U232}
\end{figure}%
%-----------------------------------------------------------------

Wenn sich die Biegesteifigkeit $EI$ eines Elementes \"{a}ndert, dann addieren wir ein Element $\Omega_e$ zu dem System
und koppeln es mit den Kr\"{a}ften/Momenten
\begin{align}\label{Eq72}
f_1^+ = - V_a^+,\qquad f_2^+ = - M_a^+, \qquad f_3^+ = V_b^+, \qquad f_4^+ = M_b^+\nn
\end{align}
an die Struktur. Wir wissen, dass diese Gr\"{o}{\ss}en im Gleichgewicht sind
\begin{align}
(\Sigma V = 0) \qquad f_1^+ + f_3^+ = 0\,\qquad  f_2^+ + f_4^+ - f_3^+ \cdot l_e = 0 \qquad (\Sigma M = 0)\,,
\end{align}
anders gesagt, wenn man die Einflussfunktion, die wir hier der Einfachheit halber $g(y)$ nennen, auf dem Element $(a,b)$ linear interpoliert,
\begin{align}
g(y) = g(a) + (g(b) - g(a)) \cdot \frac{y - a}{b - a} = g(a) + m\,(y - a)\,,
\end{align}
dann sind die Gr\"{o}{\ss}en (\ref{Eq72}) orthogonal zu dieser Interpolierenden, leisten keine Arbeit auf den Knotenverschiebungen/-verdrehungen des Weges $g(y)$. Bei der Auswertung der Einflussfunktion verbleibt also nur der Term (die Momente und $g'$ drehen entgegengesetzt)
\begin{align}\label{Eq73}
- f_2^+ \cdot (g'(a) - m) - f_4^+ \cdot (g'(b) - m)\,,
\end{align}
dessen Gr\"{o}{\ss}e davon abh\"{a}ngt, wie stark die Neigung der Einflussfunktion $g(y)$ an den Balkenenden von der Neigung $m$ der Interpolierenden abweicht.
%-----------------------------------------------------------------
\begin{figure}[tbp]
\centering
\includegraphics[width=0.9\textwidth]{\Fpath/U112}
\caption{Durchlauftr\"{a}ger, \textbf{ a)} gleiches $EI$ in allen Feldern, \textbf{ b)} Verdopplung von $EI$ im zweiten Feld, \textbf{ c)} Erzeugung von
$M_c$ am Original mit Hilfe von Koppelmomenten $f_i^+$, \textbf{ d)} Einflussfunktion f\"{u}r $M$ (Mitte 3. Feld)}
\label{U112}
\end{figure}%
%-----------------------------------------------------------------

Wenn der Aufpunkt $ x$ weit genug weg liegt, dann d\"{u}rfen wir annehmen, dass die Differenzen $g' - m$ klein sind, und dann k\"{o}nnen wir den Effekt der Steifigkeits\"{a}nderung  vernachl\"{a}ssigen. Bei Durchlauftr\"{a}gern ist die lineare Interpolierende der Einflussfunktionen null,  weil die Einflussfunktionen in allen Lagerknoten null sind, s. Bild \ref{U112} d, und somit ist auch $m = 0$.

Gerade bei Durchlauftr\"{a}gern klingen Einflussfunktionen sehr rasch ab, s. Bild \ref{U112} d, und \"{A}nderungen von $EI$ im zweitn\"{a}chsten oder drittn\"{a}chsten Feld d\"{u}rften vernachl\"{a}ssigbar sein.

Bei dem Beispiel in Bild \ref{U112}, $EI = 3.56\cdot 10^{3}$ kNm$^2$ findet die Steifigkeits\"{a}nderung, $EI \to 2\,EI$, in dem Feld direkt neben dem Aufpunkt statt, aber trotz dieser N\"{a}he \"{a}ndert sich das Feldmoment wenig, $M = 34.6 \to M_c = 32.3$ kNm.

Die beiden Momente $f_i^+$ am Element 2 ergeben sich aus der Gleichung $- \vek \Delta\,\vek K\,\vek u_c = \vek f^+$,
\begin{align} \label{Eq74}
 - \frac{EI}{l^3} \left[
\begin{array}{r r r r}
 12 & -6l & -12 &-6l \\
 -6l & 4l^2 & 6l &2l^2 \\
 -12 & 6l & 12 & 6l \\
 -6l &2l^2 &6l &4l^2
 \end{array}
  \right]\,\left [\barr{c} \phantom{-} 0 \\ 1.32 \cdot 10^{-3} \\ \phantom{-} 0 \\ -3.62 \cdot 10^{-3} \earr \right ] = \left [\barr{c}  * \\ 1.74 \\ * \\ 12.84 \earr \right ]
\end{align}
wobei in diesem Fall $\vek \Delta \vek K = \vek K$ die $4 \times 4$-Matrix des Standardelementes ($EI$) ist (wir haben ja $EI$ verdoppelt).


Die Momente $f_i^+$ alleine erzeugen also das Zusatzmoment ($m = 0$)
\begin{align}
-f_2^+ &\cdot (g'(a) - m) - f_4^+ \cdot (g'(b) - m)\nn \\
 &= (-1.74) \cdot (-0.054) - 12.89 \cdot 0.188  = -2.3\,\text{kNm}
\end{align}
und dieses zu dem urspr\"{u}nglichen Feldmoment addiert, ergibt das neue Feldmoment
\begin{align}
M_c = M - 2.3 = 34.6 \,\text{kNm} - 2.3\,\text{kNm} = 32.3\,\text{kNm}\,.
\end{align}
Das Beispiel ist nat\"{u}rlich rein theoretisch, weil wir am modifizierten System $(\vek K + \vek \Delta\,\vek K)\,\vek u_c = \vek f$ erst die Verformungen $u_i^c$ berechnen m\"{u}ssen, die wir in (\ref{Eq74}) benutzen, um die Momente $f_i^+$ zu berechnen, mit denen wir dann am System $\vek K\,\vek u_c = \vek f + \vek f^+$ den Vektor $\vek u_c$ berechnen...

Aber hier geht es prim\"{a}r darum, zu demonstrieren, wie gro{\ss} die Verdrehungen $g'(a)$ und $g'(b)$ sind, wenn der Aufpunkt gleich im n\"{a}chsten Element liegt und wie gro{\ss} die $f_i^+$ sind. Es geht um Absch\"{a}tzungen, um das statische Verst\"{a}ndnis der Situation, nicht darum, den Computer zu schlagen.

Wir halten also fest: \\

\hspace*{-12pt}\colorbox{hellgrau}{\parbox{0.98\textwidth}{\"{A}nderungen der Biegesteifigkeit, $EI + \Delta EI $, in einem Element, f\"{u}hren zu Zusatzkr\"{a}ften und -momenten $f_i^+$ an den Elementenden. F\"{u}r die dadurch ausgel\"{o}sten Effekte sind aber nur die Momente verantwortlich, s. (\ref{Eq73}). }}\\

Dass die Querkr\"{a}fte keine Rollen spielen, versteht man beim Blick auf das Bild \ref{U232}.

%-----------------------------------------------------------------
\begin{figure}[tbp]
\centering
\includegraphics[width=0.9\textwidth]{\Fpath/U101}
\caption{Der Einfluss, den das linke bzw. rechte Moment $X_i$ auf die Durchbiegung im Punkt $x$ haben, ist nicht gleich}
\label{U101}
\end{figure}%
%-----------------------------------------------------------------

%---------------------------------------------------------------------------------
\begin{figure}
\centering
\if \bild 2 \sidecaption \fi
\includegraphics[width=.75\textwidth]{\Fpath/U151}
\caption{Einflussfunktion f\"{u}r die Normalkraft links bzw. rechts vom Knoten 3. Die beiden Einflussfunktionen sind nach Addition der lokalen L\"{o}sung gleich}
\label{U151}%
\end{figure}%
%---------------------------------------------------------------------------------

%%%%%%%%%%%%%%%%%%%%%%%%%%%%%%%%%%%%%%%%%%%%%%%%%%%%%%%%%%%%%%%%%%%%%%%%%%%%%%%%%%%%%%%%%%%%%%%%%%%
{\textcolor{blau2}{\subsection{Netzlinien und Schnittgr\"{o}{\ss}en}}}\index{Netzlinien und Schnittgr\"{o}{\ss}en}
Wir hatten in Kapitel 3, S. \pageref{Jumps}, \"{u}ber die Spr\"{u}nge in den Spannungen an den Netzlinien gesprochen. Hierzu noch die folgende Erg\"{a}nzung.

Schnittgr\"{o}{\ss}en auf einer Netzlinie auszuwerten, ist nicht ratsam. Zum einen springen die ersten Ableitungen der Ansatzfunktionen auf den Linien, so wie ein Dach im First zwei Neigungen hat, man h\"{a}tte also zwei Spannungen in demselben Punkt und zum andern k\"{o}nnen die Elemente links und rechts von der Linie unterschiedliche Moduli $E_i$ haben.

Nat\"{u}rlich kann man den Netzlinien beliebig nahe kommen, so dass das keine echte Einschr\"{a}nkung ist, und wenn wie bei Stabtragwerken, die Schnittgr\"{o}{\ss}e $N, M$ oder $V$ am \"{U}bergang zweier Elemente stetig ist, dann m\"{u}ssen auch die Einflussfunktionen f\"{u}r die Schnittgr\"{o}{\ss}e links bzw. rechts vom Knoten dieselbe Gestalt haben und dann ist es egal, ob man den Punkt links oder rechts vom Knoten als Aufpunkt w\"{a}hlt, wie Bild \ref{U151} zeigt.
%----------------------------------------------------------------------------
\begin{figure}
\centering
{\includegraphics[width=0.9\textwidth]{\Fpath/U254}}
  \caption{Einflussfunktion f\"{u}r die Normalkraft links von der Mitte, \textbf{ a)} System, \textbf{ b)} $\Np_1(x)$, \textbf{ c)} FE-Einflussfunktion $j_1 = EA_1$, \textbf{ d)} lokale L\"{o}sung am Stab 1, \textbf{ e)} EL = Summe aus \textbf{ c)} + \textbf{ d)}}
  \label{U254}
\end{figure}%%

%----------------------------------------------------------

Das Gleichungssystem f\"{u}r die Knotenverschiebungen $g_i$ der Einflussfunktion f\"{u}r $N(x)$ links
\begin{align} \label{Eq79}
\left[\barr{r r r r} 2 & - 1 & 0 & 0 \\ - 1 & 2 & -1 & 0\\ 0 & -1 & 3 &-2 \\ 0 & 0 & -2 &4\earr\right]
\,\left[\barr{c} g_1 \\g_2 \\ g_3 \\ g_4 \earr \right] = \left[\barr{r} 0 \\ -1  \\
1  \\ 0 \earr \right] \qquad \ldots =  \left[\barr{r} 0 \\0  \\
-2  \\ 2 \earr \right]
\end{align}
hat die L\"{o}sung
\begin{align}
g_1 = -0.25, \,\,g_2 = -0.5, \,\,g_3 = 0.25,\,\, g_4 = 0.125\,,
\end{align}
und f\"{u}r $N(x)$ rechts, mit dem zweiten Vektor in (\ref{Eq79}) auf der rechten Seite, hat die L\"{o}sung
\begin{align}
g_1 = -0.25, \,\,g_2 = -0.5, \,\,g_3 = -0.75, \,\,g_4 = 0.125\,.
\end{align}
Addiert man zu diesen Funktionen noch die lokale L\"{o}sung, also die L\"{a}ngsverschiebungen am ein-elementrigen Stab aus der Spreizung $[[u]] = 1$ am rechten bzw. linken Ende, dann erh\"{a}lt man die exakten Einflussfunktionen und die stimmen nat\"{u}rlich \"{u}berein.

Dasselbe gilt im \"{U}brigen auch f\"{u}r den Stab in Bild \ref{U254}, wo die Einflussfunktion f\"{u}r die Normalkraft $N(x)$ links vom Steifigkeitssprung berechnet wird. W\"{u}rde man den Aufpunkt rechts davon legen, dann w\"{u}rde $j_1 = - 5 \cdot 1.07 \cdot 10^6$ nach links dr\"{u}cken und die lokale L\"{o}sung im rechten Element w\"{u}rde von +1 im mittleren Knoten auf 0 im rechten Knoten abfallen. Der Gesamteffekt w\"{a}re aber derselbe wie vorher, d.h. die Einflussfunktion f\"{u}r $N(x)$ links und rechts von der Mitte w\"{a}re dieselbe.

All dies ist keine \"{U}berraschung. Wir zitieren dieses Beispiel auch nur, um noch einmal deutlich zu machen, wie man durch Addition der lokalen L\"{o}sung zur exakten Einflussfunktion kommt, auch dann, wenn die Steifigkeiten unterschiedlich sind. Au{\ss}erhalb des
Elements, auf dem der Aufpunkt liegt, sind FE-Einflussfunktionen bei (nicht gevouteten) Stabtragwerken sowieso exakt.


Bei Fl\"{a}chentragwerken sind die FE-Einflussfunktionen immer nur N\"{a}herungen und deswegen stimmen die Einflussfunktionen f\"{u}r, z.B. $\sigma_{xx}$, in zwei nur durch eine Netzlinie getrennten Punkten nicht \"{u}berein, auch wenn theoretisch $\sigma_{xx}^L = \sigma_{xx}^R$ sein muss. Den Grund haben wir oben erl\"{a}utert.

%---------------------------------------------------------------------------------
\begin{figure}
\centering
{\includegraphics[width=0.9\textwidth]{\Fpath/U118}}
  \caption{Finite Elemente und finite Differenzen}
  \label{U118}
\end{figure}
%---------------------------------------------------------------------------------



%%%%%%%%%%%%%%%%%%%%%%%%%%%%%%%%%%%%%%%%%%%%%%%%%%%%%%%%%%%%%%%%%%%%%%%%%%%%%%%%%%%%%%%%%%%%%%%%%%%
{\textcolor{blau2}{\subsection{Betti und finite Differenzen}}\index{Betti und finite Differenzen}
Der Vollst\"{a}ndigkeit halber wollen wir noch erw\"{a}hnen, dass man finite Differenzen als Anwendung des Satzes von Betti verstehen kann.

Es sei $u(x)$ die L\"{o}sung des Randwertproblems
\begin{align}
- u''(x) = p(x) \qquad u(0) = u(l) = 0\,,
\end{align}
und $G(y,x_i)$ bzw. $G(y,x_{i +1}) $ seien die L\"{o}sungen, die zu Punktlasten $\pm 1/h$ in den Knoten $i$ und $i+1$ geh\"{o}ren, dann gilt nach dem {\em Satz von Betti\/}, s. Bild \ref{U118},
\begin{align}
A_{1,2} = \frac{1}{h} \,u_{i+1} - \frac{1}{h}\,u_i = \int_0^{\,l} \frac{G(y,x_{i +1}) - G(y,x_i)}{h}\, p(y)\,dy = A_{2,1}
\end{align}
und somit in der Grenze, $h \to 0$,
\begin{align}
u'(x) = \int_0^{\,l} G_1(y,x)\,p(y)\,dy \qquad G_1(y,x) = \frac{d}{dx}\,G_0(y,x)\,.
\end{align}
Die zweiten Differenzen lauten
\begin{align} \label{Eq75}
\frac{u_{i+1} - 2\,u_i + u_{i+1}}{h^2}\,.
\end{align}
Das Integral einer H\"{u}tchenfunktion $\Np_i(x)$ vom Knoten $i-1$ bis zum Knoten $i+1$ ist $h$ und somit folgt, wenn wir $u_h''$ auf dem Interval $(-h,h)$ gleich dem Ausdruck (\ref{Eq75})  setzen
\begin{align}
\int_{-h}^{\,h} - u_h''\,\Np_i(x)\,dx = \frac{-u_{i+1} + 2\,u_i - u_{i-1}}{h}\,,
\end{align}
was genau die Zeile $i$ der Steifigkeitsmatrix ist. In der Gleichung $\vek K\,\vek u = \vek f$ ist die linke Seite ja gleich dem Vektor $\vek f_h$, also der Arbeit, die der FE-Lastfall auf den Wegen $\Np_i$ leistet und das erkl\"{a}rt, warum $k_{ij}$ ein Integral ist, die \"{U}berlagerung von $p_h$ mit $\Np_i$.
%---------------------------------------------------------------------------------
\begin{figure}
\centering
\if \bild 2 \sidecaption[t] \fi
{\includegraphics[width=0.9\textwidth]{\Fpath/U119}}
\caption{Lineare Interpolation = lineare finite Elemente, zu allen Kurven mit denselben Knotenwerten geh\"{o}ren dieselben Knotenkr\"{a}fte $f_i$}
\label{U119}%
\end{figure}%
%---------------------------------------------------------------------------------

Aber auch, wenn $k_{ij}$ eine Arbeit ist, also $p_h$ einmal integriert wird
\begin{align}
k_{ij} = \int_0^{\,l} p_h\,\Np_i\,dx
\end{align}
so bleibt $\vek K$ doch eine Differenzenmatrix und damit ist---im Umkehrschluss---ihre Inverse $\vek K^{-1}$ (die Flexibilit\"{a}tsmatrix $\vek F$) ein Integraloperator.

Eine Flexibilit\"{a}tsmatrix\index{Flexibilit\"{a}tsmatrix} berechnet aus Kr\"{a}ften Verformungen, $\vek F\,\vek f = \vek u$, sie integriert die Kr\"{a}fte, w\"{a}hrend eine Steifigkeitsmatrix aus Verformungen Kr\"{a}fte berechnet, $\vek K\,\vek u = \vek f$, sie differenziert die Verformungen.

%%%%%%%%%%%%%%%%%%%%%%%%%%%%%%%%%%%%%%%%%%%%%%%%%%%%%%%%%%%%%%%%%%%%%%%%%%%%%%%%%%%%%%%%%%%%%%%%%%%
{\textcolor{blau2}{\subsection{Interpolation und finite Elemente}}\index{Interpolation}
Wenn man eine Kurve $w(x)$ st\"{u}ckweise linear interpoliert, dann ist
der Polygonzug  identisch mit der FE-L\"{o}sung $w_h$, die zu dem 'Seil' $w(x)$ geh\"{o}rt, das in den H\"{o}hen $w(0)$ und $w(l)$ aufgeh\"{a}ngt ist, und die Belastung $- w''(x)$ tr\"{a}gt, s. Bild \ref{U119}.

Man kann also einer solchen linearen Interpolation \"{a}quivalente Knotenkr\"{a}fte $f_i$ zuschreiben
\begin{align}
f_i = \int_0^{\,l} - w''(x)\,\Np_i(x)\,dx= \tan\,\Np_i^L - \tan\,\Np_i^R \,,
\end{align}
die in den Knoten das konzentrieren, was im Nahfeld, 'auf der Strecke', an Kr\"{u}mmung vorhanden ist, und die so abrupt den Richtungswechsel
 bewirken, den die (technische) Kr\"{u}mmung $-w''$ gleitend vollzieht.

Auffallend ist dabei, dass die Steuerung der linearen Interpolation \"{u}ber die zweiten Ableitungen geschieht. Man k\"{o}nnte jetzt auf die Idee kommen, dass man, wenn man Messpunkte mit einem Lineal verbindet, auf diesem Weg Klarheit \"{u}ber den Verlauf der Kurve dazwischen bekommt. Aber  unterschiedliche Belastungen auf einem Seil k\"{o}nnen bekanntlich dieselben Knotenkr\"{a}fte $f_i$ erzeugen, was bedeutet, dass es nicht m\"{o}glich ist, aus den $f_i$ auf die Kr\"{u}mmung $-w''$ dazwischen zu schlie{\ss}en.

Bei der {\em Hermite-Interpolation\/}\index{Hermite-Interpolation} werden auch die Ableitungen in den Knoten interpoliert. Sie kann man nat\"{u}rlich in demselben Sinn als die FE-L\"{o}sung eines Balkenproblems lesen, weil ja die Element-Ansatzfunktionen $\Np_i^e(x)$ mit den Hermite-Polynomen\index{Hermite-Polynome} \"{u}bereinstimmen.



%----------------------------------------------------------------------------------------------------------
\begin{figure}[tbp]
\centering
\if \bild 2 \sidecaption \fi
\includegraphics[width=0.7\textwidth]{\Fpath/U150}
\caption{Einflussfunktionen f\"{u}r \textbf{ a)} die Durchbiegung im ersten Gelenk, \textbf{ b)} f\"{u}r die Verdrehung $w'$ \"{u}ber der Innenst\"{u}tze} \label{U150}
\end{figure}%
%----------------------------------------------------------------------------------------------------------

%-----------------------------------------------------------------
\begin{figure}[tbp]
\centering
\includegraphics[width=0.9\textwidth]{\Fpath/U316}
\caption{Schnittkr\"{a}fte, die bei der Generierung der Einflussfunktion f\"{u}r das Momente $M$ in der Mitte des Riegels 10 entstehen, \textbf{ a)} Momente, \textbf{ b)} Normalkr\"{a}fte}
\label{U316}
\end{figure}%
%-----------------------------------------------------------------
%%%%%%%%%%%%%%%%%%%%%%%%%%%%%%%%%%%%%%%%%%%%%%%%%%%%%%%%%%%%%%%%%%%%%%%%%%%%%%%%%%%%%%%%%%%%%%%%%%%
\textcolor{blau2}{\section{\"{A}nderungen der L\"{a}ngs- und Biegesteifigkeit}}
In Bild \ref{U316} a und b sind die Momente bzw. die Normalkr\"{a}fte angetragen, die entstehen, wenn man in der Mitte des Riegels 10 eine Spreizung der Gr\"{o}{\ss}e eins erzeugt, also die Einflussfunktion f\"{u}r das Moment $M$ generiert.

Man sieht deutlich, dass die Momente sehr schnell verebben, was bedeutet, dass die \"{A}nderung von $M$ auf Grund einer Steifigkeits\"{a}nderung $\Delta EI$ in irgendeinem Riegel, etwa dem Riegel 3,
\begin{align}
M_c - M =  \frac{\Delta EI}{EI}\int_0^{\,l_e} \frac{M_G \cdot M_c}{EI}\,dx
\end{align}
praktisch vernachl\"{a}ssigbar ist, weil $M_G$ relativ klein sein wird. Man muss gar nicht $M_c$ in dem betreffenden Riegel kennen, um diesen Schluss ziehen zu k\"{o}nnen.

Dagegen ist die \"{A}nderung der L\"{a}ngssteifigkeit $EA \to EA + \Delta EA$ kritischer, weil die Normalkr\"{a}fte, die durch die Generierung der Spreizung entstehen, viel weniger ged\"{a}mpft werden und somit der Einfluss auf das Moment im Riegel
\begin{align}
M_c - M = \frac{\Delta EA}{EA}\int_0^{\,l_e} \frac{N_G \cdot N_c}{EA}\,dx
\end{align}
eher bemerkbar sein wird.

Uns scheint, dass dieses ein generelles Merkmal von Stockwerkrahmen ist: \"{A}nderungen in der Biegesteifigkeit $EI \to EI + \Delta EI$ eines Riegels oder Stils sind weniger dramatisch als \"{A}nderungen in der L\"{a}ngssteifigkeit $EA \to EA + \Delta EA$.


%%%%%%%%%%%%%%%%%%%%%%%%%%%%%%%%%%%%%%%%%%%%%%%%%%%%%%%%%%%%%%%%%%%%%%%%%%%%%%%%%%%%%%%%%%%%%%%%%%%
\textcolor{blau2}{\subsection{Lagersenkung}}\label{Lagersenkung}
Hier noch ein Nachtrag zu dem Lastfall Lagersenkung. Wie auf S. \pageref{Eq36} angedeutet, spaltet man bei einer Berechnung von Hand die Biegelinie
\begin{align}\label{Eq116}
EI\,w^{IV} = 0 \qquad w(0) = w'(0) = 0 \qquad M(l) = 0\quad w(l) = w_{\Delta}\,,
\end{align}
in zwei Teile auf, $w(x) = w_1(x) + w_2(x)$. Der erste Teil weist am Balkenende die richtige Durchbiegung, $w_1(x) = w_\Delta $ auf, und der zweite Teil ist die L\"{o}sung der Differentialgleichung
\begin{align} \label{Eq117}
EI\,w_2^{IV}(x)= - EI\,w_1^{IV}(x) \quad w_2(0) = w_2'(0) = w_2(l) = 0 \quad M_2(l) = 0\,.
\end{align}
Ist $\delta w$ eine zul\"{a}ssige virtuelle Verr\"{u}ckung (dieselben geometrischen Lagerbedingungen wie $w_2$), dann ist
\begin{align}
\text{\normalfont\calligra G\,\,}(w_2,\delta w) = \int_0^{\,l} (- EI\,w_1^{IV})\,\delta w\,dx - \int_0^{\,l} \frac{M_2\,\delta M}{EI}\,dx = 0
\end{align}
und wegen
\begin{align}
\text{\normalfont\calligra G\,\,}(w_1,\delta w) = \int_0^{\,l} ( EI\,w_1^{IV})\,\delta w\,dx - \int_0^{\,l} \frac{M_1\,\delta M}{EI}\,dx = 0
\end{align}
folgt mit $w = w_1 + w_2$ in der Summe
\begin{align}
\text{\normalfont\calligra G\,\,}(w,\delta w) = 0 - \int_0^{\,l} \frac{(M_1 + M_2)\,\delta M}{EI}\,dx = 0\,.
\end{align}
Das erkl\"{a}rt das Ergebnis in Glg. (\ref{Eq120})
\begin{align}
\text{\normalfont\calligra G\,\,}(w,\delta w) &= -\int_0^{\,l} \frac{M \delta M}{EI}\,dx =  -\delta A_i = 0\,.
\end{align}
Mit finiten Elementen macht man f\"{u}r die L\"{o}sung des Randwertproblems (\ref{Eq117}) den Ansatz
\begin{align}
w_h(x) = 0 \cdot \Np_1(x) + 0 \cdot \Np_2(x) + w_\Delta \cdot \Np_3(x) + u_4 \cdot \Np_4(x)
\end{align}
und hat damit schon die Bedingung $w_h(l) = w_\Delta$ erf\"{u}llt. Das $u_4$ braucht man, um sp\"{a}ter die Forderung $M(l) = 0 $ zu erf\"{u}llen. F\"{u}r diesen Ansatz---wie f\"{u}r alle Funktionen---muss gelten
\begin{align}
\text{\normalfont\calligra G\,\,}(w_h,\Np_i) = \delta A_a(p_h,\Np_i) - \delta A_i(w_h,\Np_i) = 0 \qquad i = 1,2,3,4\,.
\end{align}
Auf Grund der {\em Galerkin-Orthogonalit\"{a}t\/} (hier in den \"{a}u{\ss}eren Arbeiten geschrieben, s. S. \pageref{Eq123})
\begin{align}
\delta A_a(p,\Np_i) -  \delta A_a(p_h\,\Np_i) = f_i - f_i^h =  0\qquad i = 1,2,3,4
\end{align}
ist das dasselbe wie
\begin{align}
\text{\normalfont\calligra G\,\,}(w_h,\Np_i) = f_i - \delta A_i(w_h,\Np_i) = 0 \qquad i = 1,2,3,4\,.
\end{align}
Wegen
\begin{align}
\delta A_i(w_h,\Np_i) = \sum_j\,\delta A_i(\Np_j,\Np_i)\,u_j =  \sum_j\,k_{ij}\,u_j
\end{align}
sind diese vier Gleichungen schlie{\ss}lich identisch mit dem System $\vek K\,\vek u = \vek f$
\begin{align}
 \frac{EI}{l^3} \left[
\begin{array}{r r r r}
 12 & -6l & -12 &-6l \\
 -6l & 4l^2 & 6l &2l^2 \\
 -12 & 6l & 12 & 6l \\
 -6l &2l^2 &6l &4l^2
 \end{array}
  \right]\,\left [\barr{c}0 \\ 0 \\ w_\Delta \\ u_4 \earr \right ] = \left [\barr{c}  f_1 \\ f_2 \\ f_3\\ f_4 \earr \right ]\,.
\end{align}
Man bringt nun die dritte Spalte auf die rechte Seite und streicht auch die dritte Zeile, so dass
die einzige unbekannte Weggr\"{o}{\ss}e $u_4$, der Tangens der Verdrehung des rechten Lagers, an Hand des Gleichungssystems
\begin{align}
 \frac{EI}{l^3} \left[
\begin{array}{r r  r}
 12 & -6l &-6l \\
 -6l & 4l^2  &2l^2 \\
  -6l &2l^2  &4l^2
 \end{array}
  \right]\,\left [\barr{c} 0 \\  0 \\ u_4 \earr \right ] =  \left [\barr{r}  f_1 \\ f_2 \\  f_4 = 0 \earr \right ] -\frac{EI}{l^3} \left [\barr{r}  -12 \\ 6l \\  6l \earr \right ] \cdot  w_\Delta
\end{align}
bestimmt werden kann,
\begin{align}
u_4 = -1.5 \cdot \frac{w_\Delta}{l}\,.
\end{align}
Das Ergebnis hat die Form eines Differenzenquotienten, was ja auch zur Dimension $[L/L] = [\,]$ der ersten Ableitung $u_4 = w'(l)$ als $\tan\,\Np$ passt.

%%%%%%%%%%%%%%%%%%%%%%%%%%%%%%%%%%%%%%%%%%%%%%%%%%%%%%%%%%%%%%%%%%%%%%%%%%%%%%%%%%%%%%%%%%%%%%%%%%%
{\textcolor{blau2}{\section{Die Rolle der finiten Elemente}}}%\index{Grenzen der Numerik}
Vielleicht passt an diese Stelle auch ein Wort \"{u}ber die Rolle der finiten Elemente im Bauwesen. Der Mathematiker versteht unter finiten Elementen Funktionen, w\"{a}hrend f\"{u}r den Ingenieur finite Elemente reale Bauteile sind mit denen er ein Tragwerk nachbildet und so interessiert den Ingenieur  nicht nur der numerische Fehler, sondern auch der Modellfehler.

Beide Fehler sind aber miteinander verschr\"{a}nkt und so bleibt Modell bleibt immer in der Schwebe, weil der numerische Fehler und der Modellfehler verschr\"{a}nkt sind.

Und diese Flexibilit\"{a}t, die Mischung aus mathematischer N\"{a}herung und Variation im Modell ist es, die den Ingenieur eigentlich an den finiten Elementen fasziniert.


 sind wichtig sein, aber sie sind nur eine Seite des Problems. , die den Ingenieur eigentlich und prim\"{a}r interessieren.  Das Thema Modellfehler hat die Mathematik erst seit relativ kurzer Zeit, Stichworte {\em Verification and Validation\/}, entdeckt. auf diese Problematik aufmerksam geworden ist.

 F\"{u}r ihn ist Modell exakt und der Fehler liegt bei den finiten Elementen. F\"{u}r den Ingenieur ist aber schon das Modell eine N\"{a}herung.

Mathematische Fehlersch\"{a}tzer sind eine gro{\ss}e Hilfe, um die Numerik in den Griff zu bekommen, aber gleichgewichtig muss daneben die Analyse des Modellfehlers stehen und hier ist der Sachverstand des Ingenieurs gefragt.

Punkte und Linien haben unterschiedliche 'Kapazit\"{a}t'. Ein Integral wie
\begin{align}
\int_0^{\,l} \sin x\,dx
\end{align}
\"{a}ndert sich nicht, wenn man einen Punkt wegl\"{a}sst, aber man darf kein Intervall $(a,b)$ \"{u}berspringen. Vielleicht ist das der mathematische Hintergrund f\"{u}r die beobachteten Ph\"{a}nomene.

%---------------------------------------------------------------------------------
\begin{figure}
\centering
\includegraphics[width=1.0\textwidth]{\Fpath/U307}
\caption{Wandscheibe unter Eigengewicht \textbf{ a)} Einflussfunktion f\"{u}r $\sigma_{yy}$, \textbf{ b)} Hauptspannungen }
\label{U307}%
\end{figure}%
%---------------------------------------------------------------------------------
%---------------------------------------------------------------------------------
\begin{figure}
\centering
\includegraphics[width=1.0\textwidth]{\Fpath/U310}
\caption{Platte, Einflussfunktion f\"{u}r $m_{xx}$}
\label{U310}%
\end{figure}%
%---------------------------------------------------------------------------------

Ein weiteres Beispiel ist die Scheibe in Bild \ref{U307}. Dort, wo der starre Rand in den freien Rand \"{u}bergeht, liegt ein singul\"{a}rer Punkt. Das Verh\"{a}ltnis der Kr\"{a}fte, die die Einflussfunktion f\"{u}r $\sigma_{yy}$ generieren, betr\"{a}gt wieder 2:1, so dass die nach oben gerichteten Knotenkr\"{a}fte die \"{U}berhand haben und die Einflussfunktion f\"{u}r $\sigma_{yy}$ am Schluss bis in den Punkt $\infty$ ausschwingt.

Zum Schluss noch ein Blick auf die Einflussfunktion f\"{u}r das Moment $m_{xx}$ in einer gelenkig gelagerten Quadratplatte, Bild \ref{U310}. In der Stabstatik werden die Einflussfunktionen f\"{u}r ein Moment $M(x)$ mit einem {\em Quadropol\/} erzeugt, s. S. \pageref{U303}. Wenn man einen solchen Quadropol (in einem k\"{u}hnen Schritt---die Mathematiker m\"{u}ssen wegschauen) auf die Platte \"{u}bertr\"{a}gt, dann steht es zwar 2:2 zwischen den ab- und aufw\"{a}rts gerichteten Kr\"{a}ften, aber die Kr\"{a}fte, die nach unten zeigen, liegen in einer Spur, w\"{a}hrend die anderen beiden Kr\"{a}fte einen Abstand voneinander haben, und dieses handicap f\"{u}hrt wohl dazu, dass die abw\"{a}rts gerichteten Kr\"{a}fte 'gewinnen', und die Platte nach unten geht.

Wir hatten in Kapitel 3, S. \pageref{Punktlager}, die Singularit\"{a}t in den Punktlagern auf das ungleiche Verh\"{a}ltnis der Kr\"{a}fte, 2:1, zur\"{u}ckgef\"{u}hrt, die das Element mit dem Lagerknoten auseinandertreiben, um die Einflussfunktion f\"{u}r $\sigma_{yy}$ zu generieren. Vielleicht l\"{a}sst sich auch so  die Singularit\"{a}t in einspringenden Ecken, wie in Bild \ref{U134} erkl\"{a}ren.

Das Verh\"{a}ltnis der Kr\"{a}fte, die die Einflussfunktion f\"{u}r $\sigma_{yy}$ generieren, betr\"{a}gt zwar 2:2, aber die Kr\"{a}fte, die nach oben dr\"{u}cken, treffen auf einen gr\"{o}{\ss}eren Widerstand als die beiden Kr\"{a}fte, die die offene Flanke, den Rand der \"{O}ffnung, nach unten dr\"{u}cken, s. Bild \ref{U306} b. Vielleicht erkl\"{a}rt dies die Singularit\"{a}t in den Ecken der \"{O}ffnungen.


%---------------------------------------------------------------------------------
\begin{figure}
\centering
\includegraphics[width=0.6\textwidth]{\Fpath/U306}
\caption{Knotenkr\"{a}fte f\"{u}r Einflussfunktionen, Spannung $\sigma_{yy} $ in einer einspringenden Ecke}
\label{U306}%
\end{figure}%
%---------------------------------------------------------------------------------


%---------------------------------------------------------------------------------
\begin{figure}
\centering
\includegraphics[width=0.8\textwidth]{\Fpath/U136}
\caption{Die Einflussfunktion f\"{u}r die Durchbiegung in Plattenmitte \"{a}ndert sich praktisch nicht, wenn man die Umgebung des Aufpunktes verfeinert}
\label{U136}%
\end{figure}%
%---------------------------------------------------------------------------------

%%%%%%%%%%%%%%%%%%%%%%%%%%%%%%%%%%%%%%%%%%%%%%%%%%%%%%%%%%%%%%%%%%%%%%%%%%%%%%%%%%%%%%%%%%%%%%%%%%%
\textcolor{blau2}{\section{'Explodierende' Einflussfunktionen}}
Mit den Singularit\"{a}ten h\"{a}ngt ein bemerkenswerter Effekt zusammen, n\"{a}mlich, dass ihre Einflussfunktionen die Punkte eines Tragwerks in den Punkt $\infty$ verschieben. Wenn man die Einflussfunktion f\"{u}r die Spannung $\sigma_{yy}$ im Punktlager einer Scheibe berechnet, s. Bild \ref{U137}, und die Scheibe adaptiv verfeinert, dann werden die Werte der Einflussfunktion am oberen Rand der Scheibe immer gr\"{o}{\ss}er und bei unendlich feinem Netz werden sie wohl unendlich gro{\ss}, w\"{a}hrend Einflussfunktionen, die zu beschr\"{a}nkten, 'normalen' Werten geh\"{o}ren, in der Ferne abklingen, s. Bild \ref{U136}, auch dann, wenn man die Umgebung des Aufpunktes adaptiv verfeinert. Einflussfunktionen, die zu singul\"{a}ren Werten geh\"{o}ren, lassen dagegen eine Scheibe 'explodieren'.

Es scheint nun theoretisch denkbar, dass es Singularit\"{a}ten gibt, die nur bei vertikal oder horizontal gerichteten Lasten auftreten. Aber dann w\"{u}rde der winzigste Drift der Belastung in  die singul\"{a}re Richtung sofort zu unendlich gro{\ss}en Spannungen f\"{u}hren, und diese m\"{u}ssten ebenso abrupt wieder verschwinden, wenn man die Last wieder zur\"{u}ckstellt. Wir haben keine Idee, wie so etwas aussehen k\"{o}nnte, wie ein Tragwerk so 'auf Kipp' stehen k\"{o}nnte.


%---------------------------------------------------------------------------------
\begin{figure}
\centering
\includegraphics[width=0.8\textwidth]{\Fpath/U137}
\caption{Einflussfunktion f\"{u}r die Spannung $\sigma_{yy}$ im Lagerknoten. Mit einer adaptiven Verfeinerung \"{a}ndern sich die Ergebnisse, wie z.B. die Verschiebung des oberen Randes, merkbar}
\label{U137}%
\end{figure}%
%---------------------------------------------------------------------------------


\footnote{Auch die Fachwerkmodelle die beim Durchstanznachweis das Tragverhalten des Betons nachbilden, sind finite Elemente!}

%%%%%%%%%%%%%%%%%%%%%%%%%%%%%%%%%%%%%%%%%%%%%%%%%%%%%%%%%%%%%%%%%%%%%%%%%%%%%%%%%%%%%%%%%%%%%%%%%%%
\textcolor{blau2}{\subsection{Die Dimension der $j_i$}}\label{Dimji}
Wenn man Einflussfunktionen, $\vek K\vek g = \vek j$, berechnet, dann haben die $j_i$ die Dimension einer Arbeit. Wir wollen das am Beispiel der Einflussfunktion f\"{u}r die Spannung $\sigma_{xx}$ in einer Scheibe verifizieren.

Wie passt dazu z.B. das in dem Beispiel auf S. XXX
\begin{align}
j_i =
\end{align} 