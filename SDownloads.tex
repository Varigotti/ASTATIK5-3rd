\textcolor{chapterTitleBlue}{\chapter{Software}}\index{downloads}\index{software}

F\"{u}nf Programme stehen zum {\em download\/} bereit:\\

\begin{itemize}\label{SoftwareDownload}
  \item BE-PLATTE \qquad \,\,\href{http://www.be-statik.de/BE-Platte\_demo.html}{http://www.be-statik.de/BE-Platte\_demo.html} \index{BE-PLATTE}
  \item BE-SCHEIBE \qquad \href{http://www.be-statik.de/BE-Scheibe\_demo.html}{http://www.be-statik.de/BE-Scheibe\_demo.html} \index{BE-SCHEIBE}
  \item BE-FRAMES \qquad \,\href{http://www.be-statik.de}{http://www.be-statik.de}
\end{itemize}
und
\begin{itemize}
  \item WINFEM   \qquad \qquad \!\!\href{http://www.winfem.de/software.htm}{http://www.winfem.de/software.htm}
  \item WINFEM-P
\end{itemize}
Die ersten beiden Programme basieren auf der Methode der Randelemente. Viele der Zeichnungen in diesem Buch wurden mit diesen beiden Programmen erstellt, weil man mit Randelementen glattere Verl\"{a}ufe erh\"{a}lt, was ja gerade bei der grafischen Darstellung erw\"{u}nscht ist.

Auf \href{http://www.winfem.de/software.htm}{http://www.winfem.de/software.htm} liegen englischsprachige Versionen
\begin{itemize}
  \item BE-SLABS
  \item BE-PLATES
\end{itemize}
 der ersten beiden Programme.

Das dritte Programm BE-FRAMES ist ein englischsprachiges Lehr-Pro\-gramm zur Berechnung von ebenen Rahmen mit Focus auf der Darstellung von Einflussfunktionen. Dieses Programm demonstriert ferner die Anwendung der {\em one-click-reanalysis\/}, s. Kapitel 5, bei der man mit einfachen Mausklicks St\"{a}be entfernen kann oder in ihrer Steifigkeit modifizieren kann, ohne dass die Steifigkeitsmatrix neu aufgestellt werden muss.\\

Die WINFEM-Programme sind Programme zur Berechnung von Scheiben und Platten mit Focus auf der Berechnung von Einflussfunktionen, s. z.B. \ref{U38G}, \ref{U75}, \ref{U370}, \ref{U35}, \ref{U84}, \ref{U294}, \ref{U295}, \ref{U237}, \ref{U129}, \ref{U271}, \ref{Tottenham}, \ref{1GreenF156}. Das Scheibenprogramm rechnet auf Wunsch adaptiv, \ref{U417}. Beide Programme k\"{o}nnen den Lastfall $p_h$ darstellen, s. z.B. \ref{U28}, \ref{U168}.

Autor von WINFEM ist Prof. Dr.-Ing. T. Gr\"{a}tsch, \href{mailto:thomas.graetsch@haw-hamburg.de}{thomas.graetsch@haw-hamburg.de}.

Erg\"{a}nzungen: Prof. Dr.-Ing. D. Materna, \href{mailto:daniel.materna@hs-owl.de}{daniel.materna@hs-owl.de} \\

Handb\"{u}cher

\begin{tabular}{l l}
  BE-PLATTE & \hspace*{1.2cm}\href{http://www.be-statik.de/data/pdf/Platte.pdf}{http://www.be-statik.de/data/pdf/Platte.pdf} \\
  BE-SCHEIBE & \hspace*{1.2cm}\href{http://www.be-statik.de/data/pdf/scheibe.pdf}{http://www.be-statik.de/data/pdf/scheibe.pdf}\\
  BE-FRAMES & \hspace*{1.2cm}\href{http://www.be-statik.de/data/pdf/BE-Frames.pdf}{http://www.be-statik.de/data/pdf/BE-Frames.pdf}\\
  WINFEM &
  \hspace*{1.2cm}\href{http://www.be-statik.de/data/pdf/WinFem.pdf}{http://www.be-statik.de/data/pdf/WinFem.pdf}
 \end{tabular}

